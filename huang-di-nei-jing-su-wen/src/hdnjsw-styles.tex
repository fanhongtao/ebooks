% 定义本书中使用到的格式

\ctexset {
  chapter = {
    pagestyle = chapterFancy,
    numbering = false,
  }
}

\usepackage{fancyhdr}
\fancypagestyle{chapterFancy}{
  \fancyhf{}                          % 清除所有的页眉和页脚样式
  \renewcommand\headrulewidth{0pt}    % 页眉的下划线宽度置为0,即不显示
  \setlength{\headheight}{13.6pt}
  \fancyhead[R]{\thepage}             % 页码
  \fancyhead[L]{\disablepinyin\leftmark\enablepinyin} % Chapter title
}

\usepackage{geometry}
\geometry{a4paper,left=2cm,right=2cm,top=2cm,bottom=1cm}

\usepackage{hyperref}
\hypersetup{colorlinks=true, linkcolor=red}

\usepackage{xpinyin}

\ExplSyntaxOn
\int_new:N \g_pinyin_int % 是否显示拼音的开关。0:不显示拼音; 1: 仅显示指定拼音的字; 2: 显示所有汉字的拼音
\newcommand{\xianshipinyin}{\int_set:Nn \g_pinyin_int {2}}  % 显示拼音
\newcommand{\zuishaopinyin}{\int_set:Nn \g_pinyin_int {1}}  % 最少拼音
\newcommand{\yincangpinyin}{\int_set:Nn \g_pinyin_int {0}}  % 隐藏拼音

% 确保在“最少拼音”情况下,某个汉字要显示拼音
% 命令有两个参数,
%    第一个是要显示的汉字,
%    第二个是该汉字的拼音。
\newcommand{\zhuyin}[2]{% 注音
  \bool_if:NTF{\int_compare_p:n {\g_pinyin_int != 0}}{
    \begin{pinyinscope}
      \setpinyin{#1}{#2}#1
    \end{pinyinscope}
  }{
    #1
  }
}
% 确保在“最少拼音”情况下,某个汉字要显示拼音
% 命令有一个参数,
%    第一个是要显示的汉字。(可以是多个汉字)
\newcommand{\xianyin}[1]{% 显(示拼)音
  \bool_if:NTF{\int_compare_p:n {\g_pinyin_int != 0}}{
    \begin{pinyinscope}
      #1
    \end{pinyinscope}
  }{
    #1
  }
}

\newenvironment{mypinyin}{
  \bool_if:NTF{\int_compare_p:n {\g_pinyin_int == 2}}{
    \begin{pinyinscope}
  }{}
}{
  \bool_if:NTF{\int_compare_p:n {\g_pinyin_int == 2}}{
    \end{pinyinscope}
  }{}
}
\ExplSyntaxOff
