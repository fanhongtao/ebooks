\chapter{黄帝内经素问序}

\centerline{启玄子 \quad 王冰 \quad  撰\footnote{王冰,号启玄子,又作启元子。约生于唐景云元年(710年),卒于贞元二十年(805年),里居籍贯不详,唐宝应中(762~763年)为太仆令,故称为王太仆。}}

(新校正:按《唐人物志》,冰仕唐,为太仆令,年八十余以寿终。)

夫释缚脱艰,全真导气,拯黎元于仁寿,济羸劣以获安者,非三圣道,则不能致之矣。
孔安国序《尚书》曰:“伏羲、神农、黄帝之书,谓之三坟,言大道也。”
班固《汉书·艺文志》曰:“《黄帝内经》十八卷。”
《素问》即其经之九卷也,兼《灵枢》九卷,乃其数焉。

虽复年移代革,而授学犹存,惧非其人,而时有所隐,故第七一卷,师氏藏之,今之奉行,惟八卷尔。
然而其文简,其意博,其理奥,其趣深,天地之象分,阴阳之候列,变化之由表,死生之兆彰,不谋而遐迩自同,勿约而幽明斯契。
稽其言有征,验之事不\xianyin{忒}。
诚可谓至道之宗,奉生之始矣。

假若天机迅发,妙识玄通,\xianyin{蒇}谋虽属乎生知,标格亦资於\xianyin{诂}训,未尝有行不由迳、出不由户者也。
然刻意研精,探微索隐,或识契真要,则目牛无全,故动则有成,犹鬼神幽赞,而命世奇杰,时时间出焉。
则周有秦公,汉有淳于公,魏有张公、华公,皆得斯妙道者也。
咸日新其用,大济蒸人,华叶递荣,声实相副,盖教之著矣,亦天之假也。

冰弱龄慕道,夙好养生,幸遇真经,式为龟镜;而世本\xianyin{纰}\zhuyin{缪}{miu4},篇目重叠,前后不伦,文义悬隔,施行不易,披会亦难。
岁月既淹,袭以成弊。或一篇重出,而别立二名;或两论并吞,而都为一目;或问答未已,别树篇题;或脱简不书,而云世\xianyin{阙};
\zhuyin{重}{chong2}《经合》而冠《针服》,并《方宜》而为《咳篇》;隔《虚实》而为《逆从》,
合《经络》而为《论要》;节《皮部》为《经络》,退《至教》以先《针》。
诸如此流,不可胜数。且将升岱岳,非径奚为?欲诣扶桑,无舟莫适!
乃精勤博访,而并有其人,历十二年,方\xianyin{臻}理要,询谋得失,深遂夙心。

时于先生郭子斋堂,受得先师张公秘本,文字昭晰,义理环周,一以参详,群疑冰释。
恐散于末学,绝彼师资,因而撰注,用传不朽。兼旧藏之卷,合八十一篇,二十四卷,勒成一部。
冀乎究尾明首,寻注会经,开发童蒙,宣扬至理而已。
其中简脱文断,义不相接者,搜求经论所有,迁移以补其处;
篇目坠缺,指事不明者,量其意趣,加字以昭其义;
篇论吞并,义不相涉,阙漏名目者,区分事类,别目以冠篇首;
君臣请问,礼仪乖失者,考校尊卑,增益以光其意;
错简碎文,前後重叠者,详其指趣,削去繁杂,以存其要;
辞理秘密,难粗论述者,别撰《玄珠》,以陈其道。
凡所加字,皆朱书其文,使今古必分,字不杂\xianyin{糅}。
庶厥昭彰圣旨,敷畅玄言,有如列宿高悬,奎张不乱,深泉净\xianyin{滢},鳞介咸分,
君臣无夭枉之期,夷夏有延龄之望,\xianyin{俾}工徒勿误,学者惟明。
至道流行,徽音\zhuyin{累}{lei3}属,千载之后,方知大圣之慈惠无穷。

时大唐宝应元年岁次壬寅序。
