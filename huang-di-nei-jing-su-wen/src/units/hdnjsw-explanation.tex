《黄帝内经》分《灵枢》、《素问》两部分,是中国最早的医学典籍,传统医学四大经典著作之一。
《黄帝内经》是一本综合性的医书,在黄老道家理论上建立了中医学上的“阴阳五行学说”、“脉象学说”、
“藏象学说”、“经络学说”、“病因学说”、“病机学说”、“病症”、“诊法”、“论治”及“养生学”、“运气学”等学说。
其基本素材来源于中国古人对生命现象的长期观察、大量的临床实践以及简单的解剖学知识。
《黄帝内经》奠定了人体生理、病理、诊断以及治疗的认识基础,是中国影响极大的一部医学著作,被称为医之始祖。
