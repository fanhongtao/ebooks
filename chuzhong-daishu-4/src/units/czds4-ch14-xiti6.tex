\xiti
\begin{xiaotis}

\xiaoti{当压力不变时,气体的体积与温度之间的函数关系式是 $V_t = V_0 + 0.0037 V_0\,t$,
    其中 $V_t$ 是气体在 $t$ ℃ 时的体积,$V_0$ 是在 $0$ ℃ 时的体积。
    现在有一定量的气体,压力不变,在 $0$ ℃ 时的体积是 $100$ 升。求:
}
\begin{xiaoxiaotis}

    \xxt{温度是 $30$ ℃ 时气体的体积;}

    \xxt{温度是多少时气体的体积是 $101$ 升。}

\end{xiaoxiaotis}


\xiaoti{已知 $y = P + z$,这里 $P$ 是一个常数,$z$ 与 $x$ 成正比例,
    且 $x = 2$ 时 $y = 1$, $x = 3$ 时 $y = -1$。
}
\begin{xiaoxiaotis}

    \xxt{写出 $y$ 与 $x$ 之间的函数关系式;}

    \xxt{计算 $x = 0$ 时 $y$ 的值;}

    \xxt{计算 $y = 0$ 时 $x$ 的值。}

\end{xiaoxiaotis}


\xiaoti{已知 $y + b$ 与 $x + a$(其中 $a$,$b$ 是常数)成正比例,求证 $y$ 是 $x$ 的一次函数。
    如果 $x = 3$ 时 $y = 5$,$x = 2$ 时 $y = 2$,把 $y$ 表示成 $x$ 的函数。
}

\xiaoti{根据实验知道,酒精的体积与温度之间的关系在一定范围内接近于一次函数。
    现在测得一定量的酒精在 $0$ ℃ 时的体积是 $5.250$ 升,
    在 $40$ ℃ 时的体积是 $5.481$ 升,计算这些酒精在 $10$ ℃ 与 $30$ ℃ 时的体积。
}

\xiaoti{声音在空气里的传播速度 $v$(米/秒)与温度 $t$(℃)的函数关系式是 $v = 331 + 0.6\,t$。
    画出函数的图象,并根据图象求当 $t = -5$ ℃ 与 $t = 15$ ℃ 时的声音的传播速度。
}


\xiaoti{}%
\begin{xiaoxiaotis}%
    \xxt[\xxtsep]{已知一次函数 $y = kx + b$ 在 $x = -4$ 时的值为 $9$,在 $x = 6$ 时的值为 $3$, 求 $k$ 与 $b$;}

    \xxt{已知直线 $y = kx + b$ 经过点 $(-4,\, 9)$ 和点 $(6,\, 3)$, 求 $k$ 与 $b$,并画出这条直线。}

\end{xiaoxiaotis}


\xiaoti{}%
\begin{xiaoxiaotis}%
    \xxt[\xxtsep]{在同一坐标系内, 画出下列直线:
        \begin{align*}
            & y = 2x + 3, && y = 2x - 3, \\
            & y = -x + 3, && y = -x - 3;
        \end{align*}
    }

    \xxt{这四条直线是不是围成一个平行四边形, 为什么?}

\end{xiaoxiaotis}


\xiaoti{}%
\begin{xiaoxiaotis}%
    \xxt[\xxtsep]{在同一坐标系内,画出函数 $y = 3x - 2$ 与 $y = 2x + 3$ 的图象;}

    \xxt{根据这两个图象,求 $x$ 等于什么值时,函数 $y = 3x - 2$ 与 $y = 2x + 3$ 有相同的值;}

    \xxt{用图象法求方程 $6x + 3 = 4x - 7$ 的解。}

\end{xiaoxiaotis}


\xiaoti{已知 $y = kx + b$ 的图象上的点在下列范围内,分别画出它的图象的大致位置,并指出 $k$ 与 $b$ 的符号。}
\begin{xiaoxiaotis}

    \xxt{在第一、二、三象限内;}

    \xxt{在第一、二、四象限内;}

    \xxt{在第一、三、四象限内;}

    \xxt{在第二、三、四象限内。}

\end{xiaoxiaotis}


\xiaoti{画出函数 $y = 3x + 12$ 的图象。利用图象,}
\begin{xiaoxiaotis}

\begin{enhancedline}
    \xxt{求当 $x = -2$,$-1$,$\dfrac{1}{2}$ 时 $y$ 的值;}

    \xxt{求当 $y = 3$,$9$,$-3$ 时对应的 $x$ 值;}
\end{enhancedline}

    \xxt{求图象与坐标轴的两个交点的坐标及交点间的距离;}

    \xxt{求方程 $3x + 12 = 0$ 的解;}

    \xxt{求不等式 $3x + 12 > 0$ 的解集;}

    \xxt{如果 $y$ 的取值范围为 $-6 \leqslant y \leqslant 6$,求 $x$ 的取值范围。}

\end{xiaoxiaotis}

\end{xiaotis}

