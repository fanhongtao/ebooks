\subsubsection{同一轴上两点间的距离}

图 \ref{fig:14-7} 是一条数轴,数轴上的点 $A$,$B$,$C$,$D$ 的坐标分别是 $7$,$2$,$-3$,$-7$。
我们来看一看数轴上任意两点间的距离(也就是连结这两点的线段的长度)能不能用这两点的坐标来表示。

\begin{figure}[htbp]
    \centering
    
\begin{tikzpicture}[>=Stealth, scale=0.6]
    \draw [->] (-9, 0) -- (9, 0);
    \foreach \x in {-8, ..., 8} {
        \draw (\x, 0.2) -- (\x, 0) node[anchor=north] {\small $\x$};
    }

    \filldraw [fill=black] ( 7, 0) circle (0.05) node [anchor=south] {\small $A$};
    \filldraw [fill=black] ( 2, 0) circle (0.05) node [anchor=south] {\small $B$};
    \filldraw [fill=black] (-3, 0) circle (0.05) node [anchor=south] {\small $C$};
    \filldraw [fill=black] (-7, 0) circle (0.05) node [anchor=south] {\small $D$};
    \filldraw [fill=black] ( 0, 0) circle (0.05) node [anchor=south] {\small $O$};
\end{tikzpicture}


    \caption{}\label{fig:14-7}
\end{figure}

由图 \ref{fig:14-7} 可以看出,线段 $BA$,$OA$,$OB$ 之间的关系是 $BA = OA - OB$。
因为 $OA = 7$,$OB = 2$,所以 $BA = 7 - 2$。但 $7$,$2$ 分别是点 $A$,$B$ 在数轴上的坐标,
这就是说,线段 $BA$ 的长度等于点 $A$ 的坐标减去点 $B$ 的坐标。
我们知道,$7 - 2 = |7 - 2| = |2 - 7|$,所以利用绝对值符号,我们就可以说,
线段 $AB$ 的长度(也就是点 $A$ 与 $B$ 之间的距离)等于 $A$,$B$ 两点的坐标之差的绝对值。类似地,
\begin{align*}
    & CB = OB + CO = 2 + 3 = |2 - (-3)| = |(-3) - 2| \douhao \\
    & DC = DO - CO = 7 - 3 = |(-7) - (-3)| = |(-3) - (-7)| \juhao
\end{align*}

一般地,\zhongdian{数轴上任意两点间的距离,等于这两点的坐标之差的绝对值。}
也就是说,如果数轴上 $A$,$B$ 两点的坐标分别为  $x_{_A}$,$x_{_B}$,%说明:因为写成 $x_A$,$x_B$ 时,下标的位置太靠上,不好看,所以写成 $x_{_A}$
那么 \zhongdian{$A$,$B$ 两点间的距离公式为}
\begin{center}
    \framebox{\quad \zhongdian{$\bm{AB = |x_{_B} - x_{_A}|}$。}\quad }
\end{center}


\lianxi
\begin{xiaotis}

\xiaoti{在 (1) 到 (6) 各图(数轴上每一格等于一个单位长度)中,就线段 $AB$ 填写下表:}

\begin{figure}[H]
    \centering
    \begin{tikzpicture}[>=Stealth, scale=0.6, transform shape]
    \tikzset{
        pics/shuzhou/.style n args={4}{
            code = {
                \draw [->] (-1, 0) -- (9, 0);
                \foreach \x in {0, ..., 8} {
                    \draw (\x, 0.2) -- (\x, 0);
                }
                \draw (4, -1.2) node {\Large $(#1)$};
                \draw (#2, 0) node [below=0.4em] {\Large $O$};
                \filldraw [fill=black] (#3, 0) circle (0.08) node [below=0.4em] {\Large $A$};
                \filldraw [fill=black] (#4, 0) circle (0.08) node [below=0.4em] {\Large $B$};
            }
        }
    }

    \draw (0, 0) pic {shuzhou={1}{2}{4}{7}};
    \draw (12, 0) pic {shuzhou={2}{2}{0}{7}};

    \draw (0,  -2.5) pic {shuzhou={3}{7}{1}{5}};
    \draw (12, -2.5) pic {shuzhou={4}{1}{7}{3}};

    \draw (0,  -5) pic {shuzhou={5}{6}{8}{1}};
    \draw (12, -5) pic {shuzhou={6}{6}{4}{0}};
\end{tikzpicture}


    \caption*{(第 1 题)}
\end{figure}

\begin{center}
    \begin{tblr}{hlines, vlines,
        columns={c},
        column{2-5}={6em},
    }
        图号 & $x_{_A}$ & $x_{_B}$ & $x_{_B} - x_{_A}$ & $AB$的长度 \\
        (1) &          &          &                    &            \\
        (2) &          &          &                    &            \\
        (3) &          &          &                    &            \\
        (4) &          &          &                    &            \\
        (5) & $2$      & $-5$     &    $-7$            &  $7$       \\
        (6) &          &          &                    &            \\
    \end{tblr}
\end{center}

\xiaoti{已知数轴上的点 $A$,$B$,$C$,$D$ 的坐标分别是 $-5$,$7$,$-2$,$3$,
    求点 $A$ 与 $B$,$B$ 与 $C$,$C$ 与 $D$,$C$ 与 $A$ 之间的距离。
}

\end{xiaotis}

