\subsection{直角三角形中边与角间的关系}\label{subsec:15-4}

\begin{wrapfigure}[7]{r}{5cm}
    \centering
    \begin{tikzpicture}
    \pgfmathsetmacro{\a}{1.5}
    \pgfmathsetmacro{\b}{3}
    \pgfmathsetmacro{\c}{sqrt(\a*\a + \b*\b)}

    \coordinate (A) at (0, 0);
    \coordinate (B) at (\b, \a);
    \coordinate (C) at (\b, 0);

    \draw (A) node [below] {$A$} to [chuizu] node [midway, below] {$b$} (C) ;
    \draw (C) node [below] {$C$} -- (B) node [midway, right] {$a$};
    \draw (B) node [above] {$B$} -- (A) node [midway, above left] {$c$};
\end{tikzpicture}


    \caption{}\label{fig:15-7}
\end{wrapfigure}

在生产实际和科学研究中,经常需要求出线段的长度或角的大小,这种问题常常可以归结为求一个三角形的边长或角的大小。
由三角形中已知的边和角,计算未知的边或角,叫做\zhongdian{解三角形}。现在先来研究解直角三角形的问题。
为此,我们用三角函数来表示直角三角形中边与角间的关系。

如图 \ref{fig:15-7},直角三角形 $ABC$ 中,$C$ 是直角,斜边是 $c$;
锐角 $A$ 的对边是 $a$,邻边是 $b$;
锐角 $B$ 的对边是 $b$,邻边是 $a$。\footnote{本章中,直角三角形 $ABC$ 的边和角的符号都是这样表示。}
如图 \ref{fig:15-8} (1) 或(2), 建立直角坐标系。

\begin{figure}[htbp]
    \centering
    \begin{minipage}[b]{6cm}
        \centering
        \begin{tikzpicture}[>=Stealth]
    \pgfmathsetmacro{\a}{1.5}
    \pgfmathsetmacro{\b}{3}
    \pgfmathsetmacro{\c}{sqrt(\a*\a + \b*\b)}
    \pgfmathsetmacro{\r}{0.5}
    \pgfmathsetmacro{\jiao}{asin(\a/\c)}

    \draw [->] (-0.5, 0) -- (4, 0) node[below=0.2em] {$x$} coordinate(x axis);
    \draw [->] (0, -0.5) -- (0, 4) node[left=0.2em]  {$y$} coordinate(y axis);
    \draw (0, 0) coordinate (O) node [below left=0.3em] {\small $O$};

    \coordinate (A) at (0, 0);
    \coordinate (B) at (\b, \a);
    \coordinate (C) at (\b, 0);

    \draw (A) node [below right] {$A$} to[chuizu] node [midway, below] {$b$} (C);
    \draw (C) node [below] {$C$} -- (B) node [midway, right] {$a$};
    \draw (B) node [above] {$B(b,a)$} -- (A) node [midway, above left] {$c$};
    \draw (0:\r) arc (0:\jiao:\r);
\end{tikzpicture}


        \caption*{(1)}
    \end{minipage}
    \qquad
    \begin{minipage}[b]{6cm}
        \centering
        \begin{tikzpicture}[>=Stealth]
    \pgfmathsetmacro{\a}{1.5}
    \pgfmathsetmacro{\b}{3}
    \pgfmathsetmacro{\c}{sqrt(\a*\a + \b*\b)}
    \pgfmathsetmacro{\r}{0.5}
    \pgfmathsetmacro{\jiao}{asin(\b/\c)}

    \draw [->] (-0.5, 0) -- (4, 0) node[below=0.2em] {$x$} coordinate(x axis);
    \draw [->] (0, -0.5) -- (0, 4) node[left=0.2em]  {$y$} coordinate(y axis);
    \draw (0, 0) coordinate (O) node [below left=0.3em] {\small $O$};

    \coordinate (A) at (\a, \b);
    \coordinate (B) at (0, 0);
    \coordinate (C) at (\a, 0);

    \draw (A) node [above] {$A(a,b)$} to [chuizu={direction=left}] node [midway, right] {$b$} (C);
    \draw (C) node [below] {$C$} -- (B) node [midway, below] {$a$};
    \draw (B) node [below right] {$B$} -- (A) node [midway, above left] {$c$};
    \draw [double] (0:\r) arc (0:\jiao:\r);
\end{tikzpicture}


        \caption*{(2)}
    \end{minipage}
    \caption{}\label{fig:15-8}
\end{figure}

根据三角函数的定义,可得:

(1) \begin{tblr}[t]{columns={mode=math}}
    \sin A = \dfrac{a}{c} \douhao & \cos A = \dfrac{b}{c} \douhao \\
    \tan A = \dfrac{a}{b} \douhao & \cot A = \dfrac{b}{a} \fenhao
\end{tblr}

(2) \begin{tblr}[t]{columns={mode=math}}
    \sin B = \dfrac{b}{c} \douhao & \cos B = \dfrac{a}{c} \douhao \\
    \tan B = \dfrac{b}{a} \douhao & \cot B = \dfrac{a}{b} \juhao
\end{tblr}


% \begin{wrapfigure}[7]{r}{5cm}
%     \centering
%     \begin{tikzpicture}
    \pgfmathsetmacro{\a}{1.5}
    \pgfmathsetmacro{\b}{3}
    \pgfmathsetmacro{\c}{sqrt(\a*\a + \b*\b)}
    \pgfmathsetmacro{\r}{0.5}
    \pgfmathsetmacro{\jiao}{asin(\a/\c)}

    \coordinate (A) at (0, 0);
    \coordinate (B) at (\b, \a);
    \coordinate (C) at (\b, 0);

    \draw (A) to [chuizu] node [midway, below] {邻边} (C);
    \draw (C)  -- (B) node [midway, right] {对边};
    \draw (B) -- (A) node [pos=0.3, above left, rotate=\jiao] {斜边};
    \draw (0:\r) arc (0:\jiao:\r) node [pos=0.8, right] {$\alpha$};
\end{tikzpicture}


%     \caption{}\label{fig:15-9}
% \end{wrapfigure}

如果用 $\alpha$ 表示直角三角形的一个锐角,那么(1) 和 (2) 可以概括为(图 \ref{fig:15-9}):


\begin{figure}[htbp]
    \centering
    \begin{minipage}[b]{7cm}
        \centering
        \begin{tikzpicture}
    \pgfmathsetmacro{\a}{1.5}
    \pgfmathsetmacro{\b}{3}
    \pgfmathsetmacro{\c}{sqrt(\a*\a + \b*\b)}
    \pgfmathsetmacro{\r}{0.5}
    \pgfmathsetmacro{\jiao}{asin(\a/\c)}

    \coordinate (A) at (0, 0);
    \coordinate (B) at (\b, \a);
    \coordinate (C) at (\b, 0);

    \draw (A) to [chuizu] node [midway, below] {邻边} (C);
    \draw (C)  -- (B) node [midway, right] {对边};
    \draw (B) -- (A) node [pos=0.3, above left, rotate=\jiao] {斜边};
    \draw (0:\r) arc (0:\jiao:\r) node [pos=0.8, right] {$\alpha$};
\end{tikzpicture}


        \caption{}\label{fig:15-9}
    \end{minipage}
    \qquad
    \begin{minipage}[b]{7cm}
        \centering
        \begin{tikzpicture}
    \pgfmathsetmacro{\a}{3.6}
    \pgfmathsetmacro{\b}{1.5}
    \pgfmathsetmacro{\c}{sqrt(\a*\a + \b*\b)}
    \pgfmathsetmacro{\r}{0.5}
    \pgfmathsetmacro{\jiaoa}{asin(\a/\c)}
    \pgfmathsetmacro{\jiaob}{asin(\b/\c)}

    \coordinate (A) at (\b, 0);
    \coordinate (B) at (0, \a);
    \coordinate (C) at (0, 0);

    \draw (A) node [below] {$A$} to [chuizu={direction=left}] node [midway, below] {$b$} (C);
    \draw (C) node [below] {$C$} -- (B) node [midway, left] {$a$};
    \draw (B) node [left]  {$B$} -- (A) node [midway, above right] {$c$};
    \draw (\b, 0) + (180:\r) arc (180:(180-\jiaoa):\r);
    \draw [double] (0, \a) + (270:\r) arc (270:(270+\jiaob):\r);
\end{tikzpicture}


        \caption{}\label{fig:15-10}
    \end{minipage}
\end{figure}


\begin{center}
\framebox[22em]{\zhongdian{
    \begin{tblr}[t]{columns={mode=math}, rows={rowsep=0.5em}}
        \bm{\sin \alpha = \dfrac{\alpha \text{的对边}}{\text{斜边}}} \douhao          & \bm{\cos \alpha = \dfrac{\alpha \text{的邻边}}{\text{斜边}}} \douhao \\
        \bm{\tan \alpha = \dfrac{\alpha \text{的对边}}{\alpha \text{的邻边}}} \douhao & \bm{\cot \alpha = \dfrac{\alpha \text{的邻边}}{\alpha \text{的对边}}} \juhao
    \end{tblr}
}}
\end{center}

这四个式子给出了直角三角形中边与角之间的关系。
今后在解直角三角形时,可以不必借助于直角坐标系,直接应用这些关系式。

由于 $B = 90^\circ - A$,从 (1) 和 (2) 还可得到
\begin{center}
    \framebox[25em]{
        \begin{tblr}[t]{columns={mode=math}}
            \bm{\sin (90^\circ - A) = \cos A} \douhao  & \bm{\cos (90^\circ - A) = \sin A} \douhao \\
            \bm{\tan (90^\circ - A) = \cot A} \douhao  & \bm{\cot (90^\circ - A) = \tan A} \juhao
        \end{tblr}
    }
\end{center}
因为 $90^\circ - A$ 与 $A$ 的三角函数之间有上述关系,所以在求三角数值的表中,
正弦与余弦可以合用一个表,正切与余切可以合用一个表。


% \begin{wrapfigure}[10]{r}{5cm}
%     \centering
%     \begin{tikzpicture}
    \pgfmathsetmacro{\a}{3.6}
    \pgfmathsetmacro{\b}{1.5}
    \pgfmathsetmacro{\c}{sqrt(\a*\a + \b*\b)}
    \pgfmathsetmacro{\r}{0.5}
    \pgfmathsetmacro{\jiaoa}{asin(\a/\c)}
    \pgfmathsetmacro{\jiaob}{asin(\b/\c)}

    \coordinate (A) at (\b, 0);
    \coordinate (B) at (0, \a);
    \coordinate (C) at (0, 0);

    \draw (A) node [below] {$A$} to [chuizu={direction=left}] node [midway, below] {$b$} (C);
    \draw (C) node [below] {$C$} -- (B) node [midway, left] {$a$};
    \draw (B) node [left]  {$B$} -- (A) node [midway, above right] {$c$};
    \draw (\b, 0) + (180:\r) arc (180:(180-\jiaoa):\r);
    \draw [double] (0, \a) + (270:\r) arc (270:(270+\jiaob):\r);
\end{tikzpicture}


%     \caption{}\label{fig:15-10}
% \end{wrapfigure}

\liti[0] 在直角三角形 $ABC$ 中,已知 $a = 12$,$b = 5$。求角 $A$、角 $B$ 的四个三角函数值(图 \ref{fig:15-10})。

\jie 由勾股定理,得

\hspace*{1.5em} $c = \sqrt{a^2 + b^2} = \sqrt{12^2 + 5^2} = 13$。

$\therefore$ \quad \begin{tblr}[t]{columns={mode=math}, rows={rowsep=0.5em}}
    \sin A = \dfrac{a}{c} = \dfrac{12}{13} \douhao & \cos A = \dfrac{b}{c} = \dfrac{5}{13} \douhao &  \\
    \tan A = \dfrac{a}{b} = \dfrac{12}{5} \douhao  & \cot A = \dfrac{b}{a} = \dfrac{5}{12} \douhao &  \\
    \sin B = \cos A = \dfrac{5}{13} \douhao        & \cos B = \sin A = \dfrac{12}{13} \douhao \\
    \tan B = \cot A = \dfrac{5}{12} \douhao        & \cot B = \tan A = \dfrac{12}{5} \juhao \\
\end{tblr}



\lianxi
\begin{xiaotis}

\xiaoti{(口答)分别说出图中角 $A$、角 $B$ 的四个三角函数值:}

\begin{figure}[htbp]
    \centering
    \begin{minipage}[b]{7cm}
        \centering
        \begin{tikzpicture}
    \pgfmathsetmacro{\a}{3}
    \pgfmathsetmacro{\b}{4}
    \pgfmathsetmacro{\c}{sqrt(\a*\a + \b*\b)}

    \coordinate (A) at (\b, 0);
    \coordinate (B) at (0, \a);
    \coordinate (C) at (0, 0);

    \draw (A) node [below] {$A$}  to [chuizu={direction=left}] node [midway, below] {$8$} (C);
    \draw (C) node [below] {$C$} -- (B) node [midway, left] {$6$};
    \draw (B) node [left]  {$B$} -- (A) node [midway, above right] {$10$};
\end{tikzpicture}


        \caption*{(1)}
    \end{minipage}
    \qquad
    \begin{minipage}[b]{7cm}
        \centering
        \begin{tikzpicture}
    \pgfmathsetmacro{\a}{sqrt(14)}
    \pgfmathsetmacro{\b}{2.5}
    \pgfmathsetmacro{\c}{sqrt(\a*\a + \b*\b)}
    \pgfmathsetmacro{\jiao}{asin(\b/\c)}

    \begin{scope}[rotate=-90-\jiao]
        \coordinate (A) at (\b, 0);
        \coordinate (B) at (0, \a);
        \coordinate (C) at (0, 0);

        \draw (A) node [below] {$A$} to [chuizu={direction=left}] node [midway, above left] {$5$}  (C);
        \draw (C) node [above] {$C$} -- (B) node [midway, above right] {$2\sqrt{14}$};
        \draw (B) node [below] {$B$} -- (A) node [midway, below] {$9$};
    \end{scope}
\end{tikzpicture}



        \caption*{(2)}
    \end{minipage}
    \caption*{(第 1 题)}
\end{figure}


\xiaoti{在直角三角形 $ABC$ 中:}
\begin{xiaoxiaotis}

    \xxt{已知 $a = 2$,$b = 1$,求角 $A$ 的四个三角函数值;}

    \xxt{已知 $a = 3$,$c = 4$,求角 $B$ 的四个三角函数值;}

    \xxt{已知 $b = 2$,$c = \sqrt{29}$,求角 $A$、角 $B$ 的四个三角函数值。}

\end{xiaoxiaotis}


\xiaoti{把下列各式写成角 $A$ 或角 $B$ 的三角函数的形式:}
\begin{xiaoxiaotis}

    \begin{tblr}{columns={18em, colsep=0pt}}
        \xxt{$\cos (90^\circ - A)$;} & \xxt{$\tan (90^\circ - B)$;} \\
        \xxt{$\sin (90^\circ - B)$;} & \xxt{$\cot (90^\circ - A)$。}
    \end{tblr}
\end{xiaoxiaotis}


\end{xiaotis}

