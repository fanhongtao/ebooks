\subsubsection{反比例函数的图象}
\begin{enhancedline}

\liti 画出反比例函数 $y = \dfrac{6}{x}$ 与 $y = -\dfrac{6}{x}$ 的图象。

\jie 第一个函数的自变量 $x$ 的取值范围是所有非零实数。
在这个范围内选取 $x$ 的一些值,算出 $y$ 的对应值,列表如下:
\begin{table}[H]
    \begin{tblr}{
        hlines, vlines,
        columns={mode=math, c},
        column{1}={2em},
    }
        x & \cdots & -6 & -5   & -4   & -3 & -2 & -1 & 1 & 2 & 3 & 4   & 5   & 6 & \cdots \\
        y & \cdots & -1 & -1.2 & -1.5 & -2 & -3 & -6 & 6 & 3 & 2 & 1.5 & 1.2 & 1 & \cdots
    \end{tblr}
\end{table}

用表里各组对应值作为点的坐标,描出各个点。
顺次连结第一象限内的各点并延长,得到图象的一个分支;
顺次连结第三象限内的各点并延长,得到图象的另一个分支。
这两个分支合起来,就是函数 $y = \dfrac{6}{x}$ 的图象(图 \ref{fig:14-17})。

\begin{figure}[htbp]
    \centering
    \begin{minipage}[b]{7cm}
    \centering
    \begin{tikzpicture}[>=Stealth, scale=0.4,
    every node/.style={fill=white, inner sep=1pt},
]
    \draw [->] (-8, 0) -- (8, 0) node[below=0.2em] {$x$} coordinate(x axis);
    \draw [->] (0, -8) -- (0, 8) node[left=0.2em]  {$y$} coordinate(y axis);
    \draw (0, 0) node [below left=0.3em] {\small $O$};
    \foreach \x in {1, ..., 6} {
        \draw (\x, 0.2) -- (\x, 0) node [below=0.2em] {\small $\x$};
    }
    \foreach \y in {1, ..., 6} {
        \draw (0.2, \y) -- (0, \y) node [left=0.2em] {\small $\y$};
    }

    \draw[domain=0.9:6.5,   samples=50] plot (\x, {6/\x});
    \draw[domain=-0.9:-6.5, samples=50] plot (\x, {6/\x});
    \draw (3.5, 5)  node {$y = \dfrac{6}{x}$};

    \foreach \x in {1, ..., 6} {
        \coordinate (a) at (\x, 6/\x);
        \draw [fill=black] (a) circle(0.1);
        \draw [dashed] (a |- x axis) -- (a) -- (a -| y axis);
    }
\end{tikzpicture}


    \caption{}\label{fig:14-17}
    \end{minipage}
    \qquad
    \begin{minipage}[b]{7cm}
    \centering
    \input{../pic/czds4-ch14-18}
    \caption{}\label{fig:14-18}
    \end{minipage}
\end{figure}

用同样的方法,可以画出 $y = -\dfrac{6}{x}$ 的图象(图 \ref{fig:14-18})。

反比例函数 $y = \dfrac{k}{x} \; (k \neq 0)$ 的图象叫做\zhongdian{双曲线}。

由图 \ref{fig:14-17} 和图 \ref{fig:14-18} , 我们可以看出\zhongdian{反比例函数 $\bm{y = \dfrac{k}{x}}$ 有下列性质:}

\zhongdian{(1)
    当 $\bm{k > 0}$ 时,函数图象的两个分支分别位于第一、三象限内,在每一个象限内,$\bm{y}$ 随 $\bm{x}$ 的增大而减小;
    当 $\bm{k < 0}$ 时,两个分支分别位于第二、四象限内,在每一个象限内,$\bm{y}$ 随 $\bm{x}$ 的增大而增大。
}

\zhongdian{(2) 两个分支都无限接近但永远不能达到 $\bm{x}$ 轴和 $\bm{y}$ 轴。}


\lianxi
\begin{xiaotis}

\xiaoti{(口答)下列各小题中的两个变量是否成反比例,为什么?}
\begin{xiaoxiaotis}

    \xxt{时间不变时,匀速运动所走的路程与运动的速度;}

    \xxt{路程不变时,匀速运动所需的时间与运动的速度。}

\end{xiaoxiaotis}


\xiaoti{已知变量 $y$ 与 $x$ 成反比例,并且当 $x = 3$ 时, $y = 7$。求:}
\begin{xiaoxiaotis}

    \xxt{$y$ 和 $x$ 之问的函数关系式;}

    \xxt{当 $x = 2\dfrac{1}{3}$ 时 $y$ 的值;}

    \xxt{当 $y = 3$ 时 $x$ 的值。}

\end{xiaoxiaotis}


\xiaoti{在同一坐标系内,画出下列函数的图象:
    $$y = \dfrac{5}{x} \nsep y = \dfrac{-5}{x} \juhao $$
}

\end{xiaotis}
\end{enhancedline}

