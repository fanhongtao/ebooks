\subsection{一次函数的图象和性质}\label{subsec:14-8}
\begin{enhancedline}

我们来画函数 $y = \dfrac{2}{3}x + 4$ 的图象,并且把它同直线 $y = \dfrac{2}{3}x$ 相比较。

在 $x$ 的取值范围内列出这两个函数的对应值表:
\begin{table}[H]
    \hspace*{2em}
    \begin{tblr}{
        hlines, vlines,
        columns={mode=math, c, 3em},
        column{1}={5em},
        rows={rowsep=0.5em},
    }
        x                     & \cdots & -2                & -1                  & 0   & 1                & 2                & \cdots \\
        y = \dfrac{2}{3}x     & \cdots & -\dfrac{4}{3}     & -\dfrac{2}{3}       & 0   & \dfrac{2}{3}     & \dfrac{4}{3}     & \cdots \\
        y = \dfrac{2}{3}x + 4 & \cdots & -\dfrac{4}{3} + 4 & -\dfrac{2}{3} + 4   & 4   & \dfrac{2}{3} + 4 & \dfrac{4}{3} + 4 & \cdots \\

    \end{tblr}
\end{table}

它们的图象如图 \ref{fig:14-19} 所示。

\begin{figure}[htbp]
    \centering
    \begin{tikzpicture}[>=Stealth, scale=0.5,
    every node/.style={fill=white, inner sep=1pt},
]
    \draw [->] (-7, 0)   -- (5, 0) node[below=0.2em] {$x$};
    \draw [->] (0, -5.5) -- (0, 7) node[left=0.2em]  {$y$};
    \draw (0, 0) node [below right=0.3em] {\small $O$};
    \foreach \x in {-3, -2, -1, 1, 2, 3, 4} {
        \draw (\x, 0.2) -- (\x, 0) node [below] {\small $\x$};
    }
    \foreach \x in {-4, ..., -6} {
        \draw (\x, 0.2) -- (\x, 0);
    }
    \foreach \y in {-3, -2, -1, 1, 2, 3} {
        \draw (0, \y) -- (-0.2, \y) node [left] {\small $\y$};
    }
    \foreach \y in {-4, -5, 4, 5} {
        \draw (0, \y) -- (-0.2, \y);
    }

    \draw[domain=-7:4.2,  samples=5] plot (\x, {2*\x/3 + 4}) (-4, 2) node [above, rotate=30] {$y = \frac{2}{3}x + 4$};
    \draw[domain=-7:4.2,  samples=5] plot (\x, {2*\x/3}) (4, 2) node [below, rotate=30] {$y = \frac{2}{3}x$};

    \foreach \x in {3, 1, -2.2, -5} {
        \coordinate (a) at (\x, 2*\x/3 + 4);
        \coordinate (b) at (\x, 2*\x/3);
        \draw [dashed] (a) -- (b);
        \draw[decorate,decoration={brace,mirror,amplitude=0.2cm}] (a) -- (b)
                node [pos=0.5, left=0.5em] {\small $4$};
    }
\end{tikzpicture}


    \caption{}\label{fig:14-19}
\end{figure}

可以看出,对于 $x$ 的每一个值,函数 $y = \dfrac{2}{3}x + 4$ 的值都比函数 $y = \dfrac{2}{3}x$ 的值多 $4$ 个单位,
因此,把直线 $y = \dfrac{2}{3}x$ 向上平行移动 $4$ 个单位,就可以得到函数 $y = \dfrac{2}{3}x + 4$ 的图象。
由此可见,一次函数 $y = \dfrac{2}{3}x + 4$ 的图象是经过点 $(0,\, 4)$ 且平行于直线 $y = \dfrac{2}{3}x$ 的一条直线。

一般地,一次函数 $y = kx + b$ 的图象是经过点 $(0,\, b)$ 且平行于直线 $y = kx$ 的一条直线。
因此,我们以后把一次函数 $y = kx + b$ 的图象叫做直线 $y = kx + b$。

直线 $y = kx + b$ 与 $y$ 轴相交于点 $B(0,\, b)$,$b$ 叫做直线 $y = kx + b$ 在 $y$ 轴上的截距,简称\zhongdian{截距}。

\zhongdian{一次函数 $\bm{y = kx + b}$ 有下列性质:}

\zhongdian{
    当 $\bm{k > 0}$ 时,$\bm{y}$ 随 $\bm{x}$ 的增大而增大;
    当 $\bm{k < 0}$ 时,$\bm{y}$ 随 $\bm{x}$ 的增大而减小。
}

由于直线的位置可以由直线上的任意两点唯一确定,所以要画 $y = kx + b$ 的图象,
只要先确定这条直线上的任意两点,然后过这两点画一条直线就行了。


\liti 画出直线 $y = \dfrac{1}{2}x + 2$。

\jie 取 $x = 0$,得 $y = 2$;取 $y = 0$, 得 $x = -4$。

经过点 $A(0,\, 2)$ 与点 $B(-4,\, 0)$ 画直线,它就是所求的直线(图 \ref{fig:14-20})。

\begin{figure}[htbp]
    \centering
    \begin{minipage}[b]{7cm}
    \centering
    \begin{tikzpicture}[>=Stealth, scale=0.5,
    every node/.style={fill=white, inner sep=1pt},
]
    \draw [->] (-5, 0) -- (5, 0) node[below=0.2em] {$x$};
    \draw [->] (0, -1) -- (0, 4) node[left=0.2em]  {$y$};
    \draw (0, 0) node [below left=0.3em] {\small $O$};
    \foreach \x in {-4, ..., -1} {
        \draw (\x, 0.2) -- (\x, 0);
    }
    \foreach \y in {1, ..., 3} {
        \draw (0, \y) -- (-0.2, \y);
    }

    \draw[domain=-5:3,  samples=5] plot (\x, {\x/2 + 2}) (2, 2) node [below] {$y = \frac{1}{2}x + 2$};
    \draw [fill=black] (0,  2) circle(0.1) node[above left=0.3em] {\small $A$};
    \draw [fill=black] (-4, 0) circle(0.1) node[below=0.3em] {\small $B$};
\end{tikzpicture}


    \caption{}\label{fig:14-20}
    \end{minipage}
    \qquad
    \begin{minipage}[b]{7cm}
    \centering
    \begin{tikzpicture}[>=Stealth, scale=0.6,
    every node/.style={fill=white, inner sep=1pt},
]
    \draw [->] (-0.5, 0) -- (5.5, 0) node[below=0.2em, xshift=1em] {$x$(千克)};
    \draw [->] (0, -0.5) -- (0, 5) node[left=0.2em]  {$y$(厘米)};
    \draw (0, 0) node [below left=0.3em] {\small $O$};
    \foreach \x in {5, 10, 15, 20} {
        \draw (\x/5, 0.2) -- (\x/5, 0) node[below=0.3em] {$\x$};
    }
    \foreach \y in {5, 10, 15, 20} {
        \draw (0, \y/5) -- (-0.2, \y/5) node[left=0.3em] {$\y$};
    }

    \coordinate (A) at (0, 12/5);
    \coordinate (B) at (15/5, 19.5/5);
    \draw (A) -- (B);
    \draw [fill=black] (A) circle(0.1) node[below right=0.3em] {\small $A$};
    \draw [fill=black] (B) circle(0.1) node[right=0.3em] {\small $B$};
\end{tikzpicture}


    \caption{}\label{fig:14-21}
    \end{minipage}
\end{figure}

\liti 一根弹簧的原长是 $12$ 厘米,它能挂的重量不能超过 $15$ 千克,并且每挂重 $1$ 千克就伸长 $\dfrac{1}{2}$ 厘米。
写出挂重后的弹簧长度 $y$(厘米)与挂重 $x$(千克)之间的函数关系式,并且画出它的图象。

分析:因为弹簧每挂重 $1$ 千克就伸长 $\dfrac{1}{2}$ 厘米,所以挂重 $x$ 千克就伸长 $\dfrac{1}{2}x$ 厘米。

又因为弹簧的原长是 $12$ 厘米,所以挂重 $x$ 千克后的长是 $\left(\dfrac{1}{2}x + 12\right)$ 厘米。

\jie $y$ 与 $x$ 之间的函数关系式为
$$ y = \dfrac{1}{2}x + 12 \; (0 \leqslant x \leqslant 15) \juhao $$

(括号中的不等式 $0 \leqslant x \leqslant 15$ 表示 $x$ 的取值范围。)

下面画 $y = \dfrac{1}{2}x + 12 \; (0 \leqslant x \leqslant 15)$  的图象。

取 $x = 0$, 得 $y = 12$;
取 $x = 15$, 得 $y = 19\dfrac{1}{2}$。

描出点 $A(0,\, 12)$ 和点 $B\left(15,\, 19\dfrac{1}{2}\right)$,然后连成线段 $AB$(想一想为什么不画直线),
这条线段就是所求的图象(图 \ref{fig:14-21})。


\lianxi
\begin{xiaotis}

\xiaoti{在同一坐标系内,画出下列函数的图象,并把它们与直线 $y = \dfrac{1}{3}x$ 相比较}
\begin{xiaoxiaotis}

    \begin{tblr}{columns={18em, colsep=0pt}}
        \xxt{$y = \dfrac{1}{3}x + 4$;} & \xxt{$y = \dfrac{1}{3}x - 2$。}
    \end{tblr}
\end{xiaoxiaotis}


\xiaoti{}%
\begin{xiaoxiaotis}%
    \xxt[\xxtsep]{已知一次函数 $y = kx + 2$ 在 $x = 5$ 时的值为 $4$, 求 $k$;}

    \xxt{已知直线 $y = kx + 2$ 经过点 $P(5,\, 4)$,画出这条直线。}

\end{xiaoxiaotis}

\end{xiaotis}

\end{enhancedline}
