\begin{starred}
\subsection{累积频率分布\footnotemark}\label{subsec:16-6}
\end{starred}
\footnotetext{本节是选学内容。}

在上节关于大麦穗长的例题中,穗长小于 $4.55$ 厘米的频率等于前两个小组的频率之和,即
$$ 0.01 + 0.01 = 0.02 \fenhao $$
穗长小于 $4.85$ 厘米的频率等于前三个小组的频率之和,即
$$ 0.01 + 0.01 + 0.02 = 0.04 \fenhao $$
依此类推。这种数据小于某一数值的频率叫做该数值的\zhongdian{累积频率}。
这个例子中各个分点的累积频率如表 \ref{tab:16-6} 最后一列所示,它可以作为频率分布表的一个补充。

根据算出的累积频率,可以绘出\zhongdian{累积频率分布图},如图 \ref{fig:16-3} 下图所示。
图中的横轴表示穗长,纵轴表示累积频率。
按照表 \ref{tab:16-6} 中各累积频率,在图中描出相应的各点。
例如,分点 $4.25$ 的累积频率是 $0.01$, 就在图中描出横坐标是 $4.25$、纵坐标是 $0.01$ 的一个点。
然后,用线段将各点依次连结起来,就得到一条折线,它就是累积频率分布图。

利用样本的累积频率分布图,可以对总体的相应情况作出估计。
例如,为了估计长度小于 $5$ 厘米的麦穗在这块麦地里所占的比例大小,
我们在累积频率分布图中找到横坐标是 $5$ (厘米)的点,然后量出这一点的纵坐标 $0.07$。
它告诉我们,长度小于 $5$ 厘米的麦穗在这块麦地里约占 $7\%$。


\lianxi

根据表 \ref{tab:16-5},计算各分点的累积频率,并绘出累积频率分布图。

