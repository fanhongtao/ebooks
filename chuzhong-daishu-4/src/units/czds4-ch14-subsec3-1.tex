\subsubsection{常量和变量}

看下面的例子:

(1) 火车以 $60 \; \text{公里/时}$ 的速度行驶,它走过的路程 $s$(公里)
与时间 $t$(时)之间的关系是 $s = 60\,t$。

(2) 一个圆的面积 $A$($\pflm$)与它的半径 $r$(厘米)之间的关系是 $A = \pi r^2$。

可以看出: 在例 (1) 中,利用公式 $s = 60\,t$ 计算火车在不同的时间内所走过的路程时,
$t$,$s$ 可以取不同的数值,而速度的数值保持不变;
在例 (2) 中,利用公式 $A = \pi r^2$ 计算不同半径的圆的面积时,
$r$,$A$ 可以取不同的数值,而 $\pi$ 的数值保持不变。

在某一过程中可以取不同数值的量,叫做\zhongdian{变量},
如上面例子中的 $t$ 时、$s$ 公里、$r$ 厘米、$A \; \pflm$。
在过程中保持同一数值的量或数,叫做\zhongdian{常量}或\zhongdian{常数},
如上面例子中的 $60 \; \text{公里/时}$ 是常量,$\pi$ 是常数。
常量和变量是对某一过程来说的,是相对的。
在例 (1) 中速度是常量,路程和时间是变量;
如果在同一时间内,研究路程与速度之间的对应关系,那么时间是常量,路程和速度是变量。

