\fuxiti
\begin{xiaotis}

\xiaoti{在坐标平面内,}
\begin{xiaoxiaotis}

    \xxt{位于第一象限的点的横坐标是什么符号?纵坐标是什么符号?}

    \xxt{位于第四象限的点的横坐标是什么符号?纵坐标是什么符号?}

    \xxt{横坐标为负数、纵坐标为正数的点位于那一象限?}

    \xxt{横坐标、纵坐标都是负数的点位于哪一象限?}

\end{xiaoxiaotis}


\xiaoti{以 $P_1(0,\; 2)$,$P_2(8,\; -4)$,$P_3(5,\; -8)$,$P_4(-3,\; -2)$ 四个点
    为顶点的四边形是不是平行四边形?是不是矩形?在坐标平面内画出这个四边形。
}


\begin{enhancedline}
\xiaoti{已知圆锥的体积公式是 $V = \dfrac{1}{3}\pi r^2h$,这里 $V$ 是圆锥的体积,
    $r$ 是它的底面圆半径,$h$ 是它的高。
}
\begin{xiaoxiaotis}

    \xxt{在 $r$ 是常量时,$V$ 与 $h$ 之间是什么对应关系?}

    \xxt{在 $h$ 是常量时,$V$ 与底面积 $A \; (=\pi r^2)$ 之间是什么对应关系?}

    \xxt{在 $V$ 是常量时,$A$ 与 $h$ 之间是什么对应关系?}

    \xxt{在 $V$ 是常量时,$r$ 与 $h$ 是否成反比例,为什么?}

    \xxt{画出 $r = 1$ 时 $V$ 随着 $h$ 变化的图象。}

    \xxt{画出 $V = 6$ 时 $A$ 随着 $h$ 变化的图象。}

\end{xiaoxiaotis}
\end{enhancedline}


\xiaoti{设从管口流出的水量 $Q$(升)与时间 $t$(秒)之间的函数关系式是 $Q = kt$,这里 $k$ 是常量。}
\begin{xiaoxiaotis}

    \xxt{$Q$ 与 $t$ 之间是什么对应关系?}

    \xxt{已知 $5$ 秒内管口流出的水量是 $120$ 升,求比例系数 $k$;}

    \xxt{根据第(2)小题中求出的比例系数,求 $8.5$ 秒内管口流出的水量;}

    \xxt{根据第(2)小题中求出的比例系数,求管口流出 $320$ 升水所需要的时间。}

\end{xiaoxiaotis}


\xiaoti{}%
\begin{xiaoxiaotis}%
    \xxt[\xxtsep]{如果 $x$ 与 $y$ 成正比例,$y$ 与 $z$ 成正比例,那么 $x$ 与 $z$ 之间是什么对应关系?}

    \xxt{如果 $x$ 与 $y$ 成反比例,$y$ 与 $z$ 成反比例,那么 $x$ 与 $z$ 之间是什么对应关系?}

    \xxt{如果 $x$ 与 $y$ 成正比例,$y$ 与 $z$ 成反比例,那么 $x$ 与 $z$ 之间是什么对应关系?}

\end{xiaoxiaotis}


\xiaoti{已知两个函数 $y_1 = k_1x + b_1$,$y_2 = k_2x + b_2$。}
\begin{xiaoxiaotis}

    \xxt{如果 $k_1 = k_2$,$b_1 \neq b_2$,那么这两个函数的图象之间有什么关系?}

    \xxt{如果 $k_1 \neq k_2$,$b_1 = b_2$,那么这两个函数的图象之间有什么关系?}

    \xxt{如果 $k_1 = k_2$,$b_1 = b_2$,那么这两个函数的图象之间有什么关系?}

\end{xiaoxiaotis}


\xiaoti{已知点 $(x_1,\; y_1)$,$(x_2,\; y_2)$ 的坐标满足一次函数关系式 $y = kx + b$,求这个一次函数。}


\xiaoti{已知函数 $y = 3x - 15$。}
\begin{xiaoxiaotis}

    \xxt{画出这个函数的图象。}

    \xxt{从图象上观察当 $x$ 取什么值时,函数的值 \\
        \begin{tblr}{columns={8em}}
            (i) 大于零; & (ii) 小于零;& (iii) 等于零。
        \end{tblr}
    }

    \hspace*{1.5em} 由此你能发现一次函数与一元一次方程及一元一次不等式三者之间有什么联系吗?

\end{xiaoxiaotis}


\xiaoti{利用 $y = 2x - 3$ 的图象,在图上表示:}
\begin{xiaoxiaotis}

    \xxt{当 $y = -2$ 时 $x$ 的值;}

    \xxt{当 $x < 0$ 时 $y$ 的取值范围;}

    \xxt{当 $y > 3$ 时 $x$ 的取值范围;}

    \xxt{当 $y < 5$ 时 $x$ 的取值范围。}

\end{xiaoxiaotis}


\xiaoti{求经过 $A(0,\; 1)$,$B(-1,\; 1)$,$C(1,\; -1)$ 三点且对称轴平行于 $y$ 轴的抛物线,并求其顶点坐标和对称轴。}


\xiaoti{已知对称轴平行于 $y$ 轴的抛物线的顶点在点 $(2,\; 3)$, 且抛物线经过点 $(3,\; 1)$, 求这条抛物线。}


\hspace*{-.5em}*\xiaoti{求下列函数的最大值或最小值,并且说明这时自变量取什么值。}
\begin{xiaoxiaotis}

    \begin{tblr}{columns={18em, colsep=0pt}}
        \xxt{$y = 2x^2 + 5$;} & \xxt{$y = (x - 3)^2 - 2$;} \\
        \xxt{$y = ax^2 - bx$(要讨论);} & \xxt{$y = (a + x)(b - x)$。}
    \end{tblr}
\end{xiaoxiaotis}

\hspace*{-.5em}*\xiaoti{解关于 $x$ 的不等式 $ax + b > cx + d$(要讨论)。}


\xiaoti{解下列不等式组:}
\begin{xiaoxiaotis}

    \begin{tblr}{columns={18em, colsep=0pt}}
        \xxt{$\begin{cases}
                1 - \dfrac{x + 1}{2} \leqslant 2 - \dfrac{x + 2}{3} \douhao \\[1em]
                x(x - 1) \geqslant (x + 3)(x - 3) \fenhao
            \end{cases}$} & \xxt{$\begin{cases}
                3 + x < 4 + 2x \douhao \\
                5x - 3 < 4x - 1 \douhao \\
                7 + 2x > 6 + 3x \juhao
            \end{cases}$}
    \end{tblr}
\end{xiaoxiaotis}


\hspace*{-.5em}*\xiaoti{解下列关于 $x$ 的不等式(要讨论):}
\begin{xiaoxiaotis}

    \begin{tblr}{columns={18em, colsep=0pt}}
        \xxt{$|x - a| < b$;} & \xxt{$|x - a| > b$。}
    \end{tblr}
\end{xiaoxiaotis}


\xiaoti{求不等式组
    $$\begin{cases}
        x(x^2 + 1) \geqslant (x + 1)(x^2 - x + 1) \douhao \\
        1 - 2x > 3(x - 9)
    \end{cases}$$
    的整数解。
}


\xiaoti{解不等式 $0 < x^2 - x - 2 < 4$。}

\end{xiaotis}

