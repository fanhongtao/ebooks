\subsection{二次函数}\label{subsec:14-9}

看下面的例子:

(1) 正方形的边长是 $x$(厘米),则它的面积 $y$ 与边长 $x$ 之间的函数关系式是 $y = x^2$($\pflm$)。

(2) 农机厂第一个月水泵的产量为 $50$(台),第三个月的产量 $y$(台)与月平均增率 $x$ 之间的函数关系式是
$$ y = 50(1 + x)^2 \douhao $$
即
$$ y = 50x^2 + 100x + 50 \juhao $$

上面两个函数关系式中,自变量 $x$ 的最高次数是 $2$。
我们把形如 $y = ax^2 + bx + c$(其中 $a$,$b$,$c$ 是常数,且 $a \neq 0$)
的函数叫做\zhongdian{二次函数}。


\lianxi
\begin{xiaotis}

\xiaoti{矩形木扳长 $a$ 厘米、宽 $b$ 厘米。如果长、宽各锯去 $x$ 厘米,求加工后木板的面积 $y$($\pflm$)与 $x$(厘米)之的函数关系式。}

\xiaoti{设圆柱的高 $h$(厘米)是常量,写出圆桂的体积 $V$($\lflm$)与底面周长 $C$(厘米)之间的函数关系式。}

\end{xiaotis}

