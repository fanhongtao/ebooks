\subsection{$30^\circ$,$45^\circ$,$60^\circ$ 角的三角函数值}\label{subsec:15-2}

我们知道,对于给定的角 $\alpha$,它的四个三角函数值是唯一确定的,对于某些特殊角,
我们可以用下面的方法求出它的三角函数值。

(1) 如图 \ref{fig:15-4}, $\alpha = 30^\circ$,我们在角 $\alpha$ 的终边上取点 $P$。
设点 $P$ 的纵坐标是 $a$,过点 $P$ 画 $x$ 轴的线 $MP$。
在直角三角形 $OPM$ 中,$\angle POM = 30^\circ$,$MP = a$,
则 $OP = 2a$。(为什么?)由勾股定理,
$$ OM = \sqrt{(2a)^2 - a^2} = \sqrt{3}a \juhao $$
也就是说, 点 $P$ 的坐标是 $(\sqrt{3}a,\; a)$,$r = 2a$,所以

\hspace*{2em} \begin{tblr}{columns={mode=math, colsep=0pt}, rows={rowsep=0.5em}}
    \sin 30^\circ = \dfrac{y}{r} = \dfrac{a}{2a} = \dfrac{1}{2} \douhao \\
    \cos 30^\circ = \dfrac{x}{r} = \dfrac{\sqrt{3}a}{2a} = \dfrac{\sqrt{3}}{2} \douhao \\
    \tan 30^\circ = \dfrac{y}{x} = \dfrac{a}{\sqrt{3}a} = \dfrac{\sqrt{3}}{3} \douhao \\
    \cot 30^\circ = \dfrac{x}{y} = \dfrac{\sqrt{3}a}{a} = \sqrt{3} \juhao \\
\end{tblr}

\begin{figure}[htbp]
    \centering
    \begin{minipage}[b]{5cm}
    \centering
    \begin{tikzpicture}[>=Stealth]
    \draw [->] (-0.5, 0) -- (4, 0) node[below=0.2em] {$x$} coordinate(x axis);
    \draw [->] (0, -0.5) -- (0, 3) node[left=0.2em]  {$y$} coordinate(y axis);
    \draw (0, 0) coordinate (O) node [below left=0.3em] {\small $O$};

    \pgfmathsetmacro{\jiao}{30}
    \coordinate (A) at (\jiao:4);
    \coordinate (P) at (\jiao:3);
    \coordinate (M) at (P |- x axis);

    \draw [thick] (O) -- (A);
    \draw [->] (1.5, 0) arc (0:\jiao:1.5) node [pos=0.7, below left] {$\jiao^\circ$};
    \draw [dashed] (P) -- (P -| y axis) node [left] {$a$};
    \draw (P)  node [above, xshift=0.3em] {$P$}  to[chuizu={direction=left}] (M)  node [below] {$M$};
    \path (P) -- (M) node [midway, right] {$a$};
    \path (O) -- (M) node [midway, below] {$\sqrt{3}a$};
    \path (O) -- (P) node [midway, above, rotate=\jiao] {$2a$};
\end{tikzpicture}


    \caption{}\label{fig:15-4}
    \end{minipage}
    \qquad
    \begin{minipage}[b]{5cm}
    \centering
    \begin{tikzpicture}[>=Stealth]
    \draw [->] (-0.5, 0) -- (3.5, 0) node[below=0.2em] {$x$} coordinate(x axis);
    \draw [->] (0, -0.5) -- (0, 3) node[left=0.2em]  {$y$} coordinate(y axis);
    \draw (0, 0) coordinate (O) node [below left=0.3em] {\small $O$};

    \pgfmathsetmacro{\jiao}{45}
    \coordinate (A) at (\jiao:3.5);
    \coordinate (P)  at (\jiao:3);
    \coordinate (M) at (P |- x axis);

    \draw [thick] (O) -- (A);
    \draw [->] (1, 0) arc (0:\jiao:1) node [midway, right] {$\jiao^\circ$};
    \draw (P)  node [above] {$P$}  to[chuizu={direction=left}] (M)  node [below] {$M$};
    \path (P) -- (M) node [midway, right] {$a$};
    \path (O) -- (M) node [midway, below] {$a$};
    \path (O) -- (P) node [midway, above, rotate=\jiao] {$\sqrt{2}a$};
\end{tikzpicture}


    \caption{}\label{fig:15-5}
    \end{minipage}
    \qquad
    \begin{minipage}[b]{4.5cm}
    \centering
    \begin{tikzpicture}[>=Stealth]
    \draw [->] (-0.5, 0) -- (3, 0) node[below=0.2em] {$x$} coordinate(x axis);
    \draw [->] (0, -0.5) -- (0, 3.5) node[left=0.2em]  {$y$} coordinate(y axis);
    \draw (0, 0) coordinate (O) node [below left=0.3em] {\small $O$};

    \pgfmathsetmacro{\jiao}{60}
    \coordinate (A) at (\jiao:3.5);
    \coordinate (P)  at (\jiao:3);
    \coordinate (M) at (P |- x axis);

    \draw [thick] (O) -- (A);
    \draw [->] (0.5, 0) arc (0:\jiao:0.5) node [midway, right] {$\jiao^\circ$};
    \draw (P)  node [above, xshift=-0.3em] {$P$}  to[chuizu={direction=left}] (M)  node [below] {$M$};
    \path (P) -- (M) node [midway, right] {$\sqrt{3}a$};
    \path (O) -- (M) node [midway, below] {$a$};
    \path (O) -- (P) node [midway, above, rotate=\jiao] {$2a$};
\end{tikzpicture}


    \caption{}\label{fig:15-6}
    \end{minipage}
\end{figure}

(2) 如图 \ref{fig:15-5},$\alpha = 45^\circ$,我们在角 $\alpha$ 的终边上取点 $P$。
设点 $P$ 的纵坐标是 $a$,则点 $P$ 的横坐标也是 $a$。(为什么?)
由勾股定理,$r = \sqrt{2}a$。所以

\hspace*{2em} \begin{tblr}{columns={mode=math, colsep=0pt}, rows={rowsep=0.5em}}
    \sin 45^\circ = \dfrac{y}{r} = \dfrac{a}{\sqrt{2}a} = \dfrac{\sqrt{2}}{2} \douhao \\
    \cos 45^\circ = \dfrac{x}{r} = \dfrac{a}{\sqrt{2}a} = \dfrac{\sqrt{2}}{2} \douhao \\
    \tan 45^\circ = \dfrac{y}{x} = \dfrac{a}{a} = 1 \douhao \\
    \cot 45^\circ = \dfrac{x}{y} = \dfrac{a}{a} = 1 \juhao \\
\end{tblr}


(3) 如图 \ref{fig:15-6},$\alpha = 60^\circ$,我们在角 $\alpha$ 的终边上取点 $P$。
设点 $P$ 的横坐标为 $a$,则 $r = OP = 2a$。(为什么?)
由勾股定理,点 $P$ 的纵坐标 $y = \sqrt{(2a)^2 - a^2} = \sqrt{3}a$。所以

\hspace*{2em} \begin{tblr}{columns={mode=math, colsep=0pt}, rows={rowsep=0.5em}}
    \sin 60^\circ = \dfrac{y}{r} = \dfrac{\sqrt{3}a}{2a} = \dfrac{\sqrt{3}}{2} \douhao \\
    \cos 60^\circ = \dfrac{x}{r} = \dfrac{a}{2a} = \dfrac{1}{2} \douhao \\
    \tan 60^\circ = \dfrac{y}{x} = \dfrac{\sqrt{3}a}{a} = \sqrt{3} \douhao \\
    \cot 60^\circ = \dfrac{x}{y} = \dfrac{a}{\sqrt{3}a} = \dfrac{\sqrt{3}}{3} \juhao \\
\end{tblr}

以上这些特殊角的三角函数值,今后经常要用到。为了便于记忆,列表如下:

\begin{table}[H]
    \centering
    \begin{tblr}{
        hlines,vlines,
        columns={c},
        column{2-4}={5em, mode=math, colsep=0pt},
        rows={m, rowsep=0.5em},
    }
        \diagboxthree{三角函数}{三角函数值}{角} & 30^\circ            & 45^\circ             & 60^\circ \\
        正弦                                   & \dfrac{1}{2}        & \dfrac{\sqrt{2}}{2}  & \dfrac{\sqrt{3}}{2} \\
        余弦                                   & \dfrac{\sqrt{3}}{2} & \dfrac{\sqrt{2}}{2}  & \dfrac{1}{2} \\
        正切                                   & \dfrac{\sqrt{3}}{3} & 1                    & \sqrt{3} \\
        余切                                   & \sqrt{3}            & 1                    & \dfrac{\sqrt{3}}{3} \\
    \end{tblr}
\end{table}


\begin{enhancedline}
\liti[0]  求下列各式的值:
\begin{xiaoxiaotis}

    \xxt{$2\sin 30^\circ + 3\cos 60^\circ + \tan 45^\circ$;}

    \xxt{$\sin^2 45^\circ + \cot 60^\circ \cos 30^\circ$;}

    \xxt{$\dfrac{1}{2}\cos 30^\circ + \dfrac{\sqrt{2}}{2}\cos 45^\circ + \sin 60^\circ \cos 60^\circ$。}

\resetxxt
\jie  \xxt{$2\sin 30^\circ + 3\cos 60^\circ + \tan 45^\circ = 2 \times \dfrac{1}{2} + 3 \times \dfrac{1}{2} + 1 = 3\dfrac{1}{2}$;}

\xxt{$\sin^2 45^\circ + \cot 60^\circ \cos 30^\circ = \left(\dfrac{\sqrt{2}}{2}\right)^2 + \dfrac{\sqrt{3}}{3} \times \dfrac{\sqrt{3}}{2} = \dfrac{1}{2} + \dfrac{1}{2} = 1$;}

\xxt{$\dfrac{1}{2}\cos 30^\circ + \dfrac{\sqrt{2}}{2}\cos 45^\circ + \sin 60^\circ \cos 60^\circ = \dfrac{1}{2} \times \dfrac{\sqrt{3}}{2} + \dfrac{\sqrt{2}}{2} \times \dfrac{\sqrt{2}}{2} + \dfrac{\sqrt{3}}{2} \times \dfrac{1}{2} = \dfrac{2 + 2\sqrt{3}}{4} = \dfrac{1 + \sqrt{3}}{2}$。}

\end{xiaoxiaotis}


\lianxi
\begin{xiaotis}

\xiaoti{(口答)$\sin 30^\circ$ 与 $\cos 60^\circ$ 的值各是多少?
    $\tan 45^\circ$ 与 $\cot 45^\circ$ 呢?
    $\sin 45^\circ$ 与 $\cos 45^\circ$ 呢?
    $\sin 60^\circ$ 与 $\cos 30^\circ$ 呢?
    $\tan 60^\circ$ 与 $\cot 30^\circ$ 呢?
    $\tan 30^\circ$ 与 $\cot 60^\circ$ 呢?
}

\xiaoti{求下列各式的值:}
\begin{xiaoxiaotis}

    \xxt{$\sin 30^\circ - 3\tan 30^\circ + 2\cos 30^\circ$;}

    \xxt{$2\cos 30^\circ + \tan 60^\circ - 6\cot 60^\circ$;}

    \xxt{$5\cot 30^\circ - 2\cos 60^\circ + 2\sin 60^\circ$;}

    \xxt{$\cos^2 45^\circ + \sin^2 45^\circ$;}

    \xxt{$\dfrac{\sin 60^\circ - \cot 45^\circ}{\tan 60^\circ - 2 \tan 45^\circ}$。}

\end{xiaoxiaotis}


\xiaoti{在直角坐标系中,以原点为顶点,$x$ 轴的正半轴为始边,在第一象限内画出 $40^\circ$ 的角。
    量出它的终边上一点的坐标及这个点到原点的距离,然后计算 $40^\circ$ 角的四个三角函数值(精确到 $0.01$)。
}

\end{xiaotis}
\end{enhancedline}

