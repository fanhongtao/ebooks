\xiti
\begin{xiaotis}

\xiaoti{分别写出下图中角 $A$、角 $B$ 的四个三角函数值:}

\begin{figure}[htbp]
    \centering
    \begin{minipage}[b]{4.5cm}
        \centering
        \begin{tikzpicture}
    \pgfmathsetmacro{\factor}{0.04}
    \pgfmathsetmacro{\b}{55 * \factor}
    \pgfmathsetmacro{\c}{73 * \factor}
    \pgfmathsetmacro{\a}{sqrt(\c*\c - \b*\b)}

    \coordinate (B) at (0, 0);
    \coordinate (A) at (\a, -\b);
    \coordinate (C) at (\a, 0);

    \draw (A) node [below] {$A$} to [chuizu] node [midway, right] {$55$} (C) ;
    \draw (C) node [above] {$C$} -- (B);
    \draw (B) node [above] {$B$} -- (A) node [midway, below left] {$73$};
\end{tikzpicture}


        \caption*{(1)}
    \end{minipage}
    \qquad
    \begin{minipage}[b]{4.5cm}
        \centering
        
\begin{tikzpicture}
    \pgfmathsetmacro{\factor}{0.2}
    \pgfmathsetmacro{\a}{16 * \factor}
    \pgfmathsetmacro{\b}{10 * \factor}
    \pgfmathsetmacro{\c}{sqrt(\a*\a + \b*\b)}
    \pgfmathsetmacro{\jiao}{asin(\b/\c)}

    \begin{scope}[rotate=-90-\jiao]
        \coordinate (A) at (\b, 0);
        \coordinate (B) at (0, \a);
        \coordinate (C) at (0, 0);

        \draw (A) node [below] {$A$} to [chuizu={direction=left}] node [midway, above left] {$10$}  (C);
        \draw (C) node [above] {$C$} -- (B) node [midway, above right] {$16$};
        \draw (B) node [below] {$B$} -- (A);
    \end{scope}
\end{tikzpicture}


        \caption*{(2)}
    \end{minipage}
    \qquad
    \begin{minipage}[b]{4.5cm}
        \centering
        \begin{tikzpicture}
    \pgfmathsetmacro{\factor}{0.4}
    \pgfmathsetmacro{\a}{8 * \factor}
    \pgfmathsetmacro{\c}{sqrt{73} * \factor}
    \pgfmathsetmacro{\b}{sqrt(\c*\c - \a*\a)}

    \begin{scope}[rotate=-75]
        \coordinate (A) at (\b, 0);
        \coordinate (B) at (0, \a);
        \coordinate (C) at (0, 0);

        \draw (A) node [below] {$A$} to [chuizu={direction=left}] (C);
        \draw (C) node [above] {$C$} -- (B) node [midway, above] {$8$};
        \draw (B) node [right]  {$B$} -- (A) node [midway, below right] {$\sqrt{73}$};
    \end{scope}
\end{tikzpicture}


        \caption*{(3)}
    \end{minipage}
    \caption*{(第 1 题)}
\end{figure}

\xiaoti{在直角三角形 $ABC$ 中($C = 90^\circ$):}
\begin{xiaoxiaotis}

    \xxt{已知 $a = 9$,$c = 15$,求角 $A$ 的四个三角函数值;}

    \xxt{已知 $b = 21$,$c = 29$,求角 $A$ 的四个三角函数值;}

    \xxt{已知 $a = 2$,$b = 6$,求角 $A$ 和 角 $B$ 的四个三角函数值。}

\end{xiaoxiaotis}


\xiaoti{判断边长为 $8$ cm,$15$ cm,$17$ cm 的三角形是不是直角三角形,如果是直角三角形,
    求最小边所对角的四个三角函数值。
}

\begin{enhancedline}
\xiaoti{在直角三角形 $ABC$ 中($C = 90^\circ$):}
\begin{xiaoxiaotis}

    \xxt{$\exdfrac{a}{c}$ 是角 $A$ 的什么三角函数,是角 $B$ 的什么三角函数?}

    \xxt{$\exdfrac{b}{c}$ 是角 $A$ 的什么三角函数,是角 $B$ 的什么三角函数?}

    \xxt{$\exdfrac{a}{b}$ 是角 $A$ 的什么三角函数,是角 $B$ 的什么三角函数?}

\end{xiaoxiaotis}


\xiaoti{根据下列条件解直角三角形(不用查表):}
\begin{xiaoxiaotis}

    \begin{tblr}{columns={18em, colsep=0pt}}
        \xxt{$c = 10 \nsep A = 45^\circ$;}   & \xxt{$a = 6 \nsep B = 30^\circ$;} \\
        \xxt{$a = 50 \nsep c = 50\sqrt{2}$;} & \xxt{$a = 8\sqrt{5} \nsep b = 8\sqrt{15}$。}
    \end{tblr}
\end{xiaoxiaotis}
\end{enhancedline}


\xiaoti{根据下列条件利用对数解直角三角形:}
\begin{xiaoxiaotis}

    \xxt{$c = 8.035 \nsep A = 38^\circ19'$;}

    \xxt{$b = 7.234 \nsep A = 7^\circ20'$;}

    \xxt{$a = 25.64 \nsep b = 32.48$。}

\end{xiaoxiaotis}


\xiaoti{已知等腰三角形的顶角为 $78^\circ4'$,底边上的高是 $28.5$ cm,求腰长和面积(保留三个有效数字)。}

\xiaoti{如图,在离铁塔 $150$ 米的 $A$ 处,用测角仪器测得塔顶的仰角为 $30^\circ12'$。
    已知测角仪器高 $AD = 1.52$ 米,求铁塔高 $BE$(精确到 $0.1$ 米)。\footnotemark
}
\footnotetext{
    在视线与水平线所成的角中,视线在水平线上方的叫做\zhongdian{仰角},在水平线下方的叫做\zhongdian{俯角}。
    \begin{tikzpicture}[>=Stealth,]
    \coordinate (O) at (0, 0);
    \pgfmathsetmacro{\jiao}{30}

    \draw (O) -- +(3, 0) node [above] {水平线};
    \draw (0, -1) -- (0, 1) node [midway, left, align=center] {铅\\[-0.5em]垂\\[-0.5em]线};

    \pgfmathsetmacro{\jiao}{20}
    \pgfmathsetmacro{\r}{1.0}
    \draw [->] (O) -- (\jiao:2.5) node [right] {视线};
    \draw (0:\r) arc (0:\jiao:\r) node [pos=0.7, right=0.5em] {\small 仰角};

    \pgfmathsetmacro{\r}{0.7}
    \draw [->] (O) -- (-\jiao:2.5) node [right] {视线};
    \draw [double] (0:\r) arc (0:-\jiao:\r) node [pos=0.8, right=1.0em] {\small 俯角};
\end{tikzpicture}

}

\begin{figure}[htbp]
    \centering
    \begin{minipage}[b]{7cm}
        \centering
        \begin{tikzpicture}[>=Stealth,
    every node/.style={fill=white, inner sep=1pt, outer sep=3pt},
]
% 各坐标点的相对位置
%  B
%  |
%  |       A
%  E       D
    \pgfmathsetmacro{\factor}{0.04}
    \pgfmathsetmacro{\b}{150 * \factor}
    \pgfmathsetmacro{\jiaoa}{30.2} % 30度12分
    \pgfmathsetmacro{\a}{tan(\jiaoa) * \b}
    \pgfmathsetmacro{\ad}{1.52 * \factor}
    \pgfmathsetmacro{\r}{0.7}

    \coordinate (E) at (0, 0);
    \coordinate (D) at (4, 0);
    \coordinate (A) at (4, 0.5);

    \draw [name path=p1] (E) -- +(0, 3);
    \path [name path=p2] (A) -- +(180-\jiaoa:5);
    \path [name intersections={of=p1 and p2, by=B}];

    \draw [thick, ground={angle=-135}] (-1, 0) -- (5, 0);
    \draw [<->] ([yshift=-1.8em] E) to [xianduan={above=1.8em}] node {$150$米} ([yshift=-1.8em] D);

    \draw (E) node [below, xshift=-0.7em] {$E$}  -- (B) node [above] {$B$};
    \draw (D) node [below, xshift=0.7em]  {$D$} -- (A) node [right] {$A$};
    \draw (B) -- (A) -- +(180:2);
    \draw (A) + (180:\r) arc (180:180-\jiaoa:\r) node [midway, left] {$30^\circ12'$};

    \coordinate (B1) at (-0.4, 0);
    \coordinate (B2) at ( 0.4, 0);
    \draw [name path=b1] (B) -- (B1);
    \draw [name path=b2] (B) -- (B2);
    \foreach \y in {0.8, 1.4, 1.9, 2.3, 2.5, 2.7} {
        \path [name path=xline] (-1, \y) -- (1, \y);
        \path [name intersections={of=b1 and xline, by=NB1}];
        \path [name intersections={of=b2 and xline, by=NB2}];
        \draw (B1) -- (NB2) -- (NB1) -- (B2);
        \coordinate (B1) at (NB1);
        \coordinate (B2) at (NB2);
    }
\end{tikzpicture}


        \caption*{(第 8 题)}
    \end{minipage}
    \qquad
    \begin{minipage}[b]{7cm}
        \centering
        \begin{tikzpicture}[>=Stealth,
    every node/.style={fill=white, inner sep=1pt, outer sep=3pt},
]
% 各坐标点的相对位置
%          D -- E
%  A ---   C    |
%  |       B    |
%  G------------F
    \pgfmathsetmacro{\factor}{0.016}
    \pgfmathsetmacro{\ag}{124 * \factor}
    \pgfmathsetmacro{\ac}{140 * \factor}
    \pgfmathsetmacro{\bd}{83  * \factor}
    \pgfmathsetmacro{\ef}{150 * \factor}
    \pgfmathsetmacro{\bc}{\ag + \bd - \ef}
    \pgfmathsetmacro{\jiaoa}{atan(\bc/\ac)}
    %\pgfmathsetmacro{\jiaoa}{atan2(\bc,\ac)}
    \pgfmathsetmacro{\r}{0.7}

    \coordinate (G) at (0,   0);
    \coordinate (A) at (0,   \ag);
    \coordinate (B) at (\ac, \bc);
    \coordinate (C) at (\ac, \ag);
    \coordinate (D) at (\ac, \ef);
    \coordinate (E) at ($(D) + (1, 0)$);
    \coordinate (F) at ($(E) - (0, \ef)$);

    \draw [thick] (G) -- (A) -- (B) -- (C) -- (D) -- (E) -- (F) -- cycle;
    \draw [dashed] (A) -- (C);
    \draw (A) + (0:\r) arc (0:-\jiaoa:\r) node [pos=0.8, right] {$\alpha$};
    \draw [<->] ([yshift= 0.7em] A) to [xianduan={above=0.3em, below=0.7em}] node {$140$} ([yshift= 0.7em] C);
    \draw [<->] ([xshift=-1em] G) to [xianduan={below=1em}] node [rotate=90] {$124$} ([xshift=-1em] A);
    \draw [<->] ([xshift= 1em] B) to [xianduan={above=1em}] node [rotate=90] {$83$}  ([xshift= 1em] D);
    \draw [<->] ([xshift= 1em] F) to [xianduan={above=1em}] node [rotate=90] {$150$} ([xshift= 1em] E);
\end{tikzpicture}


        \caption*{(第 9 题)}
    \end{minipage}
\end{figure}



\xiaoti{在加工如图的垫模时,需计算斜角 $\alpha$,根据图示尺寸求 $\alpha$。}

\xiaoti{如图,一拦水坝的横断面为梯形 $ABCD$,根据图示数据求:}
\begin{xiaoxiaotis}

    \xxt{坡角 $\alpha$ 和 $\beta$;}

    \xxt{坝底宽 $AD$ 和斜坡 $AB$ 的长(精确到 $0.1$ 米)。}

\end{xiaoxiaotis}

\begin{figure}[htbp]
    \centering
    \begin{tikzpicture}[>=Stealth,
    every node/.style={fill=white, inner sep=1pt, outer sep=3pt},
]
% 各坐标点的相对位置
%          B -- C
%          |    |
%  A ---   E -- F -- D
    \pgfmathsetmacro{\factor}{0.5}
    \pgfmathsetmacro{\bc}{2.8 * \factor}
    \pgfmathsetmacro{\be}{4.2 * \factor}
    \pgfmathsetmacro{\cd}{7.5 * \factor}
    \pgfmathsetmacro{\ae}{\be * 2.5}
    \pgfmathsetmacro{\jiaoa}{atan(\be/\ae)}
    \pgfmathsetmacro{\jiaod}{asin(\be/\cd)}
    \pgfmathsetmacro{\df}{\be * cot(\jiaod)} % \cf = \be
    \pgfmathsetmacro{\r}{0.7}

    \coordinate (A) at (0, 0);
    \coordinate (B) at (\ae, \be);
    \coordinate (C) at (\ae + \bc, \be);
    \coordinate (E) at (\ae, 0);
    \coordinate (D) at (\ae + \bc + \df, 0); % \ef = \bc

    \draw (A) node [below] {$A$}
        -- (B) node [above left] {$B$} node [midway, above, rotate=\jiaoa] {$i = 1:2.5$}
        -- (C) node [above right] {$C$} node [midway, above] {$2.8$米}
        -- (D) node [below] {$D$} node [midway, above, rotate=-\jiaod] {$7.5$米}
        -- (E) node [below] {$E$}
        -- cycle;
    \draw [dashed] (B) -- (E) node [midway, below, rotate=90] {$4.2$米};
    \draw (B) to[chuizu={skipline=true}] (E);

    \draw (A) + (0:\r) arc (0:\jiaoa:\r) node [midway, right] {$\alpha$};
    \draw [double] (D) + (180:\r) arc (180:180-\jiaod:\r) node [midway, left] {$\beta$};
\end{tikzpicture}


    \caption*{(第 10 题)}
\end{figure}



\xiaoti{如图,利用土堤修筑一条渠道:在堤中间挖去深为 $0.6$ m 的一块(图中的(三) 是挖去部分的横断面),
    把挖出来的土填在两旁(图中的(一)、(二)是填土部分的横断面)即成一渠道。
    已知渠道内坡度为 $1:1.5$,渠道底面宽 $BC$ 为 $0.5$ m,求:
}
\begin{xiaoxiaotis}

    \xxt{横断面(等腰梯形)$ABCD$ 的面积;}

    \xxt{修一条长 $100$ m 的渠道要挖出多少方土?}

\end{xiaoxiaotis}

\begin{figure}[htbp]
    \centering
    \begin{tikzpicture}[>=Stealth,
    every node/.style={fill=white, inner sep=1pt, outer sep=3pt},
]
% 各坐标点的相对位置
%      K  L                E  F
%    J      A          D        G
%               B   C  M
%  I                               H
    \pgfmathsetmacro{\factor}{2.5}
    \pgfmathsetmacro{\bc}{0.5 * \factor}
    \pgfmathsetmacro{\jiaob}{atan(1/1.5)}

    % 下面的数据原题中没有给出,是自己填写的。
    \pgfmathsetmacro{\cm}{1.5}
    \pgfmathsetmacro{\dm}{1}   % \dm:\cm = 1:1.5
    \pgfmathsetmacro{\r}{0.7}

    % 以 BC 的中点为原点,这样,
    %    B、C 是关于 y 轴的对称点,
    %    A、D 是关于 y 轴的对称点,
    %    ……
    \coordinate (O) at (0, 0);
    \coordinate (C) at (\bc/2, 0);
    \coordinate (D) at ($(C) + (\cm, \dm)$);
    \coordinate (E) at ($(C)!1.5!(D)$);
    \coordinate (F) at ($(E) + (0.6, 0)$);
    \coordinate (H) at ($(E) + (2, -2.5)$);

    \coordinate (A) at ($(D -| O)!-1!(D)$);
    \path [name path=ad] (A) -- ($(A)!1.5!(D)$);
    \path [name path=fh] (F) -- (H);
    \path [name intersections={of=ad and fh, by=G}];

    \coordinate (B) at ($(C -| O)!-1!(C)$);
    \coordinate (I) at ($(H -| O)!-1!(H)$);
    \coordinate (J) at ($(G -| O)!-1!(G)$);
    \coordinate (K) at ($(F -| O)!-1!(F)$);
    \coordinate (L) at ($(E -| O)!-1!(E)$);

    \fill [pattern={mylines[angle=45, distance={5pt}]}]
        (A) -- (B)  -- (C) -- (D)
        -- (G) -- (H) -- (I) -- (J) -- cycle;
    \draw [thick]
        (A) node [above, xshift=0.1em] {$A$}
        -- (B) node [below] {$B$} node [pos=0.6, above, rotate=-\jiaob] {\small $1:1.5$}
        -- (C) node [below] {$C$} node [midway, above] {(三)}
        -- (D) node [above, xshift=-0.1em] {$D$}
        -- (E) -- (F) node [pos=0.4, below] {(二)}
        -- (H) -- (I) -- (K)
        -- (L) node [pos=0.6, below] {(一)}
        --cycle;
    \draw [dashed, thick] (J) -- (G);
\end{tikzpicture}


    \caption*{(第 11 题)}
\end{figure}


\end{xiaotis}

