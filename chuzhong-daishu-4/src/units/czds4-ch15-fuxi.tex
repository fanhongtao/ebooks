\fuxiti
\begin{xiaotis}
\begin{enhancedline}

\xiaoti{已知角 $\alpha$ 的终边分别经过下列各点,求角 $\alpha$ 的四个三角函数值。}
\begin{xiaoxiaotis}

    \begin{tblr}{columns={18em, colsep=0pt}}
        \xxt{$(-8,\; 6)$;} & \xxt{$(5,\; \sqrt{7})$;} \\
        \xxt{$(-\sqrt{2},\; \sqrt{3})$;} & \xxt{$\left(0,\; \exdfrac{2}{3}\right)$。}
    \end{tblr}
\end{xiaoxiaotis}


\xiaoti{在直角三角形 $ABC$ 中,角 $C$ 为直角,$CD$ 是斜边 $AB$ 上的高。
    分别写出等于角 $A$ 的正弦、余弦、正切的线段的比。这样的比有多少个?
}


\xiaoti{求下列各式的值:}
\begin{xiaoxiaotis}

    \xxt{$\tan^2{150^\circ} + 2\sin{60^\circ} + \tan{45^\circ} \sin{90^\circ} - \tan{60^\circ} + \cos^2{30^\circ}$;}

    \xxt{$\cos{60^\circ} - \sin^2{135^\circ} + \exdfrac{3}{4}\tan^2{30^\circ} + \cos^2{30^\circ} - \sin{30^\circ}$;}

    \xxt{$\dfrac{\sin{90^\circ}}{\tan{45^\circ} - \sin{120^\circ}}$;}

    \xxt{$\dfrac{\tan{30^\circ} \cos{135^\circ} \sin{60^\circ}}{\cot{120^\circ} \sin{150^\circ}}$。}

\end{xiaoxiaotis}


\xiaoti{求下列三角函数值:}
\begin{xiaoxiaotis}

    \xxt{$\sin{162^\circ21'} \nsep \cos{103^\circ16'} \nsep \tan{93^\circ53'} \nsep \cot{145^\circ3'}$;}

    \xxt{$\sin{109^\circ43'} \nsep \cos{167^\circ4'} \nsep \tan{151^\circ32'} \nsep \cot{100^\circ32'}$。}

\end{xiaoxiaotis}


\xiaoti{求适合下列各式的三角形的内角 $\alpha \; (0^\circ < \alpha < 180^\circ)$:}
\begin{xiaoxiaotis}

    \xxt{$\sin\alpha = \dfrac{\sqrt{3}}{2} \nsep \cos\alpha = -0.6581 \nsep \tan\alpha = 35.43$;}

    \xxt{$\sin\alpha = \sin{15^\circ} \nsep \cos\alpha = -\cos{45^\circ} \nsep -\cot\alpha = \cot{115^\circ}$。}

\end{xiaoxiaotis}


\xiaoti{设 $A$,$B$,$C$ 是一个三角形的三个内角,求证:}
\begin{xiaoxiaotis}

    \xxt{$\sin{\dfrac{A + B}{2}} = \cos{\exdfrac{C}{2}}$;}

    \xxt{$\tan{\dfrac{B + C}{2} = \cot{\exdfrac{A}{2}}}$。}

\end{xiaoxiaotis}


\xiaoti{沿水库拦水坝的背水坡将坝面加宽 $2.0$ 米,坡度由原来的 $1:2$ 改成 $1:2.5$。
    已知原背水坡长 $BD = 13.4$ 米,坝长 $90$ 米,求完成这一工程需多少方土(保留两个有效数字)。
}


\begin{figure}[htbp]
    \centering
    \begin{tikzpicture}[>=Stealth]
    \pgfmathsetmacro{\factor}{0.4}
    \pgfmathsetmacro{\cd}{2 * \factor}
    \pgfmathsetmacro{\ef}{\cd}
    \pgfmathsetmacro{\bd}{13.4 * \factor}
    \pgfmathsetmacro{\dbf}{atan{1/2}}
    \pgfmathsetmacro{\cae}{atan{1/2.5}}
    \pgfmathsetmacro{\df}{sin(\dbf) * \bd}
    \pgfmathsetmacro{\bf}{2 * \df}

    \pgfmathsetmacro{\ce}{\df}
    \pgfmathsetmacro{\ae}{2.5 * \ce}

    \coordinate ["$E$" below] (E) at (0, 0);
    \coordinate ["$A$" below] (A) at (-\ae, 0);
    \coordinate ["$B$" below] (B) at (-\bf+\ef, 0);
    \coordinate ["$F$" below] (F) at (\ef, 0);
    \coordinate ["$C$" above left] (C) at (0, \ce);
    \coordinate ["$D$" above right] (D) at (\cd, \ce);

    % 下面的两个点是任意取的点
    \coordinate (D') at ($(C)!2.5!(D)$);
    \coordinate (F') at ($(B)!2.3!(F)$);

    \draw [thick] (B) -- (F') -- (D') -- (D) -- cycle node [midway, above, rotate=\dbf] {$1:2$};
    \draw [dashed] (D) -- (C) -- (A) node [midway, above, rotate=\cae] {$1:2.5$} -- (B);
    \draw [dashed] (C) -- (E) pic [draw, solid, angle radius=0.5em] {right angle=C--E--F'};
    \draw [dashed] (D) -- (F) pic [draw, solid, angle radius=0.5em] {right angle=D--F--F'};

    \begin{scope}
        \coordinate (X1) at ($(C) + (0, 0.8)$);
        \coordinate (X2) at ($(D) + (0, 0.8)$);
        \draw (C)  -- ($(C)!1.3!(X1)$);
        \draw (D)  -- ($(D)!1.3!(X2)$);
        \draw (X1) -- (X2) node [midway, above] {$2.0$};
        \draw [<-] (X1) -- ($(X1)!-1!(X2)$);
        \draw [<-] (X2) -- ($(X2)!-1!(X1)$);
    \end{scope}
    %\draw [<->] ([yshift=1.0em] C) to [xianduan={below=1.0em}] node [above] {$2.0$} ([yshift=1.0em] D);

    \coordinate (X) at ($(F') + (-1.2, 0.8)$);
    \path [path picture={\draw (path picture bounding box.north west) pic {waterwave};}]
        (X) -- ($(X) + (-1.4, 0.7)$) -- ($(X) + (0.8, 0.7)$) -- cycle;
\end{tikzpicture}


    \caption*{(第 7 题)}
\end{figure}

\xiaoti{直角三角形 $ABC$ 中,已知 $A = 32^\circ20'$, 角 $A$ 的平分线 $AT = 14.7$ cm,
    求直角边 $BC$ 和斜边 $AB$ 的长(保留三个有效数字)。
}


\xiaoti{某型号飞机的机翼形状如图,根据图示尺寸计算 $AC$、$BD$ 和 $AB$ 长度(保留三个有效数字)。}

\begin{figure}[htbp]
    \centering
    \begin{minipage}[b]{7cm}
        \centering
        \begin{tikzpicture}[>=Stealth]
    % 各坐标点的相对位置
    %  C -- E
    %  |
    %  D -- F
    %  |          A
    %  G  ----    B
    \pgfmathsetmacro{\factor}{0.5}
    \pgfmathsetmacro{\cd}{3.40 * \factor}
    \pgfmathsetmacro{\bg}{5.00 * \factor}
    \pgfmathsetmacro{\ace}{45}
    \pgfmathsetmacro{\bdf}{30}

    \pgfmathsetmacro{\dbg}{\bdf}
    \pgfmathsetmacro{\dg}{\bg * tan(\dbg)}
    \pgfmathsetmacro{\bd}{\dg / sin(\dbg)}

    \coordinate ["$B$" right] (B) at (0, 0);
    \coordinate ["$D$" below left] (D) at (-\bg, \dg);
    \coordinate ["$C$" above] (C) at (-\bg, \dg + \cd);
    \path [name path=ca] (C) -- +(-\ace:\bd*1.5); % AC < 1.5 * BD
    \path [name path=ba] (B) -- +(0, \cd);        % AB 小于 CD
    \path [name intersections={of=ca and ba, by=A}];
    \node at (A) [right] {$A$};

    \coordinate (E) at ($(C) + (1, 0)$);
    \coordinate (F) at ($(D) + (1, 0)$);
    \coordinate (G) at (-\bg, 0);

    \draw [very thick] (A) -- (B) -- (D) -- (C) -- cycle;
    \draw (C) -- (E) pic [draw, "$45^\circ$" {xshift=0.5em}, angle radius=1.5em, angle eccentricity=1.3] {angle=A--C--E};
    \draw (D) -- (F) pic [draw, "$30^\circ$" {xshift=0.5em}, angle radius=1.5em, angle eccentricity=1.3] {angle=B--D--F};

    \draw (D) -- (G);
    \draw [<->] ([yshift=-1.0em] G) to [xianduan={above=1.0em}] node [above] {$5.00$ 米} ([yshift=-1.0em] B);
    \draw [<->] ([xshift=-1.0em] C) to [xianduan={above=1.0em}] node [above, rotate=90] {$3.40$ 米} ([xshift=-1.0em] D);
\end{tikzpicture}


        \caption*{(第 9 题)}
    \end{minipage}
    \qquad
    \begin{minipage}[b]{7cm}
        \centering
        \begin{tikzpicture}[>=Stealth,]
    \pgfmathsetmacro{\factor}{0.09}
    \pgfmathsetmacro{\ab}{15.8 * \factor}
    \pgfmathsetmacro{\bc}{60.5 * \factor}
    \pgfmathsetmacro{\abc}{80}

    \coordinate (B) at (0, 0);
    \coordinate (C) at (\bc, 0);
    \coordinate (A) at (\abc:\ab);

    %---------------------------------
    \pgfmathsetmacro{\ra}{\ab/2}
    \pgfmathsetmacro{\rb}{\ab/2.5}
    \pgfmathsetmacro{\rc}{\ab/4.5}

    \draw [thick] (A) circle (\ra/2);
    \draw [thick] (A) circle (\ra);
    \draw [thick] (B) circle (\rb/4);
    \draw [thick] (B) circle (\rb/2);
    \draw [thick] (C) circle (\rc/4);
    \draw [thick] (C) circle (\rc/2);

    \draw [thick] (C) + (200:\rc) coordinate (X1)
        arc (200:440:\rc) coordinate(X2)
        to [out=155, in=25] ($(X2) + (-4.5, 0)$)
        .. controls +(-0.5, -0.2) and +(-0.0, -0.2) .. ($(A) + (-55:\ra)$)
    ;

    \draw [thick] (X1)
        to [out=145, in=25] ($(B) + (-45:\rb)$)
        arc (-45:-180:\rb)
        to [out=75, in=-65] ($(A) + (225:\ra)$)
    ;

    \draw [<->] ([yshift=-2.0em] B) to [xianduan={above=2.0em}] node [above] {$60.5$} ([yshift=-2.0em] C);
    \draw [<->,  transform canvas={shift=(170:2.5em)}] (A) to [xianduan={above=2.5em}] node [above, rotate=\abc] {$15.8$} (B);

    \begin{scope}[every node/.style={fill=white, inner sep=0pt},]
        \draw [dashed] (A) node [above=0.15em] {$A$}
            -- (B) node [below left=0.2em] {$B$}
            -- (C) node [above=0.5em] {$C$} -- cycle;
        \draw (B) + (40:1.3em) node [rotate=-40] {\small $80^\circ$};
    \end{scope}

\end{tikzpicture}


        \caption*{(第 10 题)}
    \end{minipage}
\end{figure}


\xiaoti{缝纫机挑线杆的形状如图,在加工时需要计算 $A$,$C$ 两孔中心的距离。
    已知 $BC = 60.5$ mm, $AB = 15.8$ mm, $\angle ABC = 80^\circ$,
    求 $AC$ 的长(保留三个有效数字)。
}


\xiaoti{已知梯形 $ABCD$ 的上底 $DC = 15.3$ cm, 下底 $AB = 24.2$ cm,
    一腰 $AD = 12.0$ cm, 另一腰 $BC = 11.7$ cm,
    求这梯形的各角与面积(保留三个有效数字)。
}


\xiaoti{已知 $\triangle ABC$ 的面积为 $12 \, \text{cm}^2$,边长 $a = 6$ cm, $b = 8$ cm,
    求这两边的夹角 $C$, 并画图说明解的情况。
}


\xiaoti{在 $\triangle ABC$ 中,$AD$ 为角 $A$ 的平分线,用正弦定理证明:
    $$ \dfrac{BD}{DC} = \dfrac{AB}{AC} \juhao $$
}



\renewcommand{\mi}{\mathord{\text{m}}}
\xiaoti{如图,$A$,$B$ 两点间有小山和小河,为求 $AB$ 的长,选择一点 $D$,
    使 $AD$ 可直接丈量且 $B$,$D$ 两点可以和互看到,再在 $AD$ 上选一点 $C$,
    使 $B$, $C$ 两点可以看到,已测得:\\[-0.5em]
    \hspace*{10em}
    \begin{tblr}{columns={mode=math}}
        AC = 149.46 \, \mi \douhao   & CD = 52.61 \, \mi \douhao \\
        \angle ADB = 108^\circ15' \douhao & \angle ACB = 120^\circ25' \douhao
    \end{tblr} \\[-0.5em]
    求 $AB$。
}

\begin{figure}[htbp]
    \centering
    \begin{minipage}[b]{7cm}
        \centering
        \begin{tikzpicture}[
    river/.style={decorate, decoration={random steps,segment length=3pt,amplitude=2pt}},
]
    \pgfmathsetmacro{\factor}{0.02}
    \pgfmathsetmacro{\ac}{149.46 * \factor}
    \pgfmathsetmacro{\cd}{52.61 * \factor}
    \pgfmathsetmacro{\adb}{108.25}
    \pgfmathsetmacro{\acb}{120.417} % 120度25分

    % 设 E 是 B 在 CD 上的垂足
    \pgfmathsetmacro{\bde}{180 - \adb}
    \pgfmathsetmacro{\bce}{180 - \acb}
    % BE = tan(\bde) * DE
    % BE = tan(\bce) * CE = tan(\bce) * (DE +CE)
    \pgfmathsetmacro{\de}{(\cd * tan(\bce)) / (tan(\bde) - tan(\bce))}
    \pgfmathsetmacro{\be}{\de * tan(\bde)}

    \coordinate (E) at (0, 0);
    \coordinate ["$B$" left] (B) at (0, \be);
    \coordinate ["$D$" below] (D) at (\de, 0);
    \coordinate ["$C$" below] (C) at (\de + \cd, 0);
    \coordinate ["$A$" below] (A) at (\de + \cd + \ac, 0);

    \draw [thick] (A) -- (D);
    \draw [dashed] (B) -- (A);
    \draw [dashed] (B) -- (C) pic [draw, solid, angle radius=0.8em] {angle=A--C--B};
    \draw [dashed] (B) -- (D) pic [draw, solid, angle radius=0.8em] {angle=A--D--B};


    \begin{scope}[yshift=1.1cm]
        \foreach \x in {1, ..., 5} {
            \coordinate (X) at (0, \x/5);
            \draw [river] (X) to [out=-10, in=160] +(2, -0.7)
                to [out=0, in=190] +(5, -0.6);
        }
    \end{scope}
    % 利用已有的 “山” 的 pic, 绘制后手工调整到适合的位置
    \coordinate (M) at ($(A)!0.4!(B)$);
    \pic [scale=0.087, rotate around=55:(M)] at (M) {mountain};
\end{tikzpicture}


        \caption*{(第 15 题)}
    \end{minipage}
    \qquad
    \begin{minipage}[b]{7cm}
        \centering
        \begin{tikzpicture}[>=Stealth]
    % 各坐标点的相对位置
    %
    %          M
    %          C
    %               B
    %   N
    %   A
    \pgfmathsetmacro{\factor}{0.3}
    \pgfmathsetmacro{\nac}{45}
    \pgfmathsetmacro{\ac}{10 * \factor}
    \pgfmathsetmacro{\mcb}{105}

    \pgfmathsetmacro{\mca}{180 - \nac}
    \pgfmathsetmacro{\acb}{360 - \mca - \mcb}
    \pgfmathsetmacro{\cab}{asin(sin(\acb) * 9/21)}% V_cb : V_ab
    \pgfmathsetmacro{\cba}{180 - \cab - \acb}
    \pgfmathsetmacro{\ab}{\ac * sin(\acb) / sin(\cba)}

    \coordinate ["$A$" below] (A) at (0, 0);
    \coordinate ["$B$" right] (B) at (90 - \nac - \cab:\ab);
    \coordinate ["$C$" above left] (C) at (90 - \nac:\ac);

    \draw [very thick] (A) -- (B) -- (C) -- cycle;

    \draw [->] (C) -- +(0, 1) coordinate (M) node [above] {北}
        pic [draw, solid, <-, angle radius=0.8em, "$105^\circ$" {xshift=0.5em}, angle eccentricity=1.8] {angle=B--C--M};

    \draw [->] (A) -- +(0, 2) coordinate (N) node [above] {北}
        pic [draw, solid, <-, angle radius=1.5em, "$45^\circ$" {yshift=0.2em}, angle eccentricity=1.3] {angle=C--A--N};
    \draw pic [draw, solid, <-, angle radius=2.9em, "$?$", angle eccentricity=1.3] {angle=B--A--N};

    \coordinate (M) at ($(B) + (0.7, -0.6)$);
    \pic [scale=0.03] at (M) {mountain};
\end{tikzpicture}


        \caption*{(第 15 题)}
    \end{minipage}
\end{figure}


\xiaoti{某渔轮在航行中不幸遇险,发出呼救信号。我海军舰艇在 $A$ 处茯悉后,
    立即测出该渔轮在方位角为 $45^\circ$,距离为 $10$ 海里的 $C$ 处,
    并测得渔轮正沿方位角为 $105^\circ$ 的方向,以 $9$ 海里/时 的速度向小岛 $B$ 靠拢。
    我海军舰艇立即以 $21$ 海里/时 的速度前去营救。
    求出舰艇的航向和靠近渔轮所需的时间(提示:设舰艇收到信号后 $x$ 小时在 B 处靠拢渔轮)。
}


\xiaoti{用余弦定理证明:平行四边形两对角线平方的和等于四边平方的和。}


\xiaoti{根据三角形面积公式
    $$ S_{\triangle} = \sqrt{s(s - a) (s- b) (s - c)} $$
    $\left( \text{其中} s = \exdfrac{1}{2} (a + b + c) \text{,} a \text{、} b \text{、} c \text{是三角形三边的长} \right)$,
    计算下列各题中的三角形面积 $S_\triangle$:
}
\begin{xiaoxiaotis}

    \xxt{$a = 20 \nsep b = 13 \nsep c = 21$;}

    \xxt{$a = 17 \nsep b = 21 \nsep c = 10$。}

\end{xiaoxiaotis}

\end{enhancedline}
\end{xiaotis}

