\subsection{化钝角三角函数为锐角三角函数}\label{subsec:15-7}

在生产实际和科学研究中,也常遇到解斜三角形(锐角三角形或钝角三角形)的问题。
为了研究斜三角形中边和角间的关系,我们先讨论 $90^\circ \leqslant \alpha < 180^\circ$ 时,
角 $\alpha$ 的三角函数的情况。

当 $\alpha = 90^\circ$ 时,角 $\alpha$ 的终边与 $y$ 轴的正半轴 $Oy$ 重合(图 \ref{fig:15-17}),
这时角 $\alpha$ 的终边上任一点 $P(x,\; y)$,有 $x = 0$,$y = r = OP$。所以
\begin{center}
    \begin{tblr}{rows={mode=math, rowsep=0.5em}}
        \sin{90^\circ} = \exdfrac{y}{r} = 1 \douhao & \cos{90^\circ} = \exdfrac{x}{r} = 0 \douhao \\
        \tan{90^\circ} \text{不存在} \douhao & \cot{90^\circ} = \exdfrac{x}{y} = 0 \juhao
    \end{tblr}
\end{center}

\begin{figure}[htbp]
    \centering
    \begin{minipage}[b]{6.5cm}
        \centering
        \begin{tikzpicture}[>=Stealth,]
    \draw [->] (-0.5, 0) -- (2, 0) node [below] {$x$} coordinate(x axis);
    \draw [->] (0, -0.5) -- (0, 2) node [left]  {$y$} coordinate(y axis);
    \draw (0, 0) coordinate (O) node [below left] {$O$};
    \coordinate (P) at (0, 1.2);
    \filldraw [fill=black] (P) circle (0.05) node [right] {$P(x, y)$};
    %\draw [->] (0:0.5) arc (0:90:0.5) node[midway, right=0.3em] {$90^\circ$};
    \draw pic [draw, ->, "$90^\circ$", angle eccentricity=1.8] {angle=x axis--O--P};
    \node [left] at (0, 0.8) {$r$};
\end{tikzpicture}


        \caption{}\label{fig:15-17}
    \end{minipage}
    \qquad
    \begin{minipage}[b]{8cm}
        \centering
        \begin{tikzpicture}[>=Stealth,]
    \draw [->] (-3.5, 0) -- (1.5, 0) node [below] {$x$} coordinate(x axis);
    \draw [->] (0, -1) -- (0, 2) node [left]  {$y$} coordinate(y axis);
    \draw (0, 0) coordinate (O) node [below left] {$O$};

    \pgfmathsetmacro{\jiao}{150}
    \coordinate (P) at (\jiao:3);
    \coordinate (M) at (P |- O);

    \draw (O) -- ($(O)!1.2!(P)$);
    % \draw [->] (0:0.5) arc (0:\jiao:0.5) node[midway, right=0.3em] {$\alpha$};
    \draw pic [draw, ->, "$\alpha$", angle eccentricity=1.4] {angle=x axis--O--P};
    \filldraw [fill=black] (P) circle (0.05) node [right] {$P(x, y)$};
    % \draw [dashed] (P) -- (M) node [below] {$M$} node [midway, left] {$y$};
    % \draw (P) to[chuizu={skipline=true}] (M);
    \draw [dashed] (P) -- (M) node [below] {$M$} node [midway, left] {$y$}
        pic [draw, solid, angle radius=0.5em] {right angle=P--M--x axis};
    \node [above] at ($(O)!0.5!(P)$) {$r$};
    \node [below] at ($(O)!0.5!(M)$) {$x$};
\end{tikzpicture}


        \caption{}\label{fig:15-18}
    \end{minipage}
\end{figure}


当 $90^\circ < \alpha < 180^\circ$ 时,角 $\alpha$ 的终边在第二象限(图 \ref{fig:15-18}),
这时角 $\alpha$ 的终边上任一点 $P(x,\; y)$,有 $x < 0$,$y > 0$,$r = OP > 0$。所以
\begin{center}
    \begin{tblr}{rows={mode=math, rowsep=0.5em}}
        \sin{\alpha} = \exdfrac{y}{r} > 0 \douhao & \cos{\alpha} = \exdfrac{x}{r} < 0 \douhao \\
        \tan{\alpha} = \exdfrac{y}{x} < 0 \douhao & \cot{\alpha} = \exdfrac{x}{y} < 0 \juhao
    \end{tblr}
\end{center}

我们知道锐角三角函数的值都是正的。(为什么?)
但是,对于钝角三角函数来说,除正弦的值仍是正的以外,余弦、正切、余切的值都是负的。


\liti 已知角 $\alpha$ 的终边经过点 $P(-3, 4)$,求角 $\alpha$ 在的四个三角函数值(图\ref{fig:15-19})。

\begin{wrapfigure}[4]{r}{5cm}
    \centering
    \begin{tikzpicture}[>=Stealth, scale=0.7]
    \draw [->] (-4, 0) -- (1.5, 0) node [below] {$x$} coordinate(x axis);
    \draw [->] (0, -1) -- (0, 5) node [left]  {$y$} coordinate(y axis);
    \draw (0, 0) coordinate (O) node [below left] {$O$};
    \foreach \x in {-3, ..., -1} {
        \draw (\x, 0.2) -- (\x, 0);
    }
    \foreach \y in {1, ..., 4} {
        \draw (-0.2, \y) -- (0, \y);
    }

    \coordinate (P) at (-3, 4);

    \draw (O) -- ($(O)!1.2!(P)$);
    % \pgfmathsetmacro{\jiao}{180 - asin(4/5)};
    % \draw [->] (0:0.8) arc (0:\jiao:0.8) node[midway, above] {$\alpha$};
    \draw pic [draw, ->, "$\alpha$", angle eccentricity=1.4] {angle=x axis--O--P};
    \filldraw [fill=black] (P) circle (0.05) node [right] {$P(-3, 4)$};
    \draw [dashed] (P) -- (P |- O);
\end{tikzpicture}


    \caption{}\label{fig:15-19}
\end{wrapfigure}


\jie $\because$ \quad $x = -3 \nsep y = 4$,

$\therefore$ \quad $r = \sqrt{(-3)^2 + 4^2} = \sqrt{25} = 5$。

$\therefore$ \quad  \begin{tblr}[t]{rows={mode=math, rowsep=0.5em}}
    \sin{\alpha} = \exdfrac{y}{r} = \exdfrac{4}{5} \douhao \\
    \cos{\alpha} = \exdfrac{x}{r} = -\exdfrac{3}{5} \douhao \\
    \tan{\alpha} = \exdfrac{y}{x} = -\exdfrac{4}{3} \douhao \\
    \cot{\alpha} = \exdfrac{x}{y} = -\exdfrac{3}{4} \juhao
\end{tblr}

对于给定的一个钝角,怎样求出它的三角函数值呢?

锐角的三角函数值可以查表求得,如果我们能够把钝角的三角函数转化为锐角的三角函数,
那么求钝角的三角函数值的问题就解决了。

容易知道,任意一个钝角都可以表示成 $180^\circ - \alpha$ 的形式,其中 $\alpha$ 为锐角。
例如,$120^\circ = 180^\circ - 60^\circ$。
我们来研究钝角 $180^\circ - \alpha$ 与锐角 $\alpha$ 的三角函数之间的关系。

\begin{wrapfigure}[7]{r}{6.5cm}
    \centering
    \begin{tikzpicture}[>=Stealth,]
    \draw [->] (-2.5, 0) -- (2.5, 0) node [below] {$x$} coordinate(x axis);
    \draw [->] (0, -0.5) -- (0, 3) node [left]  {$y$} coordinate(y axis);
    \draw (0, 0) coordinate (O) node [below left] {$O$};

    \coordinate (P)  at (-1.5, 2);
    \coordinate (P1) at ( 1.5, 2);

    \draw (O) -- ($(O)!1.2!(P)$);
    \draw (O) -- ($(O)!1.2!(P1)$);
    \draw [dashed] (P) node [left]  {$P(x, y)$} -- (P1) node [right] {$P_1(x, y)$};

    \draw pic [draw, ->, "$\alpha$", angle radius=2em] {angle=x axis--O--P1};
    \draw pic [draw, ->, "$180^\circ - \alpha$" {xshift=1cm, yshift=-0.5cm}, angle eccentricity=1.4, angle radius=3em] {angle=x axis--O--P};

    \node [above right, inner sep=1pt] at ($(O)!0.6!(P)$)  {$r$};
    \node [above left,  inner sep=1pt] at ($(O)!0.6!(P1)$) {$r$};
\end{tikzpicture}


    \caption{}\label{fig:15-20}
\end{wrapfigure}

如图 \ref{fig:15-20}, 在钝角 $180^\circ - \alpha$ 的终边上任取
一点 $P(x,\; y)$,设 $OP = r$。
在锐角 $\alpha$ 的终边上取点 $P_1(x_1,\; y_1)$,使 $OP_1 = r$,
那么,因为 $OP$ 和 $OP_1$ 与 $y$ 轴成相等的角,且 $OP = OP_1$,
所以点 $P$ 和 $P_1$ 关于 $y$ 轴对称。
于是,这两个点的坐标有下面的关系:
$$ x = -x_1 \nsep y = y_1 \juhao $$

$\therefore$ \quad  \begin{tblr}[t]{rows={mode=math, rowsep=0.5em}}
    \sin{(180^\circ - \alpha)} = \exdfrac{y}{r} = \dfrac{y_1}{r} = \sin{\alpha} \douhao \\
    \cos{(180^\circ - \alpha)} = \exdfrac{x}{r} = \dfrac{-x_1}{r} = -\cos{\alpha} \douhao \\
    \tan{(180^\circ - \alpha)} = \exdfrac{y}{x} = \dfrac{y_1}{-x_1} = -\tan{\alpha} \douhao \\
    \cot{(180^\circ - \alpha)} = \exdfrac{x}{y} = \dfrac{-x_1}{y_1} = -\cot{\alpha} \juhao
\end{tblr}

当 $\alpha$ 为锐角时,有

\begin{center}
    \framebox[28em]{
        \begin{tblr}[t]{rows={mode=math, rowsep=0.5em}}
            \bm{\sin (180^\circ - \alpha) =  \sin \alpha} \douhao  & \bm{\cos (180^\circ - \alpha) = -\cos \alpha} \douhao \\
            \bm{\tan (180^\circ - \alpha) = -\tan \alpha} \douhao  & \bm{\cot (180^\circ - \alpha) = -\cot \alpha} \juhao
        \end{tblr}
    }
\end{center}

这些公式今后经常用到,要记住。


\liti 求下列三角函数值:
\begin{xiaoxiaotis}

    \begin{tblr}{columns={9em, colsep=0pt}}
        \xxt{$\sin 120^\circ$;} & \xxt{$\cos 158^\circ14'$;} & \xxt{$\tan 135^\circ$;} & \xxt{$\cot 150^\circ18'$。}
    \end{tblr}

\resetxxt
\jie \begin{tblr}[t]{columns={colsep=0pt}}
    \xxt{$\sin 120^\circ = \sin (180^\circ - 60^\circ) = \sin 60^\circ = \dfrac{\sqrt{3}}{2}$;} \\
    \xxt{$\cos 158^\circ14' = \cos (180^\circ - 21^\circ46') = -\cos 21^\circ46' = -0.9287$;} \\
    \xxt{$\tan 135^\circ = \tan (180^\circ - 45^\circ) = -\tan 45^\circ = -1$;} \\
    \xxt{$\cot 150^\circ18' = \cot (180^\circ - 29^\circ42') = -\cot 29^\circ42' = -1.753$。}
\end{tblr}

\end{xiaoxiaotis}



\liti%
\begin{xiaoxiaotis}%
    \hspace*{-1.5em}\begin{tblr}[t]{columns={colsep=0pt}}
        \xxt[\xxtsep]{已知 $\sin\alpha = \exdfrac{5}{6}$,$0^\circ < \alpha < 180^\circ$,求 $\alpha$;} \\
        \xxt{已知 $\cos\alpha = -0.8728$,$0^\circ < \alpha < 180^\circ$,求 $\alpha$。}
    \end{tblr}

\begin{enhancedline}
\jie (1) 已知 $\sin\alpha = \exdfrac{5}{6} \approx 0.8333$,$0^\circ < \alpha < 180^\circ$,
所以 $\alpha$ 可以是锐角,也可以是钝角。从“正弦和余弦表”查得 $\sin 56^\circ27' = 0.8333$,
\end{enhancedline}

$\therefore$ \quad $a_1 = 56^\circ27'$。

又 $\because$ \quad $\sin (180^\circ - 56^\circ27') = \sin 56^\circ27' = 0.83333$,

$\therefore$ \quad $a_2 = 180^\circ - 56^\circ27' = 123^\circ33'$。

本题有两解:
$$ a_1 = 56^\circ27' \nsep a_2 = 123^\circ33' \juhao $$


(2)已知 $\cos\alpha$ 为负值,且 $0^\circ < \alpha < 180^\circ$,所以 $\alpha$ 是钝角。
设 $\alpha = 180^\circ - \theta$,$\theta$ 为锐角,于是
$$ \cos\alpha = \cos (180^\circ - \theta) = -\cos\theta = -0.8728 \juhao $$

可得 $\cos\theta = 0.8728$,

查表得 $\theta = 29^\circ13'$,所以
$$ \alpha = 180^\circ - 29^\circ13' = 150^\circ47' \juhao $$

\end{xiaoxiaotis}


\lianxi
\begin{xiaotis}

\xiaoti{已知角 $\alpha$ 的终边分别经过下列各点,求角 $\alpha$ 的四个三角函数值:}
\begin{xiaoxiaotis}

    \begin{tblr}{columns={9em, colsep=0pt}}
        \xxt{$(-2,\; 2)$;} & \xxt{$(-1,\; \sqrt{3})$;} & \xxt{$(-2,\; \sqrt{5})$;} & \xxt{$(0,\; 3)$。}
    \end{tblr}
\end{xiaoxiaotis}


\xiaoti{求下列三角函数值:}
\begin{xiaoxiaotis}

    \xxt{$\sin{135^\circ} \nsep \cos{120^\circ} \nsep \tan{150^\circ} \nsep \cot{150^\circ}$;}

    \xxt{$\sin{118^\circ8'} \nsep \cos{100^\circ24'} \nsep \tan{95^\circ12'} \nsep \cot{151^\circ42'}$;}

    \xxt{$\cos{123^\circ26'} \nsep \sin{90^\circ10'} \nsep \cot{134^\circ43'} \nsep \tan{172^\circ21'}$。}

\end{xiaoxiaotis}


\xiaoti{已知 $0^\circ < \theta < 180^\circ$,求下列各式中的 $\theta$ 值:}
\begin{xiaoxiaotis}

    \begin{tblr}{columns={12em, colsep=0pt}, rows={rowsep=0.5em}}
        \xxt{$\sin\theta = \exdfrac{1}{2}$;} & \xxt{$\sin\theta = 0.6517$;} & \xxt{$\cos\theta = -\dfrac{\sqrt{3}}{2}$;} \\
        \xxt{$\cos\theta = -0.3541$;} & \xxt{$\tan\theta = -3.566$;} & \xxt{$\cot\theta = -\dfrac{\sqrt{3}}{3}$。}
    \end{tblr}
\end{xiaoxiaotis}


\end{xiaotis}



