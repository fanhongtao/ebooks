\subsubsection{反比例函数}
\begin{enhancedline}

看下面的例子:

(1) 矩形的面积是 $12 \; \pflm$,这时底 $y$(厘米)与高 $x$(厘米)之间的函数关系式是
$$ y = \dfrac{12}{x} \juhao $$

(2) 走 $25$ 公里的路程, 所需的时间 $t$(时)与平均速度 $v$(公里/时)之间的函数关系式是
$$ t = \dfrac{25}{v} \juhao $$

例 (1) 中的两个变量 $y$ 与 $x$ 的积是一个常数(等于 $12$)。同样,
例 (2) 中的两个变量 $t$ 与 $v$ 之间也有这种性质。
在算术中,我们说具有这种性质的两个量成反比例。
在这里,我们把 $y = \dfrac{12}{x}$, $t = \dfrac{25}{v}$ 这样的函数都叫做反比例函数。

一般地,函数 $y = \dfrac{k}{x}$($k$ 是不等于零的常数)叫做\zhongdian{反比例函数}
(这时我们说 $y$ 与 $x$ \zhongdian{成反比例})。
在算术中,$k$ 只能取正数,现在我们把它推广到也可以取负数。
确定了 $k$ 的值,就可以确定一个反比例函数。

\begingroup
\renewcommand{\limi}{\mathord{\text{cm}}}%厘米
\renewcommand{\pflm}{\mathord{\text{cm}^2}}%平方厘米

\liti 已知圆柱体积不变,它的高 $h = 12.5 \; \limi$ 时,底面积 $S = 20 \; \pflm$。
\begin{xiaoxiaotis}

    \xxt{求 $S$ 与 $h$ 的函数关系式;}

    \xxt{求当高 $h = 5 \; \limi$ 时的底面积 $S$。}

\resetxxt
\jie  \xxt{圆柱体积不变时,它的底面积 $S$ 与高 $h$ 成反比例,所以}
$$ S = \dfrac{k}{h} \juhao $$

把 $h = 12.5$,$S = 20$ 代入,得
$$ 20 = \dfrac{k}{12.5} \nsep k = 250 \juhao $$

答:所求的函数关系式是 $S = \dfrac{250}{h}$。

\xxt{当 $h = 5 \; (\limi)$ 时,}
$$ S = \dfrac{250}{h} = \dfrac{250}{5} = 50 \; (\pflm) \juhao $$

答:高是 $5 \; \limi$ 时,底面积是 $50 \; \pflm$。

\end{xiaoxiaotis}
\endgroup

\end{enhancedline}
