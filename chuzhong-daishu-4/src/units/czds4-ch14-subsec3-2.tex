\subsubsection{函数}

在上面的例 (1) 中,时间 $t$ 的值可以在非负实数(即正实数与零)的范围内任意选取,
对于 $t$ 的每一个确定的值,路程 $s$ 都有唯一确定的值与它对应,如:

\begin{table}[H]
    \centering
    \begin{tblr}{
        hlines, vlines,
        column{1} = {4em, c},
        column{2-7}={mode=math, 3em, c}
    }
        $t$(时)  & 1  & 1.5  & 2   & 2.5 & 3   & \cdots \\
        $s$(公里)& 60 & 90   & 120 & 150 & 180 & \cdots
    \end{tblr}
\end{table}

同样, 在例 (2) 中,半径 $r$ 的值可以在正实数的范围内任意选取,
对于半径 $r$ 的每一个确定的值,圆面积 $A$ 都有唯一确定的值与它对应。

这种变量之间的对应关系,在工农业生产和科学实验中大量存在。除了例 (1)、例 (2) 外,又如:

(3) 某水库的存水量 $Q$ 与水深 $h$(指最深处的水深)之间的对应关系,经过测量如下表所示:

\begin{table}[H]
    \centering
    \begin{tblr}{
        hlines, vlines,
        column{1} = {8em, c},
        column{2-9}={mode=math, 3.5em, c, colsep=0pt},
        row{2}={m},
    }
        水深 $h$(米)
            & 0 &  5  & 10 & 15 & 20  & 25  & 30    & 35 \\
        {存水量 $Q$ \\(万立方米)}
            & 0 & 20  & 40 & 90 & 160 & 275 & 437.5 & 650
    \end{tblr}
\end{table}

有了这张表后,水深 $h$ 的值可以在表内第一行各值中任意选取,
对于 $h$ 的每一个确定的值,存水量 $Q$ 都有唯一确定的值与它对应。例如;
$h = 20$ (米)时,$Q = 160$(万立方米);
$h = 30$ (米)时,$Q = 437.5$(万立方米)。

(4) 图 \ref{fig:14-11} 是某气象站用自动温度记录仪描下的表示某一天气温变化情况的曲线。

\begin{figure}[htbp]
    \centering
    \begin{tikzpicture}[>=Stealth,
    every node/.style={fill=white, inner sep=1pt},
]
    \pgfmathsetmacro{\factor}{0.2}
    \draw [->] (-0.5, 0) -- (5.5, 0) node[right=0.1] {$t$(时)}  coordinate(x axis);
    \draw [->] (0, -0.5) -- (0, 3.5) node[left=0.2em] {$T$(℃)} coordinate(y axis);
    \draw (0, 0) coordinate(O) node [below left] {\small $0$};
    \foreach \x in {2, 4, ..., 24} {
        \draw (\factor * \x, 0.1) -- (\factor * \x, 0) node [below] {\small $\x$};
    }
    \foreach \y in {2, 4, ..., 14} {
        \draw (0.1, \factor * \y) -- (0, \factor * \y) node[left] {\small $\y$};
    }

    \coordinate (a) at ($\factor*(0, 5)$);
    \coordinate (b) at ($\factor*(4, 1.8)$);
    \coordinate (c) at ($\factor*(14, 11.8)$);
    \coordinate (d) at ($\factor*(24, 7)$);

    % 随意绘制的曲线
    \draw
        (a) to [bend left] (b)
        plot [smooth] coordinates{
            (b)
            ($\factor*(6, 3.3)$)
            ($\factor*(8, 4.2)$)
            ($\factor*(10, 6.0)$)
            ($\factor*(10.5, 7.5)$)
            ($\factor*(12, 9.5)$)
            (c)
            ($\factor*(16, 11.7)$)
            ($\factor*(18, 11.5)$)
            ($\factor*(20, 9.8)$)
            ($\factor*(22, 8.5)$)
            (d)
        };

    % 几条辅助线
    \draw [dashed] (b -| y axis) -- (b) -- (b |- x axis);
    \draw [dashed] (c -| y axis) -- (c) -- (c |- x axis);
    \draw [dashed] (d) -- (d |- x axis);
\end{tikzpicture}


    \caption{}\label{fig:14-11}
\end{figure}

图 \ref{fig:14-11} 形象地反映了变量 $T$ 与 $t$ 之间的对应关系。
有了这幅图后,时间 $t$ 的值可以在 $0$ 到 $24$ 的范围内任意选取,
对于时间 $t$ 的每一个确定的值,气温 $T$ 都有唯一确定的值与它对应。如
$t =  4$(时)时,$T = 1.8$(℃);
$t = 14$(时)时,$T = 11.8$(℃)。

设在某变化过程中有两个变量 $x$,$y$,如果
对于 $x$ 在某一范围内的每一个确定的值,$y$ 都有唯一确定的值与它对应,
那么就说 $y$ 是 $x$ 的\zhongdian{函数},$x$ 叫做\zhongdian{自变量}。
例如,路程 $s$ 是时间 $t$ 的函数,圆面积 $A$ 是半径 $r$ 的函数,
存水量 $Q$ 是水深 $h$ 的函数,气温 $T$ 是时间 $t$ 的函数。

\begin{enhancedline}
我们看到,$60\,t$ 和 $\pi r^2$ 都是含一个字母的代数式。
一般地说,含一个字母的代数式的值,是由这个字母所取的值确定的;
这个字母的值,只要不使代数式和实际问题失去意义,可以任意选取。
对于这个字母的每一个确定的值,代数式都有唯一确定的值与它对应。
因此,每一个含一个字母的代数式都是这个字母的函数。
例如,如 $x - 2$ 是 $x$ 的函数,$\dfrac{1}{1 - u^2}$ 是 $u$ 的函数,
$\dfrac{1}{\sqrt{t^2 - 5}}$ 是 $t$ 的函数,等等。


\liti 求下列函数中自变量的取值范围:
\begin{xiaoxiaotis}

    \begin{tblr}{columns={18em, colsep=0pt}}
        \xxt{$y = 2x + 3$;}            & \xxt{$y = -3x^2$;} \\
        \xxt{$y = \dfrac{1}{x - 1}$;}  & \xxt{$y = \sqrt{x - 2}$。}
    \end{tblr}

\resetxxt
解. \xxt{$x$ 取任意实数,$2x + 3$ 都有意义。因此 $x$ 的取值范围是全体实数。}

\xxt{$x$ 取任意实数,$-3x^2$ 都有意义。因此 $x$ 的取值范围是全体实数。}

\xxt{$x = 1$ 时,$\dfrac{1}{x - 1}$ 没有意义;
    $x \neq 1$ 时,$\dfrac{1}{x - 1}$ 都有意义。
    因此 $x$ 的取值范围是所有不等于 $1$ 的实数。
}

\xxt{$x < 2$ 时, $\sqrt{x - 2}$ 没有意义;
    $x \geqslant 2$ 时,$\sqrt{x - 2}$ 都有意义。
    因此 $x$ 的取值范围是所有大于或等干 $2$ 的实数,即 $x \geqslant 2$。
}

\end{xiaoxiaotis}
\end{enhancedline}

\zhuyi 在函数 $A = \pi r^2$ 中,如果仅从代数式考虑,
$\pi r^2$ 中字母 $r$ 的取值范围可以是全体实数,
但从实际问题考虑, $\pi r^2$ 中的 $r$ 表示圆的半径,
那么它的取值范围就只能是大于零的实数,即 $r > 0$。
所以遇到实际问题时,确定函数的自变量取值范围,必须使实际问题也有意义。



\liti 在例 1 中,求当 $x = 2$ 时函数 $y$ 的对应值。

分析:例 1 中的函数当 $x = 2$ 时都有意义,只要用 $2$ 代替式中的 $x$,就可得到 $y$ 的对应值。

\begin{xiaoxiaotis}

\jie \begin{tblr}[t]{}
    \xxt{$y = 2 \times 2 + 3 = 7$;} \\
    \xxt{$y = -3 \times 2^2 = -12$;} \\
    \xxt{$y = \dfrac{1}{2 - 1} = 1$;} \\
    \xxt{$y = \sqrt{2 - 2} = 0$。}
\end{tblr}

\end{xiaoxiaotis}

对于自变量在取值范围内的一个确定的值,例如 $x = a$,函数有唯一确定的对应值。
这个对应值,我们叫做当 $x = a$ 时的函数的值,简称\zhongdian{函数值}。
如例 2,就是求当 $x = 2$ 时的函数值。


\lianxi
\begin{xiaotis}
\begin{enhancedline}

\xiaoti{(口答)在下面的等式里,有哪些变量、常量或常数?}
\begin{xiaoxiaotis}

    \xxt{匀速运动公式 $s = vt$,这里 $v$ 表示速度,$t$ 表示时间,$s$ 表示在时间 $t$ 内所走的路程;}

    \xxt{球体识公式 $V = \dfrac{4}{3}\pi r^3$,这里 $r$ 表示球的半径,$V$ 表示半径是 $r$ 的球的体积;}

    \xxt{正 $n$ 边形的内角 $\alpha$ 与边数 $n$ 之问的对应关系
        $$ \alpha = \dfrac{(n - 2) 180^\circ}{n} \juhao $$
    }

\end{xiaoxiaotis}

\xiaoti{求下列函数中自变量 $x$ 的取值范围:}
\begin{xiaoxiaotis}

    \begin{tblr}{columns={18em, colsep=0pt}, rows={rowsep=0.5em}}
        \xxt{$y = \dfrac{x - 1}{2}$;}  & \xxt{$y = \dfrac{3}{x - 4}$;} \\
        \xxt{$y = -\sqrt{x - 5}$;}     & \xxt{$y = \dfrac{1}{x^2 - x - 2}$。}
    \end{tblr}
\end{xiaoxiaotis}
\end{enhancedline}


\xiaoti{在第 2 题中求当 $x = 9$,$x = 30$ 时的函数值。}

\end{xiaotis}

