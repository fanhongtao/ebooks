\fuxiti
\begin{xiaotis}

\xiaoti{\footnote{第 1 题至第 10 题是复习前三册代数内容的题目,可以在学习本书的过程中,根据情况选用。}
    $a$ 是什么数时,下列各式成立?
}
\begin{xiaoxiaotis}

    \begin{tblr}{columns={18em, colsep=0pt}}
        \xxt{$|a|= a$;}     & \xxt{$|a| = -a$;} \\
        \xxt{$|a| = |-a|$;} & \xxt{$a = -a$。}
    \end{tblr}
\end{xiaoxiaotis}


\xiaoti{}%
\begin{xiaoxiaotis}%
    \xxt[\xxtsep]{设 $(x - 3)^2 + (y + 1)^2 = 0$,且 $x$,$y$ 为实数,求 $x$,$y$;}

    \xxt{设 $x^2 + 4y^2 - 2x + 4y + 2 = 0$,且 $x$,$y$ 为实数,求 $x$,$y$。}

\end{xiaoxiaotis}


\xiaoti{计算:}
\begin{xiaoxiaotis}

    \xxt{$(a + b - c) (a - b + c) (-a + b + c) (a + b + c)$;}

    \xxt{$(a + b - c - d) (a - b - c + d)$;}

    \xxt{$(x - y) (2x + y) (x^2 + xy + y^2) (4x^2 - 2xy + y^2)$;}

    \xxt{$(x - 1) (x - 2) (x - 3) (x - 4)$;}

    \xxt{$(a - 2b)^2 (a + 2b)^2$;}

    \xxt{$(a - 2b) (a + 2b) (a^2 + 2ab + 4b^2) (a^2 - 2ab + 4b^2)$。}

\end{xiaoxiaotis}


\xiaoti{分解因式:}
\begin{xiaoxiaotis}

    \begin{tblr}{columns={18em, colsep=0pt}}
        \xxt{$a^2 + b^2 + c^2 + 2ab + 2bc + 2ac$;} & \xxt{$a^4b + a^3b^2 - a^2b^3 - ab^4$;} \\
        \xxt{$x^4 - 5x^2 + 4$;} & \xxt{$4mn + 1 - m^2 - 4n^2$;} \\
        \xxt{$m^4 + m^2 + 1$;} & \xxt{$a^3 - b^3 + a(a^2 - b^2) + b(a - b)^2$;} \\
        \xxt{$2(a^2 + b^2) (a + b)^2 - (a^2 - b^2)^2$。}
    \end{tblr}
\end{xiaoxiaotis}


\begin{enhancedline}
\xiaoti{}%
\begin{xiaoxiaotis}%
    \xxt[\xxtsep]{已知 $x + y + z = a$,$xy + yz + zx = b$,求 $x^2 + y^2 + z^2$;}

    \xxt{已知 $x + \dfrac{1}{x} = 5$,求 $x^2 + \dfrac{1}{x^2}$。}

\end{xiaoxiaotis}


\xiaoti{化简:}
\begin{xiaoxiaotis}

    \xxt{$\dfrac{1}{x + 1} - \dfrac{1}{x + 2} - \dfrac{x + 3}{(x + 1)(x + 2)}$;}

    \xxt{$\left(\dfrac{a}{a + b} - \dfrac{a^2}{a^2 + 2ab + b^2}\right) \div \left(\dfrac{a}{a + b} - \dfrac{a^2}{a^2 - b^2}\right)$;}

    \begin{tblr}{columns={18em, colsep=0pt}, rows={rowsep=0.5em}}
    \xxt{$\dfrac{a}{1 + \dfrac{1}{a}} - \dfrac{1}{a + 1} + 1$;}
        & \xxt{$\dfrac{\sqrt{20}}{8} - \sqrt{\dfrac{9}{5}} + \dfrac{5}{\sqrt{45}}$;} \\
    \xxt{$\dfrac{1}{3 + \sqrt{7}} - \dfrac{3}{2 - \sqrt{7}} - \dfrac{\sqrt{7} - 5}{2}$;}
        & \xxt{$\dfrac{\sqrt[\uproot{10}3]{5^{-\frac{3}{2}} \left(\dfrac{1}{5}\right)^{-2}}}{(\sqrt{5} - 1)^2}$。}
    \end{tblr}

\end{xiaoxiaotis}


\xiaoti{解方程:}
\begin{xiaoxiaotis}

    \begin{tblr}{columns={18em, colsep=0pt}, rows={rowsep=0.5em}}
        \xxt{$42x^2 + x - 30 = 0$;} & \xxt{$4\left(\dfrac{3}{x} + 1\right)^2 - 9 = 0$;} \\
        \xxt{$x^2 - \sqrt{3}x + \sqrt{2}x - \sqrt{6} = 0$;} & \xxt{$\dfrac{1}{2 - x} - 1 = \dfrac{1}{x - 2} - \dfrac{6 - x}{3x^2 - 12}$;} \\
        \xxt{$3x^4 - 29x^2 + 18 = 0$;} & \xxt{$\sqrt{x + 1} - \sqrt{x - 4} = \sqrt{3x + 1}$。}
    \end{tblr}
\end{xiaoxiaotis}


\xiaoti{解方程组:}
\begin{xiaoxiaotis}

    \xxt{$\begin{cases}
        cx + y = 2c + 1 \douhao \\
        x - cy= 2 - c
    \end{cases} (c \neq 0)$,求 $x$,$y$;}

    \xxt{$\begin{cases}
        a_1x + b_1y = c_1 \douhao \\
        a_2x + b_2y = c_2
    \end{cases} (a_1b_2 \neq a_2b_1)$,求 $x$,$y$;}

    \begin{tblr}{columns={18em, colsep=0pt}}
        \xxt{$\begin{cases}
                \dfrac{5x + 3y}{4} + \dfrac{y + 2x}{3} = 3 \douhao \\[1em]
                \dfrac{5x + 3y}{4} - \dfrac{y + 2x}{3} = 1 \fenhao
            \end{cases}$} & \xxt{$\begin{cases}
                \dfrac{2}{x - 3} + \dfrac{5}{2y + 3} = -4 \douhao \\[1em]
                \dfrac{6}{x - 3} - \dfrac{2}{2y + 3} = 5 \fenhao
            \end{cases}$} \\
        \xxt{$\begin{cases}
                5x - 7y = -10 \douhao \\
                9y + 4z = 1 \douhao \\
                3x + 8z = -4 \fenhao
            \end{cases}$} & \xxt{$\begin{cases}
                x^2 - xy + 2 = 0 \douhao \\
                2x - y = 1 \fenhao
            \end{cases}$} \\
        \xxt{$\begin{cases}
            3x^2 + 5x - 8y = 36 \douhao \\
            2x^2 - 3x - 4y = 3 \juhao
        \end{cases}$}
    \end{tblr}

\end{xiaoxiaotis}


\xiaoti{$k$ 为什么值时,方程 $(k - 1)x^2 - 2x + 3 = 0$ 有两个不相等的实数根?有两个相等的实数根?没有实数根?}

\xiaoti{求方程 $x^2 + px + q = 0$ 的两根的平方和及两根的立方和。}

\begin{center}
    * \hspace*{6em} * \hspace*{6em} *
\end{center}

\xiaoti{求下列各式中的 $x$:}
\begin{xiaoxiaotis}

    \begin{tblr}{columns={18em, colsep=0pt}}
        \xxt{$64^x = \dfrac{1}{4}$;} & \xxt{$2^x = 0.125$。}
    \end{tblr}
\end{xiaoxiaotis}


\xiaoti{求下列各式中的 $x$:}
\begin{xiaoxiaotis}

    \begin{tblr}{columns={18em, colsep=0pt}}
        \xxt{$\log_{8}{x} = 1$;} & \xxt{$\log_{2}{\sqrt{2}} = x$;} \\
        \xxt{$\log_{x}{8} = \dfrac{3}{2}$;} & \xxt{$\log_{7}{x} = 0$。}
    \end{tblr}
\end{xiaoxiaotis}


\xiaoti{下面的式子对不对(其中 $a > 0$ 且 $a \neq 1$, $x > 0$, $y > 0$, $n$ 是大于 $1$ 的整数)?
    如果不对,举出例子。
}
\begin{xiaoxiaotis}

    \begin{tblr}{columns={18em, colsep=0pt}}
        \xxt{$(\log_{a}{x})^2 = 2\log_{a}{x}$;} & \xxt{$-\log_{a}{x} = \log_{a}{\dfrac{1}{x}}$;} \\
        \xxt{$\dfrac{\log_{a}{x}}{\log_{a}{y}} = \log_{a}{\dfrac{x}{y}}$;} & \xxt{$\dfrac{\log_{a}{x}}{n} = \log_{a}{\sqrt[n]{x}}$。}
    \end{tblr}
\end{xiaoxiaotis}


\xiaoti{}%
\begin{xiaoxiaotis}%
    \xxt[\xxtsep]{$\lg{100N}$ 比 $\lg{\dfrac{N}{100}}$ 大多少?}

    \xxt{$\lg{0.001N}$ 比 $\lg{1000N}$ 小多少?}

\end{xiaoxiaotis}
\end{enhancedline}


\xiaoti{已知 $\lg{2} = 0.3010$, $\lg{3} = 0.4771$,求 $\lg{0.0015}$, $\lg{750}$。}

\xiaoti{查表求下列各式中的 $x$:}
\begin{xiaoxiaotis}

    \begin{tblr}{columns={12em, colsep=0pt}}
        \xxt{$2 = 10^x$;} & \xxt{$8.5 = 10^x$;} & \xxt{$7.04 = 10^x$;} \\
        \xxt{$95.02 = 10^x$;} & \xxt{$0.0317 = 10^x$;} & \xxt{$4283960 = 10^x$。}
    \end{tblr}
\end{xiaoxiaotis}


\xiaoti{查表求下列各式中的 $x$:}
\begin{xiaoxiaotis}

    \begin{tblr}{columns={12em, colsep=0pt}}
        \xxt{$\lg{x} = \sqrt{2}$;} & \xxt{$\lg{x} = -\pi$;} & \xxt{$2\lg{x} = \overline{3}.1066$。}
    \end{tblr}
\end{xiaoxiaotis}


\xiaoti{求下列指数式的值:}
\begin{xiaoxiaotis}

    \begin{tblr}{columns={12em, colsep=0pt}}
        \xxt{$10^{0.4475}$;} & \xxt{$10^{2.4475}$;} & \xxt{$10^{-2.4475}$。}
    \end{tblr}
\end{xiaoxiaotis}


\begin{enhancedline}
\xiaoti{已知 $a = 35.72$, $b = 28.17$, $c = 30.45$, $s = \dfrac{1}{2}(a + b + c)$,
    求 $\sqrt{s (s - a) (s - b) (s - c)}$。
}


\xiaoti{}%
\begin{xiaoxiaotis}%
    \xxt[\xxtsep]{确定 $2^{100}$ 是几位数,并且求出它的最高的两位数字;}

    \xxt{确定 $0.7^{100}$ 在小数点后面连续有多少个零,并且求出它的第一个不等于零的数字。}

\end{xiaoxiaotis}
\end{enhancedline}


\xiaoti{要使我国工农业的年总产值经过 $20$ 年翻两番(即由 $1980$ 年的 $7100$ 亿元增加到 $2000$ 年的 $28000$ 亿元左右),
    平均每年的增长率是百分之几?
}

\xiaoti{一些机器设备,原来价值 $72$ 万元,由丁使用磨损,每年比上一年价值降低 $5.5\%$,
    求 $5$ 年后这些机器设备的价值。
}

\end{xiaotis}

