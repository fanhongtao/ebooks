\subsection{方差的简化计算\footnote{方差简化计算公式的推导不作为要求。}}\label{subsec:16-4}
\begin{enhancedline}

按照公式 (3) 计算样本方差,计算量比较大,下面介绍方差的简化计算方法。

为了简化讨论,我们假定样本中仅包含 $3$ 个数据 $x_1$,$x_2$,$x_3$,
它们的平均数是 $\overline{x}$,那么样本方差是
$$ s^2 = \exdfrac{1}{3} [(x_1 - \overline{x})^2 + (x_2 - \overline{x})^2 + (x_3 - \overline{x})^2] \juhao $$

将方括号内的各项展开后再整理,得到
\begin{align*}
    s^2 &= \exdfrac{1}{3} [(x_1^2 - 2x_1\overline{x} + \overline{x}^2) + (x_2^2 - 2x_2\overline{x} + \overline{x}^2) + (x_3^2 - 2x_3\overline{x} + \overline{x}^2)] \\
        &= \exdfrac{1}{3} [(x_1^2 + x_2^2 + x_3^2) - 2(x_1 + x_2 + x_3)\overline{x} + 3\overline{x}^2] \\
        &= \exdfrac{1}{3} [(x_1^2 + x_2^2 + x_3^2) - 2 \times 3 \times \dfrac{(x_1 + x_2 + x_3)}{3} \overline{x} + 3\overline{x}^2] \\
        &= \exdfrac{1}{3} [(x_1^2 + x_2^2 + x_3^2) - 2 \times 3 \overline{x}^2 + 3\overline{x}^2] \\
        &= \exdfrac{1}{3} [(x_1^2 + x_2^2 + x_3^2) - 3\overline{x}^2] \juhao
\end{align*}

一般地,如果样本的容量是 $n$,那么样本方差可以用下面的公式计算:

\begin{gather*} % 因为在方框外有公式编号,所以这里采用手工绘制方框的方式。
    \tikz[overlay, >=Stealth] {
        \draw (-1.5em, -1.0em) rectangle (17em, 1.7em);
    }s^2  = \exdfrac{1}{n} [(x_1^2 + x_2^2 + \cdots + x_n^2) - n\overline{x}^2] \douhao \tag{5}
\end{gather*}
或简记作
\begin{gather*} % 因为在方框外有公式编号,所以这里采用手工绘制方框的方式。
    \tikz[overlay, >=Stealth] {
        \draw (-1.5em, -1.4em) rectangle (12em, 2.0em);
    }s^2  = \exdfrac{1}{n} (\sum_{i=1}^n x_i^2 - n\overline{x}^2) \juhao \tag{5}
\end{gather*}\vspace*{0.5em}

用公式 (5) 计算样本方差,是直接计算各个数据的平方,而不必计算各个数据与样本平均数的差的平方,
因此它比用公式  (3) 计算少一个步骤,有时比较方便。


\liti 计算下面样本的方差(结果保留到小数点后第一位):
$$ 3 \quad -1 \quad 2 \quad 1 \quad -3 \quad 3 $$

\jie 从计算知道,样本平均数不是整数。这时用公式 (3) 计算样本方差比较麻烦,我们用公式 (5) 来计算。

\begin{align*}
    s^2 &= \exdfrac{1}{6} \left[ 3^2 + (-1)^2 + 2^2 + 1^2 + (-3)^2 + 3^2 - 6 \times \left(\dfrac{3 - 1 + 2 + 1 - 3 + 3}{6}\right)^2\right] \\
        &= \exdfrac{1}{6} \left[ 9 + 1 + 4 + 1 + 9 + 9 - 6 \times \left(\dfrac{5}{6}\right)^2\right] \\
        &= \exdfrac{1}{6} \left(33 - 6 \times \dfrac{25}{36}\right) \\
        &= \exdfrac{1}{6} \times 33 - \dfrac{25}{36} \\
        &\approx 5.5 - 0.7 = 4.8 \juhao
\end{align*}


\lianxi

计算下面样本的方差(结果保留到小数点后第一位):
$$ 5 \quad 4 \quad 4 \quad 3 \quad 4 \quad 3 \quad 2 \quad 3 \quad 5 \quad 3 $$

\lianxijiange


当样本数据较大时,用公式 (5) 计算样本方差仍然比较麻烦。
如果数据相互比较接近,为了减小参与计算的数据,我们可以仿照前面简化计算平均数的办法,
将每个数据同时减去一个与样本平均数接近的常数 $a$。
假定样本中包含 $3$ 个数据 $x_1$,$x_2$,$x_3$,我们作替换
$$ x_1' = x_1 - a \nsep x_2' = x_2 - a \nsep x_3' = x_3 - a \douhao $$
那么,
$$ x_1 = x_1' + a \nsep x_2 = x_2' + a \nsep x_3 = x_3' + a \douhao $$
又根据公式 (2),有
$$ \overline{x} = \overline{x'} + a \douhao $$
其中 $\overline{x'}$ 是 $x_1'$,$x_2'$ 和 $x_3'$ 的平均数,于是,
\begin{align*}
    s^2 &= \exdfrac{1}{3} [(x_1 - \overline{x})^2 + (x_2 - \overline{x})^2 + (x_3 - \overline{x})^2] \\
        &= \exdfrac{1}{3} \{[(x_1' + a) - (\overline{x'} + a)]^2
            + [(x_2' + a) - (\overline{x'} + a)]^2
            + [(x_3' + a) - (\overline{x'} + a)]^2\} \\
        &= \exdfrac{1}{3} [(x_1' - \overline{x'})^2 + (x_2' - \overline{x'})^2 + (x_3' - \overline{x'})^2] \\
        &= \exdfrac{1}{3} [(x_1'^2 + x_2'^2 + x_3'^2) - 3\overline{x'}^2] \juhao \quad \text{(根据公式 (5))}
\end{align*}


一般地,如果样本的容量是 $n$,那么有

\begin{gather*} % 因为在方框外有公式编号,所以这里采用手工绘制方框的方式。
    \tikz[overlay, >=Stealth] {
        \draw (-1.5em, -1.0em) rectangle (17em, 1.7em);
    }s^2  = \exdfrac{1}{n} [(x_1'^2 + x_2'^2 + \cdots + x_n'^2) - n\overline{x'}^2] \douhao \tag{6}
\end{gather*}
其中
$$ x_1' = x_1 - a \nsep x_2' = x_2 - a \nsep \cdots \nsep x_n' = x_n - a \douhao $$
这里 $a$ 是接近样本平均数的常数。

上式可以简记为
\begin{gather*} % 因为在方框外有公式编号,所以这里采用手工绘制方框的方式。
    \tikz[overlay, >=Stealth] {
        \draw (-1.5em, -1.4em) rectangle (12em, 2.0em);
    }s^2  = \exdfrac{1}{n} (\sum_{i=1}^n x_i'^2 - n\overline{x'}^2) \juhao \tag{6}
\end{gather*}\vspace*{0.5em}

由于在替换 $x_i' = x_i - a \; (i = 1, 2, \cdots, n)$ 中所取的常数 $a$ 在与样本平均数 $\overline{x}$ 比较接近,
各数据 $x_i'$ 比较小,因此当样本数据较大时,用公式 (6) 计算样本方差比较简便。



\liti 从小麦地里抽取了甲、乙两种小麦各 $10$ 株,测得株高如下(单位:厘米):
\begin{data}
    \begin{datatblr}{column{1}={mode=text}, columns={2em}}
        甲 & 76 & 90 & 84 & 86 & 81 & 87 & 86 & 82 & 85 & 83 \\
        乙 & 82 & 84 & 85 & 89 & 79 & 80 & 91 & 89 & 79 & 74
    \end{datatblr}
\end{data}
哪种小麦长得比较整齐?

\jie 根据题意,是要比较两个样本方差的大小。
因为样本数据较大,我们用公式 (6) 来计算两个样本方差,步骤如下:

(1) 确定替换 $x_i' = x_i - a$ 中的常数 $a$。

由于样本数据都在 $80$ 左右波动,我们取 $a = 80$。

(2) 如表 \ref{tab:16-1} 、表 \ref{tab:16-2} 所示,分别计算各个 $x_i'$ 及 $x_i'^2 \; (i = 1, 2, \cdots, 10)$,
然后计算各个 $x_i'$ 的和以及各个 $x_i'^2$ 的和,并填入表中。

\begin{table}[H]
    \newcommand{\btr}{{$x_i'$ \\[-0.5em] $(x_i - 80)$}}
    \newcommand{\hj}{\text{合计}}
    \centering
    \begin{minipage}{7cm}
        \centering
        \caption{(甲种小麦)}\label{tab:16-1}
        \begin{tblr}{hlines, vlines, columns={4em}, rows={mode=math, c, m}, cell{1}{2}={mode=text}}
            x_i & \btr & x_i'^2 \\
            76  &  -4  &  16  \\
            90  &  10  & 100  \\
            84  &   4  &  16  \\
            86  &   6  &  36  \\
            81  &   1  &   1  \\
            87  &   7  &  49  \\
            86  &   6  &  36  \\
            82  &   2  &   4  \\
            85  &   5  &  25  \\
            83  &   3  &   9  \\
            \hj &  40  & 292
        \end{tblr}
    \end{minipage}
    \qquad
    \begin{minipage}{7cm}
        \centering
        \caption{(乙种小麦)}\label{tab:16-2}
        \begin{tblr}{hlines, vlines, columns={4em}, rows={mode=math, c, m}, cell{1}{2}={mode=text}}
            x_i & \btr & x_i'^2 \\
            82  &   2  &   4  \\
            84  &   4  &  16  \\
            85  &   5  &  25  \\
            89  &   9  &  81  \\
            79  &  -1  &   1  \\
            80  &   0  &   0  \\
            91  &  11  & 121  \\
            89  &   9  &  81  \\
            79  &  -1  &   1  \\
            74  &  -6  &  36  \\
            \hj &  32  & 366
        \end{tblr}
    \end{minipage}
\end{table}


(3) 将表中有关数据代入公式 (6) 进行计算:
\begin{align*}
    s_\text{甲}^2 &= \dfrac{1}{10} \left( \sum_{i=1}^{10} x_i'^2 - 10\overline{x'}^2 \right) \\
                  &= \dfrac{1}{10} \left[ 292 - 10 \times \left(\dfrac{40}{10}\right)^2 \right] \\
                  &= \dfrac{1}{10} [292 - 160] = \dfrac{1}{10} \times 132 \\
                  &= 13.2 \fenhao
\end{align*}

\begin{align*}
    s_\text{乙}^2 &= \dfrac{1}{10} \left( \sum_{i=1}^{10} x_i'^2 - 10\overline{x'}^2 \right) \\
                  &= \dfrac{1}{10} \left[ 366 - 10 \times \left(\dfrac{32}{10}\right)^2 \right] \\
                  &= \dfrac{1}{10} [366 - 102.4] = \dfrac{1}{10} \times 263.6 \\
                  &= 26.36 \juhao
\end{align*}

因为 $s_\text{甲}^2 < s_\text{乙}^2$,可以估计,甲种小麦比乙种小麦长得整齐。



\liti 一个农科站在 $8$ 个面积相等的试验点对某两个早稻品种进行栽培对比试验,
两个品种在各试验点的产量如下(单位:$500$ 克):
\begin{data}
    \begin{datatblr}{column{1}={mode=text}, columns={2em}}
        甲 & 804 & 984 & 989 & 817 & 919 & 840 & 912 & 1001 \\
        乙 & 856 & 932 & 930 & 855 & 872 & 910 & 897 & 918
    \end{datatblr}
\end{data}
哪个品种的产量比较稳定?

\jie 根据题意,是要比较两个样本方差的大小。
由于样本数据较大,我们仿照例 2 的步骤,根据公式 (6) 来计算两个样本方差。

(1) 由于两个样本中的数据都在 $900$ 左右波动,我们在替换 $x_i' = x_i - a$ 中,取 $a = 900$。

(2) 如 表 \ref{tab:16-3}、表 \ref{tab:16-4} 所示,分别计算各个 $x_i'$ 及 $x_i'^2$,
然后计算各个 $x_i'$ 的和以及各个 $x_i'^2$ 的和,并填入表中。

\begin{table}[H]
    \newcommand{\btr}{{$x_i'$ \\[-0.5em] $(x_i - 900)$}}
    \newcommand{\hj}{\text{合计}}
    \centering
    \begin{minipage}{7cm}
        \centering
        \caption{(甲种水稻)}\label{tab:16-3}
        \begin{tblr}{hlines, vlines, columns={4em}, rows={mode=math, c, m}, cell{1}{2}={mode=text}}
            x_i  & \btr & x_i'^2 \\
            804  & -96  &  9216  \\
            984  &  84  &  7056  \\
            989  &  89  &  7921  \\
            817  & -83  &  6889  \\
            919  &  19  &   361  \\
            840  & -60  &  3600  \\
            912  &  12  &   144  \\
            1001 & 101  & 10201  \\
            \hj  &  66  & 45388
        \end{tblr}
    \end{minipage}
    \qquad
    \begin{minipage}{7cm}
        \centering
        \caption{(乙种水稻)}\label{tab:16-4}
        \begin{tblr}{hlines, vlines, columns={4em}, rows={mode=math, c, m}, cell{1}{2}={mode=text}}
            x_i  & \btr & x_i'^2 \\
            856  & -44  &  1936  \\
            932  &  32  &  1024  \\
            930  &  30  &   900  \\
            855  & -45  &  2025  \\
            872  & -28  &   784  \\
            910  &  10  &   100  \\
            897  &  -3  &     9  \\
            918  &  18  &   324  \\
            \hj  & -30  &  7102
        \end{tblr}
    \end{minipage}
\end{table}


(3) 将表中有关数据代入公式 (6) 进行计算:

\begin{align*}
    s_\text{甲}^2 &= \exdfrac{1}{8} \left( \sum_{i=1}^{8} x_i'^2 - 8\overline{x'}^2 \right) \\
                  &= \exdfrac{1}{8} \left[ 45388 - 8 \times \left(\dfrac{66}{8}\right)^2 \right] \\
                  &= \exdfrac{1}{8} [45388 - 554.5] \\
                  &= \exdfrac{1}{8} \times 44843.5 \\
                  &\approx 5605 \fenhao
\end{align*}


\begin{align*}
    s_\text{乙}^2 &= \exdfrac{1}{8} \left( \sum_{i=1}^{8} x_i'^2 - 8\overline{x'}^2 \right) \\
                  &= \exdfrac{1}{8} \left[ 7102 - 8 \times \left(\dfrac{-30}{8}\right)^2 \right] \\
                  &= \exdfrac{1}{8} [7102 - 112.5] \\
                  &= \exdfrac{1}{8} \times 6989.5 \\
                  &\approx 874 \fenhao
\end{align*}

因为 $s_\text{乙}^2 < s_\text{甲}^2$,可以估计,乙种水稻比甲种水稻的产量稳定。


我们看到,样本平均数和方差的计算往往较烦。
为便于计算,所取的样本容量应尽可能 “整” 一点。如 $10$, $20$, $30$ 等。


\lianxi

计算下列各样本方差(结果保留到小数点后第一位):

\begin{xiaoxiaotis}

    \xxt{105 \quad 103 \quad 101 \quad 100 \quad 114 \quad 108 \quad 110 \quad 106 \quad \phantom{0}98 \quad 102}

    \xxt{423 \quad 421 \quad 419 \quad 420 \quad 421 \quad 417 \quad 422 \quad 419 \quad 423 \quad 418}

\end{xiaoxiaotis}

\end{enhancedline}

