\subsection{解直角三角形}\label{subsec:15-5}

我们知道,直角三角形 $ABC$ 的六个元素(三条边和三个角),除直角 $C$ 外,其余五个元素之间有如下的关系:

(1) 三边之间的关系
$$ a^2 + b^2 = c^2 \quad \text{(勾股定理);} $$

(2) 锐角之间的关系
$$ A + B = 90^\circ \fenhao $$

(3) 边角之间的关系
\begin{table}[H]
    \centering
    \begin{tblr}{columns={mode=math}, rows={rowsep=0.5em}}
        \sin \alpha = \dfrac{\alpha \text{的对边}}{\text{斜边}} \douhao          & \cos \alpha = \dfrac{\alpha \text{的邻边}}{\text{斜边}} \douhao \\
        \tan \alpha = \dfrac{\alpha \text{的对边}}{\alpha \text{的邻边}} \douhao & \cot \alpha = \dfrac{\alpha \text{的邻边}}{\alpha \text{的对边}} \douhao
    \end{tblr}
\end{table}
式中的 $\alpha$ 表示锐角 $A$ 或锐角 $B$。

利用这些关系,知道其中的两个元素(至少有一个是边),就可以求出其余的三个未知元素。
就是说,解直角三角形的问题,只需知道直角三角形除直角外的两个元素(至少有一个是边)就可解决。举例说明如下。

\begin{wrapfigure}[10]{r}{5cm}
    \centering
    \begin{tikzpicture}
    \pgfmathsetmacro{\factor}{0.1}
    \pgfmathsetmacro{\b}{35 * \factor}
    \pgfmathsetmacro{\c}{45 * \factor}
    \pgfmathsetmacro{\a}{sqrt(\c*\c - \b*\b)}
    \pgfmathsetmacro{\r}{0.5}
    \pgfmathsetmacro{\jiaoa}{asin(\a/\c)}
    \pgfmathsetmacro{\jiaob}{asin(\b/\c)}

    \coordinate (A) at (0, 0);
    \coordinate (B) at (\b, \a);
    \coordinate (C) at (\b, 0);

    \draw (A) node [below] {$A$} to [chuizu] node [midway, below] {$35$} (C);
    \draw (C) node [below] {$C$} -- (B) node [midway, right] {$a$};
    \draw (B) node [above] {$B$} -- (A) node [midway, above, rotate=\jiaoa] {$45$};
    \draw (0:\r) arc (0:\jiaoa:\r);
    \draw [double] (\b, \a) + (270:\r) arc (270:(270 -\jiaob):\r);
\end{tikzpicture}


    \caption{}\label{fig:15-11}
\end{wrapfigure}

\liti 如图 \ref{fig:15-11} ,在直角三角形 $ABC$ 中,已知 $b = 35$,$c = 45$,
求 $A$,$B$(精确到 $1^\circ$)和 $a$(保留两个有效数字)。

\begin{enhancedline}
\jie (1) $\cos A = \exdfrac{b}{c} = \dfrac{35}{45} \approx 0.7778$,查表得 $A \approx 39^\circ$。

(2) $B = 90^\circ - A \approx 90^\circ - 39^\circ = 51^\circ$。

(3) $a = \sqrt{c^2 - b^2} = \sqrt{45^2 - 35^2} = \sqrt{(45 + 35)(45 - 35)} = \sqrt{800}$,\\
查平方根表得
$$ a = 28.28 \approx 28 \juhao $$

在例 1 中,由于已知边 $b$,$c$,所以求角 $A$ 时,我们选取含有角 $A$ 和边 $b$,$c$ 的关系式 $\cos A = \exdfrac{b}{c}$。


\liti 在直角三角形 $ABC$ 中,已知 $a = 15$,$A = 35^\circ27'$,求 $b$,$c$ 和 $B$(边长保留两个有效数字)。

\jie (1) \begin{tblr}[t]{columns={mode=math}}
    \because   & \cot A = \exdfrac{b}{a} \douhao \\
    \therefore & b = a \, \cot A = 15 \times \cot 35^\circ27' = 15 \times 1.4045 \approx 21 \juhao
\end{tblr}

(2) \begin{tblr}[t]{columns={mode=math}, rows={rowsep=0.5em}}
    \because   & \sin A = \exdfrac{a}{c} \douhao \\
    \therefore & c = \dfrac{a}{\sin A} = \dfrac{15}{\sin 35^\circ27'} = \dfrac{15}{0.5800} \approx 26 \juhao \\[.5em]
\end{tblr}

(3) $B = 90^\circ - 35^\circ27' = 54^\circ33'$。

在例 2 中,已知 $a$ 和 $A$。求 $b$ 时,虽然也可选用关系式 $\tan A = \exdfrac{a}{b}$,
但计算时需要用除法 $b = \dfrac{a}{\tan A}$,所以我们选用关系式 $\cot A = \exdfrac{b}{a}$。
\end{enhancedline}

\lianxi
\begin{xiaotis}

\xiaoti{在直角三角形 $ABC$ 中:}
\begin{xiaoxiaotis}

    \xxt{已知 $c$,$A$,写出求 $a$ 和 $b$ 的关系式。}

    \xxt{已知 $b$,$A$,写出求 $a$ 的关系式;已知 $a$,$A$,写出求 $b$ 的关系式。}

    \xxt{已知 $a$,$b$ 怎样求 $A$?已知 $a$,$c$ 怎样求 $A$?已知 $b$,$c$ 怎样求 $A$?}

\end{xiaoxiaotis}


\xiaoti{根据下列条件解直角三角形:}
\begin{xiaoxiaotis}

    \begin{tblr}{columns={18em, colsep=0pt}}
        \xxt{$c = 10 \nsep A = 30^\circ$;} & \xxt{$b = 15 \nsep B = 45^\circ$;} \\
        \xxt{$a = 51 \nsep c = 70$;}       & \xxt{$a = 22 \nsep b = 12$。}
    \end{tblr}

    \hspace*{1.5em} (第(3),(4)题要求角度精确到 $1^\circ$,边长保留两个有效数字。)

\end{xiaoxiaotis}

\end{xiaotis}
\lianxijiange


\begin{enhancedline}
由上面的例题可以看出,在解直角三角形时,经常要进行多位数的乘、除运算,特别是对已知角的三角函数值进行乘、除运算。
例如计算 $15 \times \cot 35^\circ27'$,$\dfrac{15}{\sin 35^\circ27'}$ 等等。
为了利用对数使计算简化,我们需要求得三角函数值的对数,
例如需要求 $\lg \cot 35^\circ27'$,$\lg \sin 35^\circ27'$ 等等。
利用“正弦对数和余弦对数表”、“正切对数和余切对数表”可以直接查出已知锐角的三角函数值的对数。
这些表的用法与“正弦和余弦表”、“正切和余切表”类似,可参看表中的说明。
\end{enhancedline}


\liti 查表求下列各三角函数值的对数:
\begin{xiaoxiaotis}

    \hspace*{1.5em} \begin{tblr}{columns={18em, colsep=0pt}}
        \xxt{$\lg{\sin 34^\circ16'}$;} & \xxt{$\lg{\cos 68^\circ38'}$;} \\
        \xxt{$\lg{\tan 5^\circ48'}$;}  & \xxt{$\lg{\cot 21^\circ59'}$。}
    \end{tblr}

\resetxxt
\jie \begin{tblr}[t]{columns={18em, colsep=0pt}}
    \xxt{$\lg{\sin 34^\circ16'} = \overline{1}.7505$;} & \xxt{$\lg{\cos 68^\circ38'} = \overline{1}.5615$;} \\
    \xxt{$\lg{\tan 5^\circ48'}  = \overline{1}.0068$;} & \xxt{$\lg{\cot 21^\circ59'} = 0.3940$。}
\end{tblr}

\end{xiaoxiaotis}


反过来,利用这些表还可以由一个锐角的三角函数值的对数,查出这个锐角。

\liti 求下列各式中的锐角 $\alpha$ :
\begin{xiaoxiaotis}

    \hspace*{1.5em} \begin{tblr}{columns={18em, colsep=0pt}}
        \xxt{$\lg{\sin \alpha} = \overline{2}.9104$;} & \xxt{$\lg{\cos \alpha} = \overline{1}.9169$;} \\
        \xxt{$\lg{\tan \alpha} = 0.0021$;}            & \xxt{$\lg{\cot \alpha} = \overline{1}.9884$。}
    \end{tblr}

\resetxxt
\jie \begin{tblr}[t]{columns={18em, colsep=0pt}}
    \xxt{$\alpha = 4^\circ40'$;} & \xxt{$\alpha = 34^\circ19'$;} \\
    \xxt{$\alpha = 45^\circ8'$;} & \xxt{$\alpha = 45^\circ46'$。}
\end{tblr}

\end{xiaoxiaotis}


下面我们举例说明利用对数解直角三角形的方法。

\liti 已知直角三角形 $ABC$ 中,$c = 287.4$,$B = 42^\circ6'$,解这个三角形。\footnote{
    在本章中,如无特别说明,解三角形时。角度精确到 $1'$,边长保留四个有效数字。
}

\jie (1) $A = 90^\circ - 42^\circ6' = 47^\circ54'$。

(2) \begin{tblr}[t]{columns={mode=math}, row{1}={abovesep=0.5em}}
    \because   & \cos B = \exdfrac{a}{c} \douhao \\
    \therefore & a = c \, \cos B \juhao \\
\end{tblr}

两边取对数得

\hspace*{2em} $\lg{a} = \lg{c} + \lg{\cos B}$。

\hspace*{2em} \begin{tblr}[t]{
    columns={r, mode=math, colsep=0em},
    rows={rowsep=0em},
    hline{3}={1}{solid},
}
    \lg{287.4} = 2.4585 \\
    \lg{\cos 42^\circ6'} = \overline{1}.8704 \tikz [overlay] {\draw (1em, -0.5em) node {$(+$};} \\
    \lg{a} = 2.3289
\end{tblr}

查反对数表得

\hspace*{2em} $a = 213.2$。


(3) \begin{tblr}[t]{columns={mode=math}, row{1}={abovesep=0.5em}}
    \because   & \sin B = \exdfrac{b}{c} \douhao \\
    \therefore & b = c \, \sin B \juhao \\
\end{tblr}

两边取对数得

\hspace*{2em} $\lg b = \lg c + \lg \sin B$。

\hspace*{2em} \begin{tblr}[t]{
    columns={r, mode=math, colsep=0em},
    rows={rowsep=0em},
    hline{3}={1}{solid},
}
    \lg{287.4} = 2.4585 \\
    \lg{\sin 42^\circ6'} = \overline{1}.8264 \tikz [overlay] {\draw (1em, -0.5em) node {$(+$};} \\
    \lg{b} = 2.2849
\end{tblr}

查反对数表得

\hspace*{2em} $b = 192.7$。



\begin{enhancedline}
\liti 已知直角三角形 $ABC$ 中,$a = 104.0$,$b = 20.49$,解这个三角形。

\jie (1) $\tan A = \exdfrac{a}{b}$。

两边取对数得

\hspace*{2em} $\lg{\tan A} = \lg{a} - \lg{b}$。

\hspace*{2em} \begin{tblr}[t]{
    columns={r, mode=math, colsep=0em},
    rows={rowsep=0em},
    hline{3}={1}{solid},
}
    \lg{104.0} = 2.0170 \\
    \lg{20.49} = 1.3115 \tikz [overlay] {\draw (1em, -0.5em) node {$(-$};} \\
    \lg{\tan A} = 0.7055
\end{tblr}

查表得

\hspace*{2em} $A = 78^\circ51'$。


(2) $B = 90^\circ - 78^\circ51' = 11^\circ9'$。

(3) 因为 $\sin A = \exdfrac{a}{c}$,所以 $c = \dfrac{a}{\sin A}$。

两边取对数得

\hspace*{2em} $\lg{c} = \lg{a} - \lg{\sin A}$。

\hspace*{2em} \begin{tblr}[t]{
    columns={r, mode=math, colsep=0em},
    rows={rowsep=0em},
    hline{3}={1}{solid},
}
    \lg{104.0} = 2.0170 \\
    \lg{\sin 78^\circ51'} = \overline{1}.9917 \tikz [overlay] {\draw (1em, -0.5em) node {$(-$};} \\
    \lg{c} = 2.0253
\end{tblr}

查反对数表得

\hspace*{2em} $c = 106.0$。
\end{enhancedline}

例 6 中的 $c$ 也可利用勾股定理,查平方表、平方根表来计算。


\lianxi
\begin{xiaotis}

\xiaoti{查表求下列各三角函数值的对数:}
\begin{xiaoxiaotis}

    \begin{tblr}{columns={12em, colsep=0pt}}
        \xxt{$\lg{\sin 21^\circ36'}$;} & \xxt{$\lg{\sin 5^\circ47'}$;}  & \xxt{$\lg{\cos 32^\circ8'}$;} \\
    \end{tblr}

    \begin{tblr}{columns={12em, colsep=0pt}}
        \xxt{$\lg{\cos 66^\circ10'}$;} & \xxt{$\lg{\tan 27^\circ41'}$;} & \xxt{$\lg{\tan 70^\circ43'}$;} \\
        \xxt{$\lg{\cot 69^\circ2'}$;}  & \xxt{$\lg{\cot 89^\circ24'}$。}
    \end{tblr}
\end{xiaoxiaotis}

\begin{minipage}[t]{10cm}
    \xiaoti{求下列各式中的锐角 $\alpha$:}
    \begin{xiaoxiaotis}

        \begin{tblr}{columns={colsep=0pt}}
            \xxt{$\lg{\sin \alpha} = \overline{1}.4001$;} & \xxt{$\lg{\cos \alpha} = \overline{2}.9301$;} \\
            \xxt{$\lg{\tan \alpha} = \overline{2}.0318$;} & \xxt{$\lg{\cot \alpha} = \overline{1}.5018$。}
        \end{tblr}
    \end{xiaoxiaotis}

    \xiaoti{根据下列条件解直角三角形:}
    \begin{xiaoxiaotis}

        \xxt{$a = 30.01 \nsep B = 80^\circ24'$;}

        \xxt{$c = 0.8328 \nsep b = 0.2954$。}

    \end{xiaoxiaotis}


    \xiaoti{如图,已知 $\alpha = 37^\circ$,$OP = 26$,求点 $P$ 的坐标(保留两个有效数字)。}
\end{minipage}
\quad
\begin{minipage}[t]{4cm}
    \begin{figure}[H]
        \centering
        \begin{tikzpicture}[>=Stealth]
    \draw [->] (-0.4, 0) -- (3, 0) node[below=0.2em] {$x$} coordinate(x axis);
    \draw [->] (0, -0.4) -- (0, 2) node[left=0.2em]  {$y$} coordinate(y axis);
    \draw (0, 0) coordinate (O) node [below left] {\small $O$};

    \pgfmathsetmacro{\jiao}{36}
    \coordinate (A) at (\jiao:3);
    \coordinate (P) at (\jiao:2.6);
    \pgfmathsetmacro{\r}{0.5}

    \draw (O) -- (A);
    \draw [->] (0:\r) arc (0:\jiao:\r) node [midway, right] {$\alpha$};
    \draw [fill=black] (P) circle (0.05) node [above] {$P$};
\end{tikzpicture}

        \caption*{(第 4 题)}
    \end{figure}
\end{minipage}

\end{xiaotis}

