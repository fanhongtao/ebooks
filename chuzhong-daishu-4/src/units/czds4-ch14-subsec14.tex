\subsection{一元二次不等式及其解法}\label{subsec:14-14}

含有一个未知数并且未知数的最高次数是二次的不等式叫做\zhongdian{一元二次不等式},它的一般形式是
$$ ax^2 + bx + c > 0 \text{,或\;} ax^2 + bx + c < 0 \; (a \neq 0) \juhao $$

\begin{wrapfigure}[12]{r}{5.5cm}
    \centering
    \begin{tikzpicture}[>=Stealth, scale=0.5,
    every node/.style={fill=white, inner sep=1pt},
]
    \draw [->] (-5, 0) -- (5, 0) node[below=0.2em] {$x$} coordinate(x axis);
    \draw [->] (0, -8) -- (0, 3) node[left=0.2em]  {$y$} coordinate(y axis);
    \draw (0, 0) node [below left=0.3em] {\small $O$};
    \foreach \x in {-4, ..., 4} {
        \draw (\x, 0) -- (\x, 0.2);
    }
    \draw (-2, 0) node [below=0.2em, xshift=-0.7em] {\small $-2$};
    \draw ( 3, 0) node [below=0.2em, xshift=0.5em] {\small $3$};

    \foreach \y in {-7, ..., 2} {
        \draw (0.2, \y) -- (0, \y);
    }
    \draw (0, -6) node [left=0.3em] {\small $-6$};


    \draw[domain=-2.3:3.3,  samples=50] plot (\x, {(\x)^2 - \x - 6}) node [above] {\small $y = x^2 - x - 6$};
    \foreach \x in {-2, 0, 3} {
        \draw [fill=black] (\x, {(\x)^2 - \x - 6}) circle(0.1);
    }

    \draw (-4, 0.6) node {\small $y > 0$};
    \draw (4.2, 0.6) node {\small $y > 0$};
    \draw (1.2, -2.6) node {\small $y < 0$};
\end{tikzpicture}


    \caption{}\label{fig:14-36}
\end{wrapfigure}

下面利用二次函数的图象来讨论一元二次不等式的解法。

例如,对于二次函数 $y = x^2 - x - 6$,我们来求

(1) $x$ 取哪些值时,$y = 0$;

(2) $x$ 取哪些值时,$y > 0$;

(3) $x$ 取哪些值时,$y < 0$;


画出抛物线 $y = x^2 - x - 6$ 的图象,如图 \ref{fig:14-36} 所示,它与 $x$ 轴相交于两点 $(-2,\; 0)$ 和 $(3,\; 0)$,
这两点将 $x$ 轴分成三段。从图 \ref{fig:14-36} 可以看出:

(1) 当 $x = -2$,或 $x = 3$ 时,$y = 0$;

(2) 当 $x < -2$,或 $x > 3$ 时,$y > 0$;

(3) 当 $-2 < x < 3$ 时,$y < 0$。

这就是说,抛物线 $y = x^2 - x - 6$ 与 $x$ 轴有两个交点,
即方程 $x^2 - x - 6 = 0$ 有两个不相等的实根(相异实根)$x_1 = -2$,$x_2 = 3$。
在这个情况下,不等式
$$ x^2 - x - 6 > 0 $$
的解集是
$$ x < -2 \text{,或\;} x > 3 \fenhao $$
而不等式
$$ x^2 - x - 6 < 0 $$
的解集则是
$$ -2 < x < 3 \juhao $$

一般地,对于二次函数 $y = ax^2 + bx + c \; (a > 0)$,设 $\Delta = b^2 - 4ac$。

1. 如果 $\Delta > 0$,此时抛物线 $y = ax^2 + bx + c$ 与 $x$ 轴有两个交点(图 \ref{fig:14-37}),
即方程 $ax^2 + bx + c = 0$ 有两个相异实根 $x_1$,$x_2 \; (x_1 < x_2)$。
那么,不等式 $ax^2 + bx + c > 0$ 的解集是
$$ x < x_1 \text{,或\;} x > x_2 \fenhao $$
而不等式 $ax^2 + bx + c < 0$ 的解集则是
$$ x_1 < x < x_2 \juhao $$

\begin{figure}[htbp]
    \centering
    \begin{minipage}[b]{7cm}
    \centering
    \begin{tikzpicture}[>=Stealth, scale=0.5,
    every node/.style={fill=white, inner sep=1pt},
]
    \draw [->] (-5, 0) -- (5, 0) node[below=0.2em] {$x$} coordinate(x axis);
    \draw [->] (0, -8) -- (0, 3) node[left=0.2em]  {$y$} coordinate(y axis);
    \draw (0, 0) node [below left=0.3em] {\small $O$};

    \draw (-2, 0) node [below=0.2em, xshift=-0.7em] {\small $x_1$};
    \draw ( 3, 0) node [below=0.2em, xshift=0.5em]  {\small $x_2$};

    \draw[domain=-2.3:3.3,  samples=50] plot (\x, {(\x)^2 - \x - 6});
    \foreach \x in {-2, 3} {
        \draw [fill=black] (\x, {(\x)^2 - \x - 6}) circle(0.1);
    }

    \draw (-4, 0.6) node {\small $y > 0$};
    \draw (4.2, 0.6) node {\small $y > 0$};
    \draw (1.2, -2.6) node {\small $y < 0$};
\end{tikzpicture}


    \caption{}\label{fig:14-37}
    \end{minipage}
    \qquad
    \begin{minipage}[b]{7cm}
    \centering
    \begin{tikzpicture}[>=Stealth, scale=0.5,
    every node/.style={fill=white, inner sep=1pt},
]
    \draw [->] (-3, 0) -- (5, 0) node[below=0.2em] {$x$} coordinate(x axis);
    \draw [->] (0, -2) -- (0, 6) node[left=0.2em]  {$y$} coordinate(y axis);
    \draw (0, 0) node [below left=0.3em] {\small $O$};

    \draw (1, 0) node [below=0.3em, xshift=0.7em] {\small $x_1 = x_2$};

    \draw[domain=-1.3:3.3,  samples=50] plot (\x, {(\x)^2 - 2*\x + 1});
    \foreach \x in {1} {
        \draw [fill=black] (\x, {(\x)^2 - 2*\x + 1}) circle(0.1);
    }

    \draw (-2, 1.6) node {\small $y > 0$};
    \draw (4.2, 1.6) node {\small $y > 0$};
\end{tikzpicture}


    \caption{}\label{fig:14-38}
    \end{minipage}
\end{figure}


\begin{enhancedline}
2. 如果 $\Delta = 0$,此时抛物线 $y = ax^2 + bx + c$ 与 $x$ 轴只有一个交点(图 \ref{fig:14-38}),
即方程 $ax^2 + bx + c = 0$ 有两个相等的实根 $x_1 = x_2 = -\dfrac{b}{2a}$。
那么,不等式 $ax^2 + bx + c > 0$ 的解集是所有不等于 $-\dfrac{b}{2a}$ 的实数,
而不等式 $ax^2 + bx + c < 0$ 的解集则是空集。

\begin{wrapfigure}[8]{r}{5cm}
    \centering
    \begin{tikzpicture}[>=Stealth, scale=0.5,
    every node/.style={fill=white, inner sep=1pt},
]
    \draw [->] (-3, 0) -- (5, 0) node[below=0.2em] {$x$} coordinate(x axis);
    \draw [->] (0, -2) -- (0, 7) node[left=0.2em]  {$y$} coordinate(y axis);
    \draw (0, 0) node [below left=0.3em] {\small $O$};

    \draw[domain=-1.3:3.3,  samples=50] plot (\x, {(\x)^2 - 2*\x + 2});

    \draw (1.5, 0.6) node {\small $y > 0$};
\end{tikzpicture}


    \caption{}\label{fig:14-39}
\end{wrapfigure}

3. 如果 $\Delta < 0$,此时抛物线  $y = ax^2 + bx + c$ 与 $x$ 轴没有交点(图 \ref{fig:14-39}),
即方程 $ax^2 + bx + c = 0$ 无实根。
那么,不等式 $ax^2 + bx + c > 0$ 的解集是全体实数,
而不等式 $ax^2 + bx + c < 0$ 的解集则是空集。



二次项系数是负数(即 $a < 0$)的不等式,可以先化成二次项系数是正数的不等式,再求它的解集。


\liti 解不等式 $(x + 4)(x - 1) < 0$。

\jie  如果将 $(x + 4)(x - 1)$ 展开,它是一个二次项系数为正数的二次函数。
因为方程 $(x + 4)(x - 1) = 0$ 的根是
$$ x_1 = -4\nsep x_2 = 1 \douhao $$
所以不等式的解集是
$$ -4 < x < 1 \juhao $$


\liti 解不等式 $2x^2 - 3x - 2 > 0$。

\jie 因为 $\Delta = b^2 - 4ac > 0$,方程 $2x^2 - 3x - 2 = 0$ 的根是
$$ x_1 = -\dfrac{1}{2}\nsep x_2 = 2 \douhao $$
所以不等式的解集是
$$ x < -\dfrac{1}{2} \text{,或\;} x > 2 \juhao $$


\liti 解不等式 $-3x^2 + 6x > 2$。

\jie 两边都乘以 $-1$,移项 得
$$ 3x^2 - 6x + 2 < 0 \juhao $$

因为 $\Delta > 0$, 方程 $3x^2 - 6x + 2 = 0$ 的解是
$$ x_1 = 1 - \dfrac{\sqrt{3}}{3}\nsep x_2 = 1 + \dfrac{\sqrt{3}}{3} \douhao $$
所以原不等式的解集是
$$ 1 - \dfrac{\sqrt{3}}{3} < x < 1 + \dfrac{\sqrt{3}}{3} \juhao $$


\liti 解不等式 $4x^2 - 4x + 1 > 0$。

\jie 因为 $\Delta = 0$, 方程 $4x^2 - 4x + 1 = 0$ 有两个相等的实根
$$ x_1 = x_2 = \dfrac{1}{2} \douhao $$

所以不等式的解集是所有不等于 $\dfrac{1}{2}$ 的实数。



\liti 解不等式 $-x^2 + 2x - 3 > 0$。

\jie 两边都乘以 $-1$,得
$$ x^2 - 2x + 3 < 0 \juhao $$

因为 $\Delta < 0$, 方程 $x^2 - 2x + 3 = 0$ 无实根,
即不等式 $x^2 - 2x + 3 < 0$ 的解集是空集。
所以,原不等式 $-x^2 + 2x - 3 > 0$ 的解集是空集。
\end{enhancedline}


\liti $m$ 是什么实数的时候,方程 $x^2 - (m + 2)x + 4 = 0$ 有实根?

\jie 这个方程的判别式是
$$ [-(m + 2)]^2 - 4 \times 1 \times 4 = m^2 + 4m - 12 \juhao $$

我们知道:判别式大于或等于零的时候,原方程有实根;判别式小于零的时候,原方程没有实根。

由 $m^2 + 4m - 12 = 0$, 得 $m = 2$ 或 $m = -6$;

由 $m^2 + 4m - 12 > 0$, 得 $m < -6$ 或 $m > 2$。

由此可知,当 $m \leqslant -6$ 或 $m \geqslant 2$ 时,原方程有实根。



\lianxi
\begin{xiaotis}

\xiaoti{解下列不等式:}
\begin{xiaoxiaotis}

    \begin{tblr}{columns={18em, colsep=0pt}, row{4}={rowsep=0.5em}}
        \xxt{$(x + 2)(x - 3) > 0$;}        & \xxt{$x(x - 2) < 0$;} \\
        \xxt{$3x^2 - 7x + 2 < 0$;}         & \xxt{$4x^2 + 4x + 1 < 0$;} \\
        \xxt{$-6x^2 - x + 2 \leqslant 0$;} & \xxt{$x(x - 1) < x(2x - 3) + 2$;} \\
        \xxt{$x^2 + 10 \geqslant 6x + 1$;} & \xxt{$x^2 - 4\dfrac{1}{3}x + 5\dfrac{1}{3} \leqslant 0$。}
    \end{tblr}
\end{xiaoxiaotis}


\xiaoti{$x$ 是什么实数时,函数 $y = x^2 - 4x + 1$ 的值:}
\begin{xiaoxiaotis}

    \begin{tblr}{columns={12em, colsep=0pt}}
        \xxt{等于零?} & \xxt{是正数?} & \xxt{是负数?}
    \end{tblr}
\end{xiaoxiaotis}


\xiaoti{$x$ 是什么实数时,$\sqrt{x^2 + x - 12}$ 有意义?}


\xiaoti{$k$ 是什么实数时,方程
    $$ x^2 + 2x - 11 = k(3 - x) $$
    有实数根?
}

\end{xiaotis}


