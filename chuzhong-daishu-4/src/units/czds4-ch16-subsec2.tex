\subsection{平均数}\label{subsec:16-2}

某班第一小组在一次数学测验中的成绩如下:
$$ 86 \quad 91  \quad 100  \quad  72  \quad  93 \quad 89 \quad 90 \quad 85 \quad 75 \quad 95 $$
这个小组的平均成绩是多少?

\begin{enhancedline}
很明显,这个平均成绩是
$$ \dfrac{86 + 91 + \cdots + 95}{10} = 87.6 \juhao $$

一般地,如果有 $n$ 个数
$$ x_1 \nsep x_2 \nsep \cdots \nsep x_n \douhao $$
那么
\begin{gather}
    \overline{x} = \exdfrac{1}{n} (x_1 + x_2 +  \cdots + x_n) \tag{1}
\end{gather}
叫做这 $n$ 个数的\zhongdian{平均数},$\overline{x}$ 读作“$x$拔”。

为了书写方便,有时将 $x_1 + x_2 +  \cdots + x_n$ 记作
$\displaystyle \sum_{i=1}^n x_i$,
其中 “$\Sigma$” 是求和符号,读作“西格马”。
$\displaystyle \sum_{i=1}^n x_i$ 读作
“$\Sigma$ —— $x$  —— $i$, $i$ 从 $1$ 到 $n$”,
表示从 $x_1$ 加到 $x_n$ 的和。于是公式 (1) 可以写成
\begin{gather}
    \overline{x} = \exdfrac{1}{n} \sum_{i=1}^n x_i \juhao \tag{1}
\end{gather}
\end{enhancedline}

\lianxi
\begin{xiaotis}

\xiaoti{计算下列各组数的平均数:}
\begin{xiaoxiaotis}

    \xxt{$1.4 \quad 2.8 \quad 5.3 \quad 3.7 \quad 2.3$}

    \xxt{$-0.3 \quad 0.2 \quad 0.3 \quad -0.4 \quad 0.6 \quad 0.3 \quad 0.5 \quad -0.4$}

\end{xiaoxiaotis}


\xiaoti{用求和符号 “$\Sigma$” 表示下列各式:}
\begin{xiaoxiaotis}

    \begin{tblr}{columns={18em, colsep=0pt}}
        \xxt{$P_1 + P_2 +  \cdots + P_{10}$;} & \xxt{$y_1 + y_2 + \cdots + y_n$。}
    \end{tblr}
\end{xiaoxiaotis}

\end{xiaotis}
\lianxijiange


在一次毕业考试中,考生有一万多名。我们想了解这一万多名考生的数学平均成绩。
但如果将他们的成绩全部加在一起再除以考生总数,十分麻烦。
这时,可以采取用样本估计总体的方法,即从中抽查部分考生的成绩,
用他们的平均成绩去估计所有考生的平均成绩。

总体中所有个体的平均数叫做\zhongdian{总体平均数},
样本中所有个体的平均数叫做\zhongdian{样本平均数}。
对于由一万多名考生的成绩组成的总体来说,所有考生的平均成绩就是总体平均数,
所抽查的部分考生的平均成绩就是样本平均数。
通常我们用样本平均数去估计总体平均数。
一般来说,样本容量越大,这种估计也就越精确。
例如,抽查的考生越多,所抽查的考生的平均成绩就越接近总平均成绩。


\liti 从参加毕业考试的学生中,抽查了 $30$ 名学生的数学成绩,分数如下:
\begin{data}
    \begin{datatblr}{}
        90 & 84 & 84 & 86 & 87 & 98 & 78 & 82 & 90 & 93 \\
        68 & 95 & 84 & 71 & 78 & 61 & 94 & 88 & 77 & 100 \\
        70 & 97 & 85 & 68 & 99 & 88 & 85 & 92 & 93 & 97
    \end{datatblr}
\end{data}
计算样本平均数(结果保留到个位)。

\begin{enhancedline}
\jie $\begin{aligned}[t]
    \overline{x} &= \dfrac{1}{30} (90 + 84 + \cdots + 97) \\
                 &= \dfrac{2562}{30} \\
                 &\approx 85 \juhao
\end{aligned}$\\
即样本平均数为 $85$.
\end{enhancedline}

于是可以估计,参加毕业考试的学生的数学平均成绩约为 $85$ 分。


\liti 从一批机器零件毛坯中取出了 $20$ 件,称得它们的重量如下(单位:千克):
\begin{data}
    \begin{datatblr}{}
        210 & 208 & 200 & 205 & 202 & 218 & 206 & 214 & 215 & 207 \\
        195 & 207 & 218 & 192 & 202 & 216 & 185 & 227 & 187 & 215
    \end{datatblr}
\end{data}
计算样本平均数(结果保留到个位)。

\begin{enhancedline}
\jie $\begin{aligned}[t]
    \overline{x} &= \dfrac{1}{20} (210 + 208 + \cdots + 215) \\
                 &= \dfrac{4129}{20} \\
                 &\approx 206 \; (\qianke) \juhao
\end{aligned}$\\
即样本平均数为 $206$ 千克。
\end{enhancedline}

于是可以估计,这批机器零件毛坯平均每件约重 $206$ 千克。

本题中的样本数据较大,而且都在 $200$ 左右波动。
这时,也可以采用下面的算法:

将上面各数据同时减去 $200$,得到一组新数据:
\begin{data}
    \begin{datatblr}{colsep=0.8em}
        10 & 8 &  0 &  5 & 2 & 18 &   6 & 14 &  15 & 7 \\
        -5 & 7 & 18 & -8 & 2 & 16 & -15 & 27 & -13 & 15
    \end{datatblr}
\end{data}
计算这组新数据的平均数,得
\begin{enhancedline} \begin{align*}
    \overline{x'} &= \dfrac{1}{20} (10 + 8 + \cdots + 15) \\
                  &= \dfrac{129}{20} \\
                  &\approx 6 \douhao
\end{align*} \end{enhancedline}
于是,所求的平均数应该是
\begin{align*}
    \overline{x} &= \overline{x'} + 200 \\
                 &\approx 6 + 200 \\
                 &= 206 \; (\qianke) \juhao
\end{align*}
所得结果与前面的结果一样。我们看到,采用这种算法,
可以使参与计算的数据变小,计算起来比较简便。

一般地,如果将一组数据 $x_1$,$x_2$,$\cdots$,$x_n$
同时减去一个适当的常数 $a$\footnotemark 得到
\footnotetext{常数 $a$ 通常取接近于样本平均数(约略估计)的较“整”的数。}
$$ x_1' = x_1 - a \nsep x_2' = x_2 - a \nsep \cdots \nsep x_n' = x_n - a \douhao $$
那么
$$ x_1 = x_1' + a \nsep x_2 = x_2' + a \nsep \cdots \nsep x_n = x_n' + a \juhao $$
因此
\begin{enhancedline} \begin{align*}
    \overline{n} &= \exdfrac{1}{n}(x_1 + x_2 + \cdots + x_n) \\
                 &= \exdfrac{1}{n}[(x_1' + a) + (x_2' + a) + \cdots + (x_n' + a)] \\
                 &= \exdfrac{1}{n} [(x_1' + x_2' + \cdots + x_n') + na] \\
                 &= \exdfrac{1}{n} (x_1' + x_2' + \cdots + x_n') + \exdfrac{1}{n} \cdot na \\
                 &= \overline{x'} + a \juhao
\end{align*} \end{enhancedline}
即
\begin{gather*} % 因为在方框外有公式编号,所以这里采用手工绘制方框的方式。(虽然 \boxed 可以用,但画的框太小,不美观)
    \tikz[overlay, >=Stealth] {
        \draw (-1.5em, -0.8em) rectangle (6em, 1.5em);
    }\overline{x} = \overline{x'} + a \juhao \tag{2}
\end{gather*}\vspace*{0.5em}

在例 2 的第二种算法中,正是利用了公式 (2)。


\liti 某工人在 $30$ 天中加工一种零件的日产量,有 $2$ 天是 $51$ 件,$3$ 天是 $52$ 件,
$6$ 天是 $53$ 件,$8$ 天是 $54$ 件,$7$ 天是 $55$ 件,$3$ 天是 $56$ 件,$1$ 天是 $57$ 件。
计算这个工人 $30$ 天中的平均日产量(结果保留到个位)。

\jie 在上面 $30$ 个数据中,$51$ 出现 $2$ 次,$52$ 出现 $3$ 次,$53$ 出现 $6$ 次,
$54$ 出现 $8$ 次,$55$ 出现 $7$ 次,$56$ 出现 $3$ 次,$57$ 出现 $1$ 次。
由于这组数据都比 $50$ 稍大一点,我们利用公式 (2) 计算它们的平均数,
并将公式中的常数 $a$ 取作 $50$。

将数据 $51$,$52$,$53$,$54$,$55$,$56$, $57$ 同时去 $50$,得到
$$ 1 \quad 2 \quad 3 \quad 4 \quad 5 \quad 6 \quad 7 $$
它们出现的次数依次是
$$ 2 \quad 3 \quad 6 \quad 8 \quad 7 \quad 3 \quad 1 $$
那么,这组新数据的平均数是
\begin{enhancedline} \begin{align*}
    \overline{x'} &= \dfrac{1 \times 2 + 2 \times 3 + \cdots + 7 \times 1}{30} \\
                  &= \dfrac{118}{30} \\
                  &\approx 4 \juhao
\end{align*} \end{enhancedline}

根据公式 (2),
\begin{align*}
    \overline{x} &= \overline{x'} + a \\
                 &\approx 4 + 50 \\
                 &= 54 \; (\text{件}) \juhao
\end{align*}
即这个工人 $30$ 天中的平均日产量为 $54$ 件。

于是可以估计,这个工人在这 $30$ 天前后的一段时间内,平均日产量约为 $54$ 件。

\begin{enhancedline}
一般来说,如果在 $n$ 个数中,$x_1$ 出现 $f_1$ 次,$x_2$ 出现 $f_2$ 次,$\cdots$,
$x_k$ 出现 $f_k$ 次(这里 $f_1 + f_2 + \cdots + f_k = n$),
那么根据公式 (1) ,这 $n$ 个数的平均数可以表示为
\begin{gather*}
    \overline{x} = \dfrac{x_1f_2 + x_2f_2 + \cdots + x_kf_k}{n} \douhao \footnotemark \tag{$1'$}
\end{gather*}
\footnotetext{这个平均数叫做\zhongdian{加权平均数},其中 $f_1$, $f_2$,$\cdots$,$f_k$ 叫做 \zhongdian{权}。}
或简记作
\begin{gather*}
    \overline{x} = \exdfrac{1}{n} \sum_{i=1}^k x_if_i \juhao \tag{$1'$}
\end{gather*}
\end{enhancedline}


正如例 3 那样,当样本中有不少数据多次重复出现时,利用公式 ($1'$) 计算样本平均数比较简便。


\lianxi
\begin{xiaotis}

\xiaoti{抽测了 $10$ 株棉花的高, 结果如下(单位: 厘米):\\
    \hspace*{8em}
    \begin{datatblr}{}
        25 & 41 & 40 & 43 & 22 & 14 & 19 & 39 & 21 & 42
    \end{datatblr} \\
    计算样本平均数(结果保留到个位)。
}

\xiaoti{利用公式 (2) 计算下面各组数据的平均数(结果保留到个位):}
\begin{xiaoxiaotis}

    \xxt{$105 \quad 103 \quad 101 \quad 100 \quad 114 \quad 108 \quad 110 \quad 106 \quad 98 \quad 102$}

    \xxt{$4203 \quad 4204 \quad 4200 \quad 4194 \quad 4204 \quad 4201 \quad 4195 \quad 4199$}

\end{xiaoxiaotis}


\xiaoti{从一所中学某年级的六个班中抽测了 $24$ 名男学生的身高,结果如下〈单位:厘米):\\
    \hspace*{8em}
    \begin{datatblr}{}
        155 & 157 & 159 & 162 & 162 & 163 & 164 & 164 \\
        165 & 165 & 165 & 166 & 166 & 167 & 167 & 168 \\
        168 & 169 & 169 & 170 & 171 & 171 & 172 & 174
    \end{datatblr} \\
    这个年男学生的平均身高约是多少(结果保留到个位)?
}


\xiaoti{在一个班的 $40$ 名学生中,$14$ 岁的 $5$ 人,$15$ 岁的有 $30$ 人,$16$ 岁的有 $4$ 人,
    $17$ 岁的 $1$ 人。计算这个班学生的平均年龄(结果保留到小数点后第一位)。
}

\end{xiaotis}


