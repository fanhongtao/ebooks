\subsubsection{正比例函数的图象}

我们来画函数 $y = 2x$ 的图象。

在 $x$ 的取值范围内取一些值,算出 $y$ 的对应值,列表如下:

\begin{table}[H]
    \hspace*{2em}
    \begin{tblr}{
        hlines, vlines,
        columns={mode=math, 2em, c},
        column{1}={4em},
    }
        x & \cdots & -2 & -1 & 0 & 1 & 2 & \cdots \\
        y & \cdots & -4 & -2 & 0 & 2 & 4 & \cdots
    \end{tblr}
\end{table}

用表里各组对应值作为点的坐标,描出各个点,并且把它们依次连结起来。
可以看到,函数 $y = 2x$ 的图象是经过 $O(0,\, 0)$,$A(1,\, 2)$ 这两点的一条直线(图 \ref{fig:14-13})。

\begin{figure}[htbp]
    \centering
    \begin{minipage}[b]{7cm}
    \centering
    \begin{tikzpicture}[>=Stealth, scale=0.6]
    \draw [->] (-3.5, 0) -- (3.5, 0) node[anchor=north] {$x$};
    \draw [->] (0, -5.0) -- (0, 5) node[anchor=east] {$y$};
    \draw [xshift=-.5em] (0, 0) node [below left] {\small $O$};
    \foreach \x in {-2, -1, 1, 2} {
        \draw (\x, 0) -- (\x, 0.2) node [above]{\small $\x$};
    }
    \foreach \y in {1, ..., 4} {
        \draw (0.2, \y) -- (0, \y) node [left] {\small $\y$};
    }
    \foreach \y in {-4, ..., -1} {
        \draw (0, \y) -- (0.2, \y) node [right] {\small $\y$};
    }

    \draw[domain=-2.3:2.3,samples=50] plot (\x, {2*\x}) node [above] {$y = 2x$};
    \foreach \x in {-2, ..., 2} {
        \draw [fill=black] (\x, 2*\x) circle(0.1);
    }
\end{tikzpicture}


    \caption{}\label{fig:14-13}
    \end{minipage}
    \qquad
    \begin{minipage}[b]{7cm}
    \centering
    \begin{tikzpicture}[>=Stealth, scale=0.6]
    \draw [->] (-3.5, 0) -- (3.5, 0) node[anchor=north] {$x$};
    \draw [->] (0, -5.0) -- (0, 5) node[anchor=west] {$y$};
    \draw (0, 0) node [below left] {\small $O$};
    \foreach \x in {-2, -1, 1, 2} {
        \draw (\x, 0) -- (\x, 0.2) node [above]{\small $\x$};
    }
    \foreach \y in {1, ..., 3} {
        \draw (0, \y) -- (0.2, \y) node [right] {\small $\y$};
    }
    \foreach \y in {-3, ..., -1} {
        \draw (0, \y) -- (-0.2, \y) node [left] {\small $\y$};
    }

    \draw[domain=1.3:-1.3,samples=50] plot (\x, {-3*\x}) node [above] {$y = -3x$};
    \foreach \x in {-1, ..., 1} {
        \draw [fill=black] (\x, -3*\x) circle(0.1);
    }
    \draw [dashed] (0, -3) -- (1, -3) -- (1, 0);
\end{tikzpicture}


    \caption{}\label{fig:14-14}
    \end{minipage}
\end{figure}

同样可以知道, $y = -3x$ 的图象是经过 $O(0,\, 0)$,$B(1,\, -3)$ 这两点的一条直线(图 \ref{fig:14-14})。

一般地,正比例函数 $y = kx$ 的图象是经过 $O(0,\, 0)$,$A(1,\, k)$ 这两点的一条直线。
我们以后把正比例函数 $y = kx$ 的图象叫做直线 $y = kx$。

由于直线的位置可以由直线上的任意两点唯一确定,我们在以后画正比例函数 $y = kx$ 的图象时,
可以不用描点法,只要选取两点连成直线就行了。
我们通常取 $O(0,\, 0)$,$A(1,\, k)$ 这两点。

\begin{enhancedline}
\liti 在同一平面直角坐标系内,分别画出下列函数的图象:
$$ y = 2x \nsep y = x \nsep y = \dfrac{1}{2}x \juhao $$

\jie 因为 $O(0,\, 0)$,$A(1,\, 2)$ 是直线 $y = 2x$ 上的点,
所以过这两点画一条直线,即得函数 $y = 2x$ 的图象。
同理, 过 $O(0,\, 0)$,$B(1,\, 1)$ 的直线是函数 $y = x$ 的图象;
过 $O(0,\, 0)$,$C\left(1,\, \dfrac{1}{2}\right)$ 的直线是函数 $y = \dfrac{1}{2}x$ 的图象(图 \ref{fig:14-15})。

\begin{figure}[htbp]
    \centering
    \begin{minipage}[b]{7cm}
    \centering
    \begin{tikzpicture}[>=Stealth, scale=0.5,
    every node/.style={fill=white, inner sep=1pt},
]
    \draw [->] (-6, 0) -- (6, 0) node[below=0.2em] {$x$};
    \draw [->] (0, -6) -- (0, 6) node[left=0.2em] {$y$};
    \draw (0, 0) node [below right=0.3em] {\small $O$};
    \foreach \x in {-5, ..., 5} {
        \draw (\x, 0) -- (\x, 0.2);
    }
    \foreach \y in {-5, ..., 5} {
        \draw (0, \y) -- (0.2, \y);
    }

    \draw[domain=-2.6:2.6, samples=10] plot (\x, {2*\x}) node [above] {$y = 2x$};
    \draw[domain=-4.0:4.0, samples=10] plot (\x, {\x})   node [above] {$y = x$};
    \draw[domain=-5.0:5.0, samples=10] plot (\x, {\x/2}) node [above] {$y = \frac{1}{2}x$};
    \draw [fill=black] (0, 0) circle(0.1);
    \draw [fill=black] (1, 2)   circle(0.1) node [left=3pt] {\small $A$};
    \draw [fill=black] (1, 1)   circle(0.1) node [above right=3pt] {\small $B$};
    \draw [fill=black] (1, 0.5) circle(0.1) node [right=3pt] {\small $C$};
\end{tikzpicture}


    \caption{}\label{fig:14-15}
    \end{minipage}
    \qquad
    \begin{minipage}[b]{7cm}
    \centering
    \begin{tikzpicture}[>=Stealth, scale=0.5,
    every node/.style={fill=white, inner sep=1pt},
]
    \draw [->] (-6, 0) -- (6, 0) node[below=0.2em] {$x$};
    \draw [->] (0, -6) -- (0, 6) node[right=0.2em] {$y$};
    \draw (0, 0) node [below left=0.3em] {\small $O$};
    \foreach \x in {-5, ..., 5} {
        \draw (\x, 0) -- (\x, 0.2);
    }
    \foreach \y in {-5, ..., 5} {
        \draw (0, \y) -- (0.2, \y);
    }

    \draw[domain=1.7:-1.7, samples=10] plot (\x, {-3*\x}) node [above] {$y = -3x$};
    \draw[domain=4.0:-4.0, samples=10] plot (\x, {-\x})   node [above] {$y = x$};
    \draw[domain=5.2:-5.2, samples=10] plot (\x, {-\x/4}) node [above] {$y = -\frac{1}{4}x$};
    \draw [fill=black] (0, 0) circle(0.1);
    \draw [fill=black] (1, -3)  circle(0.1) node [right=3pt] {\small $A$};
    \draw [fill=black] (2, -2)  circle(0.1) node [right=3pt] {\small $B$};
    \draw [fill=black] (4, -1)  circle(0.1) node [above=3pt] {\small $C$};
\end{tikzpicture}


    \caption{}\label{fig:14-16}
    \end{minipage}
\end{figure}

\liti 在同一坐标系内,分别画出下列函数的图象。
$$ y = -3x \nsep y = -x \nsep y = -\dfrac{1}{4}x \juhao $$

\jie 过 $O(0,\, 0)$,$A(1,\, -3)$ 两点画一条直线,即得函数 $y = -3x$ 的图象。
过 $O(0,\, 0)$,$B(2,\, -2)$ 两点画一条直线,即得函数 $y = -x$ 的图象。同理,
过 $O(0,\, 0)$,$C(4,\, -1)$ 的直线是函数 $y = -\dfrac{1}{4}x$ 的图象(图 \ref{fig:14-16})。

由图 \ref{fig:14-15} 和图 \ref{fig:14-16}, 我们可以看出,\zhongdian{正比例函数 $\bm{y = kx}$ 有下列性质:}

\zhongdian{
    当 $\bm{k > 0}$ 时,它的图象在第一、三象限内,$\bm{y}$ 随 $\bm{x}$ 的增大而增大;
    当 $\bm{k < 0}$ 时,它的图象在第二、四象限内,$\bm{y}$ 随 $\bm{x}$ 的增大而减小。
}


\lianxi
\begin{xiaotis}

\xiaoti{〔口答)下列函数(其中 $x$ 是自变量)中,哪些是正比例函数,哪些不是,为什么?}
\begin{xiaoxiaotis}

    \begin{tblr}{columns={18em, colsep=0pt}}
        \xxt{$y = -8x$;}  & \xxt{$y = \dfrac{-8}{x}$;} \\
        \xxt{$y = 8x^2$;} & \xxt{$y = 8x + 1$。}
    \end{tblr}
\end{xiaoxiaotis}


\xiaoti{圆的面积 $A$ 是不是半径 $r$ 的正比例函数?}

\xiaoti{已知变量 $y$ 与 $x$ 成正比例,并且 $x = 2$ 时,$y = 15$, 求 $y$ 与 $x$
    之间的比例系数,并写出 $y$ 与 $x$ 之间的函数关系式。
}

\xiaoti{在同一坐标系内,画出下列函数的图象:
    $$y = \dfrac{2}{3}x \nsep y = -\dfrac{2}{3}x \nsep y = \dfrac{3}{2}x \nsep y = -\dfrac{3}{2}x \juhao $$
}

\end{xiaotis}
\end{enhancedline}
