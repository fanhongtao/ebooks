\subsubsection{正比例函数}

看下面的例子:

(1) 在一块地里施用农药,每公顷用药 $22.5$ 千克。那么所需这种农药的总量 $y$(千克)
与这块地的面积 $x$(公顷)之间的函数关系式是
$$ y = 22.5x \juhao $$

\begin{enhancedline}
(2) 铜的密度是 $8.9 \; \kmlflm$,铜的质量 $W$(克)与体积 $V$($\lflm$)之间的函数关系式是
$$ W = 8.9V \juhao $$

例 (1) 中的变量 $y$ 与变量 $x$ 的相应值的比值 $\dfrac{y}{x}$ 是一个常数(等于 $22.5$)。
同样,例 (2) 中的 $W$ 与 $V$ 之间也有这种性质。
在算术中,我们把具有这种性质的两个量叫做成正比例。
在这里,我们把 $y = 22.5x$,$W = 8.9V$ 这样的函数都叫做正比例函数。
\end{enhancedline}

一般地,函数 $y = kx$ ($k$ 是一个不等于零的常数)叫做\zhongdian{正比例函数}
(这时我们说 $y$ 与 $x$ \zhongdian{成正比例}),
常数 $k$ 叫做变量 $y$ 与 $x$ 之间的\zhongdian{比例系数}。
在算术中,$k$ 只能取正数,现在我们把它推广到也可以取负数。
确定了比例系数 $k$,就可以确定一个正比例函数。


\liti 圆的周长 $C$ 与圆的半径 $r$ 成正比例。已知 $r = 2$(单位: 厘米。下同)时,$C = 12.56$。
\begin{xiaoxiaotis}

     \xxt{求周长 $C$ 与半径 $r$ 之间的函数关系式;}

    \xxt{求半径为 $3.5$ 的圆的周长。}

\resetxxt
\jie \xxt{因为 $C$ 与 $r$ 成正比例,所以}
$$ C = kr \juhao $$
把 $r = 2$, $C = 12.56$ 代入,得
$$ 12.56 = 2k \nsep k = 6.28 \douhao $$

\fengeSuoyi{C = 6.28r \juhao}


\xxt{$r = 3.5$ 时,
    $$ C = 6.28 \times 3.5 = 21.98 \juhao $$
}

\end{xiaoxiaotis}

