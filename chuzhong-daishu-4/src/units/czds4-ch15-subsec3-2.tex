\subsubsection{正切和余切表}

正切表的用法与正弦表类似,余切表的用法与余弦表类似。

\liti 查表求下列三角函数值:
\begin{xiaoxiaotis}

    \hspace*{1.5em} \begin{tblr}{columns={18em, colsep=0pt}}
        \xxt{$\tan 53^\circ49'$;} &  \xxt{$\cot 14^\circ32'$。}
    \end{tblr}

\resetxxt
\jie \begin{tblr}[t]{columns={18em, colsep=0pt}}
    \xxt{\begin{TrigonometricTblr}
                \tan 53^\circ48' &   & = &       1.3663 &   \\
                        (+ \, 1' & ) &   & (+ \, 0.0008 & ) \\
                \tan 53^\circ49' &   & = &       1.3671 & \fenhao
            \end{TrigonometricTblr}
        } & \xxt{\begin{TrigonometricTblr}
                \cot 14^\circ30' &   & = &       3.867 &   \\
                        (+ \, 2' & ) &   & (- \, 0.009 & ) \\
                \cot 14^\circ32' &   & = &       3.858 & \juhao
            \end{TrigonometricTblr}
        }
\end{tblr}

\end{xiaoxiaotis}


\liti 已知下列三角函数值,求锐角 $\alpha$。
\begin{xiaoxiaotis}

    \hspace*{1.5em} \begin{tblr}{columns={18em, colsep=0pt}}
        \xxt{$\tan \alpha = 1.4036$;} &  \xxt{$\cot \alpha = 0.8637$。}
    \end{tblr}

\resetxxt
\jie \begin{tblr}[t]{columns={18em, colsep=0pt}}
    \xxt{\begin{TrigonometricTblr}
                      1.4019 &   & = & \tan 54^\circ30' &   \\
                (+ \, 0.0017 & ) &   &         (+ \, 2' & ) \\
                      1.4036 &   & = & \tan 54^\circ32' & \douhao
            \end{TrigonometricTblr} \\
            $\therefore$ \quad 锐角 $\alpha = 54^\circ32'$;
        } & \xxt{\begin{TrigonometricTblr}
                      0.8632 &   & = & \cot 49^\circ12' &   \\
                (+ \, 0.0005 & ) &   &         (- \, 1' & ) \\
                      0.8637 &   & = & \cot 49^\circ11' & \douhao
            \end{TrigonometricTblr} \\
            $\therefore$ \quad 锐角 $\alpha = 49^\circ11'$。
        }
\end{tblr}

\end{xiaoxiaotis}


\lianxi
\begin{xiaotis}

\xiaoti{查表求下列三角函数值:}
\begin{xiaoxiaotis}

    \xxt{$\sin 10^\circ \nsep \sin 14^\circ36' \nsep \sin 62^\circ30' \nsep \sin 2^\circ8' \nsep \sin 57^\circ33'$;}

    \xxt{$\cos 70^\circ \nsep \cos 50^\circ18' \nsep \cos 52^\circ20' \nsep \cos 75^\circ46' \nsep \cos 80^\circ27'$。}

\end{xiaoxiaotis}


\xiaoti{已知下列三角函数值,求锐角 $\alpha$ 或 $\beta$:}
\begin{xiaoxiaotis}

    \xxt{\begin{tblr}[t]{columns={mode=math}, column{1}={leftsep=0pt}}
            \sin\alpha = 0.7083 \douhao & \sin\beta = 0.9371 \douhao \\
            \sin\alpha = 0.3526 \douhao & \sin\beta = 0.5688 \fenhao \\
        \end{tblr}
    }

    \xxt{\begin{tblr}[t]{columns={mode=math}, column{1}={leftsep=0pt}}
            \cos\alpha = 0.8290 \douhao & \cos\alpha = 0.7611 \douhao \\
            \cos\beta  = 0.2996 \douhao & \cos\beta  = 0.9931 \juhao  \\
        \end{tblr}
    }

\end{xiaoxiaotis}


\xiaoti{查表求下列三角函数值:}
\begin{xiaoxiaotis}

    \xxt{$\tan 13^\circ12' \nsep \tan 40^\circ55' \nsep \tan 54^\circ28' \nsep \tan 74^\circ3' \nsep \tan 89^\circ43'$;}

    \xxt{$\cot 72^\circ18' \nsep \cot 56^\circ56' \nsep \cot 32^\circ23' \nsep \cot 15' \nsep \cot 15^\circ15'$。}

\end{xiaoxiaotis}


\xiaoti{已知下列三角函数值,求锐角 $\alpha$ 或 $\beta$:}
\begin{xiaoxiaotis}

    \xxt{\begin{tblr}[t]{columns={mode=math}, column{1}={leftsep=0pt}}
            \tan\beta  = 0.9131 \douhao & \tan\alpha = 0.3314 \douhao \\
            \tan\alpha = 2.220  \douhao & \tan\beta  = 31.80  \fenhao \\
        \end{tblr}
    }

    \xxt{\begin{tblr}[t]{columns={mode=math}, column{1}={leftsep=0pt}}
            \cot\alpha = 1.6003 \douhao & \cot\beta  = 3.590 \douhao \\
            \cot\beta  = 0.0781 \douhao & \cot\alpha = 180.9 \juhao  \\
        \end{tblr}
    }

\end{xiaoxiaotis}

\end{xiaotis}

