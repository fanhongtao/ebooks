% 原书的目录结构就是如此(缺少 section)
% 忽略这里的报错: Difference (2) between bookmark levels is greater (hyperref)	than one, level fixed.
\subsection{对数}\label{subsec:13-1}

我们已经学过了指数, 现在%“现在”二字,在原书中不可见,是根据上下文补填的内容。
来学习与指数有紧密联系的对数。

我们知道,$2$ 的 $4$ 次幂等于 $16$,可以记作
$$ 2^4 = 16 \douhao $$
这里 $16$ 是 $2$ 的 $4$ 次幂,$2$ 是底数,$4$ 是指数。

在计算中,我们还会遇到相反的问题:$2$ 的多少次幂等于 $16$?
为了表示 $16$ 是 $2$ 的多少次幂,我们采用一个新的式子:
$$ \log_{2}{16} = 4 \douhao $$
这里 $4$ 叫做以 $2$ 为底的 $16$ 的对数,$2$ 仍叫做底数,$16$ 叫做真数。

一般地说,如果 $a \; (a > 0, \; a \neq 1)$ 的 $b$ 次幂等于 $N$,就是 $a^b = N$,
数 $b$ 就叫做以 $a$ 为底的 $N$ 的\zhongdian{对数},记作 $\log_{a}{N} = b$\;\footnote{“log”是拉丁文的 logarithm(对数)的缩写。},
其中 $a$ 叫做\zhongdian{底数}(简称\zhongdian{底}),$N$ 叫做\zhongdian{真数}。

在实数范围内,正数的任何次幂都是正数。
在 $a^b = N$ 中,因为 $a$ 是不等于 $1$ 的正数,所以对于任意一个实数 $b$,
$N$ 总是正数,也就是说零与负数没有对数。

本章对数式中的字母,如不加特殊说明,底数都是不等于 $1$ 的正数,真数都是正数。

指数式 $a^b = N$ 中的底数、指数、幂与对数式 $\log_{a}{N} = b$ 中的底数、对数、真数的关系可表示如下:

\begin{figure}[htbp]
    \centering
    \begin{tikzpicture}[every node/.style={fill=white, inner sep=1pt}]
    \draw [line width=1.5pt] (0, 0) node [draw, inner sep=1.5em] {$a^b = N$};
    \draw [line width=1.5pt] (5, 0) node [draw, inner sep=1.5em] {$\log_{a}{N} = b$};

    \draw (-0.5, -0.3) -- (-0.5, -1.5) -- (4.7, -1.5) -- (4.7, -0.3);
    \draw (2.3, -1.5) node {底数};

    \draw (-0.3, 0.3) -- (-0.3, 1.6) -- (5.8, 1.6) -- (5.8, 0.3);
    \draw (2.3, 1.6) node {指数 \quad 对数};

    \draw (0.4, 0.3) -- (0.4, 1.1) -- (5.1, 1.1) -- (5.1, 0.3);
    \draw (2.45, 1.1) node {幂 \quad 真数};

    \draw [dashed] (2.3, 2.0) -- (2.3, 0.5);
\end{tikzpicture}


\end{figure}


如果把 $a^b = N$ 中的 $b$ 写成  $\log_{a}{N}$,就有
$$ a^{\log_{a}{N}} = N \juhao $$

例如在 $2^4 = 16$ 中,如果把 $4$ 写成 $\log_{2}{16}$,就有
$$ 2^{\log_{2}{16}} = 16 \juhao $$

\begin{enhancedline}
\liti 把下列指数式写成对数式:
\begin{xiaoxiaotis}

    \hspace*{1.5em} \begin{tblr}{columns={12em, colsep=0pt}}
        \xxt{$5^4 = 625$;} & \xxt{$3^{-2} = \dfrac{1}{9}$。}
    \end{tblr}

\resetxxt
\jie \begin{tblr}{columns={12em, colsep=0pt}}
    \xxt{$\log_{5}{625} = 4$;} & \xxt{$\log_{3}{\dfrac{1}{9}} = -2$。}
\end{tblr}
\end{xiaoxiaotis}


\liti 把下列对数式写成指数式:
\begin{xiaoxiaotis}

    \hspace*{1.5em} \begin{tblr}{columns={12em, colsep=0pt}}
        \xxt{$\log_{2}{8} = 3$;} & \xxt{$\log_{10}{10000} = 4$。}
    \end{tblr}

\resetxxt
\jie \begin{tblr}{columns={12em, colsep=0pt}}
    \xxt{$2^3 = 8$;} & \xxt{$10^4 = 10000$。}
\end{tblr}
\end{xiaoxiaotis}


\liti 已知 $\log_{10}{N} = -2$,求 $N$。

\jie 由 $\log_{10}{N} = -2$,得 $10^{-2} = N$。所以
$$ N = \dfrac{1}{100} \juhao $$


\liti 求下列各式的值:
\begin{xiaoxiaotis}

    \hspace*{1.5em} \begin{tblr}{columns={12em, colsep=0pt}}
        \xxt{$\log_{9}{81}$;} & \xxt{$\log_{3}{\dfrac{1}{27}}$。}
    \end{tblr}

\resetxxt
\jie \xxt{设 $\log_{9}{81} = x$,则 $9^x = 81$。 \\
    $\because \quad 9^2 = 81$, \\
    $\therefore \quad x = 2$。 \\
    即 \quad $\log_{9}{81} = 2$。
}

\hspace*{1.5em} \xxt{设 $\log_{3}{\dfrac{1}{27}} = x$,则 $3^x = \dfrac{1}{27}$。 \\
    $\because \quad 3^{-3} = \dfrac{1}{27}$, \\[.5em]
    $\therefore \quad x = -3$。 \\[.5em]
    即 \quad $\log_{3}{\dfrac{1}{27}} = -3$。
}

\end{xiaoxiaotis}
\end{enhancedline} \jiange

\liti 求下列各式的值:

\begin{xiaoxiaotis}

    \hspace*{1.5em} \begin{tblr}{columns={12em, colsep=0pt}}
        \xxt{$\log_{4}{4}$;} & \xxt{$\log_{7}{1}$。}
    \end{tblr}

\resetxxt
\jie \xxt{设 $\log_{4}{4} = x$,则 $4^x = 4$。 \\
    $\because \quad 4^1 = 4$, \\
    $\therefore \quad x = 1$。 \\
    即 \quad $\log_{4}{4} = 1$。
}

\hspace*{1.5em} \xxt{设 $\log_{7}{1} = x$,则 $7^x = 1$。 \\
    $\because \quad 7^0 = 1$, \\
    $\therefore \quad x = 0$。 \\
    即 \quad $\log_{7}{1} = 0$。
}

\end{xiaoxiaotis}


例 5 可以推广到一般情况,设 $a$ 是不等于 $1$ 的正数,则
\begin{gather*}
    \bm{\log_{a}{a} = 1 \douhao} \\
    \bm{\log_{a}{1} = 0 \juhao}
\end{gather*}

根据对数的定义,同学们可以自己证明这两个等式。



\lianxi
\begin{xiaotis}

\xiaoti{把下列指数式写成对数式:}
\begin{xiaoxiaotis}

    \begin{tblr}{columns={12em, colsep=0pt}, rows={rowsep=0.5em}}
        \xxt{$2^5 = 32$;} & \xxt{$10^3 = 1000$;} & \xxt{$7^{-2} = \dfrac{1}{49}$;} \\
        \xxt{$10^0 = 1$;} & \xxt{$8^{\frac{2}{3}} = 4$;} & \xxt{$27^{-\frac{1}{3}} = \dfrac{1}{3}$。}
    \end{tblr}
\end{xiaoxiaotis}


\xiaoti{把下列对数式写成指数式:}
\begin{xiaoxiaotis}

    \begin{tblr}{columns={12em, colsep=0pt}}
        \xxt{$\log_{3}{9} = 2$;} & \xxt{$\log_{2}{\dfrac{1}{4}} = -2$;} & \xxt{$\log_{10}{0.001} = -3$;} \\
        \xxt{$\log_{8}{2} = \dfrac{1}{3}$;} & \xxt{$\log_{5}{5} = 1$;} & \xxt{$\log_{27}{\dfrac{1}{9}} = -\dfrac{2}{3}$。}
    \end{tblr}
\end{xiaoxiaotis}


\xiaoti{求下列各式的值:}
\begin{xiaoxiaotis}

    \begin{tblr}{columns={18em, colsep=0pt}}
        \xxt{$\log_{5}{25}$;} & \xxt{$\log_{2}{\dfrac{1}{16}}$;} \\
        \xxt{$\log_{10}{0.01}$;} & \xxt{$\log_{a}{a^5}$。}
    \end{tblr}
\end{xiaoxiaotis}


\xiaoti{求下列各式中的 $x$:}
\begin{xiaoxiaotis}

    \begin{tblr}{columns={18em, colsep=0pt}}
        \xxt{$\log_{2}{x} = 5$;} & \xxt{$\log_{\frac{1}{2}}{x} = -3$;} \\
        \xxt{$\log_{5}{x} = 1$;} & \xxt{$\log_{3}{x} = 0$。}
    \end{tblr}
\end{xiaoxiaotis}


\xiaoti{求下列各式的值:}
\begin{xiaoxiaotis}

    \begin{tblr}{columns={18em, colsep=0pt}}
        \xxt{$\log_{15}{15}$;} & \xxt{$\log_{0.4}{1}$;} \\
        \xxt{$\log_{9}{1}$;}   & \xxt{$\log_{3.7}{3.7}$。}
    \end{tblr}
\end{xiaoxiaotis}


\xiaoti{求下列各式的值:}
\begin{xiaoxiaotis}

    \begin{tblr}{columns={18em, colsep=0pt}}
        \xxt{$5^{\log_{5}{10}}$;} & \xxt{$0.2^{\log_{0.2}{5}}$。}
    \end{tblr}
\end{xiaoxiaotis}

\end{xiaotis}

