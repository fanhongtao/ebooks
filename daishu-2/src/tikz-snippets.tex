%----------------------------------
% 图形相关的项
% 需要 TexLive 2020 版本

% 自定义填充格式,用于实现绘制阴影
\tikzdeclarepattern{
    name=lines,
    parameters={
        \pgfkeysvalueof{/pgf/pattern keys/size},
        \pgfkeysvalueof{/pgf/pattern keys/angle},
        \pgfkeysvalueof{/pgf/pattern keys/line width},
    },
    bounding box={(-.1pt,-.1pt) and
        (\pgfkeysvalueof{/pgf/pattern keys/size}+.1pt,
        \pgfkeysvalueof{/pgf/pattern keys/size}+.1pt)},
    tile size={(\pgfkeysvalueof{/pgf/pattern keys/size},
                \pgfkeysvalueof{/pgf/pattern keys/size})},
    tile transformation={rotate=\pgfkeysvalueof{/pgf/pattern keys/angle}},
    defaults={
        size/.initial=7pt,
        angle/.initial=0,
        line width/.initial=.4pt,
    },
    code={
        \draw[line width=\pgfkeysvalueof{/pgf/pattern keys/line width}]
            (0,0) -- (\pgfkeysvalueof{/pgf/pattern keys/size},
                      \pgfkeysvalueof{/pgf/pattern keys/size});
    }
}

\tikzset{
  % 绘制阿基米德螺线。
  % 参数1: 起始半径
  % 参数2: 相邻两条曲线之间的距离
  % 参数3: 开始角度
  % 参数4: 终止角度
  pics/luoxuan/.style args={#1, #2, #3, #4}{
    code = {
      \begin{scope}
        \draw [domain=#3:#4,samples=300,smooth,variable=\t]
            plot({(#1 + #2 * \t/360) * cos(\t)}, {(#1 + #2 * \t/360) * sin(\t)});
      \end{scope}
    }}}

% 绘制垂足符号
% 代码来自:https://tex.stackexchange.com/questions/21752/
\tikzset{
  right angle quadrant/.code={
    \pgfmathsetmacro\quadranta{{1,1,-1,-1}[#1-1]}     % Arrays for selecting quadrant
    \pgfmathsetmacro\quadrantb{{1,-1,-1,1}[#1-1]}},
  right angle quadrant=1, % Make sure it is set, even if not called explicitly
  right angle length/.code={\def\rightanglelength{#1}},   % Length of symbol
  right angle length=2ex, % Make sure it is set...
  right angle symbol/.style n args={3}{
    insert path={
        let \p0 = ($(#1)!(#3)!(#2)$) in     % Intersection
            let \p1 = ($(\p0)!\quadranta*\rightanglelength!(#3)$), % Point on base line
            \p2 = ($(\p0)!\quadrantb*\rightanglelength!(#2)$) in % Point on perpendicular line
            let \p3 = ($(\p1)+(\p2)-(\p0)$) in  % Corner point of symbol
        (\p1) -- (\p3) -- (\p2)
    }
  }
}


% 在数轴上绘制区间
\tikzset{
  infinity interval/.pic={ % 无限区间
    \begin{scope}[infinity interval options/.cd,#1]
      \pgfkeysgetvalue{/tikz/infinity interval options/start}{\start}
      \pgfkeysgetvalue{/tikz/infinity interval options/stop}{\stop}
      \pgfkeysgetvalue{/tikz/infinity interval options/height}{\height}
      \ifthenelse {\start < \stop} {
        \pgfmathsetmacro{\sign}{1}
      }{
        \pgfmathsetmacro{\sign}{-1}
      }
      \pgfmathsetmacro{\posOne}{\sign * 0.2}
      \pgfmathsetmacro{\posTwo}{\sign * 0.5}

      \draw (\start, 0) .. controls (\start+\posOne, \height) and (\start+\posTwo, \height) .. (\start+\posTwo, \height)
        -- (\stop, \height);
    \end{scope}
  },
  infinity interval options/.is family,
  infinity interval options/.cd,
    start/.initial=0,   % 指定参数 start 的默认值
    stop/.initial=1,    % 指定参数 stop 的默认值
    height/.initial=1   % 指定参数 height 的默认值
}

\tikzset{
  interval/.pic={ % 有限区间
    \begin{scope}[interval options/.cd,#1]
      \pgfkeysgetvalue{/tikz/interval options/start}{\start}
      \pgfkeysgetvalue{/tikz/interval options/stop}{\stop}
      \pgfkeysgetvalue{/tikz/interval options/height}{\height}
      \ifthenelse {\start < \stop} {
        \pgfmathsetmacro{\sign}{1}
      }{
        \pgfmathsetmacro{\sign}{-1}
      }
      \pgfmathsetmacro{\posOne}{\sign * 0.2}
      \pgfmathsetmacro{\posTwo}{\sign * 0.5}

      \draw (\start, 0) .. controls (\start+\posOne, \height) and (\start+\posTwo, \height) .. (\start+\posTwo, \height)
        -- (\stop-\posTwo, \height) .. controls (\stop-\posTwo, \height) and (\stop-\posOne, \height) .. (\stop, 0);
      \end{scope}
  },
  interval options/.is family,
  interval options/.cd,
    start/.initial=0,   % 指定参数 start 的默认值
    stop/.initial=1,    % 指定参数 stop 的默认值
    height/.initial=1   % 指定参数 height 的默认值
}
