\subsubsection{四元线性方程组}

对四元线性方程组

\fangchengzu{eq:fcz-4}(\thefangchengzu)
\begin{minipage}[c]{0.90\textwidth}
    $$\begin{cases}
        a_1 x + b_1 y + c_1 z + d_1 w = f_1, \\
        a_2 x + b_2 y + c_2 z + d_2 w = f_2, \\
        a_3 x + b_3 y + c_3 z + d_3 w = f_3, \\
        a_4 x + b_4 y + c_4 z + d_4 w = f_4,
    \end{cases}$$
\end{minipage}
利用第 \ref{subsec:4-4} 节中的两个定理,仿照第 \ref{subsec:4-5} 节中三元线性方程组 \eqref{eq:fcz-2}
的求解方法,可以得出:\textbf{当系数行列式
$$
D = \begin{vmatrix*}
	a_1 & b_1 & c_1 & d_1 \\
	a_2 & b_2 & c_2 & d_2 \\
	a_3 & b_3 & c_3 & d_3 \\
	a_4 & b_4 & c_4 & d_4
\end{vmatrix*} \neq 0
$$
时,四元线性方程组 \eqref{eq:fcz-4} 有唯一解
$\left( \dfrac{D_x}{D},\; \dfrac{D_y}{D},\; \dfrac{D_z}{D},\; \dfrac{D_w}{D} \right)$,
其中 $D_x$,$D_y$,$D_z$,$D_w$ 是将系数行列式 $D$ 中第一、二、三、四列分别换成方程组
\eqref{eq:fcz-4} 的常数项列而得出的四个四阶行列式}。

\setcounter{cntliti}{4}
\liti 利用\nameref{klmfz}解方程组
$$\begin{cases}
    2x + 3y + 11z + 5w = 2, \\
    \phantom{1}x + \phantom{1}y + \phantom{1}5z + 2w = 1, \\
    2x + \phantom{1}y + \phantom{1}3z + 2w = -3 , \\
    \phantom{1}x + \phantom{1}y + \phantom{1}3z + 4w = -3 .
\end{cases}$$

\jie
\begin{align*}
    &D = \begin{vmatrix*}[r]
            2 & 3 & 11 & 5 \\
            1 & 1 &  5 & 2 \\
            2 & 1 &  3 & 2 \\
            1 & 1 &  3 & 4
        \end{vmatrix*}
        \xlongequal[\text{第四行乘以 (-1) 加到第三行}]{
                \scriptsize\begin{gathered}
                    \text{第四行乘以 (-3) 加到第一行} \\[-1em]
                    \text{第四行乘以 (-1) 加到第二行}
                \end{gathered}
            }
        \begin{vmatrix*}[r]
            -1 & 0 & 2 & -7 \\
            0  & 0 & 2 & -2 \\
            1  & 0 & 0 & -2 \\
            1  & 1 & 3 &  4
        \end{vmatrix*}
        = \begin{vmatrix*}[r]
            -1  & 2 & -7 \\
            0   & 2 & -2 \\
            1   & 0 & -2
        \end{vmatrix*} = 14, \\
    &D_x = \begin{vmatrix*}[r]
            2  & 3 & 11 & 5 \\
            1  & 1 &  5 & 2 \\
            -3 & 1 &  3 & 2 \\
            -3 & 1 &  3 & 4
        \end{vmatrix*}
        \xlongequal{\text{第四行乘以 (-1) 加到第三行}}
        \begin{vmatrix*}[r]
            2  & 3 & 11 & 5 \\
            1  & 1 &  5 & 2 \\
            0  & 0 &  0 & -2 \\
            -3 & 1 &  3 & 4
        \end{vmatrix*}
        = 2 \begin{vmatrix*}[r]
            2  & 3 & 11 \\
            1  & 1 &  5 \\
            -3 & 1 &  3
        \end{vmatrix*} = -28, \\
    &D_y = \begin{vmatrix*}[r]
            2 &  2 & 11 & 5 \\
            1 &  1 &  5 & 2 \\
            2 & -3 &  3 & 2 \\
            1 & -3 &  3 & 4
        \end{vmatrix*}
        \begin{aligned}[t]
            &\xlongequal{\text{第四行乘以 (-1) 加到第三行}}
            \begin{vmatrix*}[r]
                2 &  2 & 11 & 5 \\
                1 &  1 &  5 & 2 \\
                1 &  0 &  0 & -2 \\
                1 & -3 &  3 & 4
            \end{vmatrix*} \\
            &\xlongequal{\text{第一列乘以 2 加到第四列}}\quad
            \begin{vmatrix*}[r]
                2 &  2 & 11 & 9 \\
                1 &  1 &  5 & 4 \\
                1 &  0 &  0 & 0 \\
                1 & -3 &  3 & 6
            \end{vmatrix*}
            = \begin{vmatrix*}[r]
                2  & 11 & 9 \\
                1  &  5 & 4 \\
                -3 &  3 & 6
            \end{vmatrix*} = 0,
        \end{aligned} \\
    &D_z = \begin{vmatrix*}[r]
            2 & 3 &  2 & 5 \\
            1 & 1 &  1 & 2 \\
            2 & 1 & -3 & 2 \\
            1 & 1 & -3 & 4
        \end{vmatrix*}
        \begin{aligned}[t]
            &\xlongequal{\text{第四行乘以 (-1) 加到第三行}}
            \begin{vmatrix*}[r]
                2 & 3 &  2 & 5 \\
                1 & 1 &  1 & 2 \\
                1 & 0 &  0 & -2 \\
                1 & 1 & -3 & 4
            \end{vmatrix*} \\
            &\xlongequal{\text{第一列乘以 2 加到第四列}}\quad
            \begin{vmatrix*}[r]
                2 & 3 &  2 & 9 \\
                1 & 1 &  1 & 4 \\
                1 & 0 &  0 & 0 \\
                1 & 1 & -3 & 6
            \end{vmatrix*}
            = \begin{vmatrix*}[r]
                3 &  2 & 9 \\
                1 &  1 & 4 \\
                1 & -3 & 6
            \end{vmatrix*} = 14,
        \end{aligned} \\
    &D_w = \begin{vmatrix*}[r]
            2 & 3 & 11 & 2 \\
            1 & 1 &  5 & 1 \\
            2 & 1 &  3 & -3 \\
            1 & 1 &  3 & -3
        \end{vmatrix*}
        \xlongequal{\text{第四行乘以 (-1) 加到第三行}}
        \begin{vmatrix*}[r]
            2 & 3 & 11 & 2 \\
            1 & 1 &  5 & 1 \\
            1 & 0 &  0 & 0 \\
            1 & 1 &  3 & -3
        \end{vmatrix*}
        = \begin{vmatrix*}[r]
            3 & 11 & 2 \\
            1 &  5 & 1 \\
            1 &  3 & -3
        \end{vmatrix*} = -14,
\end{align*}


$\therefore \quad \begin{aligned}[t]
    &\dfrac{D_x}{D} = \dfrac{-28}{14} = -2,\\
    &\dfrac{D_z}{D} = \dfrac{14}{14} = 1,
\end{aligned}
\qquad \begin{aligned}[t]
    &\dfrac{D_y}{D} = \dfrac{0}{14} = 0,\\
    &\dfrac{D_w}{D} = \dfrac{-14}{14} = -1 \text{。}
\end{aligned}
$

所以方程组的解集是 $\{ (-2,\; 0,\; 1,\; -1)\}$。


\lianxi

利用\nameref{klmfz}解方程组
$$\begin{cases}
    2x - y + 3z + 2w = 6, \\
    3x - 3y + 3z + 2w = 5,\\
    3x - y - z + 2w = 3, \\
    3x - y + 3z - w = 4 \text{。}
\end{cases}$$

