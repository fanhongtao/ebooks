\subsection{反余弦函数}\label{subsec:1-2}

从余弦函数的图象(图\ref{fig:1-4})同样可以看到,余弦函数 $y = \cos x \; (x \in (-\infty, +\infty))$
不存在反函数。但在单调区间 $[0, \pi]$ 上,对于不同的 $x$ 值,$y$ 有不同的值和它对应,
并且随着 $x$ 由 $0$ 增大到 $\pi$,$y$ 由 $1$ 减小到 $-1$,取得值域 $[-1, 1]$ 上的一切值。
因此,函数 $y = \cos x \; (x \in [0, \pi])$ 有反函数。

\begin{figure}[htbp]
    \centering
    \begin{tikzpicture}[>=Stealth]
    \draw [dashed] (0, -1) -- (pi, -1) -- (pi, 0);

    \draw [->] (-pi-0.5, 0) -- (1.5*pi+0.5, 0) node[anchor=west] {$x$};
    \draw [->] (0, -1.5) -- (0, 1.5) node[anchor=east] {$y$};
    \node [font=\footnotesize, fill=white, inner sep=0pt] at (0.3, -0.3) {$O$};
    \draw (0,1) -- (0.2, 1) node [font=\footnotesize] at (-0.2, 0.8) {$1$};
    \draw (0,-1) -- (0.2, -1) node [font=\footnotesize] at (-0.3, -1) {$-1$};
    \foreach \x / \name in {
        -0.5*pi/$-\dfrac{\pi}{2}$,
        0.5*pi/$\dfrac{\pi}{2}$,
        pi/$\pi$} {
        \draw (\x, 0) -- (\x, 0.2);
        \node [anchor=south, font=\footnotesize, fill=white, inner sep=1pt] at (\x, 0) {\name};
    }

    \draw[domain=-0.9*pi:1.5*pi,samples=50,smooth] plot (\x, {cos(\x r)});
    \node at (1, 1.2) {$y = \cos x$};
\end{tikzpicture}

    \caption{}\label{fig:1-4}
\end{figure}

函数 $y = \cos x \; (x \in [0, \pi])$ 的反函数叫做\textbf{反余弦函数},记作 $y = \arccos x$,
它的定义域是 $[-1, 1]$,值域是 $[0, \pi]$。

这样,对于属于 $[-1, 1]$ 的每一个 $x$ 值,$\arccos x$ 就表示属于 $[0, \pi]$ 的唯一确定的一个值,
它的余弦正好等于已知的 $x$。也可以说,$\arccos x$ 表示属于 $[0, \pi]$ 的唯一确定的一个角〈弧度数),
这个角的余弦恰好等于 $x$。例如,对于 $x = \dfrac{1}{2}$,$y = \arccos \dfrac{1}{2}$
就表示 $[0, \pi]$ 上使 $\cos y = \dfrac{1}{2}$ 的唯一确定的一个角,这个角是 $\dfrac{\pi}{3}$,
因为根据余弦函数 $y = \cos x$ 在 $[0, \pi]$ 上的单调性可以知道,在 $[0, \pi]$ 上,除了
$\dfrac{\pi}{3}$ 外,其他任何角的余弦都不等 $\dfrac{1}{2}$。

由此可以得到
$$\cos\left( \arccos \dfrac{1}{2} \right) = \dfrac{1}{2} \text{。}$$

一般地,根据反余弦函数的定义,可以得到
$$ \cos(\arccos x) = x \text{,}$$
其中 $x \in [-1, 1]$,$\arccos x \in [0, \pi]$。

反余弦函数 $y = \arccos x$ 的图象如图 \ref{fig:1-5} 所示,它是与余弦函数 $y = \cos x$
在 $[0, \pi]$ 上的一段图象关于直线 $y = x$ 对称的图形。

\begin{figure}[htbp]
    \centering
    \begin{tikzpicture}[>=Stealth]
    \pgfmathsetmacro{\half}{0.5 * pi};

    \draw [->] (-1.5, 0) -- (pi+0.5, 0) node[anchor=west] {$x$};
    \draw [->] (0, -1) -- (0, pi+0.8) node[anchor=east] {$y$};
    \node [font=\footnotesize, fill=white, inner sep=0pt] at (0.3, -0.3) {$O$};

    \draw (-1, 0.2) -- (-1, 0) node[below, font=\footnotesize] {$-1$};
    \node[below, font=\footnotesize] at (1, 0) {$1$};
    \node[above, font=\footnotesize] at (\half, 0) {$\frac{\pi}{2}$};
    \draw (pi, 0) -- (pi, 0.2) node[above, font=\footnotesize] {$\pi$};

    \node[right, font=\footnotesize] at (0, \half) {$\frac{\pi}{2}$};
    \node[right, font=\footnotesize] at (0, pi) {$\pi$};

    \draw (-0.5, -0.5) -- (pi, pi) node [anchor=west] {$y = x$};
    \draw[dashed, domain=0:pi,smooth] plot (\x, {cos(\x r)}) node [below] {$y = \cos x$} node[below=10pt] {$x \in [0, \pi]$};
    \draw[domain=-1:1,smooth,samples=30] plot (\x, {rad(acos(\x))}) node at (-1.4, 3.4) {$y = \arccos x$};

    \draw[dashed] (-1, pi) -- (0, pi);
    \draw[dashed] (pi, -1) -- (pi, 0);
\end{tikzpicture}

    \caption{}\label{fig:1-5}
\end{figure}

从图象上可以看出:\textbf{反余弦函 $y = \arccos x$ 在区间 $[-1, 1]$ 上是减函数。}
它既不是偶函数,也不是奇函数。

下面我们来证明:\textbf{对于任意 $x \in [-1,1]$,有}
$$ \arccos (-x) = \pi - \arccos x \text{。}$$

\zhengming 由 $-1 \leqslant x \leqslant 1$,得 $1 \geqslant -x \geqslant -1$,
即 $-x$ 属于反余弦函数的定义域 $[-1, 1]$。

根据诱导公式与反余弦函数的定义,得
$$\cos(\pi - \arccos x) = -\cos(\arccos x) - x \text{,}$$
因此,$\pi - \arccos x$ 是余弦等于 $-x$ 的一个值。

又因 $0 \leqslant \arccos x \leqslant \pi$,所以 $0 \geqslant -\arccos x \geqslant -\pi$,
由此可得 $\pi \geqslant \pi - \arccos x \geqslant 0$,即 $\pi - \arccos x \in [0, \pi]$。

因此,$\pi - \arccos x$ 是属于 $[0, \pi]$ 且它的余弦等于 $-x$ 的一个值。于是
$$\arccos(-x) = \pi - \arccos x \text{。}$$

\liti 求下列各式的值:
\begin{xiaoxiaotis}

    \renewcommand\arraystretch{1.8}
    \begin{tabular}[t]{*{2}{@{}p{16em}}}
        \xiaoxiaoti{$\arccos \dfrac{\sqrt{3}}{2}$;} & \xiaoxiaoti{$\arccos \left( -\dfrac{\sqrt{2}}{2} \right)$;} \\
        \xiaoxiaoti{$\cos \left[ \arccos\left( -\dfrac{\sqrt{2}}{3} \right)\right]$;} & \xiaoxiaoti{$\arccos\left( \cos\dfrac{11\pi}{6} \right)$。}
    \end{tabular}

\end{xiaoxiaotis}

\jie (1)因为在 $[0, \pi]$ 上,$\cos\dfrac{\pi}{6} = \dfrac{\sqrt{3}}{2}$,所以
$$\arccos\dfrac{\sqrt{3}}{2} = \dfrac{\pi}{6} \text{。} $$

(2)因为在 $[0, \pi]$ 上,$\cos\dfrac{3\pi}{4} = -\dfrac{\sqrt{2}}{2}$,所以
$$\arccos \left( -\dfrac{\sqrt{2}}{2} \right) = \dfrac{3\pi}{4} \text{。}$$

或:$\arccos\left( -\dfrac{\sqrt{2}}{2} \right) = \pi - \arccos \dfrac{\sqrt{2}}{2} = \pi - \dfrac{\pi}{4} = \dfrac{3\pi}{4}$。

(3) $\because \quad -\dfrac{\sqrt{2}}{3} \in [-1, 1]$,

$\therefore \quad \cos \left[ \arccos\left( -\dfrac{\sqrt{2}}{3} \right)\right] = -\dfrac{\sqrt{2}}{3}$。

(4)$\arccos\left( \cos\dfrac{11\pi}{6} \right) = \arccos \dfrac{\sqrt{3}}{2} = \dfrac{\pi}{6}$。

\liti 求下列各式的值:
\begin{xiaoxiaotis}

    \xiaoxiaoti{$\sin\left[ \arccos\left( -\dfrac{4}{5} \right)\right]$;}

    \xiaoxiaoti{$\tan(\arccos x), \; x \in [-1, 1]$,且 $x \neq 0$;}

    \xiaoxiaoti{$\cos\left[ \arccos\dfrac{4}{5} + \arccos\left( -\dfrac{5}{13} \right)\right]$。}

\end{xiaoxiaotis}

\jie (1)设 $\arccos\left( -\dfrac{4}{5} \right) = \alpha$,则 $\cos\alpha = -\dfrac{4}{5}$。
由 $\alpha \in [0, \pi]$,得 $\sin\alpha \geqslant 0$,可知:
$$\sin\alpha = \sqrt{1 - \cos^2\alpha} = \sqrt{1 - \left( -\dfrac{4}{5} \right)^2} = \dfrac{3}{5} \text{。}$$

$\therefore \quad \sin\left[ \arccos\left( -\dfrac{4}{5} \right)\right] = \dfrac{3}{5}$。

(2)由 $\arccos x \in [0, \pi]$,知 $\sin(\arccos x) \geqslant 0$。

$\therefore \quad
\begin{aligned}[t]
    \tan(\arccos x) &= \dfrac{\sin(\arccos x)}{\cos(\arccos x)} \\
        & = \dfrac{\sqrt{1 - [\cos(\arccos x)]^2}}{\cos(\arccos x)} \\
        & = \dfrac{\sqrt{1 - x^2}}{x} \text{。}
\end{aligned}$

(3)设 $\arccos \dfrac{4}{5} = \alpha$,则 $\cos\alpha = \dfrac{4}{5}$,$\alpha$ 是第一象限的角,

$\therefore \quad \sin\alpha = \sqrt{1 - \cos^2\alpha} = \dfrac{3}{5}$。

又设 $\arccos\left( -\dfrac{5}{13} \right) = \beta$,则 $\cos\beta = -\dfrac{5}{13}$,$\beta$ 是第二象限的角,

$\therefore \quad \sin\beta = \sqrt{1 - \cos^2\beta} = \dfrac{12}{13}$。

代入原式,得
\begin{align*}
    & \cos\left[ \arccos\dfrac{4}{5} + \arccos\left( -\dfrac{5}{13} \right)\right] \\
    = & \cos(\alpha + \beta) = \cos\alpha \cos\beta - \sin\alpha \sin\beta \\
    = & \dfrac{4}{5} \cdot \left( -\dfrac{5}{13} \right) - \dfrac{3}{5} \cdot \dfrac{12}{13} = -\dfrac{56}{65} \text{。}
\end{align*}

\lianxi
\begin{xiaotis}

\xiaoti{用反余弦的形式把下列各式中的 $x \; (x \in [0, \pi])$ 表示出来:}
\begin{xiaoxiaotis}

    \renewcommand\arraystretch{1.5}
    \begin{tabular}[t]{*{2}{@{}p{16em}}}
        \xiaoxiaoti{$\cos x = \dfrac{2}{3}$;} & \xiaoxiaoti{$\cos x = -\dfrac{1}{5}$;} \\
        \xiaoxiaoti{$\cos x = 0.8065$;} & \xiaoxiaoti{$\cos x = \alpha \quad (\alpha \in [-1, 1])$。}
    \end{tabular}

\end{xiaoxiaotis}

\xiaoti{}
\begin{xiaoxiaotis}

    \vspace{-1.7em} \begin{minipage}{0.9\textwidth}
    \xiaoxiaoti{$\arccos 1.2$ 有意义吗,为什么?}
    \end{minipage}

    \xiaoxiaoti{$\cos\left( \arccos\dfrac{\sqrt{5}}{3} \right) = \dfrac{\sqrt{5}}{3}$ 是否成立,为什么?}

\end{xiaoxiaotis}

\xiaoti{写出下列函数的定义域、值域:}
\begin{xiaoxiaotis}

    \renewcommand\arraystretch{1.5}
    \begin{tabular}[t]{*{2}{@{}p{16em}}}
        \xiaoxiaoti{$y = \arccos 3x$;} & \xiaoxiaoti{$y = -5\arccos x$;} \\
        \xiaoxiaoti{$y = \dfrac{1}{2} \arccos\dfrac{x}{4}$;} & \xiaoxiaoti{$y = 3\arccos(2 - 3x)$。}
    \end{tabular}

\end{xiaoxiaotis}

\xiaoti{求下列反余弦函数的值:}
\begin{xiaoxiaotis}

    \renewcommand\arraystretch{1.8}
    \begin{tabular}[t]{*{2}{@{}p{16em}}}
        \xiaoxiaoti{$\arccos\dfrac{\sqrt{2}}{2}$;} & \xiaoxiaoti{$\arccos 0$;} \\
        \xiaoxiaoti{$\arccos\left( -\dfrac{3}{4} \right)$;} & \xiaoxiaoti{$\arccos 0.0471$。}
    \end{tabular}

\end{xiaoxiaotis}

\xiaoti{求下列各式的值:}
\begin{xiaoxiaotis}

    \renewcommand\arraystretch{1.5}
    \begin{tabular}[t]{*{2}{@{}p{16em}}}
        \xiaoxiaoti{$\cos(\arccos 0.8795)$;} & \xiaoxiaoti{$\arccos(\cos 0.8795)$;} \\
        \xiaoxiaoti{$\cos\left[ \arccos\left( -\dfrac{1}{4} \right)\right]$;} & \xiaoxiaoti{$\arccos\left[ \cos\left( -\dfrac{1}{4} \right)\right]$。}
    \end{tabular}

\end{xiaoxiaotis}

\xiaoti{求下列各式的值:}
\begin{xiaoxiaotis}

    \renewcommand\arraystretch{1.8}
    \begin{tabular}[t]{*{2}{@{}p{16em}}}
        \xiaoxiaoti{$\sin\left( \arccos\dfrac{2}{7} \right)$;} & \xiaoxiaoti{$\cos\left( 2\arccos\dfrac{4}{5} \right)$;} \\
        \xiaoxiaoti{$\sin\left[ \dfrac{\pi}{3} + \arccos\left( -\dfrac{1}{4} \right)\right]$;} & \xiaoxiaoti{$\cot(\arccos x), \; x \in (-1, 1)$。}
    \end{tabular}

\end{xiaoxiaotis}

\end{xiaotis}
