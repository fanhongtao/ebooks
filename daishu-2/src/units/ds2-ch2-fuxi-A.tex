{\centering \nonumsubsection{A \hspace{1em} 组}}

\begin{xiaotis}


\xiaoti{数列与数的集合这两个概念有什么区别与联系?}

\xiaoti{已知数列的每一项是它的序号的平方减去序号的 $5$ 倍,求这个数列的第 $8$ 项与第 $15$ 项。 $44$ 与 $66$ 是这个数列中的项吗?}

\xiaoti{}
\begin{xiaoxiaotis}

    \vspace{-1.6em} \begin{minipage}{0.95\textwidth}
    \xiaoxiaoti{已知数列 $\{a_n\}$ 中的 $a_1 = 5$,$a_{n+1} = a_n - 2$,求证这个数列是等差数列,并写出它的通项公式;}
    \end{minipage}

    \xiaoxiaoti{已知数列 $\{a_n\}$ 中的 $a_1 = 2$,$a_{n+1} = \dfrac{a_n}{3}$ ,求证这个数列是等比数列,并写出它的通项公式。}

\end{xiaoxiaotis}

\xiaoti{在等差数列 $\{a_n\}$ 中:}
\begin{xiaoxiaotis}

    \xiaoxiaoti{已知 $a_1$,$a_n$,$n$,求 $d$;}

    \xiaoxiaoti{已知 $a_1$,$a_n$,$d$,求 $n$ 与 $S_n$;}

    \xiaoxiaoti{已知 $a_n$,$n$,$S_n$,求 $a_1$ 与 $d$;}

    \xiaoxiaoti{已知 $a_1$,$n$,$S_n$,求 $d$ 与 $a_n$。}

\end{xiaoxiaotis}


\xiaoti{在等比数列 $\{a_n\}$ 中:}
\begin{xiaoxiaotis}

    \xiaoxiaoti{已知 $n$,$q$,$a_n$,求 $a_1$ 与 $S_n$;}

    \xiaoxiaoti{已知 $q$,$n$,$S_n$,求 $a_1$ 与 $a_n$;}

    \xiaoxiaoti{已知 $a_1$,$q$,$S_n$,求 $a_n$;}

    \xiaoxiaoti{已知 $q$,$a_n$,$S_n$,求 $a_1$。}

\end{xiaoxiaotis}


\xiaoti{已知 $a^2$,$b^2$,$c^2$ 成等差数列,求证 $\dfrac{1}{b+c}$,$\dfrac{1}{c+a}$,$\dfrac{1}{a+b}$ 也成等差数列。}


\xiaoti{已知 $a$,$b$,$c$,$d$ 成等比数列,求证:}
\begin{xiaoxiaotis}

    \xiaoxiaoti{$a+b$,$b+c$,$c+d$ 成等比数列;}

    \xiaoxiaoti{$(a-d)^2 = (b-c)^2 + (c-a)^2 + (d-b)^2$。}

\end{xiaoxiaotis}


\xiaoti{}
\begin{xiaoxiaotis}

    \vspace{-1.6em} \begin{minipage}{0.95\textwidth}
    \xiaoxiaoti{在 $a$ 与 $b$ 中间插入 $10$ 个数,使这 $12$ 个数成等差数列,求这个数列的第 $6$ 项;}
    \end{minipage}

    \xiaoxiaoti{已知 $b > a > 0$,在 $a$ 与 $b$ 中间插入 $10$ 个数,使这 $12$ 个数成等比数列,求这个数列的第 $10$ 项。}

\end{xiaoxiaotis}


\xiaoti{}
\begin{xiaoxiaotis}

    \vspace{-1.6em} \begin{minipage}{0.95\textwidth}
    \xiaoxiaoti{已知  $\{x_n\}$ 是等差数列,$y_n = ax_n + b$,其中 $a$,$b$ 是常数,$a \neq 0$,求证 $\{y_n\}$ 是等差数列;}
    \end{minipage}

    \xiaoxiaoti{已知 $\{x_n\}$ 是等比数列,$y_n = ax_n$,其中 $a$ 是不为零的常数,求证 $\{y_n\}$ 是等比数列。}

\end{xiaoxiaotis}


\xiaoti{解方程 $\lg x + \lg x^2 + \cdots + \lg x^n = n^2 + n$。}


\xiaoti{成等差数列的三个正数的和等于 $15$,并且这三个数分别加上 $1$,$3$,$9$ 后又成等比数列。求这三个数。}


\xiaoti{有四个数,其中前三个数成等差数列,后三个数成等比数列,并且第一个数与第四个数的和是 $37$,
    第二个数与第三个数的和是 $36$。求这四个数。}


\xiaoti{如果 $a$,$b$,$c$ 成等差数列,$x$,$y$,$z$ 成等比数列,且 $x$,$y$,$z$ 都是正数,求证
    $$ (b - c)\log_m x + (c - a)\log_m y + (a - b)\log_m z = 0 \text{。} $$
    \vspace{-1em}
}

\xiaoti{求下面数列的前 $n$ 项的和:}
\begin{xiaoxiaotis}

    \xiaoxiaoti{$1 \times 4$,$2 \times 5$,$3 \times 6$,$\cdots$,$n(n+3)$,$\cdots$;}

    \xiaoxiaoti{$1\dfrac{1}{2}$,$2\dfrac{1}{4}$,$3\dfrac{1}{8}$,$\cdots$,$\left( n+\dfrac{1}{2^n} \right)$,$\cdots$。}

\end{xiaoxiaotis}

用数学归纳法证明( 第 $15$ 题 ~ $16$ 题)。

\xiaoti{}
\begin{xiaoxiaotis}

    \vspace{-1.6em} \begin{minipage}{0.9\textwidth}
    \xiaoxiaoti{$1 \cdot 2 \cdot 3 + 2 \cdot 3 \cdot 4 + 3 \cdot 4 \cdot 5 + \cdots + n(n+1)(n+2) = \dfrac{1}{4} n(n+1)(n+2)(n+3)$;}
    \end{minipage}

    \xiaoxiaoti{$(a_1 + a_2 + \cdots + a_n)^2 = a_1^2 + a_2^2 + \cdots + a_n^2 + 2(a_1a_2 + a_2a_3 + \cdots + a_{n-1}a_n)$。}

\end{xiaoxiaotis}


\xiaoti{}
\begin{xiaoxiaotis}

    \vspace{-1.6em} \begin{minipage}{0.9\textwidth}
    \xiaoxiaoti{$4^{2n+1} + 3^{n+2} \; (n \in N)$ 能被 $13$ 整除;}
    \end{minipage}

    \xiaoxiaoti{$6^{2n-1} + 1 \; (n \in N)$ 能被 $7$ 整数。}

\end{xiaoxiaotis}


\xiaoti{已知数列 $\{a_n\}$ 的项满足
    $$\begin{cases}
        a_1 = b, \\
        a_{n+1} = ca_n + d,
    \end{cases}$$
    其中 $c \neq 1$。证明这个数列的通项公式是
    $$ a_n = \dfrac{bc^n + (d - b)c^{n-1} - d}{c-1} \text{。}$$
}

\xiaoti{已知数列
    $$ \dfrac{1}{1 \cdot 2} \text{,} \dfrac{1}{2 \cdot 3} \text{,} \dfrac{1}{3 \cdot 4} \text{,} \cdots \text{,} \dfrac{1}{n (n+1)} \text{,} \cdots \text{。} $$
    计算 $S_1$,$S_2$,$S_3$,由此推测计算 $S_n$ 的公式,然后用数学归纳法证明这个公式。
}

\end{xiaotis}


