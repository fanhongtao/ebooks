\subsection{不等式的证明}\label{subsec:3-3}

由于不等式的形式是多种多样的,所以不等式的证明方法也就不同。下面举例说明一些常用的证明方法。

\liti 求证 $x^2 + 3 > 3x$。

我们已经知道,$a - b > 0 \iff a > b$。因此,要证明 $a > b$,只要证明 $a - b > 0$。
这是证明不等式常用的一种方法,通常叫做\textbf{比较法}\mylabel{def:bijiaofa}。

\zhengming \; $\because \begin{aligned}[t]
    & (x^2 + 3) - 3x \\
={} & x^2 - 3x + \left( \dfrac{3}{2} \right)^2 - \left( \dfrac{3}{2} \right)^2 + 3 \\
={} & \left( x - \dfrac{3}{2} \right)^2 + \dfrac{3}{4} \geqslant \dfrac{3}{4} > 0,
\end{aligned}$

$\therefore \quad x^2 + 3 > 3x$。

注意:为了确定不等式两边的差的正负,有时要把这个差变形为一个常数,或者变形为一个常数与一个或几个平方的和的形式,
也可变形为几个因式的积的形式,以便于判断其正负。


\liti 已知 $a,\; b \in R^+$,并且 $a \neq b$,求证
$$ a^5 + b^5 > a^3b^2 + a^2b^3 \text{。} $$

\zhengming $\begin{aligned}[t]
        & (a^5 + b^5) - (a^3b^2 + a^2b^3) \\
    ={} & (a^5 - a^3b^2) - (a^2b^3 - b^5) \\
    ={} & a^3(a^2 - b^2) - b^3(a^2 - b^2) \\
    ={} & (a^2 - b^2)(a^3 - b^3) \\
    ={} & (a + b)(a - b)^2(a^2 + ab + b^2) \text{。}
\end{aligned}$

$\because \quad a,\; b \in R^+$,

$\therefore \quad \begin{aligned}[t]
    & a + b > 0, \\
    & a^2 + ab + b^2 > 0 \text{。}
\end{aligned}$

又因为 $a \neq b$,可知

$\qquad (a - b)^2 > 0$。

$\therefore \quad (a + b)(a - b)^2(a^2 + ab + b^2) > 0$,\\
即

$\qquad (a^5 + b^5) - (a^3b^2 + a^2b^3) > 0 \text{。} $

$\therefore \quad a^5 + b^5 > a^3b^2 + a^2b^3 \text{。} $


\lianxi
\begin{xiaotis}

\xiaoti{求证 $(x - 3)^2 > (x - 2)(x - 4)$。}

\xiaoti{已知 $a \neq b$,求证 $a^2 + 3b^2 > 2b(a + b)$。}

\xiaoti{已知 $a,\; b \in R^+$,且 $a \neq b$,求证
    $$ a^4 + b^4 > a^3b + ab^3 \text{。} $$
}

\shangyihang
\xiaoti{已知 $a \neq 2$,求证 $\dfrac{4a}{4 + a^2} < 1$。}

\end{xiaotis}

\,

证明不等式还常常用到下面的定理和推论。

\setcounter{theorem}{0}
\begin{theorem} \label{theorem:bdszm-1}
    如果 $a,\; b \in R$,那么 $a^2 + b^2 \geqslant 2ab$ (当且仅当 $a = b$ 时取 “$=$” 号)。
\end{theorem}

\zhengming $a^2 + b^2 - 2ab = (a - b)^2$。

当 $a \neq b$ 时,$(a - b)^2 > 0$,当 $a = b$ 时,$(a - b)^2 = 0$,所以
$$ (a - b)^2 \geqslant 0 \text{,} $$
即
$$ a^2 + b^2 - 2ab \geqslant 0 \text{。} $$

$\therefore \quad a^2 + b^2 \geqslant 2ab$。

\setcounter{corollary}{0}
\begin{corollary} \label{corollary:bdszm-1-1}
    如果 $a,\; b \in R^+$,那么 $\dfrac{a + b}{2} \geqslant \sqrt{ab}$ (当且仅当 $a = b$ 时取 “$=$” 号)。
\end{corollary}

这是因为
$$ (\sqrt{a})^2 + (\sqrt{b})^2 \geqslant 2\sqrt{a}\sqrt{b} \implies a + b \geqslant 2\sqrt{ab} \implies \dfrac{a + b}{2} \geqslant \sqrt{ab} \text{。} $$

如果 $a_1$,$a_2$,$\cdots$,$a_n \in R+$,且 $n > 1$,那么
$$ \dfrac{a_1 + a_2 + \cdots + a_n}{n} $$
叫做这 $n$ 个正数的\textbf{算术平均数}\mylabel{def:suanshupingjunshu},
$$ \sqrt[n]{a_1 a_2 \cdots a_n} $$
叫做这 $n$ 个正数的\textbf{几何平均数}\mylabel{def:jihepingjunshu}。

上面的推论就是:\textbf{两个正数的算术平均数不小于(即大于或等于)它们的几何平均数。}

\liti 已知 $x,\; y \in R+$,$x + y = S$,$xy = P$。求证:

(1) 如果 $P$ 是定值,那么当且仅当 $x = y$ 时,$S$ 的值最小;

(2) 如果 $S$ 是定值,那么当且仅当 $x = y$ 时,$P$ 的值最大。

\zhengming (1) 因为 $x,\; y \in R+$,所以
$$ \dfrac{x + y}{2} \geqslant \sqrt{xy} \text{,}$$
$$ x + y \geqslant 2\sqrt{xy} \text{。}$$
即
$$ S \geqslant 2\sqrt{P} \quad \text{(当且仅当 $x = y$ 时取 “$=$” 号)。}$$

这就是说,如果 $P$ 是定值,那么当且仅当 $x = y$ 时,$S$ 有最小值 $2\sqrt{P}$。

(2) 从 (1) 的证明可知 $S \geqslant 2\sqrt{P}$。现将它化成
$$ \sqrt{P} \leqslant \dfrac{S}{2} \text{,} $$

$\therefore \quad P \leqslant \dfrac{S^2}{4} \quad \text{(当且仅当 $x = y$ 时取 “$=$” 号)。}$

这就是说,如果 $S$ 是定值,那么当且仅当 $x = y$ 时,$P$ 有最大值 $\dfrac{1}{4} S^2$。


\begin{theorem} \label{theorem:bdszm-2}
    如果 $a,\; b,\; c \in R^+$,那么 $a^3 + b^3 + c^3 \geqslant 3abc$ (当且仅当 $a = b = c$ 时取 “$=$” 号)。
\end{theorem}

\zhengming $\because \begin{aligned}[t]
        & a^3 + b^3 + c^3 - 3abc \\
    ={} & (a + b)^3 + c^3 - 3a^2b - 3ab^2 - 3abc \\
    ={} & (a + b + c)[(a + b)^2 - (a + b)c + c^2] - 3ab(a + b + c) \\
    ={}   & (a + b + c)[a^2 + 2ab + b^2 -ac - bc + c^2 -3ab] \\
    ={} & (a + b + c)(a^2 + b^2 + c^2 -ab - bc - ca) \\
    ={} & \dfrac{1}{2}(a + b + c) \times [(a - b)^2 + (b - c)^2 + (c - a)^2] \geqslant 0 \text{,}
\end{aligned}$

$\therefore \quad a^3 + b^3 + c^3 \geqslant 3abc$。

很明显,当且仅当 $a = b = c$ 时取 “$=$” 号。

\begin{corollary} \label{corollary:bdszm-2-1}
    如果 $a,\; b,\; c \in R^+$,那么 $\dfrac{a + b + c}{3} \geqslant \sqrt[3]{abc}$ (当且仅当 $a = b = c$ 时取 “$=$” 号)。
\end{corollary}

这是因为

$\quad \begin{aligned}[t]
    & (\sqrt[3]{a})^3 + (\sqrt[3]{b})^3 + (\sqrt[3]{c})^3 \geqslant 3 \sqrt[3]{a} \cdot \sqrt[3]{b} \cdot \sqrt[3]{c} \\
    & \implies a + b + c \geqslant 3 \sqrt[3]{abc} \\
    & \implies \dfrac{a + b + c}{3} \geqslant \sqrt[3]{abc} \text{。}
\end{aligned}$


\liti 已知 $a$,$b$,$c$ 是不全相等的正数,求证
$$ a(b^2 + c^2) + b(c^2 + a^2) + c(a^2 + b^2) > 6abc \text{。} $$

我们可以利用某些已经证明过的不等式(如上面的定理及其推论)作为基础,再运用不等式的性质推导出所要求证的不等式。这种证明方法通常叫做\textbf{综合法}\mylabel{def:zonghefa}。

\zhengming \; $\because$
\shangyihang $$ b^2 + c^2 \geqslant 2bc,\quad a > 0, $$

$\therefore$
\vspace{-1.5em}\begin{equation}
    a(b^2 + c^2) \geqslant 2abc \text{。} \tag{1}
\end{equation}

同理,

$\therefore$
\vspace{-1.5em}\begin{gather}
    b(c^2 + a^2) \geqslant 2abc \text{,} \tag{2} \\
    c(a^2 + b^2) \geqslant 2abc \text{。} \tag{3}
\end{gather}

因为 $a$,$b$,$c$ 不全相等,所以 (1),(2),(3) 中至少有一式不能取 “$=$” 号。

$\therefore \quad a(b^2 + c^2) + b(c^2 + a^2) + c(a^2 + b^2) > 6abc$。


\liti 已知 $a,\; b,\; c,\; d \in R^+$,求证
$$ (ab + cd)(ac + bd) \geqslant 4abcd \text{。} $$

\zhengming 由 $a,\; b,\; c,\; d \in R^+$,得
$$\begin{aligned}
    \dfrac{ab + cd}{2} \geqslant \sqrt{ab \cdot cd} > 0 , \\
    \dfrac{ac + bd}{2} \geqslant \sqrt{ac \cdot bd} > 0 .
\end{aligned}$$

$\therefore$
\vspace{-1.5em}$$\quad \dfrac{(ab + cd)(ac + bd)}{4} \geqslant abcd \text{,}$$
即
$$ (ab + cd)(ac + bd) \geqslant 4abcd \text{。} $$


\liti 已知 $x,\; y,\; z \in R^+$,求证
$$ (x + y + x)^3 \geqslant 27xyz \text{。} $$

\zhengming \; $\because \quad \dfrac{x + y + z}{3} \geqslant \sqrt[3]{xyz} > 0,$

$\therefore \quad \dfrac{(x + y + z)^3}{27} \geqslant xyz , $ \\
即
$$ (x + y + z)^3 \geqslant 27xyz \text{。} $$


\lianxi
\begin{xiaotis}
\setcounter{cntxiaoti}{0}

\xiaoti{已知 $a$,$b$,$c$ 是不全相等的正数,求证:}
\begin{xiaoxiaotis}

    \xiaoxiaoti{$(a + b)(b + c)(c + a) > 8abc$;}

    \xiaoxiaoti{$a + b + c > \sqrt{ab} + \sqrt{bc} + \sqrt{ca}$。}

\end{xiaoxiaotis}

\xiaoti{已知 $x,\; y,\; z \in R^+$,求证:}
\begin{xiaoxiaotis}

    \renewcommand\arraystretch{1.5}
    \begin{tabular}[t]{*{2}{@{}p{16em}}}
        \xiaoxiaoti{$\dfrac{x}{y} + \dfrac{y}{x} \geqslant 2$;} & \xiaoxiaoti{$\dfrac{x}{y} + \dfrac{y}{z} + \dfrac{z}{x} \geqslant 3$。}
    \end{tabular}

\end{xiaoxiaotis}


\xiaoti{求证当 $x > 0$ 时,$x + \dfrac{16}{x}$ 的最小值是 $8$。}

\end{xiaotis}

\,

\liti 已知 $a,\; b,\; m \in R^+$,并且 $a < b$,求证
$$ \dfrac{a + m}{b + m} > \dfrac{a}{b} \text{。} $$

证明不等式时,有时可以从求证的不等式出发,分析使这个不等式成立的条件,把证明这个不等式转化为
判定这些条件是否具备的问题。如果能够肯定这些条件都已具备,那么就可以断定原不等式成立。
这种证明方法通常叫做\textbf{分析法}\mylabel{def:fenxifa}。

\zhengming 因为 $a,\; b,\; m \in R^+$,为了要证明
$$ \dfrac{a + m}{b + m} > \dfrac{a}{b} \text{,} $$
只需证明
$$ (a + m)b > a(b + m) \text{,} $$
即
$$ bm > am \text{,} $$
因此,只需证明
$$ b > a \text{。} $$

因为 $b > a$ 成立(题设),所以
$$ \dfrac{a + m}{b + m} > \dfrac{a}{b} $$
成立。

\liti 求证 $\sqrt{2} + \sqrt{7} < \sqrt{3} + \sqrt{6}$。

\textbf{证法一:} 为了要证明
$$ \sqrt{2} + \sqrt{7} < \sqrt{3} + \sqrt{6} \text{,} $$
因为 $\sqrt{2} + \sqrt{7}$ 和 $\sqrt{3} + \sqrt{6}$ 都是正数,所以只需证明
$$ (\sqrt{2} + \sqrt{7})^2 < (\sqrt{3} + \sqrt{6})^2 \text{。} $$
展开得
$$ 9 + 2\sqrt{14} < 9 + 2\sqrt{18} \text{,} $$
即
\begin{align*}
    2\sqrt{14} &< 2\sqrt{18} \text{,}  \\
     \sqrt{14} &< \sqrt{18} \text{,}  \\
           14  &< 18 \text{。}
\end{align*}

因为 $14 < 18$ 成立,所以
$$ \sqrt{2} + \sqrt{7} < \sqrt{3} + \sqrt{6} $$
成立。


\textbf{证法二:} \quad $\because$
\shangyihang $$ 14 < 18, $$

$\therefore$
\shangyihang \begin{align*}
     \sqrt{14} &< \sqrt{18}, \\
    2\sqrt{14} &< 2\sqrt{18}, \\
    9 + 2\sqrt{14} &<  9 + 2\sqrt{18}, \\
    (\sqrt{2} + \sqrt{7})^2 &< (\sqrt{3} + \sqrt{6})^2,
\end{align*}

$\therefore$
\shangyihang $$ \sqrt{2} + \sqrt{7} < \sqrt{3} + \sqrt{6} \text{。} $$

注意:证法二用的是\hyperref[def:zonghefa]{综合法}。可以看出,综合过程有时正好是分析过程的逆推。

\liti 如果 $a,\; b \in R^+$,且 $a \neq b$,求证
$$ a^3 + b^3 > a^2b + ab^2 \text{。} $$

\textbf{证法一:}证明
$$ a^3 + b^3 > a^2b + ab^2 \text{,} $$
就是证明
$$ (a + b)(a^2 -ab + b^2) > ab(a + b) \text{。} $$

因为 $a + b > 0$,所以要证明上式,只需证明
\begin{align*}
    a^2 - ab + b^2 > ab, \\
    a^2 - 2ab + b^2 > 0,
\end{align*}
即
$$ (a - b)^2 > 0 \text{。} $$

因为 $a \neq b$,最后的不等式 $(a - b)^2 > 0$ 成立,所以
$$ a^3 + b^3 > a^2b + ab^2 $$
成立。

\textbf{证法二:} 因为 $a \neq b$,所以
\begin{gather*}
    (a - b)^2 > 0, \\
    a^2 - 2ab + b^2 > 0, \\
    a^2 - ab + b^2 > ab \text{。}
\end{gather*}

又因为 $a + b > 0$,所以
$$ (a + b)(a^2 - ab + b^2) > ab(a + b) \text{,} $$
即
$$ a^3 + b^3 > a^2b + ab^2 \text{。} $$

(证法一是\hyperref[def:fenxifa]{分析法},
证法二是\hyperref[def:zonghefa]{综合法}。
还可用\hyperref[def:bijiaofa]{比较法},请同学们自己证明。)

\liti 已知 $x > -1$,且 $x \neq 0$,$n \in N$,且 $n \geqslant 2$,求证
$$ (1 + x)^n > 1 + nx \text{。} $$

一个不等式如果是关于自然数的命题,可试用\hyperref[subsec:2-4]{数学归纳法}来证明。

\zhengming (1) 当 $n = 2$ 时,
\begin{align*}
    \text{左边} &= (1 + x)^2 = 1 + 2x + x^2, \\
    \text{右边} &= 1 + 2x \text{。}
\end{align*}

因为 $x^2 > 0$,所以原不等式成立。

(2) 假设当 $n = k \; (k \geqslant 2)$ 时不等式成立,就是
$$ (1 + x)^k > 1 + kx \text{。} $$

当 $n = k + 1$ 时,因为 $x > -1$,所以 $1 + x > 0$,于是

$\text{左边} = (1 + x)^{k + 1} = (1 + x)^k (1 + x) > (1 + kx)(1 + x) = 1 + (k + 1)x + kx^2$,

$\text{右边} = 1 + (k + 1) x$。

因为 $kx^2 > 0$,所以左边 $>$ 右边,即
$$ (1 + 1)^{k + 1} > 1 + (k + 1)x \text{。} $$

这就是说,原不等式当 $n = k + 1$ 时也成立。

根据 (1) 和 (2),原不等式对任何不小于 $2$ 的自然数都成立。


\lianxi
\begin{xiaotis}
\setcounter{cntxiaoti}{0}

\xiaoti{求证 $\sqrt{6} + \sqrt{7} > 2\sqrt{2} + \sqrt{5}$。}

\xiaoti{求证 $ac + bd \leqslant \sqrt{a^2 + b^2} \cdot \sqrt{c^2 + d^2}$。}

\xiaoti{求证 $2^n > n \; (n \in N)$。}

\end{xiaotis}

