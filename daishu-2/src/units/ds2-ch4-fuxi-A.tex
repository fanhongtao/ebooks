{\centering \nonumsubsection{A \hspace{1em} 组}}

\begin{xiaotis}

\xiaoti{讨论关于 $x$ 的方程 $ax = b$ 的解的几种情况。}

\xiaoti{什么叫做二元线性方程组的一个解?什么叫做二元线性方程组的解集?二元线性方程组的解可能有几种情况?}

\xiaoti{已知方程组
$$\begin{cases}
    a_1x + b_1y = c_1, \\
    a_2x + b_2y = c_2
\end{cases} \quad (a_1,\; b_1 \text{不同时为零,} a_2,\; b_2 \text{不同时为零})
$$
中的两个方程分别表示两条直线 $l_1$ 与 $l_2$,求证:
}
\begin{xiaoxiaotis}

    \xiaoxiaoti{$l_1$,$l_2$ 相交的充要条件是方程组的系数行列式 $D \neq 0$;}

    \xiaoxiaoti{$l_1$,$l_2$ 平行而不重合的充要条件是 $D = 0$,但 $D_x$,$D_y$ 中至少有一个不等于零;}

    \xiaoxiaoti{$l_1$,$l_2$ 重合的充要条件是 $D = D_x = D_y = 0$。}

\end{xiaoxiaotis}

\xiaoti{$a$,$b$ 满足什么条件时,直线 $3x - by = a$ 与 $ax + y - 3 = 0$}
\begin{xiaoxiaotis}

    \threeInLineXxt[10em]{相交?}{平行?}{重合?}

\end{xiaoxiaotis}


\xiaoti{讨论下列方程组,并画出图象来说明所得的结果:}
\begin{xiaoxiaotis}

    \renewcommand\arraystretch{1.2}
    \begin{tabular}[t]{*{2}{@{}p{16em}}}
        \xiaoxiaoti{$\begin{cases}
                x + y = 2, \\
                4x - y = 3;
            \end{cases}$}
        & \xiaoxiaoti{$\begin{cases}
                x - y = 5, \\
                2x - 2y = 7;
            \end{cases}$} \\[1.5em]
        \xiaoxiaoti{$\begin{cases}
                x + 2y = 5, \\
                2x + 4y = 10;
            \end{cases}$;}
        & \xiaoxiaoti{$\begin{cases}
                y = 3x + 2, \\
                y = 3x \text{。}
            \end{cases}$}
    \end{tabular}

\end{xiaoxiaotis}


\xiaoti{解下列关于 $x$,$y$ 的方程组,并进行讨论:}
\begin{xiaoxiaotis}

    \xiaoxiaoti{$\begin{cases}
        (a + 1)x - (2a - 1)y = 3a, \\
        (3a + 1)x - (4a - 1)y = 5a + 4;
    \end{cases}$}

    \xiaoxiaoti{$\begin{cases}
        (a - 1)x + (a + 1)y = 2(a^2 - 1), \\
        (a^2 - 1)x + (a^2 + 1)y = 2(a^3 - 1) \text{。}
    \end{cases}$}

\end{xiaoxiaotis}


\xiaoti{判断下列方程组有没有非零解,如果有,把解求出来:}
\begin{xiaoxiaotis}

    \renewcommand\arraystretch{1.2}
    \begin{tabular}[t]{*{2}{@{}p{16em}}}
        \xiaoxiaoti{$\begin{cases}
                4x - 6y = 0, \\
                6x + 9y = 0;
            \end{cases}$;}
        & \xiaoxiaoti{$\begin{cases}
                5x = 8y, \\
                10x - 16y = 0 \text{。}
            \end{cases}$。}
    \end{tabular}

\end{xiaoxiaotis}


\xiaoti{已知行列式
$$\begin{vmatrix*}[r]
    13 &  22 & 17 \\
    14 & -11 & 16 \\
    10 &   0 & 18
\end{vmatrix*} \text{,}$$
}
\begin{xiaoxiaotis}

    \xiaoxiaoti{用对角线法则展开行列式并进行计算;}

    \xiaoxiaoti{按某一行(或一列)展开行列式并进行计算;}

    \xiaoxiaoti{利用行列式的性质先化简行列式再展开,然后进行计算。}

\end{xiaoxiaotis}

\xiaoti{解下列关于 $x$ 的方程:}
\begin{xiaoxiaotis}

    \renewcommand\arraystretch{1.2}
    \begin{tabular}[t]{*{2}{@{}p{16em}}}
        \xiaoxiaoti{$\begin{vmatrix}
                1+x & 2   & 3 \\
                1   & 2+x & 3 \\
                1   & 2   & 3+x
            \end{vmatrix} = 0$;}
        & \xiaoxiaoti{$\begin{vmatrix}
                \sin x & 1 & \sin x \\
                \cos x & 0 & \sin x \\
                \cos x & 1 & \cos x
            \end{vmatrix} = 0$。}
    \end{tabular}

\end{xiaoxiaotis}


\xiaoti{展开下列行列式,并化简:}
\begin{xiaoxiaotis}

    \renewcommand\arraystretch{1.2}
    \begin{tabular}[t]{*{2}{@{}p{16em}}}
        \xiaoxiaoti{$\begin{vmatrix}
                x   & y   & x+y \\
                y   & x+y & x \\
                x+y & x   & y
            \end{vmatrix}$;}
        & \xiaoxiaoti{$\begin{vmatrix}
                b+c & a-c & a-b \\
                b-c & c+a & b-a \\
                c-b & c-a & a+b
            \end{vmatrix}$。}
    \end{tabular}

\end{xiaoxiaotis}


\xiaoti{求证:}
\begin{xiaoxiaotis}

    \xiaoxiaoti{$\begin{vmatrix}
            a-b-c & 2a    & 2a \\
            2b    & b-c-a & 2b \\
            2c    & 2c    & c-b-a
        \end{vmatrix} = (a+b+c)^3$;}

    \xiaoxiaoti{$\begin{vmatrix*}[r]
            a  &  b & c \\
            -b &  a & d \\
            -c & -d & a
        \end{vmatrix*} = \begin{vmatrix*}[r]
            a &  b &  c \\
            b & -a &  d \\
            c & -d & -a
        \end{vmatrix*}$;}

    \xiaoxiaoti{$\begin{vmatrix}
            b_1 + c_1 & c_1 + a_1 & a_1 + b_1 \\
            b_2 + c_2 & c_2 + a_2 & a_2 + b_2 \\
            b_3 + c_3 & c_3 + a_3 & a_3 + b_3
        \end{vmatrix} = 2 \begin{vmatrix}
            a_1 & b_1 & c_1 \\
            a_2 & b_2 & c_2 \\
            a_3 & b_3 & c_3
        \end{vmatrix}$;}

    \xiaoxiaoti{$\begin{vmatrix}
            \cos(\alpha + \beta) & \sin\alpha & \cos\alpha \\
            \sin(\alpha + \beta) & \cos\alpha & \sin\alpha \\
            1                    & \sin\beta  & \cos\beta
        \end{vmatrix} = 0$。}

\end{xiaoxiaotis}


\xiaoti{解不等式
    $$\begin{vmatrix}
        x-a & b   & -c \\
        a   & x-b & c \\
        -a  & b   & x-c
    \end{vmatrix} > 0 \text{。}$$}


\xiaoti{求证:}
\begin{xiaoxiaotis}

    \xiaoxiaoti{$\begin{vmatrix}
            a & b & c \\
            c & a & b \\
            b & c & a
        \end{vmatrix} = a^3 + b^3 + c^3 - 3abc$;}

    \xiaoxiaoti{$\begin{vmatrix}
            a & b & c \\
            c & a & b \\
            b & c & a
        \end{vmatrix} = (a + b + c)(a^2 + b^2 + c^2 - bc - ca - ab)$;}

    \xiaoxiaoti{如果 $\triangle ABC$ 的三边 $a$,$b$,$c$ 有 $a^3 + b^3 + c^3 = 3abc$ 的关系,那么 $\triangle ABC$ 为等边三角形。}

\end{xiaoxiaotis}


\xiaoti{已知三角形的三顶点 $A(x_1, y_1)$,$B(x_2, y_2)$,$C(x_3, y_3)$,求证三角形的面积
    $$S = \dfrac{1}{2}\begin{vmatrix}
        x_1 & y_1 & 1 \\
        x_2 & y_2 & 1 \\
        x_3 & y_3 & 1
    \end{vmatrix} \text{的绝对值。}$$
}


\xiaoti{利用上题结论,}
\begin{xiaoxiaotis}

    \xiaoxiaoti{求以 $(1, 1)$,$(3, 4)$,$(5, -2)$,$(4, -7)$ 为顶点的四边形的面积;}

    \xiaoxiaoti{求证以三角形三边中点为顶点的三角形的面积等于原三角形面积的四分之一。}

\end{xiaoxiaotis}


\xiaoti{解下列关于 $x$,$y$,$z$ 的方程组:}
\begin{xiaoxiaotis}

    \renewcommand\arraystretch{1.2}
    \begin{tabular}[t]{*{2}{@{}p{16em}}}
        \xiaoxiaoti{$\begin{cases}
                7x - 4\dfrac{1}{2}y = 9\dfrac{1}{2}, \\[1em]
                2x + 3y + 7\dfrac{1}{2}z = 22, \\[1em]
                -\dfrac{2}{3}x + 2\dfrac{1}{2}z = 3\dfrac{2}{3};
            \end{cases}$}
        & \xiaoxiaoti{$\begin{cases}
                \dfrac{x + 2}{y - 2} = \dfrac{3}{2}, \\[1em]
                \dfrac{y + 1}{z + 3} = 4, \\[1em]
                \dfrac{z + 4}{x - 1} = -\dfrac{4}{3};
            \end{cases}$} \\[4em]
        \xiaoxiaoti{$\begin{cases}
                \dfrac{2}{x} - \dfrac{3}{y} + \dfrac{4}{z} = 11, \\[1em]
                \dfrac{3}{x} - \dfrac{2}{y} - \dfrac{5}{z} = -20, \\[1em]
                \dfrac{3}{x} + \dfrac{4}{y} - \dfrac{3}{z} = 6;
            \end{cases}$}
        & \xiaoxiaoti{%\begin{minipage}{13em}
            \vspace{-4em}\begin{multline*}
                \begin{cases}
                    lx = my = nz, \\
                    ax + by + cz = d
                \end{cases} \\
                (amn + bnl + cml \neq 0);
            \end{multline*}
        %\end{minipage}
        } \\
        \xiaoxiaoti{$\begin{cases}
                ax - aby + bz = b, \\
                x + ay - z = -1, \\
                by + z = 1 \text{。}
            \end{cases}$}
    \end{tabular}

\end{xiaoxiaotis}


\xiaoti{求下列关于 $x$,$y$,$z$ 的方程组有唯一解的条件,并把在这个条件下的解求出来:}
\begin{xiaoxiaotis}

    \xiaoxiaoti{$\begin{cases}
        (\lambda + 3)x + y + 2z = \lambda, \\
        \lambda x + (\lambda - 1)y + z = 2\lambda, \\
        3(\lambda + 1)x + \lambda y + (\lambda + 3)z = 3\lambda;
    \end{cases}$}

    \xiaoxiaoti{$\begin{cases}
        x + y + z = 0, \\
        ax + by + cz = 0, \\
        bcx + cay + abz = (b -c)(c -a)(a - b) \text{。}
    \end{cases}$}

\end{xiaoxiaotis}


\renewcommand{\labelxiaoti}{*\arabic{cntxiaoti}. }

\xiaoti{解下列方程组:}
\begin{xiaoxiaotis}

    \renewcommand\arraystretch{1.2}
    \begin{tabular}[t]{*{2}{@{}p{16em}}}
        \xiaoxiaoti{$\begin{cases}
            2x + 3y + z = 0, \\
            x + 2y - 3z = 0;
        \end{cases}$}
        & \xiaoxiaoti{$\begin{cases}
            4x - 5y - z = 0, \\
            3x + 7y - 6z = 0 \text{。}
        \end{cases}$}
    \end{tabular}

\end{xiaoxiaotis}


\xiaoti{下列方程组在 $k$ 取什么值时有非零解?并把解集求出来。}
\begin{xiaoxiaotis}

    \renewcommand\arraystretch{1.2}
    \begin{tabular}[t]{@{}p{16em}@{}p{18em}}
        \xiaoxiaoti{$\begin{cases}
            kx + 3y + z = 0, \\
            x + 4y - 3z = 0, \\
            kx + y + 3z = 0;
        \end{cases}$}
        & \xiaoxiaoti{$\begin{cases}
            4x - 2y + kz = 0, \\
            kx - y + z = 0, \\
            6x - 3y + (k + 1)z = 0 \text{。}
        \end{cases}$}
    \end{tabular}

\end{xiaoxiaotis}



\xiaoti{已知方程组
    $$\begin{cases}
        ax + by + cz = 0, \\
        cx + by + az = 0, \\
        bx + ay + cz = 0
    \end{cases}$$
    有非零解,求证 $a = b$,或 $a = c$,或 $a + b + c = 0$。
}


\xiaoti{求证:方程 $a_1 x + b_1 y + c_1 = 0$,$a_2 x + b_2 y + c_2 = 0$,
    $a_3 x + b_3 y + c_3 = 0$ 表示的三直线共点的必要条件是
    $$\begin{vmatrix}
        a_1 & b_1 & c_1 \\
        a_2 & b_2 & c_2 \\
        a_3 & b_3 & c_3
    \end{vmatrix} = 0 \text{。} $$
}


\xiaoti{计算:}
\begin{xiaoxiaotis}

    \twoInLineXxt[16em]{$\begin{vmatrix}
        1 & 2 & 2 & 2 \\
        2 & 2 & 2 & 2 \\
        2 & 2 & 3 & 2 \\
        2 & 2 & 2 & 4
    \end{vmatrix}$;}{$\begin{vmatrix*}[r]
        2 &  1 & 4 & -1 \\
        3 & -1 & 2 & -1 \\
        1 &  2 & 3 & -2 \\
        5 &  0 & 6 & -2
    \end{vmatrix*}$;}

    \xiaoxiaoti{$\begin{vmatrix*}[r]
        a^2 & (a+1)^2 & (a+2)^2 & (a+3)^2 \\
        b^2 & (b+1)^2 & (b+2)^2 & (b+3)^2 \\
        c^2 & (c+1)^2 & (c+2)^2 & (c+3)^2 \\
        d^2 & (d+1)^2 & (d+2)^2 & (d+3)^2
    \end{vmatrix*}$。}

\end{xiaoxiaotis}


\xiaoti{求证:}
\begin{xiaoxiaotis}

    \xiaoxiaoti{$\begin{vmatrix*}[r]
        1 & p & q & r+s \\
        1 & q & r & s+p \\
        1 & r & s & p+q \\
        1 & s & p & q+r
    \end{vmatrix*} = 0$;}

    \xiaoxiaoti{$\begin{vmatrix}
        a & b & b & b \\
        b & a & b & b \\
        b & b & a & b \\
        b & b & b & a
    \end{vmatrix} = (a + 3b)(a - b)^3$。}

\end{xiaoxiaotis}


\xiaoti{利用克莱姆法则解下列方程组:}
\begin{xiaoxiaotis}

    \xiaoxiaoti{$\begin{cases}
        2x + y - 5z + w = 8, \\
        x - 3y - 6w = 9, \\
        2y - z + 2w = -5, \\
        x + 4y - 7z + 6w = 0;
    \end{cases}$}

    \xiaoxiaoti{$\begin{cases}
        2x - y + z - w = 0, \\
        3x + 2y + 3z - w = 0, \\
        x - 4y - z + 2w = 12, \\
        2x + 3y - 2z - 2w = -11 \text{。}
    \end{cases}$}

\end{xiaoxiaotis}


\xiaoti{用顺序消元法(矩阵表示)解下列方程组:}
\begin{xiaoxiaotis}

    \renewcommand\arraystretch{1.2}
    \begin{tabular}[t]{@{}p{16em}@{}p{18em}}
        \xiaoxiaoti{$\begin{cases}
                2x - y + 2z = 4, \\
                x - 2y - z = 1, \\
                4x + y + 4z = 2;
            \end{cases}$}
        & \xiaoxiaoti{$\begin{cases}
            x + 2y - z + 3w = 2, \\
            2x - y + 3z - 2w = 7, \\
            x + 2y -z + w = 4, \\
            x - y + z + 2w = -2 \text{。}
        \end{cases}$}
    \end{tabular}

\end{xiaoxiaotis}

\end{xiaotis}

