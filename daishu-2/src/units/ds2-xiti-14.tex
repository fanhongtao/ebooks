\xiti\mylabel{xiti-14}

\begin{xiaotis}

\xiaoti{把下列复数表示成三角形式,并且画出相应的向量:}
\begin{xiaoxiaotis}

    \renewcommand\arraystretch{1.2}
    \begin{tabular}[t]{*{2}{@{}p{16em}}}
        \xiaoxiaoti{$6$;} & \xiaoxiaoti{$1 + i$;} \\
        \xiaoxiaoti{$1 - \sqrt{3}\,i$;} & \xiaoxiaoti{$5 + 12\,i$;} \\
        \xiaoxiaoti{$-\dfrac{1}{2} - \dfrac{\sqrt{3}}{2}\,i$;} & \xiaoxiaoti{$-18.7 + 8.4\,i$。}
    \end{tabular}

\end{xiaoxiaotis}


\xiaoti{把下列复数表示成代数形式:}
\begin{xiaoxiaotis}

    \xiaoxiaoti{$3\sqrt{2} \left( \cos\dfrac{\pi}{4} + i\,\sin\dfrac{\pi}{4} \right)$;}

    \xiaoxiaoti{$8 \left( \cos\dfrac{11\pi}{6} + i\,\sin\dfrac{11\pi}{6} \right)$;}

    \xiaoxiaoti{$9(\cos\pi + i\,\sin\pi)$;}

    \xiaoxiaoti{$6 \left( \cos\dfrac{4\pi}{3} + i\,\sin\dfrac{4\pi}{3} \right)$;}

    \xiaoxiaoti{$9 \left( \cos\dfrac{7\pi}{6} + i\,\sin\dfrac{7\pi}{6} \right)$;}

    \xiaoxiaoti{$\cos\left( k \cdot \dfrac{\pi}{4} \right) + i\,\sin\left( k \cdot \dfrac{\pi}{4} \right) \quad (k = 0,\,1,\,2,\,3,\,4,\,5,\,6,\,7)$。}

\end{xiaoxiaotis}


\xiaoti{利用公式 $\sin(-\theta) = -\sin\theta$,$\cos(-\theta) = \cos\theta$,把复数 $\cos\theta - i\,\sin\theta$ 表示成三角形式。}


\xiaoti{计算:}
\begin{xiaoxiaotis}

    \xiaoxiaoti{$3 \left( \cos\dfrac{\pi}{3} + i\,\sin\dfrac{\pi}{3} \right) \cdot 3 \left( \cos\dfrac{\pi}{6} + i\,\sin\dfrac{\pi}{6} \right)$;}

    \xiaoxiaoti{$\sqrt{10} \left( \cos\dfrac{\pi}{2} + i\,\sin\dfrac{\pi}{2} \right) \cdot \sqrt{2} \left( \cos\dfrac{\pi}{4} + i\,\sin\dfrac{\pi}{4} \right)$。}

\end{xiaoxiaotis}


\xiaoti{求证:}
\begin{xiaoxiaotis}

    \xiaoxiaoti{$(\cos 75^\circ + i\,\sin 75^\circ) (\cos 15^\circ + i\,\sin 15^\circ) = i$;}

    \xiaoxiaoti{$(\cos3\theta - i\,\sin3\theta) (\cos2\theta - i\,\sin2\theta) = \cos5\theta - i\,\sin5\theta$。\\
        (提示:先把三个复数都表示成三角形式。)
    }

\end{xiaoxiaotis}


\xiaoti{用棣莫佛定理计算:}
\begin{xiaoxiaotis}

    \xiaoxiaoti{$[3(\cos10^\circ + i\,\sin10^\circ)]^6$;}

    \xiaoxiaoti{$[2(\cos15^\circ + i\,\sin15^\circ)]^6$;}

    \xiaoxiaoti{$(1 + \sqrt{3}\,i)^4$;}

    \xiaoxiaoti{$(2 - 2\sqrt{3}\,i)^4$。}

\end{xiaoxiaotis}


\xiaoti{$n\; (n \in N)$ 是什么值的时候,$(1 + \sqrt{3}\,i)^n$ 是一个实数?}


\xiaoti{计算:}
\begin{xiaoxiaotis}

    \xiaoxiaoti{$10 \left( \cos\dfrac{2\pi}{3} + i\,\sin\dfrac{2\pi}{3} \right) \div 5 \left( \cos\dfrac{\pi}{3} + i\,\sin\dfrac{\pi}{3} \right)$;}

    \xiaoxiaoti{$12 \left( \cos\dfrac{3\pi}{2} + i\,\sin\dfrac{3\pi}{2} \right) \div 6 \left( \cos\dfrac{\pi}{6} + i\,\sin\dfrac{\pi}{6} \right)$。}

\end{xiaoxiaotis}


\xiaoti{把与复数 $3 - \sqrt{3}\,i$ 对应的向量按顺时针方向旋转 $60^\circ$,求与所得的向量对应的复数。}


\xiaoti{}
\begin{xiaoxiaotis}
    \xiaoxiaoti[-2cm]{求证
        $\dfrac{1}{\cos\theta + i\,\sin\theta} = \cos\theta - i\,\sin\theta$;}

    \xiaoxiaoti{写出下列复数 $z$ 的倒数 $\dfrac{1}{z}$ 的模与辐角:\\
        $z = 4 \left( \cos\dfrac{\pi}{12} + i\,\sin\dfrac{\pi}{12} \right)$,\\
        $z = \cos\dfrac{\pi}{6} - i\,\sin\dfrac{\pi}{6}$,\\
        $z = \dfrac{\sqrt{2}}{2}(1 - i)$。}

\end{xiaoxiaotis}


\xiaoti{化简}
\begin{xiaoxiaotis}

    \xiaoxiaoti{$\dfrac{(\cos7\theta + i\,\sin7\theta) (\cos2\theta + i\,\sin2\theta)}{(\cos5\theta + i\,\sin5\theta) (\cos3\theta + i\,\sin3\theta)}$;}

    \xiaoxiaoti{$\dfrac{\cos\phi - i\,\sin\phi}{\cos\phi + i\,\sin\phi}$。}

\end{xiaoxiaotis}


\xiaoti{计算:}
\begin{xiaoxiaotis}

    \twoInLineXxt[16em]
        {$\dfrac{(\sqrt{3} + i)^5}{-1 + \sqrt{3}\,i}$;}
        {$\left( \dfrac{2 + 2\,i}{1 - \sqrt{3}\,i} \right)^8$。}

\end{xiaoxiaotis}


\xiaoti{已知 $z = \dfrac{(4 - 3\,i)^2 \cdot (-1 + \sqrt{3}\,i)^{10}}{(1 - i)^{12}}$,求 $|z|$。}


\xiaoti{已知 $n \in N$,并且规定式子 $(\cos\theta + i\,\sin\theta)^{-1}$ 的意义是 $\dfrac{1}{\cos\theta + i\,\sin\theta}$,求证:}
\begin{xiaoxiaotis}

    \xiaoxiaoti{$(\cos\theta + i\,\sin\theta)^{-n} = \cos(-n\theta) + i\,\sin(-n\theta)$;}

    \xiaoxiaoti{$(\cos\theta - i\,\sin\theta)^n = \cos n\theta - i\,\sin n\theta$。}

\end{xiaoxiaotis}


\xiaoti{利用复数证明余弦定理。}


\xiaoti{在复数集 $C$ 中解下列方程:}
\begin{xiaoxiaotis}

    \renewcommand\arraystretch{1.2}
    \begin{tabular}[t]{*{2}{@{}p{16em}}}
        \xiaoxiaoti{$4x^2 + 9 = 0$;} & \xiaoxiaoti{$2(x^2 + 4) = 5x$;} \\
        \xiaoxiaoti{$(x - 3) (x - 5) + 2 = 0$;} & \xiaoxiaoti{$\dfrac{1}{x + 3} - \dfrac{1}{x} = 1$;} \\
        \xiaoxiaoti{$x^4 + 3x^2 - 10 = 0$;} & \xiaoxiaoti{$\dfrac{x}{x^2 + 1} + \dfrac{x^2 + 1}{x} = \dfrac{5}{2}$。}
    \end{tabular}

\end{xiaoxiaotis}


\xiaoti{在复数集 $C$ 中解下列方程组:}
\begin{xiaoxiaotis}

    \twoInLineXxt[16em]
        {$\begin{cases}
            x + y = 2, \\
            xy=2;
        \end{cases}$}
        {$\begin{cases}
            a^2 + b^2 = 0, \\
            ab = 1 \text{。}
        \end{cases}$}

\end{xiaoxiaotis}


\xiaoti{求 $1$ 的 $6$ 个六次方根,并且把它们用复平面内的点表示出来。}

\xiaoti{求:}
\begin{xiaoxiaotis}

    \xiaoxiaoti{$8(\cos 60^\circ + i\,\sin 60^\circ)$ 的六次方根;}

    \xiaoxiaoti{$-i$的五次方根。}

\end{xiaoxiaotis}


\xiaoti{在复数集 $C$ 中解下列方程:}
\begin{xiaoxiaotis}

    \renewcommand\arraystretch{1.2}
    \begin{tabular}[t]{*{2}{@{}p{16em}}}
        \xiaoxiaoti{$y^4 + 81 = 0$;} & \xiaoxiaoti{$x^3 + 1 = i$;} \\
        \xiaoxiaoti{$x^{12} + 63x^6 - 64 = 0$;} & \xiaoxiaoti{$x^{10} - 32x^5 + 1024 = 0$。}
    \end{tabular}

\end{xiaoxiaotis}

\end{xiaotis}

