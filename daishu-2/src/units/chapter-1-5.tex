\subsection{最简单的三角方程}\label{subsec:1-5}

在三角方程中,$\sin x = \alpha$,$\cos x = \alpha$,$\tan x = \alpha$,$\cot x = \alpha$ 是最简单的。
其他的三角方程的求解,往往可以归结为求这种最简单的三角方程的解集。
下面我们先研究这四个最简单的三角方程的解集。

\subsubsection{$\sin x = \alpha$ 的解集}

因为 $|\sin x| \leqslant 1$,所以当 $|\alpha| > 1$ 时,方程 $\sin x = \alpha$ 的解集为 $\kongji$。

当 $|\alpha| = 1$ 时,方程 $\sin x = \alpha$ 成为 $\sin x = 1$ 或 $\sin x = -1$ 。
由于 $y = \sin x$ 的周期为 $2\pi$ ,而在长度为一个周期的区间 $[-\pi, \pi)$ 上,
方程 $\sin x = 1$ 有唯一解 $x = \dfrac{\pi}{2}$,
方程 $\sin x = -1$ 有唯一解 $x = -\dfrac{\pi}{2}$。
因此,在 $(-\infty, +\infty)$ 上,方程 $\sin x = 1$ 的解集是
$$\left\{ x \,\middle|\, x = 2k\pi + \dfrac{\pi}{2}, \; k \in Z \right\} \text{,}$$
方程 $\sin x = -1$ 的解集是
$$\left\{ x \,\middle|\, x = 2k\pi - \dfrac{\pi}{2}, \; k \in Z \right\} \text{。}$$
这就是说,当 $|\alpha| = 1$ 时,方程 $\sin x = \alpha$ 的解集是
$$\{ x \mid x = 2k\pi + \arcsin\alpha, \; k \in Z \} \text{。}$$

当 $|\alpha| < 1$ 时,由\hyperref[subsec:1-1]{反正弦函数}的定义可知,方程 $\sin x = \alpha$
在单调区间 $\left[ -\dfrac{\pi}{2}, \dfrac{\pi}{2} \right)$ 上有唯一解 $x = \arcsin\alpha$,
而在单调区间 $\left[ \dfrac{\pi}{2}, \dfrac{3\pi}{2} \right)$ 上又有唯一解 $x = \pi - \arcsin\alpha$。
因此,在长度为一个周期的区间 $\left[ -\dfrac{\pi}{2}, \dfrac{3\pi}{2} \right)$ 上,
方程 $\sin x = \alpha$ 有两个解:$x = \arcsin\alpha$,$x = \pi - \arcsin\alpha$。
于是,当 $|\alpha| < 1$ 时,在 $(-\infty, +\infty)$ 上,方程 $\sin x = \alpha$ 的解集是
$$\{x \mid x = 2k\pi + \arcsin\alpha, \; k \in Z\}$$
$$\cup \; \{x \mid x = (2k+1)\pi - \arcsin\alpha, \; k \in Z\} \text{。}$$

上面第一个集合中的元素 $x$ 等于 $\pi$ 的偶数倍与 $\arcsin\alpha$ 的和,
第二个集合中的元素 $x$ 等于 $\pi$ 的奇数倍与 $-\arcsin\alpha$ 的和。
因为当 $k$ 为偶数时 $(-1)^k = 1$,当 $k$ 为奇数时,$(-1)^k = -1$,所以上述并集等于
$$\{ x \mid x = k\pi + (-1)^k \arcsin\alpha, \; k \in Z \} \text{。}$$

因此,方程 $\sin x = \alpha$ 在 $(-\infty, +\infty)$ 上的解集如下表所示:

\begin{table}[H]
    \centering
    \renewcommand\arraystretch{1.5}
    \begin{tabular}{|w{c}{10em}|w{c}{20em}|}
        \hline
        $\alpha$ 的取值范围 & 方程 $\sin x = \alpha$ 的解集 \\ \hline
        $|\alpha| > 1$ & $\kongji$ \\ \hline
        $|\alpha| = 1$ & $\{ x \mid x = 2k\pi + \arcsin\alpha, \; k \in Z \}$ \\ \hline
        $|\alpha| < 1$ & $\{ x \mid x = k\pi + (-1)^k \arcsin\alpha, \; k \in Z \}$ \\ \hline
    \end{tabular}
\end{table}

\subsubsection{$\cos x = \alpha$ 的解集}

当 $|\alpha| > 1$ 时,方程 $\cos x = \alpha$ 的解集为 $\kongji$。

当 $|\alpha| = 1$ 时,在长度为一个周期的区间 $[0, 2\pi)$ 上,
方程 $\cos x = 1$ 有唯一解 $x = 0$;
方程 $\cos x = -1$ 有唯一解 $x = \pi$。
因此,在 $(-\infty, +\infty)$ 上,
方程 $\cos x = 1$ 的解集是 $\{ x \mid x = 2k\pi, \; k \in Z \}$,
方程 $\cos x = -1$ 的解集是 $\{ x \mid x = 2k\pi + \pi, \; k \in Z \}$。
就是说,在 $|\alpha| = 1$ 时,方程 $\cos x = \alpha$ 的解集是
$$ \{ x \mid x = 2k\pi + \arccos \alpha, \; k \in Z \} \text{。} $$

当 $|\alpha| < 1$ 时,由\hyperref[subsec:1-2]{反余弦函数}的定义可知,方程 $\cos x = \alpha$
在单调区间 $[0, \pi)$ 上有唯一解 $x = \arccos\alpha$,
在单调区间 $[-\pi, 0)$ 上又有唯一解 $x = -\arccos\alpha$。
因此,在长度为一个周期的区间 $[-\pi, \pi)$ 上,方程 $\cos x = \alpha$ 有两个解:$x = \pm \arccos\alpha$。
于是,在 $(-\infty, +\infty)$ 上,方程 $\cos x = \alpha \; (|\alpha| < 1)$ 的解集是
$$\{x \mid x = 2k\pi \pm \arccos\alpha, \; k \in Z\} \text{。}$$

方程 $\cos x = \alpha$ 在 $(-\infty, +\infty)$ 上的解集如下表所示:

\begin{table}[H]
    \centering
    \renewcommand\arraystretch{1.5}
    \begin{tabular}{|w{c}{10em}|w{c}{20em}|}
        \hline
        $\alpha$ 的取值范围 & 方程 $\cos x = \alpha$ 的解集 \\ \hline
        $|\alpha| > 1$ & $\kongji$ \\ \hline
        $|\alpha| = 1$ & $\{ x \mid x = 2k\pi + \arccos\alpha, \; k \in Z \}$ \\ \hline
        $|\alpha| < 1$ & $\{ x \mid x = 2k\pi \pm \arccos\alpha, \; k \in Z \}$ \\ \hline
    \end{tabular}
\end{table}


\subsubsection{$\tan x = \alpha$ 的解集}

由\hyperref[subsec:1-3]{反正切函数}的定义可知,在单调区间 $\left( -\dfrac{\pi}{2}, \dfrac{\pi}{2} \right)$ 上,
不论 $\alpha$ 为什么实数,方程 $\tan x = \alpha$ 都有唯一解 $x = \arctan \alpha$。
因为 $y = \tan x$ 的周期是 $\pi$,所以方程 $\tan x = \alpha$ 的解集如下表所示:

\begin{table}[H]
    \centering
    \renewcommand\arraystretch{1.5}
    \begin{tabular}{|w{c}{20em}|}
        \hline
        方程 $\tan x = \alpha$ 的解集 \\ \hline
        $\{ x \mid x = k\pi + \arctan\alpha, \; k \in Z \}$ \\ \hline
    \end{tabular}
\end{table}


\subsubsection{$\cot x = \alpha$ 的解集}

由\hyperref[subsec:1-3]{反余切函数}的定义可知,在单调区间 $(0, \pi)$ 上,
不论 $\alpha$ 为什么实数,方程 $\cot x = \alpha$ 有唯一解 $x = \arccot \alpha$。
因为 $y = \cot x$ 的周期是 $\pi$,所以方程 $\cot x = \alpha$ 的解集如下表所示:

\begin{table}[H]
    \centering
    \renewcommand\arraystretch{1.5}
    \begin{tabular}{|w{c}{20em}|}
        \hline
        方程 $\cot x = \alpha$ 的解集 \\ \hline
        $\{ x \mid x = k\pi + \arccot\alpha, \; k \in Z \}$ \\ \hline
    \end{tabular}
\end{table}

\liti 解方程 $2\sin x + \sqrt{2} = 0$。

\jie 原方程可化为
$$ \sin x = -\dfrac{\sqrt{2}}{2} \text{。}$$
$\therefore$ 解集是 $\left\{\, x \,\middle|\, x = k\pi + (-1)^k \arcsin\left(-\dfrac{\sqrt{2}}{2}\right), \; k \in Z \,\right\}$,
即
$$ \left\{ x \,\middle|\, x = k\pi + (-1)^k \cdot \left( -\dfrac{\pi}{4} \right), \; k \in Z \right\} \text{。}$$


\liti 解方程 $2\cos2x = 1$。

\jie 原方程可化为
$$\cos2x = \dfrac{1}{2},$$
$$2x = 2k\pi \pm \arccos \dfrac{1}{2} \quad (k \in Z),$$
即
$$2x = 2k\pi \pm \dfrac{\pi}{3} \quad (k \in Z) \text{。}$$
$\therefore$ 解集是 $\left\{\, x \,\middle|\, x = k\pi \pm \dfrac{\pi}{6}, \; k \in Z \,\right\}$。


\liti 解方程 $\tan(x + 15^\circ) + 1 = 0$。

\jie 原方程可化为
$$\tan(x + 15^\circ) = -1,$$
$$x + 15^\circ = k \cdot 180^\circ + (-45^\circ) \quad (k \in Z) \text{。}$$
$\therefore$ 解集是 $\{ x \mid x = k \cdot 180^\circ - 60^\circ, \; k \in Z \}$。


\liti 求适合方程 $\sin(3x - 105^\circ) = \dfrac{1}{2}$ 且小于 $360^\circ$ 的正角。

\jie 由方程 $\sin(3x - 105^\circ) = \dfrac{1}{2}$,可得
$$3x - 105^\circ = k \cdot 180^\circ + (-1)^k \cdot 30^\circ \quad (k \in Z) \text{。}$$
$\therefore$ 解集是 $\{ x \mid x = k \cdot 60^\circ + (-1)^k \cdot 10^\circ + 35^\circ, \; k \in Z \}$。

分别设 $k = 0, 1, 2, 3, 4, 5$,得适合方程且小于 $360^\circ$ 的正角是
$45^\circ$,$85^\circ$,$165^\circ$,$205^\circ$,$285^\circ$,$325^\circ$。


\lianxi
\begin{xiaotis}

\xiaoti{写出下列方程的解集:}

\begin{xiaoxiaotis}

    \renewcommand\arraystretch{1.7}
    \begin{tabular}[t]{*{2}{@{}p{18em}}}
        \xiaoxiaoti{$\sin x = \dfrac{\sqrt{3}}{2}$;} & \xiaoxiaoti{$\sin x = -\dfrac{1}{2}$;} \\
        \xiaoxiaoti{$\cos x = \dfrac{\sqrt{2}}{2}$;} & \xiaoxiaoti{$\cos x = -0.8475$;} \\
        \xiaoxiaoti{$\tan x = -\sqrt{3}$;} & \xiaoxiaoti{$\cot x = \dfrac{5}{3}$。}
    \end{tabular}

\end{xiaoxiaotis}

\xiaoti{解下列方程:}

\begin{xiaoxiaotis}

    \xiaoxiaoti{$2 \sin \dfrac{2x}{3} = 1$;}

    \xiaoxiaoti{$2 \cos(3x - 15^\circ) + 1 = 0$;}

    \xiaoxiaoti{$3 \tan \dfrac{x + 20^\circ}{3} = \sqrt{3}$;}

    \xiaoxiaoti{$\cot \left( \dfrac{x}{4} + 30^\circ \right) + 1 = 0$。}

\end{xiaoxiaotis}


\xiaoti{求 $0^\circ$ 到 $360^\circ$ 的角 $x$,已知:}

\begin{xiaoxiaotis}

    \renewcommand\arraystretch{1.5}
    \begin{tabular}[t]{*{2}{@{}p{18em}}}
        \xiaoxiaoti{$\sin 2x = -\dfrac{1}{2}$;} & \xiaoxiaoti{$\cos (3x + 20^\circ) = 0.95$。}
    \end{tabular}

\end{xiaoxiaotis}

\end{xiaotis}
