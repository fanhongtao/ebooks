\xiti\mylabel{xiti-9}

\begin{xiaotis}

\xiaoti{用对角线法则计算:}
\begin{xiaoxiaotis}

    \renewcommand\arraystretch{1.2}
    \begin{tabular}[t]{*{2}{@{}p{16em}}}
        \xiaoxiaoti{$\begin{vmatrix*}[r]
                3  \quad & -5 \quad & 1 \\
                2  \quad & 3  \quad & -6 \\
                -7 \quad & 2  \quad & 4
            \end{vmatrix*}$;}
        & \xiaoxiaoti{$\begin{vmatrix*}[r]
                a \quad & b \quad & c \\
                0 \quad & d \quad & e \\
                0 \quad & 0 \quad & f
            \end{vmatrix*}$;} \\
        \xiaoxiaoti{$\begin{vmatrix*}
                0          \quad & -\cos\alpha \quad & -\cos\beta \\
                \cos\alpha \quad & 0           \quad & -\cos\gamma \\
                \cos\beta  \quad & \cos\gamma  \quad & 0
            \end{vmatrix*}$;}
        & \xiaoxiaoti{$\begin{vmatrix*}[r]
                a \quad & h \quad & g \\
                h \quad & b \quad & f \\
                g \quad & f \quad & c
            \end{vmatrix*}$。}
    \end{tabular}

\end{xiaoxiaotis}


\xiaoti{解方程}
\begin{xiaoxiaotis}

    \xiaoxiaoti{$\begin{vmatrix*}
            0   \quad & x-1 \quad & 1 \\
            x-1 \quad & 0   \quad & x-2 \\
            1   \quad & x-2 \quad & 0
        \end{vmatrix*} = 0$;}

    \xiaoxiaoti{$\begin{vmatrix*}
            x-1 \quad & 1   \quad & 1 \\
            1   \quad & x-1 \quad & 1 \\
            1   \quad & 1   \quad & x-1
        \end{vmatrix*} = 0$。}

\end{xiaoxiaotis}


\xiaoti{求证:}
\begin{xiaoxiaotis}

    \xiaoxiaoti{$\begin{vmatrix*}
            1 \quad & \sin3\theta \quad & \cos3\theta \\
            1 \quad & \sin2\theta \quad & \cos2\theta \\
            1 \quad & \sin\theta  \quad & \cos\theta
        \end{vmatrix*} = 2\sin\theta(1 - \cos\theta)$;}

    \xiaoxiaoti{$\begin{vmatrix*}
            2\cos\theta \quad & 1           \quad & 0 \\
            1           \quad & 2\cos\theta \quad & 1 \\
            0           \quad & 1           \quad & 2\cos\theta
        \end{vmatrix*} = \dfrac{\sin4\theta}{\sin\theta} \quad (\theta \neq k\pi,\; k \in Z)$。}

\end{xiaoxiaotis}


\xiaoti{利用行列式的性质计算:}
\begin{xiaoxiaotis}

    \renewcommand\arraystretch{1.2}
    \begin{tabular}[t]{*{2}{@{}p{16em}}}
        \xiaoxiaoti{$\begin{vmatrix*}[r]
                10 \quad & 8  \quad & -2 \\
                15 \quad & 12 \quad & -3 \\
                25 \quad & 32 \quad & 7
            \end{vmatrix*}$;}
        & \xiaoxiaoti{$\begin{vmatrix*}[r]
                \dfrac{1}{2} \quad & \dfrac{1}{3} \quad & \dfrac{1}{4} \\
                12 \quad & 24  \quad & 36 \\
                -5 \quad & -4 \quad & -3
            \end{vmatrix*}$;} \\
        \xiaoxiaoti{$\begin{vmatrix*}[r]
                554 \quad & 427 \quad & 327 \\
                586 \quad & 443 \quad & 343 \\
                711 \quad & 504 \quad & 404
            \end{vmatrix*}$。}
    \end{tabular}

\end{xiaoxiaotis}


\xiaoti{利用行列式的性质计算:}
\begin{xiaoxiaotis}

    \xiaoxiaoti{$\begin{vmatrix*}[r]
            -ab \quad & bd  \quad & bf \\
            ac  \quad & -cd \quad & cf \\
            ae  \quad & de  \quad & -ef
        \end{vmatrix*}$;}

    \xiaoxiaoti{$\begin{vmatrix*}
            a  \quad & b     \quad & c \\
            2a \quad & 3a+2b \quad & 4a+3b+2c \\
            3a \quad & 6a+3b \quad & 10a+9b+3c
        \end{vmatrix*}$。}

\end{xiaoxiaotis}


\xiaoti{下列计算过程中,哪些步骤是对的,哪些不对,应怎样改正?}
\begin{xiaoxiaotis}

    \xiaoxiaoti{$\begin{vmatrix*}
            a_1 \quad & b_1 \\
            a_2 \quad & b_2
        \end{vmatrix*} = \begin{vmatrix*}
            a_1 + ka_2 \quad & b_1 + kb_2 \\
            a_2 - ha_1 \quad & b_2 - hb_1
        \end{vmatrix*}$;}

    \xiaoxiaoti{$\begin{vmatrix*}
            a_1 \quad & b_1 \quad & c_1 \\
            a_2 \quad & b_2 \quad & c_2 \\
            a_3 \quad & b_3 \quad & c_3
        \end{vmatrix*} = \begin{vmatrix*}
            a_1 \quad & b_1 \quad & ka_1 + hc_1 \\
            a_2 \quad & b_2 \quad & ka_2 + hc_2 \\
            a_3 \quad & b_3 \quad & ka_3 + hc_3
        \end{vmatrix*}$。}

\end{xiaoxiaotis}


\xiaoti{不展开行列式,求证:}
\begin{xiaoxiaotis}

    \xiaoxiaoti{$\begin{vmatrix*}
            a    \quad & a+3d \quad & a+6d \\
            a+d  \quad & a+4d \quad & a+7d \\
            a+2d \quad & a+5d \quad & a+8d
        \end{vmatrix*} = 0$;}

    \xiaoxiaoti{$\begin{vmatrix*}
            a_1 \quad & b_1 \quad & c_1 \\
            a_2 \quad & b_2 \quad & c_2 \\
            a_3 \quad & b_3 \quad & c_3
        \end{vmatrix*} = \begin{vmatrix*}
            c_3 \quad & b_3 \quad & a_3 \\
            c_2 \quad & b_2 \quad & a_2 \\
            c_1 \quad & b_1 \quad & a_1
        \end{vmatrix*}$;}

    \xiaoxiaoti{$\begin{vmatrix*}[r]
            0  \quad & am \quad & -abn \\
            -e \quad & 0  \quad & bn \\
            e  \quad & -m \quad & 0
        \end{vmatrix*} = 0$;}

    \xiaoxiaoti{$\begin{vmatrix*}
            a_1 \quad & b_1 \quad & a_1x+b_1y+c_1 \\
            a_2 \quad & b_2 \quad & a_2x+b_2y+c_2 \\
            a_3 \quad & b_3 \quad & a_3x+b_3y+c_3
        \end{vmatrix*} = \begin{vmatrix*}
            a_1 \quad & b_1 \quad & c_1 \\
            a_2 \quad & b_2 \quad & c_2 \\
            a_3 \quad & b_3 \quad & c_3
        \end{vmatrix*}$;}

    \xiaoxiaoti{$\begin{vmatrix*}
            0       \quad & (a-b)^3 \quad & (a-c)^3 \\
            (b-a)^3 \quad & 0       \quad & (b-c)^3 \\
            (c-a)^3 \quad & (c-b)^3 \quad & 0
        \end{vmatrix*} = 0$。}

\end{xiaoxiaotis}


\xiaoti{利用行列式的性质和第 \ref{subsec:4-4} 节中的 \nameref{theorem:sjhlszk-1},计算:}
\begin{xiaoxiaotis}

    \twoInLineXxt[16em]{
        $\begin{vmatrix*}[r]
            6  \quad & -4 \quad & 2 \\
            -3 \quad & 3  \quad & -1 \\
            18 \quad & 7  \quad & 5
        \end{vmatrix*}$;
    }{
        $\begin{vmatrix*}[r]
            8  \quad & 3  \quad & -7 \\
            5  \quad & 0  \quad & -4 \\
            -9 \quad & -2 \quad & 3
        \end{vmatrix*}$。
    }

\end{xiaoxiaotis}


\xiaoti{解关于 $x$ 的方程:}
\begin{xiaoxiaotis}

    \xiaoxiaoti{$\begin{vmatrix*}
            x^2 \quad & x \quad & 1 \\
            a^2 \quad & a \quad & 1 \\
            b^2 \quad & b \quad & 1
        \end{vmatrix*} = 0 \quad (a \neq b)$;}

    \xiaoxiaoti{$\begin{vmatrix*}
            x   \quad & a   \quad & b+c \\
            x   \quad & a+b \quad & c \\
            a+b \quad & b-c \quad & a+c
        \end{vmatrix*} = 0 \quad (b(a+b) \neq 0)$。}

\end{xiaoxiaotis}

\xiaoti{求证:}
\begin{xiaoxiaotis}

    \xiaoxiaoti{$\begin{vmatrix*}
            a \quad & a^2 \quad & 1 \\
            b \quad & b^2 \quad & 1 \\
            c \quad & c^2 \quad & 1
        \end{vmatrix*} = (a - b)(b - c)(c - a)$;}

    \xiaoxiaoti{$\begin{vmatrix*}
            a \quad & a^2 \quad & bc \\
            b \quad & b^2 \quad & ac \\
            c \quad & c^2 \quad & ab
        \end{vmatrix*} = (a - b)(b - c)(c - a)(ab + bc + ca)$;}

    \xiaoxiaoti{$\begin{vmatrix*}
            ax \quad & a^2+x^2 \quad & 1 \\
            ay \quad & a^2+y^2 \quad & 1 \\
            az \quad & a^2+z^2 \quad & 1
        \end{vmatrix*} = a(x - y)(y - z)(z - x)$;}

    \xiaoxiaoti{$\begin{vmatrix*}[r]
            \cos\theta \quad & \cos3\theta \quad & \sin3\theta \\
            \cos\theta \quad & \cos\theta  \quad & \sin\theta \\
            \sin\theta \quad & \sin\theta  \quad & \cos\theta
        \end{vmatrix*} = \sin\theta\sin4\theta$。}

\end{xiaoxiaotis}


\xiaoti{已知直线方程为
    $$\begin{vmatrix*}[r]
        x  \quad & y \quad & 1 \\
        3  \quad & 5 \quad & 1 \\
        -2 \quad & 3 \quad & 1
    \end{vmatrix*} = 0 \text{,}$$
    问点 $P_1 \left( \dfrac{1}{2}, 4 \right)$ 与 $P_2 \left( 4, 7 \right)$ 是否在这条直线上。
}


\xiaoti{利用克莱姆法则解下列关于 $x$,$y$,$z$ 的方程组:}
\begin{xiaoxiaotis}

    \renewcommand\arraystretch{1.2}
    % TODO: 这里使用 longtable 进行表格分页,
    % 但生成的表格与 tabular 生成的表格相比,略有一点缩进。
    \begin{longtable}[t]{*{2}{@{}p{18em}}}
        \xiaoxiaoti{$\begin{cases}
                4x - y - 2z = 4, \\
                2x + y - 4z = 8, \\
                x + 2y + z = 1;
            \end{cases}$}
        & \xiaoxiaoti{$\begin{cases}
                5x - 8y + 3z = 0, \\
                15x + 12y - 15z = 11, \\
                10x - 4y - 6z = 1;
            \end{cases}$} \\
        \xiaoxiaoti{$\begin{cases}
                x - y + z = a, \\
                x + y - z = b, \\
                -x + y + z = c;
            \end{cases}$}
        & \xiaoxiaoti{$\begin{cases}
                bx - ay = -2ab, \\
                -2cy + 3bz = bc, \\
                cx + az = 0 \quad (abc \neq 0)
            \end{cases}$}
    \end{longtable}

\end{xiaoxiaotis}


\xiaoti{求下列关于 $x$,$y$,$z$ 的方程组有唯一解的条件,并把第 (3) 题中的方程组在这个条件下的解求出来:}
\begin{xiaoxiaotis}

    \renewcommand\arraystretch{1.2}
    \begin{tabular}[t]{*{2}{@{}p{18em}}}
        \xiaoxiaoti{$\begin{cases}
                \lambda x + y + z = 1, \\
                x + \lambda y + z = \lambda, \\
                x + y + \lambda z = \lambda^2;
            \end{cases}$}
        & \xiaoxiaoti{$\begin{cases}
                ay + bz = c, \\
                cx + az = b, \\
                bx + cy = a;
            \end{cases}$} \\
        \xiaoxiaoti{$\begin{cases}
                ax + y + z = a - 3, \\
                x + ay + z = 2, \\
                x + y + az = -2 \text{。}
            \end{cases}$}
    \end{tabular}

\end{xiaoxiaotis}

\end{xiaotis}

