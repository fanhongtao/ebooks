\subsection{含有绝对值的不等式}\label{subsec:3-5}

我们知道,在实数值 $R$ 中:

$$|a| = \begin{cases}
    a & (\text{当} a > 0 \text{时}), \\
    0 & (\text{当} a = 0 \text{时}), \\
    -a & (\text{当} a < 0 \text{时}) \text{。}
\end{cases}$$

根据实数的绝对值的定义,我们有
\begin{align*}
    & |ab| = |a| \cdot |b|, \\
    & \left| \dfrac{a}{b} \right| = \dfrac{|a|}{|b|} \quad (b \neq 0) \text{。}
\end{align*}

如果 $a$ 是一个正数,那么
\begin{gather*}
    |x| < a \iff x^2 < a^2 \iff -a < x < a, \\
    |x| > a \iff x^2 > a^2 \iff x > a \text{,或} \; x < -a \text{。}
\end{gather*}

即在 $a > 0$ 时,
\begin{gather*}
    |x| < a \iff -a < x < a, \\
    |x| > a \iff x > a \text{,或} \; x < -a \text{。}
\end{gather*}

这个结果从图 \ref{fig:3-4} 也可看出。

\begin{figure}[H]
    \centering
    \begin{minipage}{8cm}
        \centering
        \begin{tikzpicture}[>=Stealth,scale=0.8]
    \pgfmathsetmacro{\a}{1}
    \draw [->] (-4,0) -- (4,0);
    \foreach \x/\txt in {-\a/-a, 0/0, \a/a} {
        \draw (\x,0.2) -- (\x,0) node[anchor=north] {$\txt$};
    }

   \pic [transform shape] {interval={start=-\a, stop=\a}};

    \foreach \x in {-\a, \a} {
        \draw [fill=white] (\x, 0) circle(0.1);
    }
\end{tikzpicture}

        \caption*{$|x| < a$ \\ (1)}
    \end{minipage}
    \quad
    \begin{minipage}{8cm}
        \centering
        \begin{tikzpicture}[>=Stealth,scale=0.8]
    \pgfmathsetmacro{\a}{1}
    \pgfmathsetmacro{\b}{3}
    \draw [->] (-4,0) -- (4,0);
    \foreach \x/\txt in {-\a/-a, 0/0, \a/a} {
        \draw (\x,0.2) -- (\x,0) node[anchor=north] {$\txt$};
    }

    \pic [transform shape] {infinity interval={start=-\a, stop=-\b}};
    \pic [transform shape] {infinity interval={start=\a, stop=\b}};

    \foreach \x in {-\a, \a} {
        \draw [fill=white] (\x, 0) circle(0.1);
    }
\end{tikzpicture}

        \caption*{$|x| > a$ \\ (2)}
    \end{minipage}
    \caption{}\label{fig:3-4}
\end{figure}

关于和差的绝对值与绝对值的和差,还有下面的性质:

\begin{theorem} \label{theorem:bds-jdz-1}
    $|a| - |b| \leqslant |a + b| \leqslant |a| + |b|$。
\end{theorem}

\zhengming \; $\because \begin{aligned}[t]
    & -|a| \leqslant a \leqslant |a|, \\
    & -|b| \leqslant b \leqslant |b|, \\
\end{aligned}$

$\therefore \quad -(|a| + |b|) \leqslant a + b \leqslant |a| + |b|,$\\
即
\begin{equation*}
    |a + b| \leqslant |a| + |b| \text{。} \tag{1}
\end{equation*}

又
$$ a = a + b - b, \quad |-b| = |b|, $$

由 (1) 得
$$ |a| = |a + b - b| \leqslant |a + b| + |-b|, $$
即
\begin{equation*}
    |a| - |b| \leqslant |a + b| \text{。} \tag{2}
\end{equation*}

由 (1),(2) 得
$$ |a| - |b| \leqslant |a + b| \leqslant |a| + |b| \text{。} $$

\begin{corollary} \label{corollary:bds-jdz-1-1}
    $|a_1 + a_2 + \cdots + a_n| \leqslant |a_1| + |a_2| + \cdots + |a_n|$
\end{corollary}


\begin{theorem} \label{theorem:bds-jdz-2}
    $|a| - |b| \leqslant |a - b| \leqslant |a| + |b|$。
\end{theorem}

\zhengming 由定理 \ref{theorem:bds-jdz-1} 可知
$$ |a| - |-b| \leqslant |a + (-b)| \leqslant |a| + |-b|,$$
即
$$ |a| - |b| \leqslant |a - b| \leqslant |a| + |b| \text{。} $$


\liti 解不等式 $|2x - 3| < 5$ 。

\jie 这个不等式等价于
$$ -5 < 2x - 3 < 5 \text{。} $$

解这个不等式,得解集
$$ \{ x \mid -1 < x < 4 \} \text{。} $$


\liti 解不等式 $|x^2 - 5x| > 6$。

\jie 这个不等式等价于
\begin{equation*}
    x^2 - 5x > 6, \tag{1}
\end{equation*}
或
\begin{equation*}
    x^2 - 5x < -6 \text{。} \tag{2}
\end{equation*}

解不等式 (1) ,得 $x < -1$,或 $x > 6$;

解不等式 (2) ,得 $2 < x < 3$。

因此,原不等式的解集是
\begin{gather*}
    \{ x \mid x < -1 \} \cup \{ x \mid 2 < x < 3 \} \cup \{ x \mid x > 6 \} \\
    = \{ x \mid x < -1 \text{,或}\; 2 < x < 3 \text{,或}\; x > 6 \} \text{。}
\end{gather*}


\liti 已知 $|x| < \dfrac{\varepsilon}{3}$,$|y| < \dfrac{\varepsilon}{6}$,$|z| < \dfrac{\varepsilon}{9}$,求证
$$ |x + 2y - 3z| < \varepsilon \text{。} $$

\zhengming \; $\because \begin{aligned}[t]
        & |x + 2y - 3z| \leqslant |x| + |2y| + |-3z| \\
    ={} & |x| + |2| \cdot |y| + |-3| \cdot |z| = |x| + 2|y| + 3|z| \text{。}
\end{aligned}$

$\because$ \quad $|x| < \dfrac{\varepsilon}{3}$,$|y| < \dfrac{\varepsilon}{6}$,$|z| < \dfrac{\varepsilon}{9}$,

$\therefore \quad |x| + 2|y| + 3|z| < \dfrac{\varepsilon}{3} + \dfrac{2\varepsilon}{6} + \dfrac{3\varepsilon}{9} = \varepsilon$。

$\therefore \quad |x + 2y - 3z| < \varepsilon$。


\liti 已知 $|a| < 1$,$|b| < 1$,求证
$$ \left| \dfrac{a + b}{1 + ab} \right| < 1 \text{。} $$

\zhengming \; $\because \begin{aligned}[t]
    \left| \dfrac{a + b}{1 + ab} \right| < 1 & \iff \dfrac{(a + b)^2}{(1 + ab)^2} < 1 \\
    & \iff a^2 + 2ab + b^2 < 1 + 2ab + a^2b^2 \\
    & \iff 1 - a^2 - b^2 + a^2b^2 > 0 \\
    & \iff (1 - a^2)(1 - b^2) > 0 \text{。}
\end{aligned}$

因为 $|a| < 1$,$|b| < 1$,$(1 - a^2)(1 - b^2) > 0$ 成立,所以
$$ \left| \dfrac{a + b}{1 + ab} \right| < 1 \text{。} $$


\lianxi
\begin{xiaotis}

\xiaoti{(口答)下列各式是不是恒等式,为什么?}
\begin{xiaoxiaotis}

    \renewcommand\arraystretch{1.5}
    \begin{tabular}[t]{*{2}{@{}p{16em}}}
        \xiaoxiaoti{$|-a| = a$;} & \xiaoxiaoti{$\sqrt{(-a)^2} = a$;} \\
        \xiaoxiaoti{$|b - a| = |a - b|$;} & \xiaoxiaoti{$\left| \dfrac{1}{a^2} \right| = \dfrac{1}{a^2}$。}
    \end{tabular}

\end{xiaoxiaotis}


\xiaoti{(口答)用不带绝对值符号的式子表示下列各式:}
\begin{xiaoxiaotis}

    \renewcommand\arraystretch{1.5}
    \begin{tabular}[t]{*{2}{@{}p{16em}}}
        \xiaoxiaoti{$|(-a)^2|$;} & \xiaoxiaoti{$|a^2 - 1| \quad (0 < a < 1)$;} \\
        \xiaoxiaoti{$\dfrac{|ab^3 + a^3b|}{a^2 + b^2} \quad (ab < 0)$。}
    \end{tabular}

\end{xiaoxiaotis}


\xiaoti{设 $\varepsilon > 0$,解不等式 $|x - A| < \varepsilon$,并且在数轴上表示出它的解集。}


\xiaoti{已知 $|A - a| < \dfrac{\varepsilon}{2}$,$|B - b| < \dfrac{\varepsilon}{2}$,求证:}
\begin{xiaoxiaotis}

    \xiaoxiaoti{$|(A + B) - (a + b)| < \varepsilon$;}

    \xiaoxiaoti{$|(A - B) - (a - b)| < \varepsilon$。}

\end{xiaoxiaotis}


\xiaoti{解下列不等式:}
\begin{xiaoxiaotis}

    %\renewcommand\arraystretch{1.5}
    \begin{tabular}[t]{*{2}{@{}p{16em}}}
        \xiaoxiaoti{$|x - 2| < 5$;} & \xiaoxiaoti{$|2x - 3| \leqslant 1$;} \\
        \xiaoxiaoti{$|x^2 -3x - 1| > 3$。}
    \end{tabular}

\end{xiaoxiaotis}

\end{xiaotis}

