\subsection{按一行(或一列)展开三阶行列式}\label{subsec:4-4}

在展开三阶行列式时,如果分别把含 $a_1$,$a_2$,$a_3$ 的项结合在一起,并提出公因子,就得

\begin{flalign*}
    \begin{vmatrix*}
        a_1 & b_1 & c_1 \\
        a_2 & b_2 & c_2 \\
        a_3 & b_3 & c_3
    \end{vmatrix*} & = a_1b_2c_3 + a_2b_3c_1 + a_3b_1c_2 - a_3b_2c_1 - a_2b_1c_3 - a_1b_3c_2 \\
    &= a_1(b_2c_3 - b_3c_2) + a_2(b_3c_1 - b_1c_3) + a_3(b_1c_2 - b_2c_1) \\
    &= a_1 \begin{vmatrix}
                b_2 & c_2 \\
                b_3 & c_3
            \end{vmatrix}
     - a_2 \begin{vmatrix}
                b_1 & c_1 \\
                b_3 & c_3
            \end{vmatrix}
     + a_3 \begin{vmatrix}
                b_1 & c_1 \\
                b_2 & c_2
            \end{vmatrix} \tag{1}\label{eq:sjhlszk-1}
\end{flalign*}

我们看到,\eqref{eq:sjhlszk-1} 式中的
$$
\begin{vmatrix}
    b_2 & c_2 \\
    b_3 & c_3
\end{vmatrix}
$$
就是在原三阶行列式中,划去 $a_1$ 所在的行和列,剩下的元素
按原行列顺序排列所组成的行列式。 把行列式中某一元素所
在的行与列划去后,剩下的元素按原行列顺序排列所组成的
行列式,叫做原行列式中对应于这个元素的\textbf{余子式}。

例如在行列式
$$
D = \begin{vmatrix*}
    a_1 & b_1 & c_1 \\
    a_2 & b_2 & c_2 \\
    a_3 & b_3 & c_3
\end{vmatrix*}
$$
中,对应于元素 $a_2$ 的余子式为
$$
\begin{vmatrix}
    b_1 & c_1 \\
    b_3 & c_3
\end{vmatrix} \text{。}
$$

设行列式中某一元素位于第 $i$ 行第 $j$ 列,把对应于这个元素的余子式乘上
 $(-1)^{i+j}$ 后所得到的式子叫做原行列式中对应于这个元素的\textbf{代数余子式}。

例如,在上面的行列式 $D$ 中,元素 $a_2$ 位于第二行第一列,
$i + j = 2 + 1 = 3$, 所以对应于 $a_2$ 的代数余子式为
$$
(-1)^{2+1}\begin{vmatrix}
    b_1 & c_1 \\
    b_3 & c_3
\end{vmatrix} \text{,}
$$
即
$$
-\begin{vmatrix}
    b_1 & c_1 \\
    b_3 & c_3
\end{vmatrix} \text{。}
$$
三阶行列式各元素的代数余子式的符号 $(-1)^{i+j}$ 可以用下图来帮助记忆:
$$
\begin{vmatrix}
	+ & - & + \\
	- & + & - \\
	+ & - & +
\end{vmatrix} \text{。}
$$

行列式
$$
D = \begin{vmatrix*}
    a_1 & b_1 & c_1 \\
    a_2 & b_2 & c_2 \\
    a_3 & b_3 & c_3
\end{vmatrix*}
$$
中某个元素的代数余子式常用这个元素相应的大写字母并附
加相同的下标来表示,例如元素 $a_1$,$b_1$,$c_1$ 的代数余子式分别
是 $A_1$,$B_1$,$C_1$,其中
\begin{align*}
    A_1 &= (-1)^{1+1}
            \begin{vmatrix}
                b_2 & c_2 \\
                b_3 & c_3
            \end{vmatrix}
        =   \begin{vmatrix}
                b_2 & c_2 \\
                b_3 & c_3
            \end{vmatrix} \text{,} \\
    B_1 &= (-1)^{1+2}
            \begin{vmatrix}
                a_2 & c_2 \\
                a_3 & c_3
            \end{vmatrix}
        = - \begin{vmatrix}
                a_2 & c_2 \\
                a_3 & c_3
            \end{vmatrix} \text{,} \\
    C_1 &= (-1)^{1+3}
            \begin{vmatrix}
                a_2 & b_2 \\
                a_3 & b_3
            \end{vmatrix}
        =   \begin{vmatrix}
                a_2 & b_2 \\
                a_3 & b_3
            \end{vmatrix} \text{。} \\
\end{align*}

这样,上面所得的 \eqref{eq:sjhlszk-1} 式就可写成
\begin{gather*}
    \begin{vmatrix*}
        a_1 & b_1 & c_1 \\
        a_2 & b_2 & c_2 \\
        a_3 & b_3 & c_3
    \end{vmatrix*} = a_1A_1 + a_2A_2 + a_3A_3 \text{,} \tag{2}\label{eq:sjhlszk-2}
\end{gather*}
它把一个三阶行列式表示成这个行列式第一列的元素与对应于它们的代数余子式的乘积的和。

一般地,有如下定理:

\begin{theorem}\label{theorem:sjhlszk-1}
    行列式等于它的任意一行(或一列)的所有元素与它们各自对应的代数余子式的乘积的和。
\end{theorem}

也就是说,我们可以按任一行(或一列)展开三阶行列式 $D$:
\begin{align*}
    D &= a_1A_1 + b_1B_1 + c_1C_1, && D = a_1A_1 + a_2A_2 + a_3A_3, \\
    D &= a_2A_2 + b_2B_2 + c_2C_2, && D = b_1B_1 + b_2B_2 + b_3B_3, \\
    D &= a_3A_3 + b_3B_3 + c_3C_3, && D = c_1C_1 + c_2C_2 + c_3C_3 \text{。}
\end{align*}
等式 $D = a_1A_1 + a_2A_2 + a_3A_3$ 前面已经证明,其他五个等式也可类似证明。


\begin{theorem}\label{theorem:sjhlszk-2}
    行列式某一行(或一列)的各元素与另一行(或一列)对应元素的代数余子式的乘积的和等于零。
\end{theorem}

\zhengming 我们来证明行列式的第二行的各元素与第一行对应元素的代数余子式的乘积的和等于零,即
$$ a_2A_1 + b_2B_1 + c_2C_1 = 0 \text{。} $$

$\because \quad
\begin{aligned}[t]
  & a_2 \begin{vmatrix}
            b_2 & c_2 \\
            b_3 & c_3
        \end{vmatrix}
    - b_2 \begin{vmatrix}
            a_2 & c_2 \\
            a_3 & c_3
        \end{vmatrix}
    + c_2 \begin{vmatrix}
            a_2 & b_2 \\
            a_3 & b_3
        \end{vmatrix} \\
 &= \begin{vmatrix*}
        a_2 & b_2 & c_2 \\
        a_2 & b_2 & c_2 \\
        a_3 & b_3 & c_3
    \end{vmatrix*}
    = 0 \text{,}
\end{aligned}$

$\therefore \quad a_2A_1 + b_2B_1 + c_2C_1 = 0 \text{。}$

其他情况可类似证明。

\liti 把行列式
$$
D =
\begin{vmatrix*}[r]
    3 & 1 & -2 \\
    5 & -2 & 7 \\
    3 & 4 & 2
\end{vmatrix*}
$$
按第一行展开,然后进行计算。

\jie \quad $\begin{aligned}[t]
    \begin{vmatrix*}[r]
        3 & 1 & -2 \\
        5 & -2 & 7 \\
        3 & 4 & 2
    \end{vmatrix*}
    &= 3 \times \begin{vmatrix*}[r]
                -2 & 7 \\
                4 & 2
            \end{vmatrix*}
        -1 \times   \begin{vmatrix*}[r]
                5 & 7 \\
                3 & 2
            \end{vmatrix*}
        + (-2) \times \begin{vmatrix*}[r]
                5 & -2 \\
                3 & 4
            \end{vmatrix*} \\
    &= 3 \times (-32) - 1 \times (-11) - 2 \times 26 \\
    &= -137 \text{。}
\end{aligned}$

按一行(或一列)展开行列式来计算时,如果先根据行列式的性质把某一行(或一列)的两个元素变为零,
就会使计算简便得多。如上题,把第二列乘以 $-3$ 加到第一列,把二列乘以 $2$ 加到第三列,可得

$\begin{vmatrix*}[r]
    3 & 1 & -2 \\
    5 & -2 & 7 \\
    3 & 4 & 2
\end{vmatrix*}
= \begin{vmatrix*}[r]
        0 & 1 & 0 \\
        11 & -2 & 3 \\
        -9 & 4 & 10
  \end{vmatrix*}
= -1 \times
    \begin{vmatrix*}[r]
        11 & 3 \\
        -9 & 10
    \end{vmatrix*}
= -137 \text{。}
$


\liti 计算:

\twoInLine[16em]{(1) \;
    $\begin{vmatrix*}[r]
        4 & -6 & 3 \\
        5 & 2 & 7 \\
        5 & -2 & 8
    \end{vmatrix*}$ ;
}{(2) \;
    $\begin{vmatrix*}[r]
        8 & -6 & 9 \\
        5 & 4 & 6 \\
        4 & 5 & 8
    \end{vmatrix*}$ 。
}

\jie

(1) \quad $
\begin{vmatrix*}[r]
    4 & -6 & 3 \\
    5 & 2 & 7 \\
    5 & -2 & 8
\end{vmatrix*}
=   \begin{vmatrix*}[r]
        19 & 0 & 24 \\
        5 & 2 & 7 \\
        10 & 0 & 15
    \end{vmatrix*}
= 2 \begin{vmatrix*}[r]
        19 & 24 \\
        10 & 15
    \end{vmatrix*}
= 90
$;

(2) \quad $
\begin{vmatrix*}[r]
    8 & -6 & 9 \\
    5 & 4 & 6 \\
    4 & 5 & 8
\end{vmatrix*}
=   \begin{vmatrix*}[r]
        8 & -6 & 9 \\
        5 & 4 & 6 \\
        -1 & 1 & 2
    \end{vmatrix*}
=   \begin{vmatrix*}[r]
        2 & -6 & 21 \\
        9 & 4 & -2 \\
        0 & 1 & 0
    \end{vmatrix*}
= - \begin{vmatrix*}[r]
        2 & 21 \\
        9 & -2
    \end{vmatrix*}
= 193
$。


\liti 解方程
$$
\begin{vmatrix}
    15 - 2x & 11 & 10 \\
    11 - 3x & 17 & 16 \\
     7 - x  & 14 & 13
\end{vmatrix} = 0 \text{。}
$$

\jie $\begin{aligned}[t]
    \begin{vmatrix}
        15 - 2x & 11 & 10 \\
        11 - 3x & 17 & 16 \\
        7 - x   & 14 & 13
    \end{vmatrix}
    &=  \begin{vmatrix}
            15 - 2x & 1 & 10 \\
            11 - 3x & 1 & 16 \\
            7 - x   & 1 & 13
        \end{vmatrix} \\
    &=  \begin{vmatrix}
            8 - x  & 0 & -3 \\
            4 - 2x & 0 & 3 \\
            7 - x  & 1 & 13
        \end{vmatrix}
    =  -\begin{vmatrix*}[r]
            8 - x  & -3 \\
            4 - 2x & 3
        \end{vmatrix*} \\
    &= 9x - 36 = 9(x - 4) \text{。}
\end{aligned}$

因为方程左边等于 $9(x - 4)$,所以原方程为 $9(x - 4) = 0$,它的解集是 $\{ 4  \}$。



\liti 求证
$$
\begin{vmatrix}
    a     & b     & c \\
    a^2   & b^2   & c^2 \\
    b + c & c + a & a + b
\end{vmatrix} = (a - b)(b - c)(c - a)(a + b + c) \text{。}
$$

\zhengming $\begin{aligned}[t]
    \begin{vmatrix}
        a     & b     & c \\
        a^2   & b^2   & c^2 \\
        b + c & c + a & a + b
    \end{vmatrix}
    &=  \begin{vmatrix}
            a - b     & b - c     & c \\
            a^2 - b^2 & b^2 - c^2 & c^2 \\
            b - a     & c - b     & a + b
        \end{vmatrix} \\
    &= (a - b)(b - c)
        \begin{vmatrix}
            1     & 1     & c \\
            a + b & b + c & c^2 \\
            -1    & -1    & a + b
        \end{vmatrix} \\
    &= (a - b)(b - c)
        \begin{vmatrix}
            1     & 1     & c \\
            a + b & b + c & c^2 \\
            0     & 0     & a + b + c
        \end{vmatrix} \\
    &= (a - b)(b - c)(a + b + c)
        \begin{vmatrix}
            1     & 1 \\
            a + b & b + c
        \end{vmatrix} \\
    &= (a - b)(b - c)(c - a)(a + b + c) \text{。}
\end{aligned}$



\liti 求证
\begin{gather*}
    \begin{vmatrix*}
        x   & y   & 1 \\
        x_1 & y_1 & 1 \\
        x_2 & y_2 & 1
    \end{vmatrix*} = 0 \tag{*}\label{eq:sjhlszk-star}
\end{gather*}
是经过不同两点 $P_1(x_1, y_1)$,$P_2(x_2, y_2)$ 的直线的方程。

\zhengming $
    \begin{vmatrix*}
        x   & y   & 1 \\
        x_1 & y_1 & 1 \\
        x_2 & y_2 & 1
    \end{vmatrix*}
    =   \begin{vmatrix*}
            y_1 & 1 \\
            y_2 & 1
        \end{vmatrix*} x
      - \begin{vmatrix*}
            x_1 & 1 \\
            x_2 & 1
        \end{vmatrix*} y
      + \begin{vmatrix*}
            x_1 & y_1 \\
            x_2 & y_2
        \end{vmatrix*}
$。

因为 $P_1$ 与 $P_2$ 是不同的两点,所以 $x_1$ 与 $x_2$,$y_1$ 与 $y_x$ 不能都相等,也就是
$$
\begin{vmatrix*}
    x_1 & 1 \\
    x_2 & 1
\end{vmatrix*}, \quad
\begin{vmatrix*}
    y_1 & 1 \\
    y_2 & 1
\end{vmatrix*}
$$
不能全为零。因此,方程 \eqref{eq:sjhlszk-star} 是关于 $x$,$y$ 的一次方程,即平面上的直线方程。又因为
$$
\begin{vmatrix*}
    x_1 & y_1 & 1 \\
    x_1 & y_1 & 1 \\
    x_2 & y_2 & 1
\end{vmatrix*} = 0,\quad
\begin{vmatrix*}
    x_2 & y_2 & 1 \\
    x_1 & y_1 & 1 \\
    x_2 & y_2 & 1
\end{vmatrix*} = 0
$$
所以点 $P_1(x_1, y_1)$,$P_2(x_2, y_2)$ 都在方程 \eqref{eq:sjhlszk-star}
表示的直线上,即方程 \eqref{eq:sjhlszk-star} 是经过点 $P_1$ 与 $P_2$ 的直线的方程。



\lianxi
\begin{xiaotis}

\xiaoti{已知行列式}
$$
    \begin{vmatrix*}[r]
        3 & 6  & 7 \\
        8 & 6  & 1 \\
        2 & -5 & 4
    \end{vmatrix*} \text{,}
$$

\begin{xiaoxiaotis}

    \xiaoxiaoti{求行列式中元素 $-5$ 的余子式与代数余子式;}

    \xiaoxiaoti{按第三列展开这一行列式;}

    \xiaoxiaoti{验证行列式第一行的各元素与第三行对应元素的代数余子式的乘积的和等于零。}

\end{xiaoxiaotis}


\xiaoti{利用行列式的性质和本节的\nameref{theorem:sjhlszk-1},计算:}
\begin{xiaoxiaotis}

    \threeInLineXxt[10em]{
        $\begin{vmatrix*}[r]
            5  & 0 & -5 \\
            3  & 2 & 7 \\
            -4 & 3 & 9
        \end{vmatrix*}$;
    }{
        $\begin{vmatrix*}[r]
            -6 & 5 & 2 \\
            2  & 1 & -1 \\
            1  & 7 & 4
        \end{vmatrix*}$;
    }{
        $\begin{vmatrix*}[r]
            2  & 6 & 7 \\
            -3 & 8 & 8 \\
            -5 & 2 & 3
        \end{vmatrix*}$。
    }

\end{xiaoxiaotis}


\xiaoti{解下列关于 $x$ 的方程:}
\begin{xiaoxiaotis}

    \twoInLineXxt[16em]{
        $\begin{vmatrix*}
            2 & x+2 & 6 \\
            1 & x   & 3 \\
            1 & 3   & x
        \end{vmatrix*} = 0$;
    }{
        $\begin{vmatrix*}[r]
            a & a & x \\
            1 & 1 & 1 \\
            b & x & b
        \end{vmatrix*} = 0$。
    }

\end{xiaoxiaotis}


\xiaoti{求证:}
\begin{xiaoxiaotis}

    \xiaoxiaoti{$\begin{vmatrix*}[r]
            1  & 1  & 1 \\
            a  & b  & c \\
            bc & ca & ab
        \end{vmatrix*} = (a - b)(b - c)(c - a)$;}

    \xiaoxiaoti{$\begin{vmatrix*}[r]
            a & b & b \\
            b & a & b \\
            b & b & a
        \end{vmatrix*} = (a + 2b)(a - b)^2$;}

    \xiaoxiaoti{$\begin{vmatrix*}[r]
            1 & p & p^3 \\
            1 & q & q^3 \\
            1 & r & r^3
        \end{vmatrix*} = (p - q)(q - r)(r - p)(p + q + r)$。}

\end{xiaoxiaotis}

\end{xiaotis}

