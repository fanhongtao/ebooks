\subsection{复数的乘法与除法}\label{subsec:5-5}

复数的乘法规定按照以下的法则进行:设 $z_1 = a + b\,i$,$z_2 = c + d\,i$ 是任意两个复数,那么它们的\textbf{积}
\begin{align*}
    (a + b\,i)(c + d\,i) &= ac + bc\,i + cd\,i + bd\,i^2 \\[-1em]
        &=(ac - bd) + (bc + ad)\,i \text{。}
\end{align*}
也就是说,复数的乘法与多项式的乘法是类似的,但必须在所得的结果中把 $i^2$ 换成 $-1$ ,
并且把实部与虚部分别合并。

很明显,两个复数的积仍然是一个复数。

容易验证,复数的乘法满足交换律、结合律以及乘法对加法的分配律,即对任何 $z_1\, z_2\, z_3 \in C$,有
\begin{align*}
    z_1 \cdot z_2 &= z_2 \cdot z_1 , \\[-1em]
    (z_1 \cdot z_2) \cdot z_3 &= z_1 \cdot (z_2 \cdot z_3) , \\[-1em]
    z_1 \cdot (z_2 + z_3) &= z_1 \cdot z_2 + z_1 \cdot z_3 \text{。}
\end{align*}

根据复数的乘法法则,对于任何复数 $z = a + b\,i$,有
\begin{align*}
    (a + b\,i)(a - b\,i) &= a^2 + b^2 + (ab - ab)\,i \\[-1em]
        &= a^2 + b^2 \text{,}
\end{align*}
因此,两个共轭复数 $z$,$\bar{z}$ 的积是一个实数,这个实数等于每一个复数的模的平方,即
$$ z \cdot \bar{z} = |z|^2 = |\bar{z}|^2 \text{。} $$

\liti 计算 $(1 -2\,i)(3 + 4\,i)(-2 + i)$。

\jie \quad $\begin{aligned}[t]
        & (1 -2\,i)(3 + 4\,i)(-2 + i) \\[-1em]
    ={} & (11 - 2\,i)(-2 + i) \\[-1em]
    ={} & -20 + 15\,i \text{。}
\end{aligned}$

计算复数的乘方,要用到虚数单位 $i$ 的乘方。因为复数的乘法满足交换律与结合律,
所以实数集 $R$ 中正整数指数幂的运算律,在复数集 $C$ 中仍然成立,即对任何
$z,\, z_1,\, z_2 \in C$ 及 $m,\, n \in N$,有
\begin{align*}
    z^m \cdot z^n &= z^{m + n}, \\
    (z^m)^n &= z^{mn}, \\
    (z_1 \cdot z_2)^n &= z_1^n \cdot z_2^n \text{。}
\end{align*}

另一方面,我们有
\begin{gather*}
    i^1 = i, \\
    i^2 = -1, \\
    i^3 = i^2 \cdot i = - i, \\
    i^4 = i^3 \cdot i = - i \cdot i = - i^2 = 1 \text{。}
\end{gather*}

从而,对于任何 $n \in N$,我们都有
\begin{align*}
    i^{4n+1} &= i^{4n} \cdot i = (i^4)^n \cdot i \\
        &= 1^n \cdot i = i \text{。}
\end{align*}
同理可证
\begin{align*}
    & i^{4n + 2} = -1, \\
    & i^{4n + 3} = -i, \\
    & i^{4n} = 1 \text{。}
\end{align*}

这就是说,如果 $n \in N$,那么
\begin{center}
    \framebox{\begin{minipage}{23em}
        \begin{gather*}
            i^{4n+1} = i,\quad i^{4n+2} = -1,\quad i^{4n+3} = -i,\quad i^{4n} = 1 \text{。}
        \end{gather*}
    \end{minipage}}
\end{center}

\liti 计算 $\left( \dfrac{1}{2} - \dfrac{\sqrt{3}}{2}\,i \right)^3$。

\jie \quad $\begin{aligned}[t]
        & \left( \dfrac{1}{2} - \dfrac{\sqrt{3}}{2}\,i \right)^3 \\
    ={} & \left( \dfrac{1}{2} \right)^3 - 3 \left( \dfrac{1}{2} \right)^2 \left( \dfrac{\sqrt{3}}{2}\,i \right)
        + 3 \left( \dfrac{1}{2} \right) \left( \dfrac{\sqrt{3}}{2}\,i \right)^2
        - \left( \dfrac{\sqrt{3}}{2}\,i \right)^3 \\
    ={} & \dfrac{1}{8} - \dfrac{3\sqrt{3}}{8}\,i - \dfrac{9}{8} + \dfrac{3\sqrt{3}}{8}\,i = -1 \text{。}
\end{aligned}$



复数的除法规定是乘法的逆运算,即把满足
$$ (c + d\,i)(x + y\,i) = a + b\,i \quad (c + d\,i \neq 0) $$
的复数 $x + y\,i$,叫做复数 $a + b\,i$ 除以复数 $c + d\,i$ 的\textbf{商},记作
$(a + b\,i) \div (c + d\,i)$ 或 $\dfrac{a + b\,i}{c + d\,i}$。

我们知道,两个共轭复数的积是一个实数,因此,两个复数相除,可以先把它们的商写成
分式的形式,然后把分子与分母都乘以分母的共轭复数,并且把结果化简,即
\begin{align*}
        & \dfrac{a + b\,i}{c + d\,i} = \dfrac{(a + b\,i)(c - d\,i)}{(c + d\,i)(c - d\,i)} \\
    ={} & \dfrac{(ac + bd) + (bc - ad)\,i}{c^2 + d^2} \\
    ={} & \dfrac{ac + bd}{c^2 + d^2} + \dfrac{bc - ad}{c^2 + d^2}\,i \qquad (c + d\,i \neq 0) \text{。}
\end{align*}
因为 $c + d\,i \neq 0$,所以 $c^2 + d^2 \neq 0$。由此可见,商 $\dfrac{a + b\,i}{c + d\,i}$ 是一个唯一确定的复数。


\liti 计算 $(1 + 2\,i) \div (3 - 4\,i)$。

\jie \quad $\begin{aligned}[t]
    & (1 + 2\,i) \div (3 - 4\,i) = \dfrac{1 + 2\,i}{3 - 4\,i} \\
    & = \dfrac{(1 + 2\,i)(3 + 4\,i)}{(3 - 4\,i)(3 + 4\,i)} = \dfrac{-5 + 10\,i}{25} = -\dfrac{1}{5} + \dfrac{2}{5}\,i \text{。}
\end{aligned}$



\lianxi
\begin{xiaotis}


\xiaoti{证明复数的乘法满足交换律、结合律以及乘法对加法的分配律。}


\xiaoti{计算:}
\begin{xiaoxiaotis}

    \renewcommand\arraystretch{1.2}
    \begin{tabular}[t]{*{2}{@{}p{16em}}}
        \xiaoxiaoti{$(-8 - 7\,i)(-3\,i)$;} & \xiaoxiaoti{$(4 - 3\,i)(-5 - 4\,i)$;} \\
        \xiaoxiaoti{$\left( -\dfrac{1}{2} + \dfrac{\sqrt{3}}{2}\,i \right)(1 + i)$;} & \xiaoxiaoti{$\left( \dfrac{\sqrt{3}}{2}\,i - \dfrac{1}{2} \right)\left( -\dfrac{1}{2} + \dfrac{\sqrt{3}}{2}\,i \right)$。}
    \end{tabular}

\end{xiaoxiaotis}


\xiaoti{(口答)$i^{11}$,$i^{25}$,$i^{26}$,$i^{36}$,$i^{70}$,$i^{101}$,$i^{355}$,$i^{400}$ 各等于什么?}


\xiaoti{计算:}
\begin{xiaoxiaotis}

    \renewcommand\arraystretch{1.2}
    \begin{tabular}[t]{*{3}{@{}p{11em}}}
        \xiaoxiaoti{$\dfrac{1}{i}$;} & \xiaoxiaoti{$\dfrac{1}{i^3}$;} & \xiaoxiaoti{$\dfrac{1}{\sqrt{2}\,i}$;} \\[1em]
        \xiaoxiaoti{$\dfrac{2\,i}{1 - i}$;} & \xiaoxiaoti{$\dfrac{2 + i}{7 + 4\,i}$;} & \xiaoxiaoti{$\dfrac{1}{(9 + 2\,i)^2}$。}
    \end{tabular}

\end{xiaoxiaotis}


\end{xiaotis}