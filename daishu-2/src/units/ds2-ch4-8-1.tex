\subsubsection{顺序消元法解线性方程组举例}

含 $n$ 个未知数 $n$ 个方程的线性方程组,当它的系数行列式不等于零时,可以利用\nameref{klmfz}得出解的公式。
克莱姆法则在理论上有重要作用,但在具体解题时,要计算 $n+1$ 个 $n$ 阶行列式,计算量较大。
在解三元或四元线性方程组时,计算已经比较麻烦,解多于四元的线性方程组,计算就更复杂。有没有别的解法呢?

回想起我们学过的消元法(加减消元法和代入消元法),是把一个系数行列式不等于零的三元线性方程组

(\ref{eq:fcz-2})
\begin{minipage}[c]{0.90\textwidth}
    $$\begin{cases}
        a_1 x + b_1 y + c_1 z = d_1 , \\
        a_2 x + b_2 y + c_2 z = d_2 , \\
        a_3 x + b_3 y + c_3 z = d_3 ,
    \end{cases}$$
\end{minipage}
通过消元,最后化为
$$
\begin{cases}
    1 \cdot x + 0 \cdot y + 0 \cdot z = x_1 , \\
    0 \cdot x + 1 \cdot y + 0 \cdot z = y_1 , \\
    0 \cdot x + 0 \cdot y + 1 \cdot z = z_1
\end{cases}
$$
的形式,从而得出方程组的解 $(x_1,\; y_1,\; z_1)$。
消元法的基本思想是把方程组中一部分方程化成含较少未知数的方程,
在系数行列式不为零的情况下,最终化到每一个方程只含一个未知数。
现在,我们再来学习一种顺序消元法,它的基本思想仍是消元,但要求
按一定的顺序进行消元。下面我们先以解三元线性方程组为例进行说明。

\liti\mylabel{liti:xyf-1}[例 1] 解线性方程组
$$\left\{
    \begin{alignedat}{3}
           &     & 2y & +{} & 3z & = -8, \\
        x  & +{} & 3y & -{} & 2z & = 2, \\
        2x & -{} & 3y & +{} & 7z & = -9 \text{。}
    \end{alignedat}
\right.$$

\jie 先把方程组中第一和第二个方程互换,使得互换后得出的方程组中
第一个方程中的 $x$ 的系数不等于零,得
$$\left\{
    \begin{alignedat}{4}
        x  & +{} & 3y & -{} & 2z & ={} & 2 \text{,} \\
           &     & 2y & +{} & 3z & ={} & -8 \text{,} \\
        2x & -{} & 3y & +{} & 7z & ={} & -9 \text{。}
    \end{alignedat}
\right.$$

从第三个方程减去第一个方程的 $2$ 倍,消去第三个方程中的 $x$
(即使 $x$ 的系数化为零),得
$$\left\{
    \begin{alignedat}{4}
        x  & +{} & 3y & -{} &  2z & ={} & 2 \text{,} \\
           &     & 2y & +{} &  3z & ={} & -8 \text{,} \\
           & -{} & 9y & +{} & 11z & ={} & -13 \text{。}
    \end{alignedat}
\right.$$

把第二个方程乘以 $\dfrac{1}{2}$,使其中 $y$ 的系数化为 $1$,得
$$\left\{
    \begin{alignedat}{4}
        x  & +{} & 3y & -{} &  2z & ={} & 2 \text{,} \\
           &     &  y & +{} & \dfrac{3}{2}z & ={} & -4 \text{,} \\
           & -{} & 9y & +{} & 11z & ={} & -13 \text{。}
    \end{alignedat}
\right.$$


把第三个方程加上第二个方程的 $9$ 倍,消去第三个方程中的 $y$,得
$$\left\{
    \begin{alignedat}{4}
        x  & +{} & 3y & -{} &  2z & ={} & 2 \text{,} \\
           &     &  y & +{} & \dfrac{3}{2}z & ={} & -4 \text{,} \\
           &     &    &     & \dfrac{49}{2}z & ={} & -49 \text{。}
    \end{alignedat}
\right.$$

把第三个方程乘以 $\dfrac{2}{49}$,使其中 $z$ 的系数化为 $1$,得
$$\left\{
    \begin{alignedat}{4}
        x  & +{} & 3y & -{} &  2z & ={} & 2 \text{,} \\
           &     &  y & +{} & \dfrac{3}{2}z & ={} & -4 \text{,} \\
           &     &    &     & z & ={} & -2 \text{。}
    \end{alignedat}
\right.$$

从第一及第二个方程分别减去第三个方程的 $-2$ 倍及 $\dfrac{3}{2}$ 倍,
消去前两个方程中的 $z$,得
$$\left\{
    \begin{alignedat}{4}
        x  & +{} & 3y &  &   & ={} & -2 \text{,} \\
           &     &  y & \phantom{={}} &   & ={} & -1 \text{,} \\
           &     &    &  & z & ={} & -2 \text{。}
    \end{alignedat}
\right.$$

从第一个方程减去第二个方程的 $3$ 倍,消去第一个方程中的 $y$,得
$$\left\{
    \begin{alignedat}{4}
        x  & \phantom{={}} &    &  &   & ={} & 1 \text{,} \\
           &     &  y & \phantom{={}} &   & ={} & -1 \text{,} \\
           &     &    &  & z & ={} & -2 \text{。}
    \end{alignedat}
\right.$$

所以方程组的解是 $(1,\; -1,\; -2)$。

从上面的求解过程可以看出,它是按一定程序来进行的。

第一步:先把第一个方程中 $x$ 的系数化为 $1$,消去后两个方程中的 $x$
(在 \nameref{liti:xyf-1} 中,第一个方程中 $x$ 的系数是 $0$,因此先把第一、第二个方程互换),
再把第二个方程中 $y$ 的系数化为 $1$,消去第三个方程中的 $y$,
然后再把第三个方程中 $z$ 的系数化为$1$。
第二步:回过头来,再按相反顺序消去第一、第二个方程中的 $z$
(相当于把 $z$ 的值 $z_1$ 代入第一、第二个方程)和消去第一个方程中的 $y$
(相当于把 $y$ 的值 $y_1$ 代入第一个方程),从而得出方程组的解 $(x_1,\; y_1,\; z_1)$。

这种用顺序消元来解线性方程组的方法,看来好象很呆板,但正因为它是
按确定的程序进行的,因此有利于用电子计算机进行计算。

