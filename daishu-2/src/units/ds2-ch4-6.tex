\starredsubsection[三元齐次线性方程组]{三元齐次线性方程组 \protect\footnote{凡标有“*”号的章、节,供第三类型和第一类型选修数学的学生选学。}}\label{subsec:4-6}

常数项为零的三元线性方程组

\fangchengzu{eq:fcz-3}(\thefangchengzu)
\begin{minipage}[c]{0.90\textwidth}
    \begin{numcases}{}
        a_1 x + b_1 y + c_1 z = 0 \text{,} \label{eq:syqcxxfcz-1} \\
        a_2 x + b_2 y + c_2 z = 0 \text{,} \label{eq:syqcxxfcz-2} \\
        a_3 x + b_3 y + c_3 z = 0 \text{。} \label{eq:syqcxxfcz-3}
    \end{numcases}
\end{minipage}
叫做\textbf{三元齐次线性方程组}。显然,三元齐次线性方程组总有解 $(0,\; 0,\; 0)$,它叫做\textbf{零解}。
下面进一步讨论方程组 \eqref{eq:fcz-3} 会不会有非零解的情况。用 $D$ 表示方程组 \eqref{eq:fcz-3} 的系数行列式。

\begin{enumerate}[(1), nosep]
    \item $D \neq 0$。方程组 \eqref{eq:fcz-3} 有唯一解 —— 零解。
    \item $D = 0$。我们来证明方程组 \eqref{eq:fcz-3} 除零解外还有无穷多非零解。\footnote{
        利用第 \ref{subsec:4-5} 节三元线性方程组 \eqref{eq:fcz-2} 当 $D=0$ 时或者无解或者
        有无穷多解的结论,容易得出三元齐次线性方程组 \eqref{eq:fcz-3} 当 $D=0$ 时一定
        有无穷多非零解。这里我们是从头证明,并同时给出了求解的方法。
    }
\end{enumerate}


(1$^\circ$) $D$ 中至少有一个元素的代数余子式不等于零。不失一般性,设
$$ C_3 = \begin{vmatrix*}
    a_1 & b_1 \\
    a_2 & b_2
\end{vmatrix*} \neq 0 ,$$

把方程 \eqref{eq:syqcxxfcz-1}, \eqref{eq:syqcxxfcz-2} 中含 $z$ 的项到等号右边,得
$$\begin{cases}
    a_1x + b_1y = -c_1z \text{,}\\
    a_2x + b_2y = -c_2z \text{。}
\end{cases}$$
把这个方程组看成关于 $x$,$y$ 的线性方程组,解出
$$\begin{cases}
    x = \hspace{0.9em} \dfrac{
            \begin{vmatrix*}
                b_1 & c_1 \\
                b_2 & c_2
            \end{vmatrix*}
        }{
            \begin{vmatrix*}
                a_1 & b_1 \\
                a_2 & b_2
            \end{vmatrix*}
        }z = \dfrac{A_3}{C_3}z \text{,} \\[4em]
    y = \dfrac{
            -\begin{vmatrix*}
                a_1 & c_1 \\
                a_2 & c_2
            \end{vmatrix*}
        }{
            \phantom{-}\begin{vmatrix*}
                a_1 & b_1 \\
                a_2 & b_2
            \end{vmatrix*}
        }z = \dfrac{B_3}{C_3}z \text{。}
\end{cases}$$

令
$$ z = C_3t \quad (\text{t 为任意常数}) \text{,}$$
得
\begin{equation}
    \begin{cases}
        x = A_3t \text{,}\\
        y = B_3t \text{,}\\
        z = C_3t
    \end{cases} \label{eq:syqcxxfcz-4}
\end{equation}
\eqref{eq:syqcxxfcz-4} 是方程 \eqref{eq:syqcxxfcz-1} 和方程 \eqref{eq:syqcxxfcz-2} 的所有公共解
的一般表示形式。把 \eqref{eq:syqcxxfcz-4} 式代入方程 \eqref{eq:syqcxxfcz-3} 左边,得
\begin{align*}
    a_3x + b_3y + c_3z &= a_3A_3t + b_3B_3t + c_3C_3t \\
        &= D \cdot t = 0,
\end{align*}
这说明 \eqref{eq:syqcxxfcz-4} 式又同时适合方程 \eqref{eq:syqcxxfcz-3}。
因此,\eqref{eq:syqcxxfcz-4} 式表示方程组 \eqref{eq:fcz-3} 的解,
而且包括方程组 \eqref{eq:fcz-3} 的所有的解。

对任意的一个 $t$ 值, \eqref{eq:syqcxxfcz-4} 式都可确定方程组 \eqref{eq:fcz-3} 的一个解,
$t$ 值不同,确定的解也不同,而只有 $t = 0$ 时它才是零解,所以方程组 \eqref{eq:fcz-3} 有无穷多非零解。


(2$^\circ$)  $D$ 中每一个元素的代数余子式都等于零。这时,如果方程组 \eqref{eq:fcz-3}
的每个系数都等于零,那么任意一组 $x$,$y$,$z$ 的值都是方程组 \eqref{eq:fcz-3} 的解,
当然它有无穷多非零解。如果系数不全为零,不失一般性,设 $b_1 \neq 0$,由
$$\begin{vmatrix*}
    a_1 & b_1 \\
    a_2 & b_2
\end{vmatrix*} = 0, \qquad
\begin{vmatrix*}
    b_1 & c_1 \\
    b_2 & c_2
\end{vmatrix*} = 0,
$$
得
$$a_1b_2 = a_2b_1, \qquad b_1c_2 = b_2c_1 \text{。}$$

$\therefore \qquad a_2 = \dfrac{a_1b_2}{b_1}, \qquad c_2 = \dfrac{b_2c_1}{b_1} \text{。}$

因此方程 \eqref{eq:syqcxxfcz-2} 就可由方程 \eqref{eq:syqcxxfcz-1} 两边同乘以常数 $\dfrac{b_2}{b_1}$ 得出。
同样,方程 \eqref{eq:syqcxxfcz-3} 可由方程 \eqref{eq:syqcxxfcz-1} 两边同乘以常数 $\dfrac{b_3}{b_1}$ 得出。
因此方程 \eqref{eq:syqcxxfcz-1} 的解就是方程组 \eqref{eq:fcz-3} 的解,所以方程组 \eqref{eq:fcz-3} 除零解
外还有无穷多非零解。

反过来, 如果方程组 \eqref{eq:fcz-3} 有非零解,那么它的系数行列式 $D = 0$。
不然的话,即如果 $D \neq 0$,那么根据\nameref{klmfz},可推出方程组 \eqref{eq:fcz-3}
只有零解,这和方程组 \eqref{eq:fcz-3} 有非零解相矛盾。

综上所述, 可以得出:

\begin{theorem}\label{theorem:syqcxxfcz-1}
    齐次线性方程组 \eqref{eq:fcz-3} 有非零解的充要条件是它的系数行列式 $D$ 等于零。
\end{theorem}


\liti 解齐次线性方程组
$$\begin{cases}
    x + y + z = 0, \\
    2x + 2y + 3z = 0, \\
    4x + 4y + 5z = 0 \text{。}
\end{cases}$$

\jie 因为
$$D = \begin{vmatrix*}
	1 & 1 & 1 \\
	2 & 2 & 3 \\
	4 & 4 & 5
\end{vmatrix*} = 0,$$
所以方程组有无穷多解。

又因为
$$\begin{vmatrix*}
    b_1 & c_1 \\
    b_2 & c_2
\end{vmatrix*} = \begin{vmatrix*}
    1 & 1 \\
    2 & 3
\end{vmatrix*} = 1 \neq 0,
$$
把第一、第二两个方程中含 $x$ 的项移到等号右边,得
$$\begin{cases}
    y + z = -x, \\
    2y + 3z = -2x \text{。}
\end{cases}$$
把这个方程组看成关于 $y$,$z$ 的线性方程组,解出
$$\begin{cases}
    y = -x, \\
    z = 0 \text{。}
\end{cases}$$
令 $x = t$,那么 $y = -t$,$z = 0$。不管 $t$ 取什么值,$(t,\; -t,\; 0)$ 总适合第三个方程。

因此,原方程组的解集是 $\{ (t,\; -t,\; 0) \mid  t\; \text{为任意常数} \}$。



\liti 求方程组
$$\begin{cases}
    a_1x + b_1y + c_1 = 0, \\
    a_2x + b_2y + c_2 = 0, \\
    a_3x + b_3y + c_3 = 0
\end{cases}$$
有解的必要条件。

\jie 如果这个方程组有解,那么至少存在一个有序数组 $(x_1,\; y_1)$,使得
$$\begin{cases}
    a_1x + b_1y + c_1 = 0, \\
    a_2x + b_2y + c_2 = 0, \\
    a_3x + b_3y + c_3 = 0,
\end{cases}$$
即
$$\begin{cases}
    a_1x + b_1y + c_1 \cdot 1 = 0, \\
    a_2x + b_2y + c_2 \cdot 1 = 0, \\
    a_3x + b_3y + c_3 \cdot 1 = 0 \text{。}
\end{cases}$$
也就是说,三元齐次线性方程组
$$\begin{cases}
    a_1x + b_1y + c_1z = 0, \\
    a_2x + b_2y + c_2z = 0, \\
    a_3x + b_3y + c_3z = 0
\end{cases}$$
有一个非零解 $(x_1,\; y_1,\; 1)$。根据齐次线性方程组有非零解的必要条件是它的系数行列式等于零,从而推出
$$D = \begin{vmatrix*}
	a_1 & b_1 & c_1 \\
	a_2 & b_2 & c_2 \\
	a_3 & b_3 & c_3
\end{vmatrix*} = 0 \text{。}$$

因此,原方程组
$$\begin{cases}
    a_1x + b_1y + c_1 = 0, \\
    a_2x + b_2y + c_2 = 0, \\
    a_3x + b_3y + c_3 = 0
\end{cases}$$
有解的必要条件是
$$\begin{vmatrix*}
	a_1 & b_1 & c_1 \\
	a_2 & b_2 & c_2 \\
	a_3 & b_3 & c_3
\end{vmatrix*} = 0 \text{。}$$

想一想,能否把题中的必要条件改为充要条件,为什么?

\lianxi

下列齐次线性方程组有没有非零解?如果有,把解集求出来。

\begin{xiaoxiaotis}
    \twoInLineXxt[16em]{
        $\begin{cases}
            x + y + z = 0, \\
            2x - y + 3z = 0, \\
            x - 2y + z = 0;
        \end{cases}$
    }{
        $\begin{cases}
            5x - 6y - 4z = 0, \\
            x + 2y + 4z = 0, \\
            3x + 2y + 6z = 0 \text{。}
        \end{cases}$
    }
\end{xiaoxiaotis}

