\subsection{复数的三角形式}\label{subsec:5-6}

我们知道,与复数 $z = a + b\,i$ 对应的向量 $\overrightarrow{OZ}$ (图 \ref{fig:5-9})的模 $r$ 叫做这个复数的模,并且
$$ r = \sqrt{a^2 + b^2} \text{。} $$

\begin{figure}[htbp]
    \centering
    \begin{tikzpicture}[>=Stealth, scale=0.8]
    \draw [->] (-1, 0) -- (4.5, 0) node[anchor=west] {$x$};
    \draw [->] (0, -1) -- (0, 4) node[anchor=east] {$y$};
    \node at (-0.3, -0.3) {$O$};

    \coordinate (O) at (0, 0);
    \coordinate (Z) at (3.5, 2.8);
    \draw[->, thick] (O) -- (Z);
    \coordinate (R) at (3.5, 0);
    \draw[dashed] (R) -- (Z);
    \draw [->] (1.0, 0) arc (0:39:1.0);
    \node at (1.2, 0.5) {$\theta$};
    \node at (1.8, -0.3) {$a$};
    \node at (3.8, 1.3) {$b$};
    \node at (1.4, 1.6) {$r$};
    \node at (4.3, 3.1) {$Z:\; a + b\,i$};
\end{tikzpicture}

    \caption{}\label{fig:5-9}
\end{figure}

以 $x$ 轴的正半轴为始边、向量 $\overrightarrow{OZ}$ 所在的射线(起点是 $O$)
为终边的角 $\theta$,叫做\textbf{复数 $z = a + b\,i$ 的辐角}。

不等于零的复数 $z = a + b\,i$ 的辐角有无限多个值,这些值相差 $2\pi$ 的整数倍。
例如,复数 $i$ 的辐角是 $\dfrac{\pi}{2} + 2k\pi$,其中 $k$ 可以取任何整数。

适合于 $0 \leqslant \theta < 2\pi$ 的辐角 $\theta$ 的值,叫做\textbf{辐角的主值},
通常记作 $\arg z$ ,即 $0 \leqslant \arg z < 2\pi$。

每一个不等于零的复数有唯一的模与辐角的主值,并且可由它的模与辐角的主值唯一确定。
因此,\textbf{两个非零复数相等当且仅当它们的模与辐角的主值分别相等}。

很明显, \textbf{当 $a \in R^+$ 时,}
\begin{align*}
    & \bm{\arg a = 0,} \\
    & \bm{\arg(-a) = \pi,} \\
    & \bm{\arg(a\,i) = \dfrac{\pi}{2},} \\
    & \bm{\arg(-a\,i) = \dfrac{3\pi}{2} \text{。}}
\end{align*}

如果 $z = 0$,那么与它对应的向量 $\overrightarrow{OZ}$ 缩成一个点(零向量),这样的向量的方向是任意的,所以复数 $0$ 的辐角也是任意的。

从图 \ref{fig:5-9} 可以看出:
$$\begin{cases}
    a = r \cos\theta, \\
    b = r \sin\theta \text{。}
\end{cases}$$

$\therefore \quad \begin{aligned}[t]
    a + b\,i &= r\cos\theta + i\,r\sin\theta \\
        &= r(\cos\theta + i\sin\theta) \text{,}
\end{aligned}$\\
其中
$$ r = \sqrt{a^2 + b^2},\quad \cos\theta = \dfrac{a}{r},\quad \sin\theta = \dfrac{b}{r} \text{。} $$
当与 $z$ 对应的点 $Z$ 不在实轴或虚轴上时,$z$ 的辐角 $\theta$ 的终边所在的象限就是点 $Z$ 所在的象限;
当点 $Z$ 在实轴或虚轴上时,辐角 $\theta$ 的终边就是从原点 $O$ 出发、经过点 $Z$ 的半条坐标轴。

因此我们可以说,任何一个复数 $z = a + b\,i$ 都可以表示成
$$ \bm{r(\cos\theta + i\sin\theta)} $$
的形式。

$r(\cos\theta + i\sin\theta)$ 叫做复数 $a + b\,i$ 的\textbf{三角形式}。
为了同三角形式区别开来,$a + b\,i$ 叫做复数的\textbf{代数形式}。


\liti 把复数 $\sqrt{3} + i$ 表示成三角形式。

\jie \quad $r = \sqrt{3 + 1} = 2, \quad \cos\theta = \dfrac{\sqrt{3}}{2}$ 。

因为与 $\sqrt{3} + i$ 对应的点在第一象限,所以 $\arg(\sqrt{3} + i) = \dfrac{\pi}{6}$,于是
$$ \sqrt{3} + i = 2 \left( \cos\dfrac{\pi}{6} + i \sin\dfrac{\pi}{6} \right) \text{。} $$



\liti 把复数 $1 - i$ 表示成三角形式。

\jie \quad $r = \sqrt{1 + 1} = \sqrt{2}, \quad \cos\theta = \dfrac{1}{\sqrt{2}} = \dfrac{\sqrt{2}}{2}$ 。

因为与 $1 - i$ 对应的点在第四象限,所以 $\arg(1 - i) = \dfrac{7\pi}{4}$,于是
$$ 1 - i = \sqrt{2} \left( \cos\dfrac{7\pi}{4} + i\sin\dfrac{7\pi}{4} \right) \text{。} $$


\liti 把复数 $-1$ 表示成三角形式。

\jie \quad $r = \sqrt{1 + 0} = 1$ 。

因为与 $-1$ 对应的点在 $x$ 轴的负半轴上,所以 $\arg(-1) = \pi$,于是
$$ -1 = \cos\pi + i\sin\pi \text{。} $$

当然,把一个复数表示成三角形式时,辐角 $\theta$ 不一定要取主值。例如:
$\sqrt{2} \left[ \cos\left( -\dfrac{\pi}{4} \right) + i\sin\left( -\dfrac{\pi}{4} \right)\right]$
也是复数 $1 - i$ 的三角形式。


\lianxi
\begin{xiaotis}

\xiaoti{把下列复数表示成三角形式,并且画出与它们对应的向量。}
\begin{xiaoxiaotis}

    \renewcommand\arraystretch{1.2}
    \begin{tabular}[t]{*{3}{@{}p{10em}}}
        \xiaoxiaoti{$4$;} & \xiaoxiaoti{$-3$;} & \xiaoxiaoti{$2\,i$;} \\
        \xiaoxiaoti{$-i$;} & \xiaoxiaoti{$-2 + 2\,i$;} & \xiaoxiaoti{$-1 - \sqrt{3}\,i$;} \\
        \xiaoxiaoti{$\dfrac{\sqrt{3}}{2} - \dfrac{1}{2}\,i$;} & \xiaoxiaoti{$3 - 4\,i$;} & \xiaoxiaoti{$-4 + 3\,i$。}
    \end{tabular}

\end{xiaoxiaotis}


\xiaoti{下列复数是不是复数的三角形式?如果不是,把它们表示成三角形式。}
\begin{xiaoxiaotis}

    \xiaoxiaoti{$\dfrac{1}{2} \left( \cos\dfrac{\pi}{4} - i\sin\dfrac{\pi}{4} \right)$;}

    \xiaoxiaoti{$-\dfrac{1}{2} \left( \cos\dfrac{\pi}{3} + i\sin\dfrac{\pi}{3} \right)$;}

    \xiaoxiaoti{$\dfrac{1}{2} \left( \sin\dfrac{3\pi}{4} + i\cos\dfrac{3\pi}{4} \right)$;}

    \xiaoxiaoti{$\cos\dfrac{7\pi}{5} + i\sin\dfrac{7\pi}{5}$。}

\end{xiaoxiaotis}


\xiaoti{把下列复数表示成代数形式:}
\begin{xiaoxiaotis}

    \xiaoxiaoti{$4\left( \cos\dfrac{\pi}{3} + i\sin\dfrac{\pi}{3} \right)$;}

    \xiaoxiaoti{$\sqrt{2} \left( \cos\dfrac{3\pi}{4} + i\sin\dfrac{3\pi}{4} \right)$;}

    \xiaoxiaoti{$6 \left( \cos\dfrac{11\pi}{6} + i\sin\dfrac{11\pi}{6} \right)$;}

    \xiaoxiaoti{$3 \left( \cos\dfrac{3\pi}{2} + i\sin\dfrac{3\pi}{2} \right)$。}

\end{xiaoxiaotis}


\end{xiaotis}


