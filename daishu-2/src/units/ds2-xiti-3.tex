\xiti

\begin{xiaotis}

\xiaoti{写出数列的一个通项公式,使它的前 $4$ 项分别是下列各数:}
\begin{xiaoxiaotis}

    \renewcommand\arraystretch{1.5}
    \begin{tabular}[t]{*{2}{@{}p{16em}}}
        \xiaoxiaoti{$3$,$6$,$9$,$12$;} & \xiaoxiaoti{$0$,$-2$,$-4$,$-6$;} \\
        \xiaoxiaoti{$\dfrac{2}{1}$,$\dfrac{3}{2}$,$\dfrac{4}{3}$,$\dfrac{5}{4}$;} & \xiaoxiaoti{$-\dfrac{1}{2 \times 1}$,$\dfrac{1}{2 \times 2}$,$-\dfrac{1}{2 \times 3}$,$\dfrac{1}{2 \times 4}$;} \\
        \xiaoxiaoti{$1$,$\dfrac{1}{4}$,$\dfrac{1}{9}$,$\dfrac{1}{16}$;} & \xiaoxiaoti{$\sqrt[3]{1}$,$-\sqrt[3]{2}$,$\sqrt[3]{3}$,$-\sqrt[3]{4}$。}
    \end{tabular}

\end{xiaoxiaotis}


\xiaoti{已知无穷数列 $1 \times 2$,$2 \times 3$,$3 \times 4$,$4 \times 5$,$\cdots$,$n (n + 1)$,$\cdots$ 。}
\begin{xiaoxiaotis}

    \xiaoxiaoti{求这个数列的第 $10$ 项,第 $31$ 项及第 $48$ 项;}

    \xiaoxiaoti{$420$ 是这个数列中的第几项?}

\end{xiaoxiaotis}


\xiaoti{}
\begin{xiaoxiaotis}

    \vspace{-1.6em} \begin{minipage}{0.9\textwidth}
    \xiaoxiaoti{已知数列 $\{a_n\}$ 的第 $1$ 项是 $1$,第 $2$ 项是 $2$,以后各项由公式
        $a_n = a_{n - 2} + a_{n - 1}$ 给出。写出这个数列的前 $10$ 项;}
    \end{minipage}

    \xiaoxiaoti{用上面的数列,通过公式 $b_n = \dfrac{a_n}{a_{n + 1}}$ 构造一个新的数列 $\{b_n\}$,
        并写出这个数列的前 $10$ 项。}

\end{xiaoxiaotis}


\xiaoti{已知数列 $\{a_n\}$ 的通项公式是 $a_n = -2n + 3$。}
\begin{xiaoxiaotis}

    \xiaoxiaoti{计算 $a_2 - a_1$,$a_3 - a_2$,$a_4 - a_3$,$a_5 - a_4$;}

    \xiaoxiaoti{计算 $a_{n + 1} - a_n$;}

    \xiaoxiaoti{证明这个数列是一个等差数列,并求出它的首项与公差。}

\end{xiaoxiaotis}


\xiaoti{}
\begin{xiaoxiaotis}

    \vspace{-1.6em} \begin{minipage}{0.9\textwidth}
    \xiaoxiaoti{一个等差数列的第 $1$ 项是 $5.6$,第 $6$ 项是 $20.6$,求它的第 $4$ 项;}
    \end{minipage}

    \xiaoxiaoti{一个等差数列的第 $3$ 项是 $9$,第 $9$ 项是 $3$,求它的第 $12$ 项。}

\end{xiaoxiaotis}


\xiaoti{求下列各题中两数的等差中项:}
\begin{xiaoxiaotis}

    \xiaoxiaoti{$647$ 与 $895$;}

    \xiaoxiaoti{$-180$ 与 $360$;}

    \xiaoxiaoti{$\dfrac{\sqrt{3} + \sqrt{2}}{\sqrt{3} - \sqrt{2}}$ 与 $\dfrac{\sqrt{3} - \sqrt{2}}{\sqrt{3} + \sqrt{2}}$;}

    \xiaoxiaoti{$(a + b)^2$ 与 $(a - b)^2$。}

\end{xiaoxiaotis}


\xiaoti{}
\begin{xiaoxiaotis}

    \vspace{-1.6em} \begin{minipage}{0.9\textwidth}
    \xiaoxiaoti{下面是全国统一鞋号中成年女鞋的各种尺码(表示鞋底长,单位是厘米);\\
        $21$,$21\dfrac{1}{2}$,$22$,$22\dfrac{1}{2}$,$23$,$23\dfrac{1}{2}$,$24$,$24\dfrac{1}{2}$,$25$。\\
        这些尺码是否成等差数列?如果是,公差是多少?
    }
    \end{minipage}

    \xiaoxiaoti{全国统一鞋号中成年男鞋共有 14 种尺码,其中最小的尺码是 $23\dfrac{1}{2}$(厘米),
        各相邻的两个尺码都相差 $\dfrac{1}{2}$ 厘米。把全部尺码从小到大列出。}

\end{xiaoxiaotis}


\xiaoti{}
\begin{xiaoxiaotis}

    \vspace{-1.6em} \begin{minipage}{0.9\textwidth}
    \xiaoxiaoti{在 $12$ 与 $60$ 之间插入 $3$ 个数,使它们同这两个数成等差数列;}
    \end{minipage}

    \xiaoxiaoti{在 $8$ 与 $36$ 之间插入 $6$ 个数,使它们同这两个数成等差数列。}

\end{xiaoxiaotis}


\xiaoti{在通常情况下,从地面到 $1$ 万米高空,高度每增加 $1$ 千米,气温就下降某一固定数值。
    如果 $1$ 千米高度的气温是 $8.5$ ℃,$5$ 千米高度的气温是 $-17.5$ ℃ ,
    求 $2$ 千米、 $4$ 千米及 $8$ 千米高度的气温。}


\xiaoti{安裝在一个公共轴上的 $5$ 个皮带轮的直径成等差数列,其中最大的与最小的皮带轮
    的直径分别是 $216$ 毫米与 $120$ 毫米,求中间三个皮带轮的直径。}

\xiaoti{一种车床变速箱的 $8$ 个齿轮的齿数成等差列, 其中首末两个齿轮的
    齿数分别是与 $24$ 与 $45$,求其余各齿轮的齿数。}


\xiaoti{}
\begin{xiaoxiaotis}

    \vspace{-1.6em} \begin{minipage}{0.9\textwidth}
    \xiaoxiaoti{在正整集合中有多少个三位数?求它们的和。}
    \end{minipage}

    \xiaoxiaoti{在三位正整数的集合中有多少个数是 $7$ 的倍数?求它们的和。}

    \xiaoxiaoti{求等差数列 $13$,$15$,$17$,$\cdots$,$81$ 的各项的和。}

    \xiaoxiaoti{求等差数列 $10$,$7$,$4$,$\cdots$,$-47$ 的各项的和。}

\end{xiaoxiaotis}


\xiaoti{根据下列各题中的条件,求相应的等差数列 $\{a_n\}$ 的有关末知数:}
\begin{xiaoxiaotis}

    \xiaoxiaoti{$a_1 = 20$,$a_n = 54$,$S_n = 999$,求 $d$ 及 $n$;}

    \xiaoxiaoti{$d = \dfrac{1}{3}$,$n = 37$,$S_n = 629$,求 $a_1$ 及 $a_n$;}

    \xiaoxiaoti{$a_1 = \dfrac{5}{6}$,$d = -\dfrac{1}{6}$,$S_n = -5$,求 $n$ 及 $a_n$;}

    \xiaoxiaoti{$d = 2$,$n = 15$,$a_n = -10$,求 $a_1$ 及 $S_n$;}

\end{xiaoxiaotis}


\xiaoti{}
\begin{xiaoxiaotis}

    \vspace{-1.6em} \begin{minipage}{0.9\textwidth}
    \xiaoxiaoti{某等差数列 $\{a_n\}$ 的通项公式是 $a_n = 3n - 2$,求它的前 $n$ 项的和的公式。}
    \end{minipage}

    \xiaoxiaoti{某等差数列 $\{a_n\}$ 的前 $n$ 项和的公式是 $S_n = 5n^2 + 3n$,求它的前 $3$ 项,并求它的通项公式。}

\end{xiaoxiaotis}


\xiaoti{一个屋顶的某一斜面成等腰梯形,最上面一层铺了瓦片 $21$ 块,往下每一层多铺 $1$ 块,
    斜面上铺了瓦片 $19$ 层,共铺瓦片多少块?}


\xiaoti{一个剧场设置了 $20$ 排座位,第一排有 $38$ 个座位,往后每一排都比前一排多 $2$ 个座位。
    这个剧场一共设置了多少个座位?}

\xiaoti{一个等差数列的第 $6$ 项是 $5$, 第 $3$ 项与第 $8$ 项的和也是 $5$。求这个等差数列前 $9$ 项的和。}

\xiaoti{三个数成等差数列,它们的和等于 $18$, 它们的平方和等于 $116$, 求这三个数。}

\xiaoti{某多边形的周长等于 $158cm$, 所有各边的长成等差数列,最大的边长等于 $44cm$ , 公差等于 $3cm$。求多边形的边数。}

\xiaoti{一个梯形两条底边的长分别是 $12cm$ 与 $22cm$,将梯形的一条腰 $10$ 等分,过每个分点画平行于梯形底边的直线,
    求这些直线夹在梯形两腰间的线段的长度的和。}

\end{xiaotis}




