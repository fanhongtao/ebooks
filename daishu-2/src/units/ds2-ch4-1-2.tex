\subsubsection{二元线性方程组的解的行列式表示法}

利用二阶行列式,我们也可以把公式 \eqref{eq:ejhls-5} 中的两个分子写成行列式的形式,即
$$
c_1b_2 - c_2b_1 = \begin{vmatrix}
    c_1 \quad & b_1 \\
    c_2 \quad & b_2
\end{vmatrix}, \qquad
a_1c_2 - a_2c_ = \begin{vmatrix}
    a_1 \quad & c_1 \\
    a_2 \quad & c_2
\end{vmatrix} \text{。}
$$
这样,当 $a_1b_2 - a_2b_1 \neq 0$ 时,二元线性方程组 \eqref{eq:fcz-1} 的解可以写成
\begin{equation}
    \begin{cases}
        x = \dfrac{
                \begin{vmatrix}
                    c_1 \quad & b_1 \\
                    c_2 \quad & b_2
                \end{vmatrix}
            }{
                \begin{vmatrix}
                    a_1 \quad & b_1 \\
                    a_2 \quad & b_2
                \end{vmatrix}
            } \text{,} \\
        \\
        y = \dfrac{
                \begin{vmatrix}
                    a_1 \quad & c_1 \\
                    a_2 \quad & c_2
                \end{vmatrix}
            }{
                \begin{vmatrix}
                    a_1 \quad & b_1 \\
                    a_2 \quad & b_2
                \end{vmatrix}
            } \text{。}
    \end{cases} \label{eq:ejhls-8}
\end{equation}

为了简便起见,通常用 $D$,$D_x$,$D_y$ 分别表示 \eqref{eq:ejhls-8} 式中作为分母与分子的行列式:
$$
D = \begin{vmatrix}
    a_1 \quad & b_1 \\
    a_2 \quad & b_2
\end{vmatrix}, \qquad
D_x = \begin{vmatrix}
    c_1 \quad & b_1 \\
    c_2 \quad & b_2
\end{vmatrix}, \qquad
D_y = \begin{vmatrix}
    a_1 \quad & c_1 \\
    a_2 \quad & c_2
\end{vmatrix} \text{。}
$$
行列式 $D$ 是由方程组 \eqref{eq:fcz-1} 中未知数 $x$,$y$ 的系数组成的,叫做这个方程组的\textbf{系数行列式} 。
$D$ 中第一列的元素 $a_1$,$a_2$(即 $x$ 的系数)分别换成方程组 \eqref{eq:fcz-1} 的常数项 $c_1$,$c_2$,就得到行列式 $D_x$;
$D$ 中第二列的元素 $b_1$,$b_2$(即 $y$ 的系数)分别换成常数项 $c_1$,$c_2$,就得到行列式 $D_y$。

于是,当 $D \neq 0$ 时,二元线性方程组 \eqref{eq:fcz-1} 的唯一解可以写成
\begin{equation*}
    \begin{cases}
        x = \dfrac{D_x}{D}, \\[1.5em]
        y = \dfrac{D_y}{D},
    \end{cases} \tag{9}\label{eq:ejhls-9}
\end{equation*}
也可以记为 $\left( \dfrac{D_x}{D}, \dfrac{D_y}{D} \right)$。
方程组 \eqref{eq:fcz-1} 的解集是 $\left\{\left( \dfrac{D_x}{D}, \dfrac{D_y}{D} \right)\right\}$。


\setcounter{cntliti}{1}
\liti 利用行列式解方程组
$$\begin{cases}
    11x - 2y + 5 = 0 \text{,} \\
    3x + 7y + 24 = 0 \text{。}
\end{cases}$$

\jie 先把方程组写成一般形式
$$\begin{cases}
    11x - 2y = -5 \text{,} \\
    3x + 7y = -24 \text{。}
\end{cases}$$

由
\begin{align*}
    D ={} & \begin{vmatrix}
            11  \quad & -2 \\
            3   \quad & 7
        \end{vmatrix} = 77 - (-6) = 83 \neq 0, \\
    D_x ={} & \begin{vmatrix}
            -5  \quad & -2 \\
            -24 \quad & 7
        \end{vmatrix} = -35 - 48 = -83, \\
    D_y ={} & \begin{vmatrix}
            11  \quad & -5 \\
            3   \quad & -24
        \end{vmatrix} = -264 - (-15) = -249, \\
\end{align*}
得
$$ \dfrac{D_x}{D} = \dfrac{-83}{83} = -1, \qquad \dfrac{D_y}{D} = \dfrac{-249}{83} = -3 \text{。} $$

所以方程组的解集是 $\{(-1, -3)\}$。


\lianxi

利用二阶行列式解下列方程组:

\begin{xiaoxiaotis}

    \twoInLineXxt[16em]{
        $\begin{cases}
            7x - 8y = 10,\\
            6x - 7y = 11;
        \end{cases}$
    }{
        $\begin{cases}
            14x - 6y + 1 = 0,\\
            3x + 7y - 6 = 0 \text{。}
        \end{cases}$
    }

\end{xiaoxiaotis}


