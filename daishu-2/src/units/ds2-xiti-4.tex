\xiti

\begin{xiaotis}

\xiaoti{已知数列 $\{a_n\}$ 的通项公式是 $a_n = \dfrac{3}{8} \times 2^n$。}
\begin{xiaoxiaotis}

    \xiaoxiaoti{计算 $\dfrac{a_2}{a_1}$,$\dfrac{a_3}{a_2}$,$\dfrac{a_4}{a_3}$,$\dfrac{a_5}{a_4}$;}

    \xiaoxiaoti{计算 $\dfrac{a_{n+1}}{a_n}$;}

    \xiaoxiaoti{这个数列是不是等比数列?它的首项与公比各是多少?}

\end{xiaoxiaotis}


\xiaoti{在等比数列 $\{a_n\}$ 中:}
\begin{xiaoxiaotis}

    \xiaoxiaoti{已知 $a_4 = 27$,$q = -3$,求 $a_7$;}

    \xiaoxiaoti{已知 $a_2 = 18$,$a_4 = 8$,求 $a_1$ 与 $q$;}

    \xiaoxiaoti{已知 $a_5 = 4$,$a_7 = 6$,求 $a_9$;}

    \xiaoxiaoti{已知 $a_5 - a_1 = 15$,$a_4 - a_2 = 6$,求 $a_3$。}

\end{xiaoxiaotis}


\xiaoti{求下列各题中两数的等比中项:}
\begin{xiaoxiaotis}

    \renewcommand\arraystretch{1.5}
    \begin{tabular}[t]{@{}p{15em}@{}p{20em}}
        \xiaoxiaoti{$45$ 与 $80$;} & \xiaoxiaoti{$9\dfrac{3}{8}$ 与 $1\dfrac{1}{2}$;} \\
        \xiaoxiaoti{$7 + 3\sqrt{5}$ 与 $7 - 3\sqrt{5}$;} & \xiaoxiaoti{$a^4 + a^2b^2$ 与 $b^4 + a^2b^2$ $(a \neq 0,\; b \neq 0)$。}
    \end{tabular}

\end{xiaoxiaotis}


\xiaoti{}
\begin{xiaoxiaotis}

    \vspace{-1.6em} \begin{minipage}{0.9\textwidth}
    \xiaoxiaoti{在 $9$ 与 $243$ 中间插入两个数,使它们同这两个数成等比数列;}
    \end{minipage}

    \xiaoxiaoti{在 $160$ 与 $5$ 中间插入 $4$ 个数,使它们同这两个数成等比数列。}

\end{xiaoxiaotis}


\xiaoti{某林场计划第一年造林 $80$ 亩,以后每年比前一年多造林 $20\%$ 。第五年造林多少亩(保留到个位)?}

\xiaoti{从盛满 $20$ 升纯酒精的容器里倒出 $1$ 升,然后用水填满,再倒出 $1$ 升混合溶液,用水填满,
    这样继续进行,一共倒了 $3$ 次,这时容器里还有多少升纯酒精(保留到个位)?}

\xiaoti{抽气机的活塞每运动 $1$ 次,从容器里抽出 $\dfrac{1}{8}$ 的空气,因而使容器里空气的压强降低为原来的 $\dfrac{7}{8}$ 。
    已知最初容器里空气的压强是 $760$ 毫米高水银柱,求活塞运动 $5$ 次后容器里空气的压强( 保留到个位)。}


\xiaoti{某种细菌在培养过程中,每 $30$ 分钟分裂一次〈一个分裂为两个), 经过 $4$ 小时,这种细菌由 $1$ 个可繁殖成多少个?}

\xiaoti{电动机轴的直径从小到大共有 $5$ 种尺寸,它们的数值(单位:mm) 组成一个等比数列,其中最小的数值为 $40$ ,
    最大的数值为 $100$ , 求它们的公比(保留到千分位)。}

\xiaoti{一个工厂今年生产某种机器 $1080$ 台,计划到后年,把产量提高到每年生产机器 $1920$ 台。
    如果每一年比上一年增长的百分率相同,这个百分率是多少(精确到 $1\%$ )?}


\xiaoti{在等比数列 $\{a_n\}$ 中:}
\begin{xiaoxiaotis}

    \xiaoxiaoti{已知 $a_1 = -1.5$,$a_4 = 96$,求 $q$ 与 $S_4$;}

    \xiaoxiaoti{已知 $q = \dfrac{1}{2}$,$S_5 = 3\dfrac{7}{8}$,求 $a_1$ 与 $a_5$;}

    \xiaoxiaoti{已知 $a_1 = 2$,$S_3 = 26$,求 $q$ 与 $a_3$;}

    \xiaoxiaoti{已知 $a_3 = 1\dfrac{1}{2}$,$S_3 = 4\dfrac{1}{2}$,求 $a_1$ 与 $q$。}

\end{xiaoxiaotis}


\xiaoti{某工厂去年的产值是 $138$ 万元,计划在今后 $5$ 年内每年比上一年产值增长 $10\%$。从今年起,
    到第 $5$ 年这个工厂的年产值是多少?这 $5$ 年的总产值是多少(精确到万元)?}


\xiaoti{画一个边长 $2$ 厘米的正方形, 再以这个正方形的对角线为边画第 $2$ 个正方形,
    以第 $2$ 个正方形的对角线为边画第 $3$ 个正方形,这样一共画了 $10$ 个正方形。求:}
\begin{xiaoxiaotis}

    \xiaoxiaoti{第 $10$ 个正方形的面积;}

    \xiaoxiaoti{这 $10$ 个正方形的面积的和。}

\end{xiaoxiaotis}


\xiaoti{一个球从 $100$ 米高处自落下,每次着地后又跳回到原高度的一半再落下。当它第 $10$ 次着地时,共经过了多少米(保留到个位)?}


\xiaoti{求和:}
\begin{xiaoxiaotis}

    \xiaoxiaoti{$(a - 1) + (a^2 - 2) + (a^3 - 3) + \cdots + (a^n - n)$;}

    \xiaoxiaoti{$\left( x + \dfrac{1}{y} \right) + \left( x^2 + \dfrac{1}{y^2} \right) + \left( x^3 + \dfrac{1}{y^3} \right) + \cdots + \left( x^n + \dfrac{1}{y^n} \right)$。}

\end{xiaoxiaotis}


\xiaoti{三个数成等比数列,它们的和等于 $14$,它们的积等于 $64$,求这三个数。}


\xiaoti{设等比数列 $a_1$,$a_2$,$\cdots$,$a_n$ 的公比是 $q$,求证
    $$ a_1 a_2 \cdots a_n = a_1^n q^\frac{n(n - 1)}{2} \text{。} $$
}


\xiaoti{一个等比数列的各项都是正数,求证这个数列的各项的对数组成等差数列。}


\xiaoti{已知 $a_1$,$a_2$,$a_3$,$\cdots$ 是等差数列,$C$ 是正的常数,求证
    $C^{a_1}$,$C^{a_2}$,$C^{a_3}$,$\cdots$ 是等比数列。}


\xiaoti{已知无穷数列 $10^\frac{0}{10}$,$10^\frac{1}{10}$,$10^\frac{2}{10}$,$\cdots$,$10^\frac{n-1}{10}$,$\cdots$,求证:}
\begin{xiaoxiaotis}

    \xiaoxiaoti{这个数列是以 $10^\frac{1}{10}$ 为公比的等比数列;}

    \xiaoxiaoti{这个数列中的任意一项是它后面第 $10$ 项的 $\dfrac{1}{10}$;}

    \xiaoxiaoti{这个数列中的任意两项的积仍然在这个数列中。}
\end{xiaoxiaotis}

\end{xiaotis}

