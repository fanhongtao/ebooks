\subsection{不等式}\label{subsec:3-1} % 原书的目录结构就是如此(缺少 section),所以忽略此处的告警

我们已经学过一些简单的不等式,例如:

\begin{align}
    a + 2 > a + 1, \tag{$1$} \\
    a^2 + 3 > 3a, \tag{$2$} \\
    3x + 1 < 2x + 6, \tag{$3$} \\
    x^2 < a \tag{$4$} \text{。}
\end{align}

在两个不等式中,
如果每一个的左边都大于右边,如 (1) 和 (2),
或者每一个的左边都小于右边,如 (3) 和 (4),那么这样的两个不等式就是同向不等式。
如果一个不等式的左边大于右边,
而另一个不等式的左边小于右边,如 (1) 和 (3),那么这两个不等式就是异向不等式。

我们知道,实数可以比较大小。在数轴上,两个不同的点 $A$ 与 $B$ 分别表示两个不同的实数 $a$ 与 $b$,
右边的点表示的数比左边的点表示的数大。从实数减法在数轴上的表示可以看出,$a$,$b$ 之间具有以下性质:

\textbf{
    如果 $a - b$ 是正数,那么 $a > b$;
    如果 $a - b$ 是负数,那么 $a < b$;
    如果 $a - b$ 等于零,那么 $a = b$。
反过来也对。}这就是说:

\begin{align*}
    a - b > 0 &\iff a > b \; ; \\
    a - b = 0 &\iff a = b \; ; \\
    a - b < 0 &\iff a < b \; \text{。} \\
\end{align*}

由此可见,要比较两个实数的大小,只要考察它们的差就可以了。

\liti 比较 $(x+1)(x+2)$ 与 $(x-3)(x+6)$ 的大小。

\jie $\because \begin{aligned}[t]
         & (x+1)(x+2) - (x-3)(x+6) \\
    ={} & (x^2 + 3x + 2) - (x^2 + 3x -18) \\
    ={} & 20 > 0,
\end{aligned}$

$\therefore \quad (x+1)(x+2) > (x-3)(x+6)$。


\liti 已知 $x \neq 0$,比较 $(x^2 + 1)^2$ 与 $x^4 + x^2 + 1$ 的大小。

\jie $\begin{aligned}[t]
         & (x^2 + 1)^2 - (x^4 + x^2 + 1) &\\
    ={} & x^4 + 2x^2 + 1 - x^4 -x^2 -1 & \\
    ={} & x^2 \text{。}
\end{aligned}$

由 $x \neq 0$,得 $x^2 > 0$。从而
$$ (x^2 + 1)^2 > x^4 + x^2 + 1 \text{。} $$


\lianxi

\begin{xiaotis}

\xiaoti{比较 $(x+5)(x+7)$ 与 $(x+6)^2$ 的大小。}

\xiaoti{已经 $a \neq 0$,比较 $(a^2 + \sqrt{2}a + 1)(a^2 -\sqrt{2}a + 1)$ 与 $(a^2 + a + 1)(a^2 - a + 1)$ 的大小。}

\xiaoti{比较 $\left( \dfrac{n}{\sqrt{6}} + 1 \right)^3 - \left( \dfrac{n}{\sqrt{6}} - 1 \right)^3$ 与 $2$ 的大小 $(n \neq 0)$ 。}

\end{xiaotis}

