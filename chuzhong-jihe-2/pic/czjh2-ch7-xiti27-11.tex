\begin{tikzpicture}[scale=.7]
    \pgfmathsetmacro{\r}{0.2}

    % 设小圆的圆心为 O, OD 垂直于 O_1O_2, 垂足为 D。
    % 则,在直角三角形 DOO_1
    % OO1 = R + r
    % 因为:DO_1O = 30 度,所以, OD = OO1/2 = (R + r)/2
    %   R^ + [(R + r)/2]^2 = (R + r)^2
    % 化简得
    %   R^2 - 6Rr - 3r^2 = 0
    % 代入 r = 0.2 得
    %   R = 1.29
    \pgfmathsetmacro{\R}{1.29}

    \tkzDefPoints{0/0/O}
    \tkzDefPoint(210:\R+\r){O_1}
    \tkzDefPoint(330:\R+\r){O_2}
    \tkzDefPoint(90:\R+\r){O_3}

    \tkzDefCircle[R](O,\r)  \tkzGetPoint{o}
    \tkzDrawCircle[very thick](O, o)
    \foreach \n in {1,2,3} {
        \tkzDefCircle[R](O_\n,\R)  \tkzGetPoint{o_\n}
        \tkzDrawCircle[very thick](O_\n, o_\n)
    }
    \tkzDrawPolygon[dashed](O_1,O_2,O_3)
    \tkzDrawSegments[dashed](O,O_1)

    \tkzLabelPoints[left](O_1)
    \tkzLabelPoints[right](O_2)
    \tkzLabelPoints[above](O_3)
\end{tikzpicture}

