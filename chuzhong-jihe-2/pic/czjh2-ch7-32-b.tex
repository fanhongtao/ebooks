\begin{tikzpicture}
    \tkzDefPoints{0/0/O}
    \tkzDefPoint(0:1.5){A}
    \tkzDefPoint(270:1.5){P}
    \tkzDefLine[perpendicular=through P,normed](O,P)  \tkzGetPoint{q}
    \tkzDefPointOnLine[pos=2.0](P,q)  \tkzGetPoint{Q}

    \tkzDrawCircle[thick](O,A)
    \tkzDrawPoint(O)
    \tkzDrawSegment(O,P)
    \tkzDrawLine[add=0 and 1](Q,P)
    \tkzMarkRightAngle[size=.2](O,P,Q)
    \tkzLabelPoints[above](O)
    \tkzLabelPoints[below](P)
    \tkzLabelSegment[pos=1, right](P,Q){$l$}

    % 通过显示一个无色的 "P",实现三张图片的 “圆点” 在同一水平线。
    \tkzDefPoint(270:1.8){P}
    \tkzLabelPoints[below,transparent](P)
\end{tikzpicture}

