\begin{enhancedline}

我们早就学过:“$\text{圆周长} = \text{直径} \times \text{圆周率}\;\pi$”。
可是,圆周长与直径的比值,即圆周率 $\pi$ 的值是怎样计算出来的呢?

在半径为 $R$ 的圆中(图 \ref{fig:czjh2-7-86}),内接正六边形的周长是 $p_6 = 6R$,
圆内接正六边形的周长与圆的直径的比是 $\dfrac{6R}{2R} = 3$,
这个比值与 $R$ 无关,也就是说,不管圆的大小怎样,它是一个常数。

\begin{figure}[htbp]
    \centering
    \begin{minipage}[b]{7cm}
        \centering
        \begin{tikzpicture}
    \pgfmathsetmacro{\R}{1.5}
    \tkzDefPoints{0/0/O}
    \tkzDefPoint(60:\R){A}
    \tkzDrawCircle[very thick](O,A)
    \tkzDrawSegment(O,A)
    \tkzDrawPoint(O)
    \tkzLabelSegment[right](O,A){$R$}
    \tkzLabelPoints[left](O)

    \tkzDefRegPolygon[center,sides=6,name=P](O,A)
    \tkzDrawPolygon[thick,red](P1,P...,P6)

    \tkzDefRegPolygon[center,sides=12,name=P](O,A)
    \tkzDrawPolygon[thick,blue](P1,P...,P12)
\end{tikzpicture}


        \caption{}\label{fig:czjh2-7-86}
    \end{minipage}
    \qquad
    \begin{minipage}[b]{7cm}
        \centering
        \begin{tikzpicture}
    \pgfmathsetmacro{\R}{1.5}
    \tkzDefPoints{0/0/O}
    \tkzDefPoint(230:\R){A}
    \tkzDefPoint(310:\R){B}
    \tkzDefLine[altitude](A,O,B)  \tkzGetPoint{D}
    \tkzInterLC(O,D)(O,A)  \tkzGetSecondPoint{C}

    \tkzDrawCircle[very thick](O,A)
    \tkzDrawSegments(A,B  O,C  A,C)
    \tkzDrawSegments[dashed](O,A  O,B)
    \tkzDrawPoint(O)
    \tkzMarkRightAngle[size=.2](B,D,O)
    \tkzLabelSegment[above, xshift=-.2em](O,A){$R$}
    \tkzLabelPoints[above](O)
    \tkzLabelPoints[left=.2em](A)
    \tkzLabelPoints[right=.2em](B)
    \tkzLabelPoints[below](C)
    \tkzLabelPoints[above, xshift=.5em, yshift=.3em](D)
\end{tikzpicture}


        \caption{}\label{fig:czjh2-7-87}
    \end{minipage}
\end{figure}

如果把圆内接正 6 边形的周长看作是圆的周长的近似值,
把圆内接正 6 边形的周长与直径的比(等于3) 看作是圆的周长与直径的比的近似值,
当然,误差是很大的。

把圆内接正 6 边形的边数加倍,可以得到圆内接正 12 边形;
再加倍,可以得到圆内接正 24 边形; ……。
我们可以把这样一些圆内接正多边形的周长看作是圆的周长的近似值,
把这些圆内接正多边形的周长与直径的比作为圆的周长与直径的比的近似值。
当圆内接正多边形的边数不断地成倍增加时,它们的周长 $p_n$ 不断地增大,越来越接近于圆的周长, % 书中原文为 P_n,但结合上下文,写为 p_n 更合适。下一行也是这样。
圆内接正多边形的周长 $p_n$ 和直径 $2R$ 的比值就越来越接近于圆周长 $C$ 和直径 $2R$ 的比值,
误差越来越小,只要边数 $n$ 充分大,误差可以任意地小。

为了说明,我们先证明下列 “倍边公式”。

设 $\yuan\,O$ 的半径为 $R$ (图 \ref{fig:czjh2-7-87}),
圆内接正 $n$ 边形及正 $2n$ 边形的边长分别为 $AB = a_n$ 及 $AC = a_{2n}$。
因为半径 $OC$ 垂直平分 $AB$,由勾股定理可知,

$\begin{aligned}
    a_{2n}^2 &= AC^2 = AD^2 + CD^2 \\
             &= AD^2 + (OC - OD)^2 \\
             &= AD^2 + (OC - \sqrt{OA^2 - AD^2})^2 \\
             &= AD^2 + OC^2 - 2 OC \sqrt{OA^2 - AD^2} + OA^2 - AD^2 \\
             &= OC^2 + OA^2 - 2 OC \sqrt{OA^2 - AD^2} \\
             &= 2R^2 - 2R \sqrt{R^2 - \exdfrac{1}{4} a_n^2} \juhao
\end{aligned}$

$\therefore$ \quad $a_{2n} = \sqrt{2R^2 - R \sqrt{4R^2 - a_n^2}}$。

由于 $a_6 = R$,依据这个公式,就可依次计算得
\begin{align*}
    & a_{12} = \sqrt{2 - \sqrt{3\,}} \, R \fenhao \\
    & a_{24} = \sqrt{2 - \sqrt{2 + \sqrt{3\,}}} \, R \fenhao \\
    & a_{48} = \sqrt{2 - \sqrt{2 + \sqrt{2 + \sqrt{3\,}}}} \, R \fenhao \\
    & \cdots \cdots
\end{align*}
利用这个等式,半径为 $R$ 的圆内接正 6、12、24、… 边形的边长、周长以及周长与直径的比,
就都可以计算出来(如下表)。% 如下页表

可以看出,每一个圆内接正 $n$ 边形的周长和直径的比 $\left(\dfrac{p_n}{2R}\right)$
都是与半径 $R$ 无关的常数,所以,圆的周长和直径的比 $\left(\dfrac{C}{2R}\right)$
也是一个与 $R$ 无关的常数,这个常数就是 $\pi$。


\begin{tblr}{hlines, vlines,
    columns={c},
}
    边数$n$ & 边长 $a_n$ & 周长 $p_n$ & 周长与直径的比 $\dfrac{p_n}{2R}$ \\
    6   & 1.00000000 $R$ & 6.00000000 $R$ & 3.00000000 \\
    12  & 0.51763809 $R$ & 6.21165708 $R$ & 3.10582854 \\
    24  & 0.26105238 $R$ & 6.26525722 $R$ & 3.13262861 \\
    48  & 0.13080626 $R$ & 6.27870041 $R$ & 3.13935021 \\
    96  & 0.06543817 $R$ & 6.28206396 $R$ & 3.14103198 \\
    192 & 0.03272316 $R$ & 6.28290510 $R$ & 3.14145255 \\
    384 & 0.01636228 $R$ & 6.28311544 $R$ & 3.14155772 \\
    768 & 0.00818121 $R$ & 6.28316941 $R$ & 3.14158471 \\
\end{tblr}


这样,我们就得到了一种计算圆周率 $\pi$ 的近似值的方法。

我国古代数学家祖冲之,在公元五世纪就已算得 $\pi$ 的值在 3.1415926 与 3.1415927 之间,
比其他国家早一千年左右。现代利用电子计算机,已有人把 $\pi$ 的值算到小数点后几十万位。
$\pi$ 是一个无限不循环小数,就是说,是一个无理数。

\begin{wrapfigure}[10]{r}{6cm}
    \centering
    \begin{tikzpicture}
    \pgfmathsetmacro{\R}{2}
    \tkzDefPoints{0/0/O}
    \tkzDefPoint(60:\R){A}
    \tkzDrawCircle[very thick](O,A)
    \tkzDrawSegment(O,A)
    \tkzLabelSegment[left](O,A){$R$}
    \tkzLabelPoints[left](O)

    \tkzDefRegPolygon[center,sides=6,name=P](O,A)
    \tkzDrawPolygon[thick,red](P1,P...,P6)
    \tkzDefMidPoint(P4,P5)  \tkzGetPoint{H}
    \tkzDrawSegment(O,H)
    \tkzLabelSegment[right](O,H){$r_n$}

    \tkzDefRegPolygon[center,sides=12,name=P](O,A)
    \tkzDrawPolygon[thick,blue](P1,P...,P12)
\end{tikzpicture}


    \caption{}\label{fig:czjh2-7-88}
\end{wrapfigure}

由于 $\dfrac{C}{2R} = \pi$, 就可得到圆周长的计算公式
$$ C = 2 \pi R \juhao $$

此外,我们知道,如图 \ref{fig:czjh2-7-88}, 边心距为 $r_n$、周长为 $p_n$ 的正 $n$ 边形的面积 $S_n$ 等于 $\exdfrac{1}{2} r_n p_n$ 。

在半径为 $R$ 的圆中,当内接正多边形的边数不断地成倍增加时,
正多边形的边心距 $r_n$ 越来越接近于圆的半径 $R$,
正多边形的周长 $p_n$ 越来越接近于圆周长 $C$,
而正多边形的面积 $S_n$ 就越来越接近于圆面积 $S$。
这样,从正 $n$ 边形的面积公式 $S_n = \exdfrac{1}{2} r_n p_n$ 就可以得到圆面积公式
$$ S = \exdfrac{1}{2} R \cdot C = \exdfrac{1}{2} R \cdot 2 \pi R = \pi R^2 \juhao $$

\end{enhancedline}


