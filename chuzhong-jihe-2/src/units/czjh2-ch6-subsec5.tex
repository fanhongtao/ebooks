\subsection{三角形角平分线的性质}\label{subsec:czjh2-6-5}
\begin{enhancedline}

\begin{dingli}[三角形内角平分线性质定理]
    三角形的内角平分线分对边所得的两条线段和这个角的两边对应成比例。
\end{dingli}

已知:$\triangle ABC$ 中,$AD$ 是角平分线(图 \ref{fig:czjh2-6-12})。

求证:$\dfrac{BD}{DC} = \dfrac{AB}{AC}$。

分析:在比例式 $\dfrac{BD}{DC} = \dfrac{AB}{AC}$ 中,$AC$ 是 $BD$、$DC$、$AB$ 的第四比例项。
从图 \ref{fig:czjh2-6-12} 中又可看出,如果过点 $C$ 作 $CE \pingxing AD$,交 $BA$ 的延长线于 $E$,
就可以得到 $BD$、$DC$、$BA$ 的第四比例项 $AE$,要证明 $\dfrac{BD}{DC} = \dfrac{AB}{AC}$,
只要证明 $AC = AE$ 即可。

\zhengming 过点 $C$ 作 $CE \pingxing DA$,交 $BA$ 的延长线于 $E$。

$\left. \begin{aligned}
    \left. \begin{aligned}
        & \quad \angle 1 = \angle 2 \\
        CE \pingxing DA  \tuichu & \left\{ \begin{aligned}
            \angle 1 = \angle E \\
            \angle 2 = \angle 3
        \end{aligned} \right.
    \end{aligned} \right\} \tuichu  \angle E = \angle 3  \tuichu AE = AC \\
    CE \pingxing DA  \tuichu  \dfrac{BD}{DC} = \dfrac{BA}{AE}
\end{aligned} \right\} \tuichu \dfrac{BD}{DC} = \dfrac{AB}{AC} \juhao
$

\begin{figure}[htbp]
    \centering
    \begin{minipage}[b]{4.5cm}
        \centering
        \begin{tikzpicture}
    \tkzDefPoints{0/0/B, 3/0/C, 2/1.5/A}
    \tkzDefLine[bisector](B,A,C)  \tkzGetPoint{d}
    \tkzInterLL(A,d)(B,C)  \tkzGetPoint{D}
    \tkzDefLine[parallel=through C](A,D)  \tkzGetPoint{e}
    \tkzInterLL(C,e)(B,A)  \tkzGetPoint{E}

    \tkzDrawPolygon(A,B,C)
    \tkzDrawSegment(A,D)
    \tkzDrawSegments[dashed](A,E  E,C)
    \extkzLabelAngel[0.5](B,A,D){$1$}
    \extkzLabelAngel[0.6](D,A,C){$2$}
    \extkzLabelAngel[0.6](E,C,A){$3$}
    \tkzLabelPoints[above](A,E)
    \tkzLabelPoints[left](B)
    \tkzLabelPoints[right](C)
    \tkzLabelPoints[below](D)
\end{tikzpicture}


        \caption{}\label{fig:czjh2-6-12}
    \end{minipage}
    \qquad
    \begin{minipage}[b]{10.5cm}
        \centering
        \begin{minipage}[b]{4.3cm}
            \centering
            \begin{tikzpicture}
    \tkzDefPoints{0/0/A, 1/0/C, 3/0/B}
    \tkzDrawSegments[xianduan={below=0pt}](A,C)
    \tkzDrawSegments[xianduan={below=0pt}](C,B)
    \tkzDrawLine[add=0.2 and 0.2](A,B)
    \tkzLabelPoints[above=.5em](A,B,C)
\end{tikzpicture}


            \caption*{甲}
        \end{minipage}
        \qquad
        \begin{minipage}[b]{4.3cm}
            \centering
            \begin{tikzpicture}
    \tkzDefPoints{0/0/A, 2/0/B, -0.5/0/E, 2.5/0/D}
    \tkzDrawSegments[xianduan={below=0pt}](E,A)
    \tkzDrawSegments[xianduan={below=0pt}](B,D)
    \tkzDrawLine[add=0.2 and 0.2](E,D)
    \tkzLabelPoints[above=.5em](A,B,D,E)
\end{tikzpicture}


            \caption*{乙}
        \end{minipage}
        \caption{}\label{fig:czjh2-6-13}
    \end{minipage}
\end{figure}

在一条线段上的一个点,将线段分成两条线段,这个点叫做这条线段的\zhongdian{内分点}。
如图 \ref{fig:czjh2-6-13} 甲中,点 $C$ 是线段 $AB$ 的内分点,这时,点 $C$ 内分线段 $AB$ 成两条线段 $AC$、$BC$。
在一条线段的延长线上的点,有时也叫做这条线段的\zhongdian{外分点}。
外分点分线段所得的两条线段也是这个点分别和线段的两个端点确定的线段。
如图 \ref{fig:czjh2-6-13} 乙中,点 $D$($E$)是线段 $AB$ 的外分点,外分线段 $AB$ 成两条线段 $AD$、$BD$($AE$、$BE$)。

根据内分点定义,三角形内角平分线性质也可以说成
“三角形内角平分线内分对边所成的两条线段和相邻两边对应成比例”。

用类似的方法可以得到三角形外角平分线性质定理:

\begin{dingli}[三角形外角平分线性质定理]
    如果三角形的外角平分线外分对边成两条线段,那么这两条线段和相邻的两边对应成比例。
\end{dingli}

已知: $\triangle ABC$ 中, $AD$ 是外角平分线,交 $BC$ 的延长线于点 $D$( 图 \ref{fig:czjh2-6-14})。

求证: $\dfrac{BD}{CD} = \dfrac{AB}{AC}$。

同学可以根据图形自己写出证明。

\begin{figure}[htbp]
    \centering
    \begin{minipage}[b]{7cm}
        \centering
        \begin{tikzpicture}
    \tkzDefPoints{0/0/B, 2/0/C, 1.8/1.2/A}
    \tkzDefLine[bisector out](B,A,C)  \tkzGetPoint{d}
    \tkzInterLL(A,d)(B,C)  \tkzGetPoint{D}
    \tkzDefLine[parallel=through C](A,D)  \tkzGetPoint{e}
    \tkzInterLL(C,e)(B,A)  \tkzGetPoint{E}
    \tkzDefPointOnLine[pos=1.5](B,A)  \tkzGetPoint{X}

    \tkzDrawPolygon(A,B,C)
    \tkzDrawSegments(A,D A,X)
    \tkzDrawLine[add=0 and 0.2](C,D)
    \tkzDrawSegments[dashed](C,E)
    \extkzLabelAngel[0.3](D,A,X){$1$}
    \extkzLabelAngel[0.4](C,A,D){$2$}
    %\extkzLabelAngel[0.3](A,C,E){$3$}
    %\extkzLabelAngel[0.3](C,E,A){$4$}
    \tkzMarkAngle[size=0.3](A,C,E)
    \tkzLabelAngle[pos=0.5](A,C,E){$3$}
    \tkzMarkAngle[size=0.3](C,E,A)
    \tkzLabelAngle[pos=0.5](C,E,A){$4$}

    \tkzLabelPoints[above](A,E)
    \tkzLabelPoints[left](B)
    \tkzLabelPoints[below](C,D)
\end{tikzpicture}


        \caption{}\label{fig:czjh2-6-14}
    \end{minipage}
    \qquad
    \begin{minipage}[b]{7cm}
        \centering
        \begin{tikzpicture}
    \tkzDefPoints{0/0/B, 2/0/C, 1.8/1.2/A}
    \tkzDefLine[bisector](B,A,C)  \tkzGetPoint{d}
    \tkzInterLL(A,d)(B,C)  \tkzGetPoint{D}
    \tkzDefLine[bisector out](B,A,C)  \tkzGetPoint{e}
    \tkzInterLL(A,e)(B,C)  \tkzGetPoint{E}
    \tkzDefPointOnLine[pos=1.5](B,A)  \tkzGetPoint{X}

    \tkzDrawPolygon(A,B,C)
    \tkzDrawSegments(A,D A,E A,X)
    \tkzDrawLine[add=0 and 0.2](C,E)
    \extkzLabelAngel[0.5](B,A,D){$1$}
    \extkzLabelAngel[0.4](D,A,C){$2$}
    \extkzLabelAngel[0.5](C,A,E){$3$}
    \extkzLabelAngel[0.4](E,A,X){$4$}

    \tkzLabelPoints[above](A)
    \tkzLabelPoints[left](B)
    \tkzLabelPoints[below](C,D,E)
\end{tikzpicture}


        \caption{}\label{fig:czjh2-6-15}
    \end{minipage}
\end{figure}

\liti[0] 已知:如图 \ref{fig:czjh2-6-15}, $AD$ 和 $AE$ 分别是 $\triangle ABC$ 的内角平分线和外角平分线。
求证: $\dfrac{BD}{CD} = \dfrac{BE}{CE}$。

$\left. \begin{aligned}
    \text{\zhengming} \angle 1 = \angle 2  & \tuichu  \dfrac{AB}{AC} = \dfrac{BD}{CD} \\
                      \angle 3 = \angle 4  & \tuichu  \dfrac{AB}{AC} = \dfrac{BE}{CE}
\end{aligned} \right\}  \tuichu  \dfrac{BD}{CD} = \dfrac{BE}{CE} \juhao$


\begin{lianxi}

\xiaoti{}%
\begin{xiaoxiaotis}%
    (口答)\xxt[\xxtsep]{等腰三角形顶角平分线分底边所得两条线段的比值是多少?}

    \xxt{三角形外角平分线性质定理为什么在 “三角形的外角平分线分对边成两条线段” 之前一定要加如果二字?}

\end{xiaoxiaotis}


\xiaoti{总结本节两个定理的证明中作铺助线的方法,即怎样根据比例作辅助线(平行线)。}

\xiaoti{已知: $\triangle ABC$ 中, $AD$ 是角平分线, $AB = 5\;\limi$, $AC = 4\;\limi$,
    $BC = 7\;\limi$。求 $BD$、$DC$ 的长。
}

\end{lianxi}
\end{enhancedline}

