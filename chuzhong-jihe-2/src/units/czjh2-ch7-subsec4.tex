\subsection{圆心角、弧、弦、弦心距之间的关系}\label{subsec:czjh2-7-4}

如图 \ref{fig:czjh2-7-16} 甲 ,在 $\yuan\,O$ 上任取一点 $A$,作直径 $AB$,则 $OA = OB$。
就是说,点 $B$ 是点 $A$ 关于点 $O$ 的对称点。
因此,\zhongdian{圆是以圆心为对称中心的中心对称图形。}

\begin{figure}[htbp]
    \centering
    \begin{minipage}[b]{6cm}
        \centering
        \begin{tikzpicture}
    \tkzDefPoints{0/0/O}
    \tkzDefPoint(155:1.5){A}
    \tkzDefPoint(335:1.5){B}

    \tkzDrawCircle[thick](O,A)
    \tkzDrawSegments(O,A  O,B)
    \tkzDrawPoint(O)
    \tkzLabelPoints[above](O)
    \tkzLabelPoints[left](A)
    \tkzLabelPoints[right](B)
\end{tikzpicture}


        \caption*{甲}
    \end{minipage}
    \qquad
    \begin{minipage}[b]{6cm}
        \centering
        \begin{tikzpicture}
    \tkzDefPoints{0/0/O}
    \tkzDefPoint(0:1.5){A}
    \tkzDefPoint(40:1.5){B}

    \tkzDrawCircle[thick](O,A)
    \tkzDrawSegments(O,A  O,B)
    \tkzMarkAngle[size=.6, -latex](A,O,B)
    \tkzLabelAngle[pos=.8](A,O,B){$\alpha$}
    \tkzDrawPoint(O)
    \tkzLabelPoints[left](O)
    \tkzLabelPoints[right](A,B)
\end{tikzpicture}


        \caption*{乙}
    \end{minipage}
    \caption{}\label{fig:czjh2-7-16}
\end{figure}

圆不仅是中心对称图形:绕圆心旋转 $180^\circ$ 后能够与原来的图形重合,并且它还有另外一个重要性质。
如图 \ref{fig:czjh2-7-16} 乙中,让圆绕中心 $O$ 旋转任意一个角度 $\alpha$, 圆上任意一点 $A$ 都能够与圆上一点 $B$ 重合。
因此,圆绕圆心旋转任意一个角度,都能够与原来的图形重合。
利用这个性质,我们还可以推出圆的其他一些性质。

顶点在圆心的角叫做\zhongdian{圆心角}。 从圆心到弦的距离叫做\zhongdian{弦心距}。
现在用上面的性质来研究在同一个圆中,圆心角、圆心角所对的弦、弧、弦心距相互之间的关系。

如图 \ref{fig:czjh2-7-17} $\yuan\,O$ 中,当圆心角 $\angle AOB = \angle A'OB'$ 时,
它们所对的弧 $\yuanhu{AB}$ 和 $\yuanhu{A'B'}$、弦 $AB$ 和 $A'B'$、
弦心距 $OM$ 和 $OM'$ 是否也相等呢?

我们把 $\angle AOB$ 连同 $\yuanhu{AB}$ 绕圆心 $O$ 旋转,使射线 $OA$ 与 $OA'$ 重合。

$\because$ \quad $\angle AOB = \angle A'OB'$,

$\therefore$ \quad  射线 $OB$ 与 $OB'$ 重合。

又$\because$ \quad  $OA = OA'$, $OB = OB'$,

$\therefore$ \quad 点 $A$ 和点 $A'$ 重合,点 $B$ 与点 $B'$ 重合。

这样,$\yuanhu{AB}$ 与 $\yuanhu{A'B'}$ 重合, $AB$ 与 $A'B'$ 重合,
从点 $O$ 到 $AB$ 的垂线段 $OM$ 和点 $O$ 到 $A'B'$ 的垂线段 $OM'$ 也重合。即
$$ \yuanhu{AB} = \yuanhu{A'B'} \douhao  AB = A'B' \douhao  OM = OM' \juhao $$

\begin{figure}[htbp]
    \centering
    \begin{minipage}[b]{4cm}
        \centering
        \begin{tikzpicture}
    \tkzDefPoints{0/0/O}
    \tkzDefPoint(0:1.5){A}
    \tkzDefPoint(50:1.5){B}
    \tkzDefPoint(80:1.5){A'}
    \tkzDefPoint(130:1.5){B'}
    \tkzDefLine[altitude](A,O,B)  \tkzGetPoint{M}
    \tkzDefLine[altitude](A',O,B')  \tkzGetPoint{M'}

    \tkzDrawCircle[thick](O,A)
    \tkzDrawSegments     (O,A   O,B   A,B    O,M)
    \tkzDrawSegments[red](O,A'  O,B'  A',B'  O,M')
    \tkzMarkAngle[size=.6, -latex](A,O,A')
    \tkzDrawPoint(O)
    \tkzLabelPoints[below left](O)
    \tkzLabelPoints[right](A)
    \tkzLabelPoints[above right](B)
    \tkzLabelPoints[above](A')
    \tkzLabelPoints[above](B')
    \tkzLabelPoints[below, xshift=-.4em](M)
    \tkzLabelPoints[below, xshift=-.3em](M')
\end{tikzpicture}


        \caption{}\label{fig:czjh2-7-17}
    \end{minipage}
    \qquad
    \begin{minipage}[b]{6cm}
        \centering
        \begin{tikzpicture}
    \tkzDefPoints{0/0/O}
    \tkzDefPoint(-40:1.5){A}
    \tkzDefPoint(20:1.5){B}
    \tkzDefPoint(50:1.5){C}
    \tkzDefPoint(51:1.5){D}

    \tkzDrawCircle[thick](O,A)
    \tkzDrawSegments(O,A  O,B  O,C  O,D)
    \tkzMarkAngle[size=.6](A,O,B)
    \tkzDrawPoint(O)
    \tkzLabelPoints[left](O)
    \tkzLabelPoints[right](A,B)

    %
    \tkzDefPoint(-20:0.6){e1}  % End point
    \tkzDefPoint(-20:1.8){b1}  % Begin point
    \tkzDrawSegment[-latex](b1,e1)
    \tkzLabelSegment[pos=0, right](b1,e1){$n^\circ$圆心角}

    \tkzDefPoint(5:1.5){e2}
    \tkzDefPoint(5:1.8){b2}
    \tkzDrawSegment[-latex](b2,e2)
    \tkzLabelSegment[pos=0, right](b2,e2){$n^\circ$弧}

    \tkzDefPoint(50.5:1.2){e3}
    \tkzDefPoint(30:1.8){b3}
    \tkzDrawSegment[-latex](b3,e3)
    \tkzLabelSegment[pos=0, right](b3,e3){$1^\circ$圆心角}

    \tkzDefPoint(50.5:1.5){e4}
    \tkzDefPoint(45:1.8){b4}
    \tkzDrawSegment[-latex](b4,e4)
    \tkzLabelSegment[pos=0, right, yshift=.3em](b4,e4){$1^\circ$弧}
\end{tikzpicture}


        \caption{}\label{fig:czjh2-7-18}
    \end{minipage}
    \qquad
    \begin{minipage}[b]{4.5cm}
        \centering
        \begin{tikzpicture}
    \tkzDefPoints{0/0/O}
    \tkzDefPoint(285:1.5){A}
    \tkzDefPoint(105:1.5){B}
    \tkzDefPoint(230:1.5){D}
    \tkzDefPoint(50:1.5){E}
    \tkzDefLine[parallel=through A](D,E)  \tkzGetPoint{c}
    \tkzInterLC[common=A](A,c)(O,A)  \tkzGetFirstPoint{C}

    \tkzDrawCircle[thick](O,A)
    \tkzDrawSegments(A,B  D,E)
    \tkzDrawSegments(A,C  C,E  E,B)
    \tkzDrawPoint(O)
    \tkzLabelPoints[left](O)
    \tkzLabelPoints[below](A,D)
    \tkzLabelPoints[above](B,E)
    \tkzLabelPoints[right](C)
\end{tikzpicture}


        \caption{}\label{fig:czjh2-7-19}
    \end{minipage}
\end{figure}

上面的结论,在两个等圆中也成立。于是有下面的定理:

\begin{dingli}[定理]
    在同圆或等圆中,相等的圆心角所对的弧相等,所对的弦相等,所对的弦的弦心距相等。
\end{dingli}

由上面的定理,可以得到下面推论:

\begin{tuilun}[推论]
    在同圆或等圆中,如果两个圆心角、两条弧、两条弦或两条弦的弦心距中有一组量相等,
    那么它们所对应的其余各组量都分别相等。
\end{tuilun}

我们知道,把顶点在圆心的周角等分成 360 份时,每一份的圆心角是 $1^\circ$ 的角。
因为同圆中相等的圆心角所对的弧相等,所以整个圆也被等分成 360 份。
我们把每一份这样的弧叫做 \zhongdian{$\bm{1^\circ}$ 的弧}。

由上述定义可知, $1^\circ$ 的圆心角对着 $1^\circ$ 的弧, $1^\circ$ 的弧对着 $1^\circ$ 的圆心角。
一般地, $n^\circ$ 的圆心角对着 $n^\circ$ 的弧, $n^\circ$ 的弧对着 $n^\circ$ 的圆心角(图 \ref{fig:czjh2-7-18})。
也就是说,圆心角的度数和它所对的弧的度数相等。


\liti[0] 如图 \ref{fig:czjh2-7-19}, $AB$、$DE$ 是 $\yuan\,O$ 的直径, $AC \pingxing DE$,交 $\yuan\,O$ 于点 $C$。

求证:  $BE = EC$。

\zhengming 在 $\yuan\,O$ 中,

$\left.\begin{aligned}
    \angle AOD = \angle BOE  \tuichu \yuanhu{AD} = \yuanhu{BE} \\
    AC \pingxing DE  \tuichu \yuanhu{AD} = \yuanhu{EC}
\end{aligned}\right\}  \tuichu \yuanhu{BE} = \yuanhu{EC}  \tuichu  BE = EC \juhao$


\begin{lianxi}

\xiaoti{如图,$\yuan\,O$ 的弦 $AB > CD$, $AB$、$CD$ 的弦心距分别为 $OM$ 和 $ON$。
    求证:$OM < ON$。
}

\begin{figure}[htbp]
    \centering
    \begin{minipage}[b]{7cm}
        \centering
        \begin{tikzpicture}
    \tkzDefPoints{0/0/O}
    \tkzDefPoint(315:1.5){A}
    \tkzDefPoint(45:1.5){B}
    \tkzDefPoint(110:1.5){C}
    \tkzDefPoint(160:1.5){D}
    \tkzDefLine[altitude](A,O,B)  \tkzGetPoint{M}
    \tkzDefLine[altitude](C,O,D)  \tkzGetPoint{N}

    \tkzDrawCircle[thick](O,A)
    \tkzDrawSegments[dashed](O,A  O,B  O,C  O,D)
    \tkzDrawSegments(A,B  C,D  O,M  O,N)
    \tkzMarkRightAngle(B,M,O)
    \tkzMarkRightAngle(O,N,C)
    \tkzDrawPoint(O)
    \tkzLabelPoints[below](O)
    \tkzLabelPoints[below](A)
    \tkzLabelPoints[right](B)
    \tkzLabelPoints[above](C)
    \tkzLabelPoints[left](D)
    \tkzLabelPoints[right,xshift=-.3em](M)
    \tkzLabelPoints[below=.2em](N)
\end{tikzpicture}


        \caption*{(第 1 题)}
    \end{minipage}
    \qquad
    \begin{minipage}[b]{7cm}
        \centering
        \begin{tikzpicture}
    \tkzDefPoints{0/0/P,  3/0/O,  4.5/0/M}
    \tkzDefPoint(15:5){E}
    \tkzDefPoint(-15:5){F}
    \tkzInterLC(P,E)(O,M)  \tkzGetPoints{B}{A}
    \tkzInterLC(P,F)(O,M)  \tkzGetPoints{C}{D}

    \tkzDrawSegments(P,E  P,F)
    \tkzDrawLine[add=0 and .8](P,O)
    \tkzDrawCircle(O,M)
    \tkzDrawPoint(O)
    \tkzMarkAngle[size=.6, arc=ll](F,P,O)
    \tkzMarkAngle[size=.8, arc=ll](O,P,E)

    \tkzLabelPoints[left](P)
    \tkzLabelPoints[right](E,F)
    \tkzLabelPoints[above](O)
    \tkzLabelPoints[above,xshift=-.3em](A)
    \tkzLabelPoints[above](B)
    \tkzLabelPoints[below,xshift=-.3em](C)
    \tkzLabelPoints[below](D)
\end{tikzpicture}


        \caption*{(第 2 题)}
    \end{minipage}
\end{figure}

\xiaoti{设 $O$ 是 $\angle EPF$ 的平分线上的一点,以 $O$ 为圆心的圆和角的两边分别相交于 $A$、$B$
    和 $C$、$D$。求证: $AB = CD$。
}

\xiaoti{(口答)在半径不等的 $\yuan\,O$ 和 $\yuan\,O'$ 中, $\yuanhu{AB}$ 和 $\yuanhu{A'B'}$
    所对的圆心角都是 $60^\circ$。
}
\begin{xiaoxiaotis}

    \xxt{$\yuanhu{AB}$ 和 $\yuanhu{A'B'}$ 各是多少度?}

    \xxt{$\yuanhu{AB}$ 和 $\yuanhu{A'B'}$ 相等吗?}

\end{xiaoxiaotis}


\end{lianxi}