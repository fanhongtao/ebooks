\subsection{直角三角形中成比例的线段}\label{subsec:czjh2-6-9}
\begin{enhancedline}

从一点到一条直线所作垂线的垂足,叫做\zhongdian{这点在这条直线上的正射影}。
图 \ref{fig:czjh2-6-30} 中,点 $P_1'$、$P_2'$、$P_3'$ 分别是点 $P_1$、$P_2$、$P_3$ 在直线 $MN$ 上的正射影。

\begin{figure}[htbp]
    \centering
    \begin{minipage}[b]{5.5cm}
        \centering
        \begin{tikzpicture}
    \tkzDefPoints{0/0/M, 3/0/N}
    \tkzDefPoints{0.5/1/P_1, 1.2/-1/P_2, 2/0/P_3}
    \tkzDefPointBy[projection= onto M--N](P_1)  \tkzGetPoint{P_1'}
    \tkzDefPointBy[projection= onto M--N](P_2)  \tkzGetPoint{P_2'}

    \tkzDrawSegment(M,N)
    \tkzDrawSegments[dashed](P_1,P_1'  P_2,P_2')
    \tkzDrawPoints(P_1, P_1', P_2, P_2', P_3)

    \tkzLabelPoints[left](M)
    \tkzLabelPoints[right](N)
    \tkzLabelPoints[above](P_1,P_2',P_3)
    \tkzLabelPoints[below](P_1',P_2)
    \tkzLabelPoint[below](P_3){$(P_3')$}
\end{tikzpicture}


        \caption{}\label{fig:czjh2-6-30}
    \end{minipage}
    \qquad
    \begin{minipage}[b]{8.5cm}
        \centering
        \begin{tikzpicture}
    \def\drawProjection{%
        \tkzDefPointBy[projection= onto M--N](A)  \tkzGetPoint{A'}
        \tkzDefPointBy[projection= onto M--N](B)  \tkzGetPoint{B'}
        \tkzDrawSegment[thick](A,B)
        \tkzDrawSegments[dashed](A,A' B,B')
    }

    \tkzDefPoints{0/0/M, 6.8/0/N}
    \tkzDrawSegment(M,N)
    \tkzLabelPoints[left](M)
    \tkzLabelPoints[right,yshift=.3em](N)

    \begin{scope}
        \tkzDefPoints{0.2/1/A, 0.8/1/B}
        \drawProjection
        \tkzLabelPoints[above](A,B)
        \tkzLabelPoints[below](A',B')
    \end{scope}

    \begin{scope}
        \tkzDefPoints{1.5/1.3/A, 2.3/1/B}
        \drawProjection
        \tkzLabelPoints[above](A,B)
        \tkzLabelPoints[below](A',B')
    \end{scope}

    \begin{scope}
        \tkzDefPoints{3.0/0.8/A, 4.0/-0.8/B}
        \drawProjection
        \tkzLabelPoints[above](A,B')
        \tkzLabelPoints[below](A',B)
    \end{scope}

    \begin{scope}
        \tkzDefPoints{4.7/0.8/A, 5.5/0/B}
        \drawProjection
        \tkzLabelPoints[above](A)
        \tkzLabelPoints[above right](B)
        \tkzLabelPoints[below](A')
        \tkzLabelPoint[below](B'){$(B')$}
    \end{scope}

    \begin{scope}
        \tkzDefPoints{6.2/1.3/A, 6.2/0.5/B}
        \tkzDrawPoints(A,B)
        \tkzDefPointBy[projection= onto M--N](A)  \tkzGetPoint{A'}
        \tkzDrawSegments[dashed](A,A')
        \tkzLabelPoints[above](A)
        \tkzLabelPoints[right](B)
        \tkzLabelPoint[below,xshift=1em](A'){$A' \; (B')$}
    \end{scope}
\end{tikzpicture}


        \caption{}\label{fig:czjh2-6-31}
    \end{minipage}
\end{figure}

一条线段的两个端点在一条直线上的正射影之间的线段,叫做\zhongdian{这条线段在这条直线上的正射影}。
图 \ref{fig:czjh2-6-31} 中的那些线段 $A'B'$ 都是对应的线段 $AB$ 在直线 $MN$ 上的正射影
(当 $AB \perp MN$ 时, $A'B'$ 缩为一个点)。

点、线段在一条直线上的正射影,简称\zhongdian{射影}。

\begin{dingli}[定理]
    直角三角形中,斜边上的高是两条直角边在斜边上的射影的比例中项;
    每一条直角边是这条直角边在斜边上的射影和斜边的比例中项。
\end{dingli}

已知:图 \ref{fig:czjh2-6-32} 中, $AB$ 是 $Rt \triangle ABC$ 的斜边, $CD$ 是高。

求证:\begin{tblr}[t]{colsep=0pt}
    (1)$CD^2 = AD \cdot BD$; \\
    (2)$AC^2 = AD \cdot AB$, $BC^2 = BD \cdot AB$。
\end{tblr}

$\left.\begin{aligned}
    \text{\zhengming (1)} && \angle ACB = 90^\circ \\
                            && CD \perp AB
\end{aligned}\right\} \tuichu  \triangle ACD \xiangsi \triangle CBD  \tuichu \dfrac{CD}{AD} = \dfrac{BD}{CD}  \tuichu CD^2 = AD \cdot BD \juhao$

(2)的证明由同学自己写出。

在第五章中,我们曾用面积割补法证明了勾股定理,现在利用上面的定理很容易证明勾股定理。
把上面的定理中 (2) 的两个关系式的两边分别相加,得

$AC^2 + BC^2 = AD \cdot AB + BD \cdot AB = AB(AD + BD) = AB^2$。

\begin{figure}[htbp]
    \centering
    \begin{minipage}[b]{7cm}
        \centering
        \begin{tikzpicture}
    \tkzDefPoints{0/0/A, 4/0/B}
    \tkzDefTriangle[two angles=60 and 30](A,B)  \tkzGetPoint{C}
    \tkzDefLine[altitude](A,C,B)  \tkzGetPoint{D}

    \tkzDrawPolygon(A,B,C)
    \tkzDrawSegment(C,D)
    \tkzMarkRightAngle(B,D,C)
    \tkzMarkRightAngle(A,C,B)
    \tkzLabelPoints[above](C)
    \tkzLabelPoints[left](A)
    \tkzLabelPoints[right](B)
    \tkzLabelPoints[below](D)
\end{tikzpicture}


        \caption{}\label{fig:czjh2-6-32}
    \end{minipage}
    \qquad
    \begin{minipage}[b]{7cm}
        \centering
        \begin{tikzpicture}
    \tkzDefPoints{0/0/A, 5/0/B}
    \tkzInterCC[R](A,3)(B,4)  \tkzGetFirstPoint{C}
    \tkzDefLine[altitude](A,C,B)  \tkzGetPoint{D}
    \tkzDefLine[bisector](A,C,B)  \tkzGetPoint{e}
    \tkzInterLL(C,e)(A,B)  \tkzGetPoint{E}

    \tkzDrawPolygon(A,B,C)
    \tkzDrawSegments(C,D  C,E)
    \tkzMarkRightAngle(C,D,A)
    \tkzLabelPoints[above](C)
    \tkzLabelPoints[left](A)
    \tkzLabelPoints[right](B)
    \tkzLabelPoints[below](D,E)
\end{tikzpicture}


        \caption{}\label{fig:czjh2-6-33}
    \end{minipage}
\end{figure}

\liti 已知:$\triangle ABC$ 中,$\angle ACB = 90^\circ$, $CD$ 是高, $CE$ 是角平分线,
$AC = 9\;\limi$, $BC = 12\;\limi$ (图 \ref{fig:czjh2-6-33})。求 $CD$、$CE$ 的长。

\jie 由勾股定理可得

$AB = \sqrt{AC^2 + BC^2} = 15 \;\limi$。

$\therefore$ \quad \begin{tblr}[t]{colsep=0pt, rowsep=.5em}
    $BD = \dfrac{BC^2}{AB} = \dfrac{144}{15} = \dfrac{48}{5} \;(\limi)$, \\
    $AD = AB - BD = \dfrac{27}{5} \;(\limi)$; \\
    $CD = \sqrt{BD \cdot AD} = \dfrac{36}{5} \;(\limi)$。
\end{tblr}

$\because$ \quad $\angle ACE = \angle BCE$,

$\therefore$ \quad $\dfrac{AC}{AE} = \dfrac{BC}{BE} = \dfrac{AC + BC}{AE + BE}$。

得 \quad \begin{tblr}[t]{colsep=0pt, rowsep=.5em}
    $BE = \dfrac{BC \cdot AB}{AC + BC} = \dfrac{12 \times 15}{9 + 12} = \dfrac{60}{7} \;(\limi)$; \\
    $DE = BD - BE = \dfrac{48}{5} - \dfrac{60}{7} = \dfrac{36}{35} \;(\limi)$。
\end{tblr}

$\therefore$ \quad $CD = \sqrt{CD^2 + DE^2} = \sqrt{\left(\dfrac{36}{5}\right)^2 + \left(\dfrac{36}{35}\right)^2} = \dfrac{36\sqrt{2}}{7} \;(\limi)$。


\begin{lianxi}

\xiaoti{证明本节定理(2)。}

\xiaoti{$CD$ 是 $Rt \triangle ABC$ 的斜边 $AB$ 上的高。}
\begin{xiaoxiaotis}

    \xxt{已知 $AD = 9$ 厘米,$DB = 4$ 厘米。求 $CD$ 和 $AC$;}

    \xxt{已知 $AB = 25$ 厘米,$BC = 15$ 厘米。求 $DB$ 和 $CD$。}

\end{xiaoxiaotis}


\xiaoti{设 $CD$ 是 $Rt \triangle ABC$ 的斜边 $AB$ 上的高。求证:}
\begin{xiaoxiaotis}

    \xxt{$\dfrac{AC^2}{CB^2} = \dfrac{AD}{DB}$;}
    \xxt{$CA \cdot CD = CB \cdot AD$。}

\end{xiaoxiaotis}

\end{lianxi}

\end{enhancedline}

