\subsection{切线的判定和性质}\label{subsec:czjh2-7-8}

如图 \ref{fig:czjh2-7-33},在 $\yuan\,O$ 中,经过半径 $OA$ 的外端点 $A$,作直线 $l \perp OA$,
则圆心 $O$ 与直线 $l$ 的距离就是半径 $r$。由上一节我们知道,这样的直线与圆一定相切。
因此有下面定理:

\begin{dingli}[切线的判定定理]
    经过半径的外端并且垂直于这条半径的直线是圆的切线。
\end{dingli}

\begin{figure}[htbp]
    \centering
    \begin{minipage}[b]{7cm}
        \centering
        \begin{tikzpicture}
    \tkzDefPoints{0/0/O}
    \tkzDefPoint(270:1.5){A}
    \tkzDefLine[perpendicular=through A,normed](O,A)  \tkzGetPoint{b}
    \tkzDefPointOnLine[pos=2.0](A,b)  \tkzGetPoint{B}

    \tkzDrawCircle[thick](O,A)
    \tkzDrawPoint(O)
    \tkzDrawSegment(O,A)
    \tkzDrawLine[add=0 and 1](B,A)
    \tkzMarkRightAngle[size=.2](O,A,B)
    \tkzLabelPoints[above](O)
    \tkzLabelPoints[below](A)
    \tkzLabelSegment[pos=1, right](A,B){$l$}
\end{tikzpicture}


        \caption{}\label{fig:czjh2-7-33}
    \end{minipage}
    \qquad
    \begin{minipage}[b]{7cm}
        \centering
        \begin{tikzpicture}
    \tkzDefPoints{0/0/O}
    \tkzDefPoint(270:1.5){C}
    \tkzDefLine[perpendicular=through C,normed](O,C)  \tkzGetPoint{b}
    \tkzDefPointOnLine[pos=2.0](C,b)  \tkzGetPoint{B}
    \tkzDefPointOnLine[pos=2](B,C)  \tkzGetPoint{A}

    \tkzDrawCircle[thick](O,C)
    \tkzDrawPoint(O)
    \tkzDrawSegment[dashed](O,C)
    \tkzDrawSegments(O,A  O,B)
    \tkzDrawLine[add=.1 and .1](A,B)
    \tkzLabelPoints[above](O)
    \tkzLabelPoints[below](A,B,C)
\end{tikzpicture}


        \caption{}\label{fig:czjh2-7-34}
    \end{minipage}
\end{figure}

\liti 已知:直线 $AB$ 经过 $\yuan\,O$ 上的点 $C$, 并且 $OA = OB$,$CA = CB$(图 \ref{fig:czjh2-7-34})。

求证: 直线 $AB$ 是 $\yuan\,O$ 的切线。

\zhengming 连结 $OC$。

$\because$ \quad $OA = OB$, $CA = CB$,

$\therefore$ \quad $OC$ 是等腰三角形 $OAB$ 底边 $AB$ 上的中线。

$\therefore$ \quad $AB \perp OC$。

因此,直线 $AB$ 经过半径 $OC$ 的外端 $C$,并且垂直于半径 $OC$, 所以 $AB$ 是 $\yuan\,O$ 的切线。


\begin{dingli}[切线的性质定理]
    圆的切线垂直于经过切点的半径。
\end{dingli}


已知: 如图 \ref{fig:czjh2-7-35},直线 $AT$ 是 $\yuan\,O$ 的切线, $A$ 为切点。

求证: $AT \perp OA$。

\zhengming 假设 $AT$ 与 $0A$ 不垂直。

经过圆心且垂直于切线的直线必经过

过圆心 $O$ 作 $OM \perp AT$, 交 $AT$ 于点 $M$。由垂线段最短,得 $OM < OA$。

因为圆心到直线 $AT$ 的距离小于半径,所以 $AT$ 与 $\yuan\,O$ 相交。 这与已知相矛盾。

$\therefore$ \quad $AT \perp OA$。


由于过已知点只有一条直线与已知直线垂直, 所以经过圆心垂直于切线的直线一定过切点;
反过来,过切点垂直于切线的直线也一定经过圆心。 由此得到:

\begin{tuilun}[推论1]
    经过圆心且垂直于切线的直线必经过切点。
\end{tuilun}

\begin{tuilun}[推论2]
    经过切点且垂直于切线的直线必经过圆心。
\end{tuilun}

\begin{figure}[htbp]
    \centering
    \begin{minipage}[b]{7cm}
        \centering
        \begin{tikzpicture}
    \tkzDefPoints{0/0/O}
    \tkzDefPoint(270:1.5){A}
    \tkzDefLine[perpendicular=through A,normed](O,A)  \tkzGetPoint{T}
    \tkzDefPointOnLine[pos=2.0](A,T)  \tkzGetPoint{T}
    \tkzDefPointOnLine[pos=.4](A,T)  \tkzGetPoint{M}

    \tkzDrawCircle[thick](O,A)
    \tkzDrawPoint(O)
    \tkzDrawSegment(O,A)
    \tkzDrawLine[add=0 and 1](T,A)
    \tkzDrawSegment[dashed](O,M)
    \tkzLabelPoints[above](O)
    \tkzLabelPoints[below](A,M)
    \tkzLabelPoints[right](T)
\end{tikzpicture}


        \caption{}\label{fig:czjh2-7-35}
    \end{minipage}
    \qquad
    \begin{minipage}[b]{7cm}
        \centering
        \begin{tikzpicture}
    \tkzDefPoints{0/0/O}
    \tkzDefPoint(180:1.5){A}
    \tkzDefPoint(0:1.5){B}
    \tkzDefLine[perpendicular=through B,normed](O,B)  \tkzGetPoint{c}
    \tkzDefPointOnLine[pos=2.0](B,c)  \tkzGetPoint{C}
    \tkzDefLine[parallel=through A](O,C)  \tkzGetPoint{d}
    \tkzInterLC[common=A](A,d)(O,A)  \tkzGetFirstPoint{D}

    \tkzDrawCircle[thick](O,A)
    \tkzDrawSegments(A,B  A,D  O,C)
    \tkzDrawLine[add=0 and .7](C,B)
    \tkzDrawLine[add=0 and .7](C,D)
    \tkzDrawSegment[dashed](O,D)
    \extkzLabelAngel[0.4](O,A,D){$1$}
    \extkzLabelAngel[0.4](A,D,O){$2$}
    \extkzLabelAngel[0.5](B,O,C){$3$}
    \extkzLabelAngel[0.4](C,O,D){$4$}
    \tkzLabelPoints[below](O)
    \tkzLabelPoints[left](A)
    \tkzLabelPoints[right](B,C)
    \tkzLabelPoints[above](D)
\end{tikzpicture}


        \caption{}\label{fig:czjh2-7-36}
    \end{minipage}
\end{figure}


\liti 已知: $AB$ 是 $\yuan\,O$ 的直径, $BC$ 是 $\yuan\,O$ 的切线,切点为 $B$,
$OC$ 平行于弦 $AD$ (图 \ref{fig:czjh2-7-36})。

求证: $DC$ 是 $\yuan\,O$ 的切线。

\zhengming 连结 $OD$。

$\left.\begin{aligned}
    OA = OD          \tuichu  \angle 1 = \angle 2 \\
    AD \pingxing OC  \tuichu  \left\{\begin{aligned}
        \angle 1 = \angle 3 \\
        \angle 2 = \angle 4
    \end{aligned}\right\}
\end{aligned}\right\}  \tuichu  \angle 3 = \angle 4 \juhao$

$\left.\begin{aligned}
    OB = OD \\
    \angle 3 = \angle 4 \\
    OC = OC
\end{aligned}\right\}  \tuichu  \triangle OBC \quandeng \triangle ODC  \tuichu  \angle OBC = \angle ODC \juhao$


$\because$ \quad $BC$ 是 $\yuan\,O$ 的切线,

$\therefore$ \quad $\angle OBC = 90^\circ$。

$\therefore$ \quad $\angle ODC = 90^\circ$。

$\therefore$ \quad $DC$ 是 $\yuan\,O$ 的切线。



\begin{lianxi}

\xiaoti{如图, $AB$ 是 $\yuan\,O$ 的直径, $\angle ABT = 45^\circ$, $AT = AB$。
    求证: $AT$ 是 $\yuan\,O$ 的切线。
}

\begin{figure}[htbp]
    \centering
    \begin{minipage}[b]{4.8cm}
        \centering
        \begin{tikzpicture}
    \tkzDefPoints{0/0/O}
    \tkzDefPoint(270:1.5){A}
    \tkzDefPoint(90:1.5){B}
    \tkzDefTriangle[two angles=90 and 45](A,B)  \tkzGetPoint{T}

    \tkzDrawCircle[thick](O,A)
    \tkzDrawPolygon(A,B,T)
    \tkzDrawPoint(O)
    \tkzMarkAngle[size=.5](T,B,A)
    \tkzLabelPoints[right](O)
    \tkzLabelPoints[below](A)
    \tkzLabelPoints[above](B)
    \tkzLabelPoints[below](T)
\end{tikzpicture}


        \caption*{(第 1 题)}
    \end{minipage}
    \qquad
    \begin{minipage}[b]{5.2cm}
        \centering
        \begin{tikzpicture}
    \tkzDefPoints{0/0/O}
    \tkzDefPoint(180:1.5){A}
    \tkzDefPoint(0:1.5){B}
    \tkzDefPointOnLine[pos=2](O,B)  \tkzGetPoint{D}
    \tkzDefPoint(60:1.5){C}  % OD = 2OC, 再根据结论:DC 是切线,角OCD = 90度,所以,角 CDO = 30 度,即:角 DOC = 60 度

    \tkzDrawCircle[thick](O,A)
    \tkzDrawPoint(O)
    \tkzDrawSegments(A,C  A,D)
    \tkzDrawLine[add=0 and 0.6](D,C)
    \tkzLabelPoints[below](O)
    \tkzLabelPoints[left](A)
    \tkzLabelPoints[below,xshift=.5em](B)
    \tkzLabelPoints[above](C)
    \tkzLabelPoints[below,xshift=-.5em](D)
\end{tikzpicture}


        \caption*{(第 2 题)}
    \end{minipage}
    \qquad
    \begin{minipage}[b]{4.5cm}
        \centering
        \begin{tikzpicture}
    \tkzDefPoints{0/0/O}
    \tkzDefPoint(70:3.5){A}
    \tkzDefPoint(0:3.5){B}
    \tkzDefPoint(35:4.5){C}
    \tkzDefPointOnLine[pos=.55](O,C)  \tkzGetPoint{D}
    \tkzDefLine[altitude](A,D,O)  \tkzGetPoint{E}

    \tkzDrawCircle[thick](D,E)
    \tkzDrawPoint(D)
    \tkzDrawSegments(O,A  O,B  O,C)
    \tkzLabelPoints[below](O)
    \tkzLabelPoints[left](A)
    \tkzLabelPoints[below](B)
    \tkzLabelPoints[below](C)
    \tkzLabelPoints[above left](D)
    \tkzLabelPoints[left](E)
\end{tikzpicture}


        \caption*{(第 4 题)}
    \end{minipage}
\end{figure}

\xiaoti{$AB$ 是 $\yuan\,O$ 的直径,点 $D$ 在 $AB$ 的延长线上, $BD = OB$,点 $C$ 在圆上,
    $\angle CAB = 30^\circ$。 求证: $DC$ 是 $\yuan\,O$ 的切线。
}

\xiaoti{求证:}
\begin{xiaoxiaotis}

    \xxt{经过圆的直径两端点的切线互相平行;}

    \xxt{圆的两条切线互相平行,则连结两个切点的线段是直径。}

\end{xiaoxiaotis}


\xiaoti{已知: $OC$ 平分 $\angle AOB$, $D$ 是 $OC$ 上任意一点,
    $\yuan\,D$ 与 $OA$ 相切于点 $E$。 求证: $OB$ 与 $\yuan\,D$ 相切。
}

\end{lianxi}
