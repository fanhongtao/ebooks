\subsection{弦切角}\label{subsec:czjh2-7-11}

顶点在圆上,一边和圆相交、另一边和圆相切的角叫做\zhongdian{弦切角}。
图 \ref{fig:czjh2-7-41} 甲、乙、丙中,$\angle BAC$ 是弦切角。
弦切角也可以看作圆周角的一边绕顶点旋转到与圆相切时所成的角。

\begin{figure}[htbp]
    \centering
    \begin{minipage}[b]{4.5cm}
        \centering
        \begin{tikzpicture}
    \tkzDefPoints{0/0/O}
    \tkzDefPoint(270:1.5){A}
    \tkzDefPoint(90:1.5){C}
    \tkzDefPoint(220:1.5){P}
    \tkzDefPoint(20:1.5){m}
    \tkzDefLine[perpendicular=through A,normed](O,A)  \tkzGetPoint{b}
    \tkzDefPointOnLine[pos=2.0](A,b)  \tkzGetPoint{B}

    \tkzDrawCircle[thick](O,A)
    \tkzDrawPoint(O)
    \tkzDrawPolygon(A,C,P)
    \tkzDrawLine[add=0 and .8](B,A)
    \tkzMarkRightAngle[](A,P,C)
    \tkzLabelPoints[right](O)
    \tkzLabelPoints[below](A)
    \tkzLabelPoints[right](B,m)
    \tkzLabelPoints[above](C)
    \tkzLabelPoints[left](P)
\end{tikzpicture}


        \caption*{甲}
    \end{minipage}
    \qquad
    \begin{minipage}[b]{4.7cm}
        \centering
        \begin{tikzpicture}
    \tkzDefPoints{0/0/O}
    \tkzDefPoint(270:1.5){A}
    \tkzDefPoint(90:1.5){Q}
    \tkzDefPoint(30:1.5){C}
    \tkzDefPoint(200:1.5){P}
    \tkzDefPoint(340:1.5){m}
    \tkzDefLine[perpendicular=through A,normed](O,A)  \tkzGetPoint{b}
    \tkzDefPointOnLine[pos=2.0](A,b)  \tkzGetPoint{B}

    \tkzDrawCircle[thick](O,A)
    \tkzDrawPoint(O)
    \tkzDrawPolygon(A,C,P)
    \tkzDrawLine[add=0 and .8](B,A)
    \tkzDrawSegments[dashed](A,Q  C,Q)
    \tkzMarkRightAngle[](A,C,Q)
    \extkzLabelAngel[0.4](C,A,Q){$1$}
    \tkzLabelPoints[right](O)
    \tkzLabelPoints[below](A)
    \tkzLabelPoints[right](B,m)
    \tkzLabelPoints[above](Q,C)
    \tkzLabelPoints[left](P)
\end{tikzpicture}


        \caption*{乙}
    \end{minipage}
    \qquad
    \begin{minipage}[b]{4.5cm}
        \centering
        \begin{tikzpicture}
    \tkzDefPoints{0/0/O}
    \tkzDefPoint(270:1.5){A}
    \tkzDefPoint(90:1.5){Q}
    \tkzDefPoint(140:1.5){C}
    \tkzDefPoint(230:1.5){P}
    \tkzDefPoint(20:1.5){m}
    \tkzDefLine[perpendicular=through A,normed](O,A)  \tkzGetPoint{b}
    \tkzDefPointOnLine[pos=2.0](A,b)  \tkzGetPoint{B}
    \tkzDefPointOnLine[pos=1.8](B,A)  \tkzGetPoint{D}

    \tkzDrawCircle[thick](O,A)
    \tkzDrawPoint(O)
    \tkzDrawPolygon(A,C,P)
    \tkzDrawSegment(B,D) %\tkzDrawLine[add=0 and 1](B,A)
    \tkzDrawSegments[dashed](A,Q  C,Q)
    \tkzMarkRightAngle[](A,C,Q)
    \tkzLabelPoints[right](O)
    \tkzLabelPoints[below](A,D)
    \tkzLabelPoints[right](B,m)
    \tkzLabelPoints[above](Q,C)
    \tkzLabelPoints[left, yshift=-.3em](P)
\end{tikzpicture}


        \caption*{丙}
    \end{minipage}
    \caption{}\label{fig:czjh2-7-41}
\end{figure}

\begin{dingli}[弦切角定理]
    弦切角等于它所夹的弧对的圆周角。
\end{dingli}

已知: $AC$ 是 $\yuan\,O$ 的弦,$AB$ 是 $\yuan\,O$ 的切线,
$\yuanhu{AmC}$ 是弦切角 $\angle BAC$ 所夹的弧,
$\angle P$ 是 $\yuanhu{AmC}$ 所对的圆周角(图 \ref{fig:czjh2-7-41})。

求证: $\angle BAC = \angle APC$。

证明:分三种情况讨论。

(1) 圆心 $O$ 在 $\angle BAC$ 的边 $AC$ 上(图 \ref{fig:czjh2-7-41} 甲)。

$\because$ \quad $AB$ 是 $\yuan\,O$ 的切线,

$\therefore$ \quad $\angle BAC = 90^\circ$。

又 $\because$ \quad $\yuanhu{AmC}$ 是半圆,

$\therefore$ \quad $\angle P = 90^\circ$。

$\therefore$ \quad $\angle BAC = \angle P$。

(2) 圆心 $O$ 在 $\angle BAC$ 的外部(图 \ref{fig:czjh2-7-41} 乙)。

作 $\yuan\,O$ 的直径 $AQ$,连结 $CQ$。

$\because$ \quad $\angle BAQ = \angle ACQ = 90^\circ$,

$\therefore$ \quad $\angle BAC = 90^\circ - \angle 1$, $\angle Q = 90^\circ - 1$。

$\therefore$ \quad $\angle BAC = \angle Q$。

又 $\because$ \quad $\angle Q = \angle P$,

$\therefore$ \quad $\angle BAC = \angle P$。

(3) 圆心 $O$ 在 $\angle BAC$ 的内部(图 \ref{fig:czjh2-7-41} 丙)。

作 $\yuan\,O$ 的直径 $AQ$,连结 $CQ$。

$\because$ \quad $\angle BAC = 180^\circ - \angle DAC$, $\angle P = 180^\circ - \angle Q$,

又由 (2) 可知, $\angle DAC = \angle Q$,

$\therefore$ \quad $\angle BAC = \angle P$。

\begin{tuilun}[推论]
    两个弦切角所夹的弧相等,这两个弦切角也相等。
\end{tuilun}

\begin{wrapfigure}[5]{r}{6.8cm}
    \centering
    \begin{tikzpicture}
    \pgfmathsetmacro{\R}{2}
    \pgfmathsetmacro{\r}{1.5}
    \tkzDefPoints{0/0/O, 2.6/0/O'}
    \tkzInterCC[R](O,\r)(O',\R)  \tkzGetPoints{A}{B}
    \tkzDefLine[perpendicular=through A](O',A)  \tkzGetPoint{c}
    \tkzInterLC[R,common=A](A,c)(O,\r)  \tkzGetFirstPoint{C}
    \tkzDefLine[perpendicular=through A](O,A)  \tkzGetPoint{d}
    \tkzInterLC[R,common=A](A,d)(O',\R)  \tkzGetFirstPoint{D}

    \tkzDrawCircle[thick](O,A)
    \tkzDrawCircle[thick](O',A)
    \tkzDrawPoints(O, O')
    \tkzDrawLine[add=0 and .4](C,A)
    \tkzDrawLine[add=0 and .4](D,A)
    \tkzDrawSegments(A,B  B,C  B,D)
    \extkzLabelAngel[0.4](B,A,D){$1$}
    \extkzLabelAngel[0.5](C,A,B){$2$}
    \tkzLabelPoints[left](O)
    \tkzLabelPoints[right](O')
    \tkzLabelPoints[above](A)
    \tkzLabelPoints[below](B)
    \tkzLabelPoints[below left](C)
    \tkzLabelPoints[below right](D)
\end{tikzpicture}


    \caption{}\label{fig:czjh2-7-42}
\end{wrapfigure}

\liti 已知: 如图 \ref{fig:czjh2-7-42}, $\yuan\,O$ 和 $\yuan\,O'$ 相交于 $A$、$B$ 两点,
$AC$ 是 $\yuan\,O'$ 的切线,交 $\yuan\,O$ 于点 $C$,
$AD$ 是 $\yuan\,O$  的切线,交 $\yuan\,O'$ 于点 $D$。

求证:$AB^2 = BC \cdot BD$。

\begin{enhancedline}
\zhengming $\because$ \quad $\angle C = \angle 1$, $\angle 2 = \angle D$,

$\therefore$ \quad $\triangle ACB \xiangsi \triangle DAB$。

$\therefore$ \quad $\dfrac{BC}{AB} = \dfrac{AB}{BD}$。

$\therefore$ \quad $AB^2 = BC \cdot BD$。
\end{enhancedline}

\begin{lianxi}

\xiaoti{如图,经过 $\yuan\,O$ 上的点 $T$ 的切线和弦 $AB$ 的延长线相交于点 $C$。
    求证:$\angle ATC = \angle TBC$。
}

\begin{figure}[htbp]
    \centering
    \begin{minipage}[b]{7cm}
        \centering
        \begin{tikzpicture}
    \tkzDefPoints{0/0/O}
    \tkzDefPoint(210:1.5){A}
    \tkzDefPoint(330:1.5){B}
    \tkzDefPoint(50:1.5){T}
    \tkzDefLine[perpendicular=through T,normed](O,T)  \tkzGetPoint{c}
    \tkzInterLL(T,c)(A,B)  \tkzGetPoint{C}

    \tkzDrawCircle[thick](O,A)
    \tkzDrawPoint(O)
    \tkzDrawSegments(A,C  A,T  B,T)
    \tkzDrawLine[add=0 and .3](C,T)
    \tkzLabelPoints[below](O)
    \tkzLabelPoints[left](A)
    \tkzLabelPoints[below](B)
    \tkzLabelPoints[right](C)
    \tkzLabelPoints[above right](T)
\end{tikzpicture}


        \caption*{(第 1 题)}
    \end{minipage}
    \qquad
    \begin{minipage}[b]{7cm}
        \centering
        \begin{tikzpicture}
    \tkzDefPoints{0/0/O}
    \tkzDefPoint(200:1.5){A}
    \tkzDefPoint(340:1.5){B}
    \tkzDefPoint(90:1.5){M}
    \tkzDefLine[perpendicular=through M,normed](O,M)  \tkzGetPoint{c}
    \tkzDefPointOnLine[pos=1.5](M,c)  \tkzGetPoint{C}
    \tkzDefPointOnLine[pos=2.0](C,M)  \tkzGetPoint{D}

    \tkzDrawCircle[thick](O,A)
    \tkzDrawPoint(O)
    \tkzDrawPolygon(A,B,M)
    \tkzDrawSegments(C,D)
    \tkzLabelPoints[above](O)
    \tkzLabelPoints[left](A,C)
    \tkzLabelPoints[right](B,D)
    \tkzLabelPoints[above](M)
\end{tikzpicture}


        \caption*{(第 2 题)}
    \end{minipage}
\end{figure}

\xiaoti{如图,$AB$ 是 $\yuan\,O$ 的弦, $CD$ 是经过 $\yuan\,O$ 上一点 $M$ 的切线。求证:}
\begin{xiaoxiaotis}

    \xxt{$AB \pingxing CD$ 时, $AM = MB$;}

    \xxt{$AM = MB$ 时,$AB \pingxing CD$。}
\end{xiaoxiaotis}

\end{lianxi}


因为弦切角和它夹的弧所对的圆周角相等,所以我们可以利用这个关系作含圆周角等于已知角的弧。


\begin{wrapfigure}[9]{r}{6cm}
    \centering
    \begin{tikzpicture}
    \pgfmathsetmacro{\a}{45}

    \begin{scope}[xshift=-2cm]
        \tkzDefPoints{0/0/O}
        \tkzDefPoint(0:1.0){A}
        \tkzDefPoint(\a:1.0){B}

        \tkzDrawSegments(O,A  O,B)
        \extkzLabelAngel[0.5](A,O,B){$\alpha$}
    \end{scope}

    \tkzDefPoints{0/0/A, 2.5/0/B}
    \tkzDrawSegment(A,B)
    \tkzLabelPoints[left](A)
    \tkzLabelPoints[right](B)

    % 1
    \tkzDefLine[mediator, K=.9](A,B)  \tkzGetPoints{M}{N}
    \tkzDrawSegment(M,N)
    \tkzLabelPoints[above](M)
    \tkzLabelPoints[below](N)

    % 2
    \tkzDefPointBy[rotation=center B angle \a](A)  \tkzGetPoint{d}
    \tkzDefPointOnLine[pos=.5](B,d)  \tkzGetPoint{D}
    \tkzDrawSegment(B,D)
    \extkzLabelAngel[0.5](A,B,D){$\alpha$}
    \tkzLabelPoints[below right](D)

    % 3
    \tkzDefLine[perpendicular=through B, K=2](D,B)  \tkzGetPoint{E}
    \tkzInterLL(B,E)(M,N)  \tkzGetPoint{O}
    \tkzDrawSegments(B,E)
    \tkzDrawPoint(O)
    \tkzMarkRightAngle(E,B,D)
    \tkzLabelPoints[left](E)
    \tkzLabelPoints[right](O)

    % 4
    \tkzCalcLength(O,A)  \tkzGetLength{rOA}
    \tkzDefShiftPoint[O](130:\rOA){m}
    \tkzDefShiftPoint[O](60:\rOA){P} % \tkzDefPointBy[rotation=center O angle 105](B)  \tkzGetPoint{P}
    \tkzDrawArc[thick](O,B)(A)
    \tkzDrawSegments[thick](A,P  B,P)
    \extkzLabelAngel[0.5](A,P,B){$\alpha$}
    \tkzLabelPoints[above](P)
    \tkzLabelPoints[above, rotate=30](m)

    % 另一个弧
    \tkzDefPointBy[reflection=over A--B](O)  \tkzGetPoint{O'}
    \tkzDrawArc[thick](O',A)(B)
\end{tikzpicture}


    \caption{}\label{fig:czjh2-7-43}
\end{wrapfigure}

\liti \zhongdian{在已知线段上作弧,使它所含的圆周角等于已知角。}

已知: 线段 $AB$、$\angle \alpha$ (图 \ref{fig:czjh2-7-43})。

求作: 以 $AB$ 为弦的圆弧, 使弧所含的圆周角等干 $\angle \alpha$。

分析:作弧的关键在于确定它的圆心的位置。

如图,因为 $AB$ 是 $\yuanhu{AmB}$ 所对的弦,弧要经过 $A$、$B$ 两点,
所以圆心 $O$ 必在线段 $AB$ 的垂直平分线 $MN$ 上。

设 $BD$ 是经过点 $B$ 的 $\yuan\,O$ 的切线。
因为弦切角 $\angle ABD$ 等于 $\yuanhu{AmB}$ 所含的圆周角 $\angle APB$,
所以 $\angle ABD$ 必须等于 $\angle \alpha$,
而圆心 $O$ 必在经过点 $B$ 并且垂直于 $BD$ 的直线 $BE$ 上。
因此,圆心就是这两条直线的交点。

\zuofa 1. 作线段 $AB$ 的垂直平分线 $MN$。

2. 过点 $B$ 作射线 $BD$, 使 $\angle ABD = \angle \alpha$。

3. 过点 $B$ 作 $BD$ 的垂线 $BE$ 交 $MN$ 于点 $O$。

4. 以 $O$ 为圆心, $OA$ 为半径, 在 $\angle ABD$ 的外部作 $\yuanhu{AmB}$。
$\yuanhu{AmB}$ 就是所求的弧。

证明略。

因为,过点 $B$ 与 $BA$ 成 $\angle \alpha$ 的射线可以作出两条,
所以,所求的弧可以作出两条(分别在直线 $AB$ 的两旁)。


\begin{lianxi}

\xiaoti{取线段 $AB = 3$ cm。 在 $AB$ 上作弧,使它所含的圆周角等于 $45^\circ$。}

\xiaoti{在线段 $AB$ 上作含 $90^\circ$ 的圆周角的弧得到什么图形?}

\end{lianxi}

