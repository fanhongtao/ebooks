\subsection{正多边形的有关计算}\label{subsec:czjh2-7-17}

\begin{enhancedline}

我们知道,多边形的内角和等于 $(n - 2) \cdot 180^\circ$。
因为正多边形的各角都相等,所以它的每个内角都等于
$$ \dfrac{(n - 2) \cdot 180}{n}  \juhao $$

下面研究正多边形的其他计算问题。

正 $n$ 边形的 $n$ 条半径把它分成了 $n$ 个等腰三角形,如图 \ref{fig:czjh2-7-68} ,
这些等腰三角形底边上的高,又把它们分成 $2n$ 个直角三角形,
而这些直角三角形的斜边恰好都是正 $n$ 边形的半径,
一条直角边是正 $n$ 边形的边心距,
另一条直角边是正 $n$ 边形边长的一半,
显然这些直角三角形是全等的。 因而得到:

\begin{figure}[htbp]
    \centering
    \begin{minipage}[b]{7cm}
        \centering
        \begin{tikzpicture}
    \pgfmathsetmacro{\R}{2}
    \pgfmathsetmacro{\n}{8}

    \tkzDefPoints{0/0/O}
    \tkzDefPoint(247.5:\R){A} % 270 - 360/8/2 = 270 - 22.5 = 247.5
    \tkzDefRegPolygon[center,sides=\n,name=P](O,A)
    \coordinate (B) at (P2);

    \tkzDefLine[altitude](A,O,B)  \tkzGetPoint{G}

    \tkzDrawCircle[very thick](O,A)
    \tkzDrawSegments(B,P3  P8,A)
    \tkzDrawSegments[dim={$a_n$,-1em,}](A,B)
    \tkzDrawSegments[dashed](O,A  O,B  O,G)

    \tkzMarkRightAngle[size=.2](B,G,O)
    \extkzLabelAngel[0.5](A,O,B){$\alpha_n$}
    \tkzLabelSegment[right](O,B){$R$}
    \tkzLabelSegment[pos=.8, left, xshift=.2em](O,G){$r_n$}
    \tkzLabelPoints[above](O)
    \tkzLabelPoints[left,  yshift=-.3em](A)
    \tkzLabelPoints[right, yshift=-.3em](B)
\end{tikzpicture}


        \caption{}\label{fig:czjh2-7-68}
    \end{minipage}
    \qquad
    \begin{minipage}[b]{7cm}
        \centering
        \begin{tikzpicture}
    \pgfmathsetmacro{\R}{2}
    \pgfmathsetmacro{\n}{6}

    \tkzDefPoints{0/0/O}
    \tkzDefPoint(240:\R){A} % 270 - 360/6/2 = 270 - 30 = 240
    \tkzDefRegPolygon[center,sides=\n,name=P](O,A)
    \foreach \P [count=\i from 2] in {B,C,...,F} {
        \coordinate (\P) at (P\i);
    }
    \tkzDefLine[altitude](A,O,B)  \tkzGetPoint{G}

    \tkzDrawCircle[very thick](O,A)
    \tkzDrawPolygon(P1,P...,P\n)
    \tkzDrawSegments[dim={$a_6$,-1.2em,}](A,B)
    \tkzDrawSegments(O,A  O,B  O,G)

    \tkzMarkRightAngle[size=.2](B,G,O)
    \tkzLabelSegment[right](O,B){$R$}
    \tkzLabelSegment[pos=.3, xshift=-.5em](O,G){$r_6$}
    \tkzLabelPoints[above](O)
    \tkzLabelPoints[left,  yshift=-.3em](A)
    \tkzLabelPoints[right, yshift=-.3em](B)
    \tkzAutoLabelPoints[center=O, centered, dist= .2](C,...,F)
    \tkzLabelPoints[above, xshift=-.5em](G)
\end{tikzpicture}


        \caption{}\label{fig:czjh2-7-69}
    \end{minipage}
\end{figure}


\begin{dingli}[定理]
    正 $n$ 边形的半径和边心距把正 $n$ 边形分成 $2n$ 个全等的直角三角形。
\end{dingli}

由于正 $n$ 边形的中心角 $\alpha_n = \dfrac{360^\circ}{n}$,我们应用上面定理,
就可以把正 $n$ 边形的边长 $a_n$, 边心距 $r_n$, 周长 $p_n$ 和面积 $S_n$
等的计算问题归结为有关直角三角形的计算问题。



\liti[0] 已知正六边形 $ABCDEF$ 的半径为 $R$,求这个正六边形的边长 $a_6$;
周长 $p_6$ 和面积 $S_6$。

\jie 连结半径 $OA$、$OB$, 作 $\triangle OAB$ 的高 $OG$,得 $Rt \triangle OGB$(图 \ref{fig:czjh2-7-69})。

$\because$ \quad $\angle GOB = \dfrac{360^\circ}{2n} = 30^\circ$,

$\therefore$ \quad $a_6 = 2 R \sin 30^\circ = 2 \cdot \exdfrac{R}{2} = R$。

$\therefore$ \quad $p_6 = 6 \cdot a_6 = 6 R$。

$\because$ \quad $r_6 = R \cos 30^\circ = \dfrac{\sqrt{3}}{2} R$,

$\therefore$ \quad $S_6 = 6 \cdot \exdfrac{1}{2} \cdot r_6 \cdot a_6 = \dfrac{3\sqrt{3}}{2} R^2$。


\begin{lianxi}

\xiaoti{已知圆的半径为 $R$,求它的内接正三角形、正方形的边长、边心距及面积。}

\xiaoti{在半径为 20 cm 的圆中,利用三角函数表,计算内接正九边形的边长、边心距、周长及面积(保留四个有效数字)。}

\xiaoti{一个正七边形的边长为 12 cm, 利用三角函数表,计算它的面积(保留三个有效数字)。}

\xiaoti{设正三角形的边长为 $a$,求它的边心距、半径和高。并证明:
    $\text{边心距} : \text{半径} : \text{高} = 1 : 2 : 3$。
}

\end{lianxi}

\end{enhancedline}
