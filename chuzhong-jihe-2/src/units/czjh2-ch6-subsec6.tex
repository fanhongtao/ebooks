\subsection{相似三角形}\label{subsec:czjh2-6-6}
\begin{enhancedline}

前面我们学过,有一些图形的形状相同,但大小不一定相同。
我们知道两个全等三角形的形状相同,大小也相同。
有些三角形的形状是相同的,但大小不一定相同。
如图 \ref{fig:czjh2-6-16} 中, $\triangle ABC$, $\triangle A'B'C'$, $\triangle A''B''C''$
就是形状相同大小不同的三角形。

\begin{figure}[htbp]
    \centering
    \begin{tikzpicture}
    \begin{scope}
        \tkzDefPoints{0/0/A, 2/0/B, 1.5/1.3/C}
        \tkzDrawPolygon(A,B,C)
        \tkzLabelPoints[left](A)
        \tkzLabelPoints[right](B)
        \tkzLabelPoints[above](C)
    \end{scope}

    \begin{scope}[xshift=3.5cm, scale=1.6]
        \tkzDefPoints{0/0/A', 2/0/B', 1.5/1.3/C'}
        \tkzDrawPolygon(A',B',C')
        \tkzLabelPoints[left](A')
        \tkzLabelPoints[right](B')
        \tkzLabelPoints[above](C')
    \end{scope}

    \begin{scope}[xshift=8.2cm,scale=0.8]
        \tkzDefPoints{0/0/B'', 2/0/A'', 0.5/1.3/C''}
        \tkzDrawPolygon(A'',B'',C'')
        \tkzLabelPoints[right](A'')
        \tkzLabelPoints[left](B'')
        \tkzLabelPoints[above](C'')
    \end{scope}
\end{tikzpicture}


    \caption{}\label{fig:czjh2-6-16}
\end{figure}


仅依靠观察是不能确定两个三角形的形状是否相同的。
因而,必须研究两个形状相同的三角形之间有什么关系。
为此,我们来测量 $\triangle ABC$ 与 $\triangle A'B'C'$ 的各边和各角,可以得出:
\begin{gather*}
    \angle A = \angle A' \douhao \angle B = \angle B' \douhao \angle C = \angle C' \douhao \\
    \dfrac{AB}{A'B'} = \dfrac{BC}{B'C'} = \dfrac{CA}{C'A'} \juhao
\end{gather*}
这就是说,这两个三角形的对应角都相等,对应边都成比例。

两个对应角相等,对应边成比例的三角形,叫做\zhongdian{相似三角形}。
相似用符号“$\xiangsi$”来表示,读作“相似于”。
如图 \ref{fig:czjh2-6-16} 中的 $\triangle ABC$ 与 $\triangle A'B'C'$ 相似,记作
$$ \triangle ABC \xiangsi \triangle A'B'C' \juhao $$

和记两个三角形全等一样,在记两个三角形相似时,通常把表示对应顶点的字母写在对应的位置上,
这使得我们可以比较容易地找出相似三角形的对应角和对应边。

现在我们来研究下面两个图形。

如图 \ref{fig:czjh2-6-17}, $\triangle ABC$ 中, $EF \pingxing BC$,可得

$\left.\begin{aligned}
    EF \pingxing BC \tuichu & \left\{\begin{aligned}
                                & \dfrac{AE}{AB} = \dfrac{AF}{AC} = \dfrac{EF}{BC} \\
                                & \angle 1 = \angle B \\
                                & \angle 2 = \angle C
                              \end{aligned}\right. \\
                            & \quad \angle A = \angle A
\end{aligned}\right\} \tuichu \triangle AEF \xiangsi \triangle ABC$。

\begin{figure}[htbp]
    \centering
    \begin{minipage}[b]{7cm}
        \centering
        \begin{tikzpicture}
    \tkzDefPoints{0/0/B, 4/0/C, 2.8/2.8/A}
    \tkzDefPointOnLine[pos=0.4](A,B)  \tkzGetPoint{E}
    \tkzDefPointOnLine[pos=0.4](A,C)  \tkzGetPoint{F}

    \tkzDrawPolygon(A,B,C)
    \tkzDrawSegment(E,F)
    \tkzMarkAngle[size=0.75](F,E,A)
    \tkzLabelAngle[pos=0.5](F,E,A){$1$}
    \tkzMarkAngle[size=0.65](A,F,E)
    \tkzLabelAngle[pos=0.4](A,F,E){$2$}
    \tkzLabelPoints[above](A)
    \tkzLabelPoints[left](B,E)
    \tkzLabelPoints[right](C,F)
\end{tikzpicture}


        \caption{}\label{fig:czjh2-6-17}
    \end{minipage}
    \qquad
    \begin{minipage}[b]{7cm}
        \centering
        \begin{tikzpicture}
    \tkzDefPoints{0/0/B, 4/0/C, 2.2/1.6/A}
    \tkzDefPointOnLine[pos=1.5](B,A)  \tkzGetPoint{E}
    \tkzDefPointOnLine[pos=1.5](C,A)  \tkzGetPoint{F}

    \tkzDrawPolygon(A,B,C)
    \tkzDrawPolygon(A,E,F)
    \tkzLabelPoints[left=.5em](A)
    \tkzLabelPoints[left](B,F)
    \tkzLabelPoints[right](C,E)
\end{tikzpicture}


        \caption{}\label{fig:czjh2-6-18}
    \end{minipage}
\end{figure}

类似地,可以证明图 \ref{fig:czjh2-6-18} 中,当 $EF \pingxing BC$ 时, $\triangle AEF \xiangsi \triangle ABC$。

由此,可以得到下面的定理:

\begin{dingli}[定理]
    平等于三角形一边的直线和其他两边(或两边的延长线)相交,所构成的三角形与原三角形相似。
\end{dingli}

相似三角形对应边的比,叫做两个相似三角形的\zhongdian{相似比}(或\zhongdian{相似系数})。
但要注意,如图 \ref{fig:czjh2-6-16}, $\triangle ABC$ 与 $\triangle A'B'C'$ 的相似比是 $k_1$,
$\triangle A'B'C'$ 与 $\triangle ABC$ 的相似比是 $k_2$,$k_1 = \dfrac{1}{k_2}$。
只有当 $\triangle ABC \quandeng \triangle A'B'C'$ 时,相似比 $k_1 = k_2 = 1$。


\begin{lianxi}

\xiaoti{}%
\begin{xiaoxiaotis}%
    \xxt[\xxtsep]{所有的等腰三角形都相似吗?所有的等边三角形呢?为什么?}

    \xxt{所有的直角三角形都相似吗?所有的等腰直角三角形呢?为什么?}

\end{xiaoxiaotis}


\xiaoti{已知:如图,}
\begin{xiaoxiaotis}

\begin{figure}[htbp]
    \centering
    \begin{minipage}[b]{4.5cm}
        \centering
        \begin{tikzpicture}
    \tkzDefPoints{0/0/B, 3/0/C, 2.0/2.5/A}
    \tkzDefPointOnLine[pos=0.5](A,B)  \tkzGetPoint{D}
    \tkzDefPointOnLine[pos=0.5](A,C)  \tkzGetPoint{E}

    \tkzDrawPolygon(A,B,C)
    \tkzDrawSegment(D,E)
    \tkzLabelPoints[above](A)
    \tkzLabelPoints[left](B,D)
    \tkzLabelPoints[right](C,E)
\end{tikzpicture}


        \caption*{(1)}
    \end{minipage}
    \qquad
    \begin{minipage}[b]{4.5cm}
        \centering
        \begin{tikzpicture}
    \tkzDefPoints{0/0/A, 3/0/B, 2.2/1.5/O}
    \tkzDefPointOnLine[pos=1.5](A,O)  \tkzGetPoint{A'}
    \tkzDefPointOnLine[pos=1.5](B,O)  \tkzGetPoint{B'}

    \tkzDrawPolygon(O,A,B)
    \tkzDrawPolygon(O,A',B')
    \tkzLabelPoints[left=.5em](O)
    \tkzLabelPoints[left](A,B')
    \tkzLabelPoints[right](B,A')
\end{tikzpicture}


        \caption*{(2)}
    \end{minipage}
    \qquad
    \begin{minipage}[b]{4.5cm}
        \centering
        \input{../pic/czjh2-ch6-subsec6-lx-02-c}
        \caption*{(3)}
    \end{minipage}
    \caption*{(第 2 题)}
\end{figure}

    \xxt{$\triangle ABC \xiangsi \triangle ADE$,其中 $DE \pingxing BC$;}

    \xxt{$\triangle OAB \xiangsi \triangle OA'B'$,其中 $A'B' \pingxing AB$;}

    \xxt{$\triangle ABC \xiangsi \triangle ADE$,其中 $\angle ADE = \angle B$。}

    \qquad 写出各组相似三角形的对应边的比例式。

\end{xiaoxiaotis}



\xiaoti{$\triangle ABC$ 中, $BC = 52$ 厘米,$CA = 46$ 厘米, $AB = 63$ 厘米。
    另一个和它相似的三角形的最短边为 12 厘米,求其余两边的长度。
}

\xiaoti{已知: $\triangle ABC \xiangsi \triangle A_1B_1C_1$,
    $\triangle A_1B_1C_1 \xiangsi \triangle A_2B_2C_2$。
    求证:$\triangle ABC \xiangsi \triangle A_2B_2C_2$。
}

\end{lianxi}
\end{enhancedline}

