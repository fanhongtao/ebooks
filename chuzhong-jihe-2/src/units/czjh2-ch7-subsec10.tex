\subsection{三角形的内切圆}\label{subsec:czjh2-7-10}

从一块三角形的材料上裁下一块圆形的用料,怎样才能使圆的半径尽可能大呢?这实际是下面的问题。

\zhongdian{作圆,使它和已知三角形的各边都相切。}

已知: $\triangle ABC$ (图 \ref{fig:czjh2-7-39})。

求作: 和 $\triangle ABC$ 的各边都相切的圆。

分析:要作一个圆与  $\triangle ABC$ 三边都相切,就是要求出一点作为圆心,使它到三边的距离相等。
以前我们学过三角形三个内角的平分线相交于一点,这一点到三边的距离相等。由此可得三角形内切圆的作法。

\zuofa 1. 作 $\angle B$、$\angle C$ 的平分线 $BM$ 和 $CN$,交点为 $I$。

2. 过点 $I$ 作 $ID \perp BC$,垂足为 $D$。

3. 以 $I$ 为圆心, $ID$ 为半径作 $\yuan\,I$。

$\yuan\,I$ 就是所求的圆。

\zhengming 过点 $I$ 分别作 $CA$、$AB$ 的垂线,垂足为 $E$、$F$。

$\because$ \quad  $I$ 在 $\angle ABC$、$\angle ACB$ 的平分线上,

$\therefore$ \quad $IF = ID$,$IE = ID$。

$\therefore$ \quad $D$、$E$、$F$ 都在 $\yuan\,I$ 上。

又因为 $BC$、$CA$、$AB$ 经过点 $D$、$E$、$F$, 且 $BC \perp ID$、$CA \perp IE$、$AB \perp IF$,
所以 $\triangle ABC$ 的三边 $BC$、$CA$、$AB$ 都与 $\yuan\,I$ 相切。

因为三角形的三条角平分线有一个且只有一个交点,所以,
和三角形的各边都相切的圆可以作出一个且只可以作出一个。

和三角形各边都相切的圆叫做\zhongdian{三角形的内切圆},
内切圆的圆心叫做三角形的\zhongdian{内心},
这个三角形叫做\zhongdian{圆的外切三角形}。

一般地,和多边形的各边都相切的圆叫做\zhongdian{多边形的内切圆},
这个多边形叫做\zhongdian{圆的外切多边形}。

\begin{figure}[htbp]
    \centering
    \begin{minipage}[b]{7cm}
        \centering
        \begin{tikzpicture}
    \tkzDefPoints{0/0/B, 4/0/C, 3.5/2.5/A}
    \tkzDrawPolygon(A,B,C)
    \tkzLabelPoints[above](A)
    \tkzLabelPoints[left](B)
    \tkzLabelPoints[right](C)

    % 1
    \tkzDefLine[bisector](C,B,A)  \tkzGetPoint{M}
    \tkzDefLine[bisector](A,C,B)  \tkzGetPoint{N}
    \tkzInterLL(B,M)(C,N)  \tkzGetPoint{I}
    \tkzDrawSegments(B,M  C,N)
    \tkzLabelPoints[above](I)
    \tkzLabelPoints[left,yshift=.5em](M)
    \tkzLabelPoints[left,xshift=.3em](N)

    % 2
    \tkzDefLine[altitude](B,I,C)  \tkzGetPoint{D}
    \tkzDrawSegment(I,D)
    \tkzMarkRightAngle[size=.2](I,D,B)
    \tkzLabelPoints[below](D)

    % 3
    \tkzDrawCircle[thick](I,D)

    % 证明
    \tkzDefLine[altitude](C,I,A)  \tkzGetPoint{E}
    \tkzDefLine[altitude](A,I,B)  \tkzGetPoint{F}
    \tkzDrawSegments[dashed](I,E  I,F)
    \tkzLabelPoints[right](E)
    \tkzLabelPoints[above left](F)
\end{tikzpicture}


        \caption{}\label{fig:czjh2-7-39}
    \end{minipage}
    \qquad
    \begin{minipage}[b]{7cm}
        \centering
        \begin{tikzpicture}
    % 圆的外切四边形的两组对边的和相等
    % AB + CD  = AD  + BC
    % 4  + 2.5 = 3.5 + 3
    \tkzDefPoints{0/0/A, 4/0/B}
    \tkzDefPoint(70:3.5){D}
    \tkzInterCC[R](D,2.5)(B,3)  \tkzGetFirstPoint{C}

    % 切线长定理
    \tkzDefLine[bisector](B,A,D)  \tkzGetPoint{a}
    \tkzDefLine[bisector](C,B,A)  \tkzGetPoint{b}
    \tkzInterLL(A,a)(B,b)  \tkzGetPoint{O}
    \tkzDefLine[altitude](A,O,B)  \tkzGetPoint{L}
    \tkzDefLine[altitude](B,O,C)  \tkzGetPoint{M}
    \tkzDefLine[altitude](C,O,D)  \tkzGetPoint{N}
    \tkzDefLine[altitude](D,O,A)  \tkzGetPoint{P}

    \tkzDrawPolygon(A,B,C,D)
    \tkzDrawCircle[thick](O,L)
    \tkzDrawPoint(O)
    \tkzLabelPoints[left](A,D,P)
    \tkzLabelPoints[right](B,C,M)
    \tkzLabelPoints[above](N)
    \tkzLabelPoints[below](L,O)
\end{tikzpicture}


        \caption{}\label{fig:czjh2-7-40}
    \end{minipage}
\end{figure}

图 \ref{fig:czjh2-7-40} 中, $\yuan\,O$ 是四边形 $ABCD$ 的内切圆,四边形 $ABCD$ 是 $\yuan\,O$ 的外切四边形。


\liti[0] \zhongdian{圆的外切四边形的两组对边的和相等。}

已知:四边形 $ABCD$ 的边 $AB$、$BC$、$CD$、$DA$ 和 $\yuan\,O$
分别相切于点 $L$、$M$、$N$、$P$(图 \ref{fig:czjh2-7-40})。

求证: $AB + CD = AD + BC$。

证明: 因为 $AB$、$BC$、$CD$、$DA$ 都与 $\yuan\,O$ 相切,$L$、$M$、$N$、$P$ 是切点,

$\therefore$ \quad $AL = AP$, $LB = MB$, $DN = DP$, $NC = MC$。

$\therefore$ \quad $\begin{aligned}[t]
      & AL + LB + DN + NC \\
    = & AP + MB + DP + MC \\
    = & AP + DP + MB + MC \douhao
\end{aligned}$

即 \quad $AB + CD = AD + BC$。


\begin{lianxi}

\xiaoti{设 $\triangle ABC$ 的内切圆 $I$ 和各边分别相切于点 $D$、$E$、$F$,
    $\angle DIE = 118^\circ$, $\angle FID = 144^\circ$。
    求 $\triangle ABC$ 各内角的度数。
}

\begin{figure}[htbp]
    \centering
    \begin{minipage}[b]{5.6cm}
        \centering
        \begin{tikzpicture}
    \tkzDefPoints{0/0/B, 5/0/C}
    \tkzDefTriangle[two angles=36 and 62](B,C)  \tkzGetPoint{A}
    \tkzDefLine[bisector](B,A,C)  \tkzGetPoint{a}
    \tkzDefLine[bisector](C,B,A)  \tkzGetPoint{b}
    \tkzInterLL(A,a)(B,b)  \tkzGetPoint{I}
    \tkzDefLine[altitude](B,I,C)  \tkzGetPoint{D}
    \tkzDefLine[altitude](C,I,A)  \tkzGetPoint{E}
    \tkzDefLine[altitude](A,I,B)  \tkzGetPoint{F}

    \tkzDrawPolygon(A,B,C)
    \tkzDrawCircle[thick](I,D)
    \tkzDrawSegments(I,D  I,E  I,F)
    \extkzLabelAngel[0.4](F,I,D){$144^\circ$}
    \extkzLabelAngel[0.3](D,I,E){$118^\circ$}

    \tkzLabelPoints[above](A)
    \tkzLabelPoints[above, xshift=.2em](I)
    \tkzLabelPoints[below](B,C,D)
    \tkzLabelPoints[right, yshift=.5em](E)
    \tkzLabelPoints[left, yshift=.5em](F)
\end{tikzpicture}


        \caption*{(第 1 题)}
    \end{minipage}
    \qquad
    \begin{minipage}[b]{4.6cm}
        \centering
        \begin{tikzpicture}
    \tkzDefPoints{0/0/B, 4/0/C, 3/2.5/A}
    \tkzDefLine[bisector](B,A,C)  \tkzGetPoint{a}
    \tkzDefLine[bisector](C,B,A)  \tkzGetPoint{b}
    \tkzInterLL(A,a)(B,b)  \tkzGetPoint{I}
    \tkzDefLine[altitude](B,I,C)  \tkzGetPoint{D}
    \tkzDefLine[altitude](C,I,A)  \tkzGetPoint{E}
    \tkzDefLine[altitude](A,I,B)  \tkzGetPoint{F}

    \tkzDrawPolygon(A,B,C)
    \tkzDrawCircle[thick](I,D)
    \tkzDrawSegments[dashed](I,D  I,E  I,F)

    \tkzLabelPoints[above](A)
    \tkzLabelPoints[above, xshift=.2em](I)
    \tkzLabelPoints[below](B,C,D)
    \tkzLabelPoints[right, yshift=.5em](E)
    \tkzLabelPoints[left, yshift=.5em](F)
\end{tikzpicture}


        \caption*{(第 2 题)}
    \end{minipage}
    \qquad
    \begin{minipage}[b]{4.9cm}
        \centering
        \begin{tikzpicture}
    \tkzDefPoints{0/0/A, 4/0/C}
    \tkzDefTriangle[two angles=30 and 90](A,C)  \tkzGetPoint{B}
    \tkzDefLine[bisector](B,A,C)  \tkzGetPoint{a}
    \tkzDefLine[bisector](C,B,A)  \tkzGetPoint{b}
    \tkzInterLL(A,a)(B,b)  \tkzGetPoint{I}
    \tkzDefLine[altitude](B,I,C)  \tkzGetPoint{D}
    \tkzDefLine[altitude](C,I,A)  \tkzGetPoint{E}
    \tkzDefLine[altitude](A,I,B)  \tkzGetPoint{F}

    \tkzDrawPolygon(A,B,C)
    \tkzDrawCircle[thick](I,D)
    \tkzDrawSegments(I,D  I,E)
    \tkzMarkRightAngle[size=.2](B,C,A)

    \tkzLabelPoints[below](A,C,E)
    \tkzLabelPoints[left](I)
    \tkzLabelPoints[right](B,D)
    \tkzLabelPoints[below](E)
    \tkzLabelPoints[left, yshift=.5em](F)
\end{tikzpicture}


        \caption*{(第 3 题)}
    \end{minipage}
\end{figure}

\xiaoti{设 $\triangle ABC$ 的边 $BC = a$、 $CA = b$、 $AB = c$, $s = \exdfrac{1}{2} (a + b + c)$,
    内切圆 $I$ 和 $BC$、$AC$、$AB$ 分别相切于点 $D$、$E$、$F$。求证:
    \begin{gather*}
        AB = AF = s - a \douhao \\
        BF = BD = s - b \douhao \\
        CD = CE = s - c \juhao
    \end{gather*}
}

\xiaoti{$\triangle ABC$ 中, $\angle C$ 是直角, 内切圆 $I$ 和边 $BC$、$CA$、$AB$ 分别相切于点 $D$、$E$、$F$。}
\begin{xiaoxiaotis}

    \xxt{求证:四边形 $CDIE$ 是正方形。}

    \xxt{设 $BC = a$、 $CA = b$。 用 $a$、$b$ 表示内切圆半径 $r$。}
\end{xiaoxiaotis}

\end{lianxi}
