\xiti
\begin{xiaotis}

\xiaoti{用厚纸做一个正四棱台的模型。}

\xiaoti{已知上、下底面的边长和侧棱的长分别是 $a$、$b$、$c$。求下面各棱台的高和斜高:
    (1)正三棱台; (2)正四棱台; (3)正六棱台。
}

\xiaoti{正四棱台上下底面的边长分别是 $a$ 和 $b$,侧面和底面成 $45^\circ$ 的二面角。 求它的斜高和侧棱长。}

\begin{enhancedline}
\xiaoti{棱台的上、下底面的面积各是 $Q'$ 和 $Q$。 求证:这个棱台的高和截得这个棱台的原棱锥的高的比是 $\dfrac{Q - \sqrt{QQ'\;}}{Q}$。}

\xiaoti{一个正三棱锥底面边长是 10 cm,高是 15 cm。 中截面把棱锥分成一个小棱锥和一个棱台。选择适当的比例尺,画出它们的直观图。}
\end{enhancedline}

\xiaoti{一个正三棱台的上、下底面的边长分别是 3 cm 和 6 cm,侧面与底面成 $60^\circ$ 的二面角。 求它的全面积。}

\xiaoti{正四棱台的高是 12 cm,两底面的边长相差 10 cm,全面积是 $512\;\pflm$。求两底面的边长。}

\xiaoti{已知正六棱台的两底边长分别是 $a$、$2a$,高是 $a$。求这个正六棱台的侧面积以及过相对侧棱的截面面积。}

\xiaoti{正四棱台的上、下底面的边长各为 $a$、$b$,侧面积等于两底面积的和。它的高是多少?}

\xiaoti{把一个棱锥用平行于底面的平面截成棱台,使棱台上、下底面积的比为 $1:2$,求截平面的位置。}

\xiaoti{棱锥的中截面把它截成两部分。求这两部分的侧面积的比。}

\xiaoti{在一个正四棱台内有一个以它的上底面为底面,下底面中心为顶点的棱锥。
    如果棱台的上、下底面边长分别为 3 cm 和 4 cm,棱锥与棱台的侧面积相等,求棱台的高。
}

\end{xiaotis}

