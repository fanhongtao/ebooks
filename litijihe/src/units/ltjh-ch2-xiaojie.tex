\xiaojie

一、本章的主要内容是多面体和旋转体中常见的柱、锥、台、球的概念、性质、直观图的画法以及面积、体积的计算。
重点研究了应用比较广泛的直棱柱、正棱锥、正棱台、圆柱、圆锥、圆台、球和球缺。


二、这些几何体的性质都是在第一章线面关系的基础上由定义推出来的。这些性质包括:
它们的棱、面的性质;平行于底面的截面的性质;经过侧棱(或高线、轴线)的截面(或它的一部分)的性质。
通过这样的研究,我们对这些几何体就有了一个比较全面的认识。


三、本章介绍了两种直观图的画法:斜二测和正等测。画图时,可以根据情况任选一种。
画多面体时,常用斜二测,画旋转体时,常用正等测。 画多面体和旋转体组合图形时,多用正等测。
这时,要注意不要两种方法混用。


四、几种多面体和旋转体的表面积,除球面和球冠外,都是通过它们的展开图求得的。
这些公式不但互相区别,而且互相联系。
除前面讲过的关系外,直棱柱、正棱锥、正棱台、圆柱、圆锥、圆台的侧面积公式,
还可以统一写成 $S_\text{侧} = c_0 \, l$,其中 $c_0$ 是中截面周长,$l$ 分别是侧棱、斜高或母线长。
球面、球冠、球带的面积,可以统一写成 $S = 2 \pi Rh$,其中 $R$ 是球的半径,$h$ 是高(或直径)。

\begin{enhancedline}
五、几种多面体和旋转体的体积公式是在两个体积公理的基础上推导出来的。
在这一章里,我们是把柱体、锥体、台体当作不同的几何体定义的。
如果把柱体、锥体当作台体的特殊形式,那么它们,甚至包括球体的体积公式,
都可以统一写成 $V = \exdfrac{1}{6} h (S + 4S_0 + S')$,
其中 $S$、$S'$是上、下底面积,$S_0$ 是中截面面积,$h$ 是高。
\end{enhancedline}


六、本章公式较多,为了便于记忆和应用,把它们列成\hyperref[ch:gongshibiao]{公式表},放在本书的附录中。

