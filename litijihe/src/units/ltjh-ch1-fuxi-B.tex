{\centering \nonumsubsection{B \hspace{1em} 组}}
\begin{xiaotis}
\setcounter{cntxiaoti}{14}

\xiaoti{$a$、$b$ 是异面直线,平面 $\alpha$ 经过直线 $b$ 与直线 $a$ 平行。
    平面 $\beta$ 经过直线 $a$ 与平面 $\alpha$ 相交于直线 $c$。求证:
    (1)直线 $b$、$c$ 所夹的不大于直角的角就是异面直线 $a$、$b$ 所成的角;
    (2)如果 $\alpha \perp \beta$, $b \cap c = A$,在平面 $\beta$ 内,作 $AB \perp c$ 交直线 $a$ 于点 $B$,
        那么线段 $AB$ 就是异面直线 $a$、$b$ 的公垂线,直线 $a$ 与平面 $\alpha$ 的距离就是异面直线 $a$、$b$ 的距离。
}

\xiaoti{两个不全等的三角形不在同一平面内,它们的边两两对应平行。 证明:
    (1)三条对应顶点的连线交于一点;
    (2)这两个三角形相似。
}

\xiaoti{直线 $a$ 与 $b$ 不平行,如果 $\alpha \perp a$, $\beta \perp b$,
    那么平面 $\alpha$ 与 $\beta$ 必定相交,并且交线必垂直于直线 $a$、$b$。
}

\xiaoti{}%
\begin{xiaoxiaotis}%
    \xxt[\xxtsep]{由平面 $\alpha$ 外一点 $P$ 引平面的三条相等的斜线段,斜足分别为 $A$、$B$、$C$,
        $O$为 $\triangle ABC$ 的外心。求证:$OP \perp \alpha$。
    }

    \xxt{平面 $ABC$ 外一点 $P$ 到 $\triangle ABC$ 三边的距离相等, $O$ 是 $\triangle ABC$ 内的一点,
        且 $OP \perp \text{平面}\;ABC$。求证:点 $O$ 是 $\triangle ABC$ 的内心。
    }

\end{xiaoxiaotis}


\xiaoti{夹在互相垂直的两个平面之间长为 $2a$ 的线段,和这两个平面所成的角分别为 $45^\circ$、$30^\circ$,
    过这条线段的两个端点分别在这两个平面内作交线的垂线。求两垂足的距离。
}

\xiaoti{平面 $\alpha$ 过 $\triangle ABC$ 的重心 $G$。 求证:在平面 $\alpha$ 同侧的
    两个顶点到平面 $\alpha$ 的距离的和,等于另一顶点到平面的距离。
}

\xiaoti{已知:平面 $\alpha$ 和空间两点 $A$、$B$。 在平面 $\alpha$ 内找一点 $C$,使 $AC + BC$ 最小。}

\end{xiaotis}
