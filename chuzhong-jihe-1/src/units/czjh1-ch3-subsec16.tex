\subsection{角平分线}\label{subsec:czjh1-3-16}

仿照对线段垂直平分线的讨论,我们可以得到下面的定理:

\begin{dingli}[定理]
    在角的平分线上的点到这个角的两边的距离相等。
\end{dingli}

已知: $OC$ 是 $\angle AOB$ 的平分线, 点 $P$ 在 $OC$ 上, $PD \perp OA$,
$PE \perp OB$,垂足分别是 $D$、$E$(图 \ref{fig:czjh1-3-57})。

求证: $PD = PE$。

\zhengming $\because$ \quad $PD \perp OA$, $PE \perp OB$,

$\therefore$ \quad $\angle PDO = \angle PEO = Rt \angle$。

在 $\triangle PDO$ 和 $\triangle PEO$ 中,

$\angle PDO = \angle PEO$, $\angle AOC = \angle BOC$, $OP = OP$,

$\therefore$ \quad $\triangle PDO \quandeng \triangle PEO$ ($AAS$) 。

$\therefore$ \quad $PD = PE$。

\begin{figure}[htbp]
    \centering
    \begin{minipage}[b]{7cm}
        \centering
        \input{../pic/czjh1-ch3-57}
        \caption{}\label{fig:czjh1-3-57}
    \end{minipage}
    \qquad
    \begin{minipage}[b]{7cm}
        \centering
        \begin{tikzpicture}
    \tkzDefPoints{0/0/O, 4/0/C}
    \tkzDefPoint(30:3.5){A}
    \tkzDefPoint(-30:3.5){B}
    \tkzDefPointOnLine[pos=0.8](O,C)  \tkzGetPoint{P}
    \tkzDefLine[altitude](O,P,A)  \tkzGetPoint{D}
    \tkzDefLine[altitude](O,P,B)  \tkzGetPoint{E}

    \tkzDrawSegments(O,A  O,B  P,D  P,E)
    \tkzDrawSegments[dashed](O,C)
    \tkzMarkRightAngle(P,D,O)
    \tkzMarkRightAngle(P,E,O)
    \tkzLabelPoints[left](O)
    \tkzLabelPoints[above left](D)
    \tkzLabelPoints[above](A)
    \tkzLabelPoints[below](B)
    \tkzLabelPoints[below, xshift=0.3em](P)
    \tkzLabelPoints[below left](E)
    \tkzLabelPoints[above](C)
\end{tikzpicture}


        \caption{}\label{fig:czjh1-3-58}
    \end{minipage}
\end{figure}


\begin{dingli}[逆定理]
    到一个角的两边的距离相等的点,在这个角的平分线上。
\end{dingli}

已知: $PD \perp OA$, $PE \perp OB$,垂足分别是 $D$、$E$, $PD = PE$(图 \ref{fig:czjh1-3-58} )。

求证: 点 $P$ 在 $\angle AOB$ 的平分线上。

\zhengming 经过点 $P$ 作射线 $OC$,

$\because$ \quad $PD \perp OA$, $PE \perp OB$,

$\therefore$ \quad $\angle PDO = \angle PEO = Rt \angle$。

在 $Rt \triangle PDO$ 和 $Rt \triangle PEO$ 中,

$OP = OP$, $PD = PE$,

$\therefore$ \quad $Rt \triangle PDO \quandeng Rt \triangle PEO$ ($HL$)。

$\therefore$ \quad $\angle AOC = \angle BOC$。

$\therefore$ \quad $OC$ 是 $\angle AOB$ 的平分线。

就是说,点 $P$ 在 $\angle AOB$ 的平分线上。

根据上述的定理和逆定理可以知道:
\zhongdian{角的平分线可以看作是到角的两边的距离相等的所有点的集合。}

\liti[0] 求证: 三角形三条角平分线交一点。

已知: $AM$、$BN$、$CP$ 是 $\triangle ABC$ 的三条角平分线(图 \ref{fig:czjh1-3-59})。

求证: $AM$、$BN$、$CP$ 交于一点。

\zhengming 设 $AM$、$BN$ 交于点 $R$。 过点 $R$ 分别作边 $BC$、 $AC$、 $AB$ 的垂线,垂足分别为 $D$、$E$、$F$。

$\because$ \quad 点 $R$ 在 $AM$ 上,

$\therefore$ \quad $RE = RF$(在角的平分线上的点到这个角的两边的距离相等)。

同理 \quad $RD = RF$。

$\therefore$ \quad $RD = RE$。

$\therefore$ \quad 点 $R$ 在 $CP$ 上(到一个角的两边的距离相等的点,在这个角的平分线上)。

因此,$AM$、$BN$、$CP$ 交于一点。

\begin{figure}[htbp]
    \centering
    \begin{minipage}[b]{7cm}
        \centering
        \begin{tikzpicture}
    \tkzDefPoints{0/0/B, 4/0/C, 3.6/2.8/A}
    \tkzDefSpcTriangle[in](A,B,C){M,N,P}
    \tkzInterLL(A,M)(B,N)  \tkzGetPoint{R}
    \tkzDefSpcTriangle[intouch](A,B,C){D,E,F}
    \tkzDrawPolygon(A,B,C)
    \tkzDrawSegments(A,M  B,N  C,P)
    \tkzDrawSegments[dashed](R,D  R,E  R,F)
    \tkzMarkRightAngle[size=0.2](R,F,A)
    \tkzMarkRightAngle[size=0.2](R,D,C)
    \tkzMarkRightAngle[size=0.2](R,E,C)
    \tkzLabelPoints[above](A)
    \tkzLabelPoints[left](B)
    \tkzLabelPoints[right](C)
    \tkzLabelPoints[below](D)
    \tkzLabelPoints[right](E)
    \tkzLabelPoints[above](F)
    \tkzLabelPoints[below,xshift=-0.2em](M)
    \tkzLabelPoints[above right](N)
    \tkzLabelPoints[left](P)
    \tkzLabelPoints[above right](R)
\end{tikzpicture}


        \caption{}\label{fig:czjh1-3-59}
    \end{minipage}
    \qquad
    \begin{minipage}[b]{7cm}
        \centering
        \begin{tikzpicture}
    \tkzDefPoints{0/0/O, 4/0/A, 3/3/B,  0/1.5/M, 4/1.5/N}
    \tkzDrawSegments(O,A  O,B  M,N)
    \tkzLabelPoints[below](O,A)
    \tkzLabelPoints[left](B)
    \tkzLabelPoints[above](M,N)
\end{tikzpicture}


        \caption*{(第 1 题)}
    \end{minipage}
\end{figure}

\begin{lianxi}

\xiaoti{如图,在直线 $MN$ 上求作一点 $P$,使点 $P$ 到射线 $OA$ 和 $OB$ 的距离相等。}

\xiaoti{已知:$\triangle ABC$。 求作:一点 $P$,使 $P$ 到 $BC$、$CA$、$AB$ 的距离都相等。}

\end{lianxi}

