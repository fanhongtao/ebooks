\xiti
\begin{xiaotis}

\xiaoti{在同一平面内,两条直线没有公共点,它们的位置关系怎样?有一个而且只有一个公共点呢?有两个公共点呢?}

\xiaoti{画线段 $AB = 45\;\haomi$。 画任意射线 $AX$。 在 $AX$ 上取 $C'$、$D'$、 $B'$ 三点,
    使 $AC' = C'D' = D'B'$。 连结 $BB'$。用三角板画 $C'C \pingxing B'B$、$D'D \pingxing B'B$,
    分别交 $AB$ 于 $C$、$D$。 量 $AC$、$CD$、$DB$ 的长(精确到 $1\;\haomi$)。
}

\xiaoti{用三角板和直尺画平行线:}
\begin{figure}[htbp]
    \centering
    \begin{minipage}[b]{4.5cm}
        \centering
        \begin{tikzpicture}
    \tkzDefPoints{2.1/2/A, 0/0/B, 3/0/C}

    \tkzDrawPolygon(A,B,C)
    \tkzLabelPoints[above](A)
    \tkzLabelPoints[left](B)
    \tkzLabelPoints[right](C)
\end{tikzpicture}


        \caption*{甲}
    \end{minipage}
    \qquad
    \begin{minipage}[b]{4.5cm}
        \centering
        \begin{tikzpicture}
    \tkzDefPoints{0/0/O, 2.5/2/A, 3/0/B,  2.3/1.0/P}

    \tkzDrawSegments(O,A  O,B)
    \tkzDrawPoint[fill=black](P)
    \tkzLabelPoints[left](O)
    \tkzLabelPoints[right](A,B,P)
\end{tikzpicture}


        \caption*{乙}
    \end{minipage}
    \begin{minipage}[b]{5.0cm}
        \centering
        \begin{tikzpicture}
    \tkzDefPoints{0/0/A, 3/0/B, 2.5/2/C, 0.8/2/D}

    \tkzDrawPolygon(A,B,C,D)
    \tkzDrawSegment(B,D)
    \tkzLabelPoints[left](A,D)
    \tkzLabelPoints[right](B,C)
\end{tikzpicture}


        \caption*{丙}
    \end{minipage}
    \caption*{(第 3 题)}
\end{figure}
\begin{xiaoxiaotis}

    \xxt{过点 $A$ 画 $MN \pingxing BC$ (图甲);}

    \xxt{过点 $P$ 画 $PE \pingxing OA$,交 $OB$ 于点 $E$;
                  画 $PH \pingxing OB$, 交 $OA$ 于点 $H$(图乙);
    }

    \xxt{过点 $C$ 画 $CE \pingxing DA$,和 $AB$ 交于点 $E$;
         过点 $C$ 画 $CF \pingxing DB$,和 $AB$ 的延长线交于点 $F$(图丙)。
    }

\end{xiaoxiaotis}


\xiaoti{已知直线 $AB$ 和 $AB$ 外一点 $P$,经过点 $P$ 画直线 $CD$ 平行于 $AB$:}
\begin{xiaoxiaotis}

    \xxt{根据 “同位角相等,两直线平行” 来画;}

    \xxt{根据 “内错角相等,两直线平行” 来画。}

\end{xiaoxiaotis}


\xiaoti{如图,直线 $a$、$b$、$c$ 被直线 $l$ 所截, 量得 $\angle 1 = 78^\circ$,
    $\angle 2 = 78^\circ$, $\angle 3 = 78^\circ$。
}
\begin{xiaoxiaotis}

    \xxt{$\angle 1$ 和 $\angle 2$ 有什么关系?从这个关系可以推出哪两条直线平行?它的根据是什么?}

    \xxt{$\angle 1$ 和 $\angle 3$ 有什么关系?从这个关系可以推出哪两条直线平行?它的根据是什么?}

    \xxt{$\angle 2$ 和 $\angle 3$ 有什么关系?从这个关系可以推出哪两条直线平行?它的根据是什么?}

\end{xiaoxiaotis}

\begin{figure}[htbp]
    \centering
    \begin{minipage}[b]{7cm}
        \centering
        \begin{tikzpicture}
    \tkzDefPoints{0/0/C1, 4/0/C2, 0/1.2/B1, 4/1.2/B2, 0/2.2/A1, 4/2.2/A2,  2/0/C3}
    \tkzDefPointOnCircle[R = center C3 angle 78  radius 3] \tkzGetPoint{L1}
    \tkzDefPointOnLine[pos=1.3](L1,C3)  \tkzGetPoint{L2}

    \tkzInterLL(A1,A2)(L1,L2)  \tkzGetPoint{A3}
    \tkzInterLL(B1,B2)(L1,L2)  \tkzGetPoint{B3}

    \tkzDrawSegments(A1,A2  B1,B2  C1,C2  L1,L2)
    \tkzMarkAngles[size=0.3](A1,A3,L2  B1,B3,L2  C2,C3,L1)
    \tkzLabelAngle[pos=0.5](A1,A3,L2){$1$}
    \tkzLabelAngle[pos=0.5](B1,B3,L2){$2$}
    \tkzLabelAngle[pos=0.5](C2,C3,L1){$3$}
    \tkzLabelSegment[pos=1.0, right](A1,A2){$a$}
    \tkzLabelSegment[pos=1.0, right](B1,B2){$b$}
    \tkzLabelSegment[pos=1.0, right](C1,C2){$c$}
    \tkzLabelSegment[pos=0.0, right](L1,L2){$l$}
\end{tikzpicture}


        \caption*{(第 5 题)}
    \end{minipage}
    \qquad
    \begin{minipage}[b]{7cm}
        \centering
        \begin{tikzpicture}
    \tkzDefPoints{0/0/B, 3/0/C, 4.5/0/G, 1.2/2.5/A, 4.2/2.5/D}
    \tkzDefPointOnLine[pos=0.4](A,B)  \tkzGetPoint{E}
    \tkzDefPointOnLine[pos=0.4](D,C)  \tkzGetPoint{F}

    \tkzDrawPolygon(A,B,C,D)
    \tkzDrawSegments(C,G  E,F)
    \tkzLabelPoints[left](A,B,E)
    \tkzLabelPoints[right](D,F)
    \tkzLabelPoints[below](C,G)
\end{tikzpicture}


        \caption*{(第 6 题)}
    \end{minipage}
\end{figure}


\xiaoti{如图, $E$ 是 $AB$ 上一点, $F$ 是 $DC$ 上一点, $G$ 是 $BC$ 的廷长线上一点。}
\begin{xiaoxiaotis}

    \xxt{知道了 $\angle B =  \angle DCG$,可以断定哪两条直线平行?为什么?}

    \xxt{知道了 $\angle DCG = \angle D$, 可以断定哪两条直线平行?为什么?}

    \xxt{知道了 $\angle DFE + \angle D = 180^\circ$, 可以断定哪两条直线平行?为什么?}

\end{xiaoxiaotis}


\xiaoti{填空:}
\begin{figure}[htbp]
    \centering
    \begin{minipage}[b]{7cm}
        \centering
        \begin{tikzpicture}
    \tkzDefPoints{0/0/E, 4/0/F, 1/0/O, 3/0/P}

    \tkzDefPointOnCircle[R = center O angle 102  radius 1.5] \tkzGetPoint{A}
    \tkzDefPointOnCircle[R = center O angle -78  radius 1.5] \tkzGetPoint{B}
    \tkzDefPointOnCircle[R = center P angle 102  radius 1.5] \tkzGetPoint{C}
    \tkzDefPointOnCircle[R = center P angle -78  radius 1.5] \tkzGetPoint{D}

    \tkzDrawSegments(A,B  C,D  E,F)
    \tkzMarkAngles[size=0.3](C,P,E  F,O,A)
    \tkzMarkAngles[size=0.4](A,O,E  B,O,F)
    \tkzLabelAngle[pos=0.6](A,O,E){$1$}
    \tkzLabelAngle[pos=0.6](B,O,F){$3$}
    \tkzLabelAngle[pos=0.5](C,P,E){$2$}
    \tkzLabelAngle[pos=0.5](F,O,A){$4$}
    \tkzLabelPoints[below](E,F)
    \tkzLabelPoints[right](A,B,C,D)
\end{tikzpicture}


        \caption*{甲}
    \end{minipage}
    \qquad
    \begin{minipage}[b]{7cm}
        \centering
        \begin{tikzpicture}
    \tkzDefPoints{0/0/B, 4/0/C, 1/2/A, 5/2/D}
    \tkzDefPointOnLine[pos=1.4](B,A)  \tkzGetPoint{E}

    \tkzDrawPolygon(A,B,C,D)
    \tkzDrawSegments(A,E  A,C)
    \tkzMarkAngles[size=0.3](D,A,E  B,A,C  D,C,A)
    \tkzMarkAngles[size=0.4](C,A,D  A,C,B)
    \tkzLabelAngle[pos=0.5](D,A,E){$1$}
    \tkzLabelAngle[pos=0.5](B,A,C){$3$}
    \tkzLabelAngle[pos=0.5](D,C,A){$5$}
    \tkzLabelAngle[pos=0.6](C,A,D){$2$}
    \tkzLabelAngle[pos=0.6](A,C,B){$4$}
    \tkzLabelPoints[left](A,B)
    \tkzLabelPoints[right](C,D)
\end{tikzpicture}


        \caption*{乙}
    \end{minipage}
    \caption*{(第 7 题)}
\end{figure}
\begin{xiaoxiaotis}

    \xxt{如图甲, $\because$ \quad $\angle 1 = 78^\circ$(已知)\\
        \hspace*{2em} $\angle 2 = 78^\circ$(已知)\\
        $\therefore$ \quad $\angle 1 = \angle 2$ (\hspace*{2cm})。\\
        $\therefore$ \quad (\hspace*{1cm}) $\pingxing$ (\hspace*{1cm}) (\hspace*{2cm})。
    }

    \xxt{如图甲, $\because$ \quad $\angle 2 = 78^\circ$(已知)\\
        \hspace*{2em} $\angle 3 = 78^\circ$(已知)\\
        $\therefore$ \quad $\angle 2 = \angle 3$ (\hspace*{2cm})。\\
        $\therefore$ \quad (\hspace*{1cm}) $\pingxing$ (\hspace*{1cm}) (\hspace*{2cm})。
    }

    \xxt{如图甲, $\because$ \quad $\angle 2 = 78^\circ$(\hspace*{2cm})\\
        \hspace*{2em} $\angle 4 = 102^\circ$(\hspace*{2cm})\\
        $\therefore$ \quad $\angle 2 + \angle 4 = 180^\circ$ (等式的性质)。\\
        $\therefore$ \quad (\hspace*{1cm}) $\pingxing$ (\hspace*{1cm}) (\hspace*{2cm})。
    }

    \xxt{如图乙, \\
        \hspace*{3em}\begin{zmtblr}{column{1}={2.5em}}
            (i)   & $\because$ \quad $\angle 1 = \angle B$(已知)\\
                  & $\therefore$ \quad (\hspace*{1cm}) $\pingxing$ (\hspace*{1cm}) (\hspace*{2cm});\\
            (ii)  & $\because$ \quad $\angle 3 = \angle 5$(已知)\\
                  & $\therefore$ \quad (\hspace*{1cm}) $\pingxing$ (\hspace*{1cm}) (\hspace*{2cm});\\
            (iii) & $\because$ \quad $\angle 2 = \angle 4$(已知)\\
                  & $\therefore$ \quad (\hspace*{1cm}) $\pingxing$ (\hspace*{1cm}) (\hspace*{2cm});\\
            (iv)  & $\because$ \quad $\angle 1 = \angle D$(已知)\\
                  & $\therefore$ \quad (\hspace*{1cm}) $\pingxing$ (\hspace*{1cm}) (\hspace*{2cm});\\
            (v)   & $\because$ \quad $\angle B + \angle BCD = 180^\circ$(已知)\\
                  & $\therefore$ \quad (\hspace*{1cm}) $\pingxing$ (\hspace*{1cm}) (\hspace*{2cm})。\\
        \end{zmtblr}
    }
\end{xiaoxiaotis}


\xiaoti{如图,已知直线 $DE$ 经过点 $A$, $DE \pingxing BC$, $\angle B = 44^\circ$, $\angle C = 57^\circ$。}
\begin{xiaoxiaotis}

    \xxt{$\angle DAB$ 等于多少度?为什么?}

    \xxt{$\angle EAC$ 等于多少度?为什么?}

    \xxt{$\angle BAC$、$\angle BAC + \angle B + \angle C$ 各等于多少度?}

\end{xiaoxiaotis}


\begin{figure}[htbp]
    \centering
    \begin{minipage}[b]{7cm}
        \centering
        \begin{tikzpicture}
    \tkzDefPoints{0/0/B, 4/0/C}
    \tkzDefPointOnCircle[R = center B angle 44  radius 3] \tkzGetPoint{a1}
    \tkzDefPointOnCircle[R = center C angle 123 radius 3] \tkzGetPoint{a2}
    \tkzInterLL(B,a1)(C,a2)  \tkzGetPoint{A}

    \tkzDefLine[parallel=through A](B,C) \tkzGetPoint{d}
    \tkzDefPointBy[projection= onto A--d](C)  \tkzGetPoint{E}
    \tkzDefPointBy[projection= onto A--d](B)  \tkzGetPoint{D}

    \tkzDrawPolygon(A,B,C)
    \tkzDrawSegments(D,E)
    \tkzLabelPoints[above](A)
    \tkzLabelPoints[below](D,E)
    \tkzLabelPoints[left](B)
    \tkzLabelPoints[right](C)
\end{tikzpicture}


        \caption*{(第 8 题)}
    \end{minipage}
    \qquad
    \begin{minipage}[b]{7cm}
        \centering
        \begin{tikzpicture}
    \tkzDefPoints{0/0/E, 4/0/F, 0/2/A, 4.5/2/D}
    \tkzDefPointOnCircle[R = center E angle 58  radius 3] \tkzGetPoint{b}
    \tkzDefPointOnCircle[R = center F angle 102 radius 3] \tkzGetPoint{c}
    \tkzInterLL(E,b)(A,D)  \tkzGetPoint{B}
    \tkzInterLL(F,c)(A,D)  \tkzGetPoint{C}

    \tkzDrawPolygon(B,C,F,E)
    \tkzDrawSegments(A,D)
    \tkzMarkAngles[size=0.3](A,B,E  A,C,F)
    \tkzMarkAngles[size=0.4](E,B,D  F,C,D)
    \tkzLabelAngle[pos=0.5](A,B,E){$1$}
    \tkzLabelAngle[pos=0.5](A,C,F){$3$}
    \tkzLabelAngle[pos=0.6](E,B,D){$2$}
    \tkzLabelAngle[pos=0.6](F,C,D){$4$}
    \tkzLabelPoints[above](B,C)
    \tkzLabelPoints[left](A,E)
    \tkzLabelPoints[right](D,F)
\end{tikzpicture}


        \caption*{(第 9 题)}
    \end{minipage}
\end{figure}


\xiaoti{如图, $ABCD$ 成一直线,$AD \pingxing EF$, $\angle E = 58^\circ$, $\angle F = 78^\circ$。}
\begin{xiaoxiaotis}

    \xxt{$\angle E = 58^\circ$ 时, $\angle 1$、$\angle 2$ 各等于多少度?为什么?}

    \xxt{$\angle F = 78^\circ$ 时, $\angle 3$、$\angle 4$ 各等于多少度?为什么?}

\end{xiaoxiaotis}


\xiaoti{填空:}
\begin{figure}[htbp]
    \centering
    \begin{minipage}[b]{4.5cm}
        \centering
        \begin{tikzpicture}
    \tkzDefPoints{0/0/C, 3/0/D, 0.3/2/A, 2.6/2/B}

    \tkzDrawSegments(A,B  B,C  C,D)
    \tkzLabelPoints[left](A,C)
    \tkzLabelPoints[right](B,D)
\end{tikzpicture}


        \caption*{甲}
    \end{minipage}
    \qquad
    \begin{minipage}[b]{4.5cm}
        \centering
        \input{../pic/czjh1-ch2-xiti4-10-b}
        \caption*{乙}
    \end{minipage}
    \begin{minipage}[b]{5.0cm}
        \centering
        \input{../pic/czjh1-ch2-xiti4-10-c}
        \caption*{丙}
    \end{minipage}
    \caption*{(第 10 题)}
\end{figure}
\begin{xiaoxiaotis}

    \xxt{如图甲, $\because$ \quad $AB \pingxing CD$(已知)\\
        $\therefore$ \quad $\angle B = \angle C$ (\hspace*{2cm})。
    }

    \xxt{如图乙, $\because$ \quad $\angle ADE = \angle B$(已知)\\
        $\therefore$ \quad $DE \pingxing BC$ (\hspace*{2cm}), \\
        $\therefore$ \quad $\angle CED + \angle C = 180^\circ$ (\hspace*{2cm})。
    }

    \xxt{如图丙, $\because$ \quad $AB \pingxing CD$(已知)\\
        $\therefore$ \quad $\angle 1 = \angle 3$ (\hspace*{2cm}) 。\\
        $\because$   \quad $\angle 3 = \angle 2$ (\hspace*{2cm}), \\
        $\therefore$ \quad $\angle 1 = \angle 2$ (\hspace*{2cm})。
    }

\end{xiaoxiaotis}


\xiaoti{用式子来表示下列句子:如图,}
\begin{xiaoxiaotis}

    \xxt{已知 $\angle 1$ 和 $\angle 2$ 相等,根据内错角相等,两直线平行,所以 $AB$ 和 $EF$ 平行。}

    \xxt{已知 $DE$ 和 $BC$ 平行,根据两直线平行,同位角相等,所以 $\angle 1$ 等于 $\angle B$,$\angle 3$ 等于 $\angle C$。}

\end{xiaoxiaotis}


\begin{figure}[htbp]
    \centering
    \begin{minipage}[b]{7cm}
        \centering
        \begin{tikzpicture}
    \tkzDefPoints{0/0/B, 4/0/C, 2.5/3/A}
    \tkzDefPointOnLine[pos=0.4](A,B)  \tkzGetPoint{D}
    \tkzDefPointOnLine[pos=0.4](A,C)  \tkzGetPoint{E}
    \tkzDefLine[parallel=through E](A,B) \tkzGetPoint{f}
    \tkzInterLL(E,f)(B,C)  \tkzGetPoint{F}

    \tkzDrawPolygon(A,B,C)
    \tkzDrawSegments(D,E  E,F)
    \tkzMarkAngles[size=0.3](E,D,A  D,E,F  C,F,E)
    \tkzMarkAngles[size=0.4](A,E,D)
    \tkzLabelAngle[pos=0.5](E,D,A){$1$}
    \tkzLabelAngle[pos=0.5](D,E,F){$2$}
    \tkzLabelAngle[pos=0.5](C,F,E){$4$}
    \tkzLabelAngle[pos=0.6](A,E,D){$3$}
    \tkzLabelPoints[above](A)
    \tkzLabelPoints[left](B,D)
    \tkzLabelPoints[right](E)
    \tkzLabelPoints[below](C,F)
\end{tikzpicture}


        \caption*{(第 11 题)}
    \end{minipage}
    \qquad
    \begin{minipage}[b]{7cm}
        \centering
        \begin{tikzpicture}
    \tkzDefPoints{0/0/B, 4/0/C, 2.5/3/A, 2/0/D}
    \tkzDefLine[parallel=through D](A,C) \tkzGetPoint{e}
    \tkzInterLL(D,e)(A,B)  \tkzGetPoint{E}
    \tkzDefLine[parallel=through D](A,B) \tkzGetPoint{f}
    \tkzInterLL(D,f)(A,C)  \tkzGetPoint{F}

    \tkzDrawPolygon(A,B,C)
    \tkzDrawSegments(D,E  D,F)
    \tkzMarkAngles[size=0.3](F,D,E)
    \tkzMarkAngles[size=0.4](E,D,B  C,D,F)
    \tkzLabelAngle[pos=0.5](F,D,E){$2$}
    \tkzLabelAngle[pos=0.6](E,D,B){$1$}
    \tkzLabelAngle[pos=0.6](C,D,F){$3$}
    \tkzLabelPoints[above](A)
    \tkzLabelPoints[left](E)
    \tkzLabelPoints[right](F)
    \tkzLabelPoints[below](B,C,D)
\end{tikzpicture}


        \caption*{(第 12 题)}
    \end{minipage}
\end{figure}


\xiaoti{填空:}
\begin{xiaoxiaotis}

    \xxt{$\because$ \quad $\angle A = \text{(\hspace*{1cm})}$ (已知),\\
        $\therefore$ \quad $AC \pingxing ED$ (\hspace*{2cm})。
    }

    \xxt{$\because$ \quad $\angle 2 = \text{(\hspace*{1cm})}$ (已知),\\
        $\therefore$ \quad $AC \pingxing ED$ (\hspace*{2cm})。
    }

    \xxt{$\because$ \quad $\angle A + \text{(\hspace*{1cm})} = 180^\circ$ (已知),\\
        $\therefore$ \quad $AB \pingxing FD$ (\hspace*{2cm})。
    }

    \xxt{$\because$ \quad $AC \pingxing \text{(\hspace*{1cm})}$ (已知),\\
        $\therefore$ \quad $\angle 2 = \angle DFC$ (\hspace*{2cm})。
    }

    \xxt{$\because$ \quad $AB \pingxing \text{(\hspace*{1cm})}$ (已知),\\
        $\therefore$ \quad $\angle 2 + \angle AED = 180^\circ$ (\hspace*{2cm})。
    }

    \xxt{$\because$ \quad $AC \pingxing \text{(\hspace*{1cm})}$ (已知),\\
        $\therefore$ \quad $\angle C = \angle 1$ (\hspace*{2cm})。
    }

\end{xiaoxiaotis}

\end{xiaotis}

