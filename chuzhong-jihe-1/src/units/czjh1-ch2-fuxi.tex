\fuxiti
\begin{xiaotis}

\xiaoti{点 $O$ 是直线 $AB$ 上的一点,$OC$、$OD$ 是分别在直线 $AB$ 两旁的两条射线,且有 $\angle AOC = \angle BOD$。}
\begin{xiaoxiaotis}

    \xxt{如果 $\angle AOC = 50^\circ$, 求 $\angle COB + \angle BOD$ 的大小;}

    \xxt{如果 $\angle AOC = m^\circ$,  求 $\angle COB + \angle BOD$ 的大小;}

    \xxt{$\angle AOC$ 和 $\angle BOD$ 是不是对顶角。}

\end{xiaoxiaotis}

\begin{figure}[htbp]
    \centering
    \begin{minipage}[b]{4.5cm}
        \centering
        \begin{tikzpicture}
    \tkzDefPoints{0/0/A, 3.5/0/B, 1.75/0/O}
    \tkzDefPointOnCircle[R = center O angle 230  radius 1.5] \tkzGetPoint{C}
    \tkzDefPointOnLine[pos=2](C,O)  \tkzGetPoint{D}

    \tkzDrawSegments(A,B  C,D)
    \tkzLabelPoints[below](A,O,B)
    \tkzLabelPoints[right](C,D)
\end{tikzpicture}


        \caption*{(第 1 题)}
    \end{minipage}
    \qquad
    \begin{minipage}[b]{4.5cm}
        \centering
        \begin{tikzpicture}
    \tkzDefPoints{0/0/A, 3.5/0/B, 1.75/0/O}
    \tkzDefPointOnCircle[R = center O angle 110  radius 1.5] \tkzGetPoint{C}
    \tkzDefPointOnLine[pos=2](C,O)  \tkzGetPoint{D}

    \tkzDrawSegments(A,B  C,D)
    \tkzLabelPoints[below](A,B)
    \tkzLabelPoints[below right](O)
    \tkzLabelPoints[right](C,D)
\end{tikzpicture}


        \caption*{(第 2 题)}
    \end{minipage}
    \qquad
    \begin{minipage}[b]{5cm}
        \centering
        \begin{tikzpicture}
    \tkzDefPoints{0/0/A, 3.5/0/C, 3/0/O}
    \tkzDefPointOnCircle[R = center O angle -50  radius 0.75] \tkzGetPoint{E}
    \tkzDefPointOnLine[pos=5](E,O)  \tkzGetPoint{D}
    \tkzDefPointOnCircle[R = center A angle 45  radius 3] \tkzGetPoint{B}
    \tkzInterLL(A,B)(E,D)  \tkzGetPoint{P}

    \tkzDrawSegments(A,B  A,C  D,E)
    \tkzMarkAngles[size=0.3](D,P,A  E,O,C)
    \tkzMarkAngles[size=0.4](D,O,A  A,P,E)
    \tkzLabelAngle[pos=0.5](D,P,A){$1$}
    \tkzLabelAngle[pos=0.5](E,O,C){$2$}
    \tkzLabelAngle[pos=0.6](D,O,A){$3$}
    \tkzLabelAngle[pos=0.6](A,P,E){$4$}

    \tkzLabelPoints[below](A,E)
    \tkzLabelPoints[above right](C)
    \tkzLabelPoints[left](D)
    \tkzLabelPoints[right](B)
\end{tikzpicture}


        \caption*{(第 3 题)}
    \end{minipage}
\end{figure}

\xiaoti{已知直线 $AB$ 与 $CD$ 相交于点 $O$, 并且 $\angle AOD + \angle BOC = 220^\circ$。求 $\angle AOC$。}


\xiaoti{}%
\begin{xiaoxiaotis}%
    \xxt[\xxtsep]{指出 $\angle 1$、$\angle 2$、$\angle 3$、$\angle 4$ 与 $\angle A$ 中对顶角、
        同位角、内错角、同旁内角,并指出它们是由哪些直线相交而成的。
    }

    \xxt{已知 $\angle 1 = 95^\circ$, $\angle 2 = 50^\circ$, 求 $\angle 3$、$\angle 4$ 的大小。}

\end{xiaoxiaotis}


\xiaoti{画一条线段 $AB = 6 \;\limi$。 取 $AB$ 的中点 $C$, 过点 $C$ 画直线 $CD \perp AB$,
    垂足为 $C$。 在 $CD$ 上取一点 $E$, 使 $CE = 4 \;\limi$。 % 原书中没有单位,根据上下文判断,应该是 cm
    测量 $AE$、 $BE$ 的长,并比较它们的大小。
}


\xiaoti{下面的说法对吗?为什么?}
\begin{xiaoxiaotis}

    \xxt{画直线 $AB$ 的中垂线;}

    \xxt{已知线段 $AB$ 和点 $M$, 过点 $M$ 画 $AB$ 的垂直平分线;}

    \xxt{已知直线上 $AB$ 一点 $M$,直线外一点 $N$,连结点 $M$、$N$,使 $MN \perp AB$;}

    \xxt{已知直线 $AB$ 及 $AB$ 上一点 $C$,过点 $C$ 画 $CD \perp CA$, 画 $CE \perp CB$。}

\end{xiaoxiaotis}


\xiaoti{如图,要从点 $A$ 走到河岸 $BC$,怎样走法最近?为什么?如果要走到河岸上一点 $D$ 呢?}

\begin{figure}[htbp]
    \centering
    \begin{minipage}[b]{4.5cm}
        \centering
        \begin{tikzpicture}
    \tkzDefPoints{0/0/C, 3.5/0/B, 2.5/0/D,  1.5/2/A}

    \tkzDrawSegments(C,B)
    \tkzDrawPoints[fill=black](A,D)

    \tkzLabelPoints[below](B,C,D)
    \tkzLabelPoints[above](A)
\end{tikzpicture}


        \caption*{(第 6 题)}
    \end{minipage}
    \qquad
    \begin{minipage}[b]{4.5cm}
        \centering
        \begin{tikzpicture}
    \tkzDefPoints{0/0/A, 3/-2.5/C, 0/-1/b1, 3/0/d1}
    \tkzDefPointOnCircle[R = center C angle 165  radius 1] \tkzGetPoint{b2}
    \tkzDefPointOnCircle[R = center A angle 8    radius 1] \tkzGetPoint{d2}
    \tkzInterLL(A,b1)(C,b2)  \tkzGetPoint{B}
    \tkzInterLL(A,d2)(C,d1)  \tkzGetPoint{D}

    \tkzDrawPolygon(A,B,C,D)

    \tkzLabelPoints[left](A,B)
    \tkzLabelPoints[right](C,D)
\end{tikzpicture}


        \caption*{(第 8 题)}
    \end{minipage}
    \qquad
    \begin{minipage}[b]{5cm}
        \centering
        \begin{tikzpicture}
    \tkzDefPoints{0/0/B, 3/0/C}
    \tkzDefPointOnCircle[R = center B angle 70  radius 1] \tkzGetPoint{a1}
    \tkzDefPointOnCircle[R = center C angle 130 radius 1] \tkzGetPoint{a2}
    \tkzInterLL(B,a1)(C,a2)  \tkzGetPoint{A}

    \tkzDefLine[bisector](A,C,B) \tkzGetPoint{d}
    \tkzInterLL(A,B)(C,d)        \tkzGetPoint{D}
    \tkzDefLine[parallel=through D](B,C) \tkzGetPoint{e}
    \tkzInterLL(D,e)(A,C)        \tkzGetPoint{E}

    \tkzDrawPolygon(A,B,C)
    \tkzDrawSegments(C,D  D,E)

    \tkzLabelPoints[above](A)
    \tkzLabelPoints[below](B,C)
    \tkzLabelPoints[left](D)
    \tkzLabelPoints[right](E)
\end{tikzpicture}


        \caption*{(第 9 题)}
    \end{minipage}
\end{figure}

\xiaoti{画 $\angle BAC = 57^\circ$。 在 $AB$ 上取点 $D$,使 $AD = 30 \;\haomi$;
    在 $AC$ 上取点 $E$,使 $AE = 40 \;\haomi$。
    经过 $D$ 画 $AC$ 的平行线,经过 $E$ 画 $AB$ 的平行线,两线相交于点 $F$。
    量出 $AF$ 的长〈精确到 $1\;\haomi$) 和 $\angle BAF$、$\angle CAF$ 的大小(精确到 $1^\circ$)。
}


\xiaoti{一块钢板的两边 $AB$ 和 $CD$ 平行, 已知 $\angle A = 98^\circ$, $\angle C = 75^\circ$,
    求 $\angle B$ 和 $\angle D$ 的度数,并且说明根据。
}


\xiaoti{如图,已知 $DE \pingxing BC$, $CD$ 是 $\angle ACB$ 的平分线, $\angle B = 70^\circ$,
    $\angle ACB = 50^\circ$,求 $\angle EDC$ 和 $\angle BDC$ 的度数。
}

\xiaoti{已知 $DB \pingxing FG \pingxing EC$, $\angle ABD = 60^\circ$, $\angle ACE = 36^\circ$,
    $AP$ 是 $\angle BAC$ 的平分线,计算 $\angle PAG$ 的度数。
}

\begin{figure}[htbp]
    \centering
    \begin{minipage}[b]{7cm}
        \centering
        \begin{tikzpicture}
    \tkzDefPoints{0/0/X1, 3/0/X2, 0.5/3/D, 3.5/3/E}
    \tkzDefPointOnLine[pos=0.7](X1,D)  \tkzGetPoint{B}
    \tkzDefPointOnLine[pos=0.3](X2,E)  \tkzGetPoint{C}
    \tkzDefPointBy[rotation=center B angle -60](D)  \tkzGetPoint{a1}
    \tkzDefPointBy[rotation=center C angle  36](E)  \tkzGetPoint{a2}
    \tkzInterLL(B,a1)(C,a2)  \tkzGetPoint{A}
    \tkzDefLine[normed, parallel=through A](X1,D) \tkzGetPoint{F}
    \tkzDefPointOnLine[pos=3](F,A)  \tkzGetPoint{G}
    \tkzDefLine[bisector](B,A,C) \tkzGetPoint{p}
    \tkzDefPointOnLine[pos=1.8](A,p)  \tkzGetPoint{P}

    \tkzDrawSegments(X1,X2  X1,D  X2,E  B,A  C,A  F,G  A,P)
    \tkzMarkAngles[size=0.5](A,B,D  E,C,A)
    \tkzLabelAngle[pos=0.8](A,B,D){$60^\circ$}
    \tkzLabelAngle[pos=1.2](E,C,A){$36^\circ$}
    \tkzLabelPoints[left](B,D)
    \tkzLabelPoints[right](F,A,E,C)
    \tkzLabelPoints[below](G,P)
\end{tikzpicture}


        \caption*{(第 10 题)}
    \end{minipage}
    \qquad
    \begin{minipage}[b]{7cm}
        \centering
        \begin{tikzpicture}
    \tkzDefPoints{0/0/B, 3/0/C, -1/2.5/A, 2/2.5/D}

    \tkzDrawPolygon(A,B,C,D)
    \tkzDrawSegments(A,C  B,D)
    \tkzMarkAngles[size=0.5](C,A,D  D,B,A  A,C,B  B,D,C)
    \tkzMarkAngles[size=0.4](B,A,C  C,B,D  D,C,A  A,D,B)
    \tkzLabelAngle[pos=0.7](C,A,D){$1$}
    \tkzLabelAngle[pos=0.7](D,B,A){$3$}
    \tkzLabelAngle[pos=0.7](A,C,B){$5$}
    \tkzLabelAngle[pos=0.7](B,D,C){$7$}
    \tkzLabelAngle[pos=0.6](B,A,C){$2$}
    \tkzLabelAngle[pos=0.6](C,B,D){$4$}
    \tkzLabelAngle[pos=0.6](D,C,A){$6$}
    \tkzLabelAngle[pos=0.6](A,D,B){$8$}
    \tkzLabelPoints[above](A,D)
    \tkzLabelPoints[below](B,C)
\end{tikzpicture}


        \caption*{(第 11 题)}
    \end{minipage}
\end{figure}

\xiaoti{}%
\begin{xiaoxiaotis}%
    \xxt[\xxtsep]{已知 $AD \pingxing BC$, 可以得出哪些角相等?为什么?}

    \xxt{已知 $AB \pingxing DC$, 可以得出哪些角的和是 $180^\circ$? 为什么?}

    \xxt{已知 $\angle 3 = \angle 7$, 可以得出哪两条直线平行?为什么?}

    \xxt{已知 $\angle 1 + \angle 2 + \angle 3 + \angle 4 = 180^\circ$, 可以得出哪两条直线平行? 为什么?}

\end{xiaoxiaotis}


抄下列各题,并在括号内加注由(第 12 ~ 15 题):

\xiaoti{已知:如图 $AB$ 和 $CD$ 相交于点 $O$,$\angle A = \angle B$。\\
    求证: $\angle C = \angle D$。 \\
    \zhengming $\because$ \quad $\angle A = \angle B$ (\hspace*{2cm}), \\
    $\therefore$  \quad $AC \pingxing BD$ (\hspace*{2cm})。 \\
    $\therefore$  \quad $\angle C = \angle D$ (\hspace*{2cm})。
}

\begin{figure}[htbp]
    \centering
    \begin{minipage}[b]{7cm}
        \centering
        \begin{tikzpicture}
    \tkzDefPoints{0/0/A, 4/0/B, 1.2/1.2/C}
    \tkzDefLine[parallel=through B, normed](C,A) \tkzGetPoint{D}
    \tkzInterLL(A,B)(C,D)  \tkzGetPoint{O}

    \tkzDrawSegments(A,B  C,D  A,C  B,D)
    \tkzLabelPoints[below](A,D,O)
    \tkzLabelPoints[above](B,C)
\end{tikzpicture}


        \caption*{(第 12 题)}
    \end{minipage}
    \qquad
    \begin{minipage}[b]{7cm}
        \centering
        \begin{tikzpicture}
    \tkzDefPoints{0/0/B, 4/0/C, 2.2/3/A, 2/0/D}
    \tkzDefLine[parallel=through D](B,A) \tkzGetPoint{e}
    \tkzDefLine[parallel=through D](C,A) \tkzGetPoint{f}
    \tkzInterLL(D,e)(A,C)  \tkzGetPoint{E}
    \tkzInterLL(D,f)(A,B)  \tkzGetPoint{F}

    \tkzDrawPolygon(A,B,C)
    \tkzDrawSegments(D,F  D,E)
    \tkzLabelPoints[above](A)
    \tkzLabelPoints[below](B,C,D)
    \tkzLabelPoints[left](F)
    \tkzLabelPoints[right](E)
\end{tikzpicture}


        \caption*{(第 13 题)}
    \end{minipage}
\end{figure}

\xiaoti{已知:如图。$D$、$E$、$F$ 分别是 $BC$、$CA$、$AB$ 上的点,$DE \pingxing BA$, $DF \pingxing CA$。\\
    求证: $\angle FDE = \angle A$。 \\
    \zhengming $\because$ \quad $DE \pingxing BA$ (\hspace*{2cm}), \\
    $\therefore$  \quad $\angle FDE = \angle BFD$ (\hspace*{2cm})。 \\
    $\because$ \quad $DF \pingxing CA$ (\hspace*{2cm}), \\
    $\therefore$  \quad $\angle BFD = \angle A$ (\hspace*{2cm})。 \\
    $\therefore$  \quad $\angle FDE = \angle A$ (\hspace*{2cm})。
}


\xiaoti{已知:如图,$AB \pingxing EF$。\\
    求证: $\angle BCF = \angle B + \angle F$。 \\
    \zhengming  经过点 $C$ 画 $CD \pingxing AB$(经过直线外一点有且只有一条直线和这条直线平行), \\
    $\therefore$  \quad $\angle B = \angle 1$ (\hspace*{2cm})。 \\
    $\because$ \quad \begin{zmtblr}[t]{}
        $AB \pingxing EF$ (\hspace*{2cm}), \\
        $CD \pingxing AB$ (\hspace*{2cm}), \\
    \end{zmtblr} \\
    $\therefore$  \quad $CD \pingxing EF$ (\hspace*{2cm})。 \\
    $\therefore$  \quad $\angle F = \angle 2$ (\hspace*{2cm})。 \\
    $\therefore$  \quad $\angle 1 + \angle 2 = \angle B + \angle F$ (\hspace*{2cm})。 \\
    即 \quad $\angle BCF = \angle B + \angle F$。
}


\begin{figure}[htbp]
    \centering
    \begin{minipage}[b]{7cm}
        \centering
        \begin{tikzpicture}
    \tkzDefPoints{0/0/E, 3/0/F, 0/2/A, 3/2/B,  1.3/0.8/C}
    \tkzDefLine[parallel=through C](A,B) \tkzGetPoint{d}
    \tkzInterLL(C,d)(B,F)  \tkzGetPoint{D}

    \tkzDrawSegments(A,B  B,C  C,D  C,F  E,F)
    \tkzMarkAngles[size=0.5](D,C,B)
    \tkzMarkAngles[size=0.4](F,C,D)
    \tkzLabelAngle[pos=0.7](D,C,B){$1$}
    \tkzLabelAngle[pos=0.9](F,C,D){$2$}
    \tkzLabelPoints[below](A,D,E,F)
    \tkzLabelPoints[left](C)
    \tkzLabelPoints[right](B)
\end{tikzpicture}


        \caption*{(第 14 题)}
    \end{minipage}
    \qquad
    \begin{minipage}[b]{7cm}
        \centering
        \begin{tikzpicture}
    \tkzDefPoints{0/0/D, 4/0/E, 0.7/0/F, 3.3/0/G}
    \tkzDefPointBy[rotation=center F angle  70](G)  \tkzGetPoint{a1}
    \tkzDefPointBy[rotation=center G angle -50](F)  \tkzGetPoint{a2}
    \tkzInterLL(F,a1)(G,a2)  \tkzGetPoint{A}
    \tkzDefPointOnLine[pos=1.4](A,F)  \tkzGetPoint{B}
    \tkzDefPointOnLine[pos=1.4](A,G)  \tkzGetPoint{C}

    \tkzDrawSegments(D,E  A,B  A,C)
    \tkzMarkAngles[size=0.4](E,F,A  A,G,D  D,F,B  C,G,E)
    \tkzLabelAngle[pos=0.7](E,F,A){$1$}
    \tkzLabelAngle[pos=0.7](A,G,D){$2$}
    \tkzLabelAngle[pos=0.7](D,F,B){$3$}
    \tkzLabelAngle[pos=0.7](C,G,E){$4$}
    \tkzLabelPoints[above](A,D,E)
    \tkzLabelPoints[right](B)
    \tkzLabelPoints[left](C)
    \tkzLabelPoints[below right](F)
    \tkzLabelPoints[below left](G)
\end{tikzpicture}


        \caption*{(第 15 题)}
    \end{minipage}
\end{figure}


\xiaoti{已知:$DE$ 交 $AB$ 于点 $F$,交 $AC$ 于点 $G$, $\angle 1 > \angle 2$,\\
    求证: $\angle 3 > \angle 4$。 \\
    \zhengming $\because$ \quad $\angle 1 > \angle 2$ (\hspace*{2cm}), \\
    $\angle 1 = \angle 3$,$\angle 2 = \angle 4$ (\hspace*{2cm}), \\
    $\therefore$  \quad $\angle 3 > \angle 4$ (\hspace*{2cm})。
}


\xiaoti{已知:如图,直线 $AB$、$CD$ 被 $EF$、$GH$ 所截, $\angle 1 + \angle 2 = 180^\circ$。求证:$\angle 3 = \angle 4$。}

\begin{figure}[htbp]
    \centering
    \begin{minipage}[b]{7cm}
        \centering
        \begin{tikzpicture}
    \tkzDefPoints{0/0/C, 4/0/D, 0/2/A, 4/2/B,  1.2/3/G,  0.8/-1/H,  2.5/3/E,  3.2/-1/F}
    \tkzInterLL(G,H)(A,B)    \tkzGetPoint{X1}
    \tkzInterLL(G,H)(C,D)    \tkzGetPoint{X2}
    \tkzInterLL(E,F)(A,B)    \tkzGetPoint{X3}
    \tkzInterLL(E,F)(C,D)    \tkzGetPoint{X4}

    \tkzDrawSegments(A,B  C,D  E,F  G,H)
    \tkzMarkAngles[size=0.3](A,X1,H  G,X2,C  B,X3,E  D,X4,E)
    \tkzLabelAngle[pos=0.6](A,X1,H){$1$}
    \tkzLabelAngle[pos=0.6](G,X2,C){$2$}
    \tkzLabelAngle[pos=0.6](B,X3,E){$3$}
    \tkzLabelAngle[pos=0.6](D,X4,E){$4$}
    \tkzLabelPoints[above](G,E)
    \tkzLabelPoints[below](A,B,C,D,H,F)
\end{tikzpicture}


        \caption*{(第 16 题)}
    \end{minipage}
    \qquad
    \begin{minipage}[b]{7cm}
        \centering
        \begin{tikzpicture}
    \tkzDefPoints{0/0/B', 4/0/C', 3.5/3/A', 0/1.5/B,  4/1.5/C}
    \tkzDefLine[parallel=through B, normed](B',A') \tkzGetPoint{a}
    \tkzDefPointOnLine[pos=3](B,a)  \tkzGetPoint{A}
    \tkzInterLL(A',B')(B,C)    \tkzGetPoint{D}

    \tkzDrawSegments(B',A' B',C'  B,A  B,C)
    \tkzLabelPoints[right](A',A)
    \tkzLabelPoints[below](B',C',B,C,D)
\end{tikzpicture}


        \caption*{(第 17 题)}
    \end{minipage}
\end{figure}

\xiaoti{已知:如图,$AB \pingxing A'B'$, $BC \pingxing B'C'$, $BC$ 交 $A'B'$ 于点 $D$。求证:$\angle B = \angle B'$。}

\xiaoti{已知:如图,$AB$ 和 $CD$ 相交于点 $O$,$\angle C = \angle COA$,$\angle D = \angle BOD$。求证:$AC \pingxing BD$。}


\begin{figure}[htbp]
    \centering
    \begin{minipage}[b]{7cm}
        \centering
        \begin{tikzpicture}
    \tkzDefPoints{-0.5/1.5/C, 0.5/-1.5/D}
    \tkzDefMidPoint(C,D)  \tkzGetPoint{O}
    \tkzDefTriangle[two angles= 70 and 70](O,C)  \tkzGetPoint{A}
    \tkzDefTriangle[two angles= 70 and 70](O,D)  \tkzGetPoint{B}
    \tkzDrawPolygon(O,C,A)
    \tkzDrawPolygon(O,D,B)
    \tkzLabelPoints[left](A)
    \tkzLabelPoints[right](B)
    \tkzLabelPoints[above](C)
    \tkzLabelPoints[below](D)
    \tkzLabelPoints[below left](O)
\end{tikzpicture}

        \caption*{(第 18 题)}
    \end{minipage}
    \qquad
    \begin{minipage}[b]{7cm}
        \centering
        \begin{tikzpicture}
    \tkzDefPoints{0/0/C, 4/0/D,  0/3/A,  4/3/B}
    \tkzDefLine[bisector](A,B,C) \tkzGetPoint{E}
    \tkzDefLine[bisector](D,C,B) \tkzGetPoint{F}

    \tkzDrawSegments(A,B  B,C  C,D  B,E  C,F)
    \tkzMarkAngles[size=0.8](A,B,E  D,C,F)
    \tkzLabelAngle[pos=1.4](A,B,E){$1$}
    \tkzLabelAngle[pos=1.4](D,C,F){$2$}
    \tkzLabelPoints[below](A,E,C,D)
    \tkzLabelPoints[right](B,F)
\end{tikzpicture}


        \caption*{(第 19 题)}
    \end{minipage}
\end{figure}


\begin{enhancedline}
\xiaoti{}%
\begin{xiaoxiaotis}%
    \xxt[\xxtsep]{已知:如图,$AB \pingxing CD$,$EB \pingxing CF$。求证: $\angle 1 = \angle 2$。}

    \xxt{已知:如图,$AB \pingxing CD$,$\angle 1 = \angle 2$。求证:$BE \pingxing CF$。}

    \xxt{已知:如图,$BE \pingxing CF$,$\angle 1 = \exdfrac{1}{2} \angle ABC$, $\angle 2 = \exdfrac{1}{2} \angle BCD$。求证:$AB \pingxing CD$。}

\end{xiaoxiaotis}
\end{enhancedline}

\end{xiaotis}

