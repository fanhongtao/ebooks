\subsection{平行四边形、三角形、梯形的面积}\label{subsec:czjh1-5-2}

\begin{enhancedline}

\subsubsection{平行四边形的面积。}

已知 $\pxsbx ABCD$, 从一个顶点 $B$ 作边 $CD$ 的垂线 $BE$,垂足为 $E$,
以 $AB$、$BE$ 为邻边作矩形 $ABEF$ (图 \ref{fig:czjh1-5-6})。

\begin{wrapfigure}[6]{r}{5cm}
    \centering
    \begin{tikzpicture}
    \tkzDefPoints{0/0/A, 2/0/B, 3.2/2/C, 1.2/2/D, 2/2/E, 0/2/F}

    \tkzDrawPolygon[dashed](A,B,E,F)
    \tkzDrawPolygon(A,B,C,D)
    \tkzLabelPoints[above](C,D,E,F)
    \tkzLabelPoints[below](A,B)
\end{tikzpicture}


    \caption{}\label{fig:czjh1-5-6}
\end{wrapfigure}

$\because$ \quad $\triangle ADF \quandeng \triangle BCE$,(为什么?)

$\therefore$ \quad $S_{\pxsbx ABCD} = S_{\text{矩形} ABEF}$。

$\therefore$ \quad $S_{\pxsbx ABCD} = AB \cdot BE$。

我们把平行四边形的一条边叫做它的\zhongdian{底},
这条边和对边的距离叫做它这个底上的\zhongdian{高}。
于是得到下面的定理:

\begin{dingli}[定理]
    平行四边形的面积等于它的底 $a$ 和高 $h$ 的积。
\end{dingli}

\begin{center}
    \framebox[10em]{\zhongdian{$\bm{
        S_{\text{平行四边形}} = ah
    }$}。}
\end{center}

\begin{tuilun}[推论]
    等底等高的平行四边形的面积相等。
\end{tuilun}

例如,在图 \ref{fig:czjh1-5-7} 中, $S_{\pxsbx ABCD} = S_{\pxsbx ABEF}$。

\begin{figure}[htbp]
    \centering
    \begin{minipage}[b]{7cm}
        \centering
        \begin{tikzpicture}
    \tkzDefPoints{0/0/A, 2/0/B, 1.2/2/C, -0.8/2/D, 3.7/2/E, 1.7/2/F}

    \tkzDrawPolygon(A,B,C,D)
    \tkzDrawPolygon(A,B,E,F)
    \tkzDrawSegment[dashed](C,F)
    \tkzLabelPoints[above](C,D,E,F)
    \tkzLabelPoints[below](A,B)
\end{tikzpicture}


        \caption{}\label{fig:czjh1-5-7}
    \end{minipage}
    \qquad
    \begin{minipage}[b]{7cm}
        \centering
        \begin{tikzpicture}
    \tkzDefPoints{0/0/A, 3/0/B, 1/2/C}
    \tkzDefPointBy[projection=onto A--B](C)  \tkzGetPoint{H}
    \tkzDefPointBy[translation=from A to B](C)  \tkzGetPoint{D}

    \tkzDrawPolygon(A,B,C)
    \tkzDrawSegments(C,H)
    \tkzDrawSegments[dashed](B,D  C,D)
    \tkzLabelPoints[above](C,D)
    \tkzLabelPoints[below](A,B,H)
\end{tikzpicture}


        \caption{}\label{fig:czjh1-5-8}
    \end{minipage}
\end{figure}

\subsubsection{三角形的面积。}

已知 $\triangle ABC$,$CH$ 是高。 以 $AB$、$AC$ 为邻边作 $\pxsbx ABDC$, $BC$ 是对角线(图 \ref{fig:czjh1-5-8})。

$\because$ \quad $\triangle ABC \quandeng \triangle DCB$,

$\therefore$ \quad $S_{\triangle ABC} = \exdfrac{1}{2} S_{\pxsbx ABDC}$。

$\because$ \quad $\triangle ABC$ 与 $\pxsbx ABDC$ 的底和高相同,由平行四边形面积定理,得
$$ S_{\triangle ABC} = \exdfrac{1}{2} S_{\pxsbx ABDC} = \exdfrac{1}{2} AB \cdot CH \juhao $$

由此我们得到下面定理:

\begin{dingli}[定理]
    三角形的面积等于它的底 $a$ 与高 $h$ 的积的一半。
\end{dingli}

\begin{center}
    \framebox[10em]{\zhongdian{$\bm{
        S_{\text{三角形}} = \exdfrac{1}{2} ah
    }$}。}
\end{center}


\begin{tuilun}[推论1]
    等底等高的三角形面积相等。
\end{tuilun}

例如,在图 \ref{fig:czjh1-5-9} 中,$AB \pingxing EC$, 有 $S_{\triangle ABC} = S_{\triangle ABD} = S_{\triangle ABE}$。

\begin{figure}[htbp]
    \centering
    \begin{minipage}[b]{7cm}
        \centering
        \begin{tikzpicture}
    \tkzDefPoints{0/0/A, 3/0/B, 4/2/C, 2/2/D, -0.5/2/E}

    \tkzDrawPolygon(A,B,C)
    \tkzDrawPolygon(A,B,D)
    \tkzDrawPolygon(A,B,E)
    \tkzDrawSegments[dashed](C,E)
    \tkzLabelPoints[above](C,D,E)
    \tkzLabelPoints[below](A,B)
\end{tikzpicture}


        \caption{}\label{fig:czjh1-5-9}
    \end{minipage}
    \qquad
    \begin{minipage}[b]{7cm}
        \centering
        \begin{tikzpicture}
    \tkzDefPoints{-1.5/0/A, 1.5/0/C, 0/-1/B, 0/1/D, 0/0/O}

    \tkzDrawPolygon(A,B,C,D)
    \tkzDrawSegments(A,C  B,D)
    \tkzLabelPoints[left](A)
    \tkzLabelPoints[below](B)
    \tkzLabelPoints[right](C)
    \tkzLabelPoints[above](D)
    \tkzLabelPoints[below left](O)
\end{tikzpicture}


        \caption{}\label{fig:czjh1-5-10}
    \end{minipage}
\end{figure}

如图 \ref{fig:czjh1-5-10}, 菱形 $ABCD$ 被对角线 $AC$,$BD$ 分成四个全等的直角三角形。
它们的底和高都分别是两条对角线的一半,利用三角形的面积可得:

\begin{tuilun}[推论2]
    菱形的面积等于它的两条对角线的积的一半。
\end{tuilun}


\subsubsection{梯形的面积。}

已知梯形 $ABCD$, 高为 $DH$(图 \ref{fig:czjh1-5-11})。
作对角线 $BD$ 把梯形分为两个三角形 $ABD$、$BCD$。
显然,这两个三角形的高都等于 $DH$。于是由三角形面积定理得,

$S_{\text{梯形} ABCD} = S_{\triangle ABD} + S_{\triangle BCD} = \exdfrac{1}{2} AH \cdot DH = \exdfrac{1}{2} CD \cdot DH = \exdfrac{1}{2} (AB + CD) DH$。

\begin{wrapfigure}[6]{r}{5cm}
    \centering
    \begin{tikzpicture}
    \tkzDefPoints{0/0/A, 4/0/B, 1/2/D, 3.5/2/C}
    \tkzDefPointBy[projection=onto A--B](D)  \tkzGetPoint{H}

    \tkzDrawPolygon(A,B,C,D)
    \tkzDrawSegments(D,H)
    \tkzDrawSegments[dashed](B,D)
    \tkzLabelPoints[above](C,D)
    \tkzLabelPoints[below](A,B,H)
\end{tikzpicture}


    \caption{}\label{fig:czjh1-5-11}
\end{wrapfigure}

于是,我们得到下面定理:

\begin{dingli}[定理]
    梯形面积等于它的两底 $a$、$b$ 的和与高 $h$ 的积的一半。
\end{dingli}

\begin{center}
    \framebox[12em]{\zhongdian{$\bm{
        S_{\text{梯形}} = \exdfrac{1}{2} (a + b) h
    }$}。}
\end{center}

\begin{tuilun}[推论]
    梯形面积等于它的中位线长与高的积。
\end{tuilun}

由于任意多边形都可以看作是若干个平行四边形、三角形、梯形组成的,
而平行四边形、三角形、梯形的面积都可以计算,所以任何多边形的面积,
都可以变为这些图形的面积和或差来计算。


\liti 如图 \ref{fig:czjh1-5-12}, 有一块土地 $ABCD$, 顶点 $B$、$C$ 到一条长边 $AD$ 的垂线段为 $BN$、$CQ$,
已测得 $BN$、$CQ$、$AQ$、$ND$ 的长。将这块土地的面积用这些条件表示出来。

\jie 在多边形 $ABCD$ 中,垂线段 $BN$、$CQ$ 将多边形 $ABCD$ 分为三角形和梯形,所以

$\begin{aligned}
      & S_{\text{多边形} ABCD} = S_{\triangle ABN} + S_{\text{梯形} BCQN} + S_{\triangle CDQ} \\
    = & \exdfrac{1}{2} AN \cdot BN + \exdfrac{1}{2} (BN + CQ) \cdot NQ + \exdfrac{1}{2} QD \cdot CQ \\
    = & \exdfrac{1}{2} BN \cdot (AN + NQ) + \exdfrac{1}{2} CQ \cdot (NQ + QD) \\
    = & \exdfrac{1}{2} (BN \cdot AQ + CQ \cdot ND) \juhao
\end{aligned}$

\begin{figure}[htbp]
    \centering
    \begin{minipage}[b]{7cm}
        \centering
        \begin{tikzpicture}
    \tkzDefPoints{0/0/B, 2/0.3/C, 3/1/D, -1.5/1/A}
    \tkzDefPointBy[projection=onto A--D](B)  \tkzGetPoint{N}
    \tkzDefPointBy[projection=onto A--D](C)  \tkzGetPoint{Q}

    \tkzDrawPolygon(A,B,C,D)
    \tkzDrawSegments[dashed](B,N  C,Q)
    \tkzLabelPoints[above](N,Q)
    \tkzLabelPoints[below](B,C)
    \tkzLabelPoints[left](A)
    \tkzLabelPoints[right](D)
\end{tikzpicture}


        \caption{}\label{fig:czjh1-5-12}
    \end{minipage}
    \qquad
    \begin{minipage}[b]{7cm}
        \centering
        \begin{tikzpicture}
    \tkzDefPoints{0/0/A, 2/0/B, 2.5/2/C, 1.5/3/D}
    \tkzDrawPolygon(A,B,C,D)
    \tkzLabelPoints[below](A,B)
    \tkzLabelPoints[right](C)
    \tkzLabelPoints[above](D)

    % 1
    \tkzDrawSegments[dashed](D,B)

    % 2
    \tkzDrawLine[add=0 and 0.8](A,B)
    \tkzDefLine[parallel=through C](D,B)  \tkzGetPoint{p}
    \tkzInterLL(A,B)(C,p)  \tkzGetPoint{C'}
    \tkzLabelPoints[below](C')
    \tkzDrawSegments[dashed](C,C')

    % 3
    \tkzDrawSegments(D,C')
\end{tikzpicture}


        \caption{}\label{fig:czjh1-5-13}
    \end{minipage}
\end{figure}

\liti 将已知四边形 $ABCD$ 变为面积相等的三角形。

已知: 四边形 $ABCD$(图  \ref{fig:czjh1-5-13})。

求作: 一个三角形与四边形  $ABCD$ 面积相等。

分析: 从点 $D$ 引对角线把四边形分成两个三角形,如果能把其中一个三角形变为与它面积相等的三角形,
并且有一边在另一个三角形的一边的延长线上,问题就解决了。

\zuofa 1. 连结对角线 $DB$。

2. 过顶点 $C$ 作 $CC'$ 平行于 $DB$, 与 $AB$ 的延长线交于点 $C'$。

3. 连结 $DC'$。

$\triangle ADC'$ 就是所求的三角形。

\zhengming 根据等底等高的三角形面积相等,按作法

$\because$ \quad  $CC' \pingxing BD$,

$\therefore$ \quad $S_{\triangle BDC} = S_{\triangle BDC'}$。

$\because$ \quad $S_{\text{四边形} ABCD} = S_{\triangle ABD} + S_{\triangle BDC}$,

$\therefore$ \quad $\begin{aligned}[t]
    S_{\text{四边形} ABCD}  &= S_{\triangle ABD} + S_{\triangle BDC'} \\
                    &= S_{\triangle ADC'} \juhao
\end{aligned}$


\begin{lianxi}

\xiaoti{求证:平行四边形的两条对角线把它分成四个面积相等的三角形。}

\xiaoti{$\triangle ABE$ 与矩形 $ABCD$ 有公共边 $AB$, 顶点 $E$ 在矩形的边 $CD$ 或其延长线上。\\
    求证:$S_{\triangle ABE} = \exdfrac{1}{2} S_{\text{矩形} ABCD}$。
}

\xiaoti{一个等腰梯形的下底 18 cm、高 $4\sqrt{3}$ cm、腰长为 8 cm, 腰与下底成 $60^\circ$ 角,求它的面积。}

\xiaoti{连结三角形各边中点所成的三角形的面积,是原三角形面积的几分之几?}

\end{lianxi}

\end{enhancedline}

