\xiti
\begin{xiaotis}

\xiaoti{}%
\begin{xiaoxiaotis}%
    \xxt[\xxtsep]{直角三角形 $ABC$ 中, $\angle C = Rt \angle$, $b = 2.5$, $c = 6.5$, 求 $a$;}

    \xxt{$\triangle ABC$ 中, $a = n^2 - 1 \; (n > 1)$, $b = 2n$, $c = n^2 + 1$。
        求证: $\angle C = Rt \angle$。
    }

\end{xiaoxiaotis}


\xiaoti{隔湖有两点 $A$、$B$,从与 $BA$ 方向成直角的 $BC$ 方向上的点 $C$,
    测得 $CA = 50 \;\mi$, $CB = 40\;\mi$, 求 $AB$。
}

\begin{figure}[htbp]
    \centering
    \begin{minipage}[b]{7cm}
        \centering
        \begin{tikzpicture}
    \tkzDefPoints{0/0/C, 4/0/B, 4/3/A}
    \tkzDrawSegments(A,C)
    \tkzDrawSegments[dashed](A,B)
    \tkzDrawLine[add=0 and 0.2](C,B)
    \tkzMarkRightAngle[size=0.2](A,B,C)
    \tkzLabelPoints[above](A)
    \tkzLabelPoints[below](B)
    \tkzLabelPoints[left](C)

    %
    \tkzDefPointOnLine[pos=0.5](A,B)  \tkzGetPoint{O}
    \tkzDrawEllipse[
        pattern={mylines[angle=65, distance={4pt}]},
        decorate, decoration={random steps,segment length=3pt,amplitude=1pt}
    ](O,0.5,1.2,0)
\end{tikzpicture}


        \caption*{(第 2 题)}
    \end{minipage}
    \qquad
    \begin{minipage}[b]{7cm}
        \centering
        \begin{tikzpicture}[scale=0.8]
    \tkzDefPoints{0/0/A, 4/0/C, 4/3/B}
    \tkzDrawSegments(C,A  A,B)
    \tkzDrawSegments[-Stealth](B,C)
    \tkzLabelSegment[below](A,C){4尺}
    \tkzLabelSegment[above left](A,B){5尺}
    \tkzLabelSegment[right](B,C){3尺}
    \tkzLabelPoints[above](B)
    \tkzLabelPoints[right](C)
    \tkzLabelPoints[left](A)
\end{tikzpicture}


        \caption*{(第 3 题)}
    \end{minipage}
\end{figure}


\xiaoti{在地面上确定直角, 可以用如图所示的方法: 取一条长 $12$ 尺的测绳,
    在地面上距离为 $4$ 尺的两点 $A$、$C$ 打两个木桩,把测绳套在木桩上,
    把剩下的 $8$ 尺绳分成 $5$ 尺和 $3$ 尺两段,拉紧分点就可以在地面上确定点 $B$,
    这时 $\angle ACB$ 就是直角。 说明这种确定直角方法的根据。
}

\xiaoti{如图,车床齿轮箱壳要钻两个圆孔,两孔中心的距离 $AB$ 是 $134 \;\haomi$,
    两孔中心的水平距离 $BC$ 是 $77 \;\haomi$, 计算两孔中心的垂直距离 $AC$ (精确到 $0.1 \;\haomi$)。
}

\begin{figure}[htbp]
    \centering
    \begin{minipage}[b]{7cm}
        \centering
        \begin{tikzpicture}[scale=0.04]
    \pgfmathsetmacro{\r}{134}

    \tkzDefPoints{0/0/B, 77/0/C}
    \tkzDefLine[perpendicular=through C](B,C)  \tkzGetPoint{a}
    \tkzInterLC[R](C,a)(B,\r)  \tkzGetPoint{A}

    \tkzDrawPolygon[densely dash dot](A,B,C)

    \tkzDefShiftPoint[A](2,0){A'}
    \tkzDefShiftPoint[B](2,0){B'}
    \tkzDrawCircle[black](A,A')
    \tkzDrawCircle[black](B,B')

    \tkzLabelPoints[below right,xshift=-.4em](A)
    \tkzLabelPoints[above right, xshift=1em](B)
    \tkzLabelPoints[above left](C)

    \tkzDefLine[perpendicular=through A, K=0.1](B,A)  \tkzGetPoint{A1}
    \tkzDefLine[perpendicular=through B, K=0.1](B,A)  \tkzGetPoint{B2}
    % \tkzDrawSegments(A1,B2)

    \tkzDefLine[perpendicular=through B, K=-0.1](B,C)  \tkzGetPoint{B1}
    \tkzDefLine[perpendicular=through C, K=-0.1](B,C)  \tkzGetPoint{C2}
    % \tkzDrawSegments(B1,C2)

    \tkzDefLine[perpendicular=through C, K=-0.1](C,A)  \tkzGetPoint{C1}
    \tkzDefLine[perpendicular=through A, K=-0.1](C,A)  \tkzGetPoint{A2}
    % \tkzDrawSegments(C1,A2)

    \tkzFindAngle(C,B,A)  \tkzGetAngle{cba}
    \pgfmathsetmacro{\stopangle}{\cba + 80}
    \draw % [red]
        (A2)
        arc [radius=9.6, start angle=0,  end angle=\stopangle]
        -- (B2)
        arc [radius=8.5, start angle=\stopangle, end angle=270]
        -- (C2)
        arc [radius=7.8, start angle=270, end angle=360]
        -- (A2)
        ;

    %
    \tkzDrawSegments(A,A1 B,B2)
    \tkzDrawSegments[dim={$134$,-10pt,rotate=\cba}](A1,B2)

    \tkzDrawSegments(B,B1 C,C2)
    \tkzDrawSegments[dim={$77$,-10pt,}](B1,C2)

    \tkzDrawSegments(C,C1 A,A2)
    \tkzDrawSegments[dim={$x$,-10pt,rotate=90}](C1,A2)
\end{tikzpicture}


        \caption*{(第 4 题)}
    \end{minipage}
    \qquad
    \begin{minipage}[b]{7cm}
        \centering
        \begin{tikzpicture}[scale=0.7]
    \pgfmathsetmacro{\a}{3}
    \pgfmathsetmacro{\b}{1.5}
    \pgfmathsetmacro{\d}{-10}

    \coordinate (A) at (0,\b,0);
    \coordinate (B) at (\a,0,0);
    \coordinate (C) at (0,0,0);
    \coordinate (A') at (0,\b,\d);
    \coordinate (B') at (\a,0,\d);
    % \coordinate (C') at (0,0,\d);

    \draw (A) -- (B) -- (C) -- (A);
    \draw (A) -- (A') -- (B') -- (B);

    \foreach \n [count=\i] in {0.3,0.6,0.9} {
        \coordinate (X\i) at ($(A)!\n!(B)$);
        \coordinate (X{\i}') at ($(A')!\n!(B')$);
        \draw (X\i) -- (X{\i}');
    }

    \foreach \n [count=\i] in {0.33,0.66} {
        \coordinate (Y\i) at ($(A)!\n!(A')$);
        \coordinate (Y{\i}') at ($(B)!\n!(B')$);
        \draw (Y\i) -- (Y{\i}');
    }

    \node [below] at ($(B)!.5!(C)$) {$a$};
    \node [left]  at ($(A)!.5!(C)$) {$b$};
    \node [below right] at ($(B)!.5!(B')$) {$d$};
\end{tikzpicture}


        \caption*{(第 5 题)}
    \end{minipage}
\end{figure}

\xiaoti{某生产队修建一个育苗棚(如图),棚宽 $a = 3 \;\mi$, 高 $b = 1.5 \;\mi$,
    长 $d = 10 \;\mi$, 求覆盖在顶面上的塑料薄膜需要多少平方米〈精确到 $0.1 \;\pfm$)。
}

\xiaoti{}%
\begin{xiaoxiaotis}%
    \xxt[\xxtsep]{正方形的边长是 $a$,求对角线长;}

    \xxt{正方形的对角线长是 $b$,求一边长。}

\end{xiaoxiaotis}


\xiaoti{求高等于 $h$ 的等边三角形的边长。}

\xiaoti{在 $\triangle ABC$ 中, $\angle C = 90^\circ$, $AC = 2.1 \;\limi$, $BC = 2.8 \;\limi$,求}
\begin{xiaoxiaotis}

    \xxt{$\triangle ABC$ 的面积;}
    \xxt{斜边 $AB$;}
    \xxt{高 $CD$。}
\end{xiaoxiaotis}


\xiaoti{一艘轮船以小时 $16$ 海里的速度离开港口向东南方向航行。
    另一艘轮船在同时同地以每小时 12 海里的速度向西南方向航行。
    它们离开港口一个半小时后相距多远?
}

\xiaoti{在 $\triangle ABC$ 中, $\angle C = 90^\circ$, $\angle A = 30^\circ$, $AB = 10$, 求 $AC$。}

\xiaoti{在一个锐角等于 $30^\circ$ 的直角三角形中, $30^\circ$ 角所对的直角边的长是 $a$,
    求斜边的长和另一条直角边的长。 写出三条边长的比。
}

\xiaoti{如图,在边长为 $c$ 的正方形中,有四个斜边为 $c$ 的全等三角形,
    已知它们的直角边长为 $a$, $b$。 利用这个图证明勾股定理。
    (这个图叫做勾股方圆图, 我国古代数学家赵爽在他所著的《勾股方圆图注》中,
    用这个图证明了勾股定理。)
}

\begin{figure}[htbp]
    \centering
    \begin{minipage}[b]{7cm}
        \centering
        \begin{tikzpicture}
    \pgfmathsetmacro{\abc}{30} % 角 ABC 的度数
    \pgfmathsetmacro{\c}{4}

    \tkzDefPoints{0/0/B, \c/0/C, \c/\c/D}
    \tkzDefPointBy[rotation=center B angle \abc](C)  \tkzGetPoint{a}
    \tkzDefPointBy[projection=onto B--a](C)  \tkzGetPoint{A}
    \tkzDrawPolygon[fill=gray!20](A,B,C)

    \tkzDefPointOnLine[pos=.5](B,D)  \tkzGetPoint{O}
    \foreach \n in {90, 180, 270} {
        \tkzDefPointBy[rotation=center O angle \n](A)  \tkzGetPoint{A'}
        \tkzDefPointBy[rotation=center O angle \n](B)  \tkzGetPoint{B'}
        \tkzDefPointBy[rotation=center O angle \n](C)  \tkzGetPoint{C'}
        \tkzDrawPolygon[fill=gray!20](A',B',C')
    }

    \tkzLabelSegment[below left](A,C){$b$}
    \tkzLabelSegment[below right](A,B){$a$}
    \tkzLabelSegment[below](B,C){$c$}
    \tkzLabelSegment[right](C,D){$c$}
\end{tikzpicture}


        \caption*{(第 12 题)}
    \end{minipage}
    \qquad
    \begin{minipage}[b]{7cm}
        \centering
        \begin{tikzpicture}
    \tkzDefPoints{0/0/A, 4/0/B, 5/2/C, 1/2/D}
    \tkzDefLine[altitude](A,D,B)  \tkzGetPoint{E}
    \tkzDefLine[altitude](A,C,B)  \tkzGetPoint{F}

    \tkzDrawPolygon(A,B,C,D)
    \tkzDrawSegments(A,C  B,D)
    \tkzDrawSegments[dashed](D,E)
    \tkzDrawSegments[dashed](C,F  F,B)

    \tkzLabelSegment[below](A,B){$a$}
    \tkzLabelSegment[above left](A,D){$b$}
    \tkzLabelSegment[pos=0.7, above](A,C){$m$}
    \tkzLabelSegment[pos=0.7, above](B,D){$n$}

    \tkzLabelPoints[below](A,B,E,F)
    \tkzLabelPoints[above](C,D)
\end{tikzpicture}


        \caption*{(第 14 题)}
    \end{minipage}
\end{figure}

\xiaoti{已知: $\triangle ABC$ 中, $CD$ 是高。
    求证:$CA^2 - CB^2 = DA^2 - DB^2 = AB (DA - DB)$。
}

\xiaoti{如图,平行四边形的邻边长为 $a$、$b$, 对角线长为 $m$、$n$。
    求证: $m^2 + n^2 = 2 (a^2 + b^2)$。
}

\xiaoti{一个等腰三角形的周长是 $16$ 厘米, 底边上的高是 $4$ 厘米,
    求这个三角形各边的长。
}

\end{xiaotis}

