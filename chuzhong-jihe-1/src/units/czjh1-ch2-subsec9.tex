\subsection{证明}\label{subsec:czjh1-2-9}

前面,我们学过一些定理,如 “对顶角相等”;“内错角相等,两直线平行” 等。
它们的正确性是经过推理证实的,推理的过程就是证明。

一个命题,除了公理之外,它是真命题还是假命题,需要经过证明才能知道。
证明要从命题的题设出发,通过推理来判断命题的结论是否成立。
结论成立,就是真命题,结论不成立,就是假命题。学习几何就要学会证明。

下面我们来看一个证明的例子。

\begin{dingli}[定理]
    如果两条直线都和第三条直线垂直,那么这两条直线平行。
\end{dingli}

已知:如图 \ref{fig:czjh1-2-22},$AB \perp EF$, $CD \perp EF$。

求证:$AB \pingxing CD$。

\zhengming $\because$ \quad $AB \perp EF$(已知),

$\therefore$ \quad $\angle 1 = 90^\circ$ (垂直的定义)。

$\because$ \quad $CD \perp EF$(已知),

$\therefore$ \quad $\angle 2 = 90^\circ$ (垂直的定义)。

$\therefore$ \quad $\angle 1 = \angle 2$ (等量代换)。

$\therefore$ \quad $AB \pingxing CD$(同位角相等,两直线平行)。

\begin{figure}[htbp]
    \centering
    \begin{minipage}[b]{7cm}
        \centering
        \begin{tikzpicture}
    \tkzDefPoints{0/0/E, 4/0/F, 1.2/1.5/A, 1.2/-1.5/B, 2.5/1.5/C, 2.5/-1.5/D}
    \tkzInterLL(A,B)(E,F)  \tkzGetPoint{O}
    \tkzInterLL(C,D)(E,F)  \tkzGetPoint{P}

    \tkzDrawSegments(A,B  C,D  E,F)
    \tkzMarkRightAngles[size=0.3](F,O,A  F,P,C)
    \tkzLabelAngle[pos=0.7](F,O,A){$1$}
    \tkzLabelAngle[pos=0.7](F,P,C){$2$}
    \tkzLabelPoints[above](E,F)
    \tkzLabelPoints[right](A,B,C,D)
\end{tikzpicture}


        \caption{}\label{fig:czjh1-2-22}
    \end{minipage}
    \qquad
    \begin{minipage}[b]{7cm}
        \centering
        \input{../pic/czjh1-ch2-23}
        \caption{}\label{fig:czjh1-2-23}
    \end{minipage}
\end{figure}


从上例可以看出,证明是从题设出发,经过由 “$\because$ …,$\therefore$ …” 组成的推理,直到得出结论。
推理的每一步都必须有根据,根据是题设和已证事项、定义、公理和定理。

证明一个命题,一般步骤如下:

1. 按题意画出图形;

2. 分清命题的题设与结论,结合图形,在已知一项中写出题设,在求证一项中写出结论;

3. 在证明一项中,写出证明过程。

下面,我们按以上步骤,再来证明一个命题:

\begin{dingli}[定理]
    如果一条直线和两条平行线中的一条垂直,那么,这条直线也和另一条垂直。
\end{dingli}

我们先来画出符合题意的图形。
先画出两条平行线 $AB \pingxing CD$。再画一条直线 $EF \perp AB$ 。
这样我们就画出了符合题意的图形(图 \ref{fig:czjh1-2-23})。

下面,我们再来写已知求证。命题的题设是 “一条直线和两条平行线中的一条垂直”。
体现在图上是 $AB \pingxing CD$, $EF \perp AB$。
命题的结论是 “这条直线也和另一条垂直”。体现在图上是 $EF \perp CD$。于是有:

已知:$AB \pingxing CD$, $EF \perp AB$(图 \ref{fig:czjh1-2-23})。

求证: $EF \perp CD$。

\zhengming $\because$ \quad $AB \pingxing CD$(已知),

$\therefore$ \quad $\angle 1 = \angle 2$ (两直线平行,同位角相等)。

$\because$ \quad $EF \perp AB$ (已知),

$\therefore$ \quad $\angle 1 = 90^\circ$(垂直定义)。

$\therefore$ \quad $\angle 2 = 90^\circ$(等量代换)。

$\therefore$ \quad $EF \perp CD$(垂直定义)。

有些命题是假命题。 要证明一个命题是假命题,只要能举出一个例子说明这个命题不成立就可以。
例如,证明 “相等的角是对顶角” 是假命题时,可画一个角的平分线,得到两个相等的角,
但它们不是对顶角,就可以肯定要证的命题是假命题了。


\begin{lianxi}

\xiaoti{抄写下列各命题的证明步驟,并写出推理根据:}
\begin{figure}[htbp]
    \centering
    \begin{minipage}[b]{3.5cm}
        \centering
        \begin{tikzpicture}
    \tkzDefPoints{0/0/A, 0.6/0/B, 1.6/0/C, 2.2/0/D}

    \tkzDrawLine[add=0.2 and 0.2](A,D)
    \tkzDrawPoints[fill=black](A,B,C,D)
    \tkzLabelPoints[below](A,B,C,D)
\end{tikzpicture}


        \caption*{甲}
    \end{minipage}
    \qquad
    \begin{minipage}[b]{6.5cm}
        \centering
        \begin{tikzpicture}
    \begin{scope}
        \tkzDefPoints{0/0/B, 2.5/0/C}
        \tkzDefPoint(50:2.5){A}
        \tkzDefPoint(25:2.5){D}

        \tkzDrawSegments(B,A  B,C  B,D)
        \tkzMarkAngles[size=0.6](C,B,D)
        \tkzLabelAngle[pos=0.9](C,B,D){$1$}
        \tkzLabelPoints[below](B,C)
        \tkzLabelPoints[right](A,D)
    \end{scope}

    \begin{scope}[xshift=3.2cm]
        \tkzDefPoints{0/0/B', 2.5/0/C'}
        \tkzDefPoint(50:2.5){A'}
        \tkzDefPoint(25:2.5){D'}

        \tkzDrawSegments(B',A'  B',C'  B',D')
        \tkzMarkAngles[size=0.6](C',B',D')
        \tkzLabelAngle[pos=0.9](C',B',D'){$2$}
        \tkzLabelPoints[below](B',C')
        \tkzLabelPoints[right](A',D')
    \end{scope}
\end{tikzpicture}


        \caption*{乙}
    \end{minipage}
    \begin{minipage}[b]{4cm}
        \centering
        \begin{tikzpicture}
    \tkzDefPoints{0/0/O, 2.5/0/D}
    \tkzDefPoint(80:2.5){A}
    \tkzDefPoint(60:2.5){C}
    \tkzDefPoint(20:2.5){B}

    \tkzDrawSegments(O,A  O,B  O,C  O,D)
    \tkzMarkAngles[size=0.6](C,O,A  D,O,B)
    \tkzLabelAngle[pos=0.9](C,O,A){$1$}
    \tkzLabelAngle[pos=0.9](D,O,B){$2$}
    \tkzLabelPoints[below](O,D)
    \tkzLabelPoints[right](A,B,C)
\end{tikzpicture}


        \caption*{丙}
    \end{minipage}
    \caption*{(第 11 题)}
\end{figure}
\begin{xiaoxiaotis}

    \xxt{已知:如图甲, $A$、$B$、$C$、$D$ 四点在一条直线上,$AB = CD$。 \\
        求证: $AC = BD$。\\
        \zhengming $\because$ \quad $AB = CD$ (\hspace*{2cm}), \\
        $\therefore$ \quad $AB + BC = CD + BC$(\hspace*{2cm}), \\
        即 \quad $AC = BD$。
    }

    \begin{enhancedline}
    \xxt{已知:如图乙,$\angle ABC = \angle A'B'C'$,$BD$、$B'D'$ 分别是 $\angle ABC$、$\angle A'B'C'$ 的平分线。\\
        求证:$\angle 1 = \angle 2$。 \\
        \zhengming $\because$ \quad  $\angle ABC = \angle A'B'C'$(\hspace*{2cm}), \\
        $\therefore$ \quad $\exdfrac{1}{2} \angle ABC = \exdfrac{1}{2} \angle A'B'C'$ (\hspace*{2cm})。 \\
        $\because$ \quad \begin{zmtblr}[t]{}
            $\angle 1 = \exdfrac{1}{2} \angle ABC$ (角平分线定义), \\[1em]
            $\angle 2 = \exdfrac{1}{2} \angle A'B'C'$ (\hspace*{2cm}),
        \end{zmtblr} \\
        $\therefore$ \quad $\angle 1 = \angle 2$(\hspace*{2cm})。
    }
    \end{enhancedline}

    \xxt{已知:如图丙,$\angle AOB = \angle COD$。\\
        求证: $\angle 1 = \angle 2$。 \\
        \zhengming $\because$ \quad $\angle AOB = \angle COD$(\hspace*{2cm}), \\
        $\therefore$ \quad $\angle AOB - \angle BOC = \angle COD - \angle BOC$(\hspace*{2cm}), \\
        即 \quad $\angle 1 = \angle 2$。
    }

\end{xiaoxiaotis}


\xiaoti{证明下列命题是假命题:}
\begin{xiaoxiaotis}

    \xxt{一个角的补角大于这个角;}

    \xxt{正数与负数的和是负数。}

\end{xiaoxiaotis}

\end{lianxi}

