\xiaojie

一、本章的内容是多边形,平行四边形(包括矩形、菱形、正方形)和梯形的有关知识。


二、多边形的一些有关概念,多边形内角和等于 $(n - 2) \cdot 180^\circ$, 外角和等于 $360^\circ$,
这些知识是研究多边形的边、角关系的基础。


三、几种特殊的四边形: 平行四边形(包括矩形、菱形、正方形)和梯形,是常见的四边形。

1. 几种特殊四边形的关系如图 \ref{fig:czjh1-4-49} :

\begin{figure}[htbp]
    \centering
    \begin{tikzpicture}[
    >=Stealth,
    tixing/.style={draw, minimum height=1.2cm, trapezium, trapezium stretches=true,},
]
    %\node (SBX) [tixing, trapezium left angle=75, trapezium right angle=45] at (0, 0) {四边形};
    \node (SBX) [rectangle, minimum height=1.2cm] at (0, 0) {四边形};       % 不绘制 node 的 shape
    \draw (-0.6, 0.6) -- (0.8, 0.3) -- (1.0, -0.5) -- (-1, -0.5) -- cycle; % 自己绘制一个不规则的四边形

    \node (PXSBX) [tixing, trapezium left angle=75, trapezium right angle=-75] at (3,2.5) {平行四边形};
    \node (JX)    [draw, rectangle, minimum height=1.2cm, minimum width=2cm] at (7,4.5) {矩形};
    \node (LX)    [draw, diamond, minimum height=1.2cm, aspect=2] at (7,1.5) {菱形};
    \node (ZFX)   [draw, rectangle, minimum height=1.5cm, minimum width=1.5cm] at (12,3) {正方形};
    \node (TX)    [tixing, trapezium left angle=75, trapezium right angle=45] at (5, -2.0) {梯形};
    \node (DYTX)  [tixing] at (12,  0.5) {等腰梯形};
    \node (ZJTX)  [tixing, trapezium left angle=90, trapezium right angle=45] at (12, -2.0) {直角梯形};

    \draw [->] ($(SBX.north) + (0.2,-0.1)$) -- ($(PXSBX.south) - (0.2,0.2)$)
        node [pos=0.4, above, rotate=30] {两组对边}
        node [midway, below, rotate=30] {分别平行}
    ;
    \draw [->] ($(PXSBX.70)  + (0.2,0.2)$) -- ($(JX.160) - (0.2,0.2)$)
        node [midway, above, rotate=30] {有一个角}
        node [midway, below, rotate=30] {是直角}
    ;
    \draw [->] ($(JX.east) + (0.2,0.2)$) -- ($(ZFX.west) - (0.2,0.2)$)
        node [midway, above, rotate=-40] {有一组邻}
        node [midway, below, rotate=-40] {边相等}
    ;
    \draw [->] ($(PXSBX.east) + (0.2,0.2)$) -- ($(LX.north west) + (-0.2,0.2)$)
        node [midway, above, rotate=-20] {有一组邻}
        node [midway, below, rotate=-20] {边相等}
    ;
    \draw [->] ($(LX.east) + (0.2,0.2)$) -- ($(ZFX.south west) - (0.2,0)$)
        node [midway, above, rotate=10] {有一个角}
        node [midway, below, rotate=10] {是直角}
    ;

    \draw [->] ($(SBX.south east) + (0.2,-0.2)$) -- ($(TX.west) - (0.2,-0.2)$)
        node [midway, above, rotate=-20] {有且仅有一}
        node [midway, below, rotate=-20] {组对边平行}
    ;
    \draw [->] ($(TX.30) + (0.2,0)$) -- ($(DYTX.bottom left corner) - (0.2,0)$)
        node [midway, above, rotate=15] {两腰相等}
    ;
    \draw [->] ($(TX.east) + (0.2,0)$) -- ($(ZJTX.west) - (0.2,0)$)
        node [midway, above] {有一个角}
        node [midway, below] {是直角}
    ;
\end{tikzpicture}


    \caption{}\label{fig:czjh1-4-49}
\end{figure}

2. 几种特殊四边形的性质:

\jiange
\begin{tblr}{hlines, vlines, row{1}={c}}
    & 边 & 角 & 对角线 \\
    平行四边形 & 对边平行且相等 & 对角相等 & 两条对角线互相平分 \\
    矩形 & 对边平行且相等 & 四个角都是直角 & 两条对角线互相平分且相等 \\
    菱形 & {对边平行,\\四条边都相等} & 对角相等 & {两条对角线互相垂直平分,\\每条对角线平分一组对角} \\
    正方形 & {对边平行,\\四条边都相等} & 四个角都是直角 & {两条对角线互相垂直平分且相等,\\每条对角线平分一组对角} \\
    等腰梯形 & {两底平行,\\两腰相等} & 同一底上的两个角相等 & 两条对角线相等
\end{tblr}

\jiange
3. 几种特殊四边形的常用判定方法:

\jiange
\begin{tblr}{hlines,vlines,
    column{2}={13cm},
}
    平行四边形 & (1)两组对边分别平行;(2)两组对边分别相等;(3)一组对边平行且相等;(4)两条对角线互相平分。 \\
    矩形 & (1)有三个角是直角;(2)是平行四边形,并且有一个角是直角;(3)是平行四边形,并且两条对角线相等。 \\
    菱形 & (1)四条边都相等;(2)是平行四边形,并且有一组邻边相等;(3)是平行四边形,并且两条对角线互相垂直。\\
    正方形 & (1)是矩形,并且有一组邻边相等;(2)是菱形,并且有一个角是直角。 \\
    等腰梯形 & 是梯形,并且同一底上的两个角相等。
\end{tblr}

\jiange
四、中心对称和轴对称是两种对称图形, 它们有一个共同的性质:即对称的两个图形全等。
有一不明显区别是:关于中心对称的两个图形中,对应线段平行;而关于轴对称的两个图形中,对应线段不一定平行。
如果一个轴对称图形有两条互相垂直的对称轴(例如,矩形、菱形、正方形), 那么它必是中心对称图形,
这两条对称轴的交点就是它的对称中心。


五、以平行四边形性质为基础,推出平行线等分线段定理。
这个定理的两个推论分别是梯形、三角形的中位线的判定定理。

三角形中位线定理:三角形的中位线平行于第三边并且等于它的一半,是梯形中位线定理的基础,
由它很容易推出梯形中位线定理:梯形的中位线平行于两底并且等于两底和的一半。

