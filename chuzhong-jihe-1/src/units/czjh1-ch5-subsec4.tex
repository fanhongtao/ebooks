\subsection{勾股定理的例题}\label{subsec:czjh1-5-4}

\liti 求图 \ref{fig:czjh1-5-18} 所示矩形零件上两孔中心 $A$ 和 $B$ 的距离(精确到 $0.1\;\haomi$)。

\begin{figure}[htbp]
    \centering
    \begin{tikzpicture}
    \pgfmathsetmacro{\factor}{0.06}
    \pgfmathsetmacro{\a}{80*\factor} % 矩形的长
    \pgfmathsetmacro{\b}{60*\factor} % 矩形的宽(高)
    \pgfmathsetmacro{\r}{10*\factor}

    \tkzDefPoints{
        21*\factor/40*\factor/A,
        60*\factor/21*\factor/B,
        21*\factor/21*\factor/C,
        0/0/O,
        \a/0/P,
        \a/\b/Q,
        0/\b/R,
        21*\factor/\b/Ax,
        0/40*\factor/Ay,
        60*\factor/0/Bx,
        \a/21*\factor/By}

    %
    \tkzDrawPolygon(O,P,Q,R)
    \tkzDrawPolygon[dashed](A,B,C)
    \tkzMarkRightAngle(B,C,A)

    \tkzDefShiftPoint[A](10*\factor,0){A'}
    \tkzDefShiftPoint[B](10*\factor,0){B'}
    \tkzDrawCircle[black](A,A')
    \tkzDrawCircle[black](B,B')

    \tkzDrawLine[add=0 and 0](A, Ax)
    \tkzDrawSegments[dim={$21$,10pt,}](R,Ax)

    \tkzDrawLine[add=0.7 and 0](A, Ay)
    \tkzDrawSegments[dim={$40$,10pt,rotate=90}](O,Ay)

    \tkzDrawLine[add=0.7 and 0](B, Bx)
    \tkzDrawSegments[dim={$60$,-10pt,}](O,Bx)

    \tkzDrawLine[add=0 and 0](B, By)
    \tkzDrawSegments[dim={$21$,-10pt,rotate=90}](P,By)

    \tkzLabelPoints[above right](A,B)
    \tkzLabelPoints[below left](C)

    % \begin{scope}[>=Stealth, every node/.style={fill=white, inner sep=1pt}]
    %     \draw [<->] ($(O) + (0,-.5)$) to [xianduan={above=4cm}] node {$60$} ($(Bx) + (0,-.5)$);
    %     \draw [<->] ($(R) + (0, .5)$) to [xianduan={below=3cm}] node {$20$} ($(Ax) + (0, .5)$);
    % \end{scope}
\end{tikzpicture}


    \caption{}\label{fig:czjh1-5-18}
\end{figure}

\jie $\because$ \quad $\triangle ABC$ 是直角三角形,根据勾股定理,得

\qquad $AB^2 = AC^2 + BC^2$。

$\therefore$ \quad $AB = \sqrt{AC^2 + BC^2}$。

$\because$ \quad $AC = 40 - 21 = 19$, $BC = 60 - 21 = 39$,

$\therefore$ \quad $AB = \sqrt{19^2 + 39^2} = \sqrt{1882}$。

查表,得

\qquad $AB = 43.4 \; (\haomi)$。

答:两孔中心距离约为 $43.4 \;\haomi$。


\liti 从直角三角形的直角顶点到斜边上的垂线,将斜边上的正方形分成两个矩形。
求证:这两个矩形的面积分别等于两个直角边上的正方形的面积。

已知;如图 \ref{fig:czjh1-5-19}, $Rt \triangle ABC$ 中,$\angle C = 90^\circ$,
四边形 $ADEB$、$BKJC$、$CGFA$ 分别是 $\triangle ABC$ 三边上的正方形。
$CI \perp AB$, 垂足为 $H$,交 $DE$ 于 $I$。

求证: $S_{\text{正方形} CGFA} = S_{\text{矩形} ADIH}$, $S_{\text{正方形} BKJC} = S_{\text{矩形} HIEB}$。

\zhengming 连结 $BF$、 $CD$。

$\because$ \quad $\triangle ABF \quandeng \triangle ADC$ ($SAS$),

$\therefore$ \quad $S_{\triangle ABF} = S_{\triangle ADC}$。

$\because$ \quad \begin{zmtblr}[t]{}
    $S_{\text{正方形} CGFA} = 2 S_{\triangle ABF}$, \\
    $S_{\text{矩形} ADIH} = 2 S_{\triangle ADC}$(等底等高), \\
\end{zmtblr}

$\therefore$ \quad $S_{\text{正方形} CGFA} = S_{\text{矩形} ADIH}$。

同理可得,

\qquad $S_{\text{正方形} BKJC} = S_{\text{矩形} HIEB}$。


古代希腊数学家欧几里得曾把毕达哥拉斯定理编写在他所著的《几何原本》一书中,用上面的方法证明了这个定理。

\begin{figure}[htbp]
    \centering
    \begin{minipage}[b]{7cm}
        \centering
        \begin{tikzpicture}[scale=0.4]
    \pgfmathsetmacro{\a}{3}
    \pgfmathsetmacro{\b}{4}
    \pgfmathsetmacro{\c}{5}

    \tkzDefPoints{0/0/A,  \c/0/B}
    \tkzInterCC[R](A,\b)(B,\a)  \tkzGetFirstPoint{C}
    % \tkzMarkRightAngle(A,C,B)
    \tkzLabelPoints[above=.5em](C)
    \tkzLabelPoints[below left](A)
    \tkzLabelPoints[below right](B)

    \tkzDefSquare(B,A)  \tkzGetPoints{D}{E}
    % \tkzDrawPolygon(B,A,D,E)  % 移到 tkzFillPolygon 后
    \tkzLabelPoints[below](D,E)

    \tkzDefSquare(A,C)  \tkzGetPoints{G}{F}
    % \tkzDrawPolygon(A,C,G,F)  % 移到 tkzFillPolygon 后
    \tkzLabelPoints[above](G)
    \tkzLabelPoints[left](F)

    \tkzDefSquare(C,B)  \tkzGetPoints{K}{J}
    \tkzDrawPolygon(C,B,K,J)
    \tkzLabelPoints[right](K)
    \tkzLabelPoints[above](J)

    %====================================
    \tkzDefPointBy[projection=onto A--B](C)  \tkzGetPoint{H}
    \tkzInterLL(D,E)(C,H)  \tkzGetPoint{I}
    % \tkzDrawSegment(C,I)  % 移到 tkzFillPolygon 后
    \tkzLabelPoints[below right](H)
    \tkzLabelPoints[below](I)

    %
    \tkzFillPolygon[gray!20](A,B,F)
    \tkzFillPolygon[gray!20](A,D,C)
    \tkzDrawSegments[dashed](B,F  C,D)

    % 称位后的代码
    \tkzDrawPolygon(B,A,D,E)
    \tkzDrawPolygon(A,C,G,F)
    \tkzDrawSegment(C,I)

    % 标记两个相等的角
    \tkzMarkAngle[size=0.7cm](B,A,F)
    \tkzMarkAngle[size=1.0cm](D,A,C)

    \tkzMarkRightAngle(A,C,B)
\end{tikzpicture}


        \caption{}\label{fig:czjh1-5-19}
    \end{minipage}
    \qquad
    \begin{minipage}[b]{7cm}
        \centering
        \begin{tikzpicture}
    \tkzDefPoints{0/0/C, 1/0/A, 0/1/B_1}
    \tkzDrawPolygon(A,C,B_1)
    \tkzLabelSegment[below](C,A){$1$}
    \tkzLabelSegment[left](C,B_1){$1$}
    \tkzLabelPoints[below](A,C)
    \tkzLabelPoints[above left](B_1)

    %
    \foreach \n [remember=\n as \lastn (initially 1)] in {2,...,6} {
        \tkzDefLine[perpendicular=through B_\lastn, normed](B_\lastn,A)  \tkzGetPoint{B_\n}
        \tkzDrawPolygon(A,B_\lastn,B_\n)
        \tkzAutoLabelPoints[center=A,dist=.1,above](B_\n)
    }
    \tkzLabelSegment[above left](B_1,B_2){$1$}
\end{tikzpicture}


        \caption{}\label{fig:czjh1-5-20}
    \end{minipage}
\end{figure}


\liti 作长为 $\sqrt{2}$、$\sqrt{3}$、…、$\sqrt{7}$ 的各线段.

分析:由勾股定理,
直角边长为 $1$ 的直角三角形,斜边长就等于 $\sqrt{2}$、
直角边长为 $\sqrt{2}$、$1$ 的直角三角形的斜边长就是 $\sqrt{3}$。
以此类推,由此得到作法。


\zuofa 1. 作直角边长为 $1$ 的等腰直角三角形 $ACB_1$ (图 \ref{fig:czjh1-5-20} )。

2. 以斜边 $AB_1$ 为一直角边, 作另一直角边长为 $1$ 的直角三角形 $AB_1B_2$。

3. 顺次这样作下去,最后作到直角三角形 $AB_5B_6$, 这时斜边 $AB_1$、$AB_2$、…、$AB_6$
的长度就是 $\sqrt{2}$、$\sqrt{3}$、…、$\sqrt{7}$。

\zhengming 根据勾股定理, 在 $Rt \triangle ACB_1$ 中,

$AB_1^2 = AC^2 + B_1C^2 = 1^2 + 1^2 = 2$。

$\because$ \quad $AB_1 > 0$,

$\therefore$ \quad $AB_1 = \sqrt{2}$。

其他同理可证。


\begin{lianxi}


\xiaoti{已知 $CD$ 是 $Rt \triangle ABC$ 斜边 $AB$ 上的高,$BD = 1$, $\angle A = 30^\circ$。
    求 $\triangle ABC$ 的面积。
}

\xiaoti{证明:在四边形 $ABCD$ 中,如果对角 $AC \perp BD$,那么
    $$ AB^2 + CD^2 = AD^2 + BC^2 \juhao $$
}

\end{lianxi}

