\subsection{三角形全等的判定 II}\label{subsec:czjh1-3-6}

现在研究三角形全等的另一个判定方法。

画一个三角形,使它的两个角分别等于 $30^\circ$ 和 $45^\circ$, 它们所夹的边的长等于 2.5 cm。

\huafa 1. 画线段 $BC = 2.5 \;\limi$ (图 \ref{fig:czjh1-3-21})。

2. 在 $BC$ 的同旁,分别以 $B$、$C$ 为顶点, 画 $\angle CBD = 30^\circ$、
$\angle BCE = 45^\circ$,$BD$ 和 $CE$ 相交于 $A$。

$\triangle ABC$ 就是所求的三角形。

\begin{figure}[htbp]
    \centering
    \begin{minipage}[b]{4.5cm}
        \centering
        \begin{tikzpicture}
    \tkzDefPoints{0/0/B, 2.5/0/C}
    \tkzDefPointBy[rotation=center B angle  30](C)  \tkzGetPoint{D}
    \tkzDefPointBy[rotation=center C angle -45](B)  \tkzGetPoint{e}
    \tkzDefPointOnLine[pos=0.8](C,e)  \tkzGetPoint{E}
    \tkzInterLL(B,D)(C,E)  \tkzGetPoint{A}

    \tkzDrawSegments(B,C  B,D  C,E)
    \tkzLabelPoints[above](A,E,D)
    \tkzLabelPoints[left](B)
    \tkzLabelPoints[right](C)
\end{tikzpicture}


        \caption{}\label{fig:czjh1-3-21}
    \end{minipage}
    \qquad
    \begin{minipage}[b]{9cm}
        \centering
        \input{../pic/czjh1-ch3-22}
        \caption{}\label{fig:czjh1-3-22}
    \end{minipage}
\end{figure}


如果按照上面的条件,用同样的方法另画一个 $\triangle A'B'C'$,
再把 $\triangle A'B'C'$ 剪下来放到 $\triangle ABC$ 上,
$\triangle A'B'C'$ 与 $\triangle ABC$ 能够完全重合。
所以,只要是按上面条件画出的三角形,总是全等的。我们也把这个事实作为公理:

\begin{gongli}[角边角公理]
    有两角和它们的夹边对应相等的两个三角形全等
\end{gongli}(可以简写成 “\zhongdian{角边角}” 或 “$\bm{ASA}$” ) 。

例如, 在图 \ref{fig:czjh1-3-22} 的 $\triangle ABC$ 和 $\triangle A'B'C'$ 中,如果
$\angle B = \angle B'$, $BC = B'C'$, $\angle C = \angle C'$,那么
$$ \triangle ABC \quandeng \triangle A'B'C' \juhao $$

在 $\triangle ABC$ 和 $\triangle A'B'C'$ 中, 如果 $AB = A'B'$,$\angle B = \angle B'$,
$\angle C = \angle C'$( 图 \ref{fig:czjh1-3-22}), 那么, 由三角形内角和定理可得 $\angle A = \angle A'$。
根据角边角公理, $\triangle ABC \quandeng \triangle A'B'C'$。因此,有下面推论:

\begin{tuilun}[推论]
    有两角和其中一角的对边对应相等的两个三角形全等
\end{tuilun}(可以简写成 “\zhongdian{角角边}” 或 “$\bm{AAS}$” ) 。


\liti  已知:点 $D$ 在 $AB$ 上,点 $E$ 在 $AC$ 上,$BE$ 和 $CD$ 相交于点 $O$,
$AB = AC$, $\angle B= \angle C$ (图 \ref{fig:czjh1-3-23})。

求证: $BD = CE$。

分析: $BD$ 和 $CE$ 分别在 $\triangle BOD$ 和 $\triangle COE$ 中,
由已知条件不能直接证明 $\triangle BOD \quandeng \triangle COE$。
但已知 $AB = AC$,$AB$、$BD$ 及 $AC$、$CE$ 分别在一条直线上,
如果能证 $AD = AE$,就可以得到 $BD = CE$。
而 $AD$ 和 $AE$ 分别在 $\triangle ADC$ 和 $\triangle AEB$ 中,
可由已知条件证得 $\triangle ADC \quandeng \triangle AEB$。

\begin{wrapfigure}[7]{r}{5cm}
    \centering
    \begin{tikzpicture}
    \tkzDefPoints{0/0/B, 3.0/0/C, 1.5/2.5/A}
    \tkzDefPointOnLine[pos=0.6](A,B)  \tkzGetPoint{D}
    \tkzDefPointOnLine[pos=0.6](A,C)  \tkzGetPoint{E}
    \tkzInterLL(B,E)(C,D)  \tkzGetPoint{O}

    \tkzDrawSegments(A,B  A,C  B,E  C,D)
    \tkzLabelPoints[above](A)
    \tkzLabelPoints[left](B,D)
    \tkzLabelPoints[right](C,E)
    \tkzLabelPoints[below](O)
\end{tikzpicture}


    \caption{}\label{fig:czjh1-3-23}
\end{wrapfigure}

\zhengming 在 $\triangle ADC$ 和 $\triangle AEB$ 中,

\hspace{2em}$\begin{cases}
    \angle A = \angle A &\text{(公共角),} \\
    AC = AB & \text{(已知),} \\
    \angle C = \angle B & \text{(已知),} \\
\end{cases}$

$\therefore$ \quad  $\triangle ACD \quandeng \triangle ABE$ ($ASA$)。

$\therefore$ \quad $AD = AE$ (全等三角形对应边相等)。

又 $\because$ \quad $AB = AC$ (已知),

$\therefore$ \quad $BD = CE$ (等式性质)。


\liti 求证:全等三角形对应角的平分线相等。

已知:$\triangle ABC \quandeng \triangle A'B'C'$, $AD$、$A'D"$ 分别是
$\triangle ABC$ 和 $\triangle A'B'C'$ 的角平分线(图 \ref{fig:czjh1-3-24})。

求证:$AD = A'D'$。

\zhengming $\because$ \quad $\triangle ABC \quandeng \triangle A'B'C'$ (已知),

\begin{minipage}[b]{2em}$\therefore$ \\ \phantom{a} \end{minipage} %  为了让 “所以” 符号与 “AB = A'B'” 同行显示
% \quad
$\left.\begin{aligned}
    & AB = A'B' \\
    & \angle B = \angle B' \\
    & \angle BAC = \angle B'A'C'
\end{aligned} \right\} \qquad \text{(全等三角形对应边、对应角相等)。}$

\begin{enhancedline}
$\because$ \quad $\begin{aligned}[t]
    &\angle 1 = \exdfrac{1}{2} \angle BAC \quad \text{(已知),} \\
    &\angle 2 = \exdfrac{1}{2} \angle B'A'C' \quad \text{(已知),} \\
\end{aligned}$

$\therefore$ \quad $\angle 1 = \angle 2$(等量代换)。
\end{enhancedline}

在 $\triangle ABD$ 和 $\triangle A'B'D'$ 中,

\hspace{2em}$\begin{cases}
    \angle B = \angle B' & \text{(已证),} \\
    AB = A'B' & \text{(已证),} \\
    \angle 1 = \angle 2 & \text{(已证),} \\
\end{cases}$

$\therefore$ \quad $\triangle ABD \quandeng \triangle A'B'D'$ ($ASA$)。

$\therefore$ \quad $AD = A'D'$ (全等三角形对应边相等)。


\begin{figure}[htbp]
    \centering
    \begin{minipage}[b]{9cm}
        \centering
        \begin{tikzpicture}
    % 两个 scope 的区别,仅仅在于各点的名称不同。
    % 所以将绘制代码抽取出来(复用)
    \def\drawtriangle{
        \tkzDefPoints{0/0/B, 3.5/0/C, 2.8/2/A}
        \tkzDefLine[bisector](B,A,C) \tkzGetPoint{d}
        \tkzInterLL(A,d)(B,C)        \tkzGetPoint{D}
        \tkzDrawPolygon(A,B,C)
        \tkzDrawSegment(A,D)
        \tkzMarkAngle[size=0.4](B,A,D)
    }

    \begin{scope}
        \drawtriangle
        \tkzLabelAngle[pos=0.7](B,A,D){$1$}
        \tkzLabelPoints[above](A)
        \tkzLabelPoints[below](B,C,D)
    \end{scope}

    \begin{scope}[xshift=4.5cm]
        \drawtriangle
        \tkzLabelAngle[pos=0.7](B,A,D){$2$}
        \tkzLabelPoint[above](A){$A'$}
        \tkzLabelPoint[below](B){$B'$}
        \tkzLabelPoint[below](C){$C'$}
        \tkzLabelPoint[below](D){$D'$}
    \end{scope}
\end{tikzpicture}


        \caption{}\label{fig:czjh1-3-24}
    \end{minipage}
    \qquad
    \begin{minipage}[b]{5cm}
        \centering
        \begin{tikzpicture}
    \tkzDefPoints{0/0/C, 0/3/A}
    \tkzDefPointBy[rotation=center C angle  60](A)  \tkzGetPoint{b1}
    \tkzDefPointBy[rotation=center A angle -30](C)  \tkzGetPoint{b2}
    \tkzInterLL(C,b1)(A,b2)  \tkzGetPoint{B}
    \tkzDefPointBy[reflection = over A--C](B)  \tkzGetPoint{D}

    \tkzDrawPolygon(A,B,C)
    \tkzDrawPolygon(A,C,D)
    \tkzMarkAngle[size=0.4](B,A,C)
    \tkzMarkAngle[size=0.5](C,A,D)
    \tkzLabelAngle[pos=0.7](B,A,C){$1$}
    \tkzLabelAngle[pos=0.8](C,A,D){$2$}
    \tkzMarkRightAngle(A,B,C)
    \tkzMarkRightAngle(A,D,C)
    \tkzLabelPoints[above](A)
    \tkzLabelPoints[below](B,C,D)
\end{tikzpicture}


        \caption*{(第 1 题)}
    \end{minipage}
\end{figure}

\begin{lianxi}

\xiaoti{已知:如图,$AB \perp BC$,$AD \perp BC$,垂足分别为 $B$、$D$,$\angle 1 = \angle 2$。\\
    求证: $AB = AD$。
}

\xiaoti{求证:全等三角形对应边上的高相等。}

\end{lianxi}

