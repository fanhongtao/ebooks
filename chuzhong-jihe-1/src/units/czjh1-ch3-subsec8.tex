\subsection{等腰三角形的性质}\label{subsec:czjh1-3-8}

等腰三角形是一种特殊的三角形,它有许多重要性质。我们先做个实验,
取一张等腰三角形的纸片(图 \ref{fig:czjh1-3-31}),把两腰 $AB$、$AC$ 叠在一起,我们发现,两个底角互相重合。
这说明等腰三角形的两个底角相等。下面我们来证明这个性质。

\begin{dingli}[等腰三角形的性质定理]
    等腰三角形的两个底角等。
\end{dingli}(简写成 “\zhongdian{等边对等角}”。)

\begin{wrapfigure}[8]{r}{5cm}
    \centering
    \begin{tikzpicture}
    \tkzDefPoints{0/0/B,  2.4/0/C,  1.2/3/A,  1.2/0/D}

    \tkzDrawPolygon(A,B,C)
    \tkzDrawSegment[dashed](A,D)
    \tkzMarkRightAngle(A,D,C)
    \extkzLabelAngel[0.5](B,A,D){$1$}
    \extkzLabelAngel[0.6](D,A,C){$2$}
    \tkzLabelPoints[above](A)
    \tkzLabelPoints[below](B,C,D)
\end{tikzpicture}


    \caption{}\label{fig:czjh1-3-31}
\end{wrapfigure}


已知: $\triangle ABC$ 中, $AB = AC$(图 \ref{fig:czjh1-3-31})。

求证: $\angle B = \angle C$。

\zhengming 作顶角的平分线 $AD$。

在 $\triangle BAD$ 和 $\triangle CAD$ 中,

\hspace{2em} $\begin{cases}
    AB = AC \quad \text{(已知),} \\
    \angle 1 = \angle 2 \quad \text{(角的平分线的定义),} \\
    AD = AD \quad \text{(公共边),} \\
\end{cases}$

$\therefore$ \quad $\triangle BAD \quandeng \triangle CAD$ ($SAS$)。

$\therefore$ \quad $\angle B = \angle C$ (全等三角形的对应角相等)。

从上面的证明过程中,可以知道
$$ BD = CD \douhao  \angle ADB = \angle ADC = Rt \angle \douhao $$
所以 $AD$ 平分 $BC$, 并且 $AD \perp BC$。得

\begin{tuilun}[推论 1]
    等腰三角形顶角的平分线平分底边并且垂直于底边。
\end{tuilun}

从推论 1 可以知道,\zhongdian{等腰三角形的顶角平分线、底边上的中线、底边上的高互相重合。}

\begin{tuilun}[推论 2]
    等边三角形的各角都相等,并且每一个角都等于 $60^\circ$。
\end{tuilun}


\begin{wrapfigure}[8]{r}{5cm}
    \centering
    \begin{tikzpicture}
    \tkzDefPoints{0/0/B,  2.4/0/C,  1.2/3/A}
    \tkzDefLine[bisector](C,B,A)  \tkzGetPoint{d}
    \tkzDefLine[bisector](A,C,B)  \tkzGetPoint{e}
    \tkzInterLL(B,d)(A,C)  \tkzGetPoint{D}
    \tkzInterLL(C,e)(A,B)  \tkzGetPoint{E}

    \tkzDrawPolygon(A,B,C)
    \tkzDrawSegments(B,D  C,E)
    \extkzLabelAngel[0.5](C,B,D){$1$}
    \extkzLabelAngel[0.5](E,C,B){$2$}
    \tkzLabelPoints[above](A)
    \tkzLabelPoints[left](B,E)
    \tkzLabelPoints[right](C,D)
\end{tikzpicture}


    \caption{}\label{fig:czjh1-3-32}
\end{wrapfigure}


\liti 求证:等腰三角形两个底角的平分线相等。

已知: 在 $\triangle ABC$ 中, $AB = AC$,
$BD$ 和 $CE$ 分别是 $\angle B$ 和 $\angle C$ 的平分线(图 \ref{fig:czjh1-3-32})。

求证: $BD = CE$。

\zhengming $\because$ \quad $AB = AC$ (已知),

$\therefore$ \quad $\angle ABC = \angle ACB$ (等边对等角)。

又 $\because$ \quad \begin{zmtblr}[t]{}
    $\angle 1 = \exdfrac{1}{2} \angle ABC$, \\[.5em]
    $\angle 2 = \exdfrac{1}{2} \angle ACB$ (已知),
\end{zmtblr}

$\therefore$ \quad $\angle 1 = \angle 2$ (等式性质)。

在 $\triangle BDC$ 和 $\triangle CEB$ 中,

\hspace{2em} $\begin{cases}
    \angle ACB = \angle ABC \quad \text{(已证),} \\
    BC = CB \quad \text{(公共边),} \\
    \angle 1 = \angle 2 \quad \text{(已证),} \\
\end{cases}$

$\therefore$ \quad $\triangle BDC \quandeng \triangle CEB$ ($ASA$)。

$\therefore$ \quad $BD = CE$ (全等三角形对应边相等)。


\begin{wrapfigure}[8]{r}{5cm}
    \centering
    \begin{tikzpicture}
    \tkzDefPoints{0/0/B,  0.6/0/D,  2.4/0/E, 3.0/0/C,  1.5/2.5/A, 1.5/0/F}

    \tkzDrawPolygon(A,B,E)
    \tkzDrawSegment[dashed](A,F)
    \tkzMarkRightAngle(A,F,C)
    \tkzDrawPolygon(A,C,D)
    \tkzLabelPoints[above](A)
    \tkzLabelPoints[below](B,D,E,C,F)
\end{tikzpicture}


    \caption{}\label{fig:czjh1-3-33}
\end{wrapfigure}

\liti 已知:点 $D$、$E$ 在 $BC$ 上, $AB = AC$, $AD = AE$ (图 \ref{fig:czjh1-3-33})。

求证: $BD = CE$。

\zhengming 作 $AF \perp BC$,垂足是 $F$。

$\because$ \quad \begin{zmtblr}[t]{}
    $AB = AC$, \\
    $AD = AE$ (已知), \\
    $AF \perp BC$ (辅助线作法),
\end{zmtblr}

$\therefore$ \quad $BF = CF$, $DF = EF$ (等腰三角形底边上的高与底边上的中线互相重合)。

$\therefore$ \quad $BD = CE$ (等式性质)。



% \begin{wrapfigure}[8]{r}{5cm}
%     \centering
%     \begin{tikzpicture}
    \tkzDefPoints{0/0/B,  3.0/0/C,  2.5/2/A}
    \tkzInterLC[common=C](A,B)(A,C)  \tkzGetFirstPoint{D}

    \tkzDrawPolygon(A,B,C)
    \tkzDrawSegment[dashed](C,D)
    \tkzMarkSegments[mark=|](A,D  A,C)
    \tkzLabelPoints[above](A)
    \tkzLabelPoints[below](B,C)
    \tkzLabelPoints[left](D)
\end{tikzpicture}


%     \caption{}\label{fig:czjh1-3-34}
% \end{wrapfigure}

\liti \begin{xingzhi}
    在一个三角形中,如果两条边不等,那么它们所对的角也不等,大边所对的角较大。
\end{xingzhi}(简写成“\zhongdian{大边对大角}”。)

已知: $\triangle ABC$ 中, $AB > AC$ (图 \ref{fig:czjh1-3-34})。

求证: $\angle ACB > \angle B$。

\zhengming 在较大的边 $AB$ 上截取 $AD$, 使 $AD = AC$。连接 $CD$。

$\because$ \quad $AD = AC$ (辅助线作法),

$\therefore$ \quad $\angle ADC = \angle ACD$ (等腰三角形底角相等)。

$\because$ \quad $\angle ACB > \angle ACD$ (角的大小定义),

$\therefore$ \quad $\angle ACB > \angle ADC$ (等量代换)。

$\because$ \quad $\angle ADC > \angle B$ (三角形的外角大于不相邻的内角),

$\therefore$ \quad $\angle ACB > \angle B$ (不等式性质)。

注意:证明三角形的边或角不等的问题,常通过添加辅助线的办法,把它化成相等的情况进行研究。
辅助线的一般作法是,在大边上(或大角内)作出一部分等小边(或小角),得到等腰三角形。


\begin{figure}[htbp]
    \centering
    \begin{minipage}[b]{7cm}
        \centering
        \begin{tikzpicture}
    \tkzDefPoints{0/0/B,  3.0/0/C,  2.5/2/A}
    \tkzInterLC[common=C](A,B)(A,C)  \tkzGetFirstPoint{D}

    \tkzDrawPolygon(A,B,C)
    \tkzDrawSegment[dashed](C,D)
    \tkzMarkSegments[mark=|](A,D  A,C)
    \tkzLabelPoints[above](A)
    \tkzLabelPoints[below](B,C)
    \tkzLabelPoints[left](D)
\end{tikzpicture}


        \caption{}\label{fig:czjh1-3-34}
    \end{minipage}
    \qquad
    \begin{minipage}[b]{7cm}
        \centering
        \begin{tikzpicture}
    \tkzDefPoints{0/0/A,  -0.8/1/C}
    \tkzDefPointBy[rotation=center C angle 30](A)  \tkzGetPoint{e}
    \tkzDefPointOnLine[pos=1.4](C,e)  \tkzGetPoint{E} % CE > AC
    \tkzDefPointOnLine[pos=1.8](C,E)  \tkzGetPoint{D}
    \tkzInterLC(A,E)(D,E)             \tkzGetFirstPoint{b}
    \tkzDefPointOnLine[pos=1.4](E,b)  \tkzGetPoint{B} % BD > ED

    \tkzDrawPolygon(A,C,E)
    \tkzDrawPolygon(B,D,E)
    \tkzLabelPoints[above](C,E)
    \tkzLabelPoints[right](B)
    \tkzLabelPoints[below](A,D)
\end{tikzpicture}


        \caption*{(第 3 题)}
    \end{minipage}
\end{figure}


\begin{lianxi}

\xiaoti{(口答)}%
\begin{xiaoxiaotis}%
    \xxt[-1em]{怎样从等腰三角形的性质定理得出推论;等腰直角三角形的每一个锐角都等于 $45^\circ$?}

    \xxt{如果等腰三角形的一个底角等于 $75^\circ$, 那么它的顶角等于多少度?}

    \xxt{等腰直角三角形斜边上的高杷直角分成两个角,求这两个角的度数。}

\end{xiaoxiaotis}


\xiaoti{(口答)}%
\begin{xiaoxiaotis}%
    \xxt[-1em]{$\triangle ABC$ 中, 已知 $BC > AB > AC$, 比较三个角的大小;}

    \xxt{如果一个三角形最大的边所对的角是锐角,那么这个三角形一定是锐角三角形。为什么?}

\end{xiaoxiaotis}

\xiaoti{已知: $AB$ 和 $CD$ 相交于点 $E$, $CE > AC$, $BD > ED$。\\
    求证: $\angle A > \angle B$。
}

\end{lianxi}

