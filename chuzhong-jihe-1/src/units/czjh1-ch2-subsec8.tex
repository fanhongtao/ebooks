\subsection{命题、定理}\label{subsec:czjh1-2-8}

前面,我们学过一些图形的性质。例如, “对顶角相等”,
“如果两条直线都和第三条直线平行,那么这两条直线也互相平行”,
“两条直线被第三条直线所截,如果内错角相等,那么这两条直线平行”,
“两条平行线被第三条直线所截,同旁内角互补”,等等,都是判断某一件事情的语句。
象这样,判断一件事情的语句,叫做\zhongdian{命题}。

每一个命题都是由题设、结论两部分组成的。题设是已知事项;结论是由已知事项推出的事项。
命题常写成 “如果…,那么… ” 的形式, 用 “如果” 开始的部分就是题设,用 “那么” 开始的部分就是结论。
例如:命题 “如果两条直线都和第三条直线平行,那么这两条直线也互相平行”,
其中 “两条直线都和第三条直线平行” 是题设,“这两条直线也互相平行” 是结论。
但有的命题,如 “对顶角相等”, 它的题设与结论不十分明显,对于这样的命题,
应分清它的题设和结论,也可以将它写成 “如果…,那么…” 的形式。
例如,写成 “如果两个角是对顶角(题设),那么这两个角相等(结论)”。

上面列举的一些命题都是正确的命题,叫做\zhongdian{真命题}。
还有一类命题是错误的命题,叫做\zhongdian{假命题}。
例如,“相等的两个角是对顶角”、“内错角互补,两直线平行” 都是假命题。

前面,我们学过一些公理,如 “两点确定一条直线”, “两点之间线段最短”;
“平行公理”; “同位角相等,两直线平行” 也是命题,它们都是人们从长期的实践中总结出来的真命题,
并把它们作为判断其他命题的正确性的根据。

本节开头所列举的一些命题,都是用推理的方法判断为正确的命题。
用推理的方法判断为正确的命题叫做\zhongdian{定理}。


\begin{lianxi}

\xiaoti{判断下列语句中,哪些是命题,哪些不是命题。如果是命题,指出它的题设和结论。}
\begin{xiaoxiaotis}

    \xxt{两条直线相交,只有一个交点;}

    \xxt{在直线 $AB$ 上任取一点 $C$;}

\end{xiaoxiaotis}


\xiaoti{将下列命题改写成 “如果…,那么…” 的形式:}
\begin{xiaoxiaotis}

    \xxt{两条平行线被第三条直线所截,内错角相等;}

    \xxt{同旁内角互朴,两直线平行;}

    \xxt{同角的余角相等。}

\end{xiaoxiaotis}


\begin{minipage}[b]{8cm}
    \xiaoti{抄写下题,并在括号内加注理由。\\
    已知:如图, $\angle 1 = \angle 2$。\\
    求证: $AB \pingxing CD$。 \\
    \zhengming $\because$ \quad $\angle 1 = \angle 2$ (\hspace*{2cm}), \\
    $\angle 2 = \angle 3$  (\hspace*{2cm}),  \\
    $\therefore$ \quad $\angle 1 = \angle 3$  (\hspace*{2cm}),  \\
    $\therefore$ \quad $AB \pingxing CD$  (\hspace*{2cm})。
}
\end{minipage}
\quad
\begin{minipage}[b]{6cm}
    \centering
    \begin{tikzpicture}
    \tkzDefPoints{0/0/C, 4/0/D, 0/1.2/A, 4/1.2/B, 1/2.2/E, 3/-1/F}
    \tkzInterLL(A,B)(E,F)  \tkzGetPoint{O}
    \tkzInterLL(C,D)(E,F)  \tkzGetPoint{P}

    \tkzDrawSegments(A,B  C,D  E,F)
    \tkzMarkAngles[size=0.3](C,P,F)
    \tkzMarkAngles[size=0.4](B,O,E  D,P,E)
    \tkzLabelAngle[pos=0.5](C,P,F){$2$}
    \tkzLabelAngle[pos=0.6](B,O,E){$1$}
    \tkzLabelAngle[pos=0.6](D,P,E){$3$}
    \tkzLabelPoints[below](A,B,C,D)
    \tkzLabelPoints[right](E)
    \tkzLabelPoints[left](F)
\end{tikzpicture}


    (第 3 题)
\end{minipage}

\end{lianxi}

