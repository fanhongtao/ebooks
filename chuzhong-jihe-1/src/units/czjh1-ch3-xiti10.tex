\xiti
\begin{xiaotis}
\begin{enhancedline}

\xiaoti{证明:如果三角形一边上的中线等于这条边的一半,那么这个三角形是直角三角形。}

\xiaoti{已知:如图,$CD$、$CE$、$CM$ 分别是 $Rt \triangle ABC$ 斜边上的高、角平分线和中线。\\
    求证:$\angle 1 = \angle 2$。
}

\begin{figure}[htbp]
    \centering
    \begin{minipage}[b]{7cm}
        \centering
        \begin{tikzpicture}
    \tkzDefPoints{0/0/A, 5/0/B}
    \tkzDefTriangle[two angles=60 and 30](A,B)  \tkzGetPoint{C}
    \tkzDefLine[altitude](A,C,B)  \tkzGetPoint{D}
    \tkzDefLine[bisector,normed](A,C,B) \tkzGetPoint{e}
    \tkzInterLL(A,B)(C,e)  \tkzGetPoint{E}
    \tkzDefMidPoint(A,B)  \tkzGetPoint{M}

    \tkzDrawPolygon(A,B,C)
    \tkzDrawSegments(C,D  C,E  C,M)
    \tkzMarkRightAngle(C,D,A)
    \extkzLabelAngel[1.0](D,C,E){$1$}
    \extkzLabelAngel[0.8](E,C,M){$2$}
    \tkzLabelPoints[above](C)
    \tkzLabelPoints[below](D,E,M)
    \tkzLabelPoints[left](A)
    \tkzLabelPoints[right](B)
\end{tikzpicture}


        \caption*{(第 2 题)}
    \end{minipage}
    \qquad
    \begin{minipage}[b]{7cm}
        \centering
        \input{../pic/czjh1-ch3-xiti10-04}
        \caption*{(第 4 题)}
    \end{minipage}
\end{figure}


\xiaoti{三角形三个角的度数之比为 $1:2:3$, 它的最大边的长等于 $16 \;\limi$,求最小边的长。}

\xiaoti{已知:$\triangle ABC$ 中, $\angle ACB = 90^\circ$, $CD$ 是高,$\angle A = 30^\circ$。\\
    求证: $DB = \exdfrac{1}{4} AB$。
}

\xiaoti{等腰三角形的底角等于 $15^\circ$, 腰长等于 $2a$, 求腰上的高。}

\xiaoti{已知直角三角形一条直角边等于 $10 \;\limi$, 它所对的角为 $60^\circ$。 求斜边上的高。}

\xiaoti{已知等腰三角形底边上的高等于腰长的一半。 求这个等腰三角形各角的度数。}

\xiaoti{已知:如图, $OA = OB$, $AC \perp OA$, $BC \perp OB$。\\
    求证:$\angle AOC = \angle BOC$。
}

\xiaoti{已知:如图, $AB = CD$, $DE \perp AC$, $BF \perp AC$, $E$、$F$ 是垂足, $DE = BF$。\\
    求证:(1)$AE = CF$; (2)$AB \pingxing CD$。
}

\begin{figure}[htbp]
    \centering
    \begin{minipage}[b]{4.5cm}
        \centering
        \begin{tikzpicture}[scale=0.8]
    \tkzDefPoints{0/0/O, 4/0/C}
    \tkzDefTriangle[two angles=30 and 60](O,C)  \tkzGetPoint{A}
    \tkzDefPointBy[reflection=over O--C](A)  \tkzGetPoint{B}
    \tkzDrawPolygon(O,A,C)
    \tkzDrawPolygon(O,B,C)
    \tkzMarkRightAngle(O,A,C)
    \tkzMarkRightAngle(O,B,C)
    \tkzLabelPoints[above](A)
    \tkzLabelPoints[below](B)
    \tkzLabelPoints[left](O)
    \tkzLabelPoints[right](C)
\end{tikzpicture}


        \caption*{(第 8 题)}
    \end{minipage}
    \qquad
    \begin{minipage}[b]{5.0cm}
        \centering
        \begin{tikzpicture}[scale=0.8]
    \tkzDefPoints{0/0/A, 4/0/B, 0.8/3/D,  4.8/3/C}
    \tkzDefLine[altitude](A,B,C)  \tkzGetPoint{F}
    \tkzDefLine[altitude](A,D,C)  \tkzGetPoint{E}
    \tkzDrawPolygon(A,B,F)
    \tkzDrawPolygon(C,D,E)
    \tkzMarkRightAngle(A,F,B)
    \tkzMarkRightAngle(C,E,D)
    \tkzLabelPoints[left](A,D)
    \tkzLabelPoints[right](B,C)
    \tkzLabelPoints[above left](F)
    \tkzLabelPoints[below right](E)
\end{tikzpicture}


        \caption*{(第 9 题)}
    \end{minipage}
    \qquad
    \begin{minipage}[b]{4.5cm}
        \centering
        \begin{tikzpicture}[scale=0.8]
    \tkzDefPoints{0/0/B, 4/0/C,  3/2/A}
    \tkzDefMidPoint(B,C)  \tkzGetPoint{D}
    \tkzDefLine[altitude](A,C,D)  \tkzGetPoint{F}
    \tkzDefLine[altitude](A,B,D)  \tkzGetPoint{E}
    \tkzDrawPolygon(A,B,C)
    \tkzDrawSegments(A,E  B,E  C,F)
    \tkzMarkRightAngle(A,E,B)
    \tkzMarkRightAngle(C,F,E)
    \tkzLabelPoints[above](A)
    \tkzLabelPoints[left](B)
    \tkzLabelPoints[right](C)
    \tkzLabelPoints[below,xshift=0.3em](D)
    \tkzLabelPoints[below right](E)
    \tkzLabelPoints[left,yshift=0.1em](F)
\end{tikzpicture}


        \caption*{(第 12 题)}
    \end{minipage}
\end{figure}

\xiaoti{已知: $BE$ 和 $CF$ 是 $\triangle ABC$ 的高, $BE = CF$, $H$ 是 $BE$ 和 $CF$ 的交点。\\
    求证: $HB = HC$。
}

\xiaoti{求证: 有一条直角边和斜边上的高对应相等的两个直角三角形全等。}

\xiaoti{已知;如图, $AD$ 是 $\triangle ABC$ 的中线, $CF \perp AD$, 交 $AD$ 于 $F$,
    $BE \perp AD$,交 $AD$ 的延长线于 $E$。 \\
    求证:$CF = BE$。
}

\end{enhancedline}
\end{xiaotis}

