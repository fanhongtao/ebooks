\subsection{线段的垂直平分线}\label{subsec:czjh1-3-15}

设 $MN$ 是线段 $AB$ 的垂直平分线,点 $P$ 是直线 $MN$ 上任意一点(图 \ref{fig:czjh1-3-54})。连结 $PA$、$PB$。

\begin{figure}[htbp]
    \centering
    \begin{minipage}[b]{7cm}
        \centering
        \begin{tikzpicture}
    \tkzDefPoints{0/0/A, 3/0/B}
    \tkzDefLine[mediator, normed](A,B)  \tkzGetPoints{m}{N}
    \tkzDefPointOnLine[pos=2.5](N,m)  \tkzGetPoint{M}
    \tkzInterLL(A,B)(M,N)  \tkzGetPoint{C}
    \tkzDefPointOnLine[pos=0.2](M,N)  \tkzGetPoint{P}
    \tkzDrawSegments(A,B  M,N)
    \tkzDrawSegments[dashed](A,P  B,P)
    \tkzMarkRightAngle(B,C,M)
    \tkzLabelPoints[right](M,N,P)
    \tkzLabelPoints[below](A,B)
    \tkzLabelPoints[below right](C)
\end{tikzpicture}


        \caption{}\label{fig:czjh1-3-54}
    \end{minipage}
    \qquad
    \begin{minipage}[b]{7cm}
        \centering
        \input{../pic/czjh1-ch3-55}
        \caption{}\label{fig:czjh1-3-55}
    \end{minipage}
\end{figure}


$\because$ \quad $AC = BC$, $PC = PC$, $\angle PCA = \angle PCB = Rt \angle$,

$\therefore$ \quad $\triangle PCA \quandeng \triangle PCB$ ($SAS$)。

$\therefore$ \quad $PA = PB$。

因此,我们得到下面定理:

\begin{dingli}[定理]
    线段垂直平分线上的点和这条线段两个端点的距离相等。
\end{dingli}

这个定理有逆定理:

\begin{dingli}[逆定理]
    和一条线段两个端点距离相等的点,在这条线段的垂直平分线上。
\end{dingli}

已知:$PA = PB$ (图 \ref{fig:czjh1-3-55})。

求证:点 $P$ 在线段 $AB$ 的垂直平分线上。

\zhengming: 经过点 $P$ 作 $MN \perp AB$,垂足是 $C$,那么

$\angle PCA = \angle PCB = Rt \angle$。

在 $Rt \triangle PCA$ 和 $Rt \triangle PCB$ 中,

$PA = PB$, $PC = PC$,

$\therefore$ \quad $Rt \triangle PCA \quandeng Rt \triangle PCB$ ($HL$)。

$\therefore$ \quad $AC = BC$。

$\therefore$ \quad $MN$ 是 $AB$ 的垂直平分线。 就是说,点 $P$ 在 $AB$ 的垂直平分线上。

根据上述的定理和逆定理可以知道,
在 $AB$ 的垂直平分线 $MN$ 上的点和两点 $A$、$B$ 的距离都相等;
反过来,到两点 $A$、$B$ 的距离相等的点都在 $MN$ 上,
所以直线 $MN$ 可以看作是和两点 $A$、$B$ 的距离相等的所有点的集合。

\zhongdian{线段的垂直平分线可以看作是和线段两个端点距离相等的所有点的集合。}

\liti[0] 求证:三角形三边的垂直平分线交于一点。

已知: $MN$、$GH$、$PQ$ 分别是 $\triangle ABC$ 三边 $AB$、$BC$、$CA$ 的垂直平分线(图 \ref{fig:czjh1-3-56})。

求证:$MN$、$GH$、$PQ$ 交于一点。

分析:我们知道,两条相交直线只有一个交点。
要证明三条直线交于一点,只要证明第三条直线通过前两条直线的交点就可以了。

\zhengming 设 $MN$、$PQ$ 交于点 $R$. 连结 $AR$、$BR$、$CR$。

$\because$ \quad $MN$ 是 $AB$ 的垂直平分线,

$\therefore$ \quad $AR = BR$(线段垂直平分线上的点和这条线段两个端点的距离相等)。

同理 \quad $AR = CR$。

$\therefore$ \quad $BR = CR$。

$\therefore$ \quad 点 $R$ 在 $GH$ 上(和一条线段的两个端点距离相等的点,在这条线段的垂直平分线上)。

因此,$MN$、$GH$、$PQ$ 交于一点。

\begin{figure}[htbp]
    \centering
    \begin{minipage}[b]{7cm}
        \centering
        \begin{tikzpicture}
    \tkzDefPoints{0/0/B, 4/0/C, 3.5/2.2/A}
    \tkzDefLine[mediator, normed](A,B)  \tkzGetPoints{m}{n}
    \tkzDefLine[mediator, normed](B,C)  \tkzGetPoints{g}{h}
    \tkzDefLine[mediator, normed](C,A)  \tkzGetPoints{p}{q}
    \tkzInterLL(m,n)(g,h)  \tkzGetPoint{R}
    \tkzInterLL(m,n)(A,B)  \tkzGetPoint{N}
    \tkzInterLL(g,h)(B,C)  \tkzGetPoint{H}
    \tkzInterLL(p,q)(A,C)  \tkzGetPoint{P}
    \tkzDefPointOnLine[pos=1.8](N,R)  \tkzGetPoint{M}
    \tkzDefPointOnLine[pos=1.5](H,R)  \tkzGetPoint{G}
    \tkzDefPointOnLine[pos=1.4](P,R)  \tkzGetPoint{Q}

    \tkzDrawPolygon(A,B,C)
    \tkzDrawLines[add=0 and 0.3](M,N)
    \tkzDrawLines[add=0 and 0.3](G,H)
    \tkzDrawLines[add=0 and 0.2](Q,P)
    \tkzDrawSegments[dashed](A,R  B,R  C,R)
    \tkzMarkRightAngle[size=0.2](R,N,B)
    \tkzMarkRightAngle[size=0.2](R,H,B)
    \tkzMarkRightAngle[size=0.2](R,P,C)

    \tkzLabelPoints[above](A)
    \tkzLabelPoints[left](B)
    \tkzLabelPoints[right](C)
    \tkzLabelPoints[below,xshift=0.5em,yshift=0.3em](M)
    \tkzLabelPoints[above left](N)
    \tkzLabelPoints[right,xshift=-0.2em](G)
    \tkzLabelPoints[below right](H)
    \tkzLabelPoints[below right](P)
    \tkzLabelPoints[below](Q)
    \tkzLabelPoints[right](R)
\end{tikzpicture}


        \caption{}\label{fig:czjh1-3-56}
    \end{minipage}
    \qquad
    \begin{minipage}[b]{7cm}
        \centering
        \begin{tikzpicture}
    \tkzDefPoints{0/0/M, 4/0/N, 1/1.5/A,  2.5/1.0/B}
    \tkzDrawSegments(M,N)
    \tkzDrawPoints(A,B)
    \tkzLabelPoints[below](M,N)
    \tkzLabelPoints[left](A)
    \tkzLabelPoints[right](B)
\end{tikzpicture}


        \caption*{(第 1 题)}
    \end{minipage}
\end{figure}

\begin{lianxi}

\xiaoti{如图,在直线 $MN$ 上求作一 $P$,使 $PA = PB$。}

\xiaoti{已知: $\triangle ABC$。求作:一点 $O$,使 $OA = OB = OC$。}

\end{lianxi}

