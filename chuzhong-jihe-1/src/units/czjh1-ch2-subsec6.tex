\subsection{平行线的判定}\label{subsec:czjh1-2-6}

现在,我们来研究怎样才能判断两条直线是平行线。

在上一节画平行线 $AB$、$CD$(图 \ref{fig:czjh1-2-15}) 时,我们是把三角板由虚线位置推到实线位置画出的,
这就是说,把尺边 $EF$ 看作截线,使同位角相等,作出的直线 $AB$、$CD$ 就是平行线。
我们把这个事实作为判定两条直线平行的公理:

\begin{gongli}[公理]
    两条直线被第三条直线所截,如果同位角等,那么这两条直线平行。
\end{gongli}

这个公理可以简单说成:\begin{gongli}
    同位角相等,两直线平行。
\end{gongli}

例如,两条直线 $AB$、$CD$ 被直线 $EF$ 所截(图 \ref{fig:czjh1-2-17}),
如果已知同位角 $\angle 1$ 与 $\angle 2$ 都是 $85^\circ$,
那么 $\angle 1 = \angle 2$,就可以判定 $AB \pingxing CD$。
根据的就是 “同位角相等,两直线平行”。

\begin{figure}[htbp]
    \centering
    \begin{minipage}[b]{7cm}
        \centering
        \begin{tikzpicture}
    \tkzDefPoints{0/0/C, 4/0/D, 0/1.5/A, 4/1.5/B, 3/2.5/E, 1/-1/F}
    \tkzInterLL(A,B)(E,F)  \tkzGetPoint{O}
    \tkzInterLL(C,D)(E,F)  \tkzGetPoint{P}

    \tkzDrawSegments(A,B  C,D  E,F)
    \tkzMarkAngles[size=0.3](B,O,E  D,P,E)
    \tkzLabelAngle[pos=0.5](B,O,E){$1$}
    \tkzLabelAngle[pos=0.5](D,P,E){$2$}
    \tkzLabelPoints[below](A,B,C,D,F)
    \tkzLabelPoints[right](E)
\end{tikzpicture}


        \caption{}\label{fig:czjh1-2-17}
    \end{minipage}
    \qquad
    \begin{minipage}[b]{7cm}
        \centering
        \input{../pic/czjh1-ch2-18}
        \caption{}\label{fig:czjh1-2-18}
    \end{minipage}
\end{figure}


根据公理,由同位角相等可以判定两条直线平行。
那么由内错角、同旁内角的关系是否也可以判定两条直线平行呢?
我们来看图\ref{fig:czjh1-2-18}。 因为由同位角 $\angle 1 = \angle 2$,就可以推出 $AB \pingxing CD$。
所以只要找出在什么情形下,$\angle 1 = \angle 2$ 就可以了。
在图中我们发现 $\angle 1 = \angle 3$,$\angle 1$ 与 $\angle 4$ 互补。
如果内错角 $\angle 3 = \angle 2$, 或同旁内角 $\angle 4$ 与 $\angle 2$ 互补,
就可以推出 $\angle 1 = \angle 2$,于是 $AB \pingxing CD$。下面我们写出推理过程:

$\because$ \quad \begin{zmtblr}[t]{}
    $\angle 3 = \angle 2$(已知)\\
    $\angle 1 = \angle 3$(对顶角相等)
\end{zmtblr}

$\therefore$ \quad $\angle 1 = \angle 2$(等量代换)

$\therefore$ \quad $AB \pingxing CD$(同位角相等,两直线平行)

因此,我们得到另一个平行线的判定方法:

\begin{gongli}
    两条直线被第三条直线所截,如果内错角相等,那么这两条直线平行。
\end{gongli}

这句话可以简单说成:\begin{gongli}
    内错角相等,两直线平行。
\end{gongli}

$\because$ \quad \begin{zmtblr}[t]{}
    $\angle 4$ 与 $\angle 2$ 互补(已知)\\
    $\angle 1$ 与 $\angle 4$ 互补(邻补角定义)
\end{zmtblr}

$\therefore$ \quad $\angle 1 = \angle 2$(同角的补角相等)

$\therefore$ \quad $AB \pingxing CD$(同位角相等,两直线平行)

由此,我们得到又一个平行线的判定方法:

\begin{gongli}
    两条直线被第三条直线所截,如果同旁内角互补,那么这两条直线平行。
\end{gongli}

这句话可以简单说成\begin{gongli}%
    同旁内角互补,两直线平行。
\end{gongli}


\begin{lianxi}

口答下列各题:

\xiaoti{如图,直线 $AB$ 和 $CD$ 被直线 $EF$ 所截。}
\begin{xiaoxiaotis}

    \xxt{量得 $\angle 1 = 80^\circ$, $\angle 2 = 80^\circ$, 就可以判定 $AB \pingxing CD$。它的根据是什么?}

    \xxt{量得 $\angle 3 = 100^\circ$, $\angle 4 = 100^\circ$, 也可以判定 $AB \pingxing CD$。它的根据是什么?}

    \xxt{量得 $\angle 1 = 80^\circ$, $\angle 3 = 100^\circ$, 也可以判定 $AB \pingxing CD$。它的根据是什么?}

\end{xiaoxiaotis}

\begin{figure}[htbp]
    \centering
    \begin{minipage}[b]{7cm}
        \centering
        \begin{tikzpicture}
    \tkzDefPoints{0/0/E, 4/0/F, 1/0/O, 3/0/P}

    \tkzDefPointOnCircle[R = center O angle 80  radius 1.5] \tkzGetPoint{A}
    \tkzDefPointOnCircle[R = center O angle 260 radius 1.5] \tkzGetPoint{B}
    \tkzDefPointOnCircle[R = center P angle 80  radius 1.5] \tkzGetPoint{C}
    \tkzDefPointOnCircle[R = center P angle 260 radius 1.5] \tkzGetPoint{D}

    \tkzDrawSegments(A,B  C,D  E,F)
    \tkzMarkAngles[size=0.3](F,O,A  F,P,C)
    \tkzMarkAngles[size=0.4](C,P,E  B,O,F)
    \tkzLabelAngle[pos=0.5](F,O,A){$1$}
    \tkzLabelAngle[pos=0.5](F,P,C){$2$}
    \tkzLabelAngle[pos=0.6](C,P,E){$3$}
    \tkzLabelAngle[pos=0.6](B,O,F){$4$}
    \tkzLabelPoints[below](E,F)
    \tkzLabelPoints[right](A,B,C,D)
\end{tikzpicture}


        \caption*{(第 1 题)}
    \end{minipage}
    \qquad
    \begin{minipage}[b]{7cm}
        \centering
        \begin{tikzpicture}
    \tkzDefPoints{0/0/A, 4/0/B, 5/0/E, 1/2/D, 5/2/C}

    \tkzDrawPolygon(A,B,C,D)
    \tkzDrawSegment(B,E)
    \tkzLabelPoints[below](A,B,E)
    \tkzLabelPoints[above](C,D)
\end{tikzpicture}


        \caption*{(第 2 题)}
    \end{minipage}
\end{figure}


\xiaoti{如图, $BE$ 是 $AB$ 的延长线。量得 $\angle CBE = \angle A = \angle C$。}
\begin{xiaoxiaotis}

    \xxt{从 $\angle CBE  = \angle A$, 可以判定哪两条直线平行?它的根据是什么?}

    \xxt{从 $\angle CBE  = \angle C$, 可以判定哪两条直线平行?它的根据是什么?}

\end{xiaoxiaotis}


\xiaoti{图 \ref{fig:czjh1-2-7} 中, 木工用角尺画出工件边缘的几条垂线,这些垂线是否平行?能否用几何语言说出这个事实。}

\end{lianxi}

