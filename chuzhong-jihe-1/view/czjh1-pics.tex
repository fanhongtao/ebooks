% 将本书的 pic/ 目录下所有图片,按文件名顺序显示出来。
% 方便快速查找图片,以便借鉴其实现。
\documentclass[UTF8, 11pt, oneside]{ctexbook}
\input{../src/czjh1-styles}
\input{../src/czjh1-symbols}
\input{../src/czjh1-tikz-snippets}

\begin{document}

% czjh1-ch1-subsec6 中的定义
% 这种提前定义的写法,会使绘图的 .tex 文件更简洁,
% 但换一个地方使用,又需要自行定义。后期尽量不这样写。
\def\baseabc{
    \tkzDefPoint(0,0){O}
    \tkzDefPoint(0:2){A}
    \tkzDefPoint(40:2){B}
    \tkzDrawSegments(O,A  O,B)
    \tkzLabelPoints[below](O, A)
    \tkzLabelPoints[right](B)
}
\def\comparedabc{
    \tkzDefPoint(0,0){O}
    \tkzDefPoint(0:2){A}
    \tkzDefPoint(40:2){B}
    \tkzDrawSegments(O,A  O,B)
    \tkzLabelPoint[below](O){$(O)$}
    \tkzLabelPoint[below](A){$(A)$}

    \tkzLabelPoint[above](O){$O'$}
    \tkzLabelPoint[above](A){$A'$}
}

% all_pics.tex 文件由根目录下的 create_all_pics.sh 生成
\begin{tikzpicture}[>=Stealth,
    every node/.style={fill=white, inner sep=1pt},
    scale=0.7,
]
    \pgfmathsetmacro{\a}{8}
    \pgfmathsetmacro{\b}{6}
    \pgfmathsetmacro{\c}{1.5}

    \draw [ultra thick] (0, 0) rectangle (\a, \b);

    \draw [ultra thick] (0, 0) rectangle (\c, \c);
    \draw [ultra thick] (\a - \c, 0) rectangle (\a, \c);
    \draw [ultra thick] (0, \b - \c) rectangle (\c, \b);
    \draw [ultra thick] (\a - \c, \b - \c) rectangle (\a, \b);

    \draw [dashed] (\c, \c) rectangle (\a - \c, \b - \c);

    \draw [<->] (0, \b+0.5) to [xianduan={below=0.5cm}] node {$80$} (\a, \b+0.5);
    \draw [<->] (0, \b - \c/2) to node {$x$} (\c, \b - \c/2);
    \draw [<->] (\c, \b - \c/2) to node {$80 - 2x$} (\a - \c, \b - \c/2);

    \draw [<->] (-0.5, 0) to [xianduan={below=0.5cm}]  node [rotate=90] {$60$} (-0.5, \b);
    \draw [<->] (\c/2, 0) to node [rotate=90] {$x$} (\c/2, \c);
    \draw [<->] (\c/2, \c) to node [rotate=90] {$60 - 2x$} (\c/2, \b - \c);
\end{tikzpicture}


../pic/czds3-ch11-1.tex

\begin{tikzpicture}[>=Stealth,
    every node/.style={fill=white, inner sep=1pt},
]
    \pgfmathsetmacro{\a}{6*0.3}
    \pgfmathsetmacro{\b}{12*0.3}
    \coordinate (A) at (0, 0);
    \coordinate (B) at (\a, 0);
    \coordinate (C) at (\a, \b);
    \coordinate (P) at (2*0.3, 0);
    \coordinate (Q) at (\a, 4*0.3);

    \draw [ultra thick] (A) -- (B) -- (C) -- cycle;
    \draw (A) node [anchor=north east] {$A$};
    \draw (B) node [anchor=north west] {$B$};
    \draw (C) node [above] {$C$};

    \draw (P) + (0.5, 0.2) node {$P$};
    \draw (Q) node [left=0.25, above] {$Q$};
    \draw [ultra thick] (P) -- (Q);

    \draw [<->] (0, -0.2) to [xianduan] node {$6$\; cm} (\a, -0.2);
    \draw [<->] (\a + 0.2, 0) to [xianduan] node [rotate=90] {$12$\; cm} (\a + 0.2, \b);

    \draw [->] (0.1, 0.2) -- (2*0.3, 0.2);
    \draw [->] (\a - 0.2, 0) -- (\a - 0.2, 2*0.3);
\end{tikzpicture}



../pic/czds3-ch11-fuxi11-10.tex

\begin{tikzpicture}[>=Stealth,
    every node/.style={fill=white, inner sep=1pt},
]
    \pgfmathsetmacro{\a}{3.2}
    \pgfmathsetmacro{\b}{2}
    \pgfmathsetmacro{\c}{0.2}

    \pgfmathsetmacro{\x}{0.8}
    \pgfmathsetmacro{\y}{1.0}

    \draw [ultra thick, pattern={mylines[angle=45, distance={5pt}]}] (0, 0) rectangle (\a, \b);

    \draw [ultra thick, fill=white] (0, 0) rectangle (\x, \y);
    \draw [ultra thick, fill=white] (\x + \c, 0) rectangle (\a, \y);
    \draw [ultra thick, fill=white] (0, \y + \c) rectangle (\x, \b);
    \draw [ultra thick, fill=white] (\x + \c, \y + \c) rectangle (\a, \b);

    \draw [<->] (0, -0.3) to [xianduan={above=0.3cm}] node {$32 \; cm$} (\a, -0.3);
    \draw [<->] (-0.3, 0) to [xianduan={below=0.3cm}]  node [rotate=90] {$20 \; cm$} (-0.3, \b);
\end{tikzpicture}



../pic/czds3-ch11-subsec4-lianxi-5.tex

\begin{tikzpicture}[>=Stealth,
    every node/.style={fill=white, inner sep=1pt},
]
    \pgfmathsetmacro{\w}{18*0.3}
    \pgfmathsetmacro{\a}{15*0.3}
    \pgfmathsetmacro{\b}{10*0.3}
    \draw [<->] (-\w/2, 0.9) to [xianduan={below=0.5cm}] node {$18$\; m} (\w/2, 0.9);
    \draw [pattern={mylines[angle=45, distance={5pt}]}] (-\w/2, 0) rectangle (\w/2, .5);
    \draw (-\a/2, 0) -- (-\a/2, -\b) -- (\a/2, -\b) -- (\a/2, 0);
    \node at (0, -\b/2) {鸡场};
\end{tikzpicture}



../pic/czds3-ch11-xiti6-5.tex

\begin{tikzpicture}
    \draw (0, 3) ellipse [x radius=2, y radius=1] node[below=1.2] {正有理数集合};
    \draw (5, 3) ellipse [x radius=2, y radius=1] node[below=1.2] {负有理数集合};
    \draw (0, 0) ellipse [x radius=2, y radius=1] node[below=1.2] {正无理数集合};
    \draw (5, 0) ellipse [x radius=2, y radius=1] node[below=1.2] {负无理数集合};
\end{tikzpicture}


../pic/czds3-ch9-fuxi9-27.tex

\begin{tikzpicture}[>=Stealth]
    \foreach \a/\b/\c [count=\xi] in {
            \hphantom{+}8/-8/?,
            \hphantom{+}\frac{3}{4}/-\frac{3}{4}/?,
            \hphantom{+}?/\hphantom{+}?/121,
            \hphantom{+}?/\hphantom{+}?/0.36
    } {
        \coordinate (A) at (0, 4-\xi);
        \coordinate (B) at (0, 4-\xi - 0.5);
        \coordinate (C) at (4, 4-\xi-0.25);
        \path
            let
                \p{ac} = ($ (A) !.5! (C) $),
                \p{bc} = ($ (B) !.5! (C) $)
            in
                coordinate (AC) at (\p{ac})
                coordinate (BC) at (\p{bc});
        \draw [->] (A) node [anchor=east] {$\a$} -- (AC);
        \draw (AC) -- (C);
        \draw [->] (B) node [anchor=east] {$\b$} -- (BC);
        \draw (BC) -- (C) node [anchor=west] {$\c$};
    }

    \draw (0,   1.3) ellipse [x radius=1.0, y radius=2.2];
    \draw (4.2, 1.3) ellipse [x radius=1.0, y radius=2.2];
    \draw (0,   3.8) node {$x$};
    \draw (4.2, 3.8) node {$x^2$};
\end{tikzpicture}


../pic/czds3-ch9-subsec1-lianxi-4.tex

\begin{tikzpicture}
    \draw (0, 0) ellipse [x radius=3, y radius=1.2] node[below=1.6] {有理数集合};
    \draw (7, 0) ellipse [x radius=3, y radius=1.2] node[below=1.6] {无理数集合};
\end{tikzpicture}


../pic/czds3-ch9-xiti2-8.tex



\end{document}
