\section{实验:用天平称物体的质量}\label{sec:1-7}

\shiyan{目的} 学习用托盘天平称物体的质量。

\shiyan{器材} 托盘天平,砝码,几个相同的硬币(或者几个相伺的钮扣),铁块, 1 厘米长的一段棉线。

\shiyan{步骤}

(1) 调节天平横梁右端的螺母,使横梁平衡。

(2) 把铁块放在左盘里,先根据估计,用镊子往右盘里试加砝码,然后移动游码,直到横梁平衡。

(3) 横梁平衡后,计算砝码的总质量并观察游码所对的刻度值,得出所称的铁块的质量。

(4) 称量完毕,把砝码全部放回盒内,不要遗漏。

(5) 把几个相同的硬币放在左盘里,称出它们的总质量。

(6) 求出一个硬币的质量的平均值。

(7) \mylabel{celiang-keben-zhiliang}称出物理课本的质量,并把它记在\pageref{shiyong-keben-zhiliang} 页 \hyperref[shiyong-keben-zhiliang]{第 (4) 题} 的后边。

\shiyan{观察与思考}

(1) 将 1 厘米长的棉线放在天平左盘里,称称看,能称出它的质量来吗?想一想,用什么办法能够测出 1 厘米长的棉线的质量?

(2) 想想看,怎样用天平称液体的质量。

\lianxi

(1) 一个同学为了验证冰融化成水后质量不变,他先测出了冰的质量,然后把冰放入一个开口的烧瓶里加热,直到水沸腾了,他才去测量水的质量。
可是测量的结果表明水的质量减小了,你能找出质量减小的原因吗?

(2) 考古学家研究了一种恐龙化石后,认为这种恐龙活着时质量大约有 50 吨。一只恐龙的质量相当于多少个 50 千克的人的质量?

(3) 有一堆同一规格的小零件,每个只有几十毫克,估计有几千个。
手边有一架天平,你能利用它很快知道这堆零件的确切数目吗?用具体的数字例子来说明你的办法。

