\section{实验:研究物体浮在液面的条件}\label{sec:6-4}

在这个实验里,我们利用装着沙子的试管作为浮在液面上的物体,研究它受到的浮力跟它受到的重力有什么关系。
利用量筒可以测出试管排开的水的体积,进一步算出它受到的浮力。利用天平称出试管和沙子的质量,根据 $G = mg$ 可以算出它们的总重。

\shiyan{目的} 研究物体浮在液面的条件。

\shiyan{器材} 天平,砝码,试管,沙子,量筒,水,角匙。

\shiyan{实验步骤}

(1) 向量筒中倒入适量的水,记住水的体积。

(2) 在试管中装少量沙子。把试管放在量筒中的水里,使试管能够竖直地浮在水面上。
读出这时水面所对的刻度值,求出试管排开的水的体积,记在下面的表里。

(3) 擦干试管,用天平称出试管和沙子的质量,再算出它们的总重,记在下面的表里。

(4) 再向试管里加一些沙子,重做上面的实验。这样重复两次。

(5) 根据测得的数据算出试管受到的浮力。比较各次实验中试管和沙子的总重跟浮力的大小,
说明物体浮在液面的条件是什么。

\begin{table}[H]
    \centering
    \begin{tabular}{|w{c}{4em}|*{6}{w{c}{5em}|}}
        \hline
        \tabincell{c}{实验\\次数} & \tabincell{c}{量筒内水\\的体积\\($\lflm$)} & \tabincell{c}{放入试管后\\筒内水面升\\到的刻度\\($\lflm$)} & \tabincell{c}{试管排开\\的水的体\\积($\lflm$)} & \tabincell{c}{试管受到\\的浮力\\(牛顿)} & \tabincell{c}{试管和沙\\子的质量\\(克)} & \tabincell{c}{试管和沙\\子的总重\\(牛顿)} \\ \hline
        1 & & & & & & \\ \hline
        2 & & & & & & \\ \hline
        3 & & & & & & \\ \hline
    \end{tabular}
\end{table}


\shiyan{思考题} 试管和沙子的总重增大时,试管浸入水中的体积怎样变化?
试管受到的浮力怎样变化?想一想,试管和沙子的总重增加到多大时,试管就不能浮在水面上了?



\nonumsection{小实验:研究鸡蛋在盐水中的浮沉}

取一个新鲜鸡蛋,放在茶缸里,向茶缸里倒入一些清水,观察鸡蛋在清水里的浮沉情况。
向水里慢慢放入一些盐,轻轻用筷子搅拌,使盐溶化,观察鸡蛋的浮沉情况有什么变化。解释看到的现象。

