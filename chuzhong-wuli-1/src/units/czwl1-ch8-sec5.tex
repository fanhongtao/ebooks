\section{机械效率}\label{sec:8-5}

前面研究简单机械和功的原理时,我们都没有考虑摩擦。实际上,摩擦总是存在的。
例如,利用滑轮提起重物时,滑轮的轮和轴之间有摩擦,利用斜面提高物体时,斜面和物体之间有摩擦。
由于要克服摩擦,我们不得不比没有摩擦时用的力大,做的功也要比没有摩擦时做的多。

我们利用机械提高物体的时候,用来提高物体的功对我们是有用的,是必须做的,
而用来克服摩擦的功对我们没有用,是我们不希望做但又不得不额外做的。

利用机械工作的时候,对人们有用的功叫\textbf{有用功},
对人们没有用但又不得不额外做的功叫\textbf{额外功}。

除了摩擦,还常常由于别的原因,而不得不做额外功。
例如,利用动滑轮提高物体的时候,动滑轮本身也被提起,
因此,在提高物体做有用功的同时,不得不为提高动滑轮本身而做额外功。

有用功与额外功的和叫\textbf{总功},它是我们总共做的功。

有用功只占总功的一部分。\textbf{有用功跟总功的比值,叫做机械效率}。

如果用 $W_\text{总}$ 表示总功, $W_\text{有用}$ 表示有用功,
$\eta$\footnotemark 表示机械效率,那么
$$ \eta = \dfrac{W_\text{有用}}{W_\text{总}} \;\juhao $$
\footnotetext{$\eta$:希腊字母,汉语拼音读法是 yita 。}

\CJKunderwave{有用功总小于总功,所以机械效率总小于 1 。机械效率通常用百分数来表示}。
例如总功是 500 焦耳,有用功是 400 焦耳,机械效率就是
$\dfrac{400\jiaoer}{500\jiaoer} = 0.8 = 80\%$。
起重机的机械效率是 40 ~ $50\%$ ,
滑轮组的机械效率是 50 ~ $70\%$ ,
抽水机的机械效率是60 ~ $80\%$ 。

\liti 用一个动滑轮把重 400 牛顿的货物提高 2 米,所用的力是 250 牛顿,
求总功、有用功和这个动滑轮的机械效率。

解这道题的关键是要注意用动滑轮提起货物时,人把绳子末端拉起的距离是货物被提起的高度的两倍。
从题目知道人用的力 $F = 250$ 牛顿,货物被提起的高度 $h = 2$ 米,
所以人把绳子末端拉起的距离是 $2h = 4\mi$。

解: $W_\text{总} = F \times 2h = 250 \niudun \times 4 \mi = 1000 \jiaoer$。

  因为货物重 $G = 400$ 牛顿,所以

  $W_\text{有用} = G \times h = 400 \niudun \times 2 \mi = 800 \jiaoer$。

  $\eta = \dfrac{W_\text{有用}}{W_\text{总}} = \dfrac{800 \jiaoer}{1000 \jiaoer} = 0.8 = 80\%$。

答:总功是 1000 焦耳;有用功是 800 焦耳;这个动滑轮的机械效率是 $80\%$ 。


在使用机械来做功的时候,我们总是希望有用功在总功里占的百分比大一些,也就是机械效率高一些。
因此要尽量采取措施来减小额外功。用滚动代替滑动,使用润滑剂来减小摩擦,
改革机械结构以减轻它受到的重力,都是减小额外功,提高机械效率的有效办法。
在今后实际工作中遇到提高机械效率的问题时,希望同学们能够运用学到的知识来研究解决。

