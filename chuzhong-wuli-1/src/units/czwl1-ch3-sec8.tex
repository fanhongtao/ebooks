\section{运动和力}\label{sec:3-8}

牛顿第一运动定律告诉我们,物体在没有受到外力作用的时候,总保持匀速直线运动状态或静止状态。
那么,怎样才能改变物体的运动状态呢?

放在桌上的小车,要使它从静止变为运动,必须用力拉它。要使运动的小车静止,必须用力阻碍它。

小车运动起来以后,要使它的速度增大,必须继续用力拉它,要使它的速度减小,必须用跟运动方向相反的力阻碍它。

要使运动的小车向左转弯,必须对它加一个向左的力,要使它向右转弯,必须对它加一个向右的力。

小车从静止变为运动或从运动变为静止,小车速度的大小发生改变,小车的运动方向发生改变,
都叫做小车运动状态的改变,可见,要改变小车的运动状态,就必须对小车加力。

不仅小车,任何物体都是这样。\CJKunderwave{要改变物体的运动状态,就必须对它加力}。
列车从静止变为运动,并且速度逐渐增大,是由于受到机车的牵引力。
关闭了发动机的汽车,速度逐渐减小,最后停下来,是由于受到地面和空气的阻力。
射出去的炮弹,不断地改变运动方向而做曲线运动,是由于受到地球的引力。

因此,\textbf{力是改变物体运动状态的原因}。

我们在前面讲了物体由于惯性而做匀速直线运动或者保持静止。
现在又讲了只要物体的运动状态发生了改变,它一定是受到了力。
所以,\CJKunderwave{力的作用不是使物体运动,而是使物体的运动状态发生改变}。

