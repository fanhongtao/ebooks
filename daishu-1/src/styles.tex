% 定义本书中使用到的格式

\usepackage{amsmath}
\usepackage{amssymb}
\usepackage{caption}
\captionsetup[figure]{labelsep=none}

\usepackage{enumitem}
\usepackage{geometry}
\geometry{a4paper,left=2cm,right=2cm,top=2cm,bottom=1cm}
\usepackage{wrapfig}

% 绘制带圈的数字
\usepackage{tikz}
\usepackage{etoolbox}
\newcommand{\circled}[2][]{\tikz[baseline=(char.base)]
    {\node[shape = circle, draw, inner sep = 1pt]
    (char) {\phantom{\ifblank{#1}{#2}{#1}}};%
    \node at (char.center) {\makebox[0pt][c]{#2}};}}
\robustify{\circled}

\newcommand{\lianxi}{\textbf{练\quad 习}}
\newcommand{\xhx}[1][2em]{\underline{\hspace{#1}}}

\setcounter{secnumdepth}{3}
\renewcommand{\thesection}{\chinese{section}}
\renewcommand{\thesubsection}{\arabic{section}.\arabic{subsection}}
\renewcommand{\thesubsubsection}{\arabic{subsubsection}.}

\renewcommand{\thefigure}{\thechapter-\arabic{figure}}
\renewcommand{\thefootnote}{\circled{\arabic{footnote}}}

\usepackage{hyperref}
\hypersetup{colorlinks=true, linkcolor=black}

\newcommand{\kongji}{\varnothing}
\newcounter{cntliti}[subsection]           % 例题的计数器
\newcommand{\liti}{ % 例题的标题
  \stepcounter{cntliti}
  \textbf{例\thecntliti}
}

\newcommand{\jie}{\textbf{解: }}

% 修改数学公式与上下文的距离
\makeatletter
\renewcommand\normalsize{%
    \abovedisplayskip 1\p@ \@plus1\p@ \@minus6\p@
    \belowdisplayskip \abovedisplayskip
}
\makeatother
