% 定义本书中使用到的格式

\usepackage{array}
\usepackage{amsmath}
\usepackage{amssymb}
\usepackage{caption}
\captionsetup[figure]{labelsep=none}

\usepackage{enumitem}
\usepackage{etoolbox}
\usepackage{float}
\usepackage{geometry}
\geometry{a4paper,left=2cm,right=2cm,top=2cm,bottom=1cm}
\usepackage{wrapfig}
\usepackage{tikz}
\usetikzlibrary{
  arrows.meta,
  calc,
  intersections,
  patterns,
  patterns.meta,
}

% 绘制带圈的数字
\newcommand{\circled}[2][]{\tikz[baseline=(char.base)]
    {\node[shape = circle, draw, inner sep = 1pt]
    (char) {\phantom{\ifblank{#1}{#2}{#1}}};%
    \node at (char.center) {\makebox[0pt][c]{#2}};}}
\robustify{\circled}

\newcommand{\gongshishangyi}{\setlength\abovedisplayskip{-1.5em}} % 当 “解:” 后直接写公式时,为了“解”与公式水平对齐所需要的偏移。
\newcommand{\lianxi}{\textbf{练\quad 习}}
\newcommand{\xhx}[1][2em]{\underline{\hspace{#1}}}

\counterwithout*{subsection}{section}
\counterwithin*{subsection}{chapter}
\setcounter{secnumdepth}{3}
\renewcommand{\thesection}{\chinese{section}}
\renewcommand{\thesubsection}{\arabic{chapter}.\arabic{subsection}}
\renewcommand{\thesubsubsection}{\arabic{subsubsection}.}

\renewcommand{\thefigure}{\thechapter-\arabic{figure}}
\renewcommand{\thefootnote}{\circled{\arabic{footnote}}}

\usepackage{hyperref}
\hypersetup{colorlinks=true, linkcolor=black}

\newcounter{cntliti}[subsection]           % 例题的计数器
\newcommand{\liti}{ % 例题的标题
  \stepcounter{cntliti}
  \textbf{例\thecntliti}
}

\newcommand{\jie}{\textbf{解: }}

\newcounter{cntxiti}                       % “习题”的计数器
\newcommand{\xiti}{%
  \stepcounter{cntxiti}
  \setcounter{cntxiaoti}{0}
  \vspace{1em}
  \begin{center}
    \subsection*{\labelxiti}
  \end{center}
  \addcontentsline{toc}{subsection}{\labelxiti}
}
\newcommand{\labelxiti}{习题 \chinese{cntxiti}}

\newcounter{cntxiaoti}[subsubsection]      % 小题的计数器
\newcounter{cntxiaoxiaoti}[cntxiaoti]      % 小小题的计数器

\newenvironment{xiaotis}{ % “小题” 环境
  \newcommand{\xiaoti}[1] { % 小题的标题
    \stepcounter{cntxiaoti}
    \hangafter 1\setlength{\hangindent}{3.2em}{\labelxiaoti ##1}
  }
  \newcommand{\labelxiaoti}{\arabic{cntxiaoti}. }  % 1. 2. 3. ……
}{%
}

\newenvironment{xiaoxiaotis}{ % “小小题” 环境
  \newcommand{\xiaoxiaoti}[1] { % 小小题的标题
    \stepcounter{cntxiaoxiaoti}
    \hangafter 1\setlength{\hangindent}{5.4em}{\hspace{1em}\labelxiaoxiaoti##1}
  }
  \newcommand{\labelxiaoxiaoti}{(\arabic{cntxiaoxiaoti})} % (1) (2) (3) ……
}{%
}

\newcommand{\twoInLine}  [3][10em] {\begin{tabular}[t]{*{2}{@{}p{#1}}} #2 & #3\end{tabular}}
\newcommand{\threeInLine}[4][10em] {\begin{tabular}[t]{*{3}{@{}p{#1}}} #2 & #3 & #4\end{tabular}}
\newcommand{\fourInLine} [5][10em] {\begin{tabular}[t]{*{4}{@{}p{#1}}} #2 & #3 & #4 & #5\end{tabular}}

% 修改数学公式与上下文的距离
\makeatletter
\renewcommand\normalsize{%
    \abovedisplayskip 1\p@ \@plus1\p@ \@minus6\p@
    \belowdisplayskip \abovedisplayskip
}
\makeatother


%----------------------------------
% (重)定义一些数学符号
%  一是为了使用/记忆上的方便,二是如果以后有变动,只需要修改一处。
\newcommand{\kongji}{\varnothing}      % 空集
\newcommand{\buji}{\overline}          % 补集

%----------------------------------
% 图形相关的项
% 需要 TexLive 2020 版本

% 自定义填充格式,用于实现绘制阴影
\tikzdeclarepattern{
    name=lines,
    parameters={
        \pgfkeysvalueof{/pgf/pattern keys/size},
        \pgfkeysvalueof{/pgf/pattern keys/angle},
        \pgfkeysvalueof{/pgf/pattern keys/line width},
    },
    bounding box={(-.1pt,-.1pt) and
        (\pgfkeysvalueof{/pgf/pattern keys/size}+.1pt,
        \pgfkeysvalueof{/pgf/pattern keys/size}+.1pt)},
    tile size={(\pgfkeysvalueof{/pgf/pattern keys/size},
                \pgfkeysvalueof{/pgf/pattern keys/size})},
    tile transformation={rotate=\pgfkeysvalueof{/pgf/pattern keys/angle}},
    defaults={
        size/.initial=7pt,
        angle/.initial=0,
        line width/.initial=.4pt,
    },
    code={
        \draw[line width=\pgfkeysvalueof{/pgf/pattern keys/line width}]
            (0,0) -- (\pgfkeysvalueof{/pgf/pattern keys/size},
                      \pgfkeysvalueof{/pgf/pattern keys/size});
    }
}

%----------------------------------
\newlength{\defaultParIndent}  % 页面缺省的 \parindent 长度。
\setlength{\defaultParIndent}{\parindent}
