\xiti

\begin{xiaotis}

\xiaoti{下列各表分别表示从集合 $A$(元素 $a$)到集合 $B$(元素 $b$)的一个映射,
    判断这些映射是不是 $A$ 到 $B$ 上的一一映射:}
\begin{xiaoxiaotis}

    \vspace{0.5em}
    \xiaoxiaoti{
        \begin{tabular}{|*{5}{w{c}{2em}|}}
            \hline
            $a$ & 1 & 2 & 3 & 4 \\
            \hline
            $b$ & -1 & -1 & -1 & -1 \\
            \hline
        \end{tabular}
    }

    \vspace{0.5em}
    \xiaoxiaoti{
        \begin{tabular}{|*{5}{w{c}{2em}|}}
            \hline
            $a$ & 1 & 2 & 3 & 4 \\
            \hline
            $b$ & 3 & 6 & 9 & 12 \\
            \hline
        \end{tabular}
    }

    \vspace{0.5em}
    \xiaoxiaoti{
        \begin{tabular}{|*{5}{w{c}{2em}|}}
            \hline
            $a$ & 3 & 4 & 5 & 6 \\
            \hline
            $b$ & 2 & 3 & 2 & 4 \\
            \hline
        \end{tabular}
    }

    \vspace{0.5em}
    \xiaoxiaoti{
        \begin{tabular}{|*{6}{w{c}{2em}|}}
            \cline{1-5}
            $a$ & 3 & 4 & 5 & 6 \\
            \hline
            $b$ & 2 & 3 & 4 & 5 & 6 \\
            \hline
        \end{tabular}
    }
    \vspace{0.5em}

\end{xiaoxiaotis}

\xiaoti{设 $X = \{ \dots , -3, -2, -1, 0,1,2,3, \dots\}$,$Y = \{0,1,2,3,\dots\}$,对应法则是
    把 $X$ 中的元素“取绝对值”。这个对应是不是从 $X$ 到 $Y$ 的映射?是不是从 $X$ 到 $Y$ 上的一一映射?}

\xiaoti{求下列一一映射 $f: A \to B$ 的逆映射:}
\begin{xiaoxiaotis}

    \xiaoxiaoti{$A = \{x | x \geqslant 1\}$,$B = \{y | y \geqslant 0\}$,一一映射 $f: A \to B$
        使 $B$ 中的元素 $y = \sqrt{x - 1}$ 和 $A$ 中的元素 $x$ 对应;}

    \xiaoxiaoti{$A = \{x | x \neq 0\}$,$B = \{y | y \neq 1\}$,一一映射 $f: A \to B$
        使 $B$ 中的元素 $y = 1 - \dfrac 1 x$ 和 $A$ 中的元素 $x$ 对应。}

\end{xiaoxiaotis}

\xiaoti{下列各映射有没有逆映射?如果有,写出逆映射;如果没有,说明为什么。}
\begin{xiaoxiaotis}

    \xiaoxiaoti{$A = \{x | x \in Q\}$,$B = \{y | y \in R\}$,映射 $f: A \to B$
        使 $B$ 中的元素 $y = 2x$ 和 $A$ 中的元素 $x$ 对应;}

    \xiaoxiaoti{$A = \{x | x \in Z\}$,$B = \{y | y \in Z\}$,映射 $f: A \to B$
        使 $B$ 中的元素 $y = 3x$ 和 $A$ 中的元素 $x$ 对应;}

    \xiaoxiaoti{映射 $f: R \to R$ 使象集合中的元素 $y = x^3$ 和原象集合中的元素 $x$ 对应;}

    \xiaoxiaoti{设 $A = \{\alpha | 0^\circ \leqslant \alpha \leqslant 180^\circ \}$,$B = [0,1]$,映射 $f: A \to B$
        使 $B$ 中的元素 $y = \sin\alpha$ 和 $A$ 中的元素 $\alpha$ 对应。}

\end{xiaoxiaotis}

\xiaoti{$x$ 取什么值,函数 $y = \dfrac 1 {1 + x^2} \; (x \in \buji{R^-})$ 的值等于下列各数?}

\begin{xiaoxiaotis}
    \vspace{0.5em}
    \fourInLine[6em]{\xiaoxiaoti{$\dfrac 1 2$;}}{\xiaoxiaoti{0.1;}}{\xiaoxiaoti{1;}}{\xiaoxiaoti{$\dfrac 1 {17}$。}}
    \vspace{0.5em}


\end{xiaoxiaotis}

\xiaoti{下列函数中哪些互为反函数?\\
    \begin{tabular}{*{3}{p{8em}}}
        $y=x^3$, & $y=5+x$, & $y=2x$, \\
        $y=-4x$, & $y=\sqrt[3]{x}$, & $y=x-5$, \\
        $y=\dfrac x 2$, & $y=-\dfrac 1 4 x$。
    \end{tabular}
}
\vspace{0.5em}

\xiaoti{求下列函数的反函数:}

\begin{xiaoxiaotis}
    \renewcommand\arraystretch{1.5}
    \begin{tabular}[t]{*{2}{@{}p{16em}}} 
        \xiaoxiaoti {$y = -\dfrac 1 x + 3 \; (x \neq 0)$;} & \xiaoxiaoti {$y = x^5 + 1 \; (x \in R)$;} \\
        \xiaoxiaoti {$y = \sqrt{x + 5} \; (x \geqslant -5)$;} & \xiaoxiaoti {$y = \sqrt{2x - 4} \; (x \geqslant 2)$;} \\
        \xiaoxiaoti {$y = x^{\frac 3 5} - 2 \; (x \in R)$;} & \xiaoxiaoti {$y = \dfrac {2x} {5x + 1} \; (x \neq - \dfrac 1 5)$。}
    \end{tabular}
\end{xiaoxiaotis}

\xiaoti{已知函数 $y = \sqrt{25 - 4x^2}$。}

\begin{xiaoxiaotis}
    \vspace{0.5em}
    \xiaoxiaoti{当 $x \in \left[ -\dfrac 5 2, \dfrac 5 2 \right]$ \vspace{0.5em} 时,这个函数是否有反函数?如果有反函数,将它写出来,并指出反函数的定义域。}

    \vspace{0.5em}
    \xiaoxiaoti{就 $x \in \left[ 0, \dfrac 5 2 \right]$ 的情况重新回答第(1)题中的问题。}
    \vspace{0.5em}

\end{xiaoxiaotis}

\xiaoti{求下列函数的反函数,并写出原来的函数及其反函数的定义域:}
\begin{xiaoxiaotis}

    \twoInLine[16em]{\xiaoxiaoti{$y = \dfrac 1 {x - 1}$;}}{\xiaoxiaoti{$y = x^3 + 1$。}}
    \vspace{0.5em}

\end{xiaoxiaotis}

\xiaoti{求下列函数的值域:}

\begin{xiaoxiaotis}
    \renewcommand\arraystretch{1.5}
    \begin{tabular}[t]{*{2}{@{}p{16em}}}
        \xiaoxiaoti {$y = \dfrac 7 {x + 2} \; (x \neq -2)$;} & \xiaoxiaoti {$y = \dfrac x {x + 1} \; (x \neq -1)$;} \\
        \xiaoxiaoti {$y = \sqrt{16 - x^2} \; (0 \leqslant x \leqslant 4)$;} & \xiaoxiaoti {$y = \sqrt{x^2 - 49} \; (x \leqslant -7)$。}
    \end{tabular}
\end{xiaoxiaotis}

\xiaoti{已知函数 $y = 2|x|$。}

\begin{xiaoxiaotis}
    \xiaoxiaoti{当 $x \in [0, +\infty)$ 时,这个函数是否有反函数?如果有反函数,将它写出来,并指出反函数的定义域。 $x \in (-\infty, +\infty)$ 时呢?}

    \xiaoxiaoti{如果有反函数,在同一坐标系内画出函数 $y = 2|x|$ 及其反函数的图象。}

\end{xiaoxiaotis}

\vspace{0.5em}
\xiaoti{求证函数 $y = \dfrac {1 - x} {1 + x} \; (x \neq -1)$ \vspace{0.5em} 的反函数就是它本身。然后说明这个函数的
    图象关于直线 $y = x$ 具有什么特点,并利用这特点画出函数的简图(用描点法)。}

\end{xiaotis}
