\subsection{逆映射}\label{subsec:1-9}

先看图 \ref{fig:1-24} 所示的映射。

容易看出,这两个映射分别是集合 $A$ 到集合 $B$ 上、集合 $B$ 到集合 $A$ 上的一一映射。
在映射 $f: A \to B$ 作用下的象及原象,分别是在映射 $g: B \to A$ 作用下的原象及象。

\begin{figure}[htbp]
    \centering
    \begin{minipage}{8cm}
    \centering
    \begin{tikzpicture}[>=Stealth]
    \draw (0,0) circle [x radius=1cm, y radius=3cm];
    \node at (-0.6,0) {$A$};
    \node at (0,2) {$1$};
    \node at (0,1) {$2$};
    \node at (0,0) {$3$};
    \node at (0,-1) {$4$};
    \node at (0,-2) {$5$};

    \draw (3,0) circle [x radius=1cm, y radius=3cm];
    \node at (3.6,0) {$B$};
    \node at (3,2) {$2$};
    \node at (3,1) {$4$};
    \node at (3,0) {$6$};
    \node at (3,-1) {$8$};
    \node at (3,-2) {$10$};

    \draw [->] (0.2,2) .. controls(1.0,2.5) and (2.0,2.5) .. (2.8,2);
    \draw [->] (0.2,1) .. controls(1.0,1.5) and (2.0,1.5) .. (2.8,1);
    \draw [->] (0.2,0) .. controls(1.0,0.5) and (2.0,0.5) .. (2.8,0);
    \draw [->] (0.2,-1) .. controls(1.0,-0.5) and (2.0,-0.5) .. (2.8,-1);
    \draw [->] (0.2,-2) .. controls(1.0,-1.5) and (2.0,-1.5) .. (2.8,-2);

    \node at (1.5,3) {乘$2$};
    \node at (1.5,-3) {$f: A \to B$};
\end{tikzpicture}

    \end{minipage}
    \qquad
    \begin{minipage}{8cm}
    \centering
    \begin{tikzpicture}[>=Stealth]
    \draw (0,0) circle [x radius=1cm, y radius=3cm];
    \node at (-0.6,0) {$B$};
    \node at (0,2) {$2$};
    \node at (0,1) {$4$};
    \node at (0,0) {$6$};
    \node at (0,-1) {$8$};
    \node at (0,-2) {$10$};

    \draw (3,0) circle [x radius=1cm, y radius=3cm];
    \node at (3.6,0) {$A$};
    \node at (3,2) {$1$};
    \node at (3,1) {$2$};
    \node at (3,0) {$3$};
    \node at (3,-1) {$4$};
    \node at (3,-2) {$5$};

    \draw [->] (0.2,2) .. controls(1.0,1.5) and (2.0,1.5) .. (2.8,2);
    \draw [->] (0.2,1) .. controls(1.0,0.5) and (2.0,0.5) .. (2.8,1);
    \draw [->] (0.2,0) .. controls(1.0,-0.5) and (2.0,-0.5) .. (2.8,0);
    \draw [->] (0.2,-1) .. controls(1.0,-1.5) and (2.0,-1.5) .. (2.8,-1);
    \draw [->] (0.2,-2) .. controls(1.0,-2.5) and (2.0,-2.5) .. (2.8,-2);

    %\node at (1.5,3) {乘$\frac 1 2$};
    \draw (1.5,2.8) node {乘$-$} (1.75,3.0) node {$1$} (1.75,2.5) node {$2$};
    \node at (1.5,-3) {$g: B \to A$};
\end{tikzpicture}

    \end{minipage}
    \caption{}\label{fig:1-24}
\end{figure}

一般地,设 $f: A \to B$ 是集合 $A$ 到集合 $B$ 上的一一映射,如果对于 $B$ 中的
每一个元素 $b$,使 $b$ 在 $A$ 中的原象 $a$ 和它对应,这样所得的映射叫做
映射 $f: A \to B$ 的\textbf{逆映射},记作
$$f^{-1}: B \to A \text{。}$$

从逆映射的定义可以知道,映射 $f: A \to B$ 也是映射 $f^{-1}: B \to A$ 的
逆映射,而且 $f^{-1}: B \to A$ 也是一一映射(从 $B$ 到 $A$ 上的一一映射)。

这样,图 \ref{fig:1-24} 中的映射 $g: B \to A$ 就是 $f: A \to B$ 的逆映射。

现在我们来求第 \ref{subsec:1-8} 节例子中的一一映射的逆映射。

在 \hyperref[li:1-8-1]{(1)} 中的一一映射 $f: X \to Y$ 是使 $Y$ 中的元素 $y = 2x + 1$ 和 $X$ 中的元素 $x$ 对应。
由 $y = 2x + 1$,得 $x = \dfrac{y -1}{2}$。于是逆映射 $f^{-1}: Y \to X$ 就使 $X$ 中的元素
$x = \dfrac{y-1}{2}$ 和 $Y$ 中的元素 $y$对应。

在 \hyperref[li:1-8-4]{(4)} 中的一一映射 $f': \buji{R^-} \to \buji{R^-}$ 是使象集合 $\buji{R^-}$ 中的元素 $y=x^2$
和原象集合 $\buji{R^-}$ 中的元素 $x$ 对应。由 $y = x^2$,得 $x = \sqrt{y}$($x = -\sqrt{y} \notin \buji{R^-}$,舍去)。
于是逆映射 $(f')^{-1}: \buji{R^-} \to \buji{R^-}$ 就使 $f'$ 的原象集合 $\buji{R^-}$ 中的元素
$x = \sqrt{y}$ 和 $f'$ 的象集合中的元素 $y$ 对应。

应该注意:只有对于一一映射,我们才研究它的逆映射。

\lianxi

\begin{xiaotis}
\xiaoti{求下列一一映射 $f: A \to B$ 的逆映射:}

\begin{xiaoxiaotis}
    \xiaoxiaoti{$A = \{1,2,3,4,5,\dots\}$,$B = \{1, \dfrac 1 2, \dfrac 1 3, \dfrac 1 4, \dfrac 1 5, \dots \}$,
    一一映射 $f: A \to B$ 使 $B$ 中的元素 $y = \dfrac 1 x$ 和 $A$ 中的元素 $x$ 对应;}

    \xiaoxiaoti{$A = \{1,2,3,4,5,\dots\}$,$B = \{2,5,10,17,26,\dots\}$,
    一一映射 $f: A \to B$ 使 $B$ 中的元素 $y = x^2 + 1$ 和 $A$ 中的元素 $x$ 对应;}

    \xiaoxiaoti{$A = \{1,2,3,4,5,\dots\}$,$B = \{2, \dfrac 3 2, \dfrac 4 3, \dfrac 5 4, \dfrac 6 5, \dots \}$,
    一一映射 $f: A \to B$ 使 $B$ 中的元素 $y = \dfrac {x + 1} x$ 和 $A$ 中的元素 $x$ 对应。}
    \vspace{0.5em}

\end{xiaoxiaotis}

\xiaoti{为什么第 1.8 节例 (2),(3)中的映射没有逆映射?}

\xiaoti{举出集合 $A$ 到集合 $B$ 上的一一映射的例子,并求出它的逆映射。}

\end{xiaotis}
