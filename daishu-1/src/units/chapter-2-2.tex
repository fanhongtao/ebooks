\subsection{弧度制}\label{subsec:2-2}

我们在平面几何里研究过角的度量,规定周角的 $\dfrac{1}{360}$ \vspace{0.5em} 为 $1$ 度的角。这种用度做单位来度量角的
制度叫做\textbf{角度制}.下面,再介绍在数学和其他许多科学研究中还要经常用到的另一种度量角的制度——弧度制。

我们把等于半径长的圆弧所对的圆心角叫做 \textbf{$1$ 弧度的角}。如图 \ref{fig:2-6},$\hudu{AB}$ 的长等于半径 $r$,
$\hudu{AB}$ 所对的圆心角$\angle AOB$ 就是 $1$ 弧度的角。在图 \ref{fig:2-7} 中,圆心角 $\angle AOC$ 所对的
$\hudu{AC}$ 的长 $l = 2r$,那么 $\angle AOC$ 的弧度数就是
$$\dfrac l r = \dfrac{2r}{r} = 2 \text{。}$$

\begin{figure}[htbp]
    \centering
    \begin{minipage}{8cm}
    \centering
    \begin{tikzpicture}[>=Stealth]
    \draw (0, 0) circle(2);
    \node at (-0.3,-0.3) {$O$};

    \draw (0, 0) -- (2, 0) node[anchor=west] {$A$};
    \draw [rotate=57.3] (0, 0) -- (2, 0) node[anchor=south] {$B$};

    \draw [->] (0.5, 0) arc (0:57.3:0.5);
    \node at (1.2, 0.4) {$1$弧度};

    \node at (1.2, -0.3) {$r$};
    \node at (2.1, 1.0) {$r$};
\end{tikzpicture}

    \caption{}\label{fig:2-6}
    \end{minipage}
    \qquad
    \begin{minipage}{8cm}
    \centering
    \begin{tikzpicture}[>=Stealth]
    \draw (0, 0) circle(2);
    \node at (-0.3,-0.3) {$O$};

    \draw (0, 0) -- (2, 0) node[anchor=west] {$A$};
    \draw [rotate=114.6] (0, 0) -- (2, 0) node[anchor=south] {$C$};

    \draw [->] (0.5, 0) arc (0:114.6:0.5);

    \node at (1.2, -0.3) {$r$};
    \node at (2.1, 1.5) {$l = 2r$};
\end{tikzpicture}

    \caption{}\label{fig:2-7}
    \end{minipage}
\end{figure}

如果圆心角所对的弧的长 $l = 2\pi r$(即弧是一个整圆),那么这个圆心角的弧度数是
$$\dfrac l r = \dfrac{2\pi r}{r} = 2\pi \text{。}$$

\vspace{0.5em}
如果圆心角表示一个负角,且它所对的弧的长 $l = 4\pi r$,那么这个角的弧度数的绝对值是
\vspace{0.5em}
$$\dfrac l r = \dfrac{4\pi r}{r} = 4\pi \text{,}$$
即这个角的弧度数是 $-4\pi$。

一般地,我们规定:\textbf{正角的弧度数为正數,负角的弧度数为负数,零角的弧度数为零,
任一已知角 $\alpha$ 的弧度数的绝对值
$$|\alpha| = \dfrac l r \text{,}$$
其中 $l$ 为以角 $\alpha$ 作为圆心角时所对圆弧的长,$r$ 为圆的半径。}
这种用“弧度”做单位来度量角的制度叫做\textbf{弧度制}。

根据上面的公式 $|\alpha| = \dfrac l r$,可以得到
$$l = |\alpha| r \text{,}$$
这就是说,圆弧的长等于圆弧所对圆心角的弧度数的绝对值与半径的积。
这个圆弧长公式比采用角度制时的相应公式 $\left( l = \dfrac{n \pi r}{180} \right)$
要简单一些。以后还要遇到一些公式,用弧度制表示比用角度制表示简便得多。

\liti 利用弧度制来推导扇形面积公式 $S = \dfrac 1 2 l R$,
其中 $l$ 是扇形的弧长,$R$ 是圆的半径(图\ref{fig:2-8})。

\begin{figure}[htbp]
    \centering
    \begin{tikzpicture}[>=Stealth]
    \draw (0, 0) circle(2);
    \node at (-0.3, 0) {$O$};

    \filldraw[pattern={lines[angle=110]}] (0,0) -- (-45:2) arc[start angle=-45, end angle=45, radius=2] -- cycle;

    \node [fill=white, inner sep=1pt] at (1.2, 0) {$S$};
    \node at (2.2, 0) {$l$};
    \node at (0.5, 0.9) {$R$};
\end{tikzpicture}

    \caption{}\label{fig:2-8}
\end{figure}

\jie 因为圆心角为 $1$ 弧度的扇形的面积为 $\dfrac{1}{2\pi} \cdot \pi R^2$,
而弧长为 $l$ 的扇形的圆心角的弧度数为 $\dfrac l R$,所以它的面积为
$$S = \dfrac l R \cdot \dfrac{1}{2\pi} \cdot \pi R^2 = \dfrac 1 2 l R \text{。}$$

对于同一个角,当分别用弧度为单位和用度为单位来度量时,所得的量数除零角以外都是不同的。
下面介绍它们之间的换算关系。

上面指出,周角的弧度数是 $2\pi$,而在角度制里它是 $360^\circ$,因此

\begin{center}
    \framebox{\begin{minipage}{15em}
        \begin{gather*}
            360^\circ = 2\pi \text{弧度,} \\
            180^\circ = \pi \text{弧度。}
        \end{gather*}
    \end{minipage}}
\end{center}

由此还可得到:

\begin{gather*}
    1^\circ = \dfrac{\pi}{180} \text{弧度} \approx 0.01745 \text{弧度;} \\
    1 \text{弧度} = \left( \dfrac{180}{\pi} \right)^\circ \approx 57.30^\circ = 57^\circ 18' \text{。}
\end{gather*}

\liti 把 $67^\circ 30'$ 化成弧度。

\vspace{0.5em}
\jie $\because \quad 67^\circ 30' = \left( 67 \dfrac 1 2 \right)^\circ ,$
\vspace{0.5em}

$\therefore \quad  67^\circ 30' = \dfrac{\pi}{180} \text{弧度} \times 67 \dfrac 1 2 = \dfrac 3 8 \pi \text{弧度。}$

\vspace{0.5em}
\liti 把 $\dfrac 3 5 \pi$ 弧度化成度。
\vspace{0.5em}

\jie $\dfrac 3 5 \pi \text{弧度} = \dfrac 3 5 \times 180^\circ = 108^\circ$。
\vspace{0.5em}

下面是一些特殊角的度数与弧度数的对应表:

\begin{table}[H]
\renewcommand\arraystretch{2}
\begin{tabular}{|w{c}{5em}|*{8}{w{c}{2em}|}}
    \hline
    度 & $0^\circ$ & $30^\circ$ & $45^\circ$ & $60^\circ$ & $90^\circ$ & $180^\circ$ & $270^\circ$ & $360^\circ$ \\
    \hline
    弧度 & $0$ & $\dfrac \pi 6$ & $\dfrac \pi 4$ & $\dfrac \pi 3$ & $\dfrac \pi 2$ & $\pi$ & $\dfrac{3\pi}{2}$ & $2\pi$ \\
    \hline
\end{tabular}
\end{table}

\begin{wrapfigure}[8]{r}{4.5cm}
    \centering
    \begin{tikzpicture}[>=Stealth]
    \draw (0,0) circle [x radius=1cm, y radius=1.6cm];
    \node at (0, 1.1) {正角};
    \node at (0, 0) {零角};
    \node at (0, -1) {负角};
    \node at (0, -1.9) {任意角的集合};

    \draw (3,0) circle [x radius=1cm, y radius=1.6cm];
    \node at (3, 1.1) {正实数};
    \node at (3, 0) {零};
    \node at (3, -1) {负实数};
    \node at (3, -1.9) {实数集 $R$};

    \draw [->] (0.5,1.1) -- (2.4,1.1);
    \draw [<-] (0.5,0.9) -- (2.4,0.9);
    \draw [->] (0.5,0.1) -- (2.4,0.1);
    \draw [<-] (0.5,-0.1) -- (2.4,-0.1);
    \draw [->] (0.5,-0.9) -- (2.4,-0.9);
    \draw [<-] (0.5,-1.1) -- (2.4,-1.1);
\end{tikzpicture}

    \vspace{-20pt}
    \caption{}\label{fig:2-9}
\end{wrapfigure}

度数与弧度数的换算,还可以利用《中学数学用表》中的《度、分、秒化弧度表》、《弧度化度、分、秒表》来进行,用法见表中说明。

用弧度制来度量角,实际上是在角的集合与实数集 $R$ 之间建立了这样的一一对应关系:
每一个角都有一个实数(即这个角的弧度数)与它对应,不同的角有不同的实数与它对应;
反过来,每一个实数也都有一个角(角的弧度数等于这个实数)与它对应,如图 \ref{fig:2-9} 所示。

今后我们用弧度制表示角的时候,“弧度”二字通常略去不写,而只写这个角所对应的弧度数。
例如,角 $\alpha = 2$ 就表示 $\alpha$ 是 $2$ 弧度的角,$\sin \dfrac \pi 3$
就表示 $\dfrac \pi 3$ 弧度的角的正弦。

\liti 计算:
\begin{xiaoxiaotis}

    \vspace{0.5em}
    \twoInLine[8em]{\xiaoxiaoti{$\sin \dfrac{3\pi}{4}$;}}{\xiaoxiaoti{$\tan 1.5$。}}\footnote{录注:甲种本原书中使用的 tg 和 ctg 表示正切、余切,录入时统一替换成 tan 和 cot 。}

\end{xiaoxiaotis}

\jie (1)$\because \quad \dfrac 3 4 \pi \text{弧度} = 135^\circ$,

$\therefore \quad \sin \dfrac{3\pi}{4} = \sin 135^\circ = \sin 45^\circ = \dfrac{\sqrt{2}}{2}$;
\vspace{0.5em}

(2)$\because 1.5 \text{弧度} \approx 57.30^\circ \times 1.5 = 85.95^\circ = 85^\circ 57'$,

$\therefore \quad \tan 1.5 \approx \tan 85^\circ 57' = 14.12$。

\liti 将下列各角化成 $2k\pi + \alpha \, (0 \leqslant \alpha < 2\pi,\, k \in Z)$ 的形式:
\begin{xiaoxiaotis}

    \vspace{0.5em}
    \twoInLine[8em]{\xiaoxiaoti{$\dfrac{19}{3} \pi$;}}{\xiaoxiaoti{$-315^\circ$。}}

\end{xiaoxiaotis}

\jie (1)$\dfrac{19}{3} \pi = 6\pi + \dfrac{\pi}{3} \, \left( \alpha = \dfrac{\pi}{3},\, k = 3 \right)$;
\vspace{0.5em}

(2)$-315^\circ = -\left( \dfrac{\pi}{180} \times 315 \right) = -\dfrac 7 4 \pi = -2\pi + \dfrac \pi 4 ,\, \left( \alpha = \dfrac{\pi}{4},\, k = -1 \right)$。
\vspace{0.5em}

\liti 如图 \ref{fig:2-10},求公路弯道部分 $\hudu{AB}$ 的长 $l$(精确到 $1$ 米。图中长度单位:米)。

\begin{figure}[htbp]
    \centering
    \begin{tikzpicture}[>=Stealth, rotate=60]
    \draw (0, 0) -- (6, 0);
    \draw [rotate=60] (0, 0) -- (6, 0);
    \draw (4, -1) -- (4, 0) arc (0:60:4) -- +(-1, 0.5);
    \draw [loosely dash dot] (4.5, -1) -- (4.5, 0) arc (0:60:4.5) -- +(-1, 0.5);
    \draw (5, -1) -- (5, 0) arc (0:60:5) -- +(-1, 0.5);

    \draw [<->] (5.5, 0) arc (0:60:5.5);
    \node [fill=white, inner sep = 1pt] at (4.8,2.8) {$60^\circ$};

    \draw [->, rotate=30] (0, 0) -- (4.5, 0);
    \node [rotate=90] at (2.3,1.6) {$R45$};
    \node [fill=white, inner sep = 1pt, rotate=0] at (2.0,4.0) {$A$};
    \node [fill=white, inner sep = 1pt] at (4.5,-0.3) {$B$};
\end{tikzpicture}

    \caption{}\label{fig:2-10}
\end{figure}

\jie $\because \quad 60^\circ = \dfrac \pi 3 \text{弧度}$,
\vspace{0.5em}

$\therefore \quad l = |\alpha| r = \dfrac \pi 3 \times 45 \approx 3.14 \times 15 \approx 47 $(米)。
\vspace{0.5em}

答:弯道部分 $\hudu{AB}$ 的长约为 $47$ 米。

\lianxi
\begin{xiaotis}

\xiaoti{(口答)下列各度分别是多少 $\pi$ 弧度?}
\begin{xiaoxiaotis}

    \begin{tabular}[t]{*{4}{@{}p{8em}}}
        \xiaoxiaoti {$180^\circ$;} & \xiaoxiaoti {$90^\circ$;} & \xiaoxiaoti {$60^\circ$;} & \xiaoxiaoti {$45^\circ$;} \\
        \xiaoxiaoti {$30^\circ$;} & \xiaoxiaoti {$120^\circ$;} & \xiaoxiaoti {$270^\circ$;} & \xiaoxiaoti {$360^\circ$。}
    \end{tabular}

\end{xiaoxiaotis}

\xiaoti{(口答)下列各弧度分别是多少度?}
\begin{xiaoxiaotis}

    \renewcommand\arraystretch{1.5}
    \begin{tabular}[t]{*{4}{@{}p{8em}}}
        \xiaoxiaoti {$\pi$;} & \xiaoxiaoti {$2\pi$;} & \xiaoxiaoti {$\dfrac 1 2 \pi$;} & \xiaoxiaoti {$\dfrac 1 3 \pi$;} \\
        \xiaoxiaoti {$\dfrac{2\pi}{3}$;} & \xiaoxiaoti {$\dfrac \pi 6$;} & \xiaoxiaoti {$\dfrac \pi 4$;} & \xiaoxiaoti {$\dfrac{3\pi}{4}$。}
    \end{tabular}
    \vspace{0.5em}

\end{xiaoxiaotis}

\xiaoti{把下列各度化成弧度(写成多少 $\pi$ 的形式):}
\begin{xiaoxiaotis}

    \begin{tabular}[t]{*{3}{@{}p{8em}}}
        \xiaoxiaoti {$12^\circ$;} & \xiaoxiaoti {$75^\circ$;} & \xiaoxiaoti {$-210^\circ$;} \\
        \xiaoxiaoti {$135^\circ$;} & \xiaoxiaoti {$300^\circ$;} & \xiaoxiaoti {$22^\circ 30'$。}
    \end{tabular}

\end{xiaoxiaotis}

\xiaoti{把下列各弧度化成度:}
\begin{xiaoxiaotis}

    \renewcommand\arraystretch{1.5}
    \begin{tabular}[t]{*{3}{@{}p{8em}}}
        \xiaoxiaoti {$\dfrac{\pi}{12}$;} & \xiaoxiaoti {$-\dfrac 4 3 \pi$;} & \xiaoxiaoti {$\dfrac{3}{10} \pi$;} \\
        \xiaoxiaoti {$-\dfrac{\pi}{5}$;} & \xiaoxiaoti {$-12 \pi$;} & \xiaoxiaoti {$\dfrac 5 6 \pi$。}
    \end{tabular}
    \vspace{0.5em}

\end{xiaoxiaotis}

\xiaoti{用弧度表示:}
\begin{xiaoxiaotis}

    \xiaoxiaoti{终边在 $x$ 轴上的角的集合;}

    \xiaoxiaoti{终边在 $y$ 轴上的角的集合。}

\end{xiaoxiaotis}

\xiaoti{求下列各三角函数的值:}
\begin{xiaoxiaotis}

    \vspace{0.5em}
    \fourInLine[8em]{\xiaoxiaoti{$\sin \dfrac{2\pi}{3}$;}}
        {\xiaoxiaoti{$\tan \dfrac \pi 6$;}}
        {\xiaoxiaoti{$\cos 1.2$;}}
        {\xiaoxiaoti{$\sin 1$。}}
    \vspace{0.5em}

\end{xiaoxiaotis}

\xiaoti{(口答)时间经过 $4$ 小时,时针和分针各转了多少度,等于多少弧度?}

\xiaoti{用度和弧度表示的弧长公式分别计算在半径为 $1$ 米的圆中,$60^\circ$ 的圆心角所对的圆弧的长。}

\xiaoti{把下列各角化成 $2k\pi + \alpha \, (0 \leqslant \alpha < 2\pi,\, k \in Z)$ 的形式,并指出它们是第几象限的角:}
\begin{xiaoxiaotis}

    \vspace{0.5em}
    \twoInLine[8em]{\xiaoxiaoti{$\dfrac{23}{6} \pi$;}}{\xiaoxiaoti{$-1500^\circ$。}}
    \vspace{0.5em}

\end{xiaoxiaotis}

\xiaoti{己知在半径为 $120mm$ 的圆上的一条弧的长是 $144mm$,求这条弧所对的圆心角的弧度数和度数。}

\end{xiaotis}
