\subsection{互为反函数的函数图象间的关系}\label{subsec:1-11}

看下面的例题:

\liti 求函数 $y = 3x - 2 \; (x \in R)$ 的反函数,并且画出原来的函数和它的反函数的图象。

\vspace{0.5em}
\jie 从 $y = 3x - 2$,得 $x = \dfrac {y+2}{3}$,因此,函数 $y = 3x - 2 \; (x \in R)$
的反函数是 $y = \dfrac {x+2}{3} \; (x \in R)$。

函数 $y = 3x - 2 \; (x \in R)$ 和它的反函数 $y = \dfrac {x+2}{3} \; (x \in R)$ 的图象如图 \ref{fig:1-25} 中所示。

\begin{figure}[htbp]
    \centering
    \begin{minipage}{8cm}
    \centering
    \begin{tikzpicture}[>=Stealth]
    \draw [->] (-3.5,0) -- (3.5,0) node[anchor=north] {$x$};
    \draw [->] (0,-3.5) -- (0,3.5) node[anchor=east] {$y$};
    \node at (0.3,-0.3) {$O$};
    \foreach \x in {1} {
        \draw (\x,0.2) -- (\x,0) node[anchor=north] {$\x$};
    }
    \foreach \y in {1} {
        \draw (0.2,\y) -- (0,\y) node[anchor=east] {$\y$};
    }
    
    \draw[domain=-0.5:1.7] plot (\x, {3 * \x - 2}) +(0, 0.3) node {$y = 3x - 2$};
    \draw[domain=-3:3, dash pattern=on 5mm off 2mm] plot (\x, \x) +(0.5, -0.5) node {$y = x$};
    \draw[domain=-3:3] plot (\x, {(\x + 2) / 3}) +(0, -0.5) node {$y = \displaystyle \frac{x + 2} 3$};
\end{tikzpicture}

    \caption{}\label{fig:1-25}
    \end{minipage}
    \qquad
    \begin{minipage}{8cm}
    \centering
    \begin{tikzpicture}[>=Stealth]
    \draw [->] (-3.5,0) -- (3.5,0) node[anchor=north] {$x$};
    \draw [->] (0,-3.5) -- (0,3.5) node[anchor=east] {$y$};
    \node at (0.3,-0.3) {$O$};
    \foreach \x in {-1,1} {
        \draw (\x,0.2) -- (\x,0) node[anchor=north] {$\x$};
    }
    \foreach \y in {-1,1} {
        \draw (0.2,\y) -- (0,\y) node[anchor=east] {$\y$};
    }
    
    \draw[domain=-1.3:1.3,samples=200] plot (\x, {\x^3}) +(0, 0.3) node {$y = x^3$};
    \draw[domain=-2:2, dash pattern=on 5mm off 2mm] plot (\x, \x) +(0.5, -0.3) node {$y = x$};
    \draw[domain=-2:2,,samples=200] plot (\x, {\x^(1/3)}) +(0, -0.5) node {$y = \sqrt[3]{x}$};
\end{tikzpicture}

    \caption{}\label{fig:1-26}
    \end{minipage}
\end{figure}

\liti 求函数 $y = x^3 \; (x \in R)$ 的反函数,并且画出原来的函数和它的反函数的图象。

解:从 $y = x^3$,得 $x = \sqrt[3]{y}$。 因此,函数 $y = x^3 \; (x \in R)$ 的反函数是 $y = \sqrt[3]{x} \; (x \in R)$。

函数  $y = x^3$ 和它的反函数 $y = x^3 \; (x \in R)$ 的图象如图 \ref{fig:1-26} 中所示。

\vspace{0.5em}
从图 \ref{fig:1-25} 可以看出,函数 $y = 3x - 2$ 和它的反函数 $y = \dfrac {x+2}{3} \; (x \in R)$
的图象是以直线 $y = x$ 为对称轴的对称图形(以后简称\textbf{关于直线 $y = x$ 对称};
同样,以原点为对称中心的对称图形也简称 \textbf{关于原点对称})。从图 \ref{fig:1-26} 也可以看出,
函数 $y = x^3 \; (x \in R)$ 和它的反函数 $y = \sqrt[3]{x} \; (x \in R)$ 的图象关于直线 $y = x$ 对称。

现在我们来证明下面的定理:

\begin{theorem}
    函数 $y = f(x)$ 的图象和它的反函数 $y = f^{-1}(x)$ 的图象关于直线 $y = x$ 对称。
\end{theorem}

\zhengming 设 $M(a,b)$ 是 $y = f(x)$ 的图象上的任意一点,那么 $x = a$ 时,$f(x)$ 有唯一的值 $f(a) = b$。
因为 $y = f(x)$ 有反函数 $y = f^{-1}(x)$,所以 $x = b$ 时,$f^{-1}(x)$ 有唯一的值 $f^{-1}(b) = a$,即
点 $M'(b,a)$ 在反函数 $y = f^{-1}(x)$  的图象上。

如果 $a = b$,那么 $M$,$M'$ 是直线 $y = x$ 上的同一个点,因此它们关于直线 $y = x$ 对称。

\begin{figure}[htbp]
    \centering
    \begin{tikzpicture}[>=Stealth]
    \draw [->] (-2.5,0) -- (4.5,0) node[anchor=north] {$x$};
    \draw [->] (0,-2.5) -- (0,3.5) node[anchor=east] {$y$};
    \node at (0.3,-0.3) {$O$};
    
    \coordinate (O) at (0, 0);
    \coordinate [label=180:{$P(c,c)$}] (P) at (-1,-1);
    \coordinate [label=90:{$M(a,b)$}] (M) at (0.7,2);
    \coordinate [label=300:{$M'(b,a)$}] (M') at (2,0.7);
    
    \draw[dash pattern=on 5mm off 2mm] (-1.5, -1.5) -- (2.5, 2.5) +(0.5, -0.3) node {$y = x$};
    \draw (O) -- (M) -- (M') -- cycle;
    \draw (P) -- (M) -- (M') -- cycle;
\end{tikzpicture}

    \caption{}\label{fig:1-27}
\end{figure}

现设 $a \neq b$。如图 \ref{fig:1-27},在直线 $y = x$ 上任取一点 $P(c,c)$,连接 $PM$,$PM'$ 及 $MM'$。
由两点间距离公式,

\begin{align*}
    PM &= \sqrt{(a - c)^2 + (b - c)^2} \text{,} \\
    PM' &= \sqrt{(b - c)^2 + (a - c)^2} \text{,} \\
    \therefore &\quad PM = PM' \text{。}
\end{align*}

由此可知,直线 $y = x$ 上任意一点到两个定点 $M$,$M'$ 的距离相等,因此直线 $y = x$ 是线段 $MM'$
的垂直平分线,从而点 $M$,$M'$ 关于直线 $y = x$ 对称。

因为点 $M$ 是 $y = f(x)$ 的图象上的任意一点,所以 $y = f(x)$ 图象上任意一点关于直线 $y = x$ 的
对称点都在它的反函数 $y = f^{-1}(x)$ 的图象上。由 $f$,$f^{-1}$ 的互逆性可知,函数  $y = f^{-1}(x)$
图象上任意一点关于直线 $y = x$ 的对称点也都在它的反函数 $y = f(x)$ 的图象上。这就是说,$y = f(x)$
和  $y = f^{-1}(x)$ 的图象关于直线 $y = x$ 对称。

\lianxi

\begin{xiaotis}

\xiaoti{}

\begin{xiaoxiaotis}
    \vspace{-1em} %\vspace{-1.7em}
    \begin{minipage}{0.9\textwidth}
    \xiaoxiaoti{在直角坐标系内,画出直线 $y = x$,然后找出下面这些点关于直线 $y = x$
        的对称点,并写出它们的坐标(不必说明理由):\\
        $A(2,3)$,$B(1,0)$,$C(-2,-1)$,$D(0,-1)$。
    }
    \end{minipage}
    \vspace{0.7em}

    \xiaoxiaoti{上面所求得的各对称点的坐标同原来的点的坐标有什么关系?}

    \xiaoxiaoti{求出点 $P(x_0, y_0)$ 关于直线 $y = x$ 的对称点 $Q$ 的坐标。}

\end{xiaoxiaotis}

\xiaoti{设 $a \neq 0$,$b \neq 0$,求证下列各题中的两个点关于直线 $y = x$ 对称:}

\begin{xiaoxiaotis}
    \begin{tabular}[t]{@{}p{14em}@{}p{20em}} 
        \xiaoxiaoti {$M(a,0)$,$M'(0,a)$;} & \xiaoxiaoti {$M(a,a)$,$M'(a,a)$;} \\
        \xiaoxiaoti {$M(a,b)$,$M'(b,a)$。}
    \end{tabular}
\end{xiaoxiaotis}

\xiaoti{求下列函数的反函数,并画出函数及其反函数的图象:}

\begin{xiaoxiaotis}
    \xiaoxiaoti{$y = 4x - \dfrac 1 2 \; (x \in R)$;}

    \vspace{0.5em}
    \xiaoxiaoti{$y = \dfrac 1 {x + 3} \; (x \in R \text{,且} x \neq -3)$。}
    \vspace{0.5em}
\end{xiaoxiaotis}

\xiaoti{画出函数 $y = x^2 \; (x \in [0, +\infty))$ 的图象,再利用对称关系画出它的反函数的图象。}

\end{xiaotis}

