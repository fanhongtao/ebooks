\xiaojie

一、本章主要内容是在引入集合的概念、集合同集合之间的关系以及定义
映射、一一映射、逆映射这些概念的基础上,进一步阐明函数与反函数的概念,
研究函数的单调性与奇偶性,并具体研究幂函数、指数函数和对数函数以及
简单的指数方程和对数方程等。

二、一组对象的全体形成一集合,集合里的各个对象就是这个集合的元素。
对象同集合的关系是属于或不属于。

集合 $A$ 是集合 $B$ 的子集,记作 $A \subseteq B$。如果 $A \subseteq B$
而且 $B \subseteq A$,则 $A = B$。子集、真子集这两个概念是不同的。

集合 $A$,$B$ 的交集,记作 $A \cap B$,它是 $A$ 的子集也是 $B$ 的子集,
特别地,$A \cap A = A$,$A \cap \kongji = \kongji$。

集合 $A$,$B$ 的并集,记作 $A \cup B$,集合 $A$,$B$ 都是 $A \cup B$ 的子集,
特别地,$A \cup A = A$,$A \cup \kongji = A$。

一个集合的补集是相对于给定的全集来说的。如果 $A$ 是全集 $I$ 的子集,则补集记 $\buji{A}$,
而 $\buji{A}$ 也是 $I$ 的子集。$A$ 及 $\buji{A}$ 及 $I$ 的关系是 $A \cup \buji{A} = I$,
$A \cap \buji{A} = \kongji$,$\buji{\buji{A}} = A$。

三、给定两个集合 $A$,$B$,如果按照某种对应法则 $f$,对于集合 $A$ 中的任何一个元素,在
集合 $B$ 中都有唯一的元素和它对应,这样的对应(包括集合 $A$,$B$ 及对应法则 $f$)就是
从集合 $A$ 到集合 $B$ 的映射,表示为 $f: A \to B$。

设 $f: A \to B$ 是从集合 $A$ 到集合 $B$ 的一个映射,如果在这个映射的作用下,对于 $A$ 中
的不同元素,在 $B$ 中有不同的象,而且 $B$ 中的每一个元素都有原象,那么这个映射就是 $A$
到 $B$ 上的一一映射。这时,如果对于 $B$ 中的每一个元素 $b$,使 $b$ 在 $A$ 中的原象 $a$
和它对应,这样得到的映射就是映射 $f: A \to B$ 的逆映射,表示为 $f^{-1}: B \to A$。
显然,映射 $f: A \to B$ 与映射 $f^{-1}: B \to A$ 互为逆映射,它们互相依存。

以实数 $x$ 为自变量的函数 $y = f(x)$ 实际上是 $x$ 取值的集合到 $y$ 取值的集合上的映射,
其中 $x$ 取值的集合就是函数 $f(x)$ 的定义域,和 $x$ 对应的的 $y$ 值就是函数值,函数值
的集合就是函数 $f(x)$ 的值域。

如果确定函数 $y = f(x)$ 的映射 $f: A \to B$ 是函数 $f(x)$ 的定义域 $A$ 到值域 $B$
上的一一映射,那么这个映射的逆映射 $f^{-1}: B \to A$ 所确定的函数 $y = f^{-1}(x)$
就是函数 $y = f(x)$ 的反函数, $y = f^{-1}(x)$ 的定义域、值域分别是 $y = f(x)$ 的
值域、定义域。由 $y = f(x)$ 求 $y = f^{-1}(x)$ 的步驟是:(1)由 $y = f(x)$ 中解出
$x = f^{-1}(y)$;(2)把 $x = f^{-1}(y)$ 改写为 $y = f^{-1}(x)$。函数 $y = f(x)$
和 $y = f^{-1}(x)$ 的图象关于直线 $y = x$ 对称。

四、在一个区间上,如果对于自变量 $x$ 的任意两个值 $x_1$,$x_2$,且 $x_1 < x_2$,都有
$f(x_1) < f(x_2)$,那么函数 $f(x)$ 在这个区间上是增函数;如果对于任意的 $x_1$,$x_2$,
且 $x_1 < x_2$,都有 $f(x_1) > f(x_2)$,那么函数在这个区间上是减函数。

如果对于函数定义域内任意一个 $x$ 都有 $f(-x) = -f(x)$,那么函数 $f(x)$ 是奇函数。
如果对于函数定义域内任意一个 $x$ 都有 $f(-x) = f(x)$, 那么函数 $f(x)$ 是偶函数。

奇函数的图象关于原点对称;反过来,如果一个函数的图象关于原点对称,那么这个函数是奇函数。
偶函数的图象关于 $y$ 轴对称;反过来,如果一个函数的图象关于 $y$ 轴对称,那么这个函数是偶函数。

五、函数 $y = x^a$ 就是幂函数,其中 $a$ 是一个常量。我们巳讨论了 $a$ 是有理数 $n$ 的情况:
当 $n$ 为正数时,幂函数的定义域是实数集 $R$;
当 $n$ 为零或负整数时,幂函数的定义域是除 $x = 0$ 以外的所有实数;
当 $n$ 为正分数 $\dfrac p q$ 或负分数 $-\dfrac p q$,($p$,$q$ 是互质的正整数,$q > 1$)时,
$x^n$ 的意义分别是 $\sqrt[q]{x^p}$ 或 $\dfrac{1}{\sqrt[q]{x^p}}$,幂函数的定义域分别是使
$\sqrt[q]{x^p}$ 或 $\dfrac{1}{\sqrt[q]{x^p}}$ 有意义的实数的集合。

六、函数 $y = a^x$ 就是指数函数, 其中 $a$ 是一个大于零且不等于 $1$ 的常量,函数的定义域是实数集 $R$。
函数 $y = \log_a x$ 就是对数函数,其中 $a$ 是一个大于零且不等于 $1$ 的常量,函数的定义域是 $R^+$。
指数函数 $y = a^x$ 和对数函数 $y = \log_a x$ 互为反函数,它们的图象关于直线 $y = x$ 对称。

七、在指数方程和对数方程中,我们只能解一些比较特殊的方程。解这些特殊方程一般是根据指数、对数的定义,
或采取将方程两边化成同底幂、同底对数,从而得到幂指数相等、真数相等的新方程,或在方程两边同时取对数,
从而得到新方程等。解对数方程时,可能产生增根,因此,检验是解对数方程整个过程中不可缺少的一步。
