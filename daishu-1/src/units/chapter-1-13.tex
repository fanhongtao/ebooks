\subsection{对数函数}\label{subsec:1-13}

前面我们讲过细胞分裂问题,知道有的细胞分裂时,得到的细胞的个数 $y$ 是分裂次数 $x$ 的函数,这个函数可以用指数函数 $y = 2^x$ 表示。
现在我们来研究相反的问题,例如,如果目的是求一个这样的细胞经过多少次分裂,大约可以得到1万个,10万个,……细胞,那么分裂次数 $x$ 就是
得到的细胞的个数 $y$ 的函数,这个函数写成对数的形式就是
$$x = \log_2 y \text{。}$$

按照习惯,如果用 $x$ 表示自变量,用 $y$ 表示函数,那么这个函数就是
$$y = \log_2 x \text{。}$$

由指数的概念知道,确定函数 $y = f(x) = 2^x$ 的映射 $f: R \to R^+$ 是 $f(x)$ 的定义域 $R$
到值域 $R^+$ 上的一一映射。由对数的概念知道,这一映射的逆映射 $f^{-1}: R^+ \to R$ 所确定的
函数是 $x = f^{-1}(y)  = \log_2 y$。所以由反函数的概念可知,函数 $y = \log_2 x$ 是指数函数
$y = 2^x$ 的反函数。

从第 \ref{subsec:1-12} 节可知,指数函数 $y = 2^x$ 的变量 $x$,$y$ 的对应值表是:

\begin{table}[H]
\renewcommand\arraystretch{2}
\begin{tabular}{|w{c}{5em}|*{9}{w{c}{2em}|}}
    \hline
    $x$ & $\dots$ & $-3$ & $-2$ & $-1$ & $0$ & $1$ & $2$ & $3$ & $\dots$ \\
    \hline
    $y$ & $\dots$ & $\dfrac 1 8$ & $\dfrac 1 4$ & $\dfrac 1 2$ & $1$ & $2$ & $4$ & $8$ & $\dots$ \\
    \hline
\end{tabular}
\end{table}

那么只要把两行的数值对调,就得到函数 $y = \log_2 x$ 的变量对应值表:

\begin{table}[H]
\renewcommand\arraystretch{2}
\begin{tabular}{|w{c}{5em}|*{9}{w{c}{2em}|}}
    \hline
    $x$ & $\dots$ & $\dfrac 1 8$ & $\dfrac 1 4$ & $\dfrac 1 2$ & $1$ & $2$ & $4$ & $8$ & $\dots$ \\
    \hline
    $y$ & $\dots$ & $-3$ & $-2$ & $-1$ & $0$ & $1$ & $2$ & $3$ & $\dots$ \\
    \hline
\end{tabular}
\end{table}

一般地,对数函数 $y = \log_a x$(这里底数 $a$ 是一个大于零且不等于 $1$ 的常量)就是指数函数 $y = a^x$的反函数。
因为 $y = a^x$ 的值域是 $(0, +\infty)$(即$R^+$),所以函数 $y = \log_a x$ 的定义域是 $(0, +\infty)$。

函数 $y = \log_a x$ 叫做\textbf{对数函数}。

现在研究对数函数 $y = \log_a x$ 的图象和性质。

因为对数函数 $y = \log_a x$ 是指数函数 $y = a^x$ 的反函数,所以 $y = \log_a x$ 的图象和
$y = a^x$ 的图象关于直线 $y = x$ 对称。因此,我们只要画出和 $y = a^x$ 的图象关于直线 $y = x$
对称的曲线,就可以得到 $y = \log_a x$ 的图象,例如,画出和第 \ref{subsec:1-12} 节中三个
函 $y = 2^x$,$y = 10^x$,$y = \left( \dfrac 1 2 \right)^x$ 的图象关于直线 $y = x$ 对称
的曲线,就可得到 $y = \log_2 x$,$y = \log_{10} x$,$y = \log_{\frac 1 2} x$ 的图象(图 \ref{fig:1-30})。

\begin{figure}[H]
    \centering
    \begin{tikzpicture}[>=Stealth]
    \draw [->] (-2.5,0) -- (5.0,0) node[anchor=north] {$x$};
    \draw [->] (0,-1.5) -- (0,4.5) node[anchor=east] {$y$};
    \node at (0.2,-0.2) {$O$};
    \node at (0.2,0.6) {$1$};
    \foreach \x in {-1,1,2,3,4} {
        \draw (\x,0.2) -- (\x,0) node[anchor=north] {$\x$};
    }
    \foreach \y in {-1,2,3,4} {
        \draw (0.2,\y) -- (0,\y) node[anchor=east] {$\y$};
    }
    
    \draw[domain=-2.1:1.9,samples=50] plot (\x, {2^\x}) +(0.5, -0.2) node {$y = 2^x$};
    \draw[domain=-1.2:0.53,samples=50] plot (\x, {10^\x}) +(0.6, 0.3) node {$y = 10^x$};
    \draw[domain=1.1:-1.9,samples=50] plot (\x, {(1/2)^\x}) +(0.5, 0.3) node {$y = \displaystyle \left(\frac 1 2 \right)^x$};

    \draw[dash pattern=on 5mm off 2mm] (-1, -1) -- (4, 4) +(0.5,-0.3) node {$y = x$};

    \draw[domain=0.233:3.732,samples=50] plot (\x, {log2(\x)}) +(0.5, -0.3) node {$y = \log_2 x$};
    \draw[domain=0.0631:3.388,samples=50] plot (\x, {log10(\x)}) +(0.5, 0.3) node {$y = \log_{10} x$};
    \draw[domain=0.4665:3.732,samples=50] plot (\x, {ln(\x) / ln(1/2)}) +(0.4, 0.4) node {$y = \log_{\frac 1 2} x$};
\end{tikzpicture}

    \caption{}\label{fig:1-30}
\end{figure}

一般地,对数函数 $y = \log_a x$ 在其底数 $a > 1$ 及 $0 < a < 1$ 这两种情况下的图象和性质如下表所示:

\begin{table}[H]
\begin{tabular}{|c|l|l|}
    \hline
    \multirow{2}{*}{图象} & \makecell[c]{$a > 1$} & \makecell[c]{$0 < a < 1$} \\
    \cline{2-3}
    & \includegraphics{pic-pdf/dui-shu-han-shu-1.pdf} & \includegraphics{pic-pdf/dui-shu-han-shu-2.pdf} \\
    \hline
    \multirow{4}{*}{性质} & \multicolumn{2}{l|}{(1)$x > 0$;} \\
    \cline{2-3}
    & \multicolumn{2}{l|}{(2)当 $x = 0$ 时,$y = 1$;} \\
    \cline{2-3}
    &  \makecell[l]{(3)当 $x>1$ 时,$y>0$,\\ \hspace{2em} $0<x<1$ 时,$y<0$ ;}  & \makecell[l]{(3)当 $x>1$ 时,$y<0$,\\ \hspace{2em} $0<x<1$ 时,$y>0$ ;} \\
    \cline{2-3}
    & (4)在 $(0, +\infty)$ 上是增函数。 & (4)在 $(0, +\infty)$ 上是减函数。 \\
    \hline
\end{tabular}
\end{table}

\liti 求下列函数的定义域:
\begin{xiaoxiaotis}

    \xiaoxiaoti{$y = \log_a (x^2)$;}

    \xiaoxiaoti{$y = \log_a (4 - x)$。}

\end{xiaoxiaotis}

\jie (1)因为 $x^2 > 0$,即 $x \neq 0$,所以函数 $y = \log_a (x^2)$ 的定义域是 $\{x | x \in R \text{,且} x \neq 0\}$。

(2)因为 $4 - x > 0$,即 $x < 4$,所以函数 $y = \log_a (4 - x)$ 的定义域是 $(-\infty, 4)$。

\liti 比较下列各组中两个值的大小:

\begin{xiaoxiaotis}
    
    \xiaoxiaoti{$\log_2 3$,$\log_2 3.5$;}

    \xiaoxiaoti{$\log_{0.7} 1.6$,$\log_{0.7} 1.8$。}

\end{xiaoxiaotis}

\jie 分别考察对数函数 $y = \log_2 x$ 与 $y = \log_{0.7} x$,根据对数函数的性质知道:

(1) $\because$ \quad $2 > 1$,$3 < 3.5$,

$\therefore \quad \log_2 3 < \log_2 3.5$;

(2) $\because$ \quad $0.7 < 1$,$1.6 < 1.8$,

$\therefore \quad \log_{0.7} 1.6 > \log_{0.7} 1.8$。

\lianxi

\begin{xiaotis}

\xiaoti{画出函数 $y = \log_3 x$ 及 $y = \log_{\frac 1 3} x$ 的图象,并且说明这两个函数的相同性质和不同性质。}

\xiaoti{求下列函数的定义域:}
\begin{xiaoxiaotis}

    \xiaoxiaoti{$y = \log_5 (1 + x)$;}

    \vspace{0.5em}
    \xiaoxiaoti{$y = \dfrac 1 {\log_2 x}$;}

    \vspace{0.5em}
    \xiaoxiaoti{$y = \log_7 \dfrac 1 {1 - 3x}$;}
    \vspace{0.5em}

    \xiaoxiaoti{$y = \sqrt{\log_3 x}$。}

\end{xiaoxiaotis}

\xiaoti{比较下面各题中两个值的大小:}
\begin{xiaoxiaotis}

    \renewcommand\arraystretch{1.5}
    \begin{tabular}[t]{*{2}{@{}p{16em}}}
        \xiaoxiaoti {$\log_{10} 6$,$\log_{10} 8$;} & \xiaoxiaoti {$\log_{0.5} 6$,$\log_{0.5} 4$;} \\
        \xiaoxiaoti {$\log_{\frac 2 3} 0.5$,$\log_{\frac 2 3} 0.6$;} & \xiaoxiaoti {$\log_{1.5} 1.6$,$\log_{1.5} 1.4$。}
    \end{tabular}

\end{xiaoxiaotis}

\end{xiaotis}
