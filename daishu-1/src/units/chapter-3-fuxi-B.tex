{\centering \nonumsubsection{B \hspace{1em} 组}}

\begin{xiaotis}
\setcounter{cntxiaoti}{28}

\xiaoti{证明下列各恒等式:}
\begin{xiaoxiaotis}

    \jiange\xiaoxiaoti{$\dfrac{\cos\theta - \sin\theta}{\cos\theta + \sin\theta} = \sec2\theta - \tan2\theta$;}\jiange

    \xiaoxiaoti{$(\cos\alpha + \cos\beta)^2 + (\sin\alpha + \sin\beta)^2 = 4\cos^2\left(\dfrac{\alpha - \beta}{2}\right)$;}\jiange

    \xiaoxiaoti{$\sin\alpha + \sin\beta + \sin\gamma - \sin(\alpha + \beta + \gamma) = 4\sin\dfrac{\alpha + \beta}{2} \sin\dfrac{\beta + \gamma}{2} \sin\dfrac{\gamma + \alpha}{2}$;}\jiange

    \xiaoxiaoti{$\cos\alpha + \cos\beta + \cos\gamma + \cos(\alpha + \beta + \gamma) = 4\cos\dfrac{\alpha + \beta}{2} \cos\dfrac{\beta + \gamma}{2} \cos\dfrac{\gamma + \alpha}{2}$。}\jiange

\end{xiaoxiaotis}

\xiaoti{把下列各式化成积的形式:}
\begin{xiaoxiaotis}

    \xiaoxiaoti{$\sqrt{1 - \cos\alpha} + \sqrt{1 + \cos\alpha} \; (\alpha \text{在第四象限})$;}

    \jiange\xiaoxiaoti{$\sqrt{\tan x + \sin x} + \sqrt{\tan x - \sin x} \; \left(\pi < x < \dfrac{3\pi}{2}\right)$。}\jiange

\end{xiaoxiaotis}

\xiaoti{设 $\tan\dfrac{\alpha}{2} = t$,用含 $t$ 的有理式表示下列各函数:}
\begin{xiaoxiaotis}

    \jiange\xiaoxiaoti{$\dfrac{1 + \sin\alpha}{\sin\alpha (1 + \cos\alpha)}$;}\jiange

    \xiaoxiaoti{$\dfrac{\sin\alpha}{\sin\alpha + \cos\alpha}$。}\jiange

\end{xiaoxiaotis}

\xiaoti{设 $\sin\alpha$,$\sin\beta$ 是方程 \jiange \\
    $\begin{gathered}
        \qquad x^2 - (\sqrt{2}\cos20^\circ)x + \left(\cos^2 20^\circ - \dfrac{1}{2}\right) = 0
    \end{gathered}$ \jiange \\
    的两根,求 $\alpha ,\, \beta \, (0^\circ < \alpha < 90^\circ, \, 0^\circ < \beta < 90^\circ)$。
}

\xiaoti{如果方程 $x^2 + px + q = 0$ 的两个根分别是 $\tan\varphi$ 与 $\tan\left(\dfrac{\pi}{4} - \varphi\right)$,
    而且两根的比是 $\dfrac{3}{2}$,求 $p$,$q$ 的值。}

\xiaoti{已知三角形的最小内角为 $30^\circ$,它的对边长为 $2cm$,另外两个内角的差为 $60^\circ$,求最大边的长(要准确值)。}

\xiaoti{设 $A$,$B$,$C$ 是一三角形的三个内角,且
    $$\lg\sin A - \lg\cos B - \lg\sin C = \lg 2,$$
    求证这个三角形是等腰三角形。
}

\xiaoti{在 $\triangle ABC$ 中,
    $$\sin C = \dfrac{\sin A + \sin B}{\cos A + \cos B},$$
    求证这个三角形是直角三角形。
}

\xiaoti{设 $R$ 是 $\triangle ABC$ 的外接圆半径,求证:}
$$a + b + c = 8R\cos\dfrac{A}{2} \cos\dfrac{B}{2} \cos\dfrac{C}{2} \text{。}\jiange$$

\xiaoti{已知正 $n$ 边形的边长为 $a$,内切圆半径为 $r$,外接圆半径为 $R$,求证:}
$$R + r = \dfrac{1}{2}a \cdot \cot\dfrac{\pi}{2n} \text{。}\jiange$$

\xiaoti{半径分别为 $R$,$r$ ($R > r$)的两圆相外切,它们的两条外公切线的夹角为 $\theta$,求证:}
$$\sin\theta = \dfrac{4(R - r)\sqrt{Rr}}{(R + r)^2} \text{。}\jiange$$

\xiaoti{求证:}
\begin{xiaoxiaotis}

    \xiaoxiaoti{$\displaystyle \sum_{k = 1}^n \sin kx = \dfrac{\sin\dfrac{n + 1}{2}x \sin\dfrac{n}{2}x}{\sin\dfrac{x}{2}}$;}\jiange

    \xiaoxiaoti{$\displaystyle \sum_{k = 1}^n \cos kx = \dfrac{\cos\dfrac{n + 1}{2}x \sin\dfrac{n}{2}x}{\sin\dfrac{x}{2}}$。}\jiange

\end{xiaoxiaotis}

\end{xiaotis}
