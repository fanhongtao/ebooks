\subsection{半角的正弦、余弦和正切}\label{subsec:3-3}

上节我们研究了用单角的三角函数表示二倍角的三角函数。这一节我们研究如何用单角的三角函数表示半角(单角的一半)的三角函数。

由 $\cos2\alpha = 1 - 2\sin^2\alpha = 2\cos^2\alpha - 1$,得
$$ \cos\alpha = 1 - 2\sin^2\dfrac{\alpha}{2} = 2\cos^2\dfrac{\alpha}{2} - 1 \text{,} $$
即
\begin{gather*}
    2\sin^2\dfrac{\alpha}{2} = 1 - \cos\alpha \text{;} \jiange \\
    2\cos^2\dfrac{\alpha}{2} = 1 + \cos\alpha \text{。}
\end{gather*}

$\therefore$
\begin{gather}
    \sin\dfrac{\alpha}{2} = \pm\sqrt{\dfrac{1 - \cos\alpha}{2}} \text{,} \tag{$S_{\frac{\alpha}{2}}$}\label{eq:S-a/2} \\
    \cos\dfrac{\alpha}{2} = \pm\sqrt{\dfrac{1 + \cos\alpha}{2}} \text{,} \tag{$C_{\frac{\alpha}{2}}$}\label{eq:C-a/2}
\end{gather}

\jiange
将这两个公式左边、右边分别相除,又可得
\begin{gather}
    \tan\dfrac{\alpha}{2} = \pm\sqrt{\dfrac{1 - \cos\alpha}{1 + \cos\alpha}} \text{。} \tag{$T_{\frac{\alpha}{2}}$}\label{eq:T-a/2}
\end{gather}

这三个公式中根号前的符号,由 $\dfrac{\alpha}{2}$ 所在的象限来确定。如果没有给出限定符号的条件,根号前面应保持正负两个符号。

$\tan\dfrac{\alpha}{2}$ 还可以用 $\sin\alpha$,$\cos\alpha$ 的不带根号的式子来表示:
$$ \tan\dfrac{\alpha}{2} = \dfrac{\sin\dfrac{\alpha}{2}}{\cos\dfrac{\alpha}{2}}
   = \dfrac{\sin\dfrac{\alpha}{2} \cdot 2\cos\dfrac{\alpha}{2}}{\cos\dfrac{\alpha}{2} \cdot 2\cos\dfrac{\alpha}{2}}
   = \dfrac{\sin\alpha}{1 + \cos\alpha} \text{,}
$$
或
$$ \tan\dfrac{\alpha}{2} = \dfrac{\sin\dfrac{\alpha}{2}}{\cos\dfrac{\alpha}{2}}
    = \dfrac{\sin\dfrac{\alpha}{2} \cdot 2\sin\dfrac{\alpha}{2}}{\cos\dfrac{\alpha}{2} \cdot 2\sin\dfrac{\alpha}{2}}
    = \dfrac{1 - \cos\alpha}{\sin\alpha} \text{。}
$$
即
\begin{gather}
    \tan\dfrac{\alpha}{2} = \dfrac{\sin\alpha}{1 + \cos\alpha} = \dfrac{1 - \cos\alpha}{\sin\alpha} \text{。} \tag{${T_{\frac{\alpha}{2}}}'$}\label{eq:T-a/2-'}
\end{gather}

\jiange
\liti 已知 $\cos\alpha = \dfrac 1 2$,求 $\sin\dfrac{\alpha}{2}$,$\cos\dfrac{\alpha}{2}$,$\tan\dfrac{\alpha}{2}$。\jiange

\jie $\begin{aligned}[t]
    & \sin\dfrac{\alpha}{2} = \pm\sqrt{\dfrac{1 - \cos\alpha}{2}} = \pm\sqrt{\dfrac{1 - \dfrac 1 2}{2}} = \pm\dfrac 1 2 \text{,} \\
    & \cos\dfrac{\alpha}{2} = \pm\sqrt{\dfrac{1 + \cos\alpha}{2}} = \pm\sqrt{\dfrac{1 + \dfrac 1 2}{2}} = \pm\dfrac{\sqrt 3} 2 \text{,} \\
    & \tan\dfrac{\alpha}{2} = \pm\dfrac{\dfrac 1 2}{\dfrac{\sqrt 3} 2} = \pm\dfrac{\sqrt 3}{3} \text{。}
\end{aligned}$ \jiange

\liti 已知 $\cos\theta = -\dfrac 3 5$,并且 $180^\circ < \theta < 270^\circ$,求 $\tan\dfrac{\theta}{2}$。

\textbf{解法一:} 因为 $180^\circ < \theta < 270^\circ$,所以 $90^\circ < \dfrac{\theta}{2} < 135^\circ$,即 $\dfrac{\theta}{2}$ 是第二象限角。

$\therefore \quad \tan\dfrac{\theta}{2} = -\sqrt{\dfrac{1 - \cos\theta}{1 + \cos\theta}} = -\sqrt{\dfrac{1 - \left(-\dfrac 3 5\right)}{1 + \left(-\dfrac 3 5\right)}} = -2$。\jiange

\textbf{解法二:} 因为 $180^\circ < \theta < 270^\circ$,即 $\theta$ 是第三象限角。\jiange

$\therefore \quad \sin\theta = -\sqrt{1 - \cos^2\theta} = -\sqrt{1 - \dfrac{9}{25}} = -\dfrac 4 5$。\jiange

$\therefore \quad \tan\dfrac{\theta}{2} = \dfrac{1 - \cos\theta}{\sin\theta} = \dfrac{1 - \left(-\dfrac 3 5\right)}{-\dfrac 4 5} = -2$,\jiange

或 $\tan\dfrac{\theta}{2} = \dfrac{\sin\theta}{1 + \cos\theta} = \dfrac{-\dfrac 4 5}{1 + \left(-\dfrac 3 5\right)} = -2$。

如果已经知道 $\sin\alpha$,$\cos\alpha$ 的值,那么求 $\tan\dfrac{\alpha}{2}$ 时,用公式 (\ref{eq:T-a/2-'})比较方便。
一般地说,选用分母为单项式的公式 $\tan\dfrac{\alpha}{2} = \dfrac{1 - \cos\alpha}{\sin\alpha}$ 更为方便。 \jiange

\liti 求证 $\dfrac{\cos^2\alpha}{\cot\dfrac{\alpha}{2} - \tan\dfrac{\alpha}{2}} = \dfrac 1 4 \sin2\alpha$。\jiange

\textbf{证法一:} $\begin{aligned}[t]
    & \dfrac{\cos^2\alpha}{\cot\dfrac{\alpha}{2} - \tan\dfrac{\alpha}{2}} = \dfrac{\cos^2\alpha}{\dfrac{1 + \cos\alpha}{\sin\alpha} - \dfrac{1 - \cos\alpha}{\sin\alpha}} \\
    =& \dfrac{\cos^2\alpha \sin\alpha}{2\cos\alpha} = \dfrac 1 2 \sin\alpha \cos\alpha = \dfrac 1 4 \sin2\alpha \text{。}
\end{aligned}$ \jiange

\textbf{证法二:} $\begin{aligned}[t]
    \dfrac{\cos^2\alpha}{\cot\dfrac{\alpha}{2} - \tan\dfrac{\alpha}{2}} &= \dfrac{\cos^2\alpha \tan\dfrac{\alpha}{2}}{1 - \tan^2\dfrac{\alpha}{2}} = \dfrac 1 2 \cos^2\alpha \cdot \tan\alpha \\
    &= \dfrac 1 2 \sin\alpha \cos\alpha = \dfrac 1 4 \sin2\alpha \text{。}
\end{aligned}$ \jiange

\liti 用 $\tan\dfrac{\alpha}{2}$ 表示 $\sin\alpha$,$\cos\alpha$,$\tan\alpha$。

\jie $\begin{aligned}[t]
    & \sin\alpha = 2\sin\dfrac{\alpha}{2} \cos\dfrac{\alpha}{2} = 2\tan\dfrac{\alpha}{2} \cdot \cos^2\dfrac{\alpha}{2}
        = \dfrac{2\tan\dfrac{\alpha}{2}}{\sec^2\dfrac{\alpha}{2}}
        = \dfrac{2\tan\dfrac{\alpha}{2}}{1 + \tan^2\dfrac{\alpha}{2}}, \\
    & \cos\alpha = 2\cos^2\dfrac{\alpha}{2} - 1
        = \dfrac{2}{\sec^2\dfrac{\alpha}{2}} - 1
        = \dfrac{2}{1 + \tan^2\dfrac{\alpha}{2}} - 1
        = \dfrac{1 - \tan^2\dfrac{\alpha}{2}}{1 + \tan^2\dfrac{\alpha}{2}} , \\
    & \tan\alpha = \tan\left(2 \cdot \dfrac{\alpha}{2}\right) = \dfrac{2\tan\dfrac{\alpha}{2}}{1 - \tan^2\dfrac{\alpha}{2}} \text{。}
\end{aligned}$

用 $\tan\dfrac{\alpha}{2}$ 分别表示 $\sin\alpha$,$\cos\alpha$,$\tan\alpha$ 的公式,即
\begin{gather*}
    \sin\alpha = \dfrac{2\tan\dfrac{\alpha}{2}}{1 + \tan^2\dfrac{\alpha}{2}} ,\quad
    \cos\alpha = \dfrac{1 - \tan^2\dfrac{\alpha}{2}}{1 + \tan^2\dfrac{\alpha}{2}} ,\quad
    \tan\alpha = \dfrac{2\tan\dfrac{\alpha}{2}}{1 - \tan^2\dfrac{\alpha}{2}} \text{,}
\end{gather*}
通常叫做\textbf{万能公式}\mylabel{wannenggongshi}。这是因为,不论 $\alpha$ 角的哪一种三角函数,
都可以用这几个公式把它化为 $\tan\dfrac{\alpha}{2}$ 的有理式,这样就可以把
问题转化为以 $\tan\dfrac{\alpha}{2}$ 为变量的一元有理函数,往往有助于问题解决。

我们用\hyperref[wannenggongshi]{万能公式} 来证明例3:

$
\text{右边} = \dfrac 1 2 \cdot \dfrac{\tan\alpha}{1 + \tan^2\alpha}
    = \dfrac 1 2 \tan\alpha \cdot \cos^2\alpha
    = \dfrac{\tan\dfrac{\alpha}{2}}{1 - \tan^2\dfrac{\alpha}{2}} \cdot \cos^2\alpha,
$ \jiange

$
\text{左边} = \cos^2\alpha \cdot \dfrac{\tan\dfrac{\alpha}{2}}{1 - \tan^2\dfrac{\alpha}{2}} \text{。}
$ \jiange

$\therefore$ \quad 原式成立。

\lianxi
\begin{xiaotis}

\xiaoti{已知 $\cos\alpha = \dfrac 2 3$,求 $\sin\dfrac{\alpha}{2}$,$\cos\dfrac{\alpha}{2}$,$\tan\dfrac{\alpha}{2}$。}\jiange

\xiaoti{已知 $\sin\theta = -\dfrac 4 5$,且 $270^\circ < \theta < 360^\circ$,求 $\sin\dfrac{\theta}{2}$,$\cos\dfrac{\theta}{2}$,$\tan\dfrac{\theta}{2}$。} \jiange

\xiaoti{已知 $\cos A = \dfrac 4 5$,且 $\dfrac 3 2 \pi < A < 2\pi$,求 $\tan\dfrac{A}{2}$。} \jiange

\xiaoti{求证 $\tan\dfrac{\pi}{8} = \sqrt{2} - 1$。} \jiange

\xiaoti{证明下列恒等式:}
\begin{xiaoxiaotis}

    \xiaoxiaoti{$\sin^2\dfrac{\alpha}{4} = \dfrac{1 - \cos\dfrac{\alpha}{2}}{2}$;} \jiange

    \xiaoxiaoti{$1 + \sin\alpha = 2\cos^2\left( \dfrac{\pi}{4} - \dfrac{\alpha}{2} \right)$;} \jiange

    \xiaoxiaoti{$1 - \sin\theta = 2\cos^2\left( \dfrac{\pi}{4} + \dfrac{\theta}{2} \right)$;} \jiange

    \xiaoxiaoti{$\dfrac{\cos A}{\cot\dfrac{A}{2} - \tan\dfrac{A}{2}} = \dfrac 1 2 \sin A $;} \jiange

    \xiaoxiaoti{$\dfrac{\sin 2\alpha}{1 + \cos 2\alpha} \cdot \dfrac{\cos\alpha}{1 + \cos\alpha} = \tan\dfrac{\alpha}{2}$;} \jiange

    \xiaoxiaoti{$\dfrac{4\sin\alpha(1 - \tan^2\alpha)}{\sec\alpha(1 + \tan^2\alpha)} = \sin 4\alpha$。} \jiange

\end{xiaoxiaotis}

\xiaoti{已知 $\tan\alpha = -3$,求 $2\alpha$ 的各三角函数值。}

\end{xiaotis}
