\xiti
\begin{xiaotis}

\xiaoti{求下列各三角函数值:}
\begin{xiaoxiaotis}

    % \renewcommand\arraystretch{1.5}
    \begin{tabular}[t]{*{2}{@{}p{15em}}}
        \xiaoxiaoti{$\cos 210^\circ$;} & \xiaoxiaoti{$\sin 263^\circ 42'$;} \\
        \xiaoxiaoti{$\cot \dfrac 4 3 \pi$;} & \xiaoxiaoti{$\cos \left( -\dfrac \pi 6 \right)$;} \rule{0pt}{2em} \\
        \xiaoxiaoti{$\sin \left( -\dfrac 5 3 \pi \right)$;} & \xiaoxiaoti{$\cos \left( -\dfrac{11}{9} \pi \right)$;} \rule{0pt}{2em} \\
        \xiaoxiaoti{$\tan 165^\circ 18'$;} & \xiaoxiaoti{$\cos(-104^\circ 26')$;} \\
        \xiaoxiaoti{$\cot 250^\circ 24'$;} & \xiaoxiaoti{$\tan \dfrac 7 4 \pi$。}
    \end{tabular}

\end{xiaoxiaotis}

\xiaoti{化简:}
\begin{xiaoxiaotis}

    \xiaoxiaoti{$\dfrac{\sin(180^\circ + \alpha) - \tan(-\alpha) - \tan(360^\circ + \alpha)}{\tan(\alpha + 180^\circ) + \cos(-\alpha) + \cos(\alpha + 180^\circ)}$;}
    \vspace{0.5em}

    \xiaoxiaoti{$\dfrac{\sin^2(\alpha + \pi) \cdot \cos(\pi + \alpha) \cdot \cot(\alpha + 2\pi)}{\tan(\pi + \alpha) \cdot \cos^3(-\alpha - \pi)}$。}

\end{xiaoxiaotis}

\xiaoti{求证:}
\begin{xiaoxiaotis}

    \xiaoxiaoti{$\cos(-210^\circ) \cdot \tan(-240^\circ) + \sin(-30^\circ) - \cot 225^\circ = 0$;}

    \vspace{0.5em}
    \xiaoxiaoti{$\dfrac{\cot(-\alpha - \pi) \cdot \sin(\pi + \alpha)}{\cos(-\alpha) \cdot \tan(2\pi + \alpha)} = \cot\alpha$。}
    \vspace{0.5em}
\end{xiaoxiaotis}

\xiaoti{求下列各三角函数值:}
\begin{xiaoxiaotis}

    \renewcommand\arraystretch{1.5}
    \begin{tabular}[t]{*{2}{@{}p{15em}}}
        \xiaoxiaoti{$\cos \left( -\dfrac{17\pi}{4} \right)$;} & \xiaoxiaoti{$\sin(-1574^\circ)$;} \\
        \xiaoxiaoti{$\tan \dfrac{47}{15} \pi$;} & \xiaoxiaoti{$\cot \left( -\dfrac{55}{12} \pi \right)$;} \\
        \xiaoxiaoti{$\sin(-2160^\circ 52')$;} & \xiaoxiaoti{$\cos(-1751^\circ 36')$;} \\
        \xiaoxiaoti{$\tan \left( -\dfrac{70}{9} \pi \right)$;} & \xiaoxiaoti{$\cos 1615^\circ 8'$;} \\
        \xiaoxiaoti{$\sin \left( -\dfrac{26}{3} \pi \right)$;} & \xiaoxiaoti{$\sin(-23.1\pi)$;} \\
        \xiaoxiaoti{$\tan 10$;} & \xiaoxiaoti{$\cos(-3.1)$。}
    \end{tabular}

\end{xiaoxiaotis}

\xiaoti{化简:}
\begin{xiaoxiaotis}

    \xiaoxiaoti{$\sin(-1071^\circ) \cdot \sin 99^\circ + \sin(-171^\circ) \cdot \sin(-261^\circ) - \cot 1089^\circ \cdot \cot(-630^\circ)$;}

    \xiaoxiaoti{$1 + \sin(\alpha - 2\pi) \cdot \sin(\pi + \alpha) - \tan(\pi - \alpha) \cdot \cot(\alpha - \pi) - 2\cos^2(-\alpha)$。}

\end{xiaoxiaotis}

\xiaoti{求证:}
\begin{xiaoxiaotis}

    \xiaoxiaoti{$\sin(-\alpha) \cdot \sin(\pi - \alpha) - \tan(-\alpha) \cdot \cot(\alpha - \pi) - 2\cos^2(-\alpha) + 1 = \sin^2 \alpha$;}

    \xiaoxiaoti{$\dfrac{\cos(\alpha - \pi) \cot(5\pi - \alpha)}{\tan(2\pi - \alpha) \sin(-2\pi - \alpha)} = \cot^3 \alpha$。}

\end{xiaoxiaotis}

\xiaoti{根据下列条件,求三角形的内角 $A$:}
\begin{xiaoxiaotis}

    \renewcommand\arraystretch{1.5}
    \begin{tabular}[t]{*{2}{@{}p{15em}}}
        \xiaoxiaoti{$\sin A = \dfrac 1 2$;} & \xiaoxiaoti{$\cos A = -\dfrac{\sqrt{2}}{2}$;} \\
        \xiaoxiaoti{$\tan A = 1$;} & \xiaoxiaoti{$\cot A = -\sqrt{3}$。}
    \end{tabular}

\end{xiaoxiaotis}

\xiaoti{根据下列条件,求 $0$ \~{} $2\pi$ (或 $0^\circ$ \~{} $360^\circ$)间的角 $\alpha$:}
\begin{xiaoxiaotis}

    \renewcommand\arraystretch{1.5}
    \begin{tabular}[t]{*{2}{@{}p{15em}}}
        \xiaoxiaoti{$\sin\alpha = -\dfrac{\sqrt{3}}{2}$;} & \xiaoxiaoti{$\cos\alpha = 0.1896$;} \\
        \xiaoxiaoti{$\tan\alpha = 8$;} & \xiaoxiaoti{$\cot\alpha = 1$。}
    \end{tabular}

\end{xiaoxiaotis}

\xiaoti{求适合下列条件的 $x$ 的集合:}
\begin{xiaoxiaotis}

    \renewcommand\arraystretch{1.5}
    \begin{tabular}[t]{*{2}{@{}p{15em}}}
        \xiaoxiaoti{$\sin x = -1$;} & \xiaoxiaoti{$\cos x = 0$;} \\
        \xiaoxiaoti{$\sin x = \dfrac{12}{13}$;} & \xiaoxiaoti{$\tan x = -\sqrt{5}$;} \\
        \xiaoxiaoti{$\sec x = 4.023$;} & \xiaoxiaoti{$\cot x = 0.8594$。}
    \end{tabular}

\end{xiaoxiaotis}

\xiaoti{求适合下列条件的 $x$ 的集合:}
\begin{xiaoxiaotis}

    \renewcommand\arraystretch{1.5}
    \begin{tabular}[t]{*{2}{@{}p{15em}}}
        \xiaoxiaoti{$\cot x + \sqrt{3} = 0$;} & \xiaoxiaoti{$3\tan x - 1 = 0$;} \\
        \xiaoxiaoti{$\cos(\pi - x) = -\dfrac{\sqrt{3}}{2}$;} & \xiaoxiaoti{$2\sin^2 x = 1$。}
    \end{tabular}

\end{xiaoxiaotis}

\end{xiaotis}
