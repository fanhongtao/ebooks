{\centering \nonumsubsection{B \hspace{1em} 组}}
\begin{xiaotis}
\setcounter{cntxiaoti}{29}

\xiaoti{求证下列恒等式:}
\begin{xiaoxiaotis}

    \jiange
    \xiaoxiaoti{$\dfrac{\sin^2 x}{\sin x - \cos x} - \dfrac{\sin x + \cos x}{\tan^2 x - 1} = \sin x + \cos x$;}
    \jiange

    \xiaoxiaoti{$1 - (\cos^6 x + \sin^6 x) = 3\sin^2 x \cdot \cos^2 x$。}

\end{xiaoxiaotis}

\xiaoti{已知 $\sin x + \cos x = m$,求 $\sin^4 x + \cos^4 x$。}

\xiaoti{已知 $\tan^2 x + \cot^2 x = p$,$\tan^4 x + \cot^4 x = q$,求 $p$ 与 $q$ 之间的关系。}

\jiange
\xiaoti{用 $\tan\alpha$ 表示 $\dfrac{1}{\cos^2\alpha \cdot \sin^2\alpha}$。}
\jiange

\xiaoti{已知 $\dfrac x a \cos\theta + \dfrac y b \sin\theta = 1$,$\dfrac x a \sin\theta - \dfrac y b \cos\theta = 1$,求证 \jiange
    $$\dfrac{x^2}{a^2} + \dfrac{y^2}{b^2} = 2 \text{。}$$
}

\xiaoti{求适合下列条件的 $x$ 的集合:}
\begin{xiaoxiaotis}

    \jiange
    \twoInLineXxt[14em]{$\sin \dfrac x 2 = \dfrac{\sqrt 2}{2}$;}{$4\cos^2 2x = 1$。}
    \jiange

\end{xiaoxiaotis}

\xiaoti{已知函数 $f(x) = 3\sin\left( \dfrac k 5 x + \dfrac \pi 3 \right)$,其中 $k \neq 0$。}
\jiange
\begin{xiaoxiaotis}

    \xiaoxiaoti{求 $f(x)$ 的最大值、最小值;}

    \xiaoxiaoti{求最小正整数 $k$,使 $f(x)$ 的周期不大于1。}

\end{xiaoxiaotis}

\xiaoti{}
\begin{xiaoxiaotis}

    \vspace{-1.7em} \begin{minipage}{0.9\textwidth}
    \xiaoxiaoti{证明 $\pi$ 是函数 $y = \sin x \cos x$ 的一个周期;}
    \end{minipage}

    \xiaoxiaoti{证明 $\pi$ 是函数 $y = \sin x \cos x$ 的最小正周期(提示:可用反证法。
        设 $T$ 是函数 $y = \sin x \cos x$ 的最小正周期,且 $0 < T < \pi$。可通过取
        $x$ 的一个特殊值,证明这样的 $T$ 不存在)。}

\end{xiaoxiaotis}

\jiange
\xiaoti{已知 $y = \dfrac 1 2 \sqrt{2\left[ A_1^2 + A_2^2 + 2A_1A_2\cos\left( 4\pi f \dfrac x v \right)\right]}$ \jiange,
    其中,$v > 0$,$A_1$、$A_2$、$f$ 都为非负值,$A_1$、$A_2$、$f$、$v$ 都是常量,求}
\begin{xiaoxiaotis}

    \xiaoxiaoti{使 $y$ 达到最大值的 $x$ 的值的集合;}

    \xiaoxiaoti{使 $y$ 达到最小值的 $x$ 的值的集合;}

    \indent \indent(提示:当 $x \geqslant 0$ 时,函数 $y = x^{\frac 1 2}$ 是增函数。)

\end{xiaoxiaotis}

\xiaoti{利用单位圆证明:}
\begin{xiaoxiaotis}

    \xiaoxiaoti{在 $\left[ 0, \, \dfrac \pi 2 \right]$ 上,$y = \sin x$ 是增函数;\jiange}

    \xiaoxiaoti{在 $\left[ 0, \, \dfrac \pi 2 \right]$ 上,$y = \cos x$ 是减函数。\jiange}

    \indent \indent(提示,可用圆中弧、弦、弦心距间的大小关系证明。)

\end{xiaoxiaotis}

\xiaoti{研究下列函数的性质(定义域、值域、周期性、奇偶性、单调性):}
\begin{xiaoxiaotis}

    \twoInLineXxt[14em]{$y = \sin|x|$;}{$y = |\cos x|$。}

\end{xiaoxiaotis}

\end{xiaotis}

