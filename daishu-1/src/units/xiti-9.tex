\xiti
\begin{xiaotis}

\xiaoti{作出下列各角的正弦线、余弦线、正切线:}
\begin{xiaoxiaotis}

    \fourInLineXxt{$\dfrac \pi 4$;}{$-\dfrac \pi 6$;}{$-\dfrac 3 4 \pi$;}{$\dfrac{14}{3} \pi$。}
    \jiange
\end{xiaoxiaotis}

\xiaoti{作出下列函数在 $[0, \, 2\pi]$ 上的简图:}
\begin{xiaoxiaotis}

    \threeInLineXxt{$y = 1 - \sin x$;}{$y = 3\cos x$;}{$y = \dfrac 1 2 \sin x - 1$。}
    \jiange
\end{xiaoxiaotis}

\xiaoti{求下列函数的最大值、最小值及使函数取得这些值的 $x$ 的集合:}
\begin{xiaoxiaotis}

    \jiange
    \twoInLineXxt[14em]{$y = -5\sin x$;}{$y = 1 - \dfrac 1 2 \cos x$;}

    \jiange
    \twoInLineXxt[14em]{$y = 3\sin \left( 2x + \dfrac \pi 3 \right)$;}{$y = \dfrac 1 2 \sin \left( \dfrac 1 2 x + \dfrac \pi 4 \right)$。}
    \jiange

\end{xiaoxiaotis}

\xiaoti{求下列各函数的周期:}
\begin{xiaoxiaotis}

    \renewcommand\arraystretch{1.5}
    \begin{tabular}[t]{*{2}{@{}p{14em}}}
        \xiaoxiaoti {$y = \sin \dfrac 3 4 x$;} & \xiaoxiaoti {$y = \cos 4x$;} \\
        \xiaoxiaoti {$y = \dfrac 1 2 \sin 5x$;} & \xiaoxiaoti {$y = 3\sin \left( \dfrac 1 2 x + \dfrac \pi 3 \right)$。}
    \end{tabular}
    \jiange
\end{xiaoxiaotis}

\xiaoti{}
\begin{xiaoxiaotis}

    \vspace{-1.7em} \begin{minipage}{0.9\textwidth}
    \xiaoxiaoti{证明余弦曲线关于 $y$ 轴对称;}
    \end{minipage}

    \xiaoxiaoti{证明正切曲线关于坐标原点 $O$ 对称。}

\end{xiaoxiaotis}

\xiaoti{在下列函数中,哪些是奇函数?哪些是偶函数?哪些既不是奇函数也不是偶函数?为什么?}
\begin{xiaoxiaotis}

    \begin{tabular}[t]{*{2}{@{}p{14em}}}
        \xiaoxiaoti {$y = -\sin x$;} & \xiaoxiaoti {$y = |\sin x|$;} \\
        \xiaoxiaoti {$y = 3\cos x + 1$;} & \xiaoxiaoti {$y = \sin x - 1$。}
    \end{tabular}

\end{xiaoxiaotis}

\xiaoti{不通过求值,比较下列各组中两个三角函数值的大小:}
\begin{xiaoxiaotis}

    \xiaoxiaoti{$\sin 103^\circ 15'$,$\sin 164^\circ 30'$;}

    \jiange
    \xiaoxiaoti{$\cos \left( -\dfrac{47}{10} \pi \right)$,$\cos \left( -\dfrac{44}{9} \pi \right)$;}
    \jiange

    \xiaoxiaoti{$\sin 508^\circ$,$\sin 144^\circ$;}

    \xiaoxiaoti{$\cos 760^\circ$,$\cos(-770^\circ)$。}

\end{xiaoxiaotis}

\xiaoti{指出下列函数的单调区间:}
\begin{xiaoxiaotis}

    \twoInLineXxt[14em]{$y = 1 + \sin x$;}{$y = -\cos x$。}

\end{xiaoxiaotis}

\xiaoti{根据三角函数的图象,写出使下列不等式成立的 $x$ 的集合:}
\begin{xiaoxiaotis}

    \jiange
    \twoInLineXxt[14em]{$\sin x \geqslant \dfrac{\sqrt{3}}{2}$;}{$\sqrt{2} + 2\cos x \geqslant 0$。}
    \jiange

\end{xiaoxiaotis}

\xiaoti{证明:\textbf{两个三角形,如果有两边对应相等而夹角不等,那么,夹角所对的边也不等,夹角大的所对的边较大}
    (提示:利用余弦定理以及余弦函数在 $[0, \pi]$ 上是减函数这一性质)。}

\xiaoti{确定下列各函数的定义域:}
\begin{xiaoxiaotis}

    \renewcommand\arraystretch{1.5}
    \begin{tabular}[t]{*{2}{@{}p{14em}}}
        \xiaoxiaoti {$y = \dfrac{1}{1 + \sin x}$;} & \xiaoxiaoti {$y = \dfrac{1}{1 - \cos x}$;} \\
        \xiaoxiaoti {$y = \sqrt{\cos x}$;} & \xiaoxiaoti {$y = \sqrt{-2\sin x}$。}
    \end{tabular}

\end{xiaoxiaotis}

\xiaoti{作出下列函数在长度为一个周期的闭区间上的简图:}
\begin{xiaoxiaotis}

    \renewcommand\arraystretch{1.5}
    \begin{tabular}[t]{*{2}{@{}p{14em}}}
        \xiaoxiaoti {$y = 4\sin 2x$;} & \xiaoxiaoti {$y = \dfrac 1 2 \cos 3x$;} \\
        \xiaoxiaoti {$y = 3\sin \left( 2x - \dfrac \pi 6 \right) $;} & \xiaoxiaoti {$y = 2\cos \left( \dfrac 1 2 x + \dfrac \pi 4 \right)$。}
    \end{tabular}

\end{xiaoxiaotis}

\jiange
\xiaoti{作函数 $y = \sin \left( x + \dfrac \pi 2 \right)$ 的图象,把它与余弦曲线 $y = \cos x$ 进行比较,能得出什么结论?}
\jiange

\xiaoti{作函数 $y = -\cos \left( x + \dfrac \pi 2 \right)$ 的图象,把它与正弦曲线 $y = \sin x$ 进行比较,能得出什么结论?}
\jiange

\xiaoti{不画图,写出下列各函数的振幅、周期和初相,并说明这些函数的图象可由正弦曲线 $y = \sin x$ 经过怎样的变化得出:}
\begin{xiaoxiaotis}

    \jiange
    \twoInLineXxt[14em]{$y = 8\sin \left( \dfrac 1 4 x - \dfrac \pi 8 \right)$;}{$y = \dfrac 1 3 \sin \left( 3x + \dfrac \pi 7 \right)$。}
    \jiange

\end{xiaoxiaotis}

\xiaoti{电流强度 $I$ 随时间 $t$ 变化的函数关系是 $I = A\sin \omega t$。设 $\omega = 100\pi$(弧度/秒),$A = 5$(安培)。}
\begin{xiaoxiaotis}

    \xiaoxiaoti{求电流强度 $I$ 变化的周期与频率;}

    \jiange
    \xiaoxiaoti{当 $t = 0$,$\dfrac{1}{200}$,$\dfrac{1}{100}$,$\dfrac{3}{200}$,$\dfrac{1}{50}$(秒)时,求电流强度 $I$;}
    \jiange

    \xiaoxiaoti{画出电流强度 $I$ 随时间 $t$ 变化的函数的图象 \jiange
        (以 $I$ 为纵坐棕,$0.5cm$ 表示 $1$ 安培;以 $t$ 为横坐标,$1cm$ 表示 $\dfrac{1}{200}$ 秒)。}
    \jiange

\end{xiaoxiaotis}

\xiaoti{一根长 $l$ 厘米的线,一端固定,另一端悬挂一个小球。小球摆动时,离开平衡位置的位移 $S$(厘米)和时间 $t$(秒)的函数关系是
    $$S = 3\cos \left( \sqrt{\dfrac g l} t + \dfrac \pi 3 \right) \text{。}$$}
\begin{xiaoxiaotis}

    \xiaoxiaoti{求小球摆动的周期;}

    \xiaoxiaoti{巳知 $g = 980 \text{厘米}/\text{秒}^2$,要使小球摆动的周期是 $1$ 秒,线的长应当是多少厘米(确到 $0.1$ 厘米,$\pi$ 取 $3.14$)?}

\end{xiaoxiaotis}

\xiaoti{不通过求值,比较下列各组中两个三角函数值的大小:}
\begin{xiaoxiaotis}

    \jiange
    \xiaoxiaoti{$\tan \left( -\dfrac 1 5 \pi \right)$,$\tan \left( -\dfrac 3 7 \pi \right)$;}
    \jiange

    \xiaoxiaoti{$\cot 1519^\circ$,$\cot 1493^\circ$;}

    \jiange
    \xiaoxiaoti{$\tan 6\dfrac{9}{11} \pi$,$\tan \left( -5\dfrac{3}{11} \pi \right)$;}
    \jiange

    \xiaoxiaoti{$\tan \dfrac 7 8 \pi$,$\tan \dfrac{\pi}{16}$。}

\end{xiaoxiaotis}

\jiange
\xiaoti{作函数 $y = -\tan \left( x + \dfrac \pi 2 \right)$ \jiange 的图象,把它与余切曲线 $y = \cot x$ 进行比较,能得出什么结论?}

\xiaoti{求下列函数的定义域:}
\begin{xiaoxiaotis}

    \jiange
    \xiaoxiaoti{$y = \cot \left( x + \dfrac \pi 3 \right)$;}
    \jiange

    \xiaoxiaoti{$y = -\tan \left( x + \dfrac \pi 6 \right) + 2$。}
    \jiange

\end{xiaoxiaotis}

\xiaoti{求下列函数的周期:}
\begin{xiaoxiaotis}

    \jiange
    \xiaoxiaoti{$y = \tan \left( 2x - \dfrac \pi 4 \right)$;}
    \jiange

    \xiaoxiaoti{$y = 2\cot \dfrac x 3$。}
    \jiange

\end{xiaoxiaotis}

\xiaoti{下列函数是奇函数还是偶函数?为什么?}
\begin{xiaoxiaotis}

    \twoInLineXxt[14em]{$y = -\tan x$;}{$y = -|\cot x|$。}

\end{xiaoxiaotis}

\xiaoti{根据三角函数的图象,写出使下列不等式成立的 $x$ 的集合:}
\begin{xiaoxiaotis}

    \twoInLineXxt[14em]{$1 + \tan x \geqslant 0$;}{$\cot x - \sqrt{3} \geqslant 0$。}

\end{xiaoxiaotis}

\end{xiaotis}
