\subsection{幂函数}

我们已经学过函数 $y=x$,$y=x^2$ 及 $y=x^{-1}$ ,这些函数都是幂函数。

一般地,函数 $y=x^\alpha$ 叫做\textbf{幂函数},其中 $x$ 是自变量,$\alpha$ 是常数
(这里我们只讨论 $\alpha$ 是有理数 $n$ 的情况)。

我们知道,当 $n=0$ 时,$x^n = x^0 = 1\, (x \neq 0)$,函数 $y = x^0$ 成为常数函数 $y=1\, (x \neq 0)$,
它的图象是平行于 $x$ 轴并在 $x$ 轴上方 1 个单位的一条直线(除去点$(0, 1)$)。

当 $n=1$ 时,函数 $y=x^n$ 就是 $y=x$。$n$ 是其他正整数时,$x^n$ 的意义是 $x^n = x \cdot x \cdot \dots \cdot x$ 
(共 $n$ 个 $x$ 相乘)。函数的定义域是实数集 $R$。

$n$ 是一个正分数时,我们只研究 $n$ 是一个既约分数 $\dfrac{p}{q}$ 的情况($p$,$q$ 是正整数,$q>1$)。
这时,$x^n$ 的意义是 $x^\frac{p}{q} = \sqrt[q]{x^p}$,函数 $y = x^\frac{p}{q}$ 的
定义域是使 $\sqrt[q]{x^p}$ 有意义的实数 $x$ 的集合。

$n$ 是一个负整数或负分数时,例如,$n = -p$($p$ 是正整数)或 $n = -\dfrac{p}{q}$
($p$,$q$ 是素质的正整数,$q>1$)时,$x^n$ 的意义分别是 $x^n = x^{-p} = \dfrac{1}{x^p}$,
$x^n = x^{-\frac{p}{q}} = \dfrac{1}{\sqrt[q]{x^p}}$,函数的定义域是使 $\dfrac{1}{x^p}$
或 $\dfrac{1}{\sqrt[q]{x^p}}$ 有意义的实数 $x$ 的集合。

\liti 求下列幂函数的定义域:

\hspace{2em} $y = x^3$,$y = x^{\frac 1 3}$,$y = x^{\frac 1 2}$,$y = x^{-2}$,$y = x^{-\frac 1 2}$。

\jie $y = x^3$ 的定义域是 $R$,

$y = x^{\frac 1 3} = \sqrt[3]{x}$ 的定义域是 $R$,

$y = x^{\frac 1 2} = \sqrt{x}$ 的定义域是 $[0, +\infty)$,

\vspace{0.5em}
$y = x^{-2} = \dfrac{1}{x^2}$ 的定义域是 $\{x \mid x \in R \text{,且} x \neq 0\}$,

\vspace{0.5em}
$y = x^{-\frac 1 2} = \dfrac{1}{\sqrt{x}}$ 的定义域是 $(0, +\infty)$。

现在我们分 $x>0$ 及 $n<0$ 两种情况来研究幂函数的图象和性质。

\subsubsection{$n>0$。}

我们知道,$y=x$ 的图象是直线(图\ref{fig:1-8}),$y=x^2$ 的图象是抛物线(图\ref{fig:1-9})。

\begin{figure}[htbp]
    \centering
    \begin{minipage}{7cm}
    \centering
    \begin{tikzpicture}[>=Stealth]
        \draw [->] (-2.5,0) -- (2.5,0) node[anchor=north] {$x$};
        \draw [->] (0,-2.5) -- (0,2.5) node[anchor=east] {$y$};
        \node at (0.2,-0.2) {$O$};
        \foreach \x in {-4,-2,1,2,3,4} {
            \draw (\x/2,0.2) -- (\x/2,0) node[anchor=north] {$\x$};
        }
        \foreach \x in {-3,-1} {
            \draw (\x/2,0) -- (\x/2,0.2) (\x/2-0.1, 0.2) node[anchor=south] {$\x$};
        }
        \foreach \y in {1,...,4} {
            \draw (0.2,\y/2) -- (0,\y/2) node[anchor=east] {$\y$};
        }
        \foreach \y in {-4,...,-1} {
            \draw (0,\y/2) -- (0.2,\y/2) node[anchor=west] {$\y$};
        }
        \draw (-2,-2) -- (2,2) (2,1.2) node{$y=x$};
    \end{tikzpicture}
    \caption{}\label{fig:1-8}
    \end{minipage}
    \qquad
    \begin{minipage}{8cm}
    \centering
    \begin{tikzpicture}[>=Stealth,scale=0.8]
        \draw [->] (-3.5,0) -- (3.5,0) node[anchor=north] {$x$};
        \draw [->] (0,-1) -- (0,5.5) node[anchor=east] {$y$};
        \foreach \x in {-3,-2,-1,1,2,3}
            \draw (\x,0.2) -- (\x,0) node[anchor=north] {$\x$};
        \foreach \y in {1,...,5}
            \draw (0,\y) -- (0.2,\y) node[anchor=west] {$\y$};
        \node at (0.3,-0.3) {$O$};
        \draw  (-2.2,4.84) parabola bend (0,0) (2.2,4.84);
        \node at (2.5,3) {$y=x^2$};
    \end{tikzpicture}
    \caption{}\label{fig:1-9}
    \end{minipage}
\end{figure}

现在画函数 $y=x^3$,$y=x^{\frac 1 2}$ 及 $y=x^{\frac 1 3}$ 的图象。

分别列出 $x$,$y$ 的对应值表,用描点的方法,画出这三个函数的图象(图\ref{fig:1-10},图\ref{fig:1-11}及图\ref{fig:1-12})。

$y=x^3$

\begin{table}[ht]
\begin{tabular}{|c|*{9}{w{c}{3em}|}}
    \hline
    x & $\dots$ & $-1.5$ & $-1$ & $-0.5$ & $0$ & $0.5$ & $1$ & $1.5$ & $\dots$ \\
    \hline
    $y=x^3$ & $\dots$ & $-3.38$ & $-1$ & $-0.13$ & $0$ & $0.13$ & $1$ & $3.38$ & $\dots$\\
    \hline
\end{tabular}
\end{table}

$y=x^{\frac 1 2}$

\begin{table}[ht]
\begin{tabular}{|c|*{8}{w{c}{3em}|}}
    \hline
    x & $0$ & $0.5$ & $1$ & $2$ & $3$ & $4$ & $6$ & $\dots$ \\
    \hline
    $y=x^{\frac 1 2}$ & $0$ & $0.71$ & $1$ & $1.41$ & $1.73$ & $2$ & $2.45$ & $\dots$\\
    \hline
\end{tabular}
\end{table}

$y=x^{\frac 1 3}$

\begin{table}[ht]
\begin{tabular}{|c|*{9}{w{c}{3em}|}}
    \hline
    x & $\dots$ & $-3$ & $-2$ & $-1$ & $0$ & $1$ & $2$ & $3$ & $\dots$ \\
    \hline
    $y=x^{\frac 1 3}$ & $\dots$ & $-1.44$ & $-1。26$ & $-1$ & $0$ & $1$ & $1。26$ & $1.44$ & $\dots$\\
    \hline
\end{tabular}
\end{table}

\begin{figure}[H]
    \centering
    \begin{minipage}{5cm}
    \centering
    \begin{tikzpicture}[>=Stealth,scale=0.8]
        \draw [->] (-2.5,0) -- (2.5,0) node[anchor=north] {$x$};
        \draw [->] (0,-4.5) -- (0,4.5) node[anchor=east] {$y$};
        \node at (0.3,-0.3) {$O$};
        \foreach \x in {-2,-1,1,2} {
            \draw (\x,0.2) -- (\x,0) node[anchor=north] {$\x$};
        }
        \foreach \y in {1,...,4} {
            \draw (0.2,\y) -- (0,\y) node[anchor=east] {$\y$};
        }
        \foreach \y in {-4,...,-1} {
            \draw (0,\y) -- (0.2,\y) node[anchor=west] {$\y$};
        }
        \draw [domain=-1.6:1.6] plot (\x, {\x * \x * \x});
        \node at (2.3,3) {$y = x^3$};
    \end{tikzpicture}
    \caption{}\label{fig:1-10}
    \end{minipage}
    \qquad
    \begin{minipage}{8cm}
    \centering

    \begin{tikzpicture}[>=Stealth,scale=0.8]
        \draw [->] (-1,0) -- (6.5,0) node[anchor=north] {$x$};
        \draw [->] (0,-1) -- (0,4.5) node[anchor=east] {$y$};
        \foreach \x in {1,...,6}
            \draw (\x,0.2) -- (\x,0) node[anchor=north] {$\x$};
        \foreach \y in {1,...,5}
            \draw (0,\y) -- (0.2,\y) node[anchor=west] {$\y$};
        \node at (-0.3,-0.3) {$O$};
        % 前一部分变化较大,需要多一些插值点。所以分成两段绘制
        % 也可以增大 samples 个数: \draw [domain=0:6,samples=200] plot (\x, {\x ^ (1/2)});
        \draw [domain=0:0.2] plot (\x, {\x ^ (1/2)});
        \draw [domain=0.2:6] plot (\x, {\x ^ (1/2)});
        \node at (2.3,2) {$y = x^{\frac 1 2}$};
    \end{tikzpicture}
    \caption{}\label{fig:1-11}
    \end{minipage}
\end{figure}

\begin{figure}[H]
    \centering
    \begin{tikzpicture}[>=Stealth,scale=0.8]
        \draw [->] (-3.5,0) -- (3.5,0) node[anchor=north] {$x$};
        \draw [->] (0,-2.5) -- (0,2.5) node[anchor=east] {$y$};
        \foreach \x in {-3,-2,-1,1,2,3}
            \draw (\x,0.2) -- (\x,0) node[anchor=north] {$\x$};
        \foreach \y in {-2,-1,1,2}
            \draw (0,\y) -- (0.2,\y) node[anchor=west] {$\y$};
        \node at (0.3,-0.3) {$O$};
        \draw [domain=-3:3,samples=200] plot (\x, {\x ^ (1/3)});
        \node at (2.5,2) {$y=x^{\frac 1 3}$};
        %\draw[help lines] (0,0) grid [step=0.5mm] (0.1,0.6);
    \end{tikzpicture}
    \caption{}\label{fig:1-12}
\end{figure}

进一步研究可以看出:幂函数 $y=x^4$,$y=x^6$,……的图象类似于 $y=x^2$ 的图象;
$y=x^5$,$y=x^7$,……的图象类似于 $y=x^3$ 的图象;
$y=x^{\frac 1 4}$,$y=x^{\frac 1 6}$,……的图象类似于 $y=x^{\frac 1 2}$ 的图象;
$y=x^{\frac 1 5}$,$y=x^{\frac 1 7}$,……的图象类似于 $y=x^{\frac 1 3}$ 的图象。

下面在同一坐标系内画出幂函数 $y=x$,$y=x^2$,$y=x^3$,$y=x^{\frac 1 2}$,$y=x^{\frac 1 3}$ 
的图象(图\ref{fig:1-13}),我们可以看出,\textbf{当 $n>0$ 时,幂函数 $y=x^n$ 有下列性质:}

\textbf{(1)图象都通过点 $(0,0)$,$(1,1)$;}

\textbf{(2)在第一象限内,函数值随 $x$ 的增大而增大。}

\begin{figure}[H]
    \centering
    \begin{tikzpicture}[>=Stealth,scale=0.8]
        \draw [->] (-4.5,0) -- (4.5,0) node[anchor=north] {$x$};
        \draw [->] (0,-4.5) -- (0,4.5) node[anchor=east] {$y$};
        \node at (0.3,-0.3) {$O$};
        
        \draw [dashed] (1,1) -- (1,0) node[anchor=north] {$1$};
        \draw [dashed] (1,1) -- (0,1) node[anchor=east] {$1$};
        \draw [dashed] (-1,-1) -- (-1,0) node[anchor=south] {$-1$};
        \draw [dashed] (-1,-1) -- (0,-1) node[anchor=west] {$-1$};
    
        \draw [dash dot dot,domain=-1.6:1.6] plot (\x, {\x * \x * \x}) (0.8,4) node {$y = x^3$};
        \draw [dashed,domain=-2:2] plot (\x, {abs(\x) ^ 2}) (2.8,4) node {$y=x^2$};
        \draw (-4,-4) -- (4,4) (4.7,4) node{$y=x$};
        \draw [red,domain=0:4,samples=200] plot (\x, {\x ^ (1/2)}) (4.1,2.6) node {$y = x^{\frac 1 2}$};
        \draw [blue,semithick,domain=-4:4,samples=200] plot (\x, {\x ^ (1/3)}) (4.1,1.3) node {$y=x^{\frac 1 3}$};
    \end{tikzpicture}
    \caption{}\label{fig:1-13}
\end{figure}

\liti 比较下列各题 中两个值的大小:

\twoInLine[10em]{(1) $1.5^{\frac 3 5}$,$1.7^{\frac 3 5}$}
                {(2) $0.7^{1.5}$,$0.6^{1.5}$。}

\jie (1)各题中两个值都是幂运算的结果,且指数相同,因此,可以利用幂函数的性质来判断它们
的大小,考察函数 $y= x^{\frac 3 5}$,在第一象限内,$y$ 的值随 $x$ 的增大而增大。

$\because 1.5 < 1.7$,

$\therefore 1.5^{\frac 3 5} < 1.7^{\frac 3 5}$。

(2)考察幂函数 $y=x^{1.5}$,同理,

$\because 0.7 > 0.6$,

$\therefore 0.7^{1.5} > 0.6^{1.5}$。

\subsubsection{$n<0$。}

我们知道,幂函数 $y=x^{-1}$ 的图象,即反比例函数 $y = \dfrac 1 x$ 的图象,是两支曲线(图\ref{fig:1-14})。

\begin{figure}[ht]
    \centering
    \begin{tikzpicture}[>=Stealth,scale=0.8]
        \draw [->] (-4.5,0) -- (4.5,0) node[anchor=north] {$x$};
        \draw [->] (0,-4.5) -- (0,4.5) node[anchor=east] {$y$};
        \node at (0.3,-0.3) {$O$};
        \foreach \x in {1,...,4} {
            \draw (\x,0.2) -- (\x,0) node[anchor=north] {$\x$};
        }
        \foreach \x in {-4,...,-1} {
            \draw (\x,0) -- (\x,0.2) (\x-0.1, 0.2) node[anchor=south] {$\x$};
        }
        \foreach \y in {1,...,4} {
            \draw (0.2,\y) -- (0,\y) node[anchor=east] {$\y$};
        }
        \foreach \y in {-4,...,-1} {
            \draw (0,\y) -- (0.2,\y) node[anchor=west] {$\y$};
        }
        \draw [domain=0.25:4] plot (\x, {1 / \x}) (2,3) node {$y=x^{-1}$};
        \draw [domain=-4:-0.25] plot (\x, {1 / \x});
    \end{tikzpicture}
    \caption{}\label{fig:1-14}
\end{figure}

现在画函数 $y = x^{-2}$,$y = x^{-\frac 1 2}$ 的图象。

分别列出 $x$,$y$ 的对应值表,用描点的方法画出这两个函数的图象(图\ref{fig:1-15},图\ref{fig:1-16})。

\vspace{0.5em}
$y = x^{-2} = \dfrac 1 {x^2}$

\begin{table}[h]
\renewcommand\arraystretch{2}
\begin{tabular}{|c|*{10}{w{c}{2em}|}}
    \hline
    x & $\dots$ & $-3$ & $-2$ & $-1$ & $-\dfrac 1 2$ & $\dfrac 1 2$ & $1$ & $2$ & $3$ & $\dots$ \\
    \hline
    $y = x^{-2}$ & $\dots$ & $\dfrac 1 9$ & $\dfrac 1 4$ & $1$ & $4$ & $4$ & $1$ & $\dfrac 1 4$ & $\dfrac 1 9$ & $\dots$ \\
    \hline
\end{tabular}
\end{table}

\vspace{0.5em}
$y = x^{-\frac 1 2} = \dfrac 1 {\sqrt{x}}$

\begin{table}[h]
\renewcommand\arraystretch{2}
\begin{tabular}{|c|*{7}{w{c}{2em}|}}
    \hline
    x & $\dots$ & $\dfrac 1 {16}$ & $\dfrac 1 9$ & $\dfrac 1 4$ & $1$ & $4$ & $\dots$ \\
    \hline
    $y = x^{-\frac 1 2}$ & $\dots$ & $4$ & $3$ & $2$ & $1$ & $\dfrac 1 2$ & $\dots$ \\
    \hline
\end{tabular}
\end{table}

\begin{figure}[H]
    \centering
    \begin{minipage}{8cm}
    \centering
    \begin{tikzpicture}[>=Stealth,scale=0.8]
        \draw [->] (-4.5,0) -- (4.5,0) node[anchor=north] {$x$};
        \draw [->] (0,-1.5) -- (0,4.5) node[anchor=east] {$y$};
        \node at (-0.3,-0.3) {$O$};
        \foreach \x in {1,...,4} {
            \draw (\x,0.2) -- (\x,0) node[anchor=north] {$\x$};
        }
        \foreach \x in {-4,...,-1} {
            \draw (\x,0.2) -- (\x,0) node[anchor=north] {$\x$};
        }
        \foreach \y in {1,...,4} {
            \draw (0.2,\y) -- (0,\y) node[anchor=east] {$\y$};
        }
        \foreach \y in {-1} {
            \draw (0,\y) -- (0.2,\y) node[anchor=west] {$\y$};
        }
        \draw [domain=0.5:4] plot (\x, {\x^(-2)}) (2,3) node {$y=x^{-2}$};
        \draw [domain=-4:-0.5] plot (\x, {abs(\x)^(-2)});
    \end{tikzpicture}
    \caption{}\label{fig:1-15}
    \end{minipage}
    \qquad
    \begin{minipage}{8cm}
    \centering
    \begin{tikzpicture}[>=Stealth,scale=0.8]
        \draw [->] (-1.5,0) -- (5.5,0) node[anchor=north] {$x$};
        \draw [->] (0,-1.5) -- (0,4.5) node[anchor=east] {$y$};
        \node at (-0.3,-0.3) {$O$};
        \foreach \x in {-1,1,2,...,5} {
            \draw (\x,0.2) -- (\x,0) node[anchor=north] {$\x$};
        }
        \foreach \y in {1,...,4} {
            \draw (0.2,\y) -- (0,\y) node[anchor=east] {$\y$};
        }
        \foreach \y in {-1} {
            \draw (0,\y) -- (0.2,\y) node[anchor=west] {$\y$};
        }
        \draw [domain=0.07:5,samples=100] plot (\x, {1 /sqrt(\x)}) (2,3) node {$y=x^{-\frac 1 2}$};
    \end{tikzpicture}
    \caption{}\label{fig:1-16}
    \end{minipage}
\end{figure}

在同一坐标系内画出 $y = x^{-1}$,$y = x^{-2}$,$y = x^{-\dfrac 1 2}$ 的图象(图\ref{fig:1-17}),
我们可以看出,\textbf{当 $n<0$ 时,幂函数 $y = x^n$ 有下列性质:}\label{xingzhi:mihanshu-2}

(1)图象都通过点 $(1,1)$;

(2)在第一象限内,函数值随着 $x$ 的增大而减小;

(3)在第一象限内,图象向上与 $y$ 轴无限地接近,向右与 $x$ 轴无限地接近。

\begin{figure}[H]
    \centering
    \begin{tikzpicture}[>=Stealth,scale=0.8]
        \draw [->] (-4.5,0) -- (4.5,0) node[anchor=north] {$x$};
        \draw [->] (0,-4.5) -- (0,4.5) node[anchor=east] {$y$};
        \node at (0.3,-0.3) {$O$};
        \foreach \x in {1,...,4} {
            \draw (\x,0.2) -- (\x,0) node[anchor=north] {$\x$};
        }
        \foreach \x in {-4,...,-1} {
            \draw (\x,0) -- (\x,0.2) (\x-0.1, 0.2) node[anchor=south] {$\x$};
        }
        \foreach \y in {1,...,4} {
            \draw (0.2,\y) -- (0,\y) node[anchor=east] {$\y$};
        }
        \foreach \y in {-4,...,-1} {
            \draw (0,\y) -- (0.2,\y) node[anchor=west] {$\y$};
        }
        \node at (1.5,1.3) {$(1,1)$};
        
        \path[name path=a1,draw, domain=0.25:4] plot (\x, {1 / \x}) (1.8,3.5) node {$y=x^{-1}$};
        \draw [domain=-4:-0.25] plot (\x, {1 / \x}) (-2, -2) node {$y=x^{-1}$};
        \path [name path=a2] (0.35,0) -- (0.35,4);
        \draw [name intersections={of=a1 and a2, by=x}] (x)-- (1.0,3.3);
    
        \draw [name path=b1,domain=0.5:4] plot (\x, {\x^(-2)}) (2,2.5) node {$y=x^{-2}$};
        \draw [domain=-4:-0.5] plot (\x, {abs(\x)^(-2)}) (-2,2.5) node {$y=x^{-2}$};
        \path [name path=b2] (0.7,0) -- (0.7,4);
        \draw [name intersections={of=b1 and b2, by=x}] (x)-- (1.15,2.3);
    
        \draw [name path=c1,domain=0.07:5,samples=100] plot (\x, {1 /sqrt(\x)}) (3.5,1.5) node {$y=x^{-\frac 1 2}$};
        \path [name path=c2] (2.5,0) -- (2.5,4);
        \draw [name intersections={of=c1 and c2, by=x}] (x)-- (3.15,1.3);
    \end{tikzpicture}
    \caption{}\label{fig:1-17}
\end{figure}

\liti 比较下列各题中两个值的大小:

\twoInLine[10em]{(1) $2.2^{-\frac 2 3}$,$1.8^{-\frac 2 3}$}
                {(2) $0.15^{-1.2}$,$0.17^{-1.2}$。}

\jie (1)考察幂函数 $y = x^{-\frac 2 3}$,在第一象限内,$y$ 的值随 $x$ 的增大而减小。

$\because 2.2 > 1.8$,

$\therefore 2.2^{-\frac 2 3} < 1.8^{-\frac 2 3}$。

(2)考察幂函数 $y = x^{-1.2}$,同理,

$\because 0.15 < 0.17$,

$\therefore 0.15^{-1.2} > 0.17^{-1.2}$。

\lianxi

\begin{xiaotis}

\xiaoti{求下列函数的定义域:}

\begin{xiaoxiaotis}
    \begin{tabular}[t]{*{2}{@{}p{14em}}} 
        \xiaoxiaoti {$y = x^{-4}$;} & \xiaoxiaoti {$y = x^{\frac 1 5}$;} \\
        \xiaoxiaoti {$y = x^{-\frac 3 2}$;} & \xiaoxiaoti {$y = x^{\frac 2 3}$;} \\
        \xiaoxiaoti {$y = x^{\frac 3 2}$;} & \xiaoxiaoti {$y = x^{-\frac 4 5}$。}
    \end{tabular}
\end{xiaoxiaotis}


\xiaoti{画出函数 $y = x^{\frac 2 3}$ 的图象。}

\xiaoti{在同一坐标第内画出下列各题中两个函数的图象,并加以比较:}
\begin{xiaoxiaotis}

    \twoInLine[14em]{\xiaoxiaoti{$y=x^3$,$y=x^4$;}}
        {\xiaoxiaoti{$y=x^{-3}$},$y=x^{-4}$。}
\end{xiaoxiaotis}

\xiaoti{比较下列各题中两个值的大小:}
\begin{xiaoxiaotis}

    \twoInLine[14em]{\xiaoxiaoti{$1.3^{\frac 3 4}$,$1.5^{\frac 3 4}$;}}
        {\xiaoxiaoti{$0.21^{\frac 2 5}$},$0.27^{\frac 2 5}$。}
\end{xiaoxiaotis}

\xiaoti{比较下列各题中两个值的大小:}
\begin{xiaoxiaotis}

    \twoInLine[14em]{\xiaoxiaoti{$3^{-\frac 5 2}$,$3.1^{-\frac 5 2}$;}}
        {\xiaoxiaoti{$1.1^{-\frac 1 2}$},$0.9^{-\frac 1 2}$。}
\end{xiaoxiaotis}

\end{xiaotis}