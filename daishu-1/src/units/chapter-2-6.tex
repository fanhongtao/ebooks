\subsection{已知三角函数值求角}\label{subsec:2-6}

已知任意一个角(角必须属于这个函数的定义域),可以求出它的三角函数值;
反过来,如果已知一个三角函数值,也可以求出它对应的角。

\vspace{0.5em}
\liti 已知 $\sin\alpha = \dfrac{\sqrt{2}}{2}$,且 $0 \leqslant \alpha < 2\pi$,求 $\alpha$。
\vspace{0.5em}

\jie 因为 $\sin\alpha = \dfrac{\sqrt{2}}{2} > 0$,所以 $\alpha$ 是第一、二象限的角。由
$$\sin \dfrac \pi 4 = \dfrac{\sqrt{2}}{2}$$
知道,符合条件的第一象限的角是 $\dfrac \pi 4$。又由
$$\sin \left( \pi - \dfrac \pi 4 \right) = \sin \dfrac \pi 4 = \dfrac{\sqrt{2}}{2} \vspace{0.5em}$$
知道,符合条件的第二象限的角是 $\pi - \dfrac \pi 4$,即 $\dfrac{3\pi}{4}$。于是所求的 $\alpha$ 是 $\dfrac \pi 4$ 或 $\dfrac{3\pi}{4}$。

\vspace{0.5em}
也可以说,所求的 $\alpha$ 的集合是 $\left\{ \dfrac \pi 4, \dfrac{3\pi}{4} \right\}$。
\vspace{0.5em}

\liti 已知 $\cos\alpha = -0.7660$,且 $0^\circ \leqslant \alpha < 360^\circ$,求 $\alpha$。

\jie 因为 $\cos\alpha = -0.7660 < 0$,所以 $\alpha$ 是第二、三象限的角。

先求符合下面条件的锐角 $\theta$:
$$\cos\theta = 0.7660 \text{,}$$
查表得 \qquad $\theta = 40^\circ$。

由 $\cos(180^\circ - 40^\circ) = -\cos 40^\circ = -0.7660$ 知道,符合条件的第二象限的角是 $180^\circ - 40^\circ$,即 $140^\circ$。

又由 $\cos(180^\circ + 40^\circ) = -\cos 40^\circ = -0.7660$ 知道,符合条件的第三象限的角是 $180^\circ + 40^\circ$,即 $220^\circ$。

因此,所求的 $\alpha$ 是 $140^\circ$ 或 $220^\circ$。

\liti 已知 $\sin x = -0.3322$,求 $x$。

\jie 因为 $\sin x = -0.3322 < 0$,所以 $x$ 是第三、四象限的角。

先求符合下面条件的锐角 $\theta$:
$$\sin\theta = 0.3322 \text{,}$$
查表得
$$\theta = 19^\circ 24' \text{,}$$
因为
$$\begin{gathered}
    \sin(180^\circ + 19^\circ 24') = -\sin 19^\circ 24' = -0.3322, \\
    \sin(360^\circ - 19^\circ 24') = -\sin 19^\circ 24' = -0.3322,
\end{gathered}$$
所以,在 $0^\circ$ \~{} $360^\circ$ 间,符合条件的第三、四象限的角分别是 $199^\circ 24'$,
$340^\circ 36'$。由 \hyperref[gongshi:1]{公式一} 知道,与 $199^\circ 24'$,$340^\circ 36'$
有相同终边的角的正弦都等于 $-0.3322$。所以,所求的 $x$ 是
$$k \cdot 360^\circ + 199^\circ 24' \quad \text{或} \quad k \cdot 360^\circ + 340^\circ 36', \, k\in Z \text{。} $$

\vspace{0.5em}
\liti 已知 $\tan x = \dfrac 1 3$,求 $x$ 的集合。
\vspace{0.5em}

\jie 因为 $\tan x = \dfrac 1 3 > 0$,所以 $x$ 是第一、三象限的角。

查表得
$$\tan 18^\circ 26' = \dfrac 1 3 \text{,}$$
又
$$\tan(180^\circ + 18^\circ 26') = \tan 18^\circ 26' = \dfrac 1 3 \text{,}$$
因此所求的 $x$ 是
$$k \cdot 360^\circ + 18^\circ 26'$$
或
$$k \cdot 360^\circ + (180^\circ + 18^\circ 26') \, (k \in Z) \text{。}$$

因为
\begin{align}
    &k \cdot 360^\circ + 18^\circ 26' = 2k \cdot 180^\circ + 18^\circ 26', \tag{1}\label{eq:2-6-1} \\
    &k \cdot 360^\circ + 180^\circ + 18^\circ 26' =  (2k + 1) \cdot 180^\circ + 18^\circ 26', \tag{2}\label{eq:2-6-2}
\end{align}

所以把 \eqref{eq:2-6-1},\eqref{eq:2-6-2} 合并,所求的 $x$ 就是
$$n \cdot 180^\circ + 18^\circ 26', \, n \in Z \text{。}$$

因此,所求的 $x$ 的集合是
$$\{ x \mid x = n \cdot 180^\circ + 18^\circ 26', \, n \in Z \} \text{。}$$

\lianxi
\begin{xiaotis}

\xiaoti{求适合下列条件的 $\alpha$:}
\begin{xiaoxiaotis}

    \vspace{0.5em}
    \xiaoxiaoti{$\cos\alpha = \dfrac{\sqrt{3}}{2}$,且 $\dfrac{3\pi}{2} < \alpha < 2\pi$;}
    \vspace{0.5em}

    \xiaoxiaoti{$\sin\alpha = -\dfrac{1}{2}$,且 $\pi < \alpha < \dfrac{3\pi}{2}$;}
    \vspace{0.5em}

    \xiaoxiaoti{$\cot\alpha = 1$,且 $\pi < \alpha < \dfrac{3\pi}{2}$;}
    \vspace{0.5em}

    \xiaoxiaoti{$\sin\alpha = -0.8572$,且 $0^\circ < \alpha < 360^\circ$;}

    \xiaoxiaoti{$\cos\alpha = -0.4099$,且 $0^\circ < \alpha < 360^\circ$;}

    \xiaoxiaoti{$\tan\alpha = -4$,且 $0^\circ < \alpha < 360^\circ$。}

\end{xiaoxiaotis}

\xiaoti{求适合下列条件的 $x$:}
\begin{xiaoxiaotis}

    \xiaoxiaoti{$\sin x = 0.3469$,且 $x$ 在第一象限;}

    \vspace{0.5em}
    \xiaoxiaoti{$\cot x = -\dfrac{\sqrt{3}}{3}$,且 $x$ 在第二象限;}
    \vspace{0.5em}

    \xiaoxiaoti{$\cos x = \dfrac{4}{5}$,且 $x$ 在第四象限;}
    \vspace{0.5em}

    \xiaoxiaoti{$\tan x = 4.653$,且 $x$ 在第三象限。}

\end{xiaoxiaotis}

\xiaoti{求适合下列条件的 $x$ 的集合:}
\begin{xiaoxiaotis}

    \renewcommand\arraystretch{2}
    \begin{tabular}[t]{*{2}{@{}p{15em}}}
        \xiaoxiaoti{$\cos x = -\dfrac{\sqrt{3}}{2}$;} & \xiaoxiaoti{$\cot x = \dfrac{\sqrt{3}}{3}$;} \\
        \xiaoxiaoti{$\sin x = 0.7662$;} & \xiaoxiaoti{$\tan x = -29.12$。}
    \end{tabular}

\end{xiaoxiaotis}

\end{xiaotis}
