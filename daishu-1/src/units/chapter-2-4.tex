\subsection{同角三角函数的基本关系式}\label{subsec:2-4}

根据三角函数的定义,可以得到同角三角函数间的下列基本关系式:

\begin{table}[h]
\centering
\renewcommand\arraystretch{1.5}
\begin{tabular}{c}
    $\sin \alpha \cdot \csc \alpha = \dfrac y r \cdot \dfrac r y = 1 \text{。}$ \\
    $\cos \alpha \cdot \sec \alpha = \dfrac x r \cdot \dfrac r x = 1 \text{。}$ \\
    $\tan \alpha \cdot \cot \alpha = \dfrac y x \cdot \dfrac x y = 1 \text{。}$ \\
    \rule{0pt}{3em}$\tan \alpha = \dfrac y x = \dfrac{\; \dfrac y r \;}{\dfrac x r} = \dfrac{\sin \alpha}{\cos \alpha} \text{。}$ \\
    \rule{0pt}{3em}$\cot \alpha = \dfrac x y = \dfrac{\; \dfrac x r \;}{\dfrac y r} = \dfrac{\cos \alpha}{\sin \alpha} \text{。}$ \\
    \rule{0pt}{2em}$\sin^2 \alpha + \cos^2 \alpha = \left( \dfrac y r\right)^2 + \left( \dfrac x r\right)^2 = \dfrac{y^2 + x^2}{r^2} = \dfrac{r^2}{r^2} = 1 \text{。}$
\end{tabular}
\end{table}

将 $\sin^2 \alpha + \cos^2 \alpha = 1$ 的两边都除以 $\cos^2 \alpha$,可以得到
$$1 + \tan^2 \alpha = \sec^2 \alpha \text{。}$$

将 $\sin^2 \alpha + \cos^2 \alpha = 1$ 的两边都除以 $\sin^2 \alpha$,可以得到
$$1 + \cot^2 \alpha = \csc^2 \alpha \text{。}$$

以上关系式可以归纳如下:

(1)倒数关系

\begin{center}
    \framebox{\begin{minipage}{15em}
        \begin{gather*}
            \sin \alpha \cdot \csc \alpha = 1 \text{;} \\
            \cos \alpha \cdot \sec \alpha = 1 \text{;} \\
            \tan \alpha \cdot \cot \alpha = 1 \text{。}
        \end{gather*}
    \end{minipage}}
\end{center}

(2)商数关系

\begin{center}
    \framebox{
        \renewcommand\arraystretch{2}
        \begin{tabular}{w{c}{13em}}
            $\tan \alpha = \dfrac{\sin \alpha}{\cos \alpha} \text{;} $ \\
            $\cot \alpha = \dfrac{\cos \alpha}{\sin \alpha} \text{。}$
        \end{tabular}
    }
\end{center}

(3)平方关系

\begin{center}
    \framebox{\begin{minipage}{15em}
        \begin{gather*}
            \sin^2 \alpha + \cos^2 \alpha = 1 \text{;} \\
            1 + \tan^2 \alpha = \sec^2 \alpha \text{;} \\
            1 + \cot^2 \alpha = \csc^2 \alpha \text{。}
        \end{gather*}
    \end{minipage}}
\end{center}

上面这些关系式都是恒等式,即当 $\alpha$ 取使关系式的两边都有意义的任意值时,
关系式两边的值都相等。以后所说的恒等式都是指这个意义下的恒等式。

利用这些关系式,可以根据一个角的某一个三角函数值,求出这个角的其他三角函数值,
还可化简三角函数式,证明其他一些三恒等式,等等。

\liti 已知 $\sin \alpha = \dfrac 4 5$,并且 $\alpha$ 是第二象限的角,求 $\alpha$ 的其他三角函数值。
\vspace{0.5em}

\jie 由 $\sin^2 \alpha + \cos^2 \alpha = 1$,可得 $\cos \alpha = \pm \sqrt{1 - \sin^2 \alpha}$。

$\because \quad \alpha$ 是第二象限的角,$\cos \alpha < 0$,

$\therefore$
\begin{tabular}[t]{l}
    \rule{0pt}{3em}$\cos \alpha = - \sqrt{1 - \sin^2 \alpha} = - \sqrt{1 - \left( \dfrac 4 5 \right)^2} = -\dfrac 3 5$,\\
    \rule{0pt}{2em}$\tan \alpha = \dfrac{\sin \alpha}{\cos \alpha} = \dfrac{\dfrac 4 5}{-\dfrac 3 5} = -\dfrac 4 3$,\\
    \rule{0pt}{2em}$\cot \alpha = \dfrac{1}{\tan \alpha} = \dfrac{1}{-\dfrac 4 3} = -\dfrac 3 4$,\\
    \rule{0pt}{2em}$\sec \alpha = \dfrac{1}{\cos \alpha} = \dfrac{1}{-\dfrac 3 5} = -\dfrac 5 3$,\\
    \rule{0pt}{2em}$\csc \alpha = \dfrac{1}{\sin \alpha} = \dfrac{1}{\; \dfrac 4 5 \;} = \dfrac 5 4$。
\end{tabular}

\vspace{0.5em}
\liti 已知 $\cos \alpha = -\dfrac{8}{17}$,求 $\alpha$ 的其他三角函数值。
\vspace{0.5em}

\jie 因为 $\cos \alpha < 0$,所以 $\alpha$ 是第二象限的角或者是第三象限的角。

(1)如果 $\alpha$ 是第二象限的角,可以得到

\quad \begin{tabular}[t]{l}
    \rule{0pt}{3em}$\sin \alpha = \sqrt{1 - \cos^2 \alpha} = \sqrt{1 - \left(-\dfrac{8}{17} \right)^2} = \dfrac{15}{17}$,\\
    \rule{0pt}{2em}$\tan \alpha = \dfrac{\sin \alpha}{\cos \alpha} = \dfrac{\dfrac{15}{17}}{-\dfrac{8}{17}} = -\dfrac{15}{8}$,\\
    \rule{0pt}{2em}$\cot \alpha = \dfrac{1}{\tan \alpha} = \dfrac{1}{-\dfrac{15}{8}} = -\dfrac{8}{15}$,\\
    \rule{0pt}{2em}$\sec \alpha = \dfrac{1}{\cos \alpha} = \dfrac{1}{-\dfrac{8}{17}} = -\dfrac{17}{8}$,\\
    \rule{0pt}{2em}$\csc \alpha = \dfrac{1}{\sin \alpha} = \dfrac{1}{\; \dfrac{15}{17} \;} = \dfrac{17}{15}$。
\end{tabular}
\vspace{0.5em}

(2)如果 $\alpha$ 是第三象限的角,则可得到

\quad\begin{tabular}[t]{l}
    \rule{0pt}{2em}$\sin \alpha = -\dfrac{15}{17}$,$\tan \alpha = \dfrac{15}{8}$,$\cot \alpha = \dfrac{8}{15}$,\\
    \rule{0pt}{2em}$\sec \alpha = -\dfrac{17}{8}$,$\csc \alpha = -\dfrac{17}{15}$。
\end{tabular}
\vspace{0.5em}

\liti 已知 $\cot \alpha = m \, (m \neq 0)$,求 $\cos \alpha$。

\jie 由于 $\cot \alpha$ 的值为 $m$,且 $m \neq 0$,所以角 $\alpha$ 的终边不在两个坐标轴上。

(1)如果 $\alpha$ 是第一、二象限的角,可以得到

\begin{tabular}{p{2em}l}
    & $\csc \alpha = \sqrt{1 + \cot^2 \alpha} = \sqrt{m^2 + 1}$,\\
    \rule{0pt}{2em} & $\sin \alpha = \dfrac{1}{\csc \alpha} = \dfrac{1}{\sqrt{m^2 + 1}} = \dfrac{\sqrt{m^2 + 1}}{m^2 + 1}$,\\
    \rule{0pt}{2em} $\therefore$ & $\cos \alpha = \sin \alpha \cdot \cot \alpha = \dfrac{m \sqrt{m^2 + 1}}{m^2 + 1}$。
\end{tabular}
\vspace{0.5em}

(2)如果 $\alpha$ 是第三、四象限的角,可以得到

\begin{tabular}{p{2em}l}
    & $\csc \alpha = -\sqrt{m^2 + 1}$,\\
    \rule{0pt}{2em} & $\sin \alpha = -\dfrac{\sqrt{m^2 + 1}}{m^2 + 1}$,\\
    \rule{0pt}{2em} $\therefore$ & $\cos \alpha = -\dfrac{m \sqrt{m^2 + 1}}{m^2 + 1}$。
\end{tabular}
\vspace{0.5em}

\liti 已知 $\tan \alpha \neq 0$,用 $\tan \alpha$ 来表示 $\alpha$ 的其他三角函数。

\jie \begin{tabular}[t]{l}
    \rule{0pt}{2em}$\cot \alpha = \dfrac{1}{\tan \alpha}$,\\
    \rule{0pt}{2em}$\sec \alpha = \pm \sqrt{1 + \tan^2 \alpha}$ ($\alpha$ 为第一、四象限的角时取正号, \\
        \multicolumn{1}{r}{$\alpha$为第二、三象限的角时取负号,以下各式同)}\\
    \rule{0pt}{2em}$\cos \alpha = \dfrac{1}{\sec \alpha} = \dfrac{1}{\pm \sqrt{1 + \tan^2 \alpha}}$,\\
    \rule{0pt}{2em}$\sin \alpha = \cos \alpha \cdot \tan \alpha = \dfrac{\tan \alpha}{\pm \sqrt{1 + \tan^2 \alpha}}$,\\
    \rule{0pt}{2em}$\csc \alpha = \dfrac{1}{\sin \alpha} = \dfrac{\pm \sqrt{1 + \tan^2 \alpha}}{\tan \alpha}$。\\
\end{tabular}
\vspace{0.5em}

一般地,当已知角 $\alpha$ 的任一个三角函数值及角 $\alpha$ 的终边所在的象限时,都可以根据同角三角函数间的基本关系式
求出角 $\alpha$ 的其他三角函数值;当已知角 $\alpha$ 的一个三角函数值、而未指定角 $\alpha$ 的终边所在的象限时,
要根据角 $\alpha$ 的终边可能在的两个象限分别求其他三角函数值;
当已知的角 $\alpha$ 的三角函数值用字母表示时,为了确定角 $\alpha$ 的有些三角函数值表达式前面的正负号,
要对角 $\alpha$ 的终边所在的象限分别进行讨论。

\lianxi
\begin{xiaotis}
    
\xiaoti{根据下列条件,求角 $\alpha$ 的其他三角函数值:}
\begin{xiaoxiaotis}

    \xiaoxiaoti{己知 $\sin \alpha = \dfrac 1 2$,并且 $\alpha$ 为第一象限的角;}
    \vspace{0.5em}

    \xiaoxiaoti{己知 $\cos \alpha = -\dfrac 4 5$,并且 $\alpha$ 为第三象限的角。}
    \vspace{0.5em}

\end{xiaoxiaotis}

\xiaoti{已知 $\cos \theta = \dfrac 1 2$,求 $\theta$ 的其他三角函数值。}
\vspace{0.5em}

\xiaoti{已知 $\cot \varphi = -\dfrac{\sqrt{3}}{3}$,求 $\varphi$ 的其他三角函数值。}
\vspace{0.5em}

\xiaoti{已知 $\sin x = 0.35$,求 $x$ 的其他三角函数值(保留两个有效数字)。}

\end{xiaotis}

\vspace{1em}
\liti 化简下列各式:

\twoInLine[16em]{(1)$\sqrt{1 - \sin^2 100^\circ}$。}{(2)$\sqrt{\sec^2 A - 1}$。}

\jie (1)$\sqrt{1 - \sin^2 100^\circ} = \sqrt{1 - \sin^2 80^\circ} = \sqrt{\cos^2 80^\circ} = \cos 80^\circ$。($\because \quad \cos 80^\circ > 0$)

(2)$\sqrt{\sec^2 A - 1} = \sqrt{\tan^2 A} = |\tan A|$。

\liti 求证 $\cot^2 \alpha (\tan^2 \alpha - \sin^2 \alpha) = \sin^2 \alpha$。

\begin{minipage}{6.6cm}
\begin{align*}
  \text{\zhengming} \quad  & \cot^2 \alpha (\tan^2 \alpha - \sin^2 \alpha) \\
  = & \cot^2 \alpha \cdot \tan^2 \alpha - \cot^2 \alpha \cdot \sin^2 \alpha \\
  = & (\cot \alpha \cdot \tan \alpha)^2 - \dfrac{\cos^2 \alpha}{\sin^2 \alpha} \sin^2 \alpha \\
  = & 1 - \cos^2 \alpha = \sin^2 \alpha \text{。}  
\end{align*}
\end{minipage}

\liti 求证 $\sin^2 \theta \cos^2 \theta = \dfrac{1}{\sec^2 \theta + \csc^2 \theta}$。

\vspace{0.5em}
\zhengming $\begin{aligned}[t]
    & \dfrac{1}{\sec^2 \theta + \csc^2 \theta} = \dfrac{1}{\dfrac{1}{\cos^2 \theta} + \dfrac{1}{\sin^2 \theta}} \\
  = & \dfrac{\sin^2 \theta \cos^2 \theta}{\sin^2 \theta + \cos^2 \theta} = \sin^2 \theta \cos^2 \theta \text{。}
\end{aligned}$
\vspace{0.5em}

\liti 求证 $(1 - \sin^2 A)(\sec^2 A - 1) = \sin^2 A(\csc^2 A  - \cot^2 A)$。

\zhengming $\begin{aligned}[t]
    &(1 - \sin^2 A)(\sec^2 A - 1) = \cos^2 A \cdot \tan^2 A = cos^2 A \cdot \dfrac{\sin^2 A}{\cos^2 A} = \sin^2 A \text{,} \\
    &\sin^2 A (\csc^2 A - \cot^2 A) = \sin^2 A \cdot 1 = \sin^2 A \text{,}
\end{aligned}$

$\therefore \quad (1 - \sin^2 A)(\sec^2 A - 1) = \sin^2 A (\csc^2 A - \cot^2 A)$。

\vspace{0.5em}
\liti 求证 $\dfrac{\cos x}{1 - \sin x} = \dfrac{1 + \sin x}{\cos x}$。
\vspace{0.5em}

\textbf{证明一:} $\because \quad (1 - \sin x)(1 + \sin x) = 1 - \sin^2 x = \cos^2 x = \cos x \cos x$,

\vspace{0.5em}
$\therefore \dfrac{\cos x}{1 - \sin x} = \dfrac{1 + \sin x}{\cos x}$。
\vspace{0.5em}

\textbf{证明二:} $\because \begin{aligned}[t]
      &\dfrac{\cos x}{1 - \sin x} - \dfrac{1 + \sin x}{\cos x} \\
    = &\dfrac{\cos x \cdot \cos x - (1 + \sin x)(1 - \sin x)}{(1 - \sin x)\cos x} \\
    = &\dfrac{\cos^2 x - (1 - \sin^2 x)}{(1 - \sin x) \cos x} \\
    = &\dfrac{\cos^2 x - \cos^2 x}{(1 - \sin x) \cos x} = 0 \text{,}
\end{aligned}$

\vspace{0.5em}
$\therefore \quad \dfrac{\cos x}{1 - \sin x} = \dfrac{1 + \sin x}{\cos x}$。
\vspace{0.5em}

从上面的例子可以看到,证明三角恒等式,可以从任何一边开始,证得它等于另一边,也可以证明左右两边都等于同一个式子,
有时也可以先证明另一个恒等式,从而推得需要证明的恒等式,等等。要在熟练掌握各基本公式的基础上,按照由繁到简的原则,
灵活地运用各种证法。在变形的过程中,将同一式子中的正切、余切、正割、余割都化成正弦及余弦,并注意运用基本公式中的
平方关系,有时可使式子简化。

\lianxi
\setcounter{cntxiaoti}{0}
\begin{xiaotis}

\xiaoti{化简:}
\begin{xiaoxiaotis}

    \begin{tabular}{p{16em}l}
        \xiaoxiaoti{$\cos \theta \tan \theta$;} & \xiaoxiaoti{$\dfrac{1}{\sec^2 \alpha} + \dfrac{1}{\csc^2 \alpha}$;} \\
        \multicolumn{2}{@{}l}{\xiaoxiaoti{$\csc \alpha \cdot \tan \alpha \cdot \sec \alpha \cdot \sin \alpha \cdot \cos \alpha \cdot \cot \alpha$;}} \\
        \xiaoxiaoti{$\dfrac{2\cos^2 \alpha - 1}{1 - 2\sin^2 \alpha}$;} & \xiaoxiaoti{$\dfrac{\tan \alpha + \cot \alpha}{\sec \alpha \csc \alpha}$。}
    \end{tabular}
    
\end{xiaoxiaotis}

\xiaoti{化简:}
\begin{xiaoxiaotis}
    
    \xiaoxiaoti{$\sec \theta \cdot \sqrt{1 - \sin^2 \theta}$,其中 $\theta$ 为第二象限的角;}

    \vspace{0.5em}
    \xiaoxiaoti{$\dfrac{\sec \alpha}{\sqrt{\tan^2 \alpha + 1}}$。}
    \vspace{0.5em}

\end{xiaoxiaotis}

\xiaoti{求证下列恒等式:}
\begin{xiaoxiaotis}

    \xiaoxiaoti{$\sin^4 \alpha - \cos^4 \alpha = \sin^2 \alpha - \cos^2 \alpha$;}

    \xiaoxiaoti{$\sin^4 \alpha + \sin^2 \alpha \cos^2 \alpha + \cos^2 \alpha = 1$;}

    \vspace{0.5em}
    \xiaoxiaoti{$\dfrac{\sin \alpha + \cot \alpha}{\tan \alpha + \csc \alpha} = \cos \alpha$;}
    \vspace{0.5em}

    \xiaoxiaoti{$(\sin \varphi + \tan \varphi)(\cos \varphi + \cot \varphi) = (1 + \sin \varphi)(1 + \cos \varphi)$。}

\end{xiaoxiaotis}

\end{xiaotis}
