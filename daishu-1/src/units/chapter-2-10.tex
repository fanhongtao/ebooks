\subsection{正切函数、余切函数的图象和性质}\label{subsec:2-10}

由诱导公式 $\tan(x + \pi) = \tan x, \, x \in R \text{且} x \neq k\pi + \dfrac \pi 2, \, k \in Z$
知道,正切函数是周期函数。可以证明它的周期(最小正周期)是 $\pi$。现用单位圆上的正切线来作正切函数
$y = \tan x$ 在 $\left( -\dfrac \pi 2, \dfrac \pi 2 \right)$ 内的图象(图 \ref{fig:2-27})。

\begin{figure}[htbp]
    \centering
    \begin{tikzpicture}[>=Stealth]
    \draw [->] (-5.5,0) -- (2.5,0) node[anchor=north] {$x$};
    \draw [->] (0,-3.5) -- (0,4.0) node[anchor=east] {$y$};
    \node at (0.2,-0.2) {$O$};
    \draw (0,1) node[anchor=south west] {$1$};
    \draw (-0.2,-1) -- (0,-1) node[anchor=west] {$-1$};

    \draw[thick, domain=-pi/2+0.3:pi/2-0.3,samples=50, name path=tan] plot (\x, {tan(\x r)});
    \draw [dashed] (pi/2, 3.5) -- (pi/2, -3.5);
    \draw [dashed] (-pi/2, 3.5) -- (-pi/2, -3.5);
    \node [fill=white, inner sep = 0pt] at (0.7, -3.5) {$y = \tan x \quad x \in \left(-\dfrac \pi 2, \dfrac \pi 2 \right)$};

    % 绘制单位圆 及 圆内的分隔线
    \coordinate (O1) at (-3.5, 0);
    \draw [thick] (O1) circle(1) node [anchor=north east, inner sep = 1pt] {$O_1$};
    \foreach \id in {1, 2, 3, 4, -1, -2, -3, -4} {
        \draw (O1) -- ++(22.5 *\id:1) coordinate (A\id);
    }

    % 单位圆的切线
    \draw [thick, name path=tanline] (-2.5, 3.5) -- (-2.5, -3.5);

    \coordinate (xline) at (0, 0);
    \foreach \id in {1, 2, 3, -1, -2, -3} {
        \path [name path=line1] (O1) -- ($(O1)!3!(A\id)$);
        \path [name intersections={of=line1 and tanline, by=a}];
        \path [name path=line2] (a) -- +(5, 0);
        \path [name intersections={of=line2 and tan, by=b}];
        \draw (O1) -- (a);          % 单位圆分隔线延长至切线
        \draw [dashed] (a) -- (b);  % 切线 至 tan 曲线
        \draw (b) -- (b |- xline);  % tan 曲线 至 x 轴
    }

    % 为了让 pi 附近的虚线尽可能多的显示,将负号单独作为一个node
    % 普通的写法: \node [fill=white, inner sep=0pt, font=\footnotesize] at (-pi/2-0.3, 0.3) {$-\dfrac \pi 2$};
    \draw (-pi/2-0.2, 0.3) node [fill=white, inner sep=0pt, font=\footnotesize] {$\dfrac \pi 2$} +(-0.3, 0) node {$-$};
    \draw (-pi/4, 0.3) node [fill=white, inner sep=0pt, font=\footnotesize] {$\dfrac \pi 4$} +(-0.3, 0) node {$-$};

    \node [font=\footnotesize] at (pi/4, -0.4) {$\dfrac \pi 4$};
    \node [font=\footnotesize] at (pi/2-0.2, -0.4) {$\dfrac \pi 2$};
\end{tikzpicture}

    \caption{}\label{fig:2-27}
\end{figure}

根据正切函数的周期性,我们可以把图象向左、右扩展,得出 \vspace{0.5em}
$y = \tan x, \, x \in \left( -\dfrac \pi 2 + k\pi, \dfrac \pi 2 + k\pi \right)$, $k \in Z$
的图象——\textbf{正切曲线}(图 \ref{fig:2-28})。可以看出,正切曲线是由相互平行的直线
$x = \dfrac \pi 2 + k\pi \, (k \in Z)$ 隔开的无穷多支曲线所组成的。

\begin{figure}[htbp]
    \centering
    \begin{tikzpicture}[>=Stealth]
    \draw [->] (-2*pi,0) -- (2*pi,0) node[anchor=north] {$x$};
    \draw [->] (0,-3.5) -- (0,4.0) node[anchor=east] {$y$};
    \node at (0.3,-0.3) {$O$};
    \node at (-pi - 0.1, 0.3) {$-\pi$};
    \node at (pi, 0.3) {$\pi$};

    \foreach \y in {-1, 1} {
        \draw (0.2,\y) -- (0,\y) node[anchor=east] {\y};
    }

    \foreach \y / \name in {
        -1.5*pi / $-\dfrac{3\pi}{2}$,
        -0.5*pi / $-\dfrac{\pi}{2}$,
        0.5*pi / $\dfrac{\pi}{2}$,
        1.5*pi / $\dfrac{3\pi}{2}$
    } {
        \draw [dashed] (\y, 3.5) -- (\y, -3.5) (\y - 0.4, -0.5) node {\name};
    }

    \foreach \mid in {-pi, 0, pi} {
        \draw[thick, domain=\mid - pi/2 + 0.3 : \mid + pi/2 - 0.3, samples=50] plot (\x, {tan(\x r)});
    }

    \node [fill=white] at (0, -3.5) {$y = \tan x$};
\end{tikzpicture}

    \caption{}\label{fig:2-28}
\end{figure}

正切函数 $y = \tan x$ 有以下主要性质:

(1)\textbf{定义域} \mylabel{xingzhi:tan-1}

\textbf{函数 $y = \tan x$ 的定义域是 $\{ x \mid x \in R \; \text{且} \; x \neq k\pi + \dfrac \pi 2, \, k \in Z \}$。}

(2)\textbf{值域} \mylabel{xingzhi:tan-2}

从图 \ref{fig:2-28} 可以看出,
当 $x$ 小于 $\dfrac \pi 2 + k\pi, \, k \in Z$ 而无限接近于 $\dfrac \pi 2 + k\pi$ 时,\vspace{0.5em}
$\tan x$ 无限增大,即可比指定的任何正数都大,我们把这种情况记作 $\tan x \to +\infty$(读作 $\tan x$ 趋向于正无穷大);\vspace{0.5em}
当 $x$ 大于 $\dfrac \pi 2 + k\pi, \, k \in Z$ 而无限接近于 $\dfrac \pi 2 + k\pi$ 时,\vspace{0.5em}
$\tan x$ 无限减小,即取负值且它的绝对值可比指定的任何正数都大,我们把这种情况记作 $\tan x \to -\infty$(读作 $\tan x$ 趋向于负无穷大)。
这就是说,$\tan x$ 可以取任意实数值,但没有最大值、最小值。因此,
\textbf{函数 $y  = \tan x$ 的值域是实集 $R$。}

(3)\textbf{周期性} \mylabel{xingzhi:tan-3}

\textbf{$y = \tan x$ 是周期函数,周期是 $\pi$。}

(4)\textbf{奇偶性} \mylabel{xingzhi:tan-4}

\textbf{从诱导公式 $\tan(-x) = -\tan x$ 知道,$y = \tan x$ 是奇函数,它的图象关于原点对称。}

(5)\textbf{单调性} \mylabel{xingzhi:tan-5}

从图 \ref{fig:2-28} 可以看出,\textbf{函数 $y = \tan x$ 在每一个开区间
$(-\dfrac \pi 2 + k\pi, \, \dfrac \pi 2 + k\pi), \, k \in Z$ 内都是增函数。}
(想一想:正切函数在整个定义域内是增函数吗?)

用类似的方法,可以作出余切函数
$y = \cot x, \, x \in R \text{且} x \neq k\pi, \, k \in Z$
的图象——\textbf{余切曲线},如图 \ref{fig:2-29} 所示。

\begin{figure}[htbp]
    \centering
    \begin{tikzpicture}[>=Stealth]
    \draw [->] (-2*pi-1,0) -- (2*pi+1,0) node[anchor=north] {$x$};
    \draw [->] (0,-3.5) -- (0,4.0) node[anchor=east] {$y$};
    \node at (0.3,-0.3) {$O$};

    \foreach \x / \name in {
        -2*pi / $-2\pi$,
        -pi / $-\pi$,
        pi / $\pi$,
        2*pi / $2\pi$
    } {
        \draw (\x,0) node[anchor=north west] {\name};
    }

    \foreach \x / \name in {
        -1.5*pi / $-\dfrac{3\pi}{2}$,
        -0.5*pi / $-\dfrac{\pi}{2}$,
        0.5*pi / $\dfrac{\pi}{2}$,
        1.5*pi / $\dfrac{3\pi}{2}$
    } {
        \draw (\x-0.3, 0) node[anchor=south west] {\name};
    }

    \foreach \y in {-1, 1} {
        \draw (0.2,\y) -- (0,\y) node[anchor=east] {\y};
    }

    \foreach \x / \name in {
        -2*pi / $-\dfrac{3\pi}{2}$,
        -pi / $-\dfrac{\pi}{2}$,
        pi / $\dfrac{\pi}{2}$,
        2*pi / $\dfrac{3\pi}{2}$
    } {
        \draw [dashed] (\x, 3.5) -- (\x, -3.5);
    }

    \foreach \mid in {-3*pi/2, -pi/2, pi/2, 3*pi/2} {
        \draw[thick, domain=\mid - pi/2 + 0.3 : \mid + pi/2 - 0.3, samples=50] plot (\x, {cot(\x r)});
    }

    \node [fill=white] at (0, -3.5) {$y = \cot x$};
\end{tikzpicture}

    \caption{}\label{fig:2-29}
\end{figure}

余切函数 $y = \cot x$ 的主要性质如下:

(1)\textbf{定义域} \mylabel{xingzhi:cot-1}

\textbf{函数 $y = \cot x$ 的定义域是 $\{ x \mid x \in R \; \text{且} \; x \neq k\pi, \, k \in Z \}$。}

(2)\textbf{值域} \mylabel{xingzhi:cot-2}

\textbf{函数 $y  = \cot x$ 的值域是实集 $R$,没有最大值、最小值。}

(3)\textbf{周期性} \mylabel{xingzhi:cot-3}

\textbf{$y = \cot x$ 是周期函数,周期是 $\pi$。}

(4)\textbf{奇偶性} \mylabel{xingzhi:cot-4}

\textbf{$y = \cot x$ 是奇函数,它的图象关于原点对称。}

(5)\textbf{单调性} \mylabel{xingzhi:cot-5}

\textbf{$y = \cot x$ 在每一个开区间 $(k\pi, \, (k + 1)\pi), \, k \in Z$ 内都是减函数。}

\vspace{0.5em}
\liti 求函数 $y = \tan \left( x + \dfrac \pi 4 \right)$ 的定义域。
\vspace{0.5em}

\jie 令 $z = x + \dfrac \pi 4$,那么函数 $y = \tan z$ 的定义域是
$$\{ z \mid z \in R, \, \text{且} \; z \neq k\pi + \dfrac \pi 2, \, k \in Z \} \text{。}$$
由
$$x + \dfrac \pi 4 = z = k\pi + \dfrac \pi 2,$$
得
$$x = k\pi + \dfrac \pi 2 - \dfrac \pi 4 = k\pi + \dfrac \pi 4 \text{。}$$

因此,$y = \tan \left( x + \dfrac \pi 4 \right)$ 的定义域是
$$\{ x \mid x \in R, \, \text{且} \; x \neq k\pi + \dfrac \pi 4, \, k \in Z \} \text{。}$$

\lianxi
\begin{xiaotis}

\xiaoti{根据图 \ref{fig:2-27},写出 $y = \tan x, \, x \in \left( -\dfrac \pi 2, \, \dfrac \pi 2 \right)$ 的图象的作法。}
\vspace{0.5em}

\xiaoti{观察正切曲线及余切曲线,写出满足下列条件的 $x$ 的值或 $x$ 的区间:}
\begin{xiaoxiaotis}

    \begin{tabular}[t]{*{3}{@{}p{13em}}}
        \xiaoxiaoti {$\tan x > 0$;} & \xiaoxiaoti {$\tan x = 0$;}  & \xiaoxiaoti {$\tan x < 0$;} \\
        \xiaoxiaoti {$\cot x > 0$;} & \xiaoxiaoti {$\cot x = 0$;}  & \xiaoxiaoti {$\cot x < 0$。}
    \end{tabular}

\end{xiaoxiaotis}

\xiaoti{求下列函数的定义域:}
\begin{xiaoxiaotis}

    \twoInLine[13em]{\xiaoxiaoti{$y = \tan 3x$;}}{\xiaoxiaoti{$y = -3\cot 2x$。}}

\end{xiaoxiaotis}

\xiaoti{求下列函数的周期:}
\begin{xiaoxiaotis}

    \threeInLine[13em]{\xiaoxiaoti{$y = \tan 2x$;}}{\xiaoxiaoti{$y = \cot \left( x + \dfrac \pi 3 \right)$;}}{\xiaoxiaoti{$y = 5\tan \dfrac x 2$。}}

\end{xiaoxiaotis}
\vspace{0.5em}

\xiaoti{指出下列各组函数值的差哪些大于零,哪些小于零(不求值):}
\begin{xiaoxiaotis}

    \xiaoxiaoti{$\tan 138^\circ - \tan 143^\circ$;}

    \vspace{0.5em}
    \xiaoxiaoti{$\tan \left( -\dfrac{13}{4} \pi \right) - \tan \left( -\dfrac{17}{5} \pi \right)$;}
    \vspace{0.5em}

    \xiaoxiaoti{$\cot 281^\circ - \cot 305^\circ$;}

    \vspace{0.5em}
    \xiaoxiaoti{$\cot \left( -\dfrac{19}{7} \pi \right) - \cot \left( -\dfrac{23}{8} \pi \right)$。}
    \vspace{0.5em}

\end{xiaoxiaotis}

\end{xiaotis}
