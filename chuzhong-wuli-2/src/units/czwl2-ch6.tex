\chapter{热机}

燃料燃烧时释放出大量的热能。要利用这些热能来做有用的功,
例如举高重物,驱动车辆,就要设法把热能转化为机械能。
热机就是把热能转化为机械能的机器。

热机是怎样把热能转化为机械能的呢?在上一章图 \ref{fig:5-6} 的实验里,
酒精燃烧时释放出的热能传递给水和水蒸气,水蒸气在膨胀时做了功,它的热能转化为软木塞的机械能。

如果把试管换成坚固的金属汽缸,
把软木塞换成跟汽缸内壁紧密接触又能在汽缸内前后移动的活塞,
我们就得到了一台最简单、最原始的热机。

热机是在十七世纪伴随着工业革命的兴起而出现的,以后又不断地得到改进和发展,
所有的热机都具有一个共同的特点:燃料燃烧释放出热能,这些热能又传递给工作物质——水蒸气或燃气;
工作物质获得热能后,膨胀做功,把一部分热能转化为机械能,同时自己的热能减少,温度降低。

现在使用的热机有很多种,我们在下面只介绍最常用的内燃机。

\section{汽油机的工作原理}\label{sec:6-1}

内燃机的基本特点是让燃料在机器的气缸内燃烧,生成高温高压的燃气,
利用这个燃气作为工作物质去推动活塞做功。内燃机的名字就是由此而来的。

\begin{wrapfigure}{r}{6.5cm}
    \centering
    \includegraphics[width=5cm]{../pic/czwl2-ch6-1}
    \caption{}\label{fig:6-1}
\end{wrapfigure}

内燃机有两种:汽油机和柴油机。这一节,我们介绍汽油机的工作原理。

汽油机是用汽油作燃料的内燃机。它的构造如图 \ref{fig:6-1} 所示,气缸里的活塞用连杆跟曲轴相连,
气缸上面有进气门和排气门,气缸顶部有火花塞。

汽油机工作的时候,活塞在气缸里往复运动。活塞从气缸一端运动到另一端叫做一个冲程。
四冲程汽油机的工作过程是由吸气、压缩、做功、排气四个冲程组成的。

第一个冲程是吸气冲程。如 \hyperref[fig:pic3]{彩图3} 甲所示,
进气门打开,排气门关闭,活塞由最上端向下运动,气缸里气体的体积增大,压强减小(低于大气压),
于是在化油器内由汽油和空气形成的燃料混合物从进气门被吸入气缸。
当活塞到达气缸最下端的时候,进气门关闭,完成吸气冲程。

第二个冲程是压缩冲程。 如 \hyperref[fig:pic3]{彩图3} 乙所示,
进气门和排气门都关闭,活塞向上运动,燃料混合物受到压缩,压强和温度都提高了。
最后压强达到 6~15千克力/$\pflm$\footnotemark,温度升高到 250 ~ 300 ℃ 左右。
\footnotetext{千克力/$\pflm$ (即工程大气压)是工程技术上常用的压强单位,
它与帕斯卡的关系是: $1 \text{千克力/}\pflm = 9.80665 \times 10^4 \pasika$。}

第三个冲程是做功冲程。如 \hyperref[fig:pic3]{彩图3} 丙所示,
在压缩冲程末,火花塞产生电火花,燃料混合物猛烈燃烧,产生高温高压的燃气,
温度升高到 2000 ~ 2500 ℃,压达到 30 ~ 50 千克力/$\pflm$。
于是高温高压燃气推动活塞向下运动,活塞又通过连杆使曲轴转动。

第四个冲程是排气冲程。如 \hyperref[fig:pic3]{彩图3} 丁所示,
进气门关闭,排气门打开,活塞向上运动,把废气排出气缸。

排气冲程末,排气门关闭,进气门打开,活塞再向下运动,又开始了新的吸气冲程。

在汽油机工作时,上面所讲的四个冲程是周而复始循环不停的,
所以把这四个冲程叫做一个工作循环。每一个循环,活塞往复两次,曲轴转动两周。

汽油机在四个冲程中,只有做功冲程燃气对外做功;其他三个冲程是辅助冲程,
要靠安装在曲轴上的飞轮的惯性来完成。
从全局看,做功冲程固然是主要冲程,但其他三个冲程也是不可少的,
没有其他三个冲程就失去了做功冲程对外做功的条件。

汽油机在开始运转时,需要外力(人力或电动机的驱动力)先使曲轴转动起来,以后汽油机才能够自行工作。

汽油机比较轻巧,常用在汽车、飞机和小型农业机械(如插秧机、机动喷雾器)上面。


\section{柴油机的工作原理}\label{sec:6-2}

\begin{wrapfigure}[14]{r}{6.5cm}
    \centering
    \includegraphics[width=5cm]{../pic/czwl2-ch6-2}
    \caption{柴油机}\label{fig:6-2}
\end{wrapfigure}

柴油机是用柴油作燃料的内燃机。柴油机的构造跟汽油机相似,
主要不同的是柴油机的气缸顶部有一个喷油嘴,没有火花塞(图 \ref{fig:6-2})。
柴油机的每一工作循环也是由吸气、压缩、做功和排气四个冲程组成的。
但由于使用的燃料不同,柴油机和汽油机的工作过程也有所不同。

在吸气冲程,汽油机吸入气缸里的是汽油和空气的混合物,柴油机吸入气缸里的只是空气。

\begin{enhancedline}
在压缩冲程,汽油机只把燃料混合物的体积压缩到吸气冲程末的 $\dfrac{1}{6}$ ~ $\dfrac{1}{9}$。
如果压缩得更多,在压缩过程的中途,燃料混合物就因温度升高超过燃点而燃烧,汽油机将无法正常工作。
柴油机可以把空气的体积压缩到吸气冲程末的 $\dfrac{1}{16}$ ~ $\dfrac{1}{22}$,
压强达到 35 ~ 45 千克力/$\pflm$,温度升高到 500 ~ 700 ℃ ,比汽油机里燃料混合物的压强和温度都高。
\end{enhancedline}

在做功冲程,汽油机是用火花塞点火使燃料燃烧的(这种点火方式叫做点燃式)。
柴油机是在压缩冲程末,由喷油嘴向气缸内喷射雾状柴油,
雾状柴油遇到远远超过它的燃点的热空气立即燃烧(这种点火方式叫做压燃式)。
燃气的温度达到 1700 ~ 2000 ℃ ,压强达到 50 ~ 100 千克力/$\pflm$。
在柴油机里,推动活塞做功的燃气的压强比汽油机里的高,燃气做的功较多,所以效率较高。

在排气冲程,柴油机和汽油机一样,将废气排出气缸。

柴油比汽油便宜,所以柴油机比汽油机经济,但是比较笨重,主要应用在拖拉机、坦克、轮船、内燃机车、载重汽车上。
在没有电源的地方,常用它带动水泵和各种农业加工机械工作,或带动发电机发电。



