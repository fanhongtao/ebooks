\section{实验:测定小灯泡的功率}\label{sec:9-3}

在这个实验里我们将测定小灯泡的额定功率和小灯泡不在额定电压时的实际功率,
来加深我们学过的电功率知识和巩固电学实验技能。

常用的小灯泡只标明额定电压(如 1.2、2.5 、3.8 伏特),而不标明额定功率。
我们要测定它的额定功率,根据额定功率的意义和公式 $P = UI$,
必须先把它两端的电压调整为额定电压,再测出它通过的电流强度。
实验用的电源电压要高于小灯泡的额定电压。
要调整小灯泡两端的电压,需要给小灯泡串联一个滑动变阻器,分去一部分电压。
因此,除了电源、电键、导线以外,还必须有滑动变阻器、伏特表和安培表。

根据实验的要求,自己设计电路,画出电路图。
按照电路图把实验电路连接好,在闭合电键之前,调节滑动变阻器,
使电路中的电阻最大,然后再闭合电键使电路接通。

调节滑动变阻器,使小灯泡在额定电压下发光,观察发光情况。
记下伏特表和安培表的读数,算出小灯泡的额定功率。

\begin{enhancedline}
调节滑动变阻器,使小灯泡两端的电压约高于额定电压的 $\dfrac{1}{5}$,
观察这时的发光情况与额定电压时有什么不同。
记下伏特表和安培表的读数,算出这时小灯泡的实际功率。
\end{enhancedline}

调节滑动变阻器,使小灯泡两端的电压低于额定电压,
观察这时的发光情况与额定电压时有什么不同。
记下伏特表和安培表的读数,算出这时小灯泡的实际功率。

自己设计一个表格,把三次实验数据都填在表格里。



\lianxi

(1) 一辆电车的电动机的电功率是 60 千瓦,它在额定电压 600 伏特下工作时,通过的电流强度是多大?

(2) 电度表上都标着一个电压值和一个电流值,所标的电压是额定电压,所标的电流是允许通过的最大电流,
一只标着 “220V \; 5A” 的电度表,可以用在最大功率是多少瓦特的照明电路上?

(3) 一个标有 “220V \; 40W” 的灯泡,在正常发光时,通过灯丝的电流强度是多大?这时灯丝的电阻是多大?

(4) 1 度电可以供一个标有 “220V \; 40W” 的灯泡正常发光多长时间?

(5) 平均起来,一次雷电的电流强度约二万安培,电压约十亿伏特,放电时间约千分之一秒。
求平均一次雷电的电功率约多少千瓦,电流所做的功约多少度。

(6) 一间教室里有四盏 60 瓦特的电灯,由于同学们注意节约用电,做到人走灯熄,
这样平均每天每盏灯少开半小时。那么一个月(按30 天计算)可以节约多少度电?

(7) 观察一只电度表,记录电度表的数值,隔半个月再观察一次,算一算在这半个月中共用去几度电。

(8) 观察一些电器设备(如日光灯、电烙铁、电视机等)的铭牌或说明书,把它们的额定功率值记在作业本上。

(9) 弄清家中全部用电器的电功率和每个用电器平均每天用电的时间,再了解一下本地每度电的价钱。
然后算出一个月应付多少电费。到交电费时把算出的钱数同实际付的钱数相比较。
注薏,每个用电器平均每天使用多长时间,最难估得准确,这往往是两个钱数相差较多的主要原因。



\section*{小实验}

\begin{enhancedline}
一般家庭用的电度表的盘面上,标有每千瓦时的转数,利用这个数值可以测定用电器的电功率。
例如,一只电度表的盘面上标着 3000R/kWh,它表示电流通过用电器每做 1 千瓦时的功,
这只电度表的转盘(盘的边缘有个红点)转 3000 转。如果转盘转 15 转,即 3000 转的 $\dfrac{1}{200}$,
电流做的功就是 $\dfrac{1}{200}\qianwashi = 1.8 \times 10^4\jiaoer$。
如果转 15 转的时间是 180 秒,那么用电器的电功率就是 100 瓦特。

先记下所用电度表的每千瓦时的转数,然后使属于这只电度表的其他用电器停止工作,
只让待测功率的用电器工作,记下电度表转过一定转数所用的时间(可用秒表或手表的秒针来测),
算出所测用电器的电功率。
\end{enhancedline}
