\section{凸透镜的应用}\label{sec:1-8}

\xiaobiaoti{凸透镜成像的几种情况}
从上一节的实验,我们可以得到如图 \ref{fig:1-27} 所示的凸透镜成像的情况。

\begin{figure}[H]
    \centering
    \includegraphics[width=0.6\textwidth]{../pic/czwl2-ch1-27}
    \caption{凸透镜成像}\label{fig:1-27}
\end{figure}

物体在凸透镜的焦点以外的时候,在透镜另一侧的光屏上总能得到倒立的实像,并且物距越小,像就越大,像距也越大。
当 $u > 2f$ 时,像是缩小的,$f < v < 2f$ (图 \ref{fig:1-27} 甲);
当 $2f > u > f$ 时,像是放大的,$v > 2f$(图 \ref{fig:1-27} 乙)。

物体在凸透镜的焦点以内的时候,在透镜的另一侧得不到物体的实像,
透过透镜可以看到正立、放大的像(图 \ref{fig:1-27} 丙)。
这个像跟平面镜成像相似,不是物体上各点发出的光实际会聚成的,所以是虚像。

\xiaobiaoti{照相机} 当 $u > 2f$ 时,凸透镜能够成缩小的实像,照相机就是利用这种现象来拍摄照片的。

\begin{figure}[htbp]
    \centering
    \includegraphics[width=0.6\textwidth]{../pic/czwl2-ch1-28}
    \caption{照相机示意图}\label{fig:1-28}
\end{figure}

照相机(图 \ref{fig:1-28})的前部有一个镜头,通常是由一组透镜组成的,它的作用相当于一个凸透镜。
镜头后面是暗箱,照相的感光胶片就装在暗箱的底部。选定了被照景物和照相的位置后,
调节暗箱的长度,也就是调节镜头的位置,使胶片上得到被照景物的清晰的倒立、缩小的实像。
胶片上的感光物质受到形成实像的光的照射,发生了化学变化,经过处理就得到底片,
然后由底片就可以得到照片。


\xiaobiaoti{幻灯机} 当 $2f > u > f$ 时,凸透镜能够成放大的实像,幻灯机就是利用这种现象,
把幻灯片上景物的像投射到银幕上的。

\begin{figure}[htbp]
    \centering
    \begin{minipage}{9cm}
    \centering
    \includegraphics[width=8.5cm]{../pic/czwl2-ch1-29}
    \caption{幻灯机示意图}\label{fig:1-29}
    \end{minipage}
    \qquad
    \begin{minipage}{6cm}
    \centering
    \includegraphics[width=4cm]{../pic/czwl2-ch1-30}
    \caption{放大镜}\label{fig:1-30}
    \end{minipage}
\end{figure}


图 \ref{fig:1-29} 是幻灯机的示意图。它前部的镜头相当于一个凸透镜,透明的幻灯片插在凸透镜后面比焦点略微远些的地方。
幻灯片后面是聚光镜,再后面是很强的光源和反光镜。光源发出的光和反光镜反射的光,经过聚光镜强烈地照亮幻灯片,
前后调整镜头的位置,银幕上就会出现幻灯片上景物的倒立、放大的实像。
我们要看到正立的像,只要把幻灯片倒过来插就行了。

\xiaobiaoti{放大镜} 当 $u < f$ 时,凸透镜能够成放大的虚像,放大镜就是利用这种现象来观察物体的。

使用放大镜的时候,必须把要观察的物体放在焦点以内才能看到正立、放大的虚像(图 \ref{fig:1-30})。



\section*{阅读材料:电影}

\begin{wrapfigure}[29]{r}{6.5cm}
    \centering
    \includegraphics[width=5.5cm]{../pic/czwl2-ch1-31}
    \caption{电影片}\label{fig:1-31}
\end{wrapfigure}


电影放映机的构造相当复杂,但是它的投影原理跟幻灯机差不多,
也是利用相当于凸透镜的镜头把电影片上景物的像射到银幕上的。
不同的是,电影的画面是活动的。

电影片上有一连串的照片。这些照片是对活动的景物每隔了 $\dfrac{1}{24}$ 秒拍一张拍摄下来的,
也就是说,一秒钟内要依次拍摄 24 张照片。因此,后一张跟前一张的景物的变化相差很小。
例如,拍摄体操运动员 6 ~ 7 秒钟内完的一个动作,在电影片上就留下了一百多张连续的照片。
图 \ref{fig:1-31} 是从这个过程中选出来的几个画面。
放映时,用电动机带动电影片,使它上面的照片依次从镜头后面通过,每秒钟通过 24 张。
每张照片都要在镜头后面停顿大约 0.04 秒的时间,它的放大的实像也在银幕上停留相同的时间。
在换照片的时候,电影放映机上有一个特殊的装置把镜头遮住,使银幕暂时黑暗。
但是我们看电影的时候,却觉得银幕上总有像,并且像中的景物是连续活动的,这是为什么呢?

这是因为人的视觉有一种叫做视觉暂留的特性,就是外界的景物消失以后,视神经对它的印象还会保留 0.1 秒的时间。
在黑暗中,迅速地移动一支点燃着的香,会看到香的亮点变成了一条亮线,就是由于视觉暂留。
放电影时,银幕上相继出现的像相隔的时间(也就是银幕黑暗的时间)不到 0.01 秒,并且它们上面的景物变化很小,
由于视觉暂留的缘故,我们就觉得像是连续活动的了。

通常,我们看的电影,银幕上的景物活动的情况跟实际上的快慢程度是一致的。
这是因为拍摄电影跟放映电影的速度相同,即拍摄时每秒拍 24 张,放映时也是每秒放 24 张。
如果拍摄时加快拍摄速度,例如每秒拍 3900 张,而放映时仍每秒放 24 张,那么银幕上的动作就会比实际的慢一百六十多倍。
实际的动作在电影里就是慢悠悠的了。这就是电影里的慢镜头。
相反,如果放慢拍摄速度,而按正常速度放映,结果银幕上的动作就会比实际的快。
实际的动作是很缓慢的,在电影里却是刹那间完成了。这就是电影里的快镜头。

电影的快、慢镜头,不但在一般的电影里被用来达到一定的艺术效果,在科学研究上也很重要。
例如,使用慢镜头可以更仔细地观察一个物体的运动。

电影不但丰富了我们的文化生活,而且能能够把各种活动情况有声有色地记录下来。
因此,电影是一种很好的记录工具。



\lianxi

(1) 有经验的渔民在叉鱼的时候,不把叉对准所看到的鱼的位置,而是对着稍低于所看到的鱼的位置叉去,才能叉到鱼。为什么?

(2) 图 \ref{fig:1-32} 中画出了光通过透镜前后的方向,在图中填上适当类型的透镜。

\begin{figure}[htbp]
    \centering
    \includegraphics[width=0.6\textwidth]{../pic/czwl2-ch1-32}
    \caption{}\label{fig:1-32}
\end{figure}

(3) 有两个凸透镜,要想使一束跟它们的主轴平行的光通过它们后仍平行射出,这两个凸透镜应当怎样放置?
画出这一束光通过这两个凸透镜的情况。

(4) 用镜头焦距一定的照相机照相,想使照片上的人像大一些,照相机应该离被照的人近一些,还是远一些?

(5) 放映幻灯的时候,想使银幕上的像大一些,应该使幻灯机离银幕近一些,还是远一些?

