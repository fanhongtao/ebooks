\section{能的转化和守恒定律}\label{sec:5-6}

现在我们从能的转化的角度来探讨一下改变物体热能的两种方法。

用热传递的方法来改变物体的热能,实际上是热能从一个物体转移到另一个物体的过程。
参加热传递的两个温度不同的物体,一个放出热量,另一个吸收热量。
一个放出多少热量,它的热能就减少多少。
另一个吸收多少热量,它的热能就增加多少。
我们知道,如果这两个物体都没有从别的物体那里吸收热量,也没有把热量传递给别的物体,
那么一个放出的热量跟另一个吸收的热量总是相等的。
可见一个物体减小多少热能,另一个就增加多少热能。
热能从一个物体转移到另一个物体,是既不增加,也不减少,而保持守恒。

用做功的方法来改变物体的热能,实际上是热能和其他形式的能相互转化的过程。
克服摩擦做功或者压缩气体做功而使物体的热能增加,要消耗机械能,这时机械能转化为热能。
而且做多少功,即消耗多少机械能,就得到多少热能。
气体膨胀做功而使物体的热减少,这时热能转化为机械能,而且做多少功,即消多少热能,就得到多少机械能。
可见,在热能和机械能相互转化的过程中,能量同样是既不增加,也不减少,而保持守恒。

除机械能、热能外,还有其他形式的能,如电能、光能、原子能、化学能等等。
各种形式的能都可以在一定条件下相互转化。
水轮机带动发电机发电,机械能转化为电能;
电动机带动水泵把水抽到高处,电能又转化为机械能。
木柴燃烧发出热和光,化学能转化为热能和光能;
植物吸收太阳光进行光合作用,光能又转化为化学能。
大量的实验事实告诉我们,任何一种形式的能在转化成其他形式的能的过程中,总的能量都是守恒的。

所以,\textbf{能量既不会消灭,也不会创生,它只会从一种形式转化成另一种形式,
或者从一个物体转移到另一个物体,而能的总量保持不变}。这个规律叫做\textbf{能的转化和守恒定律}。

能的转化和守恒定律,是自然界最普遍、最重要的基本定律之一,是我们认识自然和改造自然的重要科学依据。

