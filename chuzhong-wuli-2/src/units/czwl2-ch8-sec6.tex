\section{电阻}\label{sec:8-6}

在上一节的实验里我们已经看到,在相同的电压下,通过导线 $AB$ 的电流跟通过导线 $CD$ 的电流并不相等。
这表明,电流强度除了跟电压有关系,还跟导体本身对电流的阻碍作用有关系。在相同的电压下,
通过导线 $AB$ 的电流比较强,是因为 $AB$ 对电流的阻碍作用比小,
通过导线 $CD$ 的电流比较弱,是因为 $CD$ 对电流的阻碍作用比大。
一切导体都有阻碍电流的性质,这种性质叫做\textbf{电阻}。
在上一节的实验中,导线 $AB$ 的电阻比较小,导线 $CD$ 的电阻比较大。

在物理学里怎样表示导体的电阻呢?研究一下上节表 \ref{tab:8-1} 和表 \ref{tab:8-2} 中的数据,可以看出,
对于电阻比较小的导线 $AB$ 来说,电压跟电流强度的比值也比较小,
对于电阻比较大的导线 $CD$ 来说,电压跟电流强度的比值也比较大。

所以,在物理学里,用导体两端的电压跟通过导体的电流强度的比值,来表示导体的电阻。

电阻的单位是\textbf{欧姆}(符号是 $\Omega$\footnotemark)。
\footnotetext{$\Omega$:希腊字母,汉语拼音读法是 omiga 。}
如果导半两端的电压是 1 伏特,通过的电流是 1 安培,那么这个导体的电阻就是 1 欧姆。
$$ 1\oumu = \dfrac{1\fute}{1\anpei} \;\juhao $$

如果导体两端的电压是 1 伏特,通过的电流是 0.5 安培,那么,这段导体的电阻就是 2 欧姆。

在上一节实验中,
导线 $AB$ 的电阻是 $2\fute / 0.4\anpei = 5\oumu$,
导线 $CD$ 的电阻是 $2\fute / 0.2\anpei = 10\oumu$。
手电筒小灯泡发光时灯丝的电阻约 10 欧姆,
日常照明用的白炽电灯发光时灯丝的电阻是几百欧姆到几千欧姆,
实验用的铜导线的电阻只有百分之几欧姆,通常可以略去不计。

比欧姆大的单位有千欧(k$\Omega$)、兆欧(M$\Omega$)。
\begin{align*}
    1 \text{千欧} &= 1000 \text{欧姆,}\\
    1 \text{兆欧} &= 1000 \text{千欧。}\\
\end{align*}

