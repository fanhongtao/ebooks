\section{燃料的燃烧值}\label{sec:3-2}

燃料燃烧时要放出热量。质量相等的不同燃料燃烧时,放出的热量是不等的。
1 千克干木柴完全燃烧能放出大约 3000 千卡的热量,
1 千克汽油完全燃烧能放出 11000 千卡的热量。
我们把 \textbf{1 千克某种燃料完全燃烧放出的热量,叫做这种燃料的燃烧值}。

燃烧值的单位是千卡/千克或卡/克,读做“千卡每千克”或“卡每克”。

\begin{table}[H]
    \centering
    \caption*{几种燃料的燃烧值(千卡/千克或卡/克)}
    \begin{tblr}{
        colspec={|cQ[r,6em]|cQ[r,4em]|},
        columns={colsep+=1em},
    }
        \hline
        干木柴 & (约)3000 & 酒精 & 7200 \\
        烟煤 & (约)7000 & 柴油 & 10200 \\
        无烟煤 & (约)8000 & 煤油 & 11000 \\
        焦炭 & 7100 & 汽油 & 11000 \\
        木炭 & 8000 & 氢 & 34000 \\
        \hline
    \end{tblr}
\end{table}

知道了燃烧值,就很容易计算出燃烧一定量的燃料可以得到多少热量。
例如,已知烟煤的燃烧值是 7000 千卡/千克,那么,
燃烧 5 千克的烟煤可以得到 35000 千卡的热量,
燃烧 9 千克的烟煤可以得到 63000 千卡的热量,等等。

实际上,燃料很难完全燃烧,因此放出的热量往往少于根据燃烧值计算出来的热量。
另外燃料燃烧放出的热量也不能完全得到利用。
例如用煤炉烧水,煤燃烧放出的热量并没有全部被水吸收,有一部分散失到空气里,
有一部分传给盛水的容器,还有一部分传给了炉壁。
我们的目的是烧水,所以被利用的只是水吸收的这一部分热量。
我国工农业生产和人民生活,每年都消费大量的煤、柴油、汽油等燃料,
因此,改善燃烧条件,使燃料尽可能燃烧完全,同时尽可能减少各种热损失,以节约燃料,是一项很重要的工作。



\lianxi

(1) 说一个物体含有多少热量,这种说法对吗? 为什么?

(2) 两杯质量相等的水。降低的温度不同,放出的热量是不是相等?
    两杯质量不等的水,降低的温度相同,放出的热量是不是相等?

(3) 120 克的水,温度从 $20\celsius$ 升高到 $100\celsius$,吸收了多少热量?

(4) 250 克的水,温度从 $90\celsius$ 降低到 $40\celsius$,放出了多少热量?

(5) 完全燃烧多少木炭,能放出 464 千卡热量?

