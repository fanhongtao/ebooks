\section{电流}\label{sec:7-4}

在图 \ref{fig:7-8} 的实验里,用金属棒把带电的验电器 $A$ 和不带电的验电器 $B$ 连接起来,
就有一部分电荷从验电器 $A$ 通过金属棒传到验电器 $B$ 上去。这表示在金属棒中发生了电荷的移动。

\textbf{电荷的定向移动形成电流}。

重作图 \ref{fig:7-8} 的实验,会看到不带电的验电器 $B$ 的金属箔张开到一定的角度,就不再张大了,
表明电荷不再通过金属棒往验电器 $B$ 上移动了,金属棒中不再有电流了。
这种情况下金属棒中的电流只存在一瞬间。

我们通常需要的是持续的电流。
例如,我们利用电来照明或者利用电来使电动机转动,都需要电流长时间地持续存在。
那么,怎样才能得到持续的电流呢?

接下手电筒的接钮,手电筒的小灯泡就发光。
小灯泡持续发光,表明我们得到的电流是持续的。
这个电流能持续存在,是因为电筒里有干电池不断供电的缘故。
如果电筒里没有干电池,无论怎样按按钮,灯都不亮。

闭合电动机的开关,电动机就转个不停,表明电流是持续的。
这个电流能持续存在,是因为发电厂的发电机不断供电的缘故。
如果发电机停止运行,即使闭合了电动机的开关,电动机也不旋转。

象干电池、发电机这些能够持续供电的装置,叫做电源。
\CJKunderwave{要得到持续的电流必须有电源}。

电流的方向是怎样的呢?导体中的电流是自由电荷定向移动形成的。
定向移动的既可能是正电荷(如正离子),也可能是负电荷(如负离子或电子),还可能是正负电荷同时向相反方向移动。
在科学上规定电流方向的时候,人们还不了解各种导体的导电情况,把\textbf{正电荷定向移动的方向规定为电流的方向}。

金属导体中形成电流的是带负电的自由电子的定向移动。
根据电流方向的规定知道,\CJKunderwave{金属导体中的电流方向跟自由电子的实际移动方向相反}。
但是,正负电荷向相反方向移动所产生的效果相同,
因此,金属导体中的电流方向虽然跟自由电子的实际移动方向相反,却不影响我们研究有关电流的问题。

