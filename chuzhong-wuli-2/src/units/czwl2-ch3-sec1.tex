\section{热量}\label{sec:3-1}

热传递过程中,低温物体吸收了热,温度升高;高温物体放出了热,温度降低。
物体吸收或者放出的热的多少叫做\textbf{热量}。

为了计量热量,要确定热量的单位,下面就来讨论这个问题。

把一壶冷水放在火炉上加热。
如果只想得到温水,烧一会儿就行了;
如果想得到热水,就需要烧较长的时间。
这表明,水的温度升得越高,需要吸收的热量越多。

但是,要确定物体温度升高时吸收的热量,只考虑物体温度升高的度数是不够的。
我们知道,烧开一满壶水比烧开半壶水需要的时间长。
这表明,质量不等的水升高相同的温度,吸收的热量并不相等,水的质量越大,需要吸收的热量越多。

由于水温度升高时吸收的热量,跟水的质量和温度升高的度数都有关系,
因此,人们把 \textbf{1 克水温度升高 $1\celsius$ 时吸收的热量}作为热量的单位,这个单位叫做\textbf{卡}。
\footnote{卡不是国际单位制的单位,但是目前它的应用还很广。国际单位制的热量单位,将在第五章里介绍。}

实验证明,1 克水温度降低 $1\celsius$ 放出的热量,跟温度升高 $1\celsius$ 吸收的热量相等,也是 1 卡。

卡是一个很小的热量单位。把一壶冷水烧开,需要的热量就是几十万卡。
要计算大量的热,用卡作单位很不方便。
因此,生产上常用 1 卡的 1000 倍来作热量的单位,这个单位叫做千卡。
$$ 1\qianka = 1000\ka \;\juhao$$

因为 1 千克是 1 克的 1000 倍,所以
\CJKunderwave{1 千克水温度升高或者降低 $1\celsius$ 时吸收或者放出的热量是 1 千卡}。

根据卡和千卡的定义,可以计算水的温度改变时吸收或者放出的热量。

例如,给 2 千克水加热,使它的温度从 $20\celsius$ 升高到 $80\celsius$,水吸收的热量可以这样计算:

我们已经知道,1 千克水,温度升高 $1\celsius$,吸收的热量是 1 千卡。
那么,2 千克水,温度升高 $1\celsius$,吸收的热量是 1 千卡的 2 倍,即 $1\qianka \times 2 = 2\qianka$。
2 千克水,温度升高 $80\celsius - 20\celsius = 60\celsius$,吸收的热量是 2 千卡的 60 倍,
因此吸收的热量是 $2\qianka \times 60 = 120\qianka$。

