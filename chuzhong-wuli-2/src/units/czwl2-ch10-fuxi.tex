\section*{复习题}

(1) 什么叫磁性?什么叫磁体?什么是磁体的磁极?磁极间有何相互作用?

(2) 从什么事实可以知道磁体的周围存在着磁场?条形磁铁和蹄形磁铁周围的磁力线分布是怎样的?

(3) 从什么实验知道通电导体的周围存在着磁场?

(4) 直线电流的磁力线方向跟电流方向之间的关系,用什么方法来判定?

(5) 用什么方法来判断通电螺线管的哪一端是 $N$ 极,哪一端是 $S$ 极?

(6) 什么叫电磁铁?它有哪些优点?举几个电磁铁应用的实例。

(7) 电磁继电器有哪些应用?

(8) 通电导体在磁场里受到的力的方向跟什么有关系?

(9) 直流电动机是根据什么原理制成的?

(10) 什么叫电磁感应?感生电流的方向跟什么有关系?

(11) 发电机是根据什么原理制成的?

