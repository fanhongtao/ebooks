\section{升华和凝华}\label{sec:4-5}

物质不但可以在固态和液态之间,或者在液态和气态之间进行变化,也可以直接在固态和气态之间进行变化。
物质从固态直接变成气态叫做\textbf{升华},从气态直接变成固态叫做\textbf{凝华}。

升华和凝华的现象是经常可以遇到的。
冬天,晾在室外的湿衣服里的水会结成冰,但冰冻的湿衣服也会干,就是因为冰直接变成了水蒸气。
在冬天的早晨,我们常常看到霜。霜就是空气中的水蒸气直接凝华而成的。

用固态的碘很容易看到升华和凝华现象。
把少量的碘放进烧瓶里,微微加热,固态的碘就升华,产生紫色的碘蒸气。
停止加热后,会在烧瓶壁上看到疑华成的固态碘。
用久了的电灯泡会发黑,是因为钨丝受热产生升华现象,然后钨的气体又在灯泡壁上凝华的缘故。

空气里总是含有水蒸气的。
当含有很多水蒸气的空气升入高空时,水蒸气温度降低就要疑结成小水滴或凝华成小冰晶。
天空中的云就是由大量的小水滴和小冰晶形成的。
在一定条件下,小水滴和小冰晶越来越大,达到一定程度时就会下落。
在下落过程中冰晶熔解成水滴,与原来的水滴一起落到地面,这就形成了雨。

物质在升华过程中要吸收热量,在凝华过程中要放出热量。
生产中可利用升华吸热的现象来取得低温。
例如,在实验室里,常用固态二氧化碳(干冰)的升华吸热来获得低温。



\lianxi

(1) 冬天可以看到呼出的“ 白气” ,而在夏天却看不见,为什么?

(2) 在北方的冬天,戴眼镜的人从外面进到嗳和的屋子里,镜片上会出现一层小水珠,为什么?

(3) 夏天在箱子里放些卫生球(用萘制的),用来预防虫蛀,过几个月后再看,卫生球变小或消失了。解释这个现象。

(4) 在很冷的冬夜里,房间门窗玻璃的内表面往往结一层冰花。试说明冰花是怎样产生的。

