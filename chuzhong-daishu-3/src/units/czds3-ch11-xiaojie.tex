\xiaojie

一、本章主要内容是一元二次方程及其解法,一元二次方程的应用,一元二次方程的根的判别式、根与系数的关系,
可化为一元二次方程的高次方程、分式方程、无理方程,简单的二元二次方程组的解法。


二、我们已学过一些整式方程(一元一次方程、一元二次方程、简单的高次方程),分式方程,
无理方程以及二元一次方程组、二元二次方程组。
在这些方程(组)中,对未知数只进行加、减、乘、除、乘方、开方运算。
解这些方程(组)的基本思想是:

1. 高次方程 $\xrightarrow{\text{降次\phantom{工}}}$ 一次方程或二次方程;

2. 分式方程 $\xrightarrow{\text{去分母}}$ 整式方程;

3. 无理方程 $\xrightarrow{\text{去根号}}$ 有理方程;

4. 多元方程 $\xrightarrow{\text{消元\phantom{工}}}$ 一元方程。


三、本章介绍了一元二次方程的四种解法——直接开平方、配方法、公式法和因式分解法。
一般说来,公式法对于解任何一元二次方程都适用,是解一元二次方程的主要方法。
但在解题时,应分折方程的特点选用适当的方法。


四、一元二次方程 $ax^2 + bx + c = 0$
当 $b^2 - 4ac > 0$ 时,有两个不相等的实数根;
当 $b^2 - 4ac = 0$ 时,有两个相等的实数根;
当 $b^2 - 4ac < 0$ 时,没有实数根。


\begin{enhancedline}
五、如果一元二次方程 $ax^2 + bx + c = 0 \quad (a \neq 0)$ 的两个根是 $x_1$,$x_2$,
那么 $x_1 + x_2 = -\dfrac{b}{a}$,$x_1 \cdot x_2 = \dfrac{c}{a}$。

并由此得出:
\end{enhancedline}

以两个数 $x_1$,$x_2$ 为根的一元二次方程(二次项系数为 $1$)是
$$ x^2 - (x_1 + x_2)x + x_1x_2 = 0 \juhao $$


六、解分式方程时,需将方程的两边都乘以各分式的最简公分母,使之变形为整式方程;
解无理方程时,需将方程的两边乘方相同的次数,使之变形为有理方程。
这两种变形都有可能产生增根。因此,必须检验从变形后的方程求得的根是否为原方程的根。


七、解简单的二元二次方程组,通常用代入法、加减法和因式分解法等进行消元或降次。
对于某些特殊形式的方程组,有时可用换元法或其他方法来解。


