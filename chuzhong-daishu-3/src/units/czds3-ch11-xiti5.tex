\xiti
\begin{enhancedline}
\begin{xiaotis}

用直接开平方法解下列方程(第 1 ~\, 3 题):

\xiaoti{}%
\begin{xiaoxiaotis}%
    \huitui\begin{tblr}[t]{columns={18em, colsep=0pt}}
        \xxt{$49x^2 - 81 = 0$;} & \xxt{$\dfrac{1}{4}y^2 = 0.01$。}
    \end{tblr}

\end{xiaoxiaotis}


\xiaoti{}%
\begin{xiaoxiaotis}%
    \huitui\begin{tblr}[t]{columns={18em, colsep=0pt}}
        \xxt{$0.2x^2 - \dfrac{3}{5} = 0$;} & \xxt{$(x + 3)(x - 3) = 9$。}
    \end{tblr}

\end{xiaoxiaotis}


\xiaoti{}%
\begin{xiaoxiaotis}%
    \huitui\begin{tblr}[t]{columns={18em, colsep=0pt}}
        \xxt{$(3x + 1)^2 = 2$;} & \xxt{$(2t + 3)^2 - 5 = 0$。}
    \end{tblr}

\end{xiaoxiaotis}


用配方法解下列方程(第 4 ~\, 5 题):

\xiaoti{}%
\begin{xiaoxiaotis}%
    \huitui\begin{tblr}[t]{columns={18em, colsep=0pt}}
        \xxt{$x^2 + 2x - 99 = 0$;} & \xxt{$y^2 + 5y + 2 = 0$。}
    \end{tblr}

\end{xiaoxiaotis}


\xiaoti{}%
\begin{xiaoxiaotis}%
    \huitui\begin{tblr}[t]{columns={18em, colsep=0pt}}
        \xxt{$3x^2 - 1 = 4x$;} & \xxt{$2x^2 + \sqrt{2}x - 30 = 0$。}
    \end{tblr}

\end{xiaoxiaotis}


\xiaoti{用配方法解关于 $x$ 的方程 $x^2 + px + q = 0$。}


\xiaoti{用公式法解下列方程:}
\begin{xiaoxiaotis}

    \begin{tblr}{columns={18em, colsep=0pt}, row{3}={rowsep=0.5em}}
        \xxt{$x^2 + 2x - 2 = 0$;}              & \xxt{$3x^2 + 4x - 7 = 0$;} \\
        \xxt{$2y^2 + 8y - 1 = 0$;}             & \xxt{$x^2 - 2.4x - 13 = 0$;} \\
        \xxt{$2x^2 - 3x + \dfrac{1}{8} = 0$;}  & \xxt{$\dfrac{3}{2}t^2 + 4t = 1$;} \\
        \xxt{$3y^2 + 1 = 2\sqrt{3}y$;}         & \xxt{$x^2 + 2(\sqrt{3} + 1)x + 2\sqrt{3} = 0$。}
    \end{tblr}
\end{xiaoxiaotis}


\xiaoti{用公式法解下列方程,并求根的近似值(精确到 $0.01$):}
\begin{xiaoxiaotis}

    \begin{tblr}{columns={18em, colsep=0pt}}
        \xxt{$x^2 - 3x - 7 = 0$;} & \xxt{$x^2 - 3\sqrt{2}x + 2 = 0$。}
    \end{tblr}
\end{xiaoxiaotis}


\xiaoti{用因式分解法解下列方程:}
\begin{xiaoxiaotis}

    \begin{tblr}{columns={18em, colsep=0pt}, row{1}={rowsep=0.5em}}
        \xxt{$8x^2 - \dfrac{1}{2}x = 3x^2 + \dfrac{1}{3}x$;} & \xxt{$\dfrac{1}{3}(y + 3)^2 = \dfrac{1}{2}(y + 3)$;} \\
        \xxt{$x^2 + 7x + 6 = 0$;}  & \xxt{$x^2 - 5x - 6 = 0$;} \\
        \xxt{$y^2 - 17y + 30 = 0$;}  & \xxt{$y^2 - 7y - 60 = 0$;} \\
        \xxt{$9(2x + 3)^2 - 4(2x - 5)^2 = 0$;} & \xxt{$(2y + 1)^2 + 3(2y + 1) + 2 = 0$。}
    \end{tblr}
\end{xiaoxiaotis}


\xiaoti{选用适当方法解下列方程:}
\begin{xiaoxiaotis}

    \begin{tblr}{columns={18em, colsep=0pt}}
        \xxt{$x^2 - 3x + 2 = 0$;}       & \xxt{$x^2 - 3x - 2 = 0$;} \\
        \xxt{$x^2 + 12x + 27 = 0$;}     & \xxt{$(x - 1)(x + 2) = 70$;} \\
        \xxt{$(3 - t)^2 + t^2 = 9$;}    & \xxt{$(y - 2)^2 = 3$;} \\
        \xxt{$(2x + 3)^2 = 3(4x + 3)$;} & \xxt{$(y + \sqrt{3})^2 = 4\sqrt{3}y$;} \\
        \xxt{$(2x - 1)(x + 3) = 4$;}    & \xxt{$(y + 1)(y - 1) = 2\sqrt{2}y$;} \\
        \xxt{$x^2 - \sqrt{3}x - \sqrt{2}x + 6 = 0$;} & \xxt{$3x(x - 1) = 2 - 2x$。}
    \end{tblr}
\end{xiaoxiaotis}


\xiaoti{解下列关于 $x$ 的方程:}
\begin{xiaoxiaotis}

    \xxt{$mx^2 - (m - n)x - n = 0 \quad (m \neq 0)$;}

    \xxt{$x^2 - (2m + 1)x + m^2 + m = 0$;}

    \xxt{$(x + a)(x - b) + (x - a)(x + b) = 2a(ax - b)$;}

    \xxt{$abx^2 - (a^4 + b^4)x + a^3b^3 = 0 \quad (ab \neq 0)$。}

\end{xiaoxiaotis}


\xiaoti{已知 $y = x^2 - 2x - 3$。
    $x$ 是什么数时,$y$ 的值等于零?
    $x$ 是什么数时,$y$ 的值等于 $-4$?
}

\xiaoti{$x$ 是什么数时,$x^2 + 6x + 5$ 的值和 $x - 1$ 的值相等?}

\xiaoti{已知 $x^2 - 7xy + 12y^2 = 0$,求证 $x = 3y$ 或 $x = 4y$。}

\xiaoti{不解方程,判别下列方程的根的情况:}
\begin{xiaoxiaotis}

    \begin{tblr}{columns={18em, colsep=0pt}}
        \xxt{$2x^2 + 4x + 35 = 0$;} & \xxt{$4m(m - 1) + 1 = 0$;} \\
        \xxt{$0.2x^2 - 5 = \dfrac{3}{2}x$;} & \xxt{$4(y^2 + 0.9) = 2.4y$;} \\
        \xxt{$\dfrac{1}{2}x^2 - \sqrt{2} = \sqrt{3}x$;} & \xxt{$2t = \sqrt{5}\left(t^2 + \dfrac{1}{5}\right)$。}
    \end{tblr}
\end{xiaoxiaotis}


\xiaoti{$m$ 取什么值时,方程 $x^2 + (2m + 1)x + (m - 2)^2 = 0$}%
\begin{xiaoxiaotis}
    \xxt[\xxtsep]{有两个不相等的实数根?}%
    \xxt[\xxtsep]{有两个相等的实数根?}%
    \xxt[\xxtsep]{没有实数根?}
\end{xiaoxiaotis}


\xiaoti{$k$ 取什么值时,方程 $4x^2 - (k + 2)x + k - 1 = 0$
    有两个相等的实数根?并求出这时方程的根。
}

\xiaoti{证明方程 $(x - 1)(x - 2) = k^2$ 有两个不相等的实数根。}

\end{xiaotis}
\end{enhancedline}

