\subsection{立方根}\label{subsec:9-4}
\begin{enhancedline}

我们来看下面的问题:

要做一只正方体的木箱,使它的容积是 $125$ $\lffm$,这个木箱的棱长应当是多少分米?

因为正方体的容积等于棱长的立方,如果设棱长为 $x$ 分米,根据题意,得
$$ x^3 = 125 \juhao $$

这就是要求出一个数,使它的立方等于 $125$。

因为 $5^3 = 125$, 所以,这个正方体木箱的棱长是 $5$ 分米。

一般地,如果一个数的立方等于 $a$,这个数就叫做 $a$ 的\zhongdian{立方根}(也叫做\zhongdian{三次方根})。
换句话说,如果 $x^3 = a$,那么 $x$ 叫做 $a$ 的立方根。
数 $a$ 的立方根用符号 “$\sqrt[3]{a}$” 表示, 读作 “三次根号 $a$”,其中 $a$ 是被开方数,$3$ 是根指数。
例如, $5^3 = 125$, $5$ 就是 $125$ 的立方根,用式子表示就是 $\sqrt[3]{125} = 5$。

求一个数的立方根的运算,叫做\zhongdian{开立方}。开立方与立方互为逆运算。

\liti 求下列各数的立方根:
\begin{xiaoxiaotis}

    \hspace*{1.5em} \begin{tblr}{columns={12em, colsep=0pt}}
        \xxt{$-8$;} & \xxt{$8$;}   & \xxt{$-\dfrac{3}{27}$;}     \\
        \xxt{$0.216$;} & \xxt{$0$。}
    \end{tblr}

\resetxxt
\jie \xxt{\begin{tblr}[t]{}
    $\because$   & $(-2)^3 = -8$, \\
    $\therefore$ & $-8$ 的立方根是 $-2$,即 \\
                 & $\sqrt[3]{-8} = -2$;
\end{tblr}}

\hspace*{1.5em} \xxt{\begin{tblr}[t]{}
    $\because$   & $2^3 = 8$, \\
    $\therefore$ & $8$ 的立方根是 $2$,即 \\
                 & $\sqrt[3]{8} = 2$;
\end{tblr}}

\hspace*{1.5em} \xxt{\begin{tblr}[t]{}
    $\because$   & $\left(-\dfrac{2}{3}\right)^3 = -\dfrac{8}{27}$, \\
    $\therefore$ & $-\dfrac{8}{27}$ 的立方根是 $-\dfrac{2}{3}$,即 \\
                 & $\sqrt[3]{-\dfrac{8}{27}} = -\dfrac{2}{3}$;
\end{tblr}}

\hspace*{1.5em} \xxt{\begin{tblr}[t]{}
    $\because$   & $0.6^3 = 0.216$, \\
    $\therefore$ & $0.216$ 的立方根是 $0.6$,即 \\
                 & $\sqrt[3]{0.216} = 0.6$;
\end{tblr}}

\hspace*{1.5em} \xxt{\begin{tblr}[t]{}
    $\because$   & $0^3 = 0$, \\
    $\therefore$ & $0$ 的立方根是 $0$,即 \\
                 & $\sqrt[3]{0} = 0$。
\end{tblr}}

\end{xiaoxiaotis}


从 例 1 可以看出,正数 $8$ 与 $0.216$ 分别有一个正的立方根 $2$ 与 $0.6$;
负数 $-8$ 与 $-\dfrac{8}{27}$ 分别有一个负的立方根 $-2$ 与 $-\dfrac{2}{3}$;
零有立方根零。一般地,
\zhongdian{正数有一个正的立方根; 负数有一个负的立方根;零的立方根仍旧是零。}

从 例 1 还可以看出,

\hspace*{5.5em} \begin{tblr}[t]{}
    $\because$   & $\sqrt[3]{-8} = -2$, \\
                 & $\sqrt[3]{8}  = 2$; \\
    $\therefore$ & $\sqrt[3]{-8} = -\sqrt[3]{8}$。
\end{tblr}

一般地, 如果 $a > 0$, 那么,$\sqrt[3]{-a} = -\sqrt[3]{a}$。
所以,求负数的立方根,只要先求出这个负数的绝对值的立方根,然后再取它的相反数。

\lianxi
\begin{xiaotis}

\xiaoti{}%
\begin{xiaoxiaotis}%
    \xxt[\xxtsep]{写出1、2、3、4、5、6、7、8、9 这九个数的立方;}

    \xxt{利用第 (1) 小题的结果,求下列各数的立方根:\\
        $27$\nsep $-64$\nsep $343$\nsep $-729$。
    }

\end{xiaoxiaotis}

\xiaoti{求下列各数的立方根: \\
    $1$\nsep $512$\nsep $\dfrac{8}{27}$\nsep $-\dfrac{27}{64}$\nsep $\dfrac{1}{8}$\nsep
    $-\dfrac{1}{8}$\nsep $0.125$\nsep $-15\dfrac{5}{8}$。
}

\end{xiaotis}

\lianxijiange

\liti 求下列各式的值:
\begin{xiaoxiaotis}

    \hspace*{1.5em} \begin{tblr}{columns={8em, colsep=0pt}}
        \xxt{$\sqrt[3]{27}$;} & \xxt{$\sqrt[3]{-27}$;}  & \xxt{$-\sqrt[3]{2\dfrac{10}{27}}$;} & \xxt{$-\sqrt[3]{-\dfrac{27}{64}}$。}
    \end{tblr}

\resetxxt
\jie \begin{tblr}[t]{columns={colsep=0pt}}
    \xxt{$\sqrt[3]{27} = 3$;} \\
    \xxt{$\sqrt[3]{-27} = -\sqrt[3]{27} = -3$;} \\
    \xxt{$-\sqrt[3]{2\dfrac{10}{27}} = -\sqrt[3]{\dfrac{64}{27}} = -\dfrac{4}{3} = -1\dfrac{1}{3}$;} \\
    \xxt{$-\sqrt[3]{-\dfrac{27}{64}} = \sqrt[3]{\dfrac{27}{64}} = \dfrac{3}{4}$。}
\end{tblr}

\end{xiaoxiaotis}


\lianxi
\begin{xiaotis}
\xiaoti{求下列各式的值:\\
    $\sqrt[3]{1000}$\nsep   $-\sqrt[3]{0.001}$\nsep  $\sqrt[3]{-\dfrac{64}{125}}$\nsep
    $-\sqrt[3]{-216}$\nsep  $\sqrt[3]{-1}$\nsep      $\sqrt[3]{3\dfrac{3}{8}}$。
}

\xiaoti{填写下表:\\
    \begin{tblr}{vlines, hlines, columns={colsep=1em, c, $$}, column{1,4}={2em}}
        a           & 0.000001  & 0.001 & 1 & 1000  & 1000000 \\
        \sqrt[3]{a} &           &       &   &       & \\
    \end{tblr} \\
    观察上表并说明当已知数 $a$ 的小数点向右(或向左)每移动三位,它的立方根 $\sqrt[3]{a}$ 的小数点的移动规律是怎样的?
}

\xiaoti{求下列各式的值:\\
    $\sqrt[3]{-125}$\nsep $\sqrt[3]{0.729}$\nsep  $-\sqrt[3]{-\dfrac{125}{216}}$\nsep
    $\sqrt[3]{4 + \dfrac{17}{27}}$\nsep  $\sqrt[3]{\dfrac{37}{64} - 1}$。
}

\end{xiaotis}
\lianxijiange


现在,我们把平方根和立方根的概念加以推广。
如果一个数的 $n$ 次方( $n$ 是大于 $1$ 的整数) 等于 $a$,这个数就叫做 $a$ 的 \zhongdian{$n$ 次方根}。
换句话说,如果 $x^n = a$, 那么 $x$ 叫做 $a$ 的 $n$ 次方根。
求 $a$ 的 $n$ 次方根的运算,叫做把 \zhongdian{$a$ 开 $n$ 次方},
$a$ 叫做\zhongdian{被开方数},
$n$ 叫做\zhongdian{根指数}。

把一个数开奇次方时,求得的方根叫做奇次方根;
把一个数开偶次方时,求得的方根叫做偶次方根。


我们知道,$2^2 = 4$, $(-2)^2 = 4$, 就是说 $2$ 与 $-2$ 是 $4$ 的平方根;
$2^4 = 16$, $(-2)^4 = 16$ ,就是说 $2$ 与 $-2$ 是 $16$ 的四次方根。
一般地,正数的偶次方根有两个,它们互为相反数。
当 $n$ 是偶数时,正数 $a$ 的正的 $n$ 次方根用符号 “$\sqrt[n]{a}$” 表示,
负的 $n$ 次方根用符号 “$-\sqrt[n]{a}$” 表示,
也可以把两个方根合起来写作 $\pm \sqrt[n]{a}$。
例如 $\sqrt[4]{16} = 2$, $-\sqrt[4]{16} = -2$,
合起来写作 $\pm \sqrt[4]{16} = \pm 2$。

我们还知道,$2^3 = 8$ , $2$ 是 $8$ 的立方根;
$(-2)^3 = -8$, $-2$ 是 $-8$ 的立方根;
$2^5 = 32$, $2$ 是 $32$ 的五次方根;
$(-2)^5 = -32$, $-2$ 是 $-32$ 的五次方根。
一般地,正数的奇次方根是一个正数,负数的奇次方根是一个负数。
当 $n$ 是奇数时, $a$ 的 $n$ 次方根用符号 “$\sqrt[n]{a}$” 表示。
例如, $\sqrt[3]{27} = 3$, $\sqrt[5]{-32} = -2$。

零的 $n$ 次方根记作 $\sqrt[n]{0}$, $\sqrt[n]{0} = 0$。

正数 $a$ 的正的 $n$ 次方根叫做 $a$ 的 \zhongdian{$n$ 次算术根}。
零的 $n$ 次方根也叫做零的算术根。

求一个数的方根的运算,叫做开方。
很明显,开 $n$ 次方与 $n$ 次方互为逆运算。

\end{enhancedline}

