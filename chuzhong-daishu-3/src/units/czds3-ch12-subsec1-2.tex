\subsubsection{负整数指数}

\begin{enhancedline}
同底数的幂相除,如果被除式的指数小于除式的指数,我们可以通过约分来计算。例如,
\begin{gather*}
    5^2 \div 5^6 = \dfrac{5^2}{5^6} = \dfrac{5^2}{5^2 \times 5^4} = \dfrac{1}{5^4} \douhao \\
    a^3 \div a^5 = \dfrac{a^3}{a^5} = \dfrac{a^3}{a^3 \cdot a^2} = \dfrac{1}{a^2} \quad (a \neq 0) \juhao
\end{gather*}

可以看到,同底数的幂相除,当被除式的指数比除式的指数小 $p$ 时,所得的商是一个分数或分式,
分子是 $1$, 分母是同底数的 $p$ 次幂。
\end{enhancedline}

另一方面,如果仿照幂的运算性质 (2) 计算这两个例子,用被除式的指数减去除式的指数,就得
\begin{gather*}
    5^2 \div 5^6 = 5^{2 - 6} = 5^{-4} \douhao \\
    a^3 \div a^5 = a^{3 - 5} = a^{-2} \quad (a \neq 0) \juhao
\end{gather*}
这时就出现了负整数指数。

\begin{enhancedline}
为了使被除式的指数小于除式的指数时,同底数幂除法的运算性质也能适用,我们规定负整数指数幂的意义是
\begin{center}
    \framebox{\quad $\bm{a^{-p} = \dfrac{1}{a^p} \quad (a \neq 0 \douhao p \; \text{\zhongdian{是正整数}})}$。\quad}
\end{center}

这就是说,任何不等于零的实数的 $-p$ ( $p$ 是正整数)次幂,等于这个数的 $p$ 次幂的倒数。

这样规定以后,上面的例子就可以这样来计算:
\begin{gather*}
    5^2 \div 5^6 = 5^{2 - 6} = 5^{-4} = \dfrac{1}{5^4}\douhao \\
    a^3 \div a^5 = a^{3 - 5} = a^{-2} = \dfrac{1}{a^2} \quad (a \neq 0) \juhao
\end{gather*}

应当注意,零的负整数次幂没有意义。

规定了零指数幂与负整数指数幂的意义,就把指数从正整数推广到了整数。
正整数指数幂的运算性质对整数指数幂都适用。例如,
\begin{gather*}
    a^3 \cdot a^0 = a^{3 + 0} = a^3 \quad (a \neq 0) \douhao \\
    a^{-3} \cdot a^2 = a^{-3 + 2} = a^{-1} = \dfrac{1}{a} \quad (a \neq 0) \douhao \\
    (a^{-3})^2 = a^{-3 \times 2} = a^{-6} = \dfrac{1}{a^6} \quad (a \neq 0) \juhao
\end{gather*}

在本章里,当指数是零或负数时,如果没有特别说明,底数都不等于零。

\liti 计算:

\hspace*{3em} $10^{-3} \nsep (-3)^{-2} \nsep \left(\dfrac{1}{2}\right)^{-3} \nsep 5^0 \times (-2)^{-1}$。

\jie \begin{tblr}[t]{columns={18em, mode=math}, rows={rowsep=0.5em}}
    10^{-3} = \dfrac{1}{10^3} = \dfrac{1}{1000} \douhao &
        (-3)^{-2} = \dfrac{1}{(-3)^2} = \dfrac{1}{9} \douhao \\
    \left(\dfrac{1}{2}\right)^{-3} = \dfrac{1}{\left(\dfrac{1}{2}\right)^3} = 8 \douhao &
        5^0 \times (-2)^{-1} = 1 \times \dfrac{1}{-2} = -\dfrac{1}{2} \juhao
\end{tblr}


\liti 用小数表示下列各式:

\hspace*{3em} $10^{-5} \nsep 7 \times 10^{-6} \nsep 3.6 \times 10^{-8}$。

\jie \begin{tblr}[t]{columns={mode=math}}
    10^{-5} = \dfrac{1}{10^5} = 0.00001 \douhao \\
    7 \times 10^{-6} = 7 \times \dfrac{1}{10^6} = 7 \times 0.000001 = 0.000007 \douhao \\
    3.6 \times 10^{-8} = 3.6 \times \dfrac{1}{10^8} = 3.6 \times 0.00000001 = 0.000000036 \juhao
\end{tblr}


\lianxi
\begin{xiaotis}

\xiaoti{(口答)下列各式的结果是什么?}
\begin{xiaoxiaotis}

    \begin{tblr}{columns={12em, colsep=0pt}}
        \xxt{$3a^2b + 2a^2b$;} & \xxt{$3a^2b \cdot 2a^2b$;} & \xxt{$(3ab^2)^2$;} \\
        \xxt{$\left(-\dfrac{2b}{a^3}\right)^3$;} & \xxt{$16a^4b^2 \div 12a^2b^2$;} & \xxt{$(a^2b^2)^3 \div a^2b$。}
    \end{tblr}
\end{xiaoxiaotis}


\xiaoti{计算:\\
    $3^0$\nsep  $3^{-1}$\nsep  $10^{-4}$\nsep  $(\sqrt{2})^0$  \nsep $7^{-2}$\nsep
    $1^{-10}$\nsep  $(-2)^{-3}$\nsep  $\left(\dfrac{1}{2}\right)^{-4}$\nsep  $(-0.1)^0$\nsep  $\left(-\dfrac{1}{2}\right)^{-3}$。
}


\xiaoti{计算:}
\begin{xiaoxiaotis}

    \begin{tblr}{columns={18em, colsep=0pt}}
        \xxt{$(-2)^3 - (-1)^0$;} & \xxt{$2^{-2} + (-2)^{-3}$;} \\
        \xxt{$\left(\dfrac{1}{2}\right)^{-2} \div \left(\dfrac{1}{2}\right)^0$;} & \xxt{$\left(-\dfrac{1}{2}\right)^{-2} \times 2^{-1}$。}
    \end{tblr}
\end{xiaoxiaotis}


\xiaoti{用小数表示下列各式:}
\begin{xiaoxiaotis}

    \begin{tblr}{columns={18em, colsep=0pt}}
        \xxt{$2 \times 10^{-5}$;} & \xxt{$3.1 \times 10^{-7}$;} \\
        \xxt{$8.04 \times 10^{-3}$;} & \xxt{$1.205 \times 10^{-2}$;} \\
        \xxt{$2.12 \times 10^{-3}$;} & \xxt{$2.12 \times 10^{-2}$;} \\
        \xxt{$2.12 \times 10^{-1}$;} & \xxt{$2.12 \times 10^0$。}
    \end{tblr}
\end{xiaoxiaotis}

\end{xiaotis}
\lianxijiange


\liti 计算 $(-a)^{-5}$,$a^{-2}b^{-1}(-2a^3)$,$(-5a^3b^{-1})^{-2}$,
并且把结果化成只含有正整数指数的式子。

\jie \begin{tblr}[t]{columns={mode=math}}
    (-a)^{-5} = \dfrac{1}{(-a)^5} = -\dfrac{1}{a^5} \douhao \\
    a^{-2}b^{-1}(-2a^3) = -2a^{-2 + 3}b^{-1} = -2ab^{-1} = -\dfrac{2a}{b} \douhao \\
    (-5a^3b^{-1})^{-2} = (-5)^{1 \times (-2)} a^{3 \times (-2)} b^{(-1) \times (-2)} = (-5)^{-2}a^{-6}b^2 = \dfrac{1}{(-5)^2} \times \dfrac{1}{a^6} \times b^2 = \dfrac{b^2}{25a^6} \juhao
\end{tblr}


\liti 计算:
\begin{xiaoxiaotis}

    \hspace*{1.5em} \begin{tblr}{columns={colsep=0pt}}
        \xxt{$\dfrac{a^{-2}b^{-3}(-3a^{-1}b^2)}{6a^{-3}b^{-2}}$;} &
            \xxt{$(x^{-2} + y^{-2})(x^{-2} - y^{-2})$;} &
            \xxt{$\dfrac{a^{-1} + b^{-1}}{a^{-1} \cdot b^{-1}}$。}
    \end{tblr}

\resetxxt
\jie \xxt{$\dfrac{a^{-2}b^{-3}(-3a^{-1}b^2)}{6a^{-3}b^{-2}} = -\dfrac{3}{6}a^{-2 + (-1) - (-3)}b^{-3 + 2 - (-2)} = -\dfrac{1}{2}a^0b = -\dfrac{1}{2}b$;}

\hspace*{1.5em} \xxt{$(x^{-2} + y^{-2})(x^{-2} - y^{-2}) = (x^{-2})^2 - (y^{-2})^2 = x^{-4} - y^{-4}$;}

\hspace*{1.5em} \xxt{$\dfrac{a^{-1} + b^{-1}}{a^{-1} \cdot b^{-1}} = \dfrac{(a^{-1} + b^{-1})ab}{(a^-1 \cdot b^-1)ab} = \dfrac{b + a}{1} = a + b$;}

\end{xiaoxiaotis}


\lianxi
\begin{xiaotis}

\xiaoti{计算下列各式,并且把结果化成只含有正整数指数的式子:}
\begin{xiaoxiaotis}

    \begin{tblr}{columns={18em, colsep=0pt}}
        \xxt{$\dfrac{ab}{c^{-2}}$;} & \xxt{$pq^{-2}r^{-1}$;} \\
        \xxt{$\dfrac{a(a + b)^{-1}}{a^{-2}b}$;} & \xxt{$\dfrac{5^{-1}xy^{-2}}{2^{-3}ab^{-4}}$。}
    \end{tblr}
\end{xiaoxiaotis}


\xiaoti{利用负整数指数把下列各式化成不含分母的式子:}
\begin{xiaoxiaotis}

    \begin{tblr}{columns={12em, colsep=0pt}}
        \xxt{$\dfrac{1}{y^5}$;} & \xxt{$\dfrac{a^2}{b^3}$;} & \xxt{$\dfrac{m^2}{x^6y}$。}
    \end{tblr}
\end{xiaoxiaotis}


\xiaoti{(口答)下列计算是否正确?如果不正确,应如何改正?}
\begin{xiaoxiaotis}

    \begin{tblr}{columns={18em, colsep=0pt}}
        \xxt{$(-1)^0 = -1$;} & \xxt{$(-1)^{-1} = 1$;} \\
        \xxt{$3a^{-2} = \dfrac{1}{3a^2}$;} & \xxt{$(-x)^5 \div (-x)^3 = -x^2$。}
    \end{tblr}
\end{xiaoxiaotis}


\xiaoti{计算下列各式,并且把结果化成只含有正整数指数的式子:}
\begin{xiaoxiaotis}

    \begin{tblr}{columns={12em, colsep=0pt}}
        \xxt{$3^{-5} \cdot 3^6$;} & \xxt{$7^{-9} \div 7^{-10}$;} & \xxt{$a^{-3} \cdot a^2$;} \\
        \xxt{$b^{-4} \div b^{-2}$;} & \xxt{$(a^{-3})^{-2}$;} & \xxt{$(x^{-2})^0$;} \\
        \xxt{$(xy)^{-2}$;} & \xxt{$\left(\dfrac{p}{q}\right)^{-2}$。}
    \end{tblr}
\end{xiaoxiaotis}


\xiaoti{计算:}
\begin{xiaoxiaotis}

    \begin{tblr}{columns={18em, colsep=0pt}}
        \xxt{$(x^4y^{-3}) \cdot (x^{-2}y^2)$;} & \xxt{$3a^{-2}b^{-3} \div 3^{-1}a^2b^{-3}$;} \\
        \xxt{$\left(\dfrac{3^{-5} \cdot 3^2}{3^{-3}}\right)^{-2}$;} & \xxt{$\dfrac{(x^{-1} + y^{-1})(x^{-1} - y^{-1})}{x^{-2}y^{-2}}$。}
    \end{tblr}
\end{xiaoxiaotis}

\end{xiaotis}
\end{enhancedline}
