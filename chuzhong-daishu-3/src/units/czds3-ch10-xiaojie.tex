\xiaojie
\begin{enhancedline}

一、本章主要内容是二次根式的性质与运算。

二、根据算术平方根的意义,二次根式有下列性质:

\begin{gather*}
    (\sqrt{a})^2 = a \quad (a \geqslant 0) \fenhao \\
    \sqrt{a^2} = |a| = \begin{cases}
        \phantom{-}a  \quad (a > 0) \douhao \\
        \phantom{-}0  \quad (a = 0) \douhao \\
                 -a   \quad (a < 0) \fenhao
    \end{cases} \\
    \sqrt{ab} = \sqrt{a} \cdot \sqrt{b} \quad (a \geqslant 0,\; b \geqslant 0) \fenhao \\
    \sqrt{\dfrac{a}{b}} = \dfrac{\sqrt{a}}{\sqrt{b}} \quad (a \geqslant 0,\; b > 0) \juhao \\
\end{gather*}

三、二次根式的性质是二次根式运算和化简的根据。

最简二次根式就是满足下列条件的二次根式:

(1) 被开方数的每一个因式的指数都小于根指数 $2$;

(2) 被开方数不含分母。

同类二次根式就是几个化成最简二次根式以后,被开方数相同的二次根式。

二次根式的加减法就是去括号与合并同类二次根式。

二次根式的乘法就是运用公式
$\sqrt{a} \cdot \sqrt{b} = \sqrt{ab} \quad (a \geqslant 0,\; b \geqslant 0)$
并参照多项式乘法法则进行运算。

二次根式的除法有时可以运用公式
$\dfrac{\sqrt{a}}{\sqrt{b}} = \sqrt{\dfrac{a}{b}} \quad (a \geqslant 0,\; b > 0)$
进行运算;一般是先写成分式的形式,然后通过分母有理化或约分进行运算。

运算结果中的二次根式,一般都要化成最简二次根式。

\end{enhancedline}
