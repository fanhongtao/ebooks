\subsubsection{零指数}

我们知道,同底数的幂相除,如果被除式的指数等于除式的指数,也就是被除式等于除式,
那么所得的商等于 $1$。例如,
\begin{gather*}
    5^2 \div 5^2 = 1 \douhao \\
    a^3 \div a^3 = 1 \quad (a \neq 0) \juhao
\end{gather*}

另一方面,如果仿照上面的运算性质 (2) 计算这两个例子,用被除式的指数减去除式的指数,就得
\begin{gather*}
    5^2 \div 5^2 = 5^{2 - 2} = 5^0 \douhao \\
    a^3 \div a^3 = a^{3 - 3} = a^0 \quad (a \neq 0) \juhao
\end{gather*}
这时就出现了零指数。

为了使被除式的指数等于除式的指数时,同底数幂除法的运算性质也能适用,我们规定零指数幂的意义是
\begin{center}
    \framebox{\quad $\bm{a^0 = 1 \quad (a \neq 0)}$。\quad}
\end{center}

这就是说,任何不等于零的实数的零次幂都等于 $1$。

这样规定以后,上面的例子就可以这样来计算:
\begin{gather*}
    5^2 \div 5^2 = 5^{2 - 2} = 5^0 = 1 \douhao \\
    a^3 \div a^3 = a^{3 - 3} = a^0 = 1 \quad (a \neq 0) \juhao
\end{gather*}

应当注意,零的零次幂没有意义。

