\xiaojie


一、本章主要内容是数的开方的有关概念,平方根表、立方根表的查法,以及实数的初步概念。

二、数的概念是在实践中不断发展的。在小学里我们先学习了自然数和零;又学习了分数和小数〈有限小数和循环小数),
就是说,学习了正有理数和零;到了初一又学习了负有理数。这样,数的概念就扩充到了有理数。
现在我们又把数的概念扩充到实数。这样,我们学过的数可以列成下表:

\vspace{4em}% 被忽略的高度
$$
    \text{实数} \smash[t]{\left\{
    \begin{aligned}
        &\begin{aligned}
            \text{有理数}
        \end{aligned}
        \smash{\left\{
            \begin{aligned}
                &\text{正有理数} \\
                &\text{零} \\
                &\text{负有理数}
            \end{aligned}
        \right.}
        \\
        \, \\
        &\text{无理数}
        \smash[b]{\left\{
            \begin{aligned}
                \text{正无理数} \\
                \text{负无理数}
            \end{aligned}
        \right.}
    \end{aligned}
    \right.}
$$\vspace{1em}

三、在进行实数运算时,有理数的运算律和运算性质同样适用。

开方与乘方互为逆运算。用乘方可以检验开方的结果是否正确。

在实数范围内,正数有 $n$ 次方根;负数有奇次方根,但没有偶次方根;零的 $n$ 次方根是零。

正数的正的方根叫做算术根,零的算术根是零。

