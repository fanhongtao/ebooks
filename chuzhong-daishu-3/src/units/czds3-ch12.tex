\chapter{指数}

我们已经学习了指数是正整数的幂,知道正整数指数幂有如下运算性质:

\begin{enhancedline}
(1) \; $a^m \cdot a^n = a^{m + n}$;

(2) \; $a^m \div a^n = a^{m - n} \; (a \neq 0,\; m > n)$;

(3) \; $(a^m)^n = a^{mn}$;

(4) \; $(ab)^n = a^n \cdot b^n$;

(5) \; $\left(\dfrac{a}{b}\right)^n = \dfrac{a^n}{b^n} \; (b \neq 0)$。

现在,我们来学习指数是零、负整数和分数时幂的意义和运算。
\end{enhancedline}

% 原书的目录结构就是如此(缺少 section)
% 忽略这里的报错: Difference (2) between bookmark levels is greater (hyperref)	than one, level fixed.
\subsection{零指数与负整数指数}\label{subsec:12-1}

对于同底数的正整数指数幂的除法,我们有上面的运算性质 (2),即同底数的幂相除,指数相减:
$$ a^m \div a^n = a^{m - n} \juhao $$
由于除式不能是零,我们规定 $a \neq 0$;为了使运算的结果仍然是正整数指数幂,还要求 $m > n$。
但在实际计算中有时会出现 $m = n$ 或 $m < n$ 的情况,下面我们来分别研究这两种情况。

\subsubsection{零指数}

我们知道,同底数的幂相除,如果被除式的指数等于除式的指数,也就是被除式等于除式,
那么所得的商等于 $1$。例如,
\begin{gather*}
    5^2 \div 5^2 = 1 \douhao \\
    a^3 \div a^3 = 1 \quad (a \neq 0) \juhao
\end{gather*}

另一方面,如果仿照上面的运算性质 (2) 计算这两个例子,用被除式的指数减去除式的指数,就得
\begin{gather*}
    5^2 \div 5^2 = 5^{2 - 2} = 5^0 \douhao \\
    a^3 \div a^3 = a^{3 - 3} = a^0 \quad (a \neq 0) \juhao
\end{gather*}
这时就出现了零指数。

为了使被除式的指数等于除式的指数时,同底数幂除法的运算性质也能适用,我们规定零指数幂的意义是
\begin{center}
    \framebox{\quad $\bm{a^0 = 1 \quad (a \neq 0)}$。\quad}
\end{center}

这就是说,任何不等于零的实数的零次幂都等于 $1$。

这样规定以后,上面的例子就可以这样来计算:
\begin{gather*}
    5^2 \div 5^2 = 5^{2 - 2} = 5^0 = 1 \douhao \\
    a^3 \div a^3 = a^{3 - 3} = a^0 = 1 \quad (a \neq 0) \juhao
\end{gather*}

应当注意,零的零次幂没有意义。


\subsubsection{负整数指数}

\begin{enhancedline}
同底数的幂相除,如果被除式的指数小于除式的指数,我们可以通过约分来计算。例如,
\begin{gather*}
    5^2 \div 5^6 = \dfrac{5^2}{5^6} = \dfrac{5^2}{5^2 \times 5^4} = \dfrac{1}{5^4} \douhao \\
    a^3 \div a^5 = \dfrac{a^3}{a^5} = \dfrac{a^3}{a^3 \cdot a^2} = \dfrac{1}{a^2} \quad (a \neq 0) \juhao
\end{gather*}

可以看到,同底数的幂相除,当被除式的指数比除式的指数小 $p$ 时,所得的商是一个分数或分式,
分子是 $1$, 分母是同底数的 $p$ 次幂。
\end{enhancedline}

另一方面,如果仿照幂的运算性质 (2) 计算这两个例子,用被除式的指数减去除式的指数,就得
\begin{gather*}
    5^2 \div 5^6 = 5^{2 - 6} = 5^{-4} \douhao \\
    a^3 \div a^5 = a^{3 - 5} = a^{-2} \quad (a \neq 0) \juhao
\end{gather*}
这时就出现了负整数指数。

\begin{enhancedline}
为了使被除式的指数小于除式的指数时,同底数幂除法的运算性质也能适用,我们规定负整数指数幂的意义是
\begin{center}
    \framebox{\quad $\bm{a^{-p} = \dfrac{1}{a^p} \quad (a \neq 0 \douhao p \; \text{\zhongdian{是正整数}})}$。\quad}
\end{center}

这就是说,任何不等于零的实数的 $-p$ ( $p$ 是正整数)次幂,等于这个数的 $p$ 次幂的倒数。

这样规定以后,上面的例子就可以这样来计算:
\begin{gather*}
    5^2 \div 5^6 = 5^{2 - 6} = 5^{-4} = \dfrac{1}{5^4}\douhao \\
    a^3 \div a^5 = a^{3 - 5} = a^{-2} = \dfrac{1}{a^2} \quad (a \neq 0) \juhao
\end{gather*}

应当注意,零的负整数次幂没有意义。

规定了零指数幂与负整数指数幂的意义,就把指数从正整数推广到了整数。
正整数指数幂的运算性质对整数指数幂都适用。例如,
\begin{gather*}
    a^3 \cdot a^0 = a^{3 + 0} = a^3 \quad (a \neq 0) \douhao \\
    a^{-3} \cdot a^2 = a^{-3 + 2} = a^{-1} = \dfrac{1}{a} \quad (a \neq 0) \douhao \\
    (a^{-3})^2 = a^{-3 \times 2} = a^{-6} = \dfrac{1}{a^6} \quad (a \neq 0) \juhao
\end{gather*}

在本章里,当指数是零或负数时,如果没有特别说明,底数都不等于零。

\liti 计算:

\hspace*{3em} $10^{-3} \nsep (-3)^{-2} \nsep \left(\dfrac{1}{2}\right)^{-3} \nsep 5^0 \times (-2)^{-1}$。

\jie \begin{tblr}[t]{columns={18em, mode=math}, rows={rowsep=0.5em}}
    10^{-3} = \dfrac{1}{10^3} = \dfrac{1}{1000} \douhao &
        (-3)^{-2} = \dfrac{1}{(-3)^2} = \dfrac{1}{9} \douhao \\
    \left(\dfrac{1}{2}\right)^{-3} = \dfrac{1}{\left(\dfrac{1}{2}\right)^3} = 8 \douhao &
        5^0 \times (-2)^{-1} = 1 \times \dfrac{1}{-2} = -\dfrac{1}{2} \juhao
\end{tblr}


\liti 用小数表示下列各式:

\hspace*{3em} $10^{-5} \nsep 7 \times 10^{-6} \nsep 3.6 \times 10^{-8}$。

\jie \begin{tblr}[t]{columns={mode=math}}
    10^{-5} = \dfrac{1}{10^5} = 0.00001 \douhao \\
    7 \times 10^{-6} = 7 \times \dfrac{1}{10^6} = 7 \times 0.000001 = 0.000007 \douhao \\
    3.6 \times 10^{-8} = 3.6 \times \dfrac{1}{10^8} = 3.6 \times 0.00000001 = 0.000000036 \juhao
\end{tblr}


\lianxi
\begin{xiaotis}

\xiaoti{(口答)下列各式的结果是什么?}
\begin{xiaoxiaotis}

    \begin{tblr}{columns={12em, colsep=0pt}}
        \xxt{$3a^2b + 2a^2b$;} & \xxt{$3a^2b \cdot 2a^2b$;} & \xxt{$(3ab^2)^2$;} \\
        \xxt{$\left(-\dfrac{2b}{a^3}\right)^3$;} & \xxt{$16a^4b^2 \div 12a^2b^2$;} & \xxt{$(a^2b^2)^3 \div a^2b$。}
    \end{tblr}
\end{xiaoxiaotis}


\xiaoti{计算:\\
    $3^0$\nsep  $3^{-1}$\nsep  $10^{-4}$\nsep  $(\sqrt{2})^0$  \nsep $7^{-2}$\nsep
    $1^{-10}$\nsep  $(-2)^{-3}$\nsep  $\left(\dfrac{1}{2}\right)^{-4}$\nsep  $(-0.1)^0$\nsep  $\left(-\dfrac{1}{2}\right)^{-3}$。
}


\xiaoti{计算:}
\begin{xiaoxiaotis}

    \begin{tblr}{columns={18em, colsep=0pt}}
        \xxt{$(-2)^3 - (-1)^0$;} & \xxt{$2^{-2} + (-2)^{-3}$;} \\
        \xxt{$\left(\dfrac{1}{2}\right)^{-2} \div \left(\dfrac{1}{2}\right)^0$;} & \xxt{$\left(-\dfrac{1}{2}\right)^{-2} \times 2^{-1}$。}
    \end{tblr}
\end{xiaoxiaotis}


\xiaoti{用小数表示下列各式:}
\begin{xiaoxiaotis}

    \begin{tblr}{columns={18em, colsep=0pt}}
        \xxt{$2 \times 10^{-5}$;} & \xxt{$3.1 \times 10^{-7}$;} \\
        \xxt{$8.04 \times 10^{-3}$;} & \xxt{$1.205 \times 10^{-2}$;} \\
        \xxt{$2.12 \times 10^{-3}$;} & \xxt{$2.12 \times 10^{-2}$;} \\
        \xxt{$2.12 \times 10^{-1}$;} & \xxt{$2.12 \times 10^0$。}
    \end{tblr}
\end{xiaoxiaotis}

\end{xiaotis}
\lianxijiange


\liti 计算 $(-a)^{-5}$,$a^{-2}b^{-1}(-2a^3)$,$(-5a^3b^{-1})^{-2}$,
并且把结果化成只含有正整数指数的式子。

\jie \begin{tblr}[t]{columns={mode=math}}
    (-a)^{-5} = \dfrac{1}{(-a)^5} = -\dfrac{1}{a^5} \douhao \\
    a^{-2}b^{-1}(-2a^3) = -2a^{-2 + 3}b^{-1} = -2ab^{-1} = -\dfrac{2a}{b} \douhao \\
    (-5a^3b^{-1})^{-2} = (-5)^{1 \times (-2)} a^{3 \times (-2)} b^{(-1) \times (-2)} = (-5)^{-2}a^{-6}b^2 = \dfrac{1}{(-5)^2} \times \dfrac{1}{a^6} \times b^2 = \dfrac{b^2}{25a^6} \juhao
\end{tblr}


\liti 计算:
\begin{xiaoxiaotis}

    \hspace*{1.5em} \begin{tblr}{columns={colsep=0pt}}
        \xxt{$\dfrac{a^{-2}b^{-3}(-3a^{-1}b^2)}{6a^{-3}b^{-2}}$;} &
            \xxt{$(x^{-2} + y^{-2})(x^{-2} - y^{-2})$;} &
            \xxt{$\dfrac{a^{-1} + b^{-1}}{a^{-1} \cdot b^{-1}}$。}
    \end{tblr}

\resetxxt
\jie \xxt{$\dfrac{a^{-2}b^{-3}(-3a^{-1}b^2)}{6a^{-3}b^{-2}} = -\dfrac{3}{6}a^{-2 + (-1) - (-3)}b^{-3 + 2 - (-2)} = -\dfrac{1}{2}a^0b = -\dfrac{1}{2}b$;}

\hspace*{1.5em} \xxt{$(x^{-2} + y^{-2})(x^{-2} - y^{-2}) = (x^{-2})^2 - (y^{-2})^2 = x^{-4} - y^{-4}$;}

\hspace*{1.5em} \xxt{$\dfrac{a^{-1} + b^{-1}}{a^{-1} \cdot b^{-1}} = \dfrac{(a^{-1} + b^{-1})ab}{(a^-1 \cdot b^-1)ab} = \dfrac{b + a}{1} = a + b$;}

\end{xiaoxiaotis}


\lianxi
\begin{xiaotis}

\xiaoti{计算下列各式,并且把结果化成只含有正整数指数的式子:}
\begin{xiaoxiaotis}

    \begin{tblr}{columns={18em, colsep=0pt}}
        \xxt{$\dfrac{ab}{c^{-2}}$;} & \xxt{$pq^{-2}r^{-1}$;} \\
        \xxt{$\dfrac{a(a + b)^{-1}}{a^{-2}b}$;} & \xxt{$\dfrac{5^{-1}xy^{-2}}{2^{-3}ab^{-4}}$。}
    \end{tblr}
\end{xiaoxiaotis}


\xiaoti{利用负整数指数把下列各式化成不含分母的式子:}
\begin{xiaoxiaotis}

    \begin{tblr}{columns={12em, colsep=0pt}}
        \xxt{$\dfrac{1}{y^5}$;} & \xxt{$\dfrac{a^2}{b^3}$;} & \xxt{$\dfrac{m^2}{x^6y}$。}
    \end{tblr}
\end{xiaoxiaotis}


\xiaoti{(口答)下列计算是否正确?如果不正确,应如何改正?}
\begin{xiaoxiaotis}

    \begin{tblr}{columns={18em, colsep=0pt}}
        \xxt{$(-1)^0 = -1$;} & \xxt{$(-1)^{-1} = 1$;} \\
        \xxt{$3a^{-2} = \dfrac{1}{3a^2}$;} & \xxt{$(-x)^5 \div (-x)^3 = -x^2$。}
    \end{tblr}
\end{xiaoxiaotis}


\xiaoti{计算下列各式,并且把结果化成只含有正整数指数的式子:}
\begin{xiaoxiaotis}

    \begin{tblr}{columns={12em, colsep=0pt}}
        \xxt{$3^{-5} \cdot 3^6$;} & \xxt{$7^{-9} \div 7^{-10}$;} & \xxt{$a^{-3} \cdot a^2$;} \\
        \xxt{$b^{-4} \div b^{-2}$;} & \xxt{$(a^{-3})^{-2}$;} & \xxt{$(x^{-2})^0$;} \\
        \xxt{$(xy)^{-2}$;} & \xxt{$\left(\dfrac{p}{q}\right)^{-2}$。}
    \end{tblr}
\end{xiaoxiaotis}


\xiaoti{计算:}
\begin{xiaoxiaotis}

    \begin{tblr}{columns={18em, colsep=0pt}}
        \xxt{$(x^4y^{-3}) \cdot (x^{-2}y^2)$;} & \xxt{$3a^{-2}b^{-3} \div 3^{-1}a^2b^{-3}$;} \\
        \xxt{$\left(\dfrac{3^{-5} \cdot 3^2}{3^{-3}}\right)^{-2}$;} & \xxt{$\dfrac{(x^{-1} + y^{-1})(x^{-1} - y^{-1})}{x^{-2}y^{-2}}$。}
    \end{tblr}
\end{xiaoxiaotis}

\end{xiaotis}
\end{enhancedline}

\subsubsection{科学记数法}

在代数第二册里,我们曾利用 $10$ 的正整数次幂来记一些数。
例如,地球的表面积约为 $510000000$ $\pfqm$,可以记作 $5.1 \times 10^8 \; \pfqm$。
现在,指数的概念从正整数推广到了整数,我们就可以利用 $10$ 的整数次幂来记任何数了。
例如,课本中一页纸的厚度约是 $0.000075$ 米,而
\begin{align*}
    0.000075 &= 7.5 \times 0.00001 \\
             &= 7.5 \times 10^{-5} \juhao
\end{align*}
这样,我们可以把一页纸的厚度记作 $7.5 \times 10^{-5}$ 米。

这种利用 $10$ 的整数次幂来记数的方法,是科学技术上常用的一种记数法,习惯上称为\zhongdian{科学记数法}。
科学记数法是把一个数记成 $\pm a \times 10^n$ 的形式,其中 $n$ 是整数, $a$ 是大于或等于 $1$ 而小于 $10$ 的数。

下面我们看两个例题。

\liti 用科学记数法表示下列各数:

\hspace*{3em} $1000000\nsep  -30000\nsep  57000000\nsep  -849000\nsep  21.23\nsep  5.08$。

\jie \begin{tblr}[t]{columns={mode=math}}
    1000000 = 1 \times 1000000 = 1 \times 10^6 \douhao \\
    -30000 = -3 \times 10000 = -3 \times 10^4 \douhao \\
    57000000 = 5.7 \times 10000000 = 5.7 \times 10^7 \douhao \\
    -849000 = -8.49 \times 100000 = -8.49 \times 10^5 \douhao \\
    21.23 = 2.123 \times 10 = 2.123 \times 10^1 \douhao \\
    5.08 = 5.08 \times 1 = 5.08 \times 10^0 \juhao
\end{tblr}

从例 5 可以看到,用科学记数法把一个绝对值大于 $1$ 的数表示成 $\pm a \times 10^n$ 的形式时,
$n$ 是一个非负整数,$n$ 等于原数整数部分的位数减去 $1$。


\liti 用科学记数法表示下列各数:

\hspace*{3em} $0.008\nsep  -0.000034\nsep  0.0000000125$。

\jie \begin{tblr}[t]{columns={mode=math}}
    0.008 = 8 \times 0.001 = 8 \times 10^{-3} \douhao \\
    -0.000034 = -3.4 \times 0.00001 = -3.4 \times 10^{-5} \douhao \\
    0.0000000125 = 1.25 \times 0.00000001 = 1.25 \times 10^{-8} \juhao
\end{tblr}

从例 6 可以看到,用科学记数法把一个绝对值小于 $1$ 的数表示成 $\pm a \times 10^n$ 的形式时,
$n$ 是一个负整数,它的绝对值等于原数中第一个非零数字前面所有的零的个数(包括小数点前面的那个零)。

用科学记数法表示位数较多的数时,读、写、计算与记忆都很方便。


\liti 地球的质量约是 $5.98 \times 10^{21}$ 吨,木星的质量约是地球质量的 $318$ 倍。
木星的质量约是多少吨(保留两个有效数字)?

\jie $\begin{aligned}[t]
        & 5.98 \times 10^{21} \times 318 \\
    ={} & 1901.64 \times 10^21 \\
    \approx{} & 1.9 \times 10^{24} \juhao
\end{aligned}$

答:木星的质量约是 $1.9 \times 10^{24}$ 吨。


\lianxi
\begin{xiaotis}

\xiaoti{用科学记数法表示下列各数: \\
    $10000\nsep  800000\nsep  56000000\nsep  2030000000\nsep  7400000$。
}

\xiaoti{下列用科学记数法表示的数,原来的数是什么?\\
    $1 \times 10^7\nsep  4 \times 10^8\nsep  8.5 \times 10^6\nsep  7.04 \times 10^5\nsep  3.96 \times 10^4$。
}


\xiaoti{用科学记数法表示下列各数:}
\begin{xiaoxiaotis}

    \begin{tblr}{columns={18em, colsep=0pt}}
        \xxt{$0.00007$;} & \xxt{$0.0000043$;} \\
        \xxt{$0.00000000807$;} & \xxt{$0.0000006002$;} \\
        \xxt{$0.301$;} & \xxt{$-0.004025$。}
    \end{tblr}
\end{xiaoxiaotis}


\xiaoti{用科学记数法表示下列各数:}
\begin{xiaoxiaotis}

    \begin{tblr}{columns={18em, colsep=0pt}}
        \xxt{$153.7$;} & \xxt{$347200000$;} \\
        \xxt{$0.0000003142$;} & \xxt{$0.00000000002001$;} \\
        \xxt{$-6$;} & \xxt{$30.5771$;} \\
        \xxt{$0.513$;} & \xxt{$0.002074$。}
    \end{tblr}
\end{xiaoxiaotis}


\xiaoti{写出下列科学记数法表示的数的原数:}
\begin{xiaoxiaotis}

    \begin{tblr}{columns={18em, colsep=0pt}}
        \xxt{$-3.05 \times 10^{-6}$;} & \xxt{$7.03 \times 10^5$;} \\
        \xxt{$-3.73 \times 10^{-1}$;} & \xxt{$2.14 \times 10^6$;} \\
        \xxt{$1 \times 10^{-3}$;}     & \xxt{$1.381 \times 10^7$;} \\
        \xxt{$7 \times 10^1$;}        & \xxt{$2.818 \times 10^3$。}
    \end{tblr}
\end{xiaoxiaotis}

\end{xiaotis}




\subsection{分数指数}\label{subsec:12-2}

\subsubsection{根式}

前面已经学过二次根式及其一些性质,现在进一步学习一般根式和它的一些性质。

\begin{enhancedline}
我们知道,当 $n$ 是奇数时,实数 $a$ 的 $n$ 次方根用符号 $\sqrt[n]{a}$ 来表示;
当 $n$ 是偶数时,非负数 $a$ 的 $n$ 次算术根用符号 $\sqrt[n]{a}$ 来表示。
式子 $\sqrt[n]{a}$ 叫做\zhongdian{根式},这里 $n$ 叫做根指数,$a$ 叫做被开方数。
我们知道,根指数 $n$ 等于 $2$ 的根式是二次根式(这时根指数 $2$ 省略不写)。
$n$ 等于 $3$,$4$,$5$,$\cdots$ 时,相应的根式是三次,四次,五次,… \; 根式。
当 $n$ 是奇数时,$a$ 可以是任何实数;
当 $n$ 是偶数时,$a$ 可以是任何非负数。
例如,$\sqrt{5}$,$\sqrt[3]{-5}$,$\sqrt[4]{\dfrac{2}{3}}$,$\sqrt[3]{a}$,
$\sqrt[6]{b^2 + 1}$,$\sqrt{(a - b)^2}$ 等都是根式, $5\sqrt[4]{x^2 + y^2}$ 也是根式。
应当注意,在实数范围内,负数的偶次方根没有意义。
\end{enhancedline}

根据方根的意义,可得

(1) \; $(\sqrt{5})^2 = 5 \nsep (\sqrt[3]{-2})^3 = -2$;

(2) \; $\sqrt[3]{(-2)^3} = -2 \nsep \sqrt[5]{2^5} = 2$;

(3) \; $\sqrt{2^2} = 2 \nsep \sqrt{(-2)^2} = |-2| = -(-2) = 2 \nsep$

\phantom{(3)} \; $\sqrt[4]{(-3)^4} = |-3| = -(-3) = 3$。

一般地,如果 $\sqrt[n]{a}$ 有意义,那么

\jiange
\framebox{\begin{minipage}{0.93\textwidth}
    \zhongdian{(1) $\bm{(\sqrt[n]{a})^n = a}$;}

    \zhongdian{(2) 当 $\bm{n}$ 为奇数时,$\bm{\sqrt[n]{a^n} = a}$;}

    \zhongdian{(3) 当 $\bm{n}$ 为偶数时,$\bm{\sqrt[n]{a^n} = |a| = }\begin{cases}
        \hspace*{1em}\bm{a \quad (a \geqslant 0) \douhao} \\
        \bm{-a \quad (a < 0) \juhao}
    \end{cases}$}
\end{minipage}}
\jiange


因为负数的偶次方根没有意义,负数的奇次方根都可以化成被开方数是正数的同次方根的相反数,
例如 $\sqrt[5]{-2} = -\sqrt[5]{2}$,所以我们研究根式的性质时,只要研究算术根的性质就可以了。
我们规定:在本章里,如果没有特别说明,根号内出现的字母,都表示正数。

根据公式 $(\sqrt[n]{a})^n = a$,当 $a \geqslant 0$ 时,可以进行下面的计算:
\begin{gather*}
    (\sqrt[8]{a^6})^8 = a^6 \fenhao \\
    (\sqrt[4]{a^3})^8 = [(\sqrt[4]{a^3})^4]^2 = (a^3)^2 = a^6 \juhao
\end{gather*}

$\sqrt[8]{a^6}$ 和 $\sqrt[4]{a^3}$ 都是 $a^6$的 $8$ 次算术根,
而 $a^6$ 的 $8$ 次算术根只有一个,所以当 $a \geqslant 0$ 时,应当有
$$ \sqrt[8]{a^6} = \sqrt[4]{a^3} \juhao $$

用同样的方法,可以推得
\begin{center}
    \framebox{\quad $\sqrt[np]{a^{mp}} = \sqrt[n]{a^m} \quad (a \geqslant 0)$。\quad}
\end{center}
\fenge{和}{$$ \sqrt[p]{a^{mp}} = a^m \quad (a \geqslant 0) \juhao $$}\\
这里 $m$ 是正整数,$n$,$p$ 都是大于 $1$ 的整数。

这就是说,一个根式的被开方数如果是一个非负数的幂,
那么这个根式的根指数与被开方数的指数都乘以或者除以同一个正整数,根式的值不变。
这个性质叫做\zhongdian{根式的基本性质}。

对于根式的基本性质,应当特别注意 $a \geqslant 0$ 这个条件,否则就不一定有这个性质。例如,
$\sqrt[6]{(-8)^2} = \sqrt[6]{64} = 2$,$\sqrt[3]{-8} = -2$,所以 $\sqrt[6]{(-8)^2} \neq \sqrt[3]{-8}$。

根指数相同的根式叫做\zhongdian{同次根式};
根指数不同的根式叫做\zhongdian{异次根式}。
利用根式的基本性质,可以把异次根式化为同次根式,或者约简某些根式中被开方数的指数及根指数。


\liti 把 $\sqrt{a}\nsep \sqrt[3]{a^2b}\nsep \sqrt[6]{a}$ 化成同次根式。

分析:这三个根式的根指数 $2$, $3$, $6$ 的最小公倍数是 $6$,可以把它们都化成六次根式。

\jie \begin{tblr}[t]{columns={$}}
    \sqrt{a} = \sqrt[6]{a^3} \douhao \\
    \sqrt[3]{a^2b} = \sqrt[6]{(a^2b)^2} = \sqrt[6]{a^4b^2} \douhao \\
    \sqrt[6]{a} = \sqrt[6]{a} \juhao
\end{tblr}


\liti 把 $\sqrt[3]{-5}\nsep \sqrt[4]{3}$ 化成同次根式。

\jie \begin{tblr}[t]{columns={$}}
    \sqrt[3]{-5} = -\sqrt[3]{5} = -\sqrt[12]{5^4} = -\sqrt[12]{625} \douhao \\
    \sqrt[4]{3} = \sqrt[12]{3^3} = \sqrt[12]{27} \juhao
\end{tblr}


\liti 约简下列根式中被开方数的指数及根指数:
\begin{xiaoxiaotis}

    \hspace*{1.5em}\xxt{$\sqrt[5]{a^{10}}$;} \xxt{$\sqrt[6]{(-3)^2x^4}$;}

\resetxxt
\jie \begin{tblr}[t]{columns={colsep=0pt}}
    \xxt{$\sqrt[5]{a^{10}} = a^2$;}\\
    \xxt{$\sqrt[6]{(-3)^2x^4} = \sqrt[6]{3^2x^4} = \sqrt[6]{(3x^2)^2} = \sqrt[3]{3x^2}$。}
\end{tblr}
\end{xiaoxiaotis}


\liti 求 $\sqrt[6]{8}$ 精确到 $0.001$ 的近似值。

\jie $\sqrt[6]{8} = \sqrt[6]{2^3} = \sqrt{2} \approx 1.414$。


\lianxi
\begin{xiaotis}

\xiaoti{设 $x$ 表示实数,求下列各式在什么条件下有意义:\\
    $\sqrt{x}\nsep  \sqrt{-x}\nsep  \sqrt[3]{x}\nsep  \sqrt[3]{-x}\nsep  \sqrt[4]{1 - x}\nsep  \sqrt[4]{x - 1}$。
}

\xiaoti{计算:}
\begin{xiaoxiaotis}

    \begin{tblr}{columns={18em, colsep=0pt}}
        \xxt{$\sqrt{x^2 - 2x + 1} \quad (x > 1)$;} & \xxt{$\sqrt[4]{(x^2 - 2x + 1)^2} \quad (x < 1)$。}
    \end{tblr}
\end{xiaoxiaotis}


\xiaoti{把下列根式化成同次根式:}
\begin{xiaoxiaotis}

    \begin{tblr}{columns={colsep=0pt}, column{1}={18em}}
        \xxt{$\sqrt{5}\nsep  \sqrt[4]{2}$;} & \xxt{$\sqrt[3]{6y^2}\nsep  \sqrt[5]{-y}$;} \\
        \xxt{$\sqrt{2mn}\nsep  \sqrt[5]{-6m^2n}\nsep  \sqrt[10]{5m}$;} & \xxt{$\sqrt{x + y}\nsep  \sqrt[4]{x^2 + y^2}\nsep  \sqrt[6]{(x + y)^5}$。}
    \end{tblr}
\end{xiaoxiaotis}


\xiaoti{约简下列根式中被开方数的指数及根指数:}
\begin{xiaoxiaotis}

    \begin{tblr}{columns={12em, colsep=0pt}}
        \xxt{$\sqrt[4]{x^2}$;} & \xxt{$\sqrt[3]{y^9}$;} & \xxt{$\sqrt[12]{x^4y^6}$;} \\
        \xxt{$\sqrt[6]{(-5)^4a^4b^2}$;} & \xxt{$\sqrt[4]{16x^8y^{12}}$;} & \xxt{$\sqrt[16]{a^{4m}b^{8n}}$。}
    \end{tblr}
\end{xiaoxiaotis}

\end{xiaotis}

\subsubsection{分数指数}
\begin{enhancedline}

看下面的两个例子。
\begin{align*}
    & \sqrt{a^6} = a^3 = a^{\frac{6}{2}} \quad (a > 0) \douhao \\
    & \sqrt[3]{x^{12}} = x^4 = x^{\frac{12}{3}} \quad (x > 0) \juhao
\end{align*}

这就是说,当根式的被开方数的指数能被根指数整除时,根式可以写成分数指数幂的形式。

为了使计算简便,当根式的被开方数的指数不能被根指数整除时,我们也把根式写成分数指数幂的形式。例如,
$$ \sqrt[3]{a^2} = a^{\frac{2}{3}} \nsep  \sqrt{b} = b^{\frac{1}{2}} \nsep  \sqrt[4]{c^5} = c^{\frac{5}{4} \juhao }$$

我们规定正数的正分数指数幂的意义是
\begin{center}
    \framebox{\quad $\bm{a^{\frac{m}{n}} = \sqrt[n]{a^m} \quad (a > 0 \douhao m \douhao n \text{\zhongdian{都是正整数,}} n > 1)}$。\quad }
\end{center}

这就是说,正数的 $\dfrac{m}{n}$ 次幂($m$,$n$ 都是正整数,$n > 1$)等于这个正数的 $m$ 次幂的 $n$ 次算术根。

正数的负分数指数幂的意义与正数的负整数指数幂的意义相仿,就是
\begin{center}
    \framebox{\quad $\bm{a^{-\frac{m}{n}} = \dfrac{1}{a^{\frac{m}{n}}} = \dfrac{1}{\sqrt[n]{a^m}} \quad (a > 0 \douhao m \douhao n \text{\zhongdian{都是正整数,}} n > 1)}$。\quad }
\end{center}

这就是说,正数的 $-\dfrac{m}{n}$ 次幂($m$,$n$ 都是正整数,$n > 1$)等于这个正数的 $\dfrac{m}{n}$ 次幂的倒数。

应当注意,零的正分数次幂是零,零的负分数次幂没有意义。

在本章里,当指数是分数时,如果没有特别说明,底数都表示正数。

规定了分数指数幂的意义以后,指数从整数又推广到了有理数。
前面学过的幂的运算性质,对于有理数指数幂也同样适用。例如,
$$ a^{\frac{2}{3}} \cdot  a^{-\frac{1}{4}} = a^{\frac{2}{3} + (-\frac{1}{4})} = a^{\frac{5}{12}} \juhao $$


\liti 求下列各式的值:

\hspace*{1.5em} $8^{\frac{2}{3}}\nsep  100^{-\frac{1}{2}}\nsep  \left(\dfrac{16}{81}\right)^{-\frac{3}{4}}$。

\jie \begin{tblr}[t]{columns={$}}
    8^{\frac{2}{3}} = (2^3)^{\frac{2}{3}} = 2^2 = 4 \douhao \\
    100^{-\frac{1}{2}} = (10^2)^{-\frac{1}{2}} = 10^{-1} = \dfrac{1}{10} \douhao \\
    \left(\dfrac{16}{81}\right)^{-\frac{3}{4}} = \left(\dfrac{2^4}{3^4}\right)^{-\frac{3}{4}} = \dfrac{2^{-3}}{3^{-3}} = \dfrac{3^3}{2^3} = \dfrac{27}{8} \juhao
\end{tblr}


\liti 计算下列各式,并且把结果化成只含有正整数指数的式子:
\begin{xiaoxiaotis}

    \xxt{$(2a^{\frac{2}{3}}b^{\frac{1}{2}}) (-6a^{\frac{1}{2}}b^{\frac{1}{3}}) \div (-3a^{\frac{1}{6}}b^{\frac{5}{6}})$;}
    \xxt{$(p^{\frac{1}{4}}q^{-\frac{3}{8}})^8$;}
    \xxt{$\sqrt[4]{\left(\dfrac{16m^{-4}}{81n^4}\right)^3}$。}

\resetxxt
\jie \begin{tblr}[t]{columns={colsep=1em}}
    \xxt{$\begin{aligned}[t]
                & (2a^{\frac{2}{3}}b^{\frac{1}{2}}) (-6a^{\frac{1}{2}}b^{\frac{1}{3}}) \div (-3a^{\frac{1}{6}}b^{\frac{5}{6}}) \\
            ={} & 4a^{\frac{2}{3} + \frac{1}{2} - \frac{1}{6}} b^{\frac{1}{2} + \frac{1}{3} - \frac{5}{6}} \\
            ={} & 4ab^0 = 4a \fenhao
        \end{aligned}$} & \xxt{$\begin{aligned}[t]
                & (p^{\frac{1}{4}}q^{-\frac{3}{8}})^8 \\
            ={} & (p^{\frac{1}{4}})^8 (q^{-\frac{3}{8}})^8 \\
            ={} & p^2 q^{-3} = \dfrac{p^2}{q^3} \fenhao
        \end{aligned}$} \\
    \xxt{$\begin{aligned}[t]
            & \sqrt[\uproot{6}4]{\left(\dfrac{16m^{-4}}{81n^4}\right)^3} = \left(\dfrac{2^4m^{-4}}{3^4n^4}\right)^{\frac{3}{4}} \\
        ={} & \dfrac{2^3m^{-3}}{3^3n^3} = \dfrac{8}{27m^3n^3} \juhao
    \end{aligned}$}
\end{tblr}

\end{xiaoxiaotis}


\lianxi
\begin{xiaotis}

\xiaoti{用分数指数幂表示下列各式(分式要化为不含分母的式子):\\
    $\sqrt[3]{x^2}\nsep  \dfrac{1}{\sqrt[3]{a}}\nsep  \sqrt[4]{(a + b)^3}\nsep  \sqrt[3]{m^2 + n^2}\nsep  \dfrac{\sqrt{x}}{\sqrt[3]{y^2}}$。
}


\xiaoti{计算:}
\begin{xiaoxiaotis}

    \begin{tblr}{columns={12em, colsep=0pt}}
        \xxt{$25^{\frac{1}{2}}$;} & \xxt{$\left(\dfrac{81}{25}\right)^{-\frac{1}{2}}$;} & \xxt{$27^{\frac{2}{3}}$;} \\
        \xxt{$10000^{\frac{1}{4}}$;} & \xxt{$4^{-\frac{1}{2}}$;} & \xxt{$\left(6\dfrac{1}{4}\right)^{\frac{3}{2}}$;} \\
        \xxt{$2^{-1} \times 64^{\frac{2}{3}}$;} & \xxt{$(0.2)^{-2} \times (0.064)^{\frac{1}{3}}$。}
    \end{tblr}
\end{xiaoxiaotis}


\xiaoti{计算:}
\begin{xiaoxiaotis}

    \begin{tblr}{columns={12em, colsep=0pt}}
        \xxt{$a^{\frac{1}{4}} \cdot a^{\frac{1}{8}} \cdot a^{\frac{5}{8}}$;}
            & \xxt{$a^{\frac{1}{3}} \cdot a^{\frac{5}{6}} \div a^{-\frac{1}{2}}$;}
            & \xxt{$(x^{\frac{1}{2}} y^{-\frac{1}{3}})^6$;} \\
        \SetCell[c=2]{l}\xxt{$4a^{\frac{2}{3}} b^{-\frac{1}{3}} \div \left(-\dfrac{2}{3} a^{-\frac{1}{3}} b^{-\frac{1}{3}}\right)$;}
            & & \xxt{$\left(\dfrac{8a^{-3}}{27b^6}\right)^{-\frac{1}{3}}$。}
    \end{tblr}
\end{xiaoxiaotis}


\xiaoti{(口答)下列计算是否正确?如果不正确,应如何改正?}
\begin{xiaoxiaotis}

    \begin{tblr}{columns={18em, colsep=0pt}}
        \xxt{$a^{\frac{2}{3}} \cdot a^{\frac{3}{2}} = a$;}  & \xxt{$x^{\frac{2}{3}} \cdot x^{-\frac{2}{3}} = 0$;} \\
        \xxt{$a^{\frac{2}{3}} \div a^{\frac{1}{3}} = a^2$;} & \xxt{$(a^{-\frac{1}{2}})^2 = a$。}
    \end{tblr}
\end{xiaoxiaotis}

\end{xiaotis}
\end{enhancedline}


\subsubsection{根式的性质}
\begin{enhancedline}

当 $m$,$n$ 都是正整数时,根据幂的运算性质可得

(1) \; $(ab)^{\frac{1}{n}} = a^{\frac{1}{n}} b^{\frac{1}{n}} \quad (a \geqslant 0,\; b \geqslant 0)$;

(2) \; $\left(\dfrac{a}{b}\right)^{\frac{1}{n}} = \dfrac{a^{\frac{1}{n}}}{b^{\frac{1}{n}}} \quad (a \geqslant 0,\; b > 0)$;

(3) \; $(a^{\frac{1}{n}})^m = a^{\frac{m}{n}} \quad (a \geqslant 0)$;

(4) \; $(a^{\frac{1}{n}})^{\frac{1}{m}} = a^{\frac{1}{mn}} \quad (a \geqslant 0)$。

按照分数指数幂的意义,可把这几个式子表示成根式的形式,即

($1'$) \; $\sqrt[n]{ab} = \sqrt[n]{a} \sqrt[n]{b} \quad (a \geqslant 0,\; b \geqslant 0)$;

($2'$) \; $\sqrt[\uproot{6}n]{\dfrac{a}{b}} = \dfrac{\sqrt[n]{a}}{\sqrt[n]{b}} \quad (a \geqslant 0,\; b > 0)$;

($3'$) \; $(\sqrt[n]{a})^m = \sqrt[n]{a^m} \quad (a \geqslant 0)$;

($4'$) \; $\sqrt[m]{\sqrt[n]{a}} = \sqrt[mn]{a} \quad (a \geqslant 0)$。

这几个公式,可以看作关于根式运算的几个性质。

($1'$) 式表明:积的算术根,等于积中各个因式的同次算术根的积。例如,
$$ \sqrt[3]{27 \times 64} = \sqrt[3]{27} \times \sqrt[3]{64} = 3 \times 4 = 12 \juhao $$

($2'$) 式表明:商的算术根,等于被除式的同次算术根除以除式的同次算术根所得的商。例如,
$$ \sqrt[\uproot{6}3]{\dfrac{27}{64}} = \dfrac{\sqrt[3]{27}}{\sqrt[3]{64}} = \dfrac{3}{4} \juhao $$

($3'$) 式表明:根式乘方,把被开方数乘方,根指数不变。例如,
$$ (\sqrt[3]{5})^2 = \sqrt[3]{5^2} = \sqrt[3]{25} \juhao $$

($4'$) 式表明:根式开方,被开方数不变,把根指数相乘。例如,
$$ \sqrt{\sqrt[3]{2}} = \sqrt[6]{2} \nsep \sqrt[3]{\sqrt[3]{2}} = \sqrt[9]{2} \juhao $$

把 ($1'$) 式和 ($2'$) 式反过来,就是根式相乘、除的公式,这就是说,
同次根式相乘(或相除),把被开方数相乘(或相除),根指数不变。例如,
\begin{align*}
    & 5\sqrt[3]{4} \cdot 2\sqrt[3]{2} = 10\sqrt[3]{8} = 20 \douhao \\
    & 5\sqrt[3]{4} \div 2\sqrt[3]{2} = \dfrac{5}{2}\sqrt[3]{2} \juhao
\end{align*}

如果是异次根式相乘(或相除),可以根据根式的基本性质,先化成同次根式,再相乘(或相除)。例如,
\begin{align*}
    & \sqrt{3} \cdot \sqrt[3]{2} = \sqrt[6]{27} \cdot \sqrt[6]{4} = \sqrt[6]{108} \douhao \\
    & \sqrt{3} \div \sqrt[3]{3} = \sqrt[6]{27} \div \sqrt[6]{9} = \sqrt[6]{3} \juhao
\end{align*}

利用这几个公式,可以进行根式的乘、除、乘方与开方运算.

利用 ($1'$) 式,可以把根式里被开方数中能开得尽方的因式用与根指数相同次数的算术根代替移到根号外面,
也可以把根号外面的非负因式乘方以后(乘方的次数与根指数相同)移到根号里面。例如,
\begin{align*}
    & \sqrt{a^2b} = \sqrt{a^2} \cdot \sqrt{b} = a\sqrt{b} \douhao \\
    & \sqrt[3]{a^6b^5} = \sqrt[3]{a^6 \cdot b^3 \cdot b^2} = \sqrt[3]{a^6} \cdot \sqrt[3]{b^3} \cdot \sqrt[3]{b^2} = a^2b\sqrt[3]{b^2} \douhao \\
    & x\sqrt[3]{y^2} = \sqrt[3]{x^3} \cdot \sqrt[3]{y^2} = \sqrt[3]{x^3y^2} \douhao \\
    & x^3\sqrt{y} = \sqrt{x^6} \cdot \sqrt{y} = \sqrt{x^6y} \; (x > 0) \juhao
\end{align*}


利用 ($2'$) 式,可以把根号里面的分母化去。例如,
\begin{align*}
    & \sqrt[\uproot{6}3]{\dfrac{2}{27}} = \sqrt[\uproot{6}3]{\dfrac{2}{3^3}} = \dfrac{1}{3}\sqrt[3]{2} \douhao \\
    & \sqrt[\uproot{6}3]{\dfrac{3}{4}} = \sqrt[\uproot{6}3]{\dfrac{3 \times 2}{2^2 \times 2}} = \dfrac{1}{2}\sqrt[3]{6} \juhao
\end{align*}

根式化简,结果应符合以下三项要求:

第一,被开方数的每一个因式的指数都小于根指数;

第二,被开方数不含分母;

第三,被开方数的指数和根指数是互质数。

符合这三项要求的根式叫做\zhongdian{最简根式}。
例如根式 $a\sqrt[3]{a^2b}$ 是最简根式, 而 $\sqrt[3]{a^4b}$,$a\sqrt[4]{a^2b^2}$
以及 $a\sqrt[\uproot{6}5]{\dfrac{b}{a^3}}$ 都不是最简根式。
计算结果用根式表示时,根式应为最简根式。

几个根式化成最简根式以后,如果被开方数都相同,根指数也都相同,
这几个根式就叫做\zhongdian{同类根式}。例如,因为
\begin{align*}
    & \sqrt{12} = \sqrt{2^2 \times 3} = 2\sqrt{3} \douhao \\
    & \sqrt[6]{27} = \sqrt[6]{3^3} = \sqrt{3} \douhao \\
    & \sqrt{\dfrac{1}{3}} = \sqrt{\dfrac{1 \times 3}{3 \times 3}} = \dfrac{1}{3}\sqrt{3} \douhao
\end{align*}
所以, $\sqrt{12}$, $\sqrt[6]{27}$, $\sqrt{\dfrac{1}{3}}$ 是同类根式。
又 $\sqrt[3]{x}$, $\sqrt{x}$ 不是同类根式,
$4\sqrt[3]{a^2}$, $4\sqrt[3]{a}$ 也不是同类根式。

\end{enhancedline}

根式相加减,就是把同类根式分别合并。例如,
\begin{align*}
    & a\sqrt[n]{x} + b\sqrt[m]{y} - c\sqrt[n]{x} + d\sqrt[m]{y} \\
    & = (a - c)\sqrt[n]{x} + (b + d)\sqrt[m]{y} \juhao
\end{align*}


\lianxi
\begin{xiaotis}

\xiaoti{计算:}
\begin{xiaoxiaotis}

    \begin{tblr}{columns={18em, colsep=0pt}}
        \xxt{$\sqrt{a^2b^4}$;} & \xxt{$\sqrt{121 \times 64 \times 256}$;} \\
        \xxt{$\sqrt[3]{a^9b^3t^{12}}$;} & \xxt{$\sqrt[3]{-343 \times 512 \times 729}$;} \\
        \xxt{$\sqrt[4]{16a^8b^{12}}$;} & \xxt{$\sqrt[n]{a^{2n} b^n c^{3n}}$。}
    \end{tblr}
\end{xiaoxiaotis}


\xiaoti{计算:}
\begin{xiaoxiaotis}

    \begin{tblr}{columns={12em, colsep=0pt}, rows={rowsep=0.5em}}
        \xxt{$\sqrt{\dfrac{2}{81}}$;}
            & \xxt{$\sqrt{\dfrac{n}{49m^4}}$;}
            & \xxt{$\sqrt[\uproot{6}3]{\dfrac{2}{27}}$;} \\
        \xxt{$\sqrt[\uproot{6}4]{\dfrac{a^5}{16b^4}}$;}
            & \xxt{$\sqrt[\uproot{6}3]{\dfrac{8x^3y^6}{27a^6b^8}}$;}
            & \xxt{$\sqrt[\uproot{6}n]{\dfrac{a^n b^{2n}}{c^{3n} d^n}}$。}
    \end{tblr}
\end{xiaoxiaotis}


\xiaoti{计算:}
\begin{xiaoxiaotis}

    \begin{tblr}{columns={18em, colsep=0pt}}
        \xxt{$(\sqrt[3]{a^2b})^2$;}  & \xxt{$(3\sqrt[5]{a^4b^3})^2$;} \\
        \xxt{$(m\sqrt[4]{mn^2})^3$;} & \xxt{$\left(-\dfrac{x}{y} \sqrt{\dfrac{y}{x}}\right)^3$。}
    \end{tblr}
\end{xiaoxiaotis}


\xiaoti{计算:}
\begin{xiaoxiaotis}

    \begin{tblr}{columns={18em, colsep=0pt}}
        \xxt{$\sqrt{\sqrt[3]{a^4b^2}}$;} & \xxt{$\sqrt[3]{2\sqrt{7}}$;} \\
        \xxt{$\sqrt{a\sqrt[3]{a}}$;} & \xxt{$\sqrt[n]{2\sqrt{2}}$。}
    \end{tblr}
\end{xiaoxiaotis}


\xiaoti{把下列各式中根号内能开得尽方的因式适当改变后移到根号外,
    使被开方数的每一个因式的指数都小于根指数:
}
\begin{xiaoxiaotis}

    \begin{tblr}{columns={12em, colsep=0pt}, rows={rowsep=0.5em}}
        \xxt{$\sqrt{8a^3}$;} & \xxt{$\sqrt{16t^5}$;} & \xxt{$\dfrac{1}{2}\sqrt{64p^3q^7}$;} \\
        \xxt{$\sqrt[3]{2t^4}$;} & \xxt{$\sqrt[3]{27a^5}$;} & \xxt{$\dfrac{1}{3}\sqrt[3]{27a^4b^5}$;} \\
        \xxt{$\sqrt[n]{a^{2n} b^{n+2}}$;} &  \SetCell[c=2]{l}\xxt{$\sqrt[4]{x^5 - x^4y} \; (x > y)$。}
    \end{tblr}
\end{xiaoxiaotis}


\xiaoti{化去根号内的分母:}
\begin{xiaoxiaotis}

    \begin{tblr}{columns={18em, colsep=0pt}, rows={rowsep=0.5em}}
        \xxt{$\sqrt{\dfrac{n^2}{8m}}$;} & \xxt{$\sqrt[\uproot{6}3]{\dfrac{b^2}{9a^2}}$;} \\
        \xxt{$\sqrt[\uproot{6}3]{\dfrac{ax^3}{27m^2n^3}}$;} & \xxt{$\dfrac{1}{x} \sqrt[\uproot{6}n]{\dfrac{1}{a^{n-2}}}$。}
    \end{tblr}
\end{xiaoxiaotis}


\xiaoti{把下列根式化成最简根式:}
\begin{xiaoxiaotis}

    \begin{tblr}{columns={18em, colsep=0pt}, rows={rowsep=0.5em}}
        \xxt{$\sqrt{\dfrac{16c^3}{9a^5b}}$;} & \xxt{$\sqrt[3]{54a^4b^7}$;} \\
        \xxt{$x^2 \sqrt[\uproot{6}3]{\dfrac{3y}{2x^2}}$;} & \xxt{$n \sqrt[\uproot{6}4]{\dfrac{1}{n^4} + \dfrac{1}{n^2}}$。}
    \end{tblr}
\end{xiaoxiaotis}


\xiaoti{计算:}
\begin{xiaoxiaotis}

    \xxt{$\sqrt{8} + \sqrt[3]{54} - 6\sqrt[\uproot{6}3]{\dfrac{2}{27}} + 3\sqrt{18}$;}

    \xxt{$7b\sqrt[3]{a} + 5\sqrt{a^2x} - b^2 \sqrt[\uproot{6}3]{\dfrac{27a}{b^3}} - 6\sqrt{\dfrac{b^2x}{9}}$。}

\end{xiaoxiaotis}
\end{xiaotis}
\lianxijiange


\liti 利用分数指数计算下列各式:
\begin{xiaoxiaotis}

    \jiange
    \xxt{$\dfrac{a^2 \cdot \sqrt[5]{a^3}}{\sqrt{a} \cdot \sqrt[10]{a^7}}$;}
    \xxt{$(\sqrt[3]{5} - \sqrt{125}) \div \sqrt[4]{5}$;}
    \xxt{$\sqrt[3]{xy^2 (\sqrt{xy})^3}$。}\jiange

\resetxxt
\jie \xxt{$\begin{aligned}[t]
    & \dfrac{a^2 \cdot \sqrt[5]{a^3}}{\sqrt{a} \cdot \sqrt[10]{a^7}} = \dfrac{a^2 \cdot a^{\frac{3}{5}}}{a^{\frac{1}{2}} \cdot a^{\frac{7}{10}}} = a^{2 + \frac{3}{5} - \frac{1}{2} - \frac{7}{10}} \\
    & = a^{\frac{7}{5}} = \sqrt[5]{a^7} = a\sqrt[5]{a^2} \fenhao
\end{aligned}$}

\xxt{$\begin{aligned}[t]
    & (\sqrt[3]{5} - \sqrt{125}) \div \sqrt[4]{5} = (5^{\frac{1}{3}} - 5^{\frac{3}{2}}) \div 5^{\frac{1}{4}} \\
    & = 5^{\frac{1}{3} - \frac{1}{4}} - 5^{\frac{3}{2} - \frac{1}{4}} = 5^{\frac{1}{12}} - 5^{\frac{5}{4}} \\
    & = \sqrt[12]{5} - \sqrt[4]{5^5} = \sqrt[12]{5} - 5\sqrt[4]{5} \fenhao
\end{aligned}$}

\xxt{$\begin{aligned}[t]
    & \sqrt[3]{xy^2 (\sqrt{xy})^3} = \sqrt[\uproot{6}3]{xy^2 (x^{\frac{1}{2}} y^{\frac{1}{2}})^3} \\
    & = (x^{\frac{5}{2}} y^{\frac{7}{2}})^{\frac{1}{3}} = x^{\frac{5}{6}} y^{\frac{7}{6}} = \sqrt[6]{x^5} \cdot \sqrt[6]{y^7} \\
    &= y\sqrt[6]{x^5y} \juhao
\end{aligned}$}

\end{xiaoxiaotis}

除特殊情况外,一般利用分数指数进行根式的乘法、除法、乘方、开方等计算比较简便。
所以,我们在进行根式运算时,一般都利用分数指数进行计算。


\lianxi

计算:
\begin{xiaoxiaotis}
\resetxxt

    \begin{tblr}{columns={18em, colsep=0pt}}
        \xxt{$2 \cdot \sqrt{2} \cdot \sqrt[4]{2} \cdot \sqrt[8]{2}$;} & \xxt{$\dfrac{\sqrt{3} \cdot \sqrt[3]{9}}{\sqrt[6]{3}}$;} \\
        \xxt{$\sqrt{\dfrac{3y}{x}} \cdot \sqrt{\dfrac{3x^2}{y}}$;} & \xxt{$\dfrac{\sqrt{x} \cdot \sqrt[3]{x^2}}{x \cdot \sqrt[6]{x}}$;} \\
        \xxt{$\sqrt{\sqrt[3]{4}}$;} & \xxt{$\sqrt[3]{a\sqrt[4]{a^3}}$;} \\
        \xxt{$\dfrac{-3 a^{\frac{2}{3}} b^{\frac{3}{4}} c^2}{9 a^{\frac{1}{3}} b^{\frac{1}{2}} c^{\frac{3}{2}}}$;} & \xxt{$(x^\frac{1}{3} y^\frac{3}{4} - x^\frac{1}{2}) x^\frac{1}{2} y^\frac{1}{4}$。}
    \end{tblr}
\end{xiaoxiaotis}




\xiti
\begin{xiaotis}
\begin{enhancedline}

\xiaoti{计算:}
\begin{xiaoxiaotis}

    \xxt{$\left(\dfrac{1}{2}\right)^2 + \left(\dfrac{1}{2}\right)^0 + \left(-\dfrac{1}{2}\right)^{-2}$;}

    \xxt{$(-3)^3 + (-3)^{-3} + \left(-\dfrac{1}{3}\right)^{-3} - \left(-\dfrac{1}{3}\right)^3$;}

    \xxt{$\left(\dfrac{1}{2}\right)^{-3} \times \left(-\dfrac{1}{3}\right)^2 \times \left(\dfrac{1}{3}\right)^{-2}$;}

    \xxt{$\left(\dfrac{b}{2a^2}\right)^3 \div \left(\dfrac{2b^2}{3a}\right)^0 \times \left(-\dfrac{b}{a}\right)^{-3}$。}

\end{xiaoxiaotis}


\xiaoti{把下列各式写成用正整数指数幂表示的形式:}
\begin{xiaoxiaotis}

    \begin{tblr}{columns={18em, colsep=0pt}}
        \xxt{$a^2b^{-1}c^3$;} & \xxt{$\dfrac{st^{-2}r}{u^{-1}v}$;} \\
        \xxt{$\dfrac{2^{-2}m^{-2}n^{-3}}{3^{-1}m^{-3}n^3x^{-2}}$;} & \xxt{$\left(\dfrac{x + y}{2x - y}\right)^{-2}$。}
    \end{tblr}
\end{xiaoxiaotis}


\xiaoti{利用负整数指数,把下列各式化成不含分母的式子:}
\begin{xiaoxiaotis}

    \begin{tblr}{columns={18em, colsep=0pt}}
        \xxt{$\dfrac{u + v}{u^4v}$;} & \xxt{$\dfrac{2x - y}{(x - y)(x + y)^2}$。}
    \end{tblr}
\end{xiaoxiaotis}


\xiaoti{计算:}
\begin{xiaoxiaotis}

    \begin{tblr}{columns={18em, colsep=0pt}}
        \xxt{$(9a^2b^{-2}c^{-4})^{-1}$;} & \xxt{$5a^{-2}b^{-3} \div 5^{-1}a^2b^{-3} \times 5^{-2}ab^4c$;} \\
        \xxt{$\dfrac{a^{-3} + b^{-3}}{a^{-1} + b^{-1}}$;} & \xxt{$\dfrac{a^{-2} - b^{-2}}{a^{-2} + b^{-2}}$;} \\
        \xxt{$(a^{-1} + b^{-1}) (a + b)^{-1}$;} & \xxt{$(x + x^{-1}) (x - x^{-1})$。}
    \end{tblr}
\end{xiaoxiaotis}


\xiaoti{用科学记数法表示下列各数:\\
    $32000$\nsep  $3200000$\nsep  $3200000000$\nsep  $0.000032$\nsep  $0.0000032$\nsep \\
    $0.000000032$\nsep  $483$\nsep  $48.3$\nsep  $4.83$\nsep  $0.483$\nsep  $0.0483$\nsep  $0.00483$。
}


\xiaoti{实际中有时用 “微米” 作为长度单位, $1 \; \weimi = 0.001 \; \haomi$。
    人的头发的直径约为 $70$ 微米,合多少毫米,多少厘米,多少米?分别用科学记数法写出来。
}

\xiaoti{地球上陆地的面积约为 $149000000 \; \pfqm$,用科学记数法把它表示出来。}

\xiaoti{一种细菌的半径是 $4 \times 10^{-5}$ 米,用小数把它表示出来(单位仍用米)。}

\xiaoti{一个氧原子约重 $2.657 \times 10^{-23}$ 克,一个氢原子约重 $1.67 \times 10^{-24}$ 克,
    一个氧原子的重量约是一个氢原子的重量的多少倍(保留两个有效数字)?
}

\xiaoti{指出下列各式中 $x$ 的值:}
\begin{xiaoxiaotis}

    \begin{tblr}{columns={18em, colsep=0pt}}
        \xxt{$8 = 2^x$;} & \xxt{$\dfrac{1}{8} = 2^x$;} \\
        \xxt{$1 = 10^x$;} & \xxt{$0.1 = 10^x$;} \\
        \xxt{$3.4 = 3.4 \times 10^x$;} & \xxt{$3400 = 3.4 \times 10^x$;} \\
        \xxt{$0.034 = 3.4 \times 10^x$;} & \xxt{$1 = 0.1^x$;} \\
        \xxt{$\dfrac{1}{64} = 2^x$;} & \xxt{$10 = 0.1^x$。}
    \end{tblr}
\end{xiaoxiaotis}


\xiaoti{用根式表示下列各式:}
\begin{xiaoxiaotis}

    \begin{tblr}{columns={9em, colsep=0pt}}
        \xxt{$4^{-\frac{1}{3}}$;}
            & \xxt{$y^{-\frac{2}{3}}$;}
            & \xxt{$a^{\frac{1}{2}} b^{-\frac{1}{2}}$;}
            & \xxt{$\dfrac{x^{-\frac{1}{4}}}{y^{-\frac{1}{4}}}$。}
    \end{tblr}
\end{xiaoxiaotis}


\xiaoti{计算:}
\begin{xiaoxiaotis}

    \begin{tblr}{columns={18em, colsep=0pt}}
        \xxt{$(-2x^{\frac{1}{4}} y^{-\frac{1}{3}}) (3x^{-\frac{1}{2}} y^{\frac{2}{3}}) (-4x^{\frac{1}{4}} y^{\frac{2}{3}})$;}
            & \xxt{$4x^{\frac{1}{4}} (-3x^{\frac{1}{4}} y^{-\frac{1}{3}}) \div (-6x^{-\frac{1}{2}} y^{-\frac{2}{3}})$;} \\
        \xxt{$\dfrac{-15a^{\frac{1}{2}} b^{\frac{1}{3}} c^{-\frac{3}{4}}}{25a^{-\frac{1}{2}} b^{-\frac{2}{3}} c^{\frac{5}{4}}}$;}
            & \xxt{$\left(\dfrac{16s^2t^{-6}}{25r^4}\right)^{-\frac{3}{2}}$。}
    \end{tblr}
\end{xiaoxiaotis}


\xiaoti{计算:}
\begin{xiaoxiaotis}

    \begin{tblr}{columns={18em, colsep=0pt}}
        \xxt{$2x^{-\frac{1}{3}} \left(\dfrac{1}{2}x^{\frac{1}{3}} - 2x^{-\frac{2}{3}}\right)$;}
            & \xxt{$(2x^{\frac{1}{2}} + 3y^{-\frac{1}{4}}) (2x^{\frac{1}{2}} - 3y^{-\frac{1}{4}})$。}
    \end{tblr}
\end{xiaoxiaotis}


\xiaoti{计算:}
\begin{xiaoxiaotis}

    \begin{tblr}{columns={18em, colsep=0pt}}
        \xxt{$\sqrt[4]{49x^2y^2}$;}
            & \xxt{$\sqrt[\uproot{6}3]{\left(\dfrac{27p^{-6}}{p^2q^{-4}}\right)^{-2}}$;} \\
        \xxt{$\sqrt[\uproot{6}5]{\dfrac{x}{y} \sqrt[\uproot{6}4]{\dfrac{y}{x}}}$;}
            & \xxt{$\sqrt{x^{-3} y^2 \sqrt[3]{xy^2}}$;} \\
        \xxt{$(\sqrt{a} \cdot \sqrt[3]{b^2})^{-3} \div \sqrt{b^{-4}a^{-1}}$;}
            & \xxt{$(\sqrt{3} - \sqrt[4]{243}) \div 2\sqrt[3]{3}$。}
    \end{tblr}
\end{xiaoxiaotis}


\xiaoti{解方程:}
\begin{xiaoxiaotis}

    \begin{tblr}{columns={18em, colsep=0pt}}
        \xxt{$x - 4\sqrt{x} + 3 = 0$;} & \xxt{$\sqrt[3]{x} + \sqrt[3]{x^2} = 2$;} \\
        \xxt{$\sqrt{x} - 3\sqrt[4]{x} + 2 = 0$。}
    \end{tblr}
\end{xiaoxiaotis}

\end{enhancedline}
\end{xiaotis}




