\xiti
\begin{enhancedline}

\begin{xiaotis}

\xiaoti{下列各语句对不对,为什么?}
\begin{xiaoxiaotis}

    \xxt{$-0.064$ 的立方根是 $-0.4$;}

    \xxt{$8$ 的平方根是 $\pm 2$;}

    \xxt{$\dfrac{1}{27}$ 的平方根是 $\dfrac{1}{3}$。}

\end{xiaoxiaotis}

\xiaoti{求下列各式的值:}
\begin{xiaoxiaotis}

    \begin{tblr}{columns={12em, colsep=0pt}}
        \xxt{$-\sqrt[3]{0.027}$;}              & \xxt{$\sqrt[3]{-343}$;}               & \xxt{$\sqrt[3]{\dfrac{125}{27}}$;} \\
        \xxt{$\sqrt[3]{1 + \dfrac{61}{64}}$;}  & \xxt{$\sqrt[3]{1 - \dfrac{19}{27}}$;} & \xxt{$-\sqrt[3]{\dfrac{7}{8} - 1}$。}
    \end{tblr}

\end{xiaoxiaotis}


\xiaoti{求下列各式中的 $x$:}
\begin{xiaoxiaotis}

    \begin{tblr}{columns={18em, colsep=0pt}}
        \xxt{$x^3 = 0.729$;}            & \xxt{$64x^3 + 125 = 0$;} \\
        \xxt{$x^3 - 3 = \dfrac{3}{8}$;} & \xxt{$(x - 1)^3 = 8$。}
    \end{tblr}

\end{xiaoxiaotis}


\xiaoti{查表求下列各数的立方根:}
\begin{xiaoxiaotis}

    \begin{tblr}{columns={10em, colsep=0pt}}
        \xxt{$0.39$;}   & \xxt{$48.3$;}    & \xxt{$-2.36$;} & \xxt{$-34.26$;} \\
        \xxt{$434.5$;}    & \xxt{$4936$;}  & \xxt{$-0.0532$;} & \xxt{$0.007283$。}
    \end{tblr}

\end{xiaoxiaotis}


\xiaoti{查表求下列各式的值:}
\begin{xiaoxiaotis}

    \begin{tblr}{columns={10em, colsep=0pt}}
        \xxt{$\sqrt[3]{0.432}$;}   & \xxt{$\sqrt[3]{-1.948}$;}         & \xxt{$-\sqrt[3]{7456.3}$;}   & \xxt{$\sqrt[3]{67.5}$;} \\
        \xxt{$\sqrt[3]{400000}$;}  & \xxt{$\sqrt[3]{-8\dfrac{5}{9}}$;} & \xxt{$\sqrt[3]{0.0000518}$;} & \xxt{$-\sqrt[3]{-350\dfrac{4}{25}}$。}
    \end{tblr}

\end{xiaoxiaotis}

\xiaoti{}%
\begin{xiaoxiaotis}%
    \xxt[\xxtsep]{正方体的表面积是 $11\ \pfm$, 求它的棱长(精确到 $0.01$ 米);}

    \xxt{正方体的体积是 $11\ \lfm$, 求它的表面积(精确到 $0.01\ \pfm$)。}

\end{xiaoxiaotis}


\xiaoti{已知 $a = 0.5$, $b = 7.5$, $c = -0.36$, 计算下列各式的值(精确到 $0.01$):}
\begin{xiaoxiaotis}

    \twoInLineXxt[18em]{$\dfrac{-b \pm \sqrt{b^2 - 4ac}}{2a}$;}{$(\sqrt[3]{b} - \sqrt[3]{a})c$。}

\end{xiaoxiaotis}


\xiaoti{把下列各数分别填在有理数集合和无理数集合的圈里:\\
    $\dfrac{22}{7}$\nsep  $3.14159265$\nsep  $\sqrt{8}$\nsep  $-8$\nsep  $\sqrt[3]{9}$\nsep
    $0.6$\nsep   $3\dfrac{1}{4}$\nsep   $36$\nsep  $1.732$\nsep   $\dfrac{\pi}{3}$。
}

\begin{figure}[htbp]
    \centering
    \begin{tikzpicture}
    \draw (0, 0) ellipse [x radius=3, y radius=1.2] node[below=1.6] {有理数集合};
    \draw (7, 0) ellipse [x radius=3, y radius=1.2] node[below=1.6] {无理数集合};
\end{tikzpicture}

    \caption*{(第 8 题)}
\end{figure}


\xiaoti{下列各语句对不对,为什么?}
\begin{xiaoxiaotis}

    \xxt{所有的有理数都可以用数轴上的点表示出来; 反过来, 数轴上的所有点都表示有理数:}

    \xxt{所有的实数都可以用数轴上的点表示出来; 反过来,数轴上的所有点都表示实数。}

\end{xiaoxiaotis}


\xiaoti{求下列各数的绝对值:\\
    $-\sqrt{3}$\nsep  $\sqrt[3]{-8}$\nsep   $\sqrt{7}$\nsep   $\dfrac{\sqrt{2}}{-3}$\nsep   $\sqrt{3} - 1.7$\nsep
    $1.4 - \sqrt{2}$。
}


\xiaoti{比较下列各组里两个数的大小:}
\begin{xiaoxiaotis}

    \begin{tblr}{columns={18em, colsep=0pt}}
        \xxt{$15$, $0$;}              & \xxt{$-\dfrac{3}{4}$, $0$;} \\
        \xxt{$-7$, $-9$;}             & \xxt{$\sqrt{2}$, $-\sqrt{3}$;} \\
        \xxt{$-\sqrt{3}$, $-1.731$;}  & \xxt{$\pi$, $3.1416$。}
    \end{tblr}

\end{xiaoxiaotis}


\xiaoti{计算(精确到 $0.01$):}
\begin{xiaoxiaotis}

    \xxt{$\sqrt{5} - \sqrt{3} + 0.145$;}

    \xxt{$\sqrt[3]{6} - \pi - \sqrt{2}$。}

\end{xiaoxiaotis}

\end{xiaotis}
\end{enhancedline}
