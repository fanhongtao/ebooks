\fuxiti
\begin{enhancedline}
\begin{xiaotis}

\xiaoti{\footnotemark 计算:}
\footnotetext{第1 ~ 18 题供复习《代数》第一册和二册学过的内容使用,可以在学习本册的过程中,根据情况选用。}
\begin{xiaoxiaotis}

    \xxt{$-8 - \left[-\dfrac{1}{7} + \left(-\dfrac{1}{6} - 0.25 \times \dfrac{2}{3}\right) \div 2\dfrac{1}{3}\right] \times 6$;}

    \xxt{$\left[\dfrac{2}{3} - \left(-\dfrac{4}{5}\right)\right] \left[\left(\dfrac{2}{3}\right)^2 + \dfrac{2}{3} \times \left(-\dfrac{4}{5}\right) + \left(-\dfrac{4}{5}\right)^2\right]$;}

    \xxt{$\dfrac{7}{9} \times (-3)^3 - \dfrac{1}{6} \times (-3)^2 + \dfrac{1}{12} \times (-3) - \dfrac{1}{4}$;}

    \xxt{$8.954^3 - 7.308^3 - 26.54^2 + 37.69^2$ (查表计算,精确到 $0.1$)。}

\end{xiaoxiaotis}

\xiaoti{各举一例说明加法交换律、加法结合律、乘法交换律、乘法结合律和分配律。}

\xiaoti{如果 $a \neq 0$, 那么 $a$ 的相反数是什么? $a$ 的倒数是什么? $a$ 的绝对值是什么?}


\xiaoti{举例说明:}
\begin{xiaoxiaotis}

    \begin{tblr}{columns={19em, colsep=0pt}}
        \xxt{$a$ 是不是一定大于 $-a$;}                 & \xxt{$\dfrac{a}{2}$ 是不是一定小于 $a$;} \\
        \xxt{已知 $a^2 = b^2$,  能不能断定 $a = b$;}  & \xxt{已知 $|a| > |b|$, 能不能断定 $a > b$。}
    \end{tblr}

\end{xiaoxiaotis}


\xiaoti{如果 $n$ 是正整数,那么}
\begin{xiaoxiaotis}

    \xxt{$2n$ 是奇数还是偶数? $2n - 1$ 呢?}

    \xxt{$(-1)^{2n} = ?$ \quad $(-1)^{2n-1} = ?$ \quad $(-a)^{2n} = ?$ \quad $(-a)^{2n-1} = ?$}

\end{xiaoxiaotis}


\xiaoti{已知 $A = x^3 - x + 6$, $B = x^3 - x^2 + x + 3$,计算:}
\begin{xiaoxiaotis}

    \xxt{$(A - x^3)(B - x^3)$;}

    \xxt{$(A + B) \div (A - B)$;}

    \xxt{$(A - B)^2$。}

\end{xiaoxiaotis}


\xiaoti{把下列各式分解因式:}
\begin{xiaoxiaotis}

    \begin{tblr}{columns={colsep=0pt}, column{1}={19em}}

        \xxt{$m^3 + 12m^2n + 36mn^2$;}     &     \xxt{$12xy - 9x^2 - 4y^2$;} \\
        \xxt{$x^4 - 13x^2 + 36$;}          &     \xxt{$a^6 - 7a^3 - 8$;} \\
        \xxt{$a^3 - b^3 - 3ab(a - b)$;}    &     \xxt{$a^3 + b^3 + 3a^2b + 3ab^2$;} \\
        \xxt{$a^3 - b^3 - a^2 + b^2$;}     &      \xxt{$a(a + 1) - b(b + 1)$;} \\
        \xxt{$a^2 - 4ab + 4b^2 - 4c^2$;}   &     \xxt{$(x + z)(x - z) + y(y - 2x)$;} \\
        \xxt{$(5x^2 + x - 20)^2 - (4x^2 - x + 4)^2$;}  & \xxt{$(a + b)^2 + (a + c)^2 - (c + d)^2 - (b + d)^2$。}

    \end{tblr}

\end{xiaoxiaotis}


\xiaoti{化简:}
\begin{xiaoxiaotis}

    \xxt{$(x - 1)(x + 4) - \{(x + 2)(x - 5) -[(x + 3)(x - 4) - (x - 1)(x + 6)]\}$;}

    \xxt{$(2x^2 - 6x + 5)^2 - (2x^2 - 6x + 4)^2 - (2x + 3)^2$;}

    \xxt{$\dfrac{a^3 - b^3}{a - b} - \dfrac{a^3 + b^3}{a + b} - 2ab$;}

    \xxt{$\dfrac{x^2 - 3x + 2}{x^2 - 9} \cdot \dfrac{x^2 - x - 12}{x^2 - 4x + 4} \div \dfrac{x^2 - 5x + 4}{x^2 - 5x + 6}$;}

    \xxt{$\dfrac{2}{x^2 - 3x + 2} - \dfrac{2}{x^2 - 4x + 3} + \dfrac{2}{x^2 - 5x + 6} - \dfrac{1}{x^2 - 1}$;}

    \xxt{$\left(\dfrac{a^2}{a + b} - \dfrac{a^3}{a^2 + 2ab + b^2}\right) \div \left(\dfrac{a}{a + b} - \dfrac{a^2}{a^2 - b^2}\right)$;}

    \twoInLineXxt[18em]{$\dfrac{1 + \dfrac{x}{y}}{1 - \dfrac{x}{y}} \div \dfrac{1 + \dfrac{y}{x}}{1 - \dfrac{y}{x}}$;}
        {$\dfrac{1}{1 - \dfrac{1}{1 - \dfrac{1}{x}}}$。}

\end{xiaoxiaotis}


\xiaoti{}%
\begin{xiaoxiaotis}%
    \xxt[\xxtsep]{已知 $A = 2x^4 - 3x^2 + 6x - 7$, $B = x^2 - x + 1$, 求 $A \div B$ 的商式 $Q$ 及余式 $R$;}

    \xxt{根据 $A = BQ + R$ 来检验第 (1) 小题中所得的结果。}

\end{xiaoxiaotis}


\xiaoti{}%
\begin{xiaoxiaotis}%
    \xxt[\xxtsep]{方程有哪几条同解原理?}

    \xxt{不等式有哪几条基本性质?}

\end{xiaoxiaotis}


\xiaoti{解下列方程:}
\begin{xiaoxiaotis}

    \xxt{$3\left\{x - \dfrac{3x - 1}{4} - \left[1 - 2\left(x - \dfrac{3 + x}{5}\right)\right]\right\} = 5x - 2$;}

    \xxt{$\dfrac{1}{2}(x - 1)^2 - \dfrac{5}{6}(x + 2)^2 + \dfrac{1}{3}(x - 3)^2 = 0$;}

    \xxt{$\dfrac{3x - 38}{4x^2 - 4} - \dfrac{4}{x + 1} = \dfrac{4}{1 - x}$;}

    \xxt{$\dfrac{x + 5}{5 - x} + \dfrac{x - 5}{5 + x} = \dfrac{100}{25 - x^2}$。}

\end{xiaoxiaotis}


\xiaoti{解下列方程组:}
\begin{xiaoxiaotis}

    \begin{tblr}{columns={19em, colsep=0pt}}
        \xxt{$\begin{cases}
                \dfrac{7x - 3y}{5} = \dfrac{5x - y}{3} - \dfrac{x + y}{2}, \\
                3(x - 1) = 5(y + 1);
            \end{cases}$}
          & \xxt{$\begin{cases}
                \dfrac{3x - 2y}{6} + \dfrac{2x + 3y}{7} = 1, \\[1em]
                \dfrac{3x - 2y}{6} - \dfrac{2x + 3y}{7} = 5;
            \end{cases}$} \\
        \xxt{$\begin{cases}
                \dfrac{11}{2x - 3y} + \dfrac{18}{3x - 2y} = 13, \\[1em]
                \dfrac{27}{3x - 2y} - \dfrac{2}{2x - 3y} = 1;
            \end{cases}$}
          & \xxt{$\begin{cases}
                3x + 14 = 4y + 6z, \\
                7y - 1 = 2z + 8x, \\
                5z + 18 = 2x + 3y \juhao
            \end{cases}$}
    \end{tblr}

\end{xiaoxiaotis}


\xiaoti{解下列方程组:}
\begin{xiaoxiaotis}

    \begin{tblr}{columns={19em, colsep=0pt}}
        \xxt{$\begin{cases}
                ax - by = a^2 + b^2, \\
                bx + ay = a^2 + b^2;
            \end{cases}$}
        & \xxt{$\begin{cases}
                \dfrac{x}{a + b} + \dfrac{y}{a - b} = 2a, \\
                x - y = 4ab;
            \end{cases}$} \\
    \xxt{$\begin{cases}
                \dfrac{a}{x} + \dfrac{b}{y} = a - b, \\[1em]
                \dfrac{a}{x} - \dfrac{b}{y} = a + b;
            \end{cases}$}
        & \xxt{$\begin{cases}
                y + z = 2a, \\
                z + x = 2b, \\
                x + y = 2c \juhao
            \end{cases}$}
    \end{tblr}

\end{xiaoxiaotis}


\xiaoti{已知方程 $x^3 + px^2 + qx + r = 0$ 有一个根是 $1$, 一个根是 $2$, 一个根是 $3$, 求 $p$、 $q$、 $r$ 的值。}


\xiaoti{}%
\begin{xiaoxiaotis}%
    \xxt[\xxtsep]{$x$ 取什么值时, 不等式 $16 - 3(2x - 5) < 25 - 4x$ 成立?}

    \xxt{$x$ 取什么值时, 代数式 $5(x - 1) - 6(x - 2)$ 为正数?}

    \xxt{$x$ 取什么值时, 代数式 $\dfrac{1}{5}(x + 2) - \dfrac{1}{3}(x - 1)$ 为负数?}

    \xxt{求不等式 $-4 < x \leqslant -1$ 的整数解,并在数轴上表示出来;}

    \xxt{求不等式 $|x| < 3$ 的整数解,并在数轴上表示出来。}

\end{xiaoxiaotis}


\xiaoti{}%
\begin{xiaoxiaotis}%
    \xxt[\xxtsep]{如果 $a \neq b$ , 那么 $(a - b)^2 > 0$, 为什么?}

    \xxt{如果 $a \neq b$ , 那么 $a^2 + b^2 > 2ab$, 为什么?}

\end{xiaoxiaotis}


\xiaoti{把一个长方形的长增加 $4$ cm, 宽减少 $1$ cm, 它的面积不变,
    把它的长减少 $2$ cm, 宽增加 $1$ cm, 它的面积也不变。求这个长方形的面积。
}


\xiaoti{甲乙两队修一条路。 如果两队合修 $4$ 天后, 剩余部分由甲队单独修, 先后共要 $6$ 天。
    如果两队合修 $3$ 天后, 剩余部分由乙队单独修, 先后也共要 $6$ 天。
    求两队单独修这条路,各需要几天?
}

\begin{center}
    * \hspace*{6em} * \hspace*{6em} *
\end{center}

\xiaoti{}%
\begin{xiaoxiaotis}%
    \xxt[\xxtsep]{设 $x^2 = a$, $a$ 是 $x$ 的什么数?  $x$ 是 $a$ 的什么数?}

    \xxt{设 $y^3 = b$, $b$ 是 $y$ 的什么数?  $y$ 是 $b$ 的什么数?}

\end{xiaoxiaotis}


\xiaoti{下列语句对不对,为什么?}
\begin{xiaoxiaotis}

    \begin{tblr}{columns={18em, colsep=0pt}}
        \xxt{$-4$ 的平方是 $16$;}    & \xxt{$16$ 的平方根是 $-4$;} \\
        \xxt{$-1$ 的立方是 $-1$;}    & \xxt{$-1$ 的立方根是 $-1$;} \\
        \xxt{$-1$ 的平方根是 $-1$;}  & \xxt{$0$ 的平方根是 $0$。}
    \end{tblr}

\end{xiaoxiaotis}


\xiaoti{什么叫算术平方根? 求出下列各数的平方根、算术平方根:\\
    $64$\nsep  $36$\nsep  $0.25$\nsep  $5$\nsep  $(-3)^2$\nsep
    $\left(\dfrac{2}{5}\right)^2$\nsep  $\left(-\dfrac{4}{13}\right)^2$。
}


\xiaoti{查表求下列各式的值:}
\begin{xiaoxiaotis}

    \begin{tblr}{columns={colsep=0pt}}
        \xxt{$\sqrt{94.3}$;}   & \xxt{$\sqrt{0.5374}$;} & \xxt{$\sqrt{0.07321}$;} & \xxt{$\sqrt{8200}$;} \\
        \xxt{$\sqrt{527.48}$;} & \xxt{$\sqrt{4\dfrac{1}{4}}$;} & \xxt{$\sqrt[3]{35.8}$;}   & \xxt{$\sqrt[3]{0.43}$;} \\
         \xxt{$\sqrt[3]{278.4}$;} & \xxt{$\sqrt[3]{1.47}$;}   & \xxt{$\sqrt[3]{0.002008}$;} & \xxt{$\sqrt[3]{4107\dfrac{1}{2}}$;}
    \end{tblr}

    \begin{tblr}{columns={18em, colsep=0pt}}
        \xxt{$\sqrt{0.00895} - \sqrt{0.0234}$;} & \xxt{$\sqrt[3]{0.0695} + \sqrt[3]{0.1783}$;} \\
        \xxt{$3.472^2 + 5.089^2$;} & \xxt{$\sqrt{3.472^2 + 5.089^2}$。}
    \end{tblr}

\end{xiaoxiaotis}


\xiaoti{求下列各式中的 $x$:}
\begin{xiaoxiaotis}

    \begin{tblr}{columns={12em, colsep=0pt}}
        \xxt{$x^2 = 169$;}      & \xxt{$121x^2 - 25 = 0$;} & \xxt{$9x^2 = 64$;} \\
        \xxt{$x^2 - 1.69 = 0$;} & \xxt{$x^3 = 64000$;}     & \xxt{$x^3 = -0.125$。}
    \end{tblr}

\end{xiaoxiaotis}


\xiaoti{已知正方形的面积是 $360 \; \text{cm}^2$,求正方形一边的长(精确到 $0.1$ cm)。}

\xiaoti{已知一个正方体的长是 $5$ 厘米,再做一个正方体,使它的体积是原正方体的体积的 $2$ 倍,
    求所做的正方体的棱长(精确到 $0.1$ 厘米)。
}

\xiaoti{要使人造地球卫星能够围绕地球运转,必须使它的速度大于第一宇宙速度而小于第二宇宙速度。
    第一宇宙速度的计算公式是 $V_1 = \sqrt{gR \vphantom{\big(}} \ (\mmm)$,
    第二宇宙速度的计算公式是 $V_2 = \sqrt{2gR \vphantom{\big(}} \ (\mmm)$。
    其中 $g = 9.8 \ (\text{米}/\text{秒}^2)$, $R = 6.4 \times 10^6 \ (\mi)$。
    求第一、第二宇宙速度(精确到 $100  \mmm$)。
}


\xiaoti{什么叫做无理数? 在下列每一个圈里,至少填入三个数:}

\begin{figure}[htbp]
    \centering
    \begin{tikzpicture}
    \draw (0, 3) ellipse [x radius=2, y radius=1] node[below=1.2] {正有理数集合};
    \draw (5, 3) ellipse [x radius=2, y radius=1] node[below=1.2] {负有理数集合};
    \draw (0, 0) ellipse [x radius=2, y radius=1] node[below=1.2] {正无理数集合};
    \draw (5, 0) ellipse [x radius=2, y radius=1] node[below=1.2] {负无理数集合};
\end{tikzpicture}

    \caption*{(第 27 题)}
\end{figure}



\xiaoti{有没有最小的自然数? 有没有最小的整数? 有没有最小的有理数?
    有没有最小的无理数? 有没有最小的实数? 有没有绝对值最小的实数?
}


\xiaoti{求下列各数的绝对值:\\
    $-2\sqrt{5}$\nsep  $\sqrt[3]{-7}$\nsep  $\sqrt{5} - \sqrt{7}$\nsep  $3.1416 - \pi$。
}


\xiaoti{比较下列各组里两个实数的大小:}
\begin{xiaoxiaotis}

    \begin{tblr}{columns={18em, colsep=0pt}}
        \xxt{$1.547$, $1.\dot{5}$;} & \xxt{$-\sqrt{5}$, $-2.24$;} \\
        \xxt{$-\pi$, $-3.1415926$;} & \xxt{$\sqrt{29}$, $5\dfrac{4}{13}$。}
    \end{tblr}

\end{xiaoxiaotis}


\xiaoti{计算(精确到 $0.01$):}
\begin{xiaoxiaotis}

    \xxt{$\pi + \sqrt{10} - \dfrac{1}{3} + 0.145$;}

    \xxt{$\sqrt{5} + \dfrac{1}{7} - \left(4.375 - \dfrac{4}{3}\right)$。}

\end{xiaoxiaotis}


\xiaoti{观察平方根表,回答下列问题:}
\begin{xiaoxiaotis}

    \xxt{被开方数由 $1.0$ 增加到 $1.1$ 时,平方根的值是增大还是减小?}

    \xxt{被开方数的值逐渐增大时,平方根的值是增大还是减小?}

\end{xiaoxiaotis}

\end{xiaotis}
\end{enhancedline}
