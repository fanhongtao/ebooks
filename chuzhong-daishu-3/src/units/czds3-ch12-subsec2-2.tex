\subsubsection{分数指数}
\begin{enhancedline}

看下面的两个例子。
\begin{align*}
    & \sqrt{a^6} = a^3 = a^{\frac{6}{2}} \quad (a > 0) \douhao \\
    & \sqrt[3]{x^{12}} = x^4 = x^{\frac{12}{3}} \quad (x > 0) \juhao
\end{align*}

这就是说,当根式的被开方数的指数能被根指数整除时,根式可以写成分数指数幂的形式。

为了使计算简便,当根式的被开方数的指数不能被根指数整除时,我们也把根式写成分数指数幂的形式。例如,
$$ \sqrt[3]{a^2} = a^{\frac{2}{3}} \nsep  \sqrt{b} = b^{\frac{1}{2}} \nsep  \sqrt[4]{c^5} = c^{\frac{5}{4} \juhao }$$

我们规定正数的正分数指数幂的意义是
\begin{center}
    \framebox{\quad $\bm{a^{\frac{m}{n}} = \sqrt[n]{a^m} \quad (a > 0 \douhao m \douhao n \text{\zhongdian{都是正整数,}} n > 1)}$。\quad }
\end{center}

这就是说,正数的 $\dfrac{m}{n}$ 次幂($m$,$n$ 都是正整数,$n > 1$)等于这个正数的 $m$ 次幂的 $n$ 次算术根。

正数的负分数指数幂的意义与正数的负整数指数幂的意义相仿,就是
\begin{center}
    \framebox{\quad $\bm{a^{-\frac{m}{n}} = \dfrac{1}{a^{\frac{m}{n}}} = \dfrac{1}{\sqrt[n]{a^m}} \quad (a > 0 \douhao m \douhao n \text{\zhongdian{都是正整数,}} n > 1)}$。\quad }
\end{center}

这就是说,正数的 $-\dfrac{m}{n}$ 次幂($m$,$n$ 都是正整数,$n > 1$)等于这个正数的 $\dfrac{m}{n}$ 次幂的倒数。

应当注意,零的正分数次幂是零,零的负分数次幂没有意义。

在本章里,当指数是分数时,如果没有特别说明,底数都表示正数。

规定了分数指数幂的意义以后,指数从整数又推广到了有理数。
前面学过的幂的运算性质,对于有理数指数幂也同样适用。例如,
$$ a^{\frac{2}{3}} \cdot  a^{-\frac{1}{4}} = a^{\frac{2}{3} + (-\frac{1}{4})} = a^{\frac{5}{12}} \juhao $$


\liti 求下列各式的值:

\hspace*{1.5em} $8^{\frac{2}{3}}\nsep  100^{-\frac{1}{2}}\nsep  \left(\dfrac{16}{81}\right)^{-\frac{3}{4}}$。

\jie \begin{tblr}[t]{columns={$}}
    8^{\frac{2}{3}} = (2^3)^{\frac{2}{3}} = 2^2 = 4 \douhao \\
    100^{-\frac{1}{2}} = (10^2)^{-\frac{1}{2}} = 10^{-1} = \dfrac{1}{10} \douhao \\
    \left(\dfrac{16}{81}\right)^{-\frac{3}{4}} = \left(\dfrac{2^4}{3^4}\right)^{-\frac{3}{4}} = \dfrac{2^{-3}}{3^{-3}} = \dfrac{3^3}{2^3} = \dfrac{27}{8} \juhao
\end{tblr}


\liti 计算下列各式,并且把结果化成只含有正整数指数的式子:
\begin{xiaoxiaotis}

    \xxt{$(2a^{\frac{2}{3}}b^{\frac{1}{2}}) (-6a^{\frac{1}{2}}b^{\frac{1}{3}}) \div (-3a^{\frac{1}{6}}b^{\frac{5}{6}})$;}
    \xxt{$(p^{\frac{1}{4}}q^{-\frac{3}{8}})^8$;}
    \xxt{$\sqrt[4]{\left(\dfrac{16m^{-4}}{81n^4}\right)^3}$。}

\resetxxt
\jie \begin{tblr}[t]{columns={colsep=1em}}
    \xxt{$\begin{aligned}[t]
                & (2a^{\frac{2}{3}}b^{\frac{1}{2}}) (-6a^{\frac{1}{2}}b^{\frac{1}{3}}) \div (-3a^{\frac{1}{6}}b^{\frac{5}{6}}) \\
            ={} & 4a^{\frac{2}{3} + \frac{1}{2} - \frac{1}{6}} b^{\frac{1}{2} + \frac{1}{3} - \frac{5}{6}} \\
            ={} & 4ab^0 = 4a \fenhao
        \end{aligned}$} & \xxt{$\begin{aligned}[t]
                & (p^{\frac{1}{4}}q^{-\frac{3}{8}})^8 \\
            ={} & (p^{\frac{1}{4}})^8 (q^{-\frac{3}{8}})^8 \\
            ={} & p^2 q^{-3} = \dfrac{p^2}{q^3} \fenhao
        \end{aligned}$} \\
    \xxt{$\begin{aligned}[t]
            & \sqrt[\uproot{6}4]{\left(\dfrac{16m^{-4}}{81n^4}\right)^3} = \left(\dfrac{2^4m^{-4}}{3^4n^4}\right)^{\frac{3}{4}} \\
        ={} & \dfrac{2^3m^{-3}}{3^3n^3} = \dfrac{8}{27m^3n^3} \juhao
    \end{aligned}$}
\end{tblr}

\end{xiaoxiaotis}


\lianxi
\begin{xiaotis}

\xiaoti{用分数指数幂表示下列各式(分式要化为不含分母的式子):\\
    $\sqrt[3]{x^2}\nsep  \dfrac{1}{\sqrt[3]{a}}\nsep  \sqrt[4]{(a + b)^3}\nsep  \sqrt[3]{m^2 + n^2}\nsep  \dfrac{\sqrt{x}}{\sqrt[3]{y^2}}$。
}


\xiaoti{计算:}
\begin{xiaoxiaotis}

    \begin{tblr}{columns={12em, colsep=0pt}}
        \xxt{$25^{\frac{1}{2}}$;} & \xxt{$\left(\dfrac{81}{25}\right)^{-\frac{1}{2}}$;} & \xxt{$27^{\frac{2}{3}}$;} \\
        \xxt{$10000^{\frac{1}{4}}$;} & \xxt{$4^{-\frac{1}{2}}$;} & \xxt{$\left(6\dfrac{1}{4}\right)^{\frac{3}{2}}$;} \\
        \xxt{$2^{-1} \times 64^{\frac{2}{3}}$;} & \xxt{$(0.2)^{-2} \times (0.064)^{\frac{1}{3}}$。}
    \end{tblr}
\end{xiaoxiaotis}


\xiaoti{计算:}
\begin{xiaoxiaotis}

    \begin{tblr}{columns={12em, colsep=0pt}}
        \xxt{$a^{\frac{1}{4}} \cdot a^{\frac{1}{8}} \cdot a^{\frac{5}{8}}$;}
            & \xxt{$a^{\frac{1}{3}} \cdot a^{\frac{5}{6}} \div a^{-\frac{1}{2}}$;}
            & \xxt{$(x^{\frac{1}{2}} y^{-\frac{1}{3}})^6$;} \\
        \SetCell[c=2]{l}\xxt{$4a^{\frac{2}{3}} b^{-\frac{1}{3}} \div \left(-\dfrac{2}{3} a^{-\frac{1}{3}} b^{-\frac{1}{3}}\right)$;}
            & & \xxt{$\left(\dfrac{8a^{-3}}{27b^6}\right)^{-\frac{1}{3}}$。}
    \end{tblr}
\end{xiaoxiaotis}


\xiaoti{(口答)下列计算是否正确?如果不正确,应如何改正?}
\begin{xiaoxiaotis}

    \begin{tblr}{columns={18em, colsep=0pt}}
        \xxt{$a^{\frac{2}{3}} \cdot a^{\frac{3}{2}} = a$;}  & \xxt{$x^{\frac{2}{3}} \cdot x^{-\frac{2}{3}} = 0$;} \\
        \xxt{$a^{\frac{2}{3}} \div a^{\frac{1}{3}} = a^2$;} & \xxt{$(a^{-\frac{1}{2}})^2 = a$。}
    \end{tblr}
\end{xiaoxiaotis}

\end{xiaotis}
\end{enhancedline}

