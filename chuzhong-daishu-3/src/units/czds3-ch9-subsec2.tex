\subsection{算术平方根}\label{subsec:9-2}
\begin{enhancedline}

我们已经知道,一个正数有两个平方根,其中一个是正数,一个是负数,并且这两个平方根互为相反数。
因此,求一个正数的平方根,只要求出它的正的平方根,就可以知道它的负的平方根。

正数 $a$ 的正的平方根,也叫做 $a$ 的\zhongdian{算术平方根},记作 $\sqrt{a}$,读作 “根号 $a$”。
例如 $9$ 的算术平方根是 $3$, 可以写成 $\sqrt{9} = 3$。
又如 $\sqrt{16} = 4$,$\sqrt{\dfrac{4}{25}} = \dfrac{2}{5}$ 等。

零的平方根也叫做零的算术平方根,因此\zhongdian{零的算术平方根仍旧是零},即 $\sqrt{0} = 0$。

\liti 求下列各数的算术平方根:
\begin{xiaoxiaotis}

    \hspace*{1.5em} \threeInLineXxt[6em]{$100$;}{$\dfrac{49}{64}$;}{$0.81$。}

\resetxxt
\jie \xxt{\begin{tblr}[t]{}
        $\because$   & $10^2 = 100$, \\
        $\therefore$ & $100$ 的算术平方根是 $10$,即 \\
                     & $\sqrt{100} = 10$;
\end{tblr}}

\hspace*{1.5em} \xxt{\begin{tblr}[t]{rows={rowsep=.5em}}
    $\because$   & $\left(\dfrac{7}{8}\right)^2 = \dfrac{49}{64}$,\\
    $\therefore$ & $\dfrac{49}{64}$ 的算术平方根是 $\dfrac{7}{8}$,即 \\
                 & $\sqrt{\dfrac{49}{64}} = \dfrac{7}{8}$;
\end{tblr}}

\hspace*{1.5em} \xxt{\begin{tblr}[t]{}
    $\because$   & $0.9^2 = 0.81$, \\
    $\therefore$ & $0.81$ 的算术平方根是 $0.9$,即 \\
                 & $\sqrt{0.81} = 0.9$。
\end{tblr}}

\end{xiaoxiaotis}


\liti 求下列各式的值:
\begin{xiaoxiaotis}

    \hspace*{1.5em} \begin{tblr}[t]{rows={rowsep=.5em}}
        \xxt{$\sqrt{10000}$;}   & \xxt{$-\sqrt{144}$;}    & \xxt{$\sqrt{\dfrac{25}{121}}$;} \\
        \xxt{$-\sqrt{0.0001}$;} & \xxt{$\pm\sqrt{625}$;}  & \xxt{$\pm\sqrt{\dfrac{49}{81}}$。}
    \end{tblr}


\resetxxt
\jie \begin{tblr}[t]{}
    \xxt{\begin{tblr}[t]{}
            $\because$   & $100^2 = 10000$, \\
            $\therefore$ & $\sqrt{10000} = 100$;
    \end{tblr}} & \xxt{\begin{tblr}[t]{}
        $\because$   & $12^2 = 144$, \\
        $\therefore$ & $-\sqrt{144} = -12$;
    \end{tblr}} \\

    \xxt{\begin{tblr}[t]{rows={rowsep=.5em}}
        $\because$   & $\left(\dfrac{5}{11}\right)^2 = \dfrac{25}{121}$, \\
        $\therefore$ & $\sqrt{\dfrac{25}{121}} = \dfrac{5}{11}$;
    \end{tblr}} & \xxt{\begin{tblr}[t]{}
        $\because$   & $(0.01)^2 = 0.0001$, \\
        $\therefore$ & $-\sqrt{0.0001} = -0.01$;
    \end{tblr}} \\

    \xxt{\begin{tblr}[t]{}
        $\because$   & $25^2 = 625$, \\
        $\therefore$ & $\pm\sqrt{625} = \pm 25$;
    \end{tblr}} & \xxt{\begin{tblr}[t]{rows={rowsep=.5em}}
        $\because$   & $\left(\dfrac{7}{9}\right)^2 = \dfrac{49}{81}$, \\
        $\therefore$ & $\pm\sqrt{\dfrac{49}{81}} = \pm\dfrac{7}{9}$。
    \end{tblr}}

\end{tblr}

\end{xiaoxiaotis}


\lianxi
\begin{xiaotis}

\xiaoti{判断下列各语句对不对:}
\begin{xiaoxiaotis}

    \xxt{$5$ 是 $25$ 的算术平方根;}

    \xxt{$-6$ 是 $36$ 的算术平方根;}

    \xxt{$6$ 是 $(-6)^2$ 的算术平方根;}

    \xxt{$0.4$ 是 $0.16$ 的算术平方根。}

\end{xiaoxiaotis}


\xiaoti{求下列各数的算术平方根:\\
    $121$\nsep $0.25$\nsep $400$\nsep $0.01$\nsep $\dfrac{1}{256}$\nsep $\dfrac{144}{169}$\nsep $0$。
}


\xiaoti{}%
\begin{xiaoxiaotis}%
    \xxt[\xxtsep]{在公式 $c = \sqrt{a^2 + b^2}$ 中,已知 $a = 6$, $b = 8$, 求 $c$;}

    \xxt{在公式 $a = \sqrt{c^2 - b^2}$ 中,已知 $c = 41$, $b = 40$, 求 $a$。}

\end{xiaoxiaotis}

\xiaoti{求下列各式的值:\\
    $\sqrt{1}$\nsep $-\sqrt{\dfrac{4}{9}}$\nsep $\sqrt{1.21}$\nsep $-\sqrt{0.0196}$\nsep
    $\pm\sqrt{\dfrac{9}{25}}$\nsep $\pm\sqrt{\dfrac{36}{169}}$。
}

\end{xiaotis}
\end{enhancedline}
