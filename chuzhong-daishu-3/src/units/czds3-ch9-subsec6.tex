\subsection{实数}\label{subsec:9-6}
\begin{enhancedline}

我们已经学过,任何一个有理数都可以写成有限小数(整数可以看作小数点后面是 $0$ 的小数)或者循环小数的形式。例如:
\begin{align*}
    3 &= 3.0, \\
    -\dfrac{3}{5} &= -0.6, \\
    \dfrac{9}{11} &= 0.\dot{8}\dot{1}, \\
    \cdots\cdots&\cdots\cdots\cdots \juhao
\end{align*}

反过来也对,即任何有限小数和循环小数都是有理数。

在实际中,我们还会遇到一种小数,它们不能写成有限小数的形式,也不能写成循坏小数的形式。
例如,$\sqrt{2} = 1.41421356\cdots$, $0.101001000100001\cdots$(两个 $1$ 之间依次多一个 $0$),
圆周率 $\pi = 3.14159265\cdots$,它们的小数位数是无限的,而且是不循环的。
这样的小数叫做\zhongdian{无限不循环小数}。

无限不循环小数叫做\zhongdian{无理数}。
如上面的 $\sqrt{2}$, $0.101001000100001\cdots$, $\pi$ 都是无理数。
无理数是很多的,例如,
\begin{align*}
    &\sqrt{3}  = 1.732050\cdots,\\
    -&\sqrt{5} = -2.236067\cdots, \\
    &\sqrt[3]{2} = 1.259921\cdots, \\
    &\dfrac{\pi}{2} = 1.57079632\cdots, \\
    &0.232332333233332\cdots \text{(两个 $2$ 之间依次多一个 $3$)}; \\
    &\cdots\cdots \hspace*{2em} \cdots\cdots
\end{align*}
也都是无理数。但是象 $\sqrt{4}$, $-\sqrt[3]{27}$ 等就不是无理数,而是有理数。

无理数可分为正无理数和负无理数。例如
$\sqrt{2}$,  $\sqrt[5]{3}$, $\pi$,  $\cdots$ 是正无理数;
$-\sqrt{2}$, $-\sqrt[5]{3}$,$-\pi$, $\cdots$ 是负无理数。

有理数和无理数统称\zhongdian{实数}。这样,我们学过的数就有:

\vspace*{5em}% 被忽略的高度
$$
    \text{实数} \smash[t]{\left\{
    \begin{aligned}
        &\begin{aligned}
            \text{有理数}
        \end{aligned}
        \smash{\left\{
            \begin{aligned}
                &\text{正有理数} \\
                &\text{零} \\
                &\text{负有理数}
            \end{aligned}
        \right\}}
        \begin{aligned}
            &\text{有限小数或}\\
            &\text{循环小数}
        \end{aligned}
        \\
        \, \\
        &\text{无理数}
        \smash[b]{\left\{
            \begin{aligned}
                \text{正无理数} \\
                \text{负无理数}
            \end{aligned}
        \right\}}
        \text{无限不循环小数}
    \end{aligned}
    \right.}
$$\vspace{1em}


有理数和无理数都有正负之分,所以实数也有正负之分。
如果 $a$ 表示一个正实数, $-a$ 就表示一个负实数。$a$ 与 $-a$ 互为相反数。
我们规定:$0$ 的相反数仍旧是 $0$。

实数的绝对值意义也和有理数一样:
\zhongdian{一个正实数的绝对值是它本身;一个负实数的绝对值是它的相反数;零的绝对值是零。}
例如,$|\sqrt{2}| = \sqrt{2}$, $|-\sqrt{2}| = \sqrt{2}$,
$|\pi| = \pi$, $|-\pi| = \pi$, $|0| = 0$ 等。

我们已经知道,每一个有理数都可以用数轴上的点来表示。
但是,数轴上所有的点并不都是表示有理数。
因为数轴上的点可以表示任何整数或小数,其中的无限不循坏小数就是无理数。

把数从有理数扩充到实数以后,实数和数轴上的点就是一一对应的,
即每一个实数都可以用数轴上的一个点来表示,反过来,数轴上的每一个点都表示一个实数。

在进行实数运算时,有理数的运算律和运算性质同样适用。
在实数范围内,正数和零总可以进行开方运算,需要注意,负数只能开奇次方,不能开偶次方。

在实数运算中,如果遇到无理数,并且需要求出结果的近似值,
可以按照所要求的精确度用近似的有限小数去代替无理数,再进行计算。

\liti[0] 计算:
\begin{xiaoxiaotis}

    \hspace*{1.5em} \xxt{$\sqrt{5} + \pi$ (精确到 $0.01$);}

    \hspace*{1.5em} \xxt{$\sqrt{3} \cdot \sqrt{2}$ (结果保留三个有效数字)。}

\resetxxt
\jie \xxt{$\sqrt{5} + \pi \approx 2.236 + 3.142 \approx 5.38$;}

\hspace*{1.5em} \xxt{$\sqrt{3} \cdot \sqrt{2} \approx 1.732 \times 1.414 \approx 2.45$。}

\end{xiaoxiaotis}


\lianxi
\begin{xiaotis}

\xiaoti{(口答) 什么叫做无理数? 举出两个无理数的例子。}

\xiaoti{(口答) 下面的语句对不对? 如果不对,怎样说才对?}
\begin{xiaoxiaotis}

    \xxt{无限小数都是无理数;}

    \xxt{无理数都是无限小数;}

    \xxt{带根号的数邶是无理数。}
\end{xiaoxiaotis}

\xiaoti{(口答)下列各数,哪些是有理数, 哪些是无理数?\\
    $-\pi$\nsep  $-3.14$\nsep  $-\sqrt{3}$\nsep  $1.732$\nsep  $0$\nsep  $0.\dot{3}$\nsep  $18$\nsep \\
    $\sqrt{\dfrac{25}{36}}$\nsep   $\dfrac{21}{31}$\nsep  $\sqrt{7}$\nsep  $-\sqrt{16}$\nsep
    $0.3131131113\cdots$(两个 $3$ 之间依次多一个 $1$)。
}

\xiaoti{}%
\begin{xiaoxiaotis}%
    \xxt[\xxtsep]{有理数都是实数吗? 实数都是有理数吗? 举例说明;}

    \xxt{无理数都是实数吗? 实数都是无理数吗? 举例说明。}

\end{xiaoxiaotis}


\xiaoti{求下列各式中的实数 $x$:}
\begin{xiaoxiaotis}

    \begin{tblr}{columns={18em, colsep=0pt}}
        \xxt{$|x| = \dfrac{2}{3}$;}    & \xxt{$|x| = 0$;} \\
        \xxt{$|x| = \sqrt{10}$;}       & \xxt{$|x| = \pi$。}
    \end{tblr}

\end{xiaoxiaotis}

\xiaoti{计算:}
\begin{xiaoxiaotis}

    \xxt{$\sqrt{10} + \dfrac{6}{7} - \pi$ (精确到 $0.01$);}

    \xxt{$(-4) \times \sqrt{7} + \dfrac{1}{2} \times \sqrt{6}$ (结果保留三个有效数字)。}

\end{xiaoxiaotis}

\end{xiaotis}
\end{enhancedline}

