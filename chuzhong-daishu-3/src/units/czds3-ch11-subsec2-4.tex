\subsubsection{因式分解法}

我们知道,一元二次方程都可以用公式法来解。
对于某些系数较为特殊的方程,例如,$x^2 = 4$, 用直接开平方法就比较简便。
现在我们再来学习一种简便方法——因式分解法。

例如,对于方程
$$ x^2 = 4 \douhao $$
除了用直接开平方法来解外,也可用下面的方法来解。

移项,得
$$ x^2 - 4 = 0 \douhao $$
这个方程的右边是 $0$, 左边可以分解成两个一次因式的积,就是
$$ x^2 - 4 = (x - 2)(x + 2) \juhao $$
因此,这个方程可变形为
$$ (x - 2)(x + 2) = 0 \juhao $$

我们知道,如果两个因式的积等于零,那么这两个因式至少要有一个等于零;
反过来,如果两个因式有一个等于零,它们的积也就等于零。
例如,要使 $(x - 2)(x + 2)$ 等于零,必须并且只需 $x - 2$ 等于零或
$x + 2$ 等于零。因此,解方程
$$ (x - 2)(x + 2) = 0 $$
就相当于解方程 $x - 2 = 0$ 或 $x + 2 = 0$ 了。
进一步解这两个一次方程,得到
$$ x = 2 \text{,或\quad} x = -2 \juhao $$

所以,原方程 $x^2 = 4$ 的两个根为
$$ x_1 = 2 \nsep  x_2 = -2 \juhao $$

这种解一元二次方程的方法叫做\zhongdian{因式分解法}。
在一元二次方程的一边是零而另一边易于分解成两个一次式时,就可以用因式分解法来解。

\liti 解方程
\begin{xiaoxiaotis}

    \hspace*{1.5em} \begin{tblr}[t]{columns={18em, colsep=0pt}}
        \xxt{$x^2 - 3x - 10 = 0$;} & \xxt{$(x + 3)(x - 1) = 5$。}
    \end{tblr}

\resetxxt
\jie {\xxt{}} 把方程的左边分解因式,得 % 注:需要将 \xxt{} 使用大括号 {} 括起来,这样,下一行中的公式才不会向右偏移。下同。
\begin{gather*}
    (x - 5)(x + 2) = 0 \juhao \\
    x - 5 = 0 \text{,或\quad} x + 2 = 0 \juhao
\end{gather*}

%\fengeSuoyi{x_1 = 5 \nsep x_2 = -2 \juhao} % 使用 \fengeSuoyi 会导致与上行公式之间多一些空白,不好看。
$\therefore$
\vspace{-1.5em}$$ x_1 = 5 \nsep x_2 = -2 \juhao $$

{\xxt{}} 原方程可变形为
$$ x^2 + 2x - 3 = 5 \douhao $$
即
$$ x^2 + 2x - 8 = 0 \juhao $$

把方程的左边分解因式,得
\begin{gather*}
    (x - 2)(x + 4) = 0 \juhao \\
    x - 2 = 0 \text{,或\quad} x + 4 = 0 \juhao
\end{gather*}

$\therefore$
\vspace{-1.5em}$$ x_1 = 2 \nsep x_2 = -4 \juhao $$

\end{xiaoxiaotis}


\liti 解方程
\begin{xiaoxiaotis}

    \hspace*{1.5em} \begin{tblr}[t]{columns={18em, colsep=0pt}}
        \xxt{$3x(x + 2) = 5(x + 2)$;} & \xxt{$(3x + 1)^2 - 4 = 0$。}
    \end{tblr}

\resetxxt
\jie {\xxt{}} 原方程就是
$$ 3x(x + 2) - 5(x + 2) = 0 \juhao $$

把方程的左边分解因式,得
\begin{gather*}
    (x + 2)(3x - 5) = 0 \juhao \\
    x + 2 = 0 \text{,或\quad} 3x - 5 = 0 \juhao
\end{gather*}

\fengeSuoyi{x_1 = -2 \nsep x_2 = \dfrac{5}{3} \juhao}


{\xxt{}} 把方程的左边分解因式,得
$$ [(3x + 1) + 2] [(3x + 1) - 2] = 0 \douhao $$
即
\begin{gather*}
    (3x + 3)(3x - 1) = 0 \juhao \\
    3x + 3 = 0 \text{,或\quad} 3x - 1 = 0 \juhao
\end{gather*}

\fengeSuoyi{x_1 = -1 \nsep x_2 = \dfrac{1}{3} \juhao}

\end{xiaoxiaotis}


\lianxi
\begin{xiaotis}

\xiaoti{(口答)说出下列方程的根是什么?}
\begin{xiaoxiaotis}

    \begin{tblr}{columns={18em, colsep=0pt}}
        \xxt{$x(x - 2) = 0$;}          & \xxt{$(y + 2)(y - 3) = 0$;} \\
        \xxt{$(3x + 2)(2x - 1) = 0$;}  & \xxt{$x^2 = x$。}
    \end{tblr}
\end{xiaoxiaotis}


\xiaoti{用因式分解法解下列方程:}
\begin{xiaoxiaotis}

    \begin{tblr}{columns={18em, colsep=0pt}}
        \xxt{$5x^2 + 4x = 0$;}     & \xxt{$\sqrt{2}y^2 = 3y$;} \\
        \xxt{$x^2 + 7x + 12 = 0$;} & \xxt{$x^2 - 10x + 16 = 0$;} \\
        \xxt{$x^2 + 3x - 10 = 0$;}   & \xxt{$x^2 - 6x - 40 = 0$;} \\
        \xxt{$t(t + 3) = 28$;}     & \xxt{$(x + 1)(x + 3) = 15$。}
    \end{tblr}
\end{xiaoxiaotis}


\xiaoti{用因式分解法解下列方程:}
\begin{xiaoxiaotis}

    \begin{tblr}{columns={18em, colsep=0pt}}
        \xxt{$(y - 1)^2 + 2y(y - 1) = 0$;} & \xxt{$6(x + 5) = x(x + 5)$;} \\
        \xxt{$(2y - 1)^2 - 9 = 0$;}        & \xxt{$(3x + 2)^2 = 4(x - 3)^2$。}
    \end{tblr}
\end{xiaoxiaotis}

\end{xiaotis}
\lianxijiange


如果学过利用十字相乘法分解因式,就可以用来解一些二次项系数不是 $1$ 的一元二次方程\footnote{
    凡学过初中《代数》第二册第七章第 $6$ 节内容的学生,可以选学本书这一段内容,并选做下面标有 “*” 号的练习。
}。


\liti 解方程
\begin{xiaoxiaotis}

    \hspace*{1.5em} \begin{tblr}[t]{columns={colsep=0pt}, column{1}={19.5em}}
        \xxt{$3x^2 - 16x + 5 = 0$;} & \xxt{$2x(4x + 13) = 7$。}
    \end{tblr}

\resetxxt
\jie \begin{minipage}[t]{7cm}
    \xxt{$\begin{aligned}[t]
        & (3x - 1)(x - 5) = 0 \juhao \\
        & 3x - 1 = 0 \text{,或\quad} x - 5 = 0 \juhao
    \end{aligned}$}

    $\therefore$ \qquad $x_1 = \dfrac{1}{3} \nsep x_2 = 5 \juhao$
\end{minipage}
\quad
\begin{minipage}[t]{7cm}
    \xxt{整理原方程,得}

    \hspace*{3em} $\begin{aligned}[t]
        & 8x^2 + 26x - 7 = 0 \juhao \\
        & (4x - 1)(2x + 7) = 0 \juhao \\
        & 4x - 1 = 0 \text{,或\quad} 2x + 7 = 0 \juhao
    \end{aligned}$

    $\therefore$ \qquad $x_1 = \dfrac{1}{4} \nsep x_2 = -\dfrac{7}{2} \juhao$
\end{minipage}
\end{xiaoxiaotis}


*\lianxi
解下列方程(第 1 ~ \; 2 题)
\begin{xiaotis}

\xiaoti{}%
\begin{xiaoxiaotis}%
    \huitui\begin{tblr}[t]{columns={18em, colsep=0pt}}
        \xxt{$3x^2 - 7x + 2 = 0$;}  & \xxt{$2x^2 - 11x - 21 = 0$;} \\
        \xxt{$14x^2 + 3x - 5 = 0$;} & \xxt{$15x^2 - 14x - 8 = 0$。}
    \end{tblr}

\end{xiaoxiaotis}


\xiaoti{}%
\begin{xiaoxiaotis}%
    \huitui\begin{tblr}[t]{columns={18em, colsep=0pt}}
        \xxt{$6(2x^2 + 1) = 17x$;} & \xxt{$2x(4x - 7) = 15$。}
    \end{tblr}

\end{xiaoxiaotis}


\xiaoti{解下列关于 $x$ 的方程:}
\begin{xiaoxiaotis}

    \begin{tblr}{columns={18em, colsep=0pt}}
        \xxt{$5m^2x^2 - 17mx + 14 = 0$;} & \xxt{$10a^2x^2 - 7abx + b^2 = 0$。}
    \end{tblr}
\end{xiaoxiaotis}

\end{xiaotis}

