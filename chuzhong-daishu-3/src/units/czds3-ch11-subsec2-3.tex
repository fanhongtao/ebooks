\subsubsection{公式法}
\begin{enhancedline}

现在我们用配方法来解一般形式的一元二次方程
$$ ax^2 + bx + c = 0 \quad (a \neq 0) \juhao $$

因为 $a \neq 0$, 所以可以根据方程的同解原理,
把方程 $ax^2 + bx + c = 0$ 的两边都除以二次项的系数 $a$, 得
$$ x^2 + \dfrac{b}{a}x + \dfrac{c}{a} = 0 \juhao $$

把常数项移到方程的右边,得
$$ x^2 + \dfrac{b}{a}x = -\dfrac{c}{a} \juhao $$

在方程的两边各加上一次项系数一半的平方,得
$$ x^2 + \dfrac{b}{a}x + \left(\dfrac{b}{2a}\right)^2 = -\dfrac{c}{a} + \left(\dfrac{b}{2a}\right)^2 \douhao $$
即
$$ \left(x + \dfrac{b}{2a}\right)^2 = \dfrac{b^2 - 4ac}{4a^2} \juhao $$

因为 $a \neq 0$,所以 $4a^2 > 0$,当 $b^2 - 4ac \geqslant 0$ 时,得
$$ x + \dfrac{b}{2a} = \pm \sqrt{\dfrac{b^2 - 4ac}{4a^2}} = \pm \dfrac{\sqrt{b^2 - 4ac}}{2a} \juhao $$

\fengeSuoyi{x = - \dfrac{b}{2a} \pm \dfrac{\sqrt{b^2 - 4ac}}{2a} \douhao} \\
即
$$ x = \dfrac{-b \pm \sqrt{b^2 - 4ac}}{2a} \juhao $$

由此得到

\begin{center}
    \framebox{\begin{minipage}{0.93\textwidth}
        \zhongdian{一元二次方程 $\bm{ax^2 + bx + c = 0 \quad (a \neq 0)}$ 的求根公式是}
        $$ \bm{x = \dfrac{-b \pm \sqrt{b^2 - 4ac}}{2a} \quad (b^2 - 4ac \geqslant 0)} \juhao $$
    \end{minipage}}
\end{center}

我们看到,一元二次方程 $ax^2 + bx + c = 0$ 的根是由系数 $a$, $b$, $c$ 确定的。
因此,在解一元二次方程时,只要先把方程化为一般形式,
然后把各项的系数 $a$, $b$, $c$ 的值代入求根公式,就可以求得方程的根。

这种解一元二次方程的方法叫做\zhongdian{公式法}。

\liti 解方程 $2x^2 + 7x - 4 = 0$。

\jie 这里 $a = 2$, $b = 7$, $c = -4$。
$$ b^2 - 4ac = 7^2 - 4 \times 2 \times (-4) = 81 \juhao $$
$$ x = \dfrac{-7 \pm \sqrt{81}}{2 \times 2} = \dfrac{-7 \pm 9}{4} \juhao $$

\fengeSuoyi{x_1 = \dfrac{-7 + 9}{4} = \dfrac{1}{2} \nsep x_2 = \dfrac{-7 - 9}{4} = -4 \juhao}


\liti 解方程 $x^2 + 2 = 2\sqrt{2}x$。

\jie 移项,得
$$ x^2 - 2\sqrt{2}x + 2 = 0 \juhao $$

这里 $a = 1$, $b = -2\sqrt{2}$, $c = 2$。
$$ b^2 - 4ac = (-2\sqrt{2})^2 - 4 \times 1 \times 2 = 0 \juhao $$
$$ x = \dfrac{2\sqrt{2} \pm 0}{2} = \sqrt{2} \juhao $$

%\fengeSuoyi{x_1 = x_2 = \sqrt{2} \juhao}
$\therefore$
\vspace{-1.5em}$$ x_1 = x_2 = \sqrt{2} \juhao $$



\zhuyi 这个方程有两个相等的实数根。


\liti 解方程 $x^2 + x - 1 = 0$ (结果精确到 $0.001$)。

\jie 这里 $a = 1$, $b = 1$, $c = -1$。
$$ b^2 - 4ac = 1^2 - 4 \times 1 \times (-1) = 5 \juhao $$
$$ x = \dfrac{-1 \pm \sqrt{5}}{2} \juhao $$

查表,得 $\sqrt{5} = 2.236$,所以
\begin{align*}
    x_1 &= \dfrac{-1 + 2.236}{2} = 0.618 \douhao \\
    x_2 &= \dfrac{-1 - 2.236}{2} = -1.618 \juhao
\end{align*}


\liti 解关于 $x$ 的方程
$$ x^2 - a(3x - 2a + b) - b^2 = 0 \juhao $$

\jie 整理原方程,得
$$ x^2 - 3ax + (2a^2 - ab - b^2) = 0 \juhao $$

这里,$x^2$ 的系数是 $1$, $x$ 的系数是 $-3a$, 常数项是 $2a^2 - ab - b^2$,而
$$ (-3a)^2 - 4 \times 1 \times (2a^2 - ab - b^2) = a^2 + 4ab + 4b^2 = (a + 2b)^2 \juhao $$
$$ x = \dfrac{3a \pm \sqrt{(a + 2b)^2}}{2} = \dfrac{3a \pm (a + 2b)}{2} \juhao \\[1em] $$

$\therefore$
\vspace*{-1.5em}\begin{align*}
    x_1 &= \dfrac{3a + a + 2b}{2} = 2a + b \douhao \\
    x_2 &= \dfrac{3a - a - 2b}{2} = a  - b \juhao
\end{align*}

\end{enhancedline}


\lianxi
\begin{xiaotis}

\xiaoti{把下列方程化成 $ax^2 + bx + c = 0$ 的形式,并写出其中 $a$, $b$, $c$ 的值:}
\begin{xiaoxiaotis}

    \begin{tblr}{columns={18em, colsep=0pt}}
        \xxt{$x^2 + 9x = 6$;}  & \xxt{$2x^2 + 1 = 7x$;} \\
        \xxt{$5x^2 = 3x + 2$;} & \xxt{$8x = 3x^2 - 1$。}
    \end{tblr}
\end{xiaoxiaotis}


\xiaoti{用公式法解下列方程:}
\begin{xiaoxiaotis}

    \begin{tblr}{columns={18em, colsep=0pt}}
        \xxt{$2x^2 + 5x - 3 = 0$;} & \xxt{$6x^2 - 13x - 5 = 0$;} \\
        \xxt{$2y^2 - 4y - 1 = 0$;} & \xxt{$t^2 + 2t = 5$;} \\
        \xxt{$p(p - 8) = 16$;}     & \xxt{$\dfrac{5}{2}y^2 + 2y = 1$;} \\
        \xxt{$0.3x^2 + x = 0.8$;}  & \xxt{$x^2 + 3 = 2\sqrt{3}x$。}
    \end{tblr}
\end{xiaoxiaotis}


\xiaoti{用公式法解下列方程,并求根的近似值(精确到 $0.01$):}
\begin{xiaoxiaotis}

    \begin{tblr}{columns={18em, colsep=0pt}}
        \xxt{$x^2 + 3x -5 = 0$;} & \xxt{$x^2 - 6x + 4 = 0$。}
    \end{tblr}
\end{xiaoxiaotis}


\xiaoti{解下列关于 $x$ 的方程:}
\begin{xiaoxiaotis}

    \begin{tblr}{columns={colsep=0pt}, column{1}={18em}}
        \xxt{$2x^2 - mx - m^2 = 0$;} & \xxt{$abx^2 - (a^2 + b^2)x + ab = 0 \quad (ab \neq 0)$。}
    \end{tblr}
\end{xiaoxiaotis}

\end{xiaotis}



