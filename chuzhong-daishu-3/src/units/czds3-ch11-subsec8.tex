\subsection{分式方程}\label{subsec:11-8}
\begin{enhancedline}

我们学过可化为一元一次方程的分式方程,现在进一步学习可化为一元二次方程的分式方程。

解可化为一元二次方程的分式方程的步骤与解可化为一元一次方程的分式方程的步骤相同。
解方程时,用同一个含有未知数的整式(各分式的最简公分母)去乘方程的两边,约去分母,化为整式方程。
这样得到的整式方程有时与原分式方程不是同解方程,有可能产生增根。因此,解分式方程时,必须进行检验。
可把变形后求得的整式方程的根代入原方程各分式的分母,如果各分母都不为零,就是原方程的根;如果有的分母为零,就是增根。
为了简便起见,也可把变形后求得的整式方程的根代入所乘的整式,
如果不使这个整式等于零,就是原方程的根; 如果使这个整式等于零,就是增根。

\liti 解方程 $\dfrac{1}{x + 2} + \dfrac{4x}{x^2 - 4} + \dfrac{2}{2 - x} = 1$。

\jie 原方程就是
$$ \dfrac{1}{x + 2} + \dfrac{4x}{(x + 2)(x - 2)} - \dfrac{2}{x - 2} = 1 \juhao $$

方程的两边都乘以 $(x + 2)(x - 2)$,约去分母,得
$$ (x - 2) + 4x - 2(x + 2) = (x + 2)(x - 2) \juhao $$

整理后,得
$$ x^2 - 3x + 2 = 0 \juhao $$

解这个方程,得
$$ x_1 = 1 \nsep x_2 = 2 \juhao $$

检验:把 $x = 1$ 代入 $(x + 2)(x - 2)$,它不等于零,

\fengeSuoyi{x = 1 \text{是原方程的根。}}

把 $x = 2$ 代入 $(x + 2)(x - 2)$,它等于零,

\fengeSuoyi{x = 2 \text{是增根。}}

所以原方程的根是 $x = 1$。

\liti 解方程 $\dfrac{2(x^2 + 1)}{x + 1} + \dfrac{6(x + 1)}{x^2 + 1} = 7$。

分析:这个方程左边两个分式中的 $\dfrac{x^2 + 1}{x + 1}$ 与 $\dfrac{x + 1}{x^2 + 1}$
互为倒数,根据这个特点,可以用换元法来解。

\jie 设 $\dfrac{x^2 + 1}{x + 1} = y$,那么 $\dfrac{x + 1}{x^2 + 1} = \dfrac{1}{y}$,
于是原方程变为
$$ 2y + \dfrac{6}{y} = 7 \juhao $$

方程的两边都乘以 $y$,约去分母,得
$$ 2y^2 - 7y + 6 = 0 \juhao $$

解这个方程,得
$$ y_1 = 2 \nsep y_2 = \dfrac{3}{2} \juhao $$

当 $y = 2$ 时, $\dfrac{x^2 + 1}{x + 1} = 2$,去分母,整理得
$$ x^2 - 2x - 1 = 0 \douhao $$

\fengeSuoyi{x = \dfrac{2 \pm \sqrt{8}}{2} = 1 \pm \sqrt{2} \fenhao}

当 $y = \dfrac{3}{2}$ 时, $\dfrac{x^2 + 1}{x + 1} = \dfrac{3}{2}$,去分母,整理得
$$ 2x^2 - 3x - 1 = 0 \douhao $$

\fengeSuoyi{x = \dfrac{3 \pm \sqrt{17}}{4} \juhao }

检验: 把 $x = 1 \pm \sqrt{2}$, $x = \dfrac{3 \pm \sqrt{17}}{4}$ 分别代入原方程的分母,
各分母都不等于零,所以它们都是原方程的根。

从而原方程的根是
\begin{center}
\vspace*{-1em}
\begin{tblr}{columns={$}}
    x_1 = 1 + \sqrt{2} \nsep             & x_2 = 1 - \sqrt{2} \douhao \\
    x_3 = \dfrac{3 + \sqrt{17}}{4} \nsep & x_4 = \dfrac{3 - \sqrt{17}}{4} \juhao
\end{tblr}
\end{center}

\liti 解关于 $x$ 的方程 $x + \dfrac{1}{x - 1} = a + \dfrac{1}{a - 1}$。

\jie 方程的两边同乘以 $(a - 1)(x - 1)$,约去分母,得
$$ x(a - 1)(x - 1) + a - 1 = a(a - 1)(x - 1) + x - 1 \juhao $$

整理后,得
$$ (a - 1)x^2 - a^2x + a^2 = 0 \juhao $$

解这个关于 $x$ 的二次方程,得
$$ x_1 = a \nsep x_2 = \dfrac{a}{a - 1} \juhao $$

检验:把 $x = a$, $x = \dfrac{a}{a - 1}$ 分别代入 $x - 1$ 都不等于零,所以它们都是原方程的根。

从而原方程的根是
$$ x_1 = a \nsep x_2 = \dfrac{a}{a - 1} \juhao $$


\lianxi
\begin{xiaotis}

\xiaoti{解下列方程:}
\begin{xiaoxiaotis}

    \begin{tblr}{columns={18em, colsep=0pt}, rows={rowsep=0.5em}}
        \xxt{$\dfrac{4}{x} - \dfrac{1}{x - 1} = 1$;}     & \xxt{$\dfrac{2}{x} - \dfrac{3}{x + 1} = 2$;} \\
        \xxt{$\dfrac{2}{1 - x} = \dfrac{1}{1 + x} + 1$;} & \xxt{$\dfrac{2}{x^2 - 4} + \dfrac{x - 4}{x^2 + 2x} = \dfrac{1}{x^2 - 2x}$。}
    \end{tblr}
\end{xiaoxiaotis}


\xiaoti{用换元法解下列方程:}
\begin{xiaoxiaotis}

    \begin{tblr}{columns={18em, colsep=0pt}}
        \xxt{$\left(\dfrac{x}{x - 1}\right)^2 - 5\left(\dfrac{x}{x - 1}\right) + 6 = 0$;}
            & \xxt{$\dfrac{3x}{x^2 - 1} + \dfrac{x^2 - 1}{3x} = \dfrac{5}{2}$。}
    \end{tblr}
\end{xiaoxiaotis}


\xiaoti{解下列关于 $x$ 的方程:}
\begin{xiaoxiaotis}

    \begin{tblr}{columns={18em, colsep=0pt}}
        \xxt{$x + \dfrac{1}{x} = c + \dfrac{1}{c}$;}
            & \xxt{$\dfrac{a - x}{b + x } = -\dfrac{4(b + x)}{a - x} \quad (a + b \neq 0)$。}
    \end{tblr}
\end{xiaoxiaotis}

\end{xiaotis}
\lianxijiange


\liti 甲乙二人同时从张庄出发,步行 $15$ 公里到李庄。
甲比乙每小时多走 $1$ 公里,结果比乙早到半小时。二人每小时各走几公里?

分析:我们可设乙每小时走 $x$ 公里,那么甲每小时走 $(x + 1)$ 公里。
乙走 $15$ 公里要用 $\dfrac{15}{x}$ 小时,
甲走 $15$ 公里要用 $\dfrac{15}{x + 1}$ 小时,
根据甲走的时间比乙走的时间少半小时,可以列出方程。

\jie 设乙每小时走 $x$ 公里,那么甲每小时走 $(x + 1)$ 公里,根据题意,得
$$ \dfrac{15}{x + 1} = \dfrac{15}{x} - \dfrac{1}{2} \juhao $$

方程的两边都乘以 $2x(x + 1)$, 约去分母,整理得
$$ x^2 + x - 30 = 0 \juhao $$

解这个方程,得
$$ x_1 = 5 \nsep x_2 = -6 \juhao $$

经检验 $x_1 = 5$, $x_2 = -6$ 都是原方程的根。
但速度为负数不合题意,所以只取 $x = 5$,这时 $x + 1 = 6$。

答:甲每小时走 $6$ 公里,乙每小时走 $5$ 公里。


\liti 一个水池有甲乙两个进水管。
单独开放甲管注满水池比单独开放乙管少用 $10$ 小时,
如果两管同时开放,$12$ 小时可把水池注满。
若单独开放一个水管,各需多少小时才能把水池注满?

分析:我们可设单独开放乙管注满水池需 $x$ 小时,那么单独开放甲管注满水池需 $(x - 10)$ 小时。
单开乙管每小时可注满水池的 $\dfrac{1}{x}$, 单开甲管每小时可注满池的 $\dfrac{1}{x - 10}$。
根据两管同时开每小时可注满水池的 $\dfrac{1}{12}$,可以列出方程.

\jie 设单独开放乙管注满水池需 $x$ 小时,那么单独开放甲管注满水池需 $(x - 10)$ 小时。
根据题意,得
$$ \dfrac{1}{x - 10} + \dfrac{1}{x} = \dfrac{1}{12} \juhao $$

方程的两边都乘以 $12x(x - 10)$,约去分母,并整理,得
$$ x^2 - 34x + 120 = 0 \juhao $$

解这个方程,得
$$ x_1 = 30 \nsep x_2 = 4 \juhao $$

经检验,$x_1 = 30$, $x_2 = 4$ 都是原方程的根。

当 $x = 30$ 时,$x - 10 = 20$。

当 $x =  4$ 时,$x - 10 = -6$。

因为注水时间不能为负数,所以只能取 $x = 30$。

答:单独开放一个水管注满水池,甲管需要 $20$ 小时,乙管需要 $30$ 小时。


\lianxi
\begin{xiaotis}

\xiaoti{某农场开挖一条长 $960$ 米的渠道,开工后每天比原计划多挖 $20$ 米,
    结果提前 $4$ 天完成任务。原计划每天挖多少米?
}

\xiaoti{某工厂贮存 $350$ 吨煤,由于改进炉灶和烧煤技术,每天能节约 $2$ 吨煤,
    使贮存的煤比原计划多用 $20$ 天。贮存的煤原计划用多少天?每天烧多少吨?
}

\xiaoti{甲乙两队学生绿化校园。如果两队合作,$6$ 天可以完成。
    如果单独工作,甲队比乙队少用 $5$ 天。两队单独工作各需多少天完成?
}

\xiaoti{甲乙两组工人合做某项工作身, $10$ 天以后,因甲组另有任务,乙组再单独做 $2$ 天才完成。
    如果单独完成这项工作,甲组比乙组可以快 $4$ 天。求各组单独完成这项工作所需要的天数。
}

\xiaoti{一小艇顺流下行 $24$ 公里到目的地,然后逆流回航到出发地,航行时间共计 $3$ 小时 $20$ 分。
    已知水流速度是每小时 $3$ 公里,小艇在静水中的速度是多少?小艇顺流下行和逆流回航的时间各是多少?
}

\end{xiaotis}

\end{enhancedline}

