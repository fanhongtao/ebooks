\subsubsection{科学记数法}

在代数第二册里,我们曾利用 $10$ 的正整数次幂来记一些数。
例如,地球的表面积约为 $510000000$ $\pfqm$,可以记作 $5.1 \times 10^8 \; \pfqm$。
现在,指数的概念从正整数推广到了整数,我们就可以利用 $10$ 的整数次幂来记任何数了。
例如,课本中一页纸的厚度约是 $0.000075$ 米,而
\begin{align*}
    0.000075 &= 7.5 \times 0.00001 \\
             &= 7.5 \times 10^{-5} \juhao
\end{align*}
这样,我们可以把一页纸的厚度记作 $7.5 \times 10^{-5}$ 米。

这种利用 $10$ 的整数次幂来记数的方法,是科学技术上常用的一种记数法,习惯上称为\zhongdian{科学记数法}。
科学记数法是把一个数记成 $\pm a \times 10^n$ 的形式,其中 $n$ 是整数, $a$ 是大于或等于 $1$ 而小于 $10$ 的数。

下面我们看两个例题。

\liti 用科学记数法表示下列各数:

\hspace*{3em} $1000000\nsep  -30000\nsep  57000000\nsep  -849000\nsep  21.23\nsep  5.08$。

\jie \begin{tblr}[t]{columns={mode=math}}
    1000000 = 1 \times 1000000 = 1 \times 10^6 \douhao \\
    -30000 = -3 \times 10000 = -3 \times 10^4 \douhao \\
    57000000 = 5.7 \times 10000000 = 5.7 \times 10^7 \douhao \\
    -849000 = -8.49 \times 100000 = -8.49 \times 10^5 \douhao \\
    21.23 = 2.123 \times 10 = 2.123 \times 10^1 \douhao \\
    5.08 = 5.08 \times 1 = 5.08 \times 10^0 \juhao
\end{tblr}

从例 5 可以看到,用科学记数法把一个绝对值大于 $1$ 的数表示成 $\pm a \times 10^n$ 的形式时,
$n$ 是一个非负整数,$n$ 等于原数整数部分的位数减去 $1$。


\liti 用科学记数法表示下列各数:

\hspace*{3em} $0.008\nsep  -0.000034\nsep  0.0000000125$。

\jie \begin{tblr}[t]{columns={mode=math}}
    0.008 = 8 \times 0.001 = 8 \times 10^{-3} \douhao \\
    -0.000034 = -3.4 \times 0.00001 = -3.4 \times 10^{-5} \douhao \\
    0.0000000125 = 1.25 \times 0.00000001 = 1.25 \times 10^{-8} \juhao
\end{tblr}

从例 6 可以看到,用科学记数法把一个绝对值小于 $1$ 的数表示成 $\pm a \times 10^n$ 的形式时,
$n$ 是一个负整数,它的绝对值等于原数中第一个非零数字前面所有的零的个数(包括小数点前面的那个零)。

用科学记数法表示位数较多的数时,读、写、计算与记忆都很方便。


\liti 地球的质量约是 $5.98 \times 10^{21}$ 吨,木星的质量约是地球质量的 $318$ 倍。
木星的质量约是多少吨(保留两个有效数字)?

\jie $\begin{aligned}[t]
        & 5.98 \times 10^{21} \times 318 \\
    ={} & 1901.64 \times 10^21 \\
    \approx{} & 1.9 \times 10^{24} \juhao
\end{aligned}$

答:木星的质量约是 $1.9 \times 10^{24}$ 吨。


\lianxi
\begin{xiaotis}

\xiaoti{用科学记数法表示下列各数: \\
    $10000\nsep  800000\nsep  56000000\nsep  2030000000\nsep  7400000$。
}

\xiaoti{下列用科学记数法表示的数,原来的数是什么?\\
    $1 \times 10^7\nsep  4 \times 10^8\nsep  8.5 \times 10^6\nsep  7.04 \times 10^5\nsep  3.96 \times 10^4$。
}


\xiaoti{用科学记数法表示下列各数:}
\begin{xiaoxiaotis}

    \begin{tblr}{columns={18em, colsep=0pt}}
        \xxt{$0.00007$;} & \xxt{$0.0000043$;} \\
        \xxt{$0.00000000807$;} & \xxt{$0.0000006002$;} \\
        \xxt{$0.301$;} & \xxt{$-0.004025$。}
    \end{tblr}
\end{xiaoxiaotis}


\xiaoti{用科学记数法表示下列各数:}
\begin{xiaoxiaotis}

    \begin{tblr}{columns={18em, colsep=0pt}}
        \xxt{$153.7$;} & \xxt{$347200000$;} \\
        \xxt{$0.0000003142$;} & \xxt{$0.00000000002001$;} \\
        \xxt{$-6$;} & \xxt{$30.5771$;} \\
        \xxt{$0.513$;} & \xxt{$0.002074$。}
    \end{tblr}
\end{xiaoxiaotis}


\xiaoti{写出下列科学记数法表示的数的原数:}
\begin{xiaoxiaotis}

    \begin{tblr}{columns={18em, colsep=0pt}}
        \xxt{$-3.05 \times 10^{-6}$;} & \xxt{$7.03 \times 10^5$;} \\
        \xxt{$-3.73 \times 10^{-1}$;} & \xxt{$2.14 \times 10^6$;} \\
        \xxt{$1 \times 10^{-3}$;}     & \xxt{$1.381 \times 10^7$;} \\
        \xxt{$7 \times 10^1$;}        & \xxt{$2.818 \times 10^3$。}
    \end{tblr}
\end{xiaoxiaotis}

\end{xiaotis}

