\subsection{由两个二元二次方程组成的方程组}\label{subsec:11-12}

对于这种形式的方程组,我们只讲一些特殊的方程组的解法。现举例如下:

\liti 解方程组
\begin{numcases}{}
    x^2 + y^2 = 20 \douhao \tag{1} \\
    x^2 - 5xy + 6y^2 = 0 \juhao \tag{2}
\end{numcases}

\jiange
分析:在这个方程组中,方程 (2) 的左边可以分解为两个一次因式的积 $(x - 2y)(x - 3y)$,
而右边为零,因此方程 (2) 可化为两个二元一次方程 $x - 2y = 0$, $x - 3y = 0$,
它们与方程 (1) 分别组成方程组
$$ \begin{cases}
    x^2 + y^2 = 20 \douhao \\
    x - 2y = 0 \douhao
\end{cases} \qquad \begin{cases}
    x^2 + y^2 = 20 \douhao \\
    x - 3y = 0 \juhao
\end{cases}$$
解这两个方程组,就得到原方程组的所有的解。

\jie 由 (2),得
$$ (x - 2y)(x - 3y) = 0 \juhao $$
$\therefore$
\vspace{-1.5em}$$ x = 2y = 0 \text{,或\quad} x - 3y = 0 \juhao $$


因此,原方程组可化为两个方程组
$$ \begin{cases}
    x^2 + y^2 = 20 \douhao \\
    x - 2y = 0 \douhao
\end{cases} \qquad \begin{cases}
    x^2 + y^2 = 20 \douhao \\
    x - 3y = 0 \juhao
\end{cases}$$

解这两个方程组,得原方程组的解为
\begin{center}
    \begin{tblr}{columns={mode=math}}
        \begin{cases}
            x_1 = 4 \douhao \\
            y_1 = 2 \fenhao
        \end{cases} & \begin{cases}
            x_2 = -4 \douhao \\
            y_2 = -2 \fenhao
        \end{cases} & \begin{cases}
            x_3 = 3\sqrt{2} \douhao \\
            y_3 = \sqrt{2} \fenhao
        \end{cases} & \begin{cases}
            x_4 = -3\sqrt{2} \douhao \\
            y_4 = -\sqrt{2} \juhao
        \end{cases}
    \end{tblr}
\end{center}


\liti 解方程组
\begin{numcases}{}
    x^2 + 2xy + y^2 = 9 \douhao \tag{1} \\
    (x - y)^2 - 3(x - y) + 2 = 0 \juhao \tag{2}
\end{numcases}

\jiange
分析:这个方程组的每一个方程都可以化为两个二元一次方程。
先将由 (1) 化得的第一个二元一次方程分别与由  (2) 化得的两个二元一次方程进行组合,可得到两个二元一次方程组;
再将由 (1) 化得的第二个二元一次方程分别与由  (2) 化得的两个二元一次方程进行组合,又可得到两个二元一次方程组,
这样一共可得到四个二元一次方程组。解这四个二元一次方程组,就可以得到原方程组的所有的解。

\jie 由 (1),得
$$ (x + y)^2 = 9 \douhao $$

$\therefore$
\vspace{-1.5em}$$ x + y = 3 \text{,或 \quad } x + y = -3 \juhao $$

由 (2),得
$$ (x - y - 1) (x - y - 2) = 0 \douhao $$

$\therefore$
\vspace{-1.5em}$$ x - y - 1 = 0 \text{,或 \quad } x - y - 2 = 0 \juhao $$

因此,原方程组可化为四个方程组
\begin{center}
    \begin{tblr}{columns={mode=math}}
        \begin{cases}
            x + y = 3 \douhao \\
            x - y - 1 = 0 \douhao
        \end{cases} & \begin{cases}
            x + y = 3 \douhao \\
            x - y - 2 = 0 \douhao
        \end{cases} \\[1em]
        \begin{cases}
            x + y = -3 \douhao \\
            x - y - 1 = 0 \douhao
        \end{cases} & \begin{cases}
            x + y = -3 \douhao \\
            x - y - 2 = 0 \juhao
        \end{cases}
    \end{tblr}
\end{center}

解这四个方程组,得原方程组的解为
$$\begin{cases}
    x_1 = 2 \douhao \\
    y_1 = 1 \fenhao
\end{cases} \quad \begin{cases}
    x_2 = \dfrac{5}{2} \douhao \\[1em]
    y_2 = \dfrac{1}{2} \fenhao
\end{cases} \quad \begin{cases}
    x_3 = -1 \douhao \\
    y_3 = -2 \fenhao
\end{cases} \quad \begin{cases}
    x_4 = -\dfrac{1}{2} \douhao \\[1em]
    y_4 = -\dfrac{5}{2} \juhao
\end{cases}$$


\lianxi
\begin{xiaotis}

\xiaoti{把下列方程代为两个二元一次方程:}
\begin{xiaoxiaotis}

    \begin{tblr}{columns={18em, colsep=0pt}}
        \xxt{$x^2 - 3xy + 2y^2 = 0$;} & \xxt{$2x^2 - 5xy - 3y^2 = 0$;} \\
        \xxt{$x^2 - 6xy + 9y^2 = 16$;} & \xxt{$(x + y)^2 - 3(x + y) - 10 = 0$;} \\
        \xxt{$x^2 - 4xy + 4y^2 - 2x + 4y = 3$。}
    \end{tblr}
\end{xiaoxiaotis}


解下列方程组(第 2 ~\, 3 题):

\xiaoti{}%
\begin{xiaoxiaotis}%
    \huitui\begin{tblr}[t]{columns={colsep=0pt}, column{1}={21em}} % 指定 21 em 是为了和第 3 题对齐。更美观
        \xxt{$\begin{cases}
                (x - y)(x - 2y) = 0 \douhao \\
                3x^2 + 2xy = 20 \fenhao
            \end{cases}$} & \xxt{$\begin{cases}
                x^2 + y^2 = 5 \douhao \\
                2x^2 - 3xy - 2y^2 = 0 \juhao
            \end{cases}$}
    \end{tblr}

\end{xiaoxiaotis}


\xiaoti{}%
\begin{xiaoxiaotis}%
    \huitui\begin{tblr}[t]{columns={colsep=0pt}, column{1}={21em}}
        \xxt{$\begin{cases}
                (x - 2y - 1)(x - 2y + 1) = 0 \douhao \\
                (3x - 2y + 1)(2x + y - 3) = 0 \fenhao
            \end{cases}$} & \xxt{$\begin{cases}
                x^2 + 2xy + y^2 = 25 \douhao \\
                9x^2 - 12xy + 4y^2 = 9 \juhao
            \end{cases}$}
    \end{tblr}

\end{xiaoxiaotis}

\end{xiaotis}
\lianxijiange


\liti 解方程组
\begin{numcases}{}
    x^2 + 3xy = 28 \douhao \tag{1} \\
    2xy - y^2 = 7 \juhao \tag{2}
\end{numcases}

\jiange
分析:这个方程组的两个方程都不含未知数的一次项,消去常数项后就可得到形如 $ax^2 + byx + cy^2 = 0$ 的方程,
解由这个方程与原方程组的任何一个方程组成的方程组,就可以求得原方程组的解。

\jie $(1) - (2) \times 4$,得
\begin{gather*}
    x^2 - 5xy + 4y^2 = 0 \douhao \\
    (x - y)(x - 4y) = 0 \juhao
\end{gather*}

\fengeSuoyi{x - y = 0 \text{,或 \quad} x - 4y = 0 \juhao}

因此,原方程组可化为两个方程组
$$\begin{cases}
    x- y = 0 \douhao \\
    2xy - y^2 = 7 \douhao
\end{cases} \qquad \begin{cases}
    x - 4y = 0 \douhao \\
    2xy - y^2 = 7 \juhao
\end{cases}$$

解这两个方程组,得原方程组的解为
\begin{center}
    \begin{tblr}{columns={mode=math}}
        \begin{cases}
            x_1 = \sqrt{7} \douhao \\
            y_1 = \sqrt{7} \fenhao
        \end{cases} & \begin{cases}
            x_2 = -\sqrt{7} \douhao \\
            y_2 = -\sqrt{7} \fenhao
        \end{cases} & \begin{cases}
            x_3 = 4 \douhao \\
            y_3 = 1 \fenhao
        \end{cases} & \begin{cases}
            x_4 = -4 \douhao \\
            y_4 = -1 \juhao
        \end{cases}
    \end{tblr}
\end{center}


\liti 解方程组
\begin{numcases}{}
    x^2 + y^2 = 5 \douhao \tag{1} \\
    xy = 2 \juhao \tag{2}
\end{numcases}

\jiange
分析:这个方程组可以象例 3 那样解。
但根据方程组的特点,也可以把方程 (1) 加上方程 $(2) \times 2$,得到一个新方程,
它的左边是一个完全平方式,右边是常数,通过两边开平方,就可以得到两个一次方程;
同样,把方程 (1) 减去方程 $(2) \times 2$, 也可以由此得到两个一次方程。
这两对一次方程同例 2 一样,一共可以组成四个二元一次方程组。
解这四个二元一次方程组,就可以得到原方程组的所有的解。

\jie $(1) + (2) \times 2$,得
$$ (x + y)^2 = 9 \douhao $$

$\therefore$
\vspace{-1.5em}\begin{equation}
    x + y = \pm 3 \juhao \tag{3}
\end{equation}

$(1) - (2) \times 2$,得
$$ (x - y)^2 = 1 \douhao $$

$\therefore$
\vspace{-1.5em}\begin{equation}
    x - y = \pm 1 \juhao \tag{4}
\end{equation}

由 (3),(4),原方程组可化为四个方程组
\begin{center}
    \begin{tblr}{columns={mode=math}}
        \begin{cases}
            x + y = 3 \douhao \\
            x - y = 1 \douhao
        \end{cases} & \begin{cases}
            x + y = 3 \douhao \\
            x - y = -1 \douhao
        \end{cases} \\
        \begin{cases}
            x + y = -3 \douhao \\
            x - y = 1 \douhao
        \end{cases} & \begin{cases}
            x + y = -3 \douhao \\
            x - y = -1 \juhao
        \end{cases}
    \end{tblr}
\end{center}

解这四个方程组,得原方程组的解为
$$\begin{cases}
    x_1 = 2 \douhao \\
    y_1 = 1 \fenhao
\end{cases} \quad \begin{cases}
    x_2 = 1 \douhao \\
    y_2 = 2 \fenhao
\end{cases} \quad \begin{cases}
    x_3 = -1 \douhao \\
    y_3 = -2 \fenhao
\end{cases} \quad \begin{cases}
    x_4 = -2 \douhao \\
    y_4 = -1 \juhao
\end{cases}$$


\lianxi
\begin{xiaotis}

解下列方程组(第 1 ~\, 2 题):

\xiaoti{}%
\begin{xiaoxiaotis}%
    \huitui\begin{tblr}[t]{columns={18em, colsep=0pt}}
        \xxt{$\begin{cases}
                x^2 - 2xy - y^2 = 2 \douhao \\
                xy + y^2 = 4 \fenhao
            \end{cases}$} & \xxt{$\begin{cases}
                3x^2 - y^2 = 8 \douhao \\
                x^2 + xy - y^2 = 4 \juhao
            \end{cases}$}
    \end{tblr}

\end{xiaoxiaotis}


\xiaoti{}%
\begin{xiaoxiaotis}%
    \huitui\begin{tblr}[t]{columns={18em, colsep=0pt}}
        \xxt{$\begin{cases}
                x^2 + y^2 = 20 \douhao \\
                xy = 8 \fenhao
            \end{cases}$} & \xxt{$\begin{cases}
                x^2 + y^2 = 13 \douhao \\
                xy  = -6 \juhao
            \end{cases}$}
    \end{tblr}

\end{xiaoxiaotis}


\xiaoti{(口答)已知方程组
    \begin{numcases}{}
        x^2 + y^2 = 5 \douhao \tag{1} \\
        x^2 - y^2 = 3 \juhao \tag{2}
    \end{numcases}
    下面的解法是否正确?如果不正确,应当怎样改正?
}

\hspace*{1.5em}\begin{minipage}{0.9\textwidth}
    \jie $[(1) + (2)] \div 2$,得
    $$ x^2 = 4 \douhao $$

    $\therefore$
    \vspace{-1.5em}$$ x = \pm 2 \juhao $$

    $[(1) - (2)] \div 2$,得
    $$ y^2 = 1 \douhao $$

    $\therefore$
    \vspace{-1.5em}$$ y = \pm 1 \juhao $$

    因此,原方程组的解是
    $$\begin{cases}
        x_1 = 2 \douhao \\
        y_1 = 1 \fenhao \\
    \end{cases} \quad \begin{cases}
        x_2 = -2 \douhao \\
        y_2 = -1 \juhao
    \end{cases}$$
\end{minipage}

\end{xiaotis}
\lianxijiange


\liti 解方程组
\begin{numcases}{}
    x^2 - 2y^2  - y = 1 \douhao \tag{1} \\
    2x^2 - 4y^2 + x = 6 \juhao \tag{2}
\end{numcases}


\begin{enhancedline}
分析:这个方程组的两个方程的二次项系数对应比例 $\left(\text{即} \dfrac{1}{2} = \dfrac{-2}{-4}\right)$,
把方程 (2) 减去方程 $(1) \times 2$,就可得到一个一次方程。
解由这个一次方程和原方程组的任何一个方程组成的方程组,就可以求得原方程组的解。
\end{enhancedline}

\jie $(2) - (1) \times 2$,得
$$ x + 2y = 4 \douhao $$

称项,得
\begin{equation}
    x = 4 - 2y \juhao \tag{3}
\end{equation}

把 (3) 代入 (1) 并整理,得
$$ 2y^2 - 17y + 15 = 0 \juhao $$

解这个方程,得
$$ y_1 = 1 \nsep y_2 = \dfrac{15}{2} \juhao $$

把 $y_1 = 1$ 代入 (3),得
$$ x_1 = 2 \fenhao $$

把 $y_2 = \dfrac{15}{2}$ 代入 (3),得
$$ x_2 = -11 \juhao $$

所以原方程组的解是
$$\begin{cases}
    x_1 = 2 \douhao \\
    y_1 = 1 \fenhao
\end{cases} \quad \begin{cases}
    x_2 = -11 \douhao \\
    y_2 = \dfrac{15}{2} \juhao
\end{cases}$$


\liti 解方程组
\begin{numcases}{}
    3x^2 - 9xy + 2y^2 - 6x - 3y + 1 = 0 \douhao \tag{1} \\
    x^2 - 3xy + y^2 - 2x - 3y + 3 = 0 \juhao \tag{2}
\end{numcases}

\begin{enhancedline}
分析:在这个方程组的两个方程中,含 $x$ 的项的系数对应成比例
$\left(\text{即} \dfrac{3}{1} = \dfrac{-9}{-3} = \dfrac{-6}{-2}\right)$,
用加减法可以得到一个只含未知数 $y$ 的方程.解这个方程求出 $y$,再代入原方程组的任何一个方程,就可求得 $x$。
\end{enhancedline}

\jie $(2) \times 3 - (1)$,得
$$ y^2 - 6y + 8 = 0 \juhao $$

解这个方程,得
$$ y = 2 \text{,或\quad} y = 4 \juhao $$

原方程组可化为两个方程组
\begin{gather*}
    \begin{cases}
        y = 2 \douhao \\
        x^2 - 3xy + y^2 - 2x - 3y + 3 = 0 \douhao
    \end{cases} \\
    \begin{cases}
        y = 4 \douhao \\
        x^2 - 3xy + y^2 - 2x - 3y + 3 = 0 \juhao
    \end{cases}
\end{gather*}

解这两个方程组,得原方程组的解为
\begin{center}
    \begin{tblr}{columns={mode=math}}
        \begin{cases}
            x_1 = 4 + \sqrt{15} \douhao \\
            y_1 = 2 \fenhao
        \end{cases} & \begin{cases}
            x_2 = 4 - \sqrt{15} \douhao \\
            y_2 = 2 \fenhao
        \end{cases} & \begin{cases}
            x_3 = 7 + \sqrt{42} \douhao \\
            y_3 = 4 \fenhao
        \end{cases} & \begin{cases}
            x_4 = 7 - \sqrt{42} \douhao \\
            y_4 = 4 \juhao
        \end{cases}
    \end{tblr}
\end{center}


\lianxi
\begin{xiaotis}

解下列方程组:

\xiaoti{}%
\begin{xiaoxiaotis}%
    \huitui\begin{tblr}[t]{columns={18em, colsep=0pt}}
        \xxt{$\begin{cases}
                x + y + xy = 5 \douhao \\
                2x + y - xy = 2 \fenhao
            \end{cases}$} & \xxt{$\begin{cases}
                2x^2 - xy - 3x = 0 \douhao \\
                xy - 2x^2 - 2y + 1 = 0 \juhao
            \end{cases}$}
    \end{tblr}

\end{xiaoxiaotis}


\xiaoti{}%
\begin{xiaoxiaotis}%
    \huitui\begin{tblr}[t]{columns={18em, colsep=0pt}}
        \xxt{$\begin{cases}
                2x^2 + y^2 + x - y = 12 \douhao \\
                x^2 + y^2 - y = 6 \fenhao
            \end{cases}$} & \xxt{$\begin{cases}
                2x^2 - y^2 + 4x - 2y = 13 \douhao \\
                x^2 - y^2 + 2x + y = 8 \juhao
            \end{cases}$}
    \end{tblr}

\end{xiaoxiaotis}

\end{xiaotis}


