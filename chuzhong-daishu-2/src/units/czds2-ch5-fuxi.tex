\fuxiti
\begin{xiaotis}

\xiaoti{}%
\begin{xiaoxiaotis}%
    \xxt[\xxtsep]{已知二元一次方程
        $$ 3x - 2y = 5 \juhao $$
        把它变形为
        $$ y = \dfrac{3x - 5}{2} \juhao $$
    }

    \begin{center}
    \begin{minipage}{3cm}
        \centering
        \begin{tikzpicture}[every node/.style={minimum size=1cm,draw}, >=Stealth]
    \foreach \text  [count=\i] in {
        x, \times 3, -5, \div 2
    } {
        \pgfmathsetmacro{\y}{-\i * 1.5}
        \draw (0, \y) node {$\text$};
        \draw [->] (0, \y-.5) -- (0, \y-1);
    }
    \draw (0, -7.5) node {$y$};
\end{tikzpicture}

    \end{minipage}
    \begin{minipage}{6cm}
        \setlength{\parindent}{2em}
        我们可以根据给定的 $x$ 值,按左边的程序框图进行计算, 求出对应的 $y$ 值。

        按照这个程序计算出右表的 $y$ 值。
    \end{minipage}
    \qquad
    \begin{minipage}{3cm}
        \begin{tblr}{columns={3em, colsep=0pt, $, c}, hlines, vlines}
            x  & y \\
            1  & -1 \\
            2  & \dfrac{1}{2} \\
            3  & \\
            0  & \\
            -1 & \\
            -2 & \\
            -3 & \\
        \end{tblr}
    \end{minipage}
    \end{center}

    \jiange
    \xxt{已知二元一次方程 $y - 4x = 7$。仿照上题编出计算 $y$ 的程序框图,
        并算出 $x = 1,\, 2,\, 0,\, -1,\, -2$ 时对应的 $y$ 值。
    }

\end{xiaoxiaotis}

\xiaoti{}%
\begin{xiaoxiaotis}%
    \xxt[\xxtsep]{有一个两位数,它的十位上的数与个位上的数的和为 5,求出所有符合这个条件的两位数;}

    \xxt{求出方程 $2x + y = 9$ 在正整数范围内的解。}

\end{xiaoxiaotis}

\xiaoti{}%
\begin{xiaoxiaotis}%
    \xxt[\xxtsep]{已知
        $$\begin{cases} x = 5, \\ y = 7 \end{cases}$$
        满足方程 $kx - 2y = 1$,求 $k$ 的值。
    }

    \xxt{已知
        $$\begin{cases} x = 2, \\ y = 1 \end{cases}$$
        是方程组
        $$\begin{cases}
            ax - 3y = 1, \\
            x + by = 5
        \end{cases}$$
        的解,求 $a$,$b$ 的值。
    }

\end{xiaoxiaotis}

\xiaoti{判断方程组
    $$\begin{cases}
        3x - 4y = 7, \\
        2x + 3y = -1
    \end{cases}$$
    的解是不是方程 $5x - y = 6$ 的一个解。
}

\xiaoti{}%
\begin{xiaoxiaotis}%
    \xxt[\xxtsep]{解不等式
        \begin{align*}
            & 2x - 3 > 5(x - 3), \\
            & \dfrac{x + 2}{4} - \dfrac{2x - 3}{6} < 1 \juhao
        \end{align*}
    }

    \xxt{这两个不等式的解集有公共部分吗?如果有,就把它在数轴上表示出来。}

\end{xiaoxiaotis}

\xiaoti{解下列方程组:}
\begin{xiaoxiaotis}

    \begin{tblr}{columns={18em, colsep=0pt}}
        \xxt{$\begin{cases} 110 = 5I_1 - I_2, \\ 110 = 9I_2 - I_1; \end{cases}$} & \xxt{$\begin{cases} 7I_1 - 3I_2 = 5, \\ -5I_1 + 6I_2 = -6; \end{cases}$} \\
    \end{tblr}

    \begin{tblr}{columns={18em, colsep=0pt}}
        \xxt{$\begin{cases} \dfrac{x}{2} + \dfrac{y}{3} = 2, \\ 0.2x + 0.3y = 2.8; \end{cases}$} & \xxt{$\begin{cases} 0.2x - 0.5y = 0, \\ 5(x + 1) - 3(y + 17) = 0;\end{cases}$} \\
        \xxt{$ 3x + 2y = 5x + 12x = -3$ \\
            提示:先把它写成方程组 \\
            \hspace*{2em} $\begin{cases}
                3x + 2y = -3, \\
                5y + 12x = -3
            \end{cases}$ \\
            的形式);}
            & \xxt{$\dfrac{2v + t}{3} = \dfrac{3v - 2t}{8} = 3$;} \\
        \xxt{$\begin{cases} \dfrac{m + n}{3} - \dfrac{n - m}{4} = 2, \\[.5em] 4m + \dfrac{n}{3} = 14; \end{cases}$} & \xxt{$\begin{cases} 7 + \dfrac{x - 3y}{4} = 2x - \dfrac{y + 5}{3}, \\[.5em] \dfrac{10(x - y) - 4(1 - x)}{3} = y; \end{cases}$} \\
        \xxt{$\begin{cases} x + y = 1, \\ y + z = 6, \\ z + x = 3; \end{cases}$} & \xxt{$\begin{cases} x + y - z = 11, \\ y + z - x = 5, \\ z + x - y = 1 \juhao \end{cases}$} \\
    \end{tblr}

\end{xiaoxiaotis}

\xiaoti{解下列关于 $x$,$y$ 的方程组:}
\begin{xiaoxiaotis}

    \begin{tblr}{columns={18em, colsep=0pt}}
        \xxt{$\begin{cases} x + y = a,  \\ x - y = b; \end{cases}$} & \xxt{$\begin{cases} y = x + c, \\ x + 2y = 5c; \end{cases}$} \\
        \xxt{$\begin{cases} \dfrac{x}{2} + \dfrac{y}{3} = 3a, \\ x - y = a; \end{cases}$} & \xxt{$\begin{cases} x + y - n = 0, \\ 5x - 3y + n = 0 \juhao \end{cases}$} \\
    \end{tblr}

\end{xiaoxiaotis}

列出一次方程组解下列应用题:

\xiaoti{有一个两位数, 十位上的数与个位上的数的和是 13, 如果把这两个数的位置对换,
    那么所得的新数比原数小 27。 求这个两位数。
}

\xiaoti{有一个两位数, 个位上的数比十位上的数大 5 , 如果把这两个数的位置对换,
    那么所得的新数与原数的和为 143。求这个两位数。
}

\begin{enhancedline}
\xiaoti{由实验得出, 一块重量是 148 千克的铜银合金在水中减轻 $14\dfrac{2}{3}$ 千克。
    已知 21 千克的银在水中减轻 2 千克, 9 千克的铜在水中减轻 1 千克。
    这块合金内含银、铜各多少千克?
}

\xiaoti{(我国古代问题)\footnotemark  有大小两种盛米的桶,
    已经知道 5 个大桶加上 1 个小桶可以盛米 3 斛(斛, 音 \pinyin{hu2}, 是古代的一种容量单位),
    1 个大桶加上 5 个小桶可以盛米 2 斛。问 1 个大桶、1 个小桶各可以盛米多少。
}
\footnotetext{这道题选自古代算书《九章算术》卷七 “盈不足”。原题是: “今有大器五小器一容三斛,
    大器一小器五容二斛,问大小器各容几何。 答曰:大器二十四分斛之十三, 小器二十四分斛之七”。
}

\xiaoti{}%
\begin{xiaoxiaotis}%
    \xxt[\xxtsep]{在公式 $s = v_0 t + \dfrac{1}{2}at^2$ 中, 当 $t = 1$ 时, $s = 13$;
        当 $t = 2$ 时, $s = 42$。求 $v_0$,$a$ 的值, 并求当 $t = 3$ 时 $s$ 的值。
    }

    \xxt{在代数式 $ax^2 + bx + c$ 中, 当 $x = 1,\, 2,\, 3$ 时, 代数式的值分别是 0, 3 , 28。
        求 $a$,$b$,$c$ 的值。 当 $x = -1$ 时,这个代数式的值是多少?
    }

\end{xiaoxiaotis}
\end{enhancedline}

\end{xiaotis}

