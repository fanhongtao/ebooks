\subsection{积的乘方}\label{subsec:6-4}

我们来计算
$$(ab)^3\nsep (ab)^4 \juhao $$

根据乘方的意义和乘法交换律、结合律,得
\begin{align*}
    (ab)^3 &= (ab) \cdot (ab) \cdot (ab) = (aaa) \cdot (bbb) = a^3b^3 \douhao \\
    (ab)^4 &= (ab) \cdot (ab) \cdot (ab) \cdot (ab) = (aaaa) (bbbb) = a^4b^4 \juhao
\end{align*}

一般地,如果 $n$ 是正整数,那么
\begin{align*}
    (ab)^n &= \overbrace{(ab) \cdot (ab) \cdot \cdots \cdot (ab)}^{n \text{个} ab} \\
        &= (\overbrace{a \cdot a \cdot \cdots \cdot a}^{n \text{个} a}) \cdot (\overbrace{b \cdot b \cdot \cdots \cdot b}^{n \text{个} b}) \\
        &= a^n b^n \douhao
\end{align*}
即
\begin{center}
    \framebox{\quad $(ab)^n = a^n b^n$。\;}
\end{center}

这就是说,\zhongdian{积的乘方,等于把积的每一个因式分别乘方,再把所得的幂相乘。}

当三个或三个以上因式的积乘方时,也具有这一性质。例如
$$ (abc)^n = a^n b^n c^n \juhao $$

\liti 计算
\begin{xiaoxiaotis}

    \begin{tblr}{columns={12em, colsep=0pt}}
        \xxt{$(xy)^5$;} & \xxt{$(2a)^4$;} & \xxt{$(-3x)^3$;} \\
        \xxt{$(-5ab)^2$;} & \xxt{$(-xy)^6$;} & \xxt{$(4xy)^2$。}
    \end{tblr}

\resetxxt
\jie \xxt{$(xy)^5 = x^5 y^5$;}

\xxt{$(2a)^4 = 2^4 a^4 = 16a^4$;}

\xxt{$(-3x)^3 = (-3)^3 x^3 = -27x^3$;}

\xxt{$(-5ab)^2 = (-5)^2 a^2 b^2 = 25a^2b^2$;}

\xxt{$(-xy)^6 = (-1)^6 x^6 y^6 = x^6y^6$;}

\xxt{$(4xy)^2 = 4^2x^2y^2 = 16x^2y^2$。}

\end{xiaoxiaotis}

\begin{enhancedline}
\liti 计算:
\begin{xiaoxiaotis}

    \begin{tblr}{columns={18em, colsep=0pt}}
        \xxt{$(xy^2)^2$;} & \xxt{$(a^2b^2)^4$;} \\
        \xxt{$(-2xy^3)^4$;} & \xxt{$\left(\dfrac{2}{3}a\right)^2$。}
    \end{tblr}

\resetxxt
\jie \xxt{$(xy^2)^2 = x^2 (y^2)^2 = x^2y^4$;}

\xxt{$(a^2b^2)^4 = (a^2)^4 \cdot (b^2)^4 = a^8b^8$;}

\xxt{$(-2xy^3)^4 = (-2)^4 \cdot x^4 \cdot (y^3)^4 = 16x^4y{12}$;}

\xxt{$\left(\dfrac{2}{3}a\right)^2 = \left(\dfrac{2}{3}\right)^2 \cdot a^2 = \dfrac{4}{9}a^2$。}

\end{xiaoxiaotis}


\liti 计算:
\begin{xiaoxiaotis}

    \twoInLine[18em]{\xxt{$(2x)^3 \cdot (-5x^2y)$;}}{\xxt{$(3xy^2)^2 + (-4xy^3) \cdot (-xy)$。}}

\resetxxt
\jie \xxt{$(2x)^3 \cdot (-5x^2y) = 8x^3 \cdot (-5x^2y) = -40x^5y$;}

\xxt{$(3xy^2)^2 + (-4xy^3) \cdot (-xy) = 9x^2y^4 + 4x^2y^4 = 13x^2y^4$。}

\end{xiaoxiaotis}
\end{enhancedline}

\lianxi
\begin{xiaotis}

\xiaoti{(口答)下列各式的结果是什么?}
\begin{xiaoxiaotis}

    \begin{tblr}{columns={12em, colsep=0pt}}
        \xxt{$(ab)^6$;} & \xxt{$(xy)^4$;} & \xxt{$(2m)^3$;} \\
        \xxt{$(5x^2)^2$;} & \xxt{$(ab^2)^3$;} & \xxt{$(-xy)^3$。}
    \end{tblr}

\end{xiaoxiaotis}

\xiaoti{计算:}
\begin{xiaoxiaotis}

    \begin{tblr}{columns={12em, colsep=0pt}}
        \xxt{$(st)^3$;} & \xxt{$(4a^3)^2$;} & \xxt{$(-2x^2y)^2$;} \\
        \xxt{$\left(\dfrac{1}{2}c^2d\right)^3$;} & \xxt{$(2 \times 10^2)^2$;} & \xxt{$(x^2 \cdot x \cdot x^5)^3$;} \\
        \xxt{$(ab^2)^3 \cdot (ab^2)^2$;} & \xxt{$(-3x^5)^3 \cdot x^2$;} & \xxt{$a(ab^2)^2$;} \\
        \xxt{$(3y)^2 \cdot (y^2)^3$。}
    \end{tblr}

\end{xiaoxiaotis}

\xiaoti{计算:}
\begin{xiaoxiaotis}

    \begin{tblr}{columns={18em, colsep=0pt}}
        \xxt{$(a^2) \cdot (a^2b^3)^2$;} & \xxt{$(3m)^2 \cdot \left(-\dfrac{1}{2}mn\right)^3$;} \\
        \xxt{$2a^2 \cdot (-2a)^3 + 2a^4 \cdot 5a$;} & \xxt{$10a^3 \cdot \dfrac{3}{5}b + (-3.5a^2) \cdot (ab)^2$。}
    \end{tblr}

\end{xiaoxiaotis}

\xiaoti{某工厂要做一个棱长为 $4 \times 10^2$ 毫米的正方体油箱,求这个油箱的容积。}

\xiaoti{下面的计算对不对,为什么?如果不对,应怎样改正?}
\begin{xiaoxiaotis}

    \begin{tblr}{columns={18em, colsep=0pt}}
        \xxt{$(ab^2)^2 = ab^4$;} & \xxt{$(3xy)^3 = 9x^3y^3$;} \\
        \xxt{$(-2a^2)^2 = -4a^4$;} & \xxt{$\left(\dfrac{4}{5}x^2y^3\right)^3 = \dfrac{64}{125}x^6y^9$。}
    \end{tblr}

\end{xiaoxiaotis}

\end{xiaotis}

