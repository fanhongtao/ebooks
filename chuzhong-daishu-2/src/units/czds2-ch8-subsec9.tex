\subsection{繁分式}\label{subsec:8-9}
\begin{enhancedline}

我们前面所学的分式,它的分子和分母都是整式。如果分式的分子或分母中含有分式,这样的分式叫做\zhongdian{繁分式}。例如
$$ \dfrac{\dfrac{c}{b}}{\enspace a \enspace} \nsep \dfrac{1}{\dfrac{1}{R_1} + \dfrac{1}{R_2}} \nsep \dfrac{\dfrac{1}{a} + \dfrac{1}{b}}{\dfrac{1}{a} - \dfrac{1}{b}} $$
等都是繁分式。

与繁分数的化简类似,可以把繁分式写成分子除以分母的形式,利用除法法则进行化简。例如
\begin{align*}
    \dfrac{\dfrac{1}{a} + \dfrac{1}{b}}{\dfrac{1}{a} - \dfrac{1}{b}}
        &= \left(\dfrac{1}{a} + \dfrac{1}{b}\right) \div \left(\dfrac{1}{a} - \dfrac{1}{b}\right) = \dfrac{b + a}{ab} \div \dfrac{b - a}{ab} \\
        &= \dfrac{b + a}{ab} \cdot \dfrac{ab}{b - a}  = \dfrac{b + a}{b - a} \juhao
\end{align*}
也可以利用分式的基本性质化简繁分式,例如
\begin{gather*}
    \dfrac{\dfrac{1}{a} + \dfrac{1}{b}}{\dfrac{1}{a} - \dfrac{1}{b}}
        = \dfrac{\left(\dfrac{1}{a} + \dfrac{1}{b}\right) \cdot ab}{\left(\dfrac{1}{a} - \dfrac{1}{b}\right) \cdot ab}
        = \dfrac{b + a}{b - a} \juhao
\end{gather*}

\lianxi
\begin{xiaotis}

\xiaoti{(口答)化简:}
\begin{xiaoxiaotis}

    \fourInLineXxt{$\dfrac{1}{\enspace \dfrac{x}{y} \enspace}$;}
                  {$\dfrac{\dfrac{1}{x}}{\enspace y \enspace}$;}
                  {$\dfrac{\dfrac{b}{a}}{\enspace \dfrac{d}{c} \enspace}$;}
                  {$\dfrac{\dfrac{b}{a}}{\enspace c \enspace}$。}

\end{xiaoxiaotis}

\xiaoti{化简:}
\begin{xiaoxiaotis}

    \threeInLineXxt{$\dfrac{\dfrac{1 + a}{b}}{\enspace \dfrac{1 - a}{b} \enspace}$;}
                  {$\dfrac{1 + \dfrac{y}{x}}{1 - \dfrac{y}{x}}$;}
                  {$\dfrac{1}{\dfrac{1}{R_1} + \dfrac{1}{R_2}}$。}

\end{xiaoxiaotis}

\end{xiaotis}

\end{enhancedline}

