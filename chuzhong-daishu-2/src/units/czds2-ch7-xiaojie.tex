\xiaojie

一、本章主要内容是因式分解的概念和多项式因式分解的几种常用方法。

二、学习多项式的因式分解,要注意因式分解与整式乘法的关系。
整式乘法是把几个整式相乘,化为一个多项式;而因式分解是把一个多项式化为几个整式相乘。
例如:把 $(a + b)(a - b)$ 化为 $a^2 - b^2$ 是整式乘法;
把 $a^2 - b^2$ 化为 $(a + b)(a - b)$ 是因式分解。这就是说,
\vspace*{0.5em} $$ (a + b)(a - b) \;
        \tikz[overlay, >=Stealth] {
            \draw [->] (0, 0.15) -- (1.5, 0.15) node [midway, above] {整式乘法};
            \draw [->] (1.5, -0.05) -- (0, -0.05) node [midway, below] {因式分解};
        } \hspace*{4em} \;
    a^2 - b^2 \vspace*{1em} \juhao $$

% 另一种使用表格来实现的写法
% \begin{center}
%     \begin{tblr}{
%         columns={colsep=0pt, halign=c, valign=m},
%         column{2}={4.5em},
%     }
%         $(a + b)(a - b)$
%             & {整式乘法\\[-.5em] \tikz[>=Stealth] {
%                 \draw [->] (0, 0.2) -- (1.5, 0.2);
%                 \draw [->] (1.5, 0) -- (0, 0);
%             } \\[-.5em] 因式分解}
%             & $a^2 - b^2$
%     \end{tblr}
% \end{center}

什么时候用整式乘法,什么时候用因式分解,是根据需要而定的。

三、提公因式法是因式分解中最基本的方法。只要多项式的各项有公因式,首先把它提出来。

运用公式法关键在于熟悉公式,掌握它们的不同形式和特点。本章学习了五个公式:
\begin{align*}
    & a^2 - b^2 = (a + b)(a - b) \douhao \\
    & a^2 \pm 2ab + b^2 = (a \pm b)^2 \douhao \\
    & a^3 \pm b^3 = (a \pm b)(a^2 \mp ab + b^2) \juhao
\end{align*}

可化为 $x^2 + (a + b)x + ab$ 型的二次三项式能分解为 $(x + a)(x + b)$。
这里关键在于掌握 $a$,$b$ 与常数项、一次项系数的关系,
要通过观察和试验把常数项分解成两个因数 $a$,$b$ 的积,而 $a$,$b$ 的和必须等于一次项的系数。

分组分解法必须预见到下一步分解的可能性。我们已学习了分组后能提公因式及分组后能运用公式这两种情况。

因为需要分解因式的多项式是多种多样的,所以必须对具体情况作具体分析,灵活运用各种方法来分解因式。

