\subsection{幂的乘方}\label{subsec:6-3}

我们来计算
$$(a^4)^3\nsep (a^3)^5 \juhao $$

$(a^4)^3$ 是把幂 $a^4$ 三次方。如果把 $a^4$ 看作是底数,那么根据乘方的意义和底数的幂的乘法性质,得
$$ (a^4)^3 = a^4 \cdot a^4 \cdot a^4 = a^{4+4+4} = a^{4 \times 3} \douhao $$
同样,得
$$ (a^3)^5 = a^3 \cdot a^3 \cdot a^3 \cdot a^3 \cdot  a^3 = a^{3+3+3+3+3} = a^{3 \times 5} \juhao $$
也就是
\begin{align*}
    (a^4)^3 &= a^{4 \times 3} \douhao \\
    (a^3)^5 &= a^{3 \times 5} \juhao
\end{align*}

一般地,如果 $m$,$n$ 都是正整数,那么
\begin{align*}
    (a^m)^n &= \overbrace{a^m \cdot a^m \cdot \cdots \cdot a^m}^{n \text{个} a^m} \\
        &= a^{m \overbrace{+ m + \cdots + }^{n \text{个} m} m} = a^{mn} \douhao
\end{align*}
即
\begin{center}
    \framebox{\quad $(a^m)^n = a^{mn}$。\;}
\end{center}

这就是说,\zhongdian{幂的乘方,底数不变,指数相乘。}


\liti 计算:

\begin{xiaoxiaotis}

    \threeInLineXxt{$(10^7)^2$;}{$(x^3)^2$;}{$(z^4)^4$。}

\resetxxt
\jie \xxt{$(10^7)^2 = 10^{7 \times 2} = 10^{14}$;}

\xxt{$(x^3)^2 = x^{3 \times 2} = x^6$;}

\xxt{$(z^4)^4 = z^{4 \times 4} = z^{16}$。}

\end{xiaoxiaotis}

\liti 计算:
\begin{xiaoxiaotis}

    \twoInLineXxt{$(a^m)^2$;}{$(b^3)^n$。}

\resetxxt
\jie \xxt{$(a^m)^2 = a^{m \times 2} = a^{2m}$;}

\xxt{$(b^3)^n = b^{3 \times n} = b^{3n}$。}

\end{xiaoxiaotis}

\liti 计算:
\begin{xiaoxiaotis}

    \twoInLineXxt{$[(x + y)^2]^4$;}{$(a^2)^4 \cdot (a^3)^3$。}

分析: 在第(1)小题中,把 $(x + y)^2$ 看作一个字母的幂进行计算;
在第(2)小题中,先分别计算 $(a^2)^4$,$(a^3)^3$,然后根据同底数的幂的乘法性质进行计算。

\resetxxt
\jie \xxt{$[(x + y)^2]^4 = (x + y)^{2 \times 4} = (x + y)^8$;}

\xxt{$(a^2)^4 \cdot (a^3)^3 = a^8 \cdot a^9 = a^{17}$。}

\end{xiaoxiaotis}


\lianxi
\begin{xiaotis}

\xiaoti{(口答)计算:}
\begin{xiaoxiaotis}

    \begin{tblr}{columns={12em, colsep=0pt}}
        \xxt{$(x^4)^2$;} & \xxt{$x^4 \cdot x^2$;} & \xxt{$(y^5)^5$;} \\
        \xxt{$y^5 \cdot y^5$;} & \xxt{$(a^m)^3$;} & \xxt{$a^m \cdot a^3$。}
    \end{tblr}

\end{xiaoxiaotis}

\xiaoti{计算:}
\begin{xiaoxiaotis}

    \begin{tblr}{columns={12em, colsep=0pt}}
        \xxt{$(10^3)^3$;} & \xxt{$(x^4)^3$;} & \xxt{$(a^2)^5$;} \\
        \xxt{$-(y^2)^4$;} & \xxt{$-(x^3)^6$;} & \xxt{$(s^m)^5$;} \\
        \xxt{$[(x + a)^3]^2$;} & \xxt{$[(x + y)^n]^2$;} & \xxt{$(a^2)^3 \cdot a^5$;} \\
        \xxt{$(a^2)^5 \cdot (a^4)^4$;} & \xxt{$(b^3)^2 \cdot (b^2)^3$;} & \xxt{$(c^2)^n \cdot c^{n+1}$。}
    \end{tblr}

\end{xiaoxiaotis}

\xiaoti{下面的计算对不对,为什么?如果不对,应怎样改正?}
\begin{xiaoxiaotis}

    \begin{tblr}{columns={18em, colsep=0pt}}
        \xxt{$(a^5)^2 = a^7$;} & \xxt{$a^5 \cdot a^2 = a^{10}$;} \\
        \xxt{$(x^4)^7 = x^{28}$;} & \xxt{$(a^{n+1})^2 = a^{2n+1}$。}
    \end{tblr}

\end{xiaoxiaotis}

\end{xiaotis}


