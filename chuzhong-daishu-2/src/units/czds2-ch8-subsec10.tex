\subsection{含有字母已知数的一元一次方程}\label{subsec:8-10}
\begin{enhancedline}

我们先看下面的问题:

一个数的 $a$ 倍 ($a \neq 0$) 等于 $b$, 求这个数。

用 $x$ 表示这个数,根据题意,可得方程
$$ ax = b \quad (a \neq 0) \juhao $$
在这个方程中, $x$ 是未知数,$a$ 和 $b$ 是用字母表示的已知数。
对 $x$ 来说,字母 $a$ 是 $x$ 的系数,叫做\zhongdian{字母系数},字母 $b$ 是常数项。
这个方程就是一个含有字母已知数的一元一次方程。

以后如果没有特别说明,在含有字母已知数的方程中,
一般用 $a$,$b$,$c$ 等表示已知数,用 $x$,$y$,$z$ 等表示未知数。

含有字母已知数的方程的解法与前面学过的只含有数字已知数的方程的解法相同。
但必须特别注意:用含有字母的式子去乘或者除方程的两边,这个式子的值不能等于零。
例如,解方程 $ax = b$ 时,只有在 $a \neq 0$ 的条件下,才能用 $a$ 去除方程的两边而得
$$ x = \dfrac{b}{a} \juhao $$

\liti 解方程 $ax + b^2 = bx + a^2 \quad (a \neq b)$。

\jie 移项,得
$$ ax - bx = a^2 - b^2 \juhao $$

合并同类项,得
$$ (a - b)x = a^2 - b^2 \juhao $$

因为 $a \neq b$,所以 $a - b \neq 0$。方程的两边都除以 $a - b$,得
$$ x = \dfrac{a^2 - b^2}{a - b} \douhao $$
就是
$$ x = a + b \juhao $$

\liti 解方程 $\dfrac{x - b}{a} = 2 - \dfrac{x - a}{b} \quad (a + b \neq 0)$。

\jie 去分母,得
\begin{gather*}
    b(x - b) = 2ab - a(x - a) \douhao \\
    bx - b^2 = 2ab - ax + a^2 \douhao \\
    ax + bx = a^2 + 2ab + b^2 \douhao \\
    (a + b)x = (a + b)^2 \juhao
\end{gather*}

\fengeYinwei{a + b \neq 0 \douhao}

\fengeSuoyi{x = a + b \juhao}

我们再看下面的问题:

汽车的行驶速度是 $v$(千米/时),行驶的时间是 $t$(小时),那么汽车行驶的路程 $s$(千米)可以用公式
$$ s = vt $$
来计算。

有时已知行驶的路程 $s$ 与行驶的速度 $v \; (v \neq 0)$,要求行驶的时间 $t$。
因为 $v \neq 0$, 在 $s = vt$ 的两边都除以 $v$,就得到
$$ t = \dfrac{s}{v} \juhao $$
这就是已知行驶的路程和速度,求行驶的时间的公式。

类似地,如果已知 $s$,$t \; (t \neq 0)$,求 $v$,可以得到
$$ v = \dfrac{s}{t} \juhao $$
这就是已知行驶的路程和时间,求行驶的速度的公式。

以上三个公式都表示路程 $s$、时间 $t$、速度 $v$ 之间的关系。当 $v$,$t$ 都不等零时,
可以把公式 $s = vt$ 变换成公式 $t = \dfrac{s}{v}$ 和 $v = \dfrac{s}{t}$ 。

象上面这样,把一个公式从一种形式变换成另一种形式,叫做\zhongdian{公式变形}。
公式变形往往就是解含有字母已知数的方程。

\liti 在 $v = v_0 + at$ 中,已知 $v$,$v_0$,$a$ ,且 $a \neq 0$,求 $t$。

\jie 移项,得
$$ v - v_0 = at \juhao $$

因为 $a \neq 0$,方程两边都除以 $a$,得
$$ t = \dfrac{v - v_0}{a} \juhao $$

\liti 在梯形面积公式 $S = \dfrac{1}{2}(a + b)h$中,已知 $S$,$b$,$h$,且 $h \neq 0$,求 $a$。

\jie 去分母,得
$$ 2S = (a + b)h \douhao $$

整理,得
$$ ah = 2S - bh \juhao $$

因为 $h \neq 0$,方程两边都除以 $h$,得
$$ a = \dfrac{2S - bh}{h} \juhao $$

\lianxi
\begin{xiaotis}

\xiaoti{解下列方程($x$ 为未知数):}
\begin{xiaoxiaotis}

    \begin{tblr}{columns={colsep=0pt}, column{1}={18em}}
        \xxt{$3a + 4x = 7x - 5b$;} & \xxt{$ax - by = 0 \quad (a \neq 0)$;} \\
        \xxt{$\dfrac{x}{a} - b = \dfrac{x}{b} - a \quad (a \neq b)$;} & \xxt{$m^2(x - n) = n^2(x - m) \quad (m^2 \neq n^2)$。}
    \end{tblr}

\end{xiaoxiaotis}

\xiaoti{(口答)在公式 $F = ma$ 中,所有字母都不等于零。}
\begin{xiaoxiaotis}

    \begin{tblr}{columns={18em, colsep=0pt}}
        \xxt{已知 $F$,$a$,求 $m$;} & \xxt{已知 $F$,$m$,求 $a$。}
    \end{tblr}

\end{xiaoxiaotis}


\xiaoti{在公式 $v = v_0 + at$ 中,所有字母都不等于零。}
\begin{xiaoxiaotis}

    \begin{tblr}{columns={18em, colsep=0pt}}
        \xxt{已知 $v$,$a$,$t$,求 $v_0$;} & \xxt{已知 $v$,$v_0$,$t$,求 $a$。}
    \end{tblr}

\end{xiaoxiaotis}


\xiaoti{在梯形面积公式 $S = \dfrac{1}{2}(a + b)h$ 中,所有字母都是正数。}
\begin{xiaoxiaotis}

    \begin{tblr}{columns={18em, colsep=0pt}}
        \xxt{已知 $S$,$a$,$b$,求 $h$;} & \xxt{已知 $S$,$a$,$h$,求 $b$。}
    \end{tblr}

\end{xiaoxiaotis}

\end{xiaotis}

\end{enhancedline}

