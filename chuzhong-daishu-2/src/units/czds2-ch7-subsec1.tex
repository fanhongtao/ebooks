% 原书的目录结构就是如此(缺少 section)
% 忽略这里的报错: Difference (2) between bookmark levels is greater (hyperref)	than one, level fixed.
\subsection{因式分解}\label{subsec:7-1}

在算术里学习分数的时候,常常要进行约分与通分,因此,常常要把一个数分解因数(即分解约数),
例如,把 33 分解成 $3 \times 11$, 把 42 分解成 $2 \times 3 \times 7$。

在代数里学习分式的时候,也常常要进行约分与通分,因此,也常常要把一个多项式化成几个整式的积。

把一个多项式化成几个整式的积的形式,叫做\zhongdian{因式分解},也可以叫做\zhongdian{分解因式}。

因式分解与乘法正好相反。例如,从 $(a + b)(a - b)$ 求得 $a^2 - b^2$ 是我们学过的乘法,反过来,
从 $a^2 - b^2$ 求得 $(a + b)(a - b)$ 就是因式分解。因此,我们可以从整式乘法得出因式分解的某些方法。

下面学习几种常用的因式分解的方法,

