% 原书的目录结构就是如此(缺少 section)
% 忽略这里的报错: Difference (2) between bookmark levels is greater (hyperref)	than one, level fixed.
\subsection{二元一次方程}\label{subsec:5-1}

我们来看下面的问题:

已知两个数的和是 7 , 求这两个数。

这个问题里有两个未知数,如果设一个数是 $x$,另一个数是 $y$,那么根据题意,可以列出方程
$$ x + y = 7 \juhao $$

这个方程含有两个未知数,并且含有未知数的项的次数都是 1, 这样的方程叫做\zhongdian{二元一次方程}。

\begin{enhancedline}
当 $x = 3$,$y = 4$ 时, 方程 $x + y = 7$ 左右两边的值相等, 我们说 $x = 3$,$y = 4$ 是适合(或满足)方程 $x + y = 7$ 的。
适合一个二元一次方程的每一对未知数的值,叫做这个\zhongdian{二元一次方程的一个解}。
例如 $x = 3$,$y = 4$  就是方程 $x + y = 7$ 的一个解,我们把它记作
$$\begin{cases}
    x = 3 \douhao \\
    y = 4 \juhao
\end{cases}$$

要求二元一次方程 $x + y = 7$ 的解,可以把这个方程变形,用含有 $x$ 的代数式表示 $y$,得
$$ y = 7 - x \juhao $$
在这个方程里,如果 $x$ 取一个值,就可以求出与它对应的 $y$ 的一个值。例如:
\end{enhancedline}

取 $x = -1$, 可以得到 $y = 8$;

取 $x = 0$, 可以得到 $y = 7$;

取 $x = 2.7$, 可以得到 $y = 4.3$;

取 $x = 5$, 可以得到 $y = 2$;

………… \hspace*{3em} …………。\\
这样得到的每一对未知数的值都适合方程 $x + y = 7$, 所以它们都是这个方程的解。

对于任何一个二元一次方程,让其中一个未知数取任意一个值,都可求出与它对应的另一个未知数的值。
因此,任何一个二元一次方程都有无数个解。

由二元一次方程的所有的解组成的集合,叫做这个二元一次方程的解集。

\lianxi
\begin{xiaotis}

\xiaoti{(口答)下列方程中,哪些是二元一次方程,哪些不是,为什么?}
\begin{xiaoxiaotis}

    \begin{tblr}{columns={18em, colsep=0pt}}
        \xxt{$2x - 3y = 9$;}           & \xxt{$x + 1 = 6z$;} \\
        \xxt{$\dfrac{1}{x} + 4 = 2y$;} & \xxt{$x -5 = 3y^2$。}
    \end{tblr}

\end{xiaoxiaotis}

\begin{enhancedline}
\xiaoti{(口答)在
    $\begin{cases}
        x = 0, \\
        y = -2,
    \end{cases}$ \quad
    $\begin{cases}
        x = 2, \\
        y = -3,
    \end{cases}$ \quad
    $\begin{cases}
        x = 1, \\
        y = -5
    \end{cases}$三对数值中,
}
\begin{xiaoxiaotis}

    \xxt{哪几对是方程 $2x - y = 7$ 的解?}

    \xxt{哪几对是方程 $x + 2y = -4$ 的解?}

\end{xiaoxiaotis}
\end{enhancedline}

\xiaoti{在下列方程中,用含 $x$ 的代数式表示 $y$:}
\begin{xiaoxiaotis}

    \begin{tblr}{columns={18em, colsep=0pt}}
        \xxt{$2x + y = 3$;} & \xxt{$3x - y = 2$;} \\
        \xxt{$x + 3y = 0$;} & \xxt{$2x - 3y + 5 = 0$。}
    \end{tblr}

\end{xiaoxiaotis}

\xiaoti{在方程 $3x + 2y = 12$ 中,设 $x = 2,\, 3,\, 4,\, 5$,分别求出对应的 $y$ 的值。}

\end{xiaotis}


