\subsection{同底数的幂的除法}\label{subsec:6-10}

现在我们来学习整式的除法。 我们知道, 除法是乘法的逆运算, 因此可以从整式乘法得出整式除法的法则。

这一节先研究两个同底数的幂的除法。 我们来计算
$$ 10^5 \div 10^3 \nsep 2^5 \div 2^3 \juhao $$

根据除法是乘法的逆运算, 我们知道, 计算被除数除以除数所得的商, 就是要求一个数, 使它与除数的积等于被除数。

\begin{tblr}{columns={$, colsep=0pt}, column{1}={.35\textwidth}}
    \because    & 10^2 \times 10^3 = 10^5 \douhao \\
    \therefore  & 10^5 \div 10^3 = 10^2 \fenhao \\
    \because    & 2^2 \times 2^3 = 2^5 \douhao \\
    \therefore  & 2^5 \div 2^3 = 2^2 \juhao
\end{tblr}\\
也就是
\begin{align*}
    10^5 \div 10^3 = 10^{5-3} \douhao \\
    2^5 \div 2^3 = 2^{5-3} \juhao
\end{align*}

同样,

%\fengeYinwei{a^2 \cdot a^3 = a^5 \douhao}

%\fengeSuoyi{a^5 \div a^3 = a^2 \juhao \footnote{这里 $a \neq 0$。在本章中,所遇到的除式的值都不等于零。}}\\
\begin{tblr}{columns={$, colsep=0pt}, column{1}={.35\textwidth}}
    \because    & a^2 \cdot a^3 = a^5 \douhao \\
    \therefore  & a^5 \div a^3 = a^2 \juhao \footnotemark
\end{tblr}\\
\footnotetext{这里 $a \neq 0$。在本章中,所遇到的除式的值都不等于零。}
也就是
$$ a^5 \div a^3 = a^{5-3} \juhao $$

一般地,如果 $m$,$n$ 都是正整数,并且 $m > n$,那么
\begin{center}
    \framebox{\quad $a^m : a^n = a^{m-n} \; (a \neq 0)$。\;}
\end{center}

这就是说,\zhongdian{同底数的幂相除,底数不变,指数相减。}

\begin{enhancedline}
同底数的幂相除,如果被除式的指数等于除式的指数,例如
$$ 3^2 \div 3^2 \nsep  \left(-\dfrac{1}{2}\right)^3 \div \left(-\dfrac{1}{2}\right)^3 \nsep a^m \div a^m \douhao $$
那么,可以看出所得的商等于 1。
\end{enhancedline}

这就是说,\zhongdian{指数相等的同底数的幂相除,商等于1。}

\liti 计算:
\begin{xiaoxiaotis}

    \begin{tblr}{columns={18em, colsep=0pt}}
        \xxt{$x^8 \div x^2$;} & \xxt{$a^9 \div a^4$;} \\
        \xxt{$(-a)^4 \div (-a)$。}
    \end{tblr}

\resetxxt
\jie \xxt{$x^8 \div x^2 = x^{8-2} = x^6$;}

\xxt{$a^9 \div a^4 = a^{9-4} = a^5$;}

\xxt{$\begin{aligned}[t]
        & (-a)^4 \div (-a) = (-a)^{4-1} \\
    ={} & (-a)^3 \\
    ={} & -a^3 \juhao
\end{aligned}$}

\end{xiaoxiaotis}


\liti 计算:
\begin{xiaoxiaotis}

    \begin{tblr}{columns={18em, colsep=0pt}}
        \xxt{$(ab)^5 \div (ab)^2$;} & \xxt{$(a + b)^3 \div (a + b)^2$;} \\
        \xxt{$y^{n+2} \div y^2$;} & \xxt{$x^{n+m} \div x^{n+m}$。}
    \end{tblr}

\resetxxt
\jie \xxt{$(ab)^5 \div (ab)^2 = (ab)^{5-2} = (ab)^3 = a^3b^3$;}

\xxt{$(a + b)^3 \div (a + b)^2 = (a + b)^{3-2} = a + b$;}

\xxt{$y^{n+2} \div y^2 = y^{n+2-2} = y^n$;}

\xxt{$x^{n+m} \div x^{n+m} = 1$。}

\end{xiaoxiaotis}


\lianxi
\begin{xiaotis}

\xiaoti{(口答)计算:}
\begin{xiaoxiaotis}

    \begin{tblr}{columns={12em, colsep=0pt}}
        \xxt{$x^7 \div x^5$;} & \xxt{$y^9 \div y^8$;} & \xxt{$z^{11} \div z^8$;} \\
        \xxt{$a^{10} \div a^3$;} & \xxt{$b^6 \div b^6$;} & \xxt{$c^7 \div c^7$。}
    \end{tblr}

\end{xiaoxiaotis}


\xiaoti{在下列各式的括号里填上适当的代数式,使等式成立:}
\begin{xiaoxiaotis}

    \begin{tblr}{columns={18em, colsep=0pt}}
        \xxt{$x^5 \cdot \ewkh = x^9$;} & \xxt{$a^6 \cdot \ewkh = a^{12}$;} \\
        \xxt{$b^3 \cdot b^3 \cdot \ewkh = b^{30}$;} & \xxt{$x^2 \cdot x^5 \cdot \ewkh = x^{20}$。}
    \end{tblr}

\end{xiaoxiaotis}

\xiaoti{下面的计算对不对?如果不对,应怎样改正?}
\begin{xiaoxiaotis}

    \begin{tblr}{columns={18em, colsep=0pt}}
        \xxt{$x^6 \div x^3 = x^2$;} & \xxt{$z^5 \div z^4 = z$;} \\
        \xxt{$a^3 \div a = a^3$;} & \xxt{$(-c)^4 \div (-c)^2 = -c^2$。}
    \end{tblr}

\end{xiaoxiaotis}


\xiaoti{计算:}
\begin{xiaoxiaotis}

    \begin{tblr}{columns={18em, colsep=0pt}}
        \xxt{$(xy)^5 \div (xy)^3$;} & \xxt{$(a + b)^5 \div (a + b)^4$;} \\
        \xxt{$a^{n+2}  \div a^{n+1}$;} & \xxt{$x^{12} \div x^3 \div x^4$;} \\
        \xxt{$y^{10} \div (y^4 \div y^2)$;} & \xxt{$(c^{4n} \div c^{2n}) \cdot c^{3n}$。}
    \end{tblr}

\end{xiaoxiaotis}

\end{xiaotis}

