\fuxiti
\begin{xiaotis}

\xiaoti{计算:}
\begin{xiaoxiaotis}

    \begin{tblr}{columns={18em, colsep=0pt}}
        \xxt{$x \cdot (x^2)^2 \cdot (x^3)^3$;} & \xxt{$2m^4 \cdot (-m^2)^2$;} \\
        \xxt{$(ab)^2 \cdot (-a)^2 \cdot (-b)^3$;} & \xxt{$2a^3b \cdot (-3ab)^3$;} \\
        \xxt{$(-0.4xy^3z) \cdot (-0.5x^2z)$;} & \xxt{$(-x^2y^n)^2 \cdot (xy)^3$;} \\
        \xxt{$\left(-\dfrac{2}{3}a^7b^5\right) \div \left(\dfrac{3}{2}a^5b^5\right)$;} & \xxt{$(2a)^3 \cdot b^4 \div 12a^3b^2$;} \\
        \xxt{$[(-2a^3b)^3]^2 \div (-3a^2b)^2$;} & \xxt{$(4x^{n+1}y^n)^2 \div [(-xy)^2]^n$。}
    \end{tblr}

\end{xiaoxiaotis}

\xiaoti{我国领土面积约是 $9.6 \times 10^6$ 平方公里,平均每平方公里的土地上,
    一年内从太阳得到的能量相当于燃烧 $1.3 \times 10^5$ 吨煤所产生的能量。 % 指数看不清楚,隐约是 5
    求我国领土上一年内从太阳得到的能量约相当于燃烧多少吨煤所产生的能量(保留两个有效数字)。
}

\xiaoti{卫星脱离地球进入太阳系的速度(即第二宇宙速度)是 $1.12 \times 10^4 \mmm$。
    计算 $3.6 \times 10^3$ 秒卫星行走多少米(保留两个有效数字)。
}


\xiaoti{计算:}
\begin{xiaoxiaotis}
\begin{enhancedline}

    \xxt{$2a^3b (3ab^2c - 2bc)$;}

    \xxt{$(0.3a^2 - 0.2a + 0.1) \times 0.2$;}

    \xxt{$\left(-\dfrac{2}{3}a\right) \left(\dfrac{1}{2}a^2 + \dfrac{1}{6}a - \dfrac{1}{4}\right)$;}

    \xxt{$(4\pi r^2h - 2\pi rh) \div 6\pi rh$;}

    \xxt{$\left(\dfrac{6}{5}a^3x^4 - 0.9ax^3\right) \div \dfrac{3}{5}ax^3$;}

    \xxt{$(3a^{n+4} + 2a^{n+1}) \div (-3a^{n-1})$;}

    \xxt{$6xy \cdot [x^2(5x + 3) - 3x^2(-4y)]$;}

    \xxt{$[5xy^2(x^2 - 3xy) - (-3x^2y)^3] \div 2x^2y^2$;}

    \xxt{$2a^2b - (-3a)^2 \cdot (2b) + (4a^3b^2)^2 \div 4a^4b^3$;}

    \xxt{$(3xy)^2 (x^2 - y^2) - (4x^2y^2)^2 \div 8y^2 + 9x^2y^4$。}

\end{enhancedline}
\end{xiaoxiaotis}

\xiaoti{任意想一个正整数 $n$,按下列程序计算下去,把答案填写在表中空格内。然后看看有什么规律,想想这是为什么。}

\begin{center}
    \begin{tikzpicture}[every node/.style={minimum size=1cm,draw}, >=Stealth]
    \foreach \text  [count=\i] in {
        $n$, 平方, $+n$, $\div n$, $- n$
    } {
        \pgfmathsetmacro{\x}{\i * 1.5}
        \draw (\x, 0) node {\text};
        \draw [->] (\x+.5, 0) -- (\x+1, 0);
    }
    \draw (9, 0) node {答案};
\end{tikzpicture}


    \jiange
    \begin{tblr}{
        columns={2em, c},
        column{1}={5em},
        hlines, vlines,
    }
        输入 $n$ & $3$ & & & & \\
        输出答案 & $1$ & & & &
    \end{tblr}
\end{center}


\xiaoti{计算:}
\begin{xiaoxiaotis}

    \begin{tblr}{columns={19.5em, colsep=0pt}, column{2}={22.5em}}
        \xxt{$(2a + 3b)(2a - 4b)$;} & \xxt{$(5a - b)(-a - 4b)$;} \\
        \xxt{$(9u - 2v)(u + v)$;} & \xxt{$(x^2 + 3)(x^2 - 2)$;} \\
        \xxt{$(3t^2 + 2r)(3t + 5r)$;} & \xxt{$(-3a^2b - 4ab) (-a^2 + 5ab^2)$;} \\
        \xxt{$(0.3a^2b - 0.4ab^2) (0.5ab^2 - 0.1a^2b)$;} & \xxt{$\left(\dfrac{3}{5}xy^3 - \dfrac{2}{3}x^2y\right) \left(\dfrac{5}{4}x^2y^2 + \dfrac{6}{5}xy\right)$;} \\
        \xxt{$(3x - 5) (x^2 - 7x + 3)$;} & \xxt{$(5a + 2b) (ab - 4a^2 + 3b^2)$;} \\
        \xxt{$3(2x - 1) (x + 6) - 5(x - 3)(x + 6)$;} & \xxt{$(x^3 + 2xy^2 - 3y^3) (2x - y) - 8xy(x^2 - y^2)$。}
    \end{tblr}

\end{xiaoxiaotis}


\xiaoti{计算:}
\begin{xiaoxiaotis}

    \begin{tblr}{columns={19.5em, colsep=0pt}}
        \xxt{$\left(\dfrac{1}{3}a^2 - \dfrac{1}{4}b\right) \left(-\dfrac{1}{4}b - \dfrac{1}{3}a^2\right)$;} & \xxt{$5x^2(x + 3)(x - 3)$;} \\
        \xxt{$\left(2x + \dfrac{1}{2}\right) \left(2x - \dfrac{1}{2}\right) \left(4x^2 + \dfrac{1}{4}\right)$;} & \xxt{$\left(\dfrac{7}{3}x + \dfrac{3}{2}y\right)^2$;} \\
        \xxt{$\left(\dfrac{2}{3}c^2 - 0.6d^2\right)^2$;} & \xxt{$4x(x - 1)^2 - x(2x + 5)(2x - 5)$;} \\
        \SetCell[c=2]{} \xxt{$3(2x + 1)(2x - 1) - 4\left(\dfrac{3}{2}x - 3\right) \left(\dfrac{3}{2}x + 3\right)$;}  \\
        \xxt{$5(2x + 5)^2 + (3x - 4)(-3x - 4)$;} & \xxt{$(x + 3y) (x^2 - 3xy + 9y^2)$;} \\
        \xxt{$(3a - 2b) (9a^2 + 6ab + 4b^2)$;} &  \xxt{$(x - y)^2 (x + y)^2$;} \\
        \xxt{$(2x + y - z)^2$;} & \xxt{$(x + y)^2 (x^2 - xy + y^2)^2$;}  \\
        \SetCell[c=2]{} \xxt{$\left[\left(\dfrac{1}{2}x - y\right)^2 - \left(\dfrac{1}{2}x + y\right)^2\right] \left(2x^2 - \dfrac{1}{2}y^2\right)$;} \\
        \xxt{$(2x + y - z + 5) (2x - y + z + 5)$;} \\
    \end{tblr}

    \begin{tblr}{
        columns={colsep=0pt} %columns={19.5em, colsep=0pt}
    }
        %\SetCell[c=2]{}
        \xxt{$(x + y - z) (x - y + z) - (x + y + z) (x - y - z)$。}
    \end{tblr}

\end{xiaoxiaotis}

\xiaoti{计算:}
\begin{xiaoxiaotis}

    \xxt{$(a + b)^2 + (a - b)^2 + (-2a - b)(a + 2b)$;}

    \xxt{$5(m + n)(m - n) - 2(m + n)^2 - 3(m -n)^2$;}

    \xxt{$(x - y)[(x + y)^2 - xy] + (x + y)[(x - y)^2 + xy]$;}

    \xxt{$(a + b + c)^2 + (a - b)^2 + (b - c)^2 + (c - a)^2$。}

\end{xiaoxiaotis}


\xiaoti{先化简,再求值:}
\begin{xiaoxiaotis}
\begin{enhancedline}

    \xxt{$(25y^2 - 5y + 1)(5y + 1) - 5(1 - 4y^2)$,其中 $y = \dfrac{2}{5}$;}

    \xxt{$8m^2 - 5m(-m + 3n) + 4m\left(-4m - \dfrac{5}{2}n\right)$,其中 $m = 2$,$n = -1$;}

    \xxt{$x(y - z) - y(z - x) + z(x - y)$,其中 $x = \dfrac{1}{2}$, $y = 1$,$z = -\dfrac{1}{2}$。}

\end{enhancedline}
\end{xiaoxiaotis}

% TODO: wrapfigure 在这里无法正常使用
\begin{minipage}{9cm}

\xiaoti{一条水渠,其横断面为梯形,长度如图所示,求横断面面积的代数式,并计算当 $a = 2$,$b = 0.8$ 时的面积。}

\xiaoti{已经甲数为 $2a$, 乙数比甲数的 2 倍多 3, 丙数比甲数的 2 倍少 3,求甲、乙、丙三数的积。当 $a = -2.5$ 时,积是多少?}

\end{minipage}
\begin{minipage}{6cm}
    \begin{figure}[H]
        \centering
        \begin{tikzpicture}[>=Stealth,
    every node/.style={fill=white, inner sep=1pt},
]
    \pgfmathsetmacro{\a}{2.5}
    \pgfmathsetmacro{\b}{1}
    \pgfmathsetmacro{\h}{\a - \b}

    \draw [ultra thick] (-\a/2, 0) -- (\a/2, 0) -- (\a/2+\b, \h)
        -- (-\a/2-\b, \h) -- cycle;
    \draw [thick] (-\a/2, 0) -- (-\a/2, \h);
    \draw [thick] (\a/2, 0) -- (\a/2, \h);

    \draw [<->] (-\a/2, -0.3)   to [xianduan={above=0.3cm}] node {$a$} (\a/2, -0.3);
    \draw [<->] (-\a/2, \h+0.3) to [xianduan={below=0.3cm}] node {$a$} (\a/2, \h+0.3);

    \draw [<->] (-\a/2-\b, \h+0.3) to [xianduan={below=0.3cm}] node {$b$} (-\a/2, \h+0.3);
    \draw [<->] (\a/2, \h+0.3)     to [xianduan={below=0.3cm}] node {$b$} (\a/2+\b, \h+0.3);

    \draw [<->] (\a/2+\b+0.3, 0) to [xianduan={above=1.3cm}]  node [rotate=90] {$a - b$} (\a/2+\b+0.3, \h);
\end{tikzpicture}

        \caption*{(第 10 题)}
    \end{figure}
\end{minipage}

\jiange
\xiaoti{}%
\begin{xiaoxiaotis}%
    \xxt[\xxtsep]{一个多项式除以 $x^2 - 4x + 1$,得 $x + 2$,求这个多项式;}

    \xxt{一个多项式乘以 $x + 4$,得 $x^3 + 3x^2 - 4x$,求这个多项式;}

    \xxt{一个多项式除以 $x^2 - 4x + 1$,商式为 $x + 1$,余式为 $3x + 1$,求这个多项式。}

\end{xiaoxiaotis}

\xiaoti{计算:}
\begin{xiaoxiaotis}

    \begin{tblr}{columns={18em, colsep=0pt}, column{2}={20em}}
        \xxt{$(4x^2 + 4x - 3) \div (2x + 3)$;} & \xxt{$(x^3 - 3x^2 - 9x + 22) \div (x - 2)$;} \\
        \xxt{$(6x^2 + 19x + 15) \div (2x + 5)$;} & \xxt{$(x^3 - x^2 + x - 1) \div (x^2 - 3x + 5)$。}
    \end{tblr}

\end{xiaoxiaotis}

\xiaoti{已知 $A = x^3 - 7x + 6$, $B = x^2 + 2x - 3$, $C = x^2 + x - 6$,计算:}
\begin{xiaoxiaotis}

    \begin{tblr}{columns={18em, colsep=0pt}}
        \xxt{$A \div (B - C)$;} & \xxt{$A \div B - A \div C$。}
    \end{tblr}

\end{xiaoxiaotis}

\begin{enhancedline}
\xiaoti{已知 $A = a^2 + b^2 + c^2$, $B = (a - b)^2 + (b - c)^2 + (c - a)^2$,计算 $A - \dfrac{B}{2}$。}

\xiaoti{}%
\begin{xiaoxiaotis}%
    \xxt[\xxtsep]{$a$, $b$ 是什么数时, $ab > 0$, $ab < 0$, $ab = 0$?}

    \xxt{$a$, $b$ 是什么数时, $a \div b$ 是正数,是负数,是零,没有意义?}

\end{xiaoxiaotis}


\xiaoti{解下列不等式:}
\begin{xiaoxiaotis}

    \xxt{$5x + 4 > 3x - 1$;}

    \xxt{$\dfrac{x - 1}{3} - \dfrac{x + 2}{6} < \dfrac{x}{2} - 2$;}

    \xxt{$(2x - 5)^2 + (3x + 1)^2 > 13(x^2 - 10)$;}

    \xxt{$(3x + 4) (3x - 4) < 9(x - 2) (x + 3)$。}

\end{xiaoxiaotis}

\xiaoti{解下列方程:}
\begin{xiaoxiaotis}

    \xxt{$x^2 - (x + 1)(x - 5) = 2(x - 5)$;}

    \xxt{$(x + 3)^2 + 2(x - 1)^2 = 3x^2 + 13$;}

    \xxt{$(x - 5)(x + 5) - (x + 1)(x + 5) = 24$;}

    \xxt{$(2x + 3)(x - 4) = (x - 2)(2x + 5)$;}

    \xxt{$(x - 1)^2 + 28 = \left(\dfrac{4}{3}x - 12\right) \left(\dfrac{3}{4}x - 12\right)$。}

\end{xiaoxiaotis}
\end{enhancedline}

\xiaoti{解下列方程组:}
\begin{xiaoxiaotis}

    \xxt{$\begin{cases}
        (x + 3)(y + 4) - xy = 17 \douhao \\
        x - y = 3 \fenhao
    \end{cases}$}

    \xxt{$\begin{cases}
        (x + 1)^2 - (x + 1)(x - 1) = y \douhao \\
        (y - 1)^2 - (y + 1)(y - 1) = x \fenhao
    \end{cases}$}

    \xxt{$\begin{cases}
        (x + 5)(y - 4) - xy = 0 \douhao \\
        3x - 2y = -1 \fenhao
    \end{cases}$}

    \xxt{$\begin{cases}
        (x + 2)^2 - (y - 3)^2 = (x + y)(x - y) \douhao \\
        x - 3y = 2 \juhao
    \end{cases}$}

\end{xiaoxiaotis}

\begin{withstar}
\xiaoti{两个整数相除,试商为 27 时,余数为 23;商为 28 时,余数为 8 。求被除数及除数。}

\xiaoti{}%
\begin{xiaoxiaotis}%
    \xxt[\xxtsep]{两个有理数的和、差、积、商(除数不为0)是不是有理数?}

    \xxt{两个整式的和、差、积是不是整式?}

\end{xiaoxiaotis}

\end{withstar}

\end{xiaotis}

