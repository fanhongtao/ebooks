% 原书的目录结构就是如此(缺少 section)
% 忽略这里的报错: Difference (2) between bookmark levels is greater (hyperref)	than one, level fixed.
\subsection{分式}\label{subsec:8-1}
\begin{enhancedline}

我们知道,两个数相除可以表示成分数的形式,即
$$ \text{被除数} \div \text{除数} = \dfrac{\text{被除数}}{\text{除数}} \juhao $$
例如,
$$ 2 \div 3 = \dfrac{2}{3} \juhao $$

因为零不能作除数,所以分数的分母不能是零。

在代数中,整式的除法也可以类似地表示。

例如,$n$ 公顷麦田共收小麦 $m$ 千克,平均每公顷产量 $(m \div n)$ 千克就可以用式子 $\dfrac{m}{n}$ 千克来表示。

又如,轮船在静水中小时走 $a$ 千米,水流的速度是 $b$ 千米每小时,轮船在逆流中航行 $s$ 千米
所需要的时间 $[s \div (a - b)]$ 小时就可以用式子 $\dfrac{s}{a - b}$ 小时来表示。

一般地,用 $A$,$B$ 表示两个整式,$A \div B$ 就可以表示成 $\dfrac{A}{B}$ 的形式。
如果除式 $B$ 中含有字母,式子 $\dfrac{A}{B}$ 就叫做\zhongdian{分式}。
其中,$A$ 叫做分式的分子,$B$ 叫做分式的分母。例如,
$$ \dfrac{m}{n}\nsep \dfrac{s}{a - b}\nsep \dfrac{x^2}{x + 3}$$
等,都是分式。

因为除式的值不能是零,所以分式的分母的值也不能是零。如果分式的分母的值是零,分式没有意义。

在本书中,如果没有特别说明,所遇到的分式都是有意义的,也就是分式里分母的值不等于零。例如,
在分式里 $\dfrac{m}{n}$ 里,$n \neq 0$;
在分式 $\dfrac{s}{a - b}$ 里,$a - b \neq 0$,即 $a \neq b$;
在分式 $\dfrac{x^2}{x + 3}$ 里,$x \neq -3$。

分式与分数有许多类似的地方,可以对比分数来学习分式。

整式和分式统称\zhongdian{有理式}。

\liti 当 $x$ 取什么数时,分式 $\dfrac{x + 1}{3x - 2}$ 有意义?

\jie 当分母等于零时,分式没有意义。此外,分式都有意义。

由分母 $3x - 2 = 0$,得 $x = \dfrac{2}{3}$。

$\therefore$ \quad 当 $x \neq \dfrac{2}{3}$ 时,分式 $\dfrac{x + 1}{3x - 2}$ 有意义。

\liti 当 $x$ 是什么数时,分式 $\dfrac{x + 2}{2x - 5}$ 的值是零?

\jie 当分子等于零而分母不等于零时,分式的值是零。

由分子 $x + 2 = 0$,得 $x = -2$。

而当 $x = -2$ 时,分母 $2x - 5 = -4 - 5 \neq 0$。

$\therefore$ \quad 当 $x = -2$ 时,分式 $\dfrac{x + 2}{2x - 5}$ 的值是零。

\lianxi
\begin{xiaotis}

\xiaoti{(口答)下列各有理式,哪些是整式,哪些是分式?\\
    $-3x \nsep \dfrac{x}{y}\nsep \dfrac{2}{3}x^2y - 7xy^2\nsep -\dfrac{1}{8}\nsep \dfrac{3}{5 + y}\nsep \dfrac{x - y}{5} \juhao$
}

\xiaoti{把下列各式中的商写成分式:
    $$s \div v \nsep 6000 \div ab\nsep (x - y) \div (x + y) \juhao$$
}

\xiaoti{当 $x$ 取什么数时,下列分式有意义?}
\begin{xiaoxiaotis}

    \threeInLineXxt{$\dfrac{1}{x}$;}{$\dfrac{2x}{x + 2}$;}{$\dfrac{x + 1}{2x - 5}$。}

\end{xiaoxiaotis}

\xiaoti{在下列分式中,$x$ 等于什么数时,分式的值等于零? $x$ 等于什么数时,分式没有意义?}
\begin{xiaoxiaotis}

    \twoInLineXxt{$\dfrac{5x}{x - 1}$;}{$\dfrac{3x - 4}{10x + 1}$。}

\end{xiaoxiaotis}

\end{xiaotis}
\lianxijiange

\end{enhancedline}

