\subsection{二元一次方程组}\label{subsec:5-2}

我们再来看下面的问题:

有甲、乙两个数, 甲数的 3 倍比乙数的 2 倍多 11 ,甲数的 2 倍与乙数的 3 倍的和是 16 , 求甲、乙两数。

这个问题, 用设一个未知数列一元一次方程的方法来求解, 比较困难。
如果设两个未知数, 例如设甲数是 $x$,乙数是 $y$, 那么就可以列出下面两个二元一次方程:
\begin{gather}
    3x - 2y = 11 \douhao \\
    2x + 3y = 16 \juhao
\end{gather}

上面的问题就是要求出既适合方程 (1), 又适合方程 (2) 的 $x$ 与 $y$ 的值,也就是求出这两个方程的公共解。

把这两个方程变形,用含有 $x$ 的代数式表示 $y$ , 得
\begin{gather}
    y = \dfrac{3}{2}x - \dfrac{11}{2} \douhao \\
    y = \dfrac{16}{3} - \dfrac{2}{3}x \juhao
\end{gather}
从 (3) 可以求得方程 (1) 的一些解
$$
    \begin{cases}
        x = 0\douhao \\
        y = -\dfrac{11}{2}\douhao
    \end{cases}\quad
    \begin{cases}
        x = 1\douhao \\
        y = -4\douhao
    \end{cases}\quad
    \begin{cases}
        x = 5\douhao \\
        y = 2\douhao
    \end{cases}\quad
    \cdots
$$
从 (4) 可以求得方程 (2) 的一些解
$$
    \begin{cases}
        x = 3\douhao \\
        y = \dfrac{10}{3}\douhao
    \end{cases}\quad
    \begin{cases}
        x = 5\douhao \\
        y = 2\douhao
    \end{cases}\quad
    \begin{cases}
        x = 7\douhao \\
        y = \dfrac{2}{3}\douhao
    \end{cases}\quad
    \cdots
$$
可以看出,其中的
$$\begin{cases}
    x = 5\douhao \\
    y = 2
\end{cases}$$
即是方程 (1) 的一个解,又是方程 (2) 的一个解,所以它就是这两个方程的公共解。

上面所说的,可以用图 \ref{fig:5-1} 来表示。

\begin{figure}[htbp]
    \centering
    \begin{tikzpicture}
    \foreach \x/\y [count=\i] in {
        0/-\dfrac{11}{2},
        1/-4,
        5/2,
        3/\dfrac{10}{3},
        7/\dfrac{2}{3}
    } {
        \node at (2*\i, 0) {$\begin{cases}
                x = \x \\
                y = \y
            \end{cases}$};
    }
    \node at (2.8, -1) {……};
    \node at (8.8, -1) {……};

    \draw (3.5, 0) ellipse [x radius=3.5, y radius=1.2];
    \draw (8.2, 0) ellipse [x radius=3.5, y radius=1.2];
    \draw (1.8, -0.7) -- (0.5, -1.6) node [below] {方程(1)的解集};
    \draw (5.8, -0.7) -- (5.6, -1.6) node [below] {方程(1),(2) 的公共解集};
    \draw (9.8, -0.7) -- (10.5, -1.6) node [below] {方程(2)的解集};
\end{tikzpicture}

    \caption{}\label{fig:5-1}
\end{figure}

由几个方程组成的一组方程,叫做\zhongdian{方程组}。
由几个一次方程组成并含有两个未知数的方程组,叫做\zhongdian{二元一次方程组}。
例如,上面的方程 (1), (2) 合在一起,就组成一个二元一次方程组,记作
$$\begin{cases}
    3x - 2y = 11 \douhao \\
    2x + 3y = 16 \juhao
\end{cases}$$

本章中所说的二元一次方程组,都是指由两个一次方程组成的二元一次方程组。

方程组里各个方程的公共解,叫做这个\zhongdian{方程组的解}。
例如,上面的方程 (1), (2) 的公共解
$$\begin{cases}
    x = 5 \douhao \\
    y = 2
\end{cases}$$
就是方程组
$$\begin{cases}
    3x - 2y = 11 \douhao \\
    2x + 3y = 16
\end{cases}$$
的解。

\lianxi
\begin{xiaotis}

\xiaoti{(口答)下列方程组中,哪些是二元一次方程组,哪些不是。为什么?}
\begin{xiaoxiaotis}

    \begin{tblr}{columns={18em, colsep=0pt}}
        \xxt{$\begin{cases}
                x + 3y = 5, \\
                2x - 3y = 3;
              \end{cases}$}
            & \xxt{$\begin{cases}
                    x + 3y = 6, \\
                    x^2 - y^2 = 8;
                \end{cases}$} \\
        \xxt{$\begin{cases}
                x + 3y = 9, \\
                y + z = 7;
              \end{cases}$}
            & \xxt{$\begin{cases}
                    x + 3y = 5, \\
                    xy = 2;
                \end{cases}$} \\
        \xxt{$\begin{cases}
                x + 3y = 3, \\
                \dfrac{x}{6} + \dfrac{2y}{3} = 1;
              \end{cases}$}
            & \xxt{$\begin{cases}
                x + 3y = 2, \\
                \dfrac{6}{x} - 2y = 3 \juhao
            \end{cases}$}
    \end{tblr}

\end{xiaoxiaotis}

\xiaoti{(口答)在
    $\begin{cases}
        x = 1, \\
        y = -1,
    \end{cases}$ \quad
    $\begin{cases}
        x = 2, \\
        y = 1,
    \end{cases}$ \quad
    $\begin{cases}
        x = 4, \\
        y = 5
    \end{cases}$
    三对数值中,哪一对是下列方程组的解?
}
\begin{xiaoxiaotis}

    \twoInLineXxt[18em]{
        $\begin{cases}
            2x - y = 3 \douhao \\
            3x + 4y = 10 \fenhao
        \end{cases}$
    }{
        $\begin{cases}
            y = 2x - 3 \douhao \\
            4x - 3y = 1 \juhao
        \end{cases}$
    }

\end{xiaoxiaotis}

\xiaoti{根据已知条件,求出 $y$ 的值;分别填入下列各图的右圈里,并找出方程组
    $$\begin{cases}
        y = 3x \douhao \\
        y - 2x = 1
    \end{cases}$$
    的解。
}

\begin{figure}[htbp]
    \centering
    \begin{tikzpicture}[>=Stealth]
    \tikzset{
        pics/yingshe/.style n args={2}{
            code = {
                \draw [->] (0, 0) node {$#1$} (0.4, 0) --(1.7, 0);
                \draw (1.7, 0) -- (3, 0) node [right] {$#2$};
            }
        }
    }

    \begin{scope}
        \foreach \x/\y [count=\i] in {
            -2/?, -1/?, \hphantom{+}0/?,
            \hphantom{+}1/?, \hphantom{+}2/6,
            \hphantom{+}3/9
        } {
            \draw (0, 0.5*\i) pic {yingshe={\x}{\y}};
        }
        \draw (0, 4) pic {yingshe={x}{y}}  (1.75, 4.1) node[above] {$y = 3x$};
        \draw (0, 1.75) ellipse [x radius=0.8, y radius=2];
        \draw (3.2, 1.75) ellipse [x radius=0.8, y radius=2];
        \draw (1.75, -.8) node {(1)};
    \end{scope}

    \begin{scope}[xshift=7cm]
        \foreach \x/\y [count=\i] in {
            -2/?, -1/?, \hphantom{+}0/1,
            \hphantom{+}1/?, \hphantom{+}2/?,
            \hphantom{+}3/7
        } {
            \draw (0, 0.5*\i) pic {yingshe={\x}{\y}};
        }
        \draw (0, 4) pic {yingshe={x}{y}}  (1.75, 4.1) node[above] {$y - 2x = 1$};
        \draw (0, 1.75) ellipse [x radius=0.8, y radius=2];
        \draw (3.2, 1.75) ellipse [x radius=0.8, y radius=2];
        \draw (1.75, -.8) node {(2)};
    \end{scope}
\end{tikzpicture}

    \caption*{(第 3 题)}
\end{figure}

\end{xiaotis}


