\xiaojie

一、溶液

凡是用来溶解其它物质的物质,叫做溶剂。

凡是被溶剂所溶解的物质,叫做溶质。

溶液是一种或一种以上的溶质分散在溶剂里,形成的均一的、稳定的混和物。

溶质溶解于水时,一般有吸热现象和放热现象,这说明溶液的形成过程中,既包括物理过程,又包括化学过程。


二、溶解度

1. 饱和溶液和不饱和溶液 \quad 在一定温度下,在一定量的溶剂里不能再溶解某种溶质的溶液,
叫做饱和溶液;还能继续溶解某种溶质的溶液,叫做不饱和溶液。

2. 固体的溶解度 \quad 在一定温度下,某物质在 100 克溶剂里达到饱和状态时所溶解的克数,
叫做这种物质在这种溶剂里的溶解度。

3. 气体的溶解度 \quad 在一定温度下,某气体(其压强为 1 标准大气压)在 1 体积溶剂里达到
饱和状态时所溶解的体积数(换算成标准状况时的体积数), 叫做这种气体在这种溶剂里的溶解度。


三、物质的结晶

1. 结晶跟溶解是相反的两个过程。
\begin{fangchengshi}
    \ce{ \text{固体溶质} <=>[\text{溶解}][\text{结晶}] \text{溶液里的溶质} }
\end{fangchengshi}

对溶解度随温度升高而增加的溶质来说,把饱和溶液的温度降低或蒸发溶剂都可使溶质从溶液里结晶出来。


2. 结晶水和结晶水合物 \quad 如物质在形成晶体时,晶体中结合一定数目的水分子,
这样的水分子叫做结晶水。含有结晶水的物质叫做结晶水合物。


四、混和物的分离

混和物分离的方法很多,常用的有过滤、结晶和蒸馏等方法。


五、溶液的浓度

一定量的溶液里所含溶质的量叫做溶液的浓度。表示溶液浓度的方法很多,
初中化学课里讲的主要是质量百分比浓度(简称百分比浓度)。
\begin{gather*}
    \text{百分比浓度} = \dfrac{\text{溶质质量}}{\text{溶质质量} + \text{溶剂质量}} \times 100\%
\end{gather*}


