\section{悬浊液 乳浊液 溶液}\label{sec:4-1}

什么是悬浊液、乳浊液?什么是溶液?它们具有哪些特点?

\begin{shiyan}
    在 4 个试管里,各加入 10 毫升水,然后分别加入泥土、植物油、蔗糖和食盐,振荡,观察发生的现象。
\end{shiyan}

泥土放进试管的水里,振荡以后,得到浑浊的液体。
液体里悬浮着由很多分子集合成的固体小颗粒,使整个液体呈浑浊状态。
这种液体不稳定。由于小颗粒比水重,静置一会儿后,它们就逐渐下沉。
这种固体小颗粒悬浮于液体里形成的混和物叫做\zhongdian{悬浊液}(或悬浮液)。

植物油注入试管的水里, 用力振荡以后,得到乳状浑浊的液体。
这种液体里分散着不溶于水的小液滴,小液滴也是由很多分子集合成的。
这种液体也不隐定,经过静置,由于一般的小液滴比水轻,所以就逐渐浮起来,分为上下两层。
这种小液滴分散到液体里形成的混和物叫做\zhongdian{乳浊液}(或乳状液) 。

悬浊液和乳浊液有广泛的应用。
为了合理使用农药,常把不溶于水的固体农药或液体农药,配制成
可湿性粉剂\footnote{可湿性粉剂是由农药原药、润湿剂和填料混和,经粉碎后制得。如西维因可制成可湿性粉剂。}.
或乳油\footnote{乳油是由农药原药、溶剂和乳化剂混和制成。如敌敌畏可制成乳油。}。
在使用时,把可湿性粉剂或乳油跟水以一定比配制成悬浊液或乳浊液,用来喷洒受病虫害的农作物。
这样农药的药液散失的少,附着在叶面的多,药液喷洒均匀,不仅使用方便,而且节省农药,提高药效。

蔗糖或食盐溶解在水里,得到的液体与悬浊液、乳浊液不同。
蔗糖的分子均一地分散到水分子中间,组成食盐的小微粒(钠离子和氯离子)也均一地分散到水分子中间。
这两种液体不但是均一的,而且是透明的。
只要水不蒸发,温度不变化,不管放置多久,蔗糖或食盐不会分离出来,
也就是说,这两种液体是很稳定的。象这样%
\zhongdian{一种或一种以上的物质分散到另一种物质里,形成均一的、稳定的混和物,叫做溶液。}
我们把\zhongdian{能溶解其它物质的物质叫做溶剂;被溶解的物质叫做溶质。}
溶液是由溶剂和溶质组成的。水能溶解很多种物质,是最常用的溶剂。
汽油、酒精等等也可以做溶剂。例如,汽油能溶解油脂,酒精能溶解碘等等。

溶质可以是固体,也可以是液体或气体。固体、气体溶于液体时,固体、气体是溶质,液体是溶剂。
两种液体互相溶解时,通常把量多的一种叫做溶剂,量少的一种叫做溶质。
用水做溶剂的溶液,叫做水溶液。用酒精做溶剂的溶液叫做酒精溶液。
通常不指明溶剂的溶液,一般指的是水溶液。

溶质和溶剂是相对而言的。例如,酒精和水互相溶解时,一般来说酒精是溶质,水是溶剂;
如果少量水溶解在酒精里,就可以把水作为溶质,酒精作为溶剂。

把两种能够起反应的固物质混和在一起,反应进行得慢;
如果把这两种物质分别配制成溶液,然后把两种溶液混和,反应就进行得快。
所以,在实验室里或化工生产中,要使两种能起反应的固体起反应,常常先把它们溶解,
然后把两种溶液混和,并加振荡或搅动,以加快反应的进行。

溶液对动植物的生理活动也有很大意义。
动物摄取食物里的养料,必须经过消化,变成溶液,才能吸收。
植物从土壤里获得各种养料,也要成为溶液,才能由根部吸收。
土壤里含有水分,里面溶解了多种物质,形成土壤溶液,土壤溶液里就含有植物需要的养料。
许多肥料,象人粪尿、牛马粪、农作物秸秆、野草等等,在施用以前都要经过腐熟的过程,
目的之一是使复杂的难溶的有机物变成简单的易溶的物质,这些物质能溶解在土壤溶液里,供农作物吸收。

我们在生产和科学实验里,常用的是水溶液,因此,本章主要介绍水溶液的知识。


\begin{xiti}

\xiaoti{把少量的下列各种物质分别放入水里,振荡,进行仔细观察,
    并指出哪个是溶液,哪个是悬浊液,哪个是乳浊液。
}
\begin{xiaoxiaotis}

    \xxt{精盐,}  \xxt{面粉,}  \xxt{煤油,}  \xxt{纯碱。}
\end{xiaoxiaotis}


\xiaoti{分别指出下列各种溶液里的溶质和溶剂。}
\begin{xiaoxiaotis}

    \xxt{碘酒,}  \xxt{糖水,}  \xxt{酒精的水溶液,}  \xxt{氢氧化钠溶液。}
\end{xiaoxiaotis}


\xiaoti{选择正确的答填写在括号里。}
\begin{xiaoxiaotis}

    \xxt{当条件不改变时, 溶液放置时间稍长, 溶质 \ewkh。\\
        \tc{1} 会沉降出来,  \tc{2} 不会分离出来, \tc{3} 会浮上来。
    }

    \xxt{一杯溶液里各部分的性质是 \ewkh 。\\
        \tc{1} 不相同的, \tc{2} 相同的, \tc{3} 上面跟下面不相同。
    }

\end{xiaoxiaotis}


\xiaoti{为什么实验室和化工生产上的许多化学反应都在溶液里进行?}

\end{xiti}


