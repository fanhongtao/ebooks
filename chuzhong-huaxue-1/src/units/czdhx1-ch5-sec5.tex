\section{常见的碱 碱的通性}\label{sec:5-5}

现在我们来学习两种常见的、重要的碱。

\subsection{氢氧化钠(\ce{NaOH})}

\begin{shiyan}
    用镊子取出一小块氢氧化钠放在表面皿上,观察它的状态、颜色,放置一会儿,再观察它的状态发生什么变化。
    把一小块氢氧化钠溶解在有少量水的试管里,注意溶液的温度有没有变化。
\end{shiyan}

纯净的氢氧化钠是一种白色固体,极易溶解于水,溶解时放出大量的热。它的水溶液有涩味和滑腻感。
暴露在空气里的氢氧化钠容易吸收水分而潮解。因此,氢氧化钠可用作某些气体的干燥剂。
由于氢氧化钠有强烈的腐蚀性,因此它又叫苛性钠、火碱或烧碱。
\zhongdian{在使用氢氧化钠时必须十分小心,防止皮肤、衣服被它腐蚀。}

下面简单介绍氢氧化钠的一些化学性质。

\subsubsection{氢氧化钠跟酸碱指示剂的反应}

氢氧化钠溶液能够使紫色的石蕊试液变成蓝色,使无色的酚酞试液变成红色。


\subsubsection{氢氧化钠跟非金属氧化物的反应}

氢氧化钠能跟二氧化碳、二氧化硫等非金属氧化物起反应。
\begin{fangchengshi}
    \ce{ 2NaOH + CO2 = \underset{\text{碳酸钠}}{\ce{Na2CO3}} + H2O } \\
    \ce{ 2NaOH + SO2 = \underset{\text{亚硫酸钠}}{\ce{Na2SO3}} + H2O }
\end{fangchengshi}

由于氢氧化钠在空气里不仅吸收水分,还能跟二氧化碳起反应,所以氢氧化钠必须密封保存。

\subsubsection{氢氧化钠跟酸的反应}

氢氧化钠不仅跟盐酸起中和反应,而且跟硫酸、硝酸等其它的酸也起类似的反应。
\begin{fangchengshi}
    \ce{ 2NaOH + H2SO4 = \underset{\text{硫酸钠}}{\ce{Na2SO4}} + 2H2O } \\
    \ce{ NaOH + HNO3 = \underset{\text{硝酸钠}}{\ce{NaNO3}} + H2O }
\end{fangchengshi}

\subsubsection{氢氧化钠跟某些盐的反应}

\begin{shiyan}
    在两个试管中分别注入少量硫酸铜溶液和氯化铁溶液,然后各加几滴氢氧化钠溶液,观察发生的现象。
\end{shiyan}

从实验可以看出,第一个试管里生成蓝色氢氧化铜沉淀,第二个试管里生成红褐色氢氧化铁沉淀。
\begin{fangchengshi}
    \ce{ CuSO4 + 2NaOH = \underset{\text{氢氧化铜}}{\ce{Cu(OH)2}} v + Na2SO4 } \\
    \ce{ FeCl3 + 3NaOH = \underset{\text{氢氧化铁}}{\ce{Fe(OH)3}} v + 3NaCl }
\end{fangchengshi}

氢氧化钠是一种重要的化工原料,广泛应用于肥皂、石油、造纸、纺织、印染等工业上。


\subsection{氢氧化钙〔\ce{Ca(OH)2}〕}

\begin{shiyan}
    在蒸发皿中放一小块生石灰,加少量水,观察有什么现象发生。
\end{shiyan}

生石灰(\ce{CaO})跟水起反应,生成氢氧化钙(俗称熟石灰或消石灰),同时放出大量的热。氢氧化钙是白色粉末状物质。
\begin{fangchengshi}
    \ce{ CaO + H2O = Ca(OH)2 }
\end{fangchengshi}

\begin{shiyan}
    取一药匙氢氧化钙放入一小烧杯里,加水约 30 毫升,用玻璃棒搅拌,观察它在水中溶解的情况。
    放置澄清,上层清液就是氢氧化钙的水溶液。
\end{shiyan}

氢氧化钙微溶于水,它的溶液俗称石灰水。氢氧化钙对皮肤、衣服等有腐蚀作用。

\begin{shiyan}
    在盛有石灰水的两个试管里,分别滴入几滴石蕊试液和酚酞试液,观察颜色的变化。
\end{shiyan}

石灰水使紫色的石蕊试液变成蓝色,使无色的酚酞试液变成红色。

我们已经知道,石灰水中通入二氧化碳后,它会变浑浊,这是由于生成了不溶于水的碳酸钙的缘故。
\begin{fangchengshi}
    \ce{ Ca(OH)2 + CO2 = CaCO3 v + H2O }
\end{fangchengshi}

氢氧化钙可以跟酸起中和反应,在农业上常用熟石灰来改良酸性土壤。

氢氧化钙也能跟某些盐起反应,例如,能跟碳酸钠起反应。

\begin{shiyan}
    在盛有石灰水的试管里,注入浓的碳酸钠溶液,观察有什么现象发生。
\end{shiyan}

生成的白色沉淀是碳酸钙。
\begin{fangchengshi}
    \ce{ Ca(OH)2 + Na2CO3 = CaCO3 v + 2NaOH }
\end{fangchengshi}

这个反应可以用来制造氢氧化钠。

熟石灰在工农业生产上的应用很广。建筑业上用熟石灰、粘土和沙子混和制成三合土,
或用石灰沙浆来砌砖、抹墙,就是利用熟石灰能吸收空气中二氧化碳变成坚固的碳酸钙这一性质。
工业上还用熟石灰作原料来制造漂白粉、氢氧化钠。
农业上用它来降低土壤的酸性,改进土壤的结构,还用它来配制农药波尔多液和石硫合剂。

除以上两种碱外,常见的、重要的碱还有氢氧化钾、氨水等,以后将会学到。


\subsection{碱的命名}

碱的命名是根据氢氧根离子和金属离子的名称,叫做 “氢氧化\;某”,
如 \ce{Mg(OH)2} 叫做氢氧化镁, \ce{Cu(OH)2} 叫做氢氧化铜等等。
如果某种金属具有可变化合价,可以形成带不同电荷的离子时,那么,
把具有高价金属离子的碱叫做 “氢氧化某”,把具有低价金属离子的碱叫做 “氢氧化亚某”。
例 \ce{Fe(OH)3} 叫做氢氧化铁, \ce{Fe(OH)2} 叫做氢氧化亚铁。


\subsection{碱的通性}

由于碱类在水溶液中都能电离,生成氢氧根离子,所以,它们有一些相似的性质。

1. 碱溶液能跟酸碱指示剂起反应,例如,
紫色的石蕊试液遇碱变蓝色,无色的酚酞试液遇碱变红色。
碱的溶液有涩味,皮肤上沾了碱的溶液有滑腻感。

2. 碱能跟多数非金属氧化物起反应,生成盐和水。

3. 碱能跟酸起中和反应,生成盐和水。

4. 碱能跟某些盐起反应,生成另一种盐和另一种碱。


\begin{xiti}

\xiaoti{以氢氧化钾为例,说明碱有哪些通性,写出有关反应的化学方程式。}

\xiaoti{生石灰是用石灰石经煅烧而制得的。1 吨含 $12\%$ 杂质的石灰石最多能制得多少生石灰?怎样检验生石灰中是否含有石灰石?}

\xiaoti{如果利用生石灰制少量氢氧化钠,还要哪些原料,经过怎样操作? 写出反应的化学方程式。}

\xiaoti{有三瓶无色溶液,已知它们分别为石灰水、氢氧化钠溶液和稀硫酸,怎样用化学方法来鉴别它们?}

\end{xiti}

