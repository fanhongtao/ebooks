\section{化学肥料}\label{sec:5-7}

在酸、碱、盐各类化合物,特别是盐类中,有好多种含有农作物所需的营养元素,因而在农业上被广泛用作化学肥料。
化学肥料简称化肥,是用矿物、空气、水等作原料,经过化学加工制成的。

农作物的生长需要碳、氢、氧、氮、磷、钾、钙、镁、硫、铁等等元素。土壤里常缺乏的是氮、磷、钾三种元素。
因此,要施用含氮、磷、钾元素的肥料。化学肥料的种类很多,主要是氮肥、磷肥和钾肥。


\subsection{氮肥}

氮是作物体内蛋白质、核酸和叶绿素的重要成分,氮肥能促使作物的茎、叶生长茂盛,叶色浓绿。
下面介绍目前农村常用的几种氮肥。

\subsubsection{氨水}

氨水是氨(\ce{NH3})的水溶液,氨分子在水中主要以水合物(\ce{NH3 . H2O})形式存在,
部分电离生成 \ce{NH4+}(铵根离子)和 \ce{OH-}, 所以,氨水显碱性。
\begin{fangchengshi}
    \ce{ NH3 + H2O = NH3 . H2O = NH4+ + OH- }
\end{fangchengshi}

纯净氨水是无色液体,工业制品因含杂质而呈浅黄色。
氨水的浓度一般为 $20\%$ 左右,折合成含氮最大约为 15—17\%。

氨水易分解、挥发,放出氨气。氨气是一种有刺激性气味的气体。氨水在浓度大、温度高时,分解、挥发更快。
因此,在运输、贮存、施用这三个环节上,都要注意防止氨气的挥发,以减少肥分的损失。
为了减少氨的挥发,可把氨水密封在容器里,放在阴凉的地方,也可在氨水的表面覆盖一层矿物油。
氨水对多种金属有腐蚀作用,因此运输和贮存氨水时,一般要用橡皮袋、陶瓷坛或内涂沥青的铁桶等耐腐蚀的容器。

氨水是一种速效肥料,施入土壤后,作物能很快吸收,不会残留有害物质,不会影响土壤的结构和性质。
因为浓氨水易挥发并能“烧伤”作物和人的皮肤,刺激人的眼、鼻和喉粘膜,施用时必须稀释,
深施到土层中,再用土盖上,或结合灌溉使氨水随水流入田地里。


\subsubsection{用作氮肥的铵盐\footnotemark}
\footnotetext{含有铵根离子(\ce{NH4+})的盐叫做铵盐。铵根离子的性质跟金属离子相似,通常把它归入金属离子之内。}

氨水是液体,运输不便,又容易挥发而降低肥效。生产上常利用酸跟氨起反应,制成固体铵盐,用作肥料。例如:
\begin{fangchengshi}
    \begin{aligned}
        \ce{ NH3 + H2O + CO2 } &= \ce{ NH4HCO3 } \\[-.5em]
        \ce{ NH3 + HNO3 }      &= \ce{ NH4NO3 } \\[-.5em]
        \ce{ 2NH3 + H2SO4 }    &= \ce{ (NH4)2SO4 } \\[-.5em]
        \ce{ NH3 + HCl }       &= \ce{ NH4Cl }
    \end{aligned}
\end{fangchengshi}

这四种铵盐都是白色晶体,易溶于水。含氮量各不相同,
碳酸氢铵(简称碳铵)约为 $17\%$,
硝酸铵(简称硝铵)约为 $35\%$,
硫酸铵(简称硫铵)约为 $21\%$,
氯化铵约为 $25\%$。

碳铵性质很不稳定,受潮时在常温下就能分解,温度越高,分解越快。
\begin{fangchengshi}
    \ce{ NH4HCO3 $\xlongequal{\Delta}$ NH3 ^ + CO2 ^ + H2O }
\end{fangchengshi}

为了避免碳铵分解,贮存和运输时都要密封,不要受潮或曝晒,也不要存放过久。
施肥后要立即盖土,或立即灌溉,以确保肥效。

碳铵施入土壤后,它的养料成分能全部被作物吸收利用,在土壤里不残留有害物质。

硝铵含氮量高,肥效大,施入土壤后,铵根离子和硝酸根离子都能被作物吸收,对土壤没有不良影响。

硝铵受热易分解。在高温或受到猛烈撞击时,迅速分解,放出大量气体而发生爆炸。
在混有可燃物如木屑、煤粉、棉花、油料等时,爆炸更为剧烈。
因此,必须注意,硝铵不能和易燃物质存放在一起。

硝铵受潮易结块,粉碎时不要用铁锤砸,最好用木棍碾碎,以免发生爆炸事故。

硫铵和氯化铵的吸湿性比较小,常温也很稳定。

如果长期施用硫铵,会使土壤酸性增加,板结硬化。

\begin{shiyan}
    在四块玻璃片上,分别放少量碳酸氢铵、硝酸铵、硫酸铵、氯化铵,各加入一些熟石灰,用玻璃棒拌和,能闻到什么气味。
\end{shiyan}

铵盐跟碱起反应,能放出刺鼻的氨气。

以氯化铵为例,反应的化学方程式如下:
\begin{fangchengshi}
    \ce{ 2NH4Cl + Ca(OH)2 = CaCl2 + 2H2O + 2NH3 ^ }
\end{fangchengshi}

因此,凡是含有 \ce{NH4+} 的氮肥,在贮存和施用时,都不要跟石灰、草木灰等碱性物质混和,否则,会降低肥效。


\subsubsection{尿素}

工业上用氨和二氧化碳在 200 标准大气压和 180 ℃ 下合成尿素〔\ce{CO(NH2)2}〕。
\begin{fangchengshi}
    \ce{ 2NH3 + CO2 $\xlongequal[\Delta]{\text{高压}}$ CO(NH2)2 + H2O }
\end{fangchengshi}

尿素是白色或淡黄色的粒状晶体,略有吸湿性,易溶于水,含氮量约为 $46\%$,是含氮量很高的一种氮肥。
尿素施入土壤后,受微生物的作用,跟水缓慢起反应生成碳酸铵,碳酸铵容易被作物吸收。
\begin{fangchengshi}
    \ce{ CO(NH2)2 + 2H2O = (NH4)2CO3 }
\end{fangchengshi}

因此,尿素的肥效较铵盐氮肥缓慢一些,但比较持久。尿素的肥效高,对土壤没有不良影响,是一种优良的氮肥。



\subsection{磷肥}

磷肥能促进作物根系发达,增强抗寒抗旱能力,还能促进作物提早成熟,穗粒增多,子粒饱满。

常用的化学磷肥的成分都是磷酸盐,制造磷肥的主要原料是磷矿石。
磷矿石的主要成分是磷酸钙〔\ce{Ca3(PO4)2}〕,它是难溶于水的矿物。
化学工业上制造磷肥的目的,就是加工磷矿石,使它转化为较易溶解于水(或弱酸)的磷酸盐,使植物易于吸收。

最简单的加工方法是把磷矿石磨碎成磷矿粉,直接施用。
由于土壤里含有的多种酸的作用,磷矿粉逐渐溶解而能被作物吸收,但非常缓慢。

把磷矿石和焦炭以及含钙、镁、硅的其他矿石一起在高炉中煅烧、熔融,可制得钙镁磷肥。
钙镁磷肥虽仍难溶于水,但较磷矿粉易溶于弱酸性溶液中,肥效有所提高。

把磷矿粉跟硫酸起反应,可制得过磷酸钙(简称普钙),它是一种常用的磷肥。
\begin{fangchengshi}
    \ce{ Ca3(PO4)2 + 2H2SO4 = $\underbrace{\ce{ Ca(H2PO4)2 + 2CaSO4 }}_{\text{过磷酸钙}}$ }
\end{fangchengshi}

过磷酸钙是磷酸二氢钙和硫酸钙的混和物,磷酸二氢钙能溶于水,肥效比磷矿粉或钙镁磷肥有较大提高。

如果用磷酸代替硫酸跟磷矿粉起反应可以制得重过磷酸钙(简称重钙)。
\begin{fangchengshi}
    \ce{ Ca3(PO4)2 + 4H3PO4 = 3Ca(H2PO4)2 }
\end{fangchengshi}

跟普钙不同,重过磷酸钙的成分是磷酸二氢钙,不含硫酸钙,所以肥效比普钙高。

磷酸二氢钙是可溶性酸式盐,它跟石灰性土壤里的碱起反应生成难溶性的磷酸钙,
又能跟酸性土壤里的铁离子、铝离子起反应生成难溶性的磷酸铁和磷酸铝,降低了肥效。
普钙和重钙最好跟农家肥料混和后施用,既减少磷肥跟土壤的直接接触,
又可因农家肥料里的酸性物质减弱磷肥变为不溶性磷酸盐的倾向。



\subsection{钾肥}

钾肥能促使作物生长健壮,茎秆粗硬,增强对病虫害和倒伏的抵抗能力,并能促进糖分和淀粉的生成。

目前农村多用草木灰作为钾肥。草木灰的主要成分是碳酸钾(\ce{K2CO3})和少量钙、镁、磷的化合物。
由于碳酸钾易溶于水,所以,堆放草木灰时要防止雨淋,以免肥分流失。
草木灰有碱性,不要跟铵态氮肥(成分是铵盐)混用,以免氨气挥发,降低肥效。

常用的钾肥还有硫酸钾(\ce{K2SO4})和氯化钾(\ce{KCl})。
这两种钾肥都是白色的晶体(含有杂质时呈淡黄色),都容易溶解于水,含钾量都很高。

跟 \ce{(NH4)2SO4} 相似,多施 \ce{K2SO4} 也会使土壤酸度增加,并使土壤板结。

植物生长不仅需要大的氮、磷、钾等营养元素,还需要微量的其他元素,例如,硼、锌、铜、锰、钼等。
含有这些元素的肥料,叫做微量元素肥料。植物缺乏这些微量元素,就会影响生长发育,减弱抗病能力。

上面简单地介绍了一些常见的化肥的品种。为了促进农业的增产丰收,必须发展化肥工业。
除了增产一般品种外,还要注意发展高效的化肥(有效肥分高的,如尿素)和
复合肥料(含有两种或两种以上营养元素的化肥,如磷酸铵、硝酸钾),等等。

除化学肥料外,我国农村还大量使用农家肥料(如厩肥、绿肥等)。
跟农家肥料相比,化学肥料具有营养元素含量大,一般易溶于水,易于被作物吸收,
肥效较快,便于工业生产等优点,但它们一般只含一种营养元素;
而农家肥料常含多种营养元素,肥效较长,便于就地取材,成本低廉,又能改良土壤结构,
提高土壤肥力,但它们的营养元素含量较小,肥效一般较慢。
因此,这两种肥料要结合施用。


\begin{xiti}

\xiaoti{现有含氨 $15\%$ 的氨水 20 千克,要用水稀释到含氨 $0.3\%$ 施用,需加水多少千克?}

\xiaoti{生产 200 吨硝酸铵,需要多少吨氨和多少吨浓度为 $70\%$ 的硝酸?}

\xiaoti{有一不纯的硫铵样品,经分析知道它含有 $20\%$ 的氮,求样品里含 \ce{(NH4)2SO4} 的百分率。}

\xiaoti{在贮存、运输和施用氨水、碳酸氢铵、过磷酸钙、草木灰等肥料时,应该怎样防止肥分的损失?}

\xiaoti{现有四小包粉状化肥,已知它们是 \ce{K2SO4}、\ce{NH4Cl}、\ce{K2CO3} 和 \ce{Ca3(PO4)2},用什么方法来鉴别它们?}

\end{xiti}

