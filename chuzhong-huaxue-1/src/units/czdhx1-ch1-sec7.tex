\section{分子式 分子量}\label{sec:1-7}

\subsection{分子式}

我们已经知道,元素是用元素符号来表示的。那么,由元素组成的各种单质和化合物用什么符号来表示呢?
人们在长期的生产实践和科学实验里,认识到各种纯净物都有一定的组成。
为了便于认识和研究物质,化学上常用元素符号来表示物质的组成。
例如,氧气分子、氢气分子和水分子的组成,可以分别用 \ce{O2}、\ce{H2}、\ce{H2O} 来表示。
这种\zhongdian{用元素符号来表示物质分子组成的式子叫做分子式。}
各种物质的分子式,是通过实验的方法,测定了物质的组成,然后得出来的。
一种物质只有一个分子式。

1. 单质分子式的写法 \quad 单质是由同种元素组成的。
写单质分子式时,首先写出组成单质的元素符号,然后在元素符号的右下角,写一个小数字,
来表示这种单质的一个分子里所含原子的数目(原子数是 $1$ 时不写上)。
例如,氧气、氮气、氢气的每一个分子里都含有两个原子,
所以这些单质的分子式分别写成 \ce{O2}、\ce{N2}、\ce{H2}。
氦气等惰性气体是由单原子组成的,通常就用元素符号表示它们的分子式。
例如,氦气、氖气分别写成 \ce{He}、\ce{Ne}。

金属单质和固态非金属单质的结构比较复杂,习惯上就用元素符号来表示它们的分子式。
例如,铁用 \ce{Fe} 表示, 磷用 \ce{P} 表示。

2. 化合物分子式的写法 \quad 化合物是由不同种元素组成的。
写化合物分子式时,必须知道这种化合物是由哪几种元素组成的,
以及这种化合物的一个分子里,每种元素各有多少个原子。
知道这些事实后,就可以先写出元素符号,然后在每种元素符号的右下角写个小数字,
以标明这种化合物的一个分子里所含该元素的原子个数。

由氧元素跟另一元素组成的化合物,书写分子式时,
一般要把氧元素符号写在右方,另一种元素符号写在左方。
例如,二氧化碳的分子式写成 \ce{CO2}, 氧化汞的分子式写成 \ce{HgO}。

由金属元素跟非金属元素组成的化合物,书写分子式时,
一般要把金属元素符号写在左方,非金属元素符号写在右方。
例如,硫化锌的分子式写成 \ce{ZnS}。

分子式用来表示物质的一个分子,如果要表示物质的几个分子,
可以在分子式前面加上系数,标明该物质的分子数。
例如,要表示两个氢分子,就写成 \ce{2H2},要表示五个水分子,就写成 \ce{5H2O}。

书写分子式时应该注意,元素符号右下角的数字和元素符号前面的数字在意义上是完全不同的。
例如,\ce{O2} 表示一个氧分子由两个氧原子组成;\ce{2O} 表示两个单个的氧原子;\ce{3O2} 表示三个氧分子。

由两种元素组成的化合物的名称,一般是从右向左读作 “某化某” 。
例如,\ce{NaCl} 读作氯化钠(俗名食盐);
有时还要读出化合物每一个分子里元素的原子个数。
例如,\ce{SO2} 读作二氧化硫, \ce{Fe3O4} 读作四氧化三铁等等。



\subsection{分子量}

\zhongdian{一个分子中各原子的原子量的总和就是分子量。}

1. 根据分子式,可以计算物质的分子量。例如:

氧气的分子式是 \ce{O2}, 那么氧气的分子量就是两个氧原子的原子量之和,
即 $\ce{O2} \text{的分子量} =  16 \times 2 = 32$。

二氧化碳的分子式是 \ce{CO2}, 那么二氧化碳的分子量就是一个碳原子的原子量和两个氧原子的原子量之和,
即 $\ce{CO2} \text{的分子量} = 12 + 16 \times 2 = 44$。

2. 根据分子式,可以计算组成物质的各元素的质量比。例如:

水的分子式是 \ce{H2O}, 那么组成水分子的氢元素和氧元素的质量比是 $1 \times 2 : 16 = 1:8$。

3. 根据分子式,还可以计算物质中某一元素的百分含量。例如:

计算化肥碳酸氢铵(\ce{NH4HCO3})中氮元素的百分含量。先根据分子式计算出分子量:

$\ce{NH4HCO3} \text{的分子量} = 14 + 1 \times 4 + 1 + 12 + 16 \times 3 = 79$, 再算出氮元素的百分含量:

$\dfrac{\ce{N}}{\ce{NH4HCO3}} \times 100\% = \dfrac{14}{79} \times 100\% \approx 17.7\%$


\begin{xiti}

\xiaoti{下列符号各表示什么意义?}
\begin{xiaoxiaotis}

    \fourInLineXxt[6em]{\ce{H},}{\ce{2H},}{\ce{H2},}{\ce{2H2}。}

\end{xiaoxiaotis}


\xiaoti{用分子式表示:}
\begin{xiaoxiaotis}

    \xxt{$4$ 个四氧化三铁分子,}

    \xxt{$3$ 个二氧化碳分子,}

    \xxt{$2$ 个氮分子。}

\end{xiaoxiaotis}


\xiaoti{在下列式子里各物质名称下面,写出这种物质的分子式:}
\begin{xiaoxiaotis}

    \xxt{\ce{ \text{氧气} + \text{硫} -> \text{二氧化硫} }}

    \xxt{\ce{ \text{氧化汞} -> \text{汞} + \text{氧气} }}

    \xxt{\ce{ \text{氧气} + \text{磷} -> \text{五氧化二磷} }}

    \xxt{\ce{ \text{水} -> \text{氢气} + \text{氧气} }}

\end{xiaoxiaotis}


\xiaoti{计算下列物质的分子量:}
\begin{xiaoxiaotis}

    \xxt{氮气(\ce{N2}),}

    \xxt{氯化钠(\ce{NaCl}),}

    \xxt{氯酸钾(\ce{KClO3}),}

    \xxt{硫酸(\ce{H2SO4}),}

    \xxt{氢氧化钙〔\ce{Ca(OH)2}〕。}

\end{xiaoxiaotis}


\xiaoti{计算化肥尿素〔\ce{CO(NH2)2}〕中氮元素的百分含量。}

\xiaoti{$1$ 千克硝酸铵(\ce{NH4NO3})跟多少千克碳酸氢铵(\ce{NH4HCO3})所含的肥效(指氮元素的含量)相当?}

\xiaoti{填空:}
\begin{xiaoxiaotis}

    \xxt{氧气的分子量是 \xhx, 氢气的分子量是 \xhx 。}

    \xxt{氧气和氢气的分子量之比为 \xhx[6em] 。}

    \xxt{相同数目的氧气分子和氢气分子的质量之比为 \xhx[6em]。}

    \xxt{实验测得相同温度、相同压强下,相同体积的氧气和氢气里所含的分子数相同。
        那么,相同温度、相同压强下,相同体积的氧气和氢气的质量之比为 \xhx[6em],
        等于它们的 \xhx[6em] 之比。
    }

\end{xiaoxiaotis}

\end{xiti}

