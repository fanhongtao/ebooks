\section{碳酸钙}\label{sec:3-6}

我们已经知道,二氧化碳跟石灰水起反应后生成碳酸钙。
碳酸钙是一种不溶于水的白色固体。碳酸钙遇到盐酸会起反应放出二氧化碳。
这是不是碳酸钙单独具有的性质呢?

\begin{shiyan}
    在分别盛有碳酸钾和碳酸钠的两个试管里,各加入少量盐酸,仔细观察有什么现象发生?
    用燃着的火柴放在管口试验,又有什么现象发生?
\end{shiyan}

生成大量的气泡是二氧化碳气体,能使燃着的火柴熄灭。
凡金属原子跟碳酸根的化合物,象碳酸钠、碳酸钾(\ce{K2CO3})、碳酸镁(\ce{MgCO3}) 等等,
都能跟盐酸起同样的反应。这是鉴定碳酸根最简便的方法。

碳酸钙还有一种性质是当遇到溶有二氧化碳的水,就会变成可溶性的碳酸氢钙〔\ce{Ca(HCO3)2}〕。
这个反应的化学方程式可表示如下:
\begin{fangchengshi}
    \ce{ CaCO3 + CO2 + H2O = Ca(HCO3)2 }
\end{fangchengshi}

这种含有碳酸氢钙的溶液如果受热;会发生分解,放出二氧化碳。
溶解了的碳酸氢钙又变成碳酸钙而沉积下来。
\begin{fangchengshi}
    \ce{ Ca(HCO3)2 \xlongequal{\Delta} CaCO3 v + H2O + CO2 ^ }
\end{fangchengshi}

碳酸钙在水壶或锅炉里能形成锅垢,就是这个缘故。

自然界里也不断发生着上述的反应。
石灰岩岩洞里的钟乳石和石笋等,就是含有碳酸钙的岩石经过上述作用而形成的。
这些悬挂的钟乳石、挺拔矗立的石笋和石柱,争奇斗丽,景象万千
(见封里彩图:广西桂林地区岩洞内的钟乳石、石笋和石柱)。

碳酸钙在自然界里分布很广。矿物里的大理石、石灰石、
白垩\footnote{垩\,音\,\pinyin{e4}。}等等
的主要成分都是碳酸钙,它们有很重要的用途。

大理石质地致密,绚丽多彩,加工琢磨以后可以用做建筑材料和装饰品。
我国很多宏伟的建筑物,象北京的人民大会堂和天坛等等都用了不少大理石做建筑材料。
白垩可用做白色涂料。

石灰石是建筑上常用的石料。粉碎后的石灰石和粘土按适当比混和,再加强热,就制得水泥。

工业上,把石灰石放入石灰窑内,经过高温煅烧,可制得生石灰。


\begin{xiti}

\xiaoti{碳酸钠和碳酸氢钠都能跟硫酸起反应生二氧化碳。根据这两个反应的化学方程式判断:}
\begin{xiaoxiaotis}

    \xxt{当硫酸的用量相同时(反应后都生成硫酸钠),哪一个反应产生的二氧化碳气体多?}

    \xxt{在通常使用的泡沫灭火器里(硫酸的量一定),是用碳酸钠好,还是用碳酸氢钠好?为什么?}

\end{xiaoxiaotis}


\xiaoti{有一种钙的化合物,为白色固体,它在水中不溶解,但溶于稀盐酸,
    并产生没有颜色的气体。把这种气体通入澄清的石灰水里时,溶液呈现白色浑浊。
    根据以上现象,指出可能是什么化合物,并说明理由。
}

\xiaoti{点燃含有 10 克氢气和 40 克氧气的混和气体,爆炸后能生成多少克水?}

\end{xiti}

