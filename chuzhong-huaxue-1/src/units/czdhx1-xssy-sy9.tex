\section{土壤酸碱性的测定 几种化肥的性质}\label{sec:xssy-sy9}

\begin{shiyanmudi}
    1. 学会用 pH 试纸测定土壤的酸碱度的简单方法; 2. 认识几种常见化肥的性质。
\end{shiyanmudi}


\begin{shiyanyongpin}
    烧杯、pH 试纸、玻璃棒、试管、漏斗、滤纸、铁架台(带铁圈)、酒精灯、坩埚钳。

    碳酸氢铵、硫酸铵、硝酸铵、尿素、过磷酸钙、石灰水、熟石灰、木炭、土样、蒸馏水。
\end{shiyanyongpin}


\begin{shiyanbuzhou}
    1. 测定土壤的酸碱度

    取 10 克土样放在烧杯里, 加 10—15 毫升蒸馏水,充分搅拌后,静置使之沉淀,待悬浊液澄清后,
    用 pH 试纸一端浸入溶液里,立即取出,把试纸颜色跟比色卡对比,就可测出土壤的 pH 值。

    2. 几种常见化肥的性质

    (1) 把碳酸氢铵、硫酸铵、硝酸铵、尿素、过磷酸钙等五种化肥各取少量(约 $0.2$ 克),分别放入五个试管里,观察它们的颜色和状态。

    (2) 闻闻这五种化肥的气味,哪一个试管里有氨的气味,为什么?

    (3) 在 5 个试管里,各加入 10 毫升水,轻轻振荡,观察每种化肥是否容易溶于水。

    (4) 把过磷酸钙和水的混和物过滤,得到过磷酸钙的饱和溶液。向澄清的石灰水里加入过磷酸钙的饱和溶液,
    有什么现象发生?稍放置一会儿,是不是有沉淀产生?如果有沉淀产生,是什么?

    (5) 取上述五种化肥各少许放在纸上,分别跟少量熟石灰混和并研磨,哪种化肥产生氨的气味?

    (6) 取上述五种化肥各少许,分别在红热的木炭上灼烧,观察各有什么现象发生。
\end{shiyanbuzhou}


\begin{wentihetaolun}
    根据实验说明可以用什么方法检验铵盐。
\end{wentihetaolun}

