\fuxiti
\begin{xiaotis}

\xiaoti{固体物质的溶解度主要决定于哪些因素?}

\xiaoti{下列说法是不是正确? 为什么?}
\begin{xiaoxiaotis}

    \xxt{饱和溶液一定是浓溶液,不饱和溶液一定是稀溶液。}

    \xxt{1 升 $10\%$ 的食盐溶液,如果从其中取出 100 毫升,那么这 100 毫升食盐溶液的浓度是 $1\%$。}

    \xxt{在 20 ℃、1 标准大气压下,氧气的溶解度是 $0.031$。这是表示在这个条件下 100 克水中最多能溶解氧气 $0.031$ 克。}

\end{xiaoxiaotis}


\xiaoti{在 20 ℃ 时,食盐的溶解度是 36 克, 能不能在 20 ℃ 时配制 $36\%$ 的食盐溶液?为什么?}


\xiaoti{有 90 ℃ 的氯化铵饱和溶液 340 克,计算:}
\begin{xiaoxiaotis}

    \xxt{蒸发掉 80 克水后,温度仍降至 90 ℃ 时,有多少克氯化铵晶体从溶液中析出?(在 90 ℃ 时氯化铵的溶解度为 $71.3$ 克。)}

    \xxt{把蒸发掉 80 克水后的氯化铵饱和溶液的温度降低到 40 ℃ 时,将有多少克氯化铵晶体从溶液中析出?(在 40 ℃时,氯化铵的溶解度为 $45.8$ 克。)}

    (提示:计算氯化铵饱和溶液的质量时,既要考虑到蒸发掉的溶剂的质量,也要考虑到由于溶剂的蒸发而析出的溶质的质量。)

\end{xiaoxiaotis}


\xiaoti{氯化钾在不同温度时的溶解度如下:\\
    \begin{tblr}{hlines, vlines, columns={mode=math, c, m, colsep+=0.5em}, column{1}={mode=text}}
        温度 (℃)  & 0     &  20 &  40   &  60 &     80 &  100 \\
        溶解度(克) & 27.6  & 34.0 & 40.0  & 45.5 & 51.5 & 56.7
    \end{tblr}
}
\begin{xiaoxiaotis}

    \xxt{绘制氯化钾的溶解度曲线。}

    \xxt{找出在 25 ℃ 时氯化钾的溶解度。}

    \xxt{计算在 25 ℃ 时的氯化钾饱和溶液的百分比浓度。}

\end{xiaoxiaotis}


\xiaoti{30 克锌可以跟 150 克硫酸溶液完全起反应,计算:}
\begin{xiaoxiaotis}

    \xxt{可制得氢气多少克?}

    \xxt{这种硫酸溶液的百分比浓度是多少?}

    \xxt{把 100 克这种硫酸溶液稀释成 $20\%$ 的硫酸溶液,需要加水多少克?}

\end{xiaoxiaotis}

\end{xiaotis}

