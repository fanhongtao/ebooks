\section{盐}\label{sec:5-6}

\subsection{盐的分类和命名}

盐类根据组成不同,一般可以分为以下几种:

1. 正盐 \quad 正盐是酸跟碱完全中和的产物,象 \ce{NaCl} 、\ce{CuSO4}、 \ce{Na2CO3} 等等都是正盐。
其中无氧酸盐的命名是在非金属元素和金属元素名称中间加一 “化” 字,叫做 “某化某”,
如 \ce{NaCl} 叫做氯化钠, \ce{K2S} 叫做硫化钾。
含氧酸盐的命名是在酸的名称后面加上金属的名称,叫做 “某酸某”,
如 \ce{CuSO4} 叫做硫酸铜,\ce{Na2CO3} 叫做碳酸钠等等。

如果一种金属元素具有多种化合价,
对于含低化合价金属元素的盐的命名,可以在金属名称的前面加个 “亚” 字;
对含有高化合价的金属元素的盐,可仍按原来方法命名。
例如, \ce{Fe2(SO4)3} 叫做硫酸铁, \ce{FeSO4} 叫做硫酸亚铁;
\ce{CuCl2} 叫做氯化铜, \ce{CuCl} 叫做氯化亚铜。


2. 酸式盐 \quad 酸式盐是酸中的氢离子部分被中和的产物,象 \ce{NaHCO3}、 \ce{KHSO4} 等等都是酸式盐。
酸式盐的命名是在酸根名称的后面加个 “氢” 字。例如,
\ce{NaHCO3} 叫做碳酸氢钠(也叫酸式碳酸钠),电离出的阴离子 \ce{HCO3-} 叫做碳酸氢根离子。

如果酸式盐中含有两个可以电离的氢原子,命名时可标明数字,
如 \ce{NaH2PO4} 叫做磷酸二氢钠, \ce{Ca(H2PO4)2} 叫做磷酸二氢钙等等。

3. 碱式盐 \quad 碱式盐是碱中的氢氧根离子部分被中和的产物,象 \ce{Cu2(OH)2CO3} 是碱式盐。
碱式盐的命名是在正盐的名称前边加 “碱式” 二字。例如,
\ce{Cu2(OH)2CO3} 叫做碱式碳酸铜。

在化学上,对于含有相同酸根离子或相同金属离子的盐,常给它们一个统称。
例如,含有 \ce{SO4^{2-}} 的盐(象 \ce{Na2SO4}、 \ce{MgSO4} 等)统称硫酸盐,
含有 \ce{K+} 的盐(象 \ce{KCl}、 \ce{K2SO4} 等)统称钾盐等等。


\subsection{盐的性质}

盐在常温下大都是晶体。不同种类的盐在水中的溶解性不同〔参看附录 \ref{app:2} \quad \nameref{app:2}〕。
一般说来,钾盐、钠盐和硝酸盐都易溶于水,而碳酸盐、磷酸盐、氢硫酸盐(硫化物)大多不溶于水。

下面简单介绍盐类在水溶液中所表现的一些化学性质:

\subsubsection{盐跟某些金属的反应}

\begin{shiyan}
    在盛有硫酸铜溶液的试管里,浸入一根洁净的(经过去锈、去油)铁钉,过一会儿,取出,观察有什么变化。
\end{shiyan}

\begin{shiyan}
    在盛有硝酸汞溶液的试管里,浸入一根洁净的铜丝,过一会儿,取出,观察有什么变化。
\end{shiyan}

\begin{shiyan}
    在盛有硫酸锌溶液的试管里,浸入一根洁净的铜丝,过一会儿,取出,观察有什么变化。
\end{shiyan}

从上面的实验可以看出,
放在硫酸铜溶液里的铁钉的表面覆盖一层红色的铜,
放在硝酸汞溶液里的铜丝的表面覆盖一层银白色的汞,
放在硫酸锌溶液里的铜丝的表面没有变化。
\begin{fangchengshi}
    \ce{ CuSO4 + Fe = FeSO4 + Cu } \\[-.5em]
    \ce{ Hg(NO3)2 + Cu = Cu(NO3)2 + Hg }
\end{fangchengshi}

从以上反应可以知道,铁能把铜从铜盐里置换出来,铜能把汞从汞盐里置换出来,
这是因为铁比铜活泼,铜比汞活泼,而铜却不能从锌盐里置换出锌来,这是因为铜不如锌活泼。
可见,在金属活动性顺序里,只有排在前面的金属,才能把排在后面的金属从它们的盐溶液里置换出来。

从以上事实可知,盐跟某些金属起反应,一般能生成另一种盐和另一种金属。

我国在西汉时期,已发现了铁能从铜盐中置换出铜的反应。
到宋初,已把这个反应用于生产,即把铁片或铁块放在硫酸铜溶液里,置换出铜来,成为金属铜粉末。
这种炼铜方法在我国最早应用,是湿法冶金术的先驱。


\subsubsection{盐跟酸的反应}

例如:
\begin{fangchengshi}
    \ce{ BaCl2 + H2SO4 = BaSO4 v + 2HCl } \\[-.5em]
    \ce{ AgNO3 + HCl = AgCl v + HNO3 }
\end{fangchengshi}

盐跟酸起反应,一般生成另一种盐和另一种酸。


\subsubsection{盐跟碱的反应}

例如:
\begin{fangchengshi}
    \begin{aligned}
        \ce{ Na2CO3 + Ca(OH)2 } &= \ce{ CaCO3 v + 2NaOH } \\[-.5em]
        \ce{ FeCl3 + 3NaOH }    &= \ce{ 3NaCl +Fe(OH)3 v }
    \end{aligned}
\end{fangchengshi}
% \begin{fangchengshi}
%     \ce{ Na2CO3 + Ca(OH)2 = CaCO3 v + 2NaOH } \\[-.5em]
%     \ce{ FeCl3 + 3NaOH = 3NaCl +Fe(OH)3 v }
% \end{fangchengshi}

盐跟碱起反应,一般生成另一种盐和另一种碱。


\subsubsection{盐跟另一种盐的反应}

例如:
\begin{fangchengshi}
    \ce{ AgNO3 + NaCl = AgCl v + NaNO3 } \\[-.5em]
    \ce{ 3AgNO3 + Na3PO4 = Ag3PO4 v + 3NaNO3 }
\end{fangchengshi}

两种盐起反应一般生成另外两种盐。



\subsection{复分解反应发生的条件}

上面所学习的盐跟酸、盐跟碱、盐跟盐之间的反应,以及以前学过的中和反应都属于复分解反应。

酸、碱、盐等电解质之间,有的能发生复分解反应,有的就不能。
根据实验证明,两种电解质在溶液中相互交换离子,生成物中如果有沉淀析出,
有气体放出或有水生成\footnote{严格地说,指有难电离的物质生成。},
那么,复分解反应就可以发生,否则就不能发生。例如:
\begin{fangchengshi}
    \begin{aligned}
        \ce{ KCl + AgNO3 }  &= \ce{ KNO3 + AgCl v } \\[-.5em]
        \ce{ CaCO2 + 2HCl } &= \ce{ CaCl2 + H2O + CO2 ^ } \\[-.5em]
        \ce{ NaOH + HCl }   &= \ce{ NaCl + H2O }
    \end{aligned}
\end{fangchengshi}
% \begin{fangchengshi}
%     \ce{ KCl + AgNO3 = KNO3 + AgCl v } \\[-.5em]
%     \ce{ CaCO2 + 2HCl = CaCl2 + H2O + CO2 ^ } \\[-.5em]
%     \ce{ NaOH + HCl = NaCl + H2O }
% \end{fangchengshi}

以上这些复分解反应都可以发生。

如果把氯化钠溶液和硝酸钾溶液混和在一起,既没有沉淀析出,也没有气体放出或水生成,实际上并没有发生复分解反应。

\begin{yuedu}
    为什么会发生以上的现象呢?我们可以从酸、碱、盐所电离出来的离子间的反应——离子反应来进行分折。以氯化钾和硝酸银的反应为例:
    \begin{fangchengshi}
        \ce{ KCl + AgNO3 = KNO3 + AgCl v }
    \end{fangchengshi}

    在上式中,把在溶液中容易电离的物质写成离子的形式,把难溶物质、难电离物质或气体用分子式来表示,可写成如下形式:
    \begin{fangchengshi}
        \ce{ K+ + Cl- + Ag+ + NO3- = K+ + NO3- + AgCl v }
    \end{fangchengshi}

    在溶液中开始时存在四种离子,由于 \ce{Ag+} 和 \ce{Cl-} 结合而成难溶于水的 \ce{AgCl} 沉淀,
    溶液中的 \ce{Ag+} 和 \ce{Cl-} 迅速减少,反应就向右方进行。

    现在再分折一下氯化纳和硝酸钾在溶液中的情况。
    \begin{fangchengshi}
        \ce{ NaCl + KNO3 = NaNO3 + KCl } \\[-.5em]
        \ce{ Na+ + Cl- + K+ + NO3- = Na+ + NO3- + K+ + Cl- }
    \end{fangchengshi}

    从上式可以看出,方程式的等号前后都是四种离子,这些离子混和后没有发生化学变化,也就是说,没有发生复分解反应。
\end{yuedu}



\begin{xiti}

\xiaoti{举出硝酸盐、钾盐、硫酸盐各三种,写出它们的电离方程式。}

\xiaoti{写出下列盐的名称:\\
    \ce{ZnS}, \quad \ce{KClO3}, \quad \ce{NaH2PO4}, \quad \ce{MgCO3}, \quad \ce{Ca(HCO3)2}。
}

\xiaoti{写出下列盐的分子式:\\
    硫酸钙, 碳酸钡, 硫化钙, 磷酸二氢钾, 碱式碳酸铜, 硝酸钙, 氯化铁。
}

\xiaoti{用熟石灰和硫酸铜来配制农药波尔多液时,为什么不能使用铁制容器?}

\xiaoti{在治疗胃酸(含稀盐酸)过多的药物中,常含有氢氧化铝或碳酸氢钠,它们起什么作用?写出有关反应的化学方程式。}

\xiaoti{下列物质间能不能发生复分解反应?如能发生反应,写出有关的化学方程式。}
\begin{xiaoxiaotis}

    \xxt{盐酸和氢氧化钾溶液,}

    \xxt{氯化钠溶液和氢氧化钾溶液,}

    \xxt{硫酸铜溶液和硫化钠(\ce{Na2S}) 溶液,}

    \xxt{氯化钙溶液和稀硝酸。}

\end{xiaoxiaotis}


\xiaoti{有一种蓝色溶液具有下列性质:}
\begin{xiaoxiaotis}

    \xxt{加入氢氧化钠溶液能生成浅蓝色沉淀;}

    \xxt{加入氯化钡溶液,生成白色沉淀,再加稀硝酸,沉淀不溶解;}

    \xxt{加入一根铁钉,铁钉表面出现红色有光泽物质。判断这是什么物质的溶液,并写出有关的化学方程式。}

\end{xiaoxiaotis}

\end{xiti}

