\xiaojie

一、水

1. 水是人类宝贵的自然财富,它对于工农业生产和维持生命都起了重大的作用。因此,应当很好利用这个资源并预防和消除对于水源的污染。

2. 水由氢、氧两种元素组成。



二、氢气的制法和性质

1. 实验室用锌跟稀硫酸(或盐酸)起反应制取氢气。

2. 氢气跟氧气起反应生成水,纯净的氢气在空气里可以安静地燃烧,
跟氧气混和后点燃就发生爆炸,所以点燃氢气前必须检验纯度。

3. 一种单质跟一种化合物起反应,生成另一种单质和另一种化合物,这类反应叫做置换反应。


三、氧化-还原反应

1. 含氧化合物里的氧被夺取的反应,叫做还原反应。使含氧化合物发生还原反应的物质,叫做还原剂。

2. 能供给氧,使别种物质发生氧化反应的物质,叫做氧化剂。

3. 一种物质被氧化,同时另一种物质被还原的反应,叫做氧化-还原反应。

4. 氢气可作还原剂。冶金工业利用氢气的还原性来冶炼某些金属。


四、核外电子的排布,化合物的形成与化合价

1. 多电子原子的电子在核外是分层排布的,这种排布有一定的规律。

2. 情性气体元素原子的最外电子层有 8 个电子(氦是 2 个),是一种稳定结构。
在化学反应中,金属元素的原子比较容易失去最外层电子,通常达到 8 个电子的稳定结构;
非金属元素的原子比较容易获得电子,通常也达到 8 个电子的稳定结构。

3. 带电的原子(或原子团)叫做离子。带正电的离子叫做阳离子,带负电的离子叫做阴离子。

4. 由阴、阳离子相互作用而构成的化合物,叫做离子化合物。

以共用电子对形成分子的化合物,叫做共价化合物。


5. 一种元素一定数目的原子跟其它元素一定数目的原子化合的性质,叫做这种元素的化合价。

在离子化合物里,元素化合价的数值,就是这种元素的一个原子得失电子的数目;
在共价化合物里,元素化合价的数值,就是这种元素的一个原子跟其它元素的原子形成的共用电子对的数目。
不论在离子化合物还是在共价化合物里,正负化合价的代数和都等于零。
根据这个原则,可以检查分子式的正误,也可以在已知化合价时书写化合物的分子式。
只有确实知道有某种化合物存在,才能根据元素的化合价写出它分子式。


五、根据化学方程式的计算

根据化学方程式的计算,首先要审清题意,写出正确的化学方程式,
然后根据已知物质和待求物质之间的质量关系列比例式求解。
计算中要注意书写格式。

