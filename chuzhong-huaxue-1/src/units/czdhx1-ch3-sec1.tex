\section{金刚石和石墨 同素异形现象}\label{sec:3-1}

光彩夺目的金刚石,深灰色的石墨,黑色的炭黑和灰黑色的木炭,表面上看来各不相同,
有些甚至差别很大,但它们实际上都是由同一种元素——碳所组成的。
这里先介绍金刚石和石墨。金刚石和石墨都是自然界里存在的游离态的碳。

金刚石 纯净的金刚石是一种无色透明的、正八面体形状的固体(见封里彩图),密度是 $3.51$ $\kmlflm$。%为了让单位能换行,分开书写
含有杂质的金刚石带棕、黑等颜色。天然采集到的金刚石并不带闪烁光泽,需要仔细琢磨成许多面,
对光才发生折射和散射,成为璀灿夺目的装饰品——钻石。
在天然物质里,金刚石的硬度最大,可以装在钻探机的钻头上,钻凿坚硬的岩层,
可以用来切割大理石,也可以用来加工非常坚硬的金属或刻划玻璃。
金刚石的熔点很高,是电的不良导体,不溶于任何通常的物质里。

石墨 石墨是一种深灰色的、有金属光泽而不透明的细鳞片状固体,密度是 $2.25 \; \kmlflm$。
用手摸石墨,有滑腻的感觉。石墨可以做润滑剂,由于熔点高,也可以在不适宜用普通润滑剂的高温下使用。
石墨是最软的矿物之一。用石墨在纸上划过,会留下深灰色的痕迹。用不同比率的石墨粉末跟粘土粉末混和,
成细棒形,烘干后可制成硬度不同的铅笔芯。石墨有优良的导电性能、可用作干电池的电极。
石墨熔点高,在高温下不易氧化,不受许多化学药品的作用,石墨还可以用作高温电炉的电极。
由于石墨的优良传热性和经受得住温度的骤然升降,可以用来制造熔融钢和其它金属的坩埚。
石墨不受许多化学药品的作用,它还可制成砖、块料和管道,用来砌工业生产上耐强酸的塔和槽。

\begin{yuedu}
    把某种合成纤维跟塑料树脂(两者都含有碳元素)结合并在一定压强下加热,会变成以石墨形式存在的碳纤维。
    碳纤维的强度很大。用碳纤维增强的塑抖,可用在飞机的构件以及气象和通讯人造卫星上。
\end{yuedu}

那末,怎样证明纯净的金刚石和石墨都是由碳元素组成的呢?
只要把金刚石和石墨分别放在氧气里燃烧,结果都同样生成唯一的产物二氧化碳。
这就表明金刚石和石墨都是由碳元素组成的单质。

\begin{yuedu}
    另外,金刚石可以从石墨制取。把石墨加热到 2000 ℃ 和加压到 5 万至 10 万标准大气压,
    并用铬、铁和铂等做催化剂,可以制出人造金刚石。
    1975 年已制出每粒质量为 1 米制克拉\footnote{米制克拉(carat)是珠宝钻石的质量单位。1 米制克拉等于 200 亳克。}重
    的人造金刚石。人造金刚石已使用于工业上。
\end{yuedu}


\zhongdian{一种元素形成几种单质的现象叫做同素异形现象。}
由同一种元素形成的多种单质,叫做这种元素的\zhongdian{同素异形体}。
金刚石和石墨都是碳的同素异形体。

金刚石和石墨既然都是碳元素形成的单质,碳元素的原子核外电子排布都 K 层 2 个电子, L 层 4 个电子,
那末,它们的物理性质为什么会有那么大的差异呢?经过研究知道,这因为金刚石和石墨里碳原子排列的不同。


\begin{xiti}

\xiaoti{学习同素异形现象这个概念以后,你对单质和元素这些概念有些什么新的理解?}

\xiaoti{石墨的用途是由它的性质确定的。\\
    它能用作:
}
\begin{xiaoxiaotis}

    \xxt{润滑剂,因为 \xhx[5cm],}

    \xxt{坩埚,因为  \xhx[5cm],}

    \xxt{电极,因为  \xhx[5cm],}

    \xxt{铅笔芯,因为  \xhx[5cm]。}

\end{xiaoxiaotis}


\xiaoti{试用铅笔芯粉末来润滑因生锈而打不开或难于打开的铁锁,然后用钥匙打开锁。}

\end{xiti}

