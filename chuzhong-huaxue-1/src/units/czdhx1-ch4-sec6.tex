\section{溶液的浓度}\label{sec:4-6}

把溶液区分为浓溶液和稀溶液,这种分法比较粗略,只指溶液中溶质含量的多少,而不能准确地表明一定量的溶液里含有多少溶质。
在实际应用上,常常要确切知道一定量的溶液里究竟含有多少溶质。
例如施用农药的时候,就要较准确地知道一定量的药液里所含农药的量。
因为药液过浓,就要毒害农作物,稀了,就不能杀死害虫病菌。
一定量的溶液里所含溶质的量叫做\zhongdian{溶液的浓度}。

表示溶液浓度的方法很多,这里要讲的主要是质量百分比浓度。

溶液的浓度用溶质的质量占全部溶液质量的百分比来表示的,叫做\zhongdian{质量百分比浓度}(简称\zhongdian{百分比浓度})。
例如,食盐溶液的浓度等于 $5\%$, 就是表示 100 克的溶液里有 5 克食盐和 95 克水,或 100 千克溶液里有 5 千克食盐和 95 千克水。

怎样来配制 100 克 $5\%$ 的食盐溶液呢?

把 5 克食盐溶解在 95 克水里,就制得 100 克 $5\%$ 的食盐溶液。

溶液的百分比浓度可以根据下式进行计算:
$$ \text{百分比浓度} = \dfrac{\text{溶质质量}}{\text{溶质质量} + \text{溶剂质量}} \times 100\% $$


\liti 在农业生产上,有时用 $10 \text{—} 20\%$ 食盐溶液来选种,
配制 150 千克 $16\%$ 食盐溶液,需要食盐和水各多少千克?

\jie 设 150 千克 $16\%$ 食盐溶液里所含食盐的质量为 $x$ 千克。
\begin{align*}
    x &= 150 \times \dfrac{16}{100} \\
    x &= 24 \; (\qianke)
\end{align*}

150 千克 $16\%$ 食盐溶液里所含水的质量:
$$ 150 - 24 = 126 \; (\qianke) $$

答:配制 $16\%$ 食盐溶液 150 千克需要食盐 24 千克和水 126 千克。


\liti 食盐在 0 ℃ 时的溶解度是 35.7 克,计算 0 ℃ 时食盐饱和溶液的百分比浓度是多少?

\jie 食盐在 0 ℃ 时溶解度是 35.7 克,也就是用 100 克水配制成的食盐饱和溶液,总质量是:
$$ 100 \ke + 35.7 \ke = 135.7 \ke $$

\begin{align*}
    \text{食盐溶液的百分比浓度} &= \dfrac{35.7}{100 + 35.7} \times 100\% \\
                              &= 26.3\%
\end{align*}

答: 0 ℃ 时食盐饱和溶液的百分比浓度是 $26.3\%$。



\liti 把 50 克 $98\%$ \ce{H2SO4} 稀释成 $20\%$ \ce{H2SO4} 溶液需要水多少克?

\jie 被稀释溶液里的溶质的质量在稀释前后不变。

设 \quad 稀释后溶液的质量为 $x$。

\begin{gather*}
    50 \times 98\% = x \times 20\% \\
    x = \dfrac{50 \times 98\%}{20\%} = 245 \; (\ke)
\end{gather*}

需要水的克数是:
$$ 245 - 50 = 195 \; (\ke) $$

答:把 50 克 $98\%$ \ce{H2SO4} 稀释成 $20\%$ \ce{H2SO4} 溶液,需要水 195 克。



\liti 配制 $20\%$ \ce{H2SO4} 溶液 460 克,问需要 $98\%$ \ce{H2SO4} 多少毫升?

\jie 设 \quad 需要 $98\%$ \ce{H2SO4} 的质量为 $x$。
\begin{gather*}
    460 \times 20\% = x \times 98\% \\
    x = \dfrac{460 \times 20\%}{98\%} = 93.88 \; (\ke)
\end{gather*}

从 \ce{H2SO4} 密度和百分比浓度对照表\footnote{
    \ce{H2SO4} 密度和百分比浓度对照表( 20 ℃ ):\\[0.5em]
    \begin{tblr}{hlines, vlines, columns={mode=math, c, m}, column{1}={mode=text}}
        {密度\\($\kmlflm$)}    & 1.01 & 1.07 & 1.14  & 1.22  & 1.30  & 1.40  & 1.50  & 1.61  & 1.73  & 1.81  & 1.84 \\
        {百分比浓度\\($\%$)} & 1   &  10  &   20  &   30  &   40  &  50   &   60  &  70   &  80   &  90   & 98
    \end{tblr}
}得知 $98\%$ \ce{H2SO4} 的密度是 $1.84 \; \kmlflm$。

需要 $98\%$ \ce{H2SO4} 的体积:
$$ 93.88 \div 1.84 \approx 51 \; (\lflm) = 51 \; (\haosheng) $$

( 1 立方厘米等于 1 毫升)

答:配制 $20\%$ \ce{H2SO4} 溶液 460 克,需要 $98\%$ \ce{H2SO4} 51 毫升。




\taolun 百分比浓度和溶解度有何区别? 在进行有关溶解度和百分比浓度计算时,应注意哪些问题?

\begin{yuedu}
    ppm 浓度

    ppm 是百万分数的符号。溶液的浓度用溶质质量占全部溶液质量的百万分比来表示的叫 ppm 浓度。
    因为有的溶液浓度极稀,仅含有百万分之几的溶质,如用百分比浓度表示,既不方便,又容易发生错误。
    因此,在溶液浓度极稀时,用 ppm 表示这种溶液的浓度,比较方便。
    例如,某溶液的浓度是百万分之三, 以 3 ppm 表示。

    如果用百分比浓度表示, 换算方法是:
    $$ \dfrac{3}{1000000} \times 100\% = 0.0003\% $$
\end{yuedu}

此外,在应用两种液体配制溶液时,有时用两种液体的体积比表示溶液的浓度,叫做体积比浓度。
例如,配制 $1:4$ 的硫酸溶液,就是指 1 体积硫酸(一般是指 $98\%$、密度为 $1.84 \; \kmlflm$ 的硫酸)
跟 4 体积水配制成的溶液。这种体积比浓度表示溶液的浓度比较粗略,但配制时简便易行,
在农业生产上稀释农药、在医疗上配制药剂、在化学实验室配制溶液,常采用这种浓度。


\begin{xiti}

\xiaoti{下面的说法是不是正确?如不正确,应该怎样改正。}
\begin{xiaoxiaotis}

    \xxt{在 20 ℃ 时, 100 克硫酸铜溶液里含有 10 克硫酸铜,硫酸铜的溶解度是 10 克。}

    \xxt{在 20 ℃ 时, 100 克水溶解 21 克硫酸铜,这种硫酸铜溶液的百分比浓度是 $21\%$。}

    \xxt{有 50 克 $10\%$ \ce{NaCl} 溶液跟 50 克 $20\%$ \ce{NaCl} 溶液相混和,得到 100 克 $30\%$ \ce{NaCl} 溶液。}

\end{xiaoxiaotis}


\xiaoti{在 80 克 $15\%$ \ce{NaNO3} 溶液里加入 20 克水或 20 克硝酸钠,计算用这两种方法制成的两种溶液的百分比浓度。}

\xiaoti{在 $1.5$ 升 $40\%$ \ce{H2SO4} 溶液里,含有多少克硫酸和多少克水?}

\xiaoti{在 10 ℃ 时,氯化铵的溶解度是 33 克, 计算 10 ℃ 的氯化铵饱和溶液的百分比浓度是多少?}

\xiaoti{在一定温度时,食盐饱和溶液的质量是 12 克,把它蒸干后,得食盐 $3.2$ 克,计算:}
\begin{xiaoxiaotis}

    \xxt{这一温度下食盐的溶解度。}

    \xxt{溶液的百分比浓度。}

    \xxt{在同一条件下,食盐饱和溶液的百分比浓度和溶解度在数值上是否相等,哪个数值大,为什么?}

\end{xiaoxiaotis}


\xiaoti{把 100 克 $90\%$ \ce{H2SO4} 稀释成 $10\%$ \ce{H2SO4} 溶液,需要加水多少克?}

\xiaoti{实验室需要配制 $10\%$ 的盐酸 500 克, 需要 $38\%$ 的盐酸多少毫升?( $38\%$ 盐酸的密度是 $1.19 \; \kmlflm$。)}

\xiaoti{某硫酸溶液 100 克,能跟 13 克锌完全起反应,这种硫酸溶液的百分比浓度是多少?}

\end{xiti}

