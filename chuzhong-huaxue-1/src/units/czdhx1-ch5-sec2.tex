\section{酸、碱、盐是电解质}\label{sec:5-2}

我们已经知道,根据化合物在溶液里或熔化状态的导电性,它们可以分为电解质和非电解质两大类。
现在我们来学习几类重要的电解质。

\subsection{酸}

\begin{shiyan}
    用图 \ref{fig:5-1} 中所示的实验装置,分别试验盐酸和硝酸溶液的导电性。
\end{shiyan}

实验表明,盐酸和硝酸溶液都能够导电,它们跟硫酸一样,都是电解质。它们的电离方程式如下:
\begin{fangchengshi}
    \ce{ \underset{\text{盐酸}}{\ce{HCl}} = H+ + Cl- } \\
    \ce{ \underset{\text{硝酸}}{\ce{HNO3}} = H+ + NO3- } \\
    \ce{ \underset{\text{硫酸}}{\ce{H2SO4}} = 2H+ + SO4^{2-} }
\end{fangchengshi}

由上式可以看出,盐酸、硝酸、硫酸在水溶液里都能电离生成氢离子(\ce{H+})。

\zhongdian{电解质电离时所生成的阳离子全部是氢离子的化合物叫做酸。}
盐酸、硝酸和硫酸都属于酸类。

在酸的分子里,除去在水溶液里能够电离生成的氢离子,余下的部分是酸根离子,
例如,\ce{Cl-}、\ce{NO3-}、\ce{SO4^{2-}} 都是酸根离子。
酸根离子所带负电荷的数目等于酸分子电离时生成的氢离子的数目。


\subsection{碱}

\begin{shiyan}
    用图 \ref{fig:5-1} 中所示的装置,分别试验氢氧化钾和氢氧化钡的溶液的导电性。
\end{shiyan}

实验表明,氢氧化钾和氢氧化钡的溶液都能够导电,它们跟氢氧化钠一样,都是电解质。它们的电离方程式如下:
\begin{fangchengshi}
    \ce{ \underset{\text{氢氧化钾}}{\ce{KOH}} = K+ + OH- } \\
    \ce{ \underset{\text{氢氧化钡}}{\ce{Ba(OH)2}} = Ba^{2+} + 2OH- } \\
    \ce{ \underset{\text{氢氧化钠}}{\ce{NaOH}} = Na+ + OH- }
\end{fangchengshi}

由上式可以看出,氢氧化钾、氢氧化钡和氢氧化钠在水溶液里都能电离生成能自由移动的氢氧根离子(\ce{OH-})。

\zhongdian{电解质电离时所生成的阴离子全部是氢氧根离子的化合物叫做碱。}
氢氧化钠、氢氧化钾、氢氧化钡都属于碱类。

由以上所举各例可知,碱在电离的时候,除生成氢氧根离子外,还生成金属离子。
氢氧根离子带一个负电荷,因此,在碱里跟一个金属离子结合的氢氧根离子的数目等于这种金属离子所带正电荷的数目。


\subsection{盐}

\begin{shiyan}
    用图 \ref{fig:5-1} 所示的分别试验碳酸钠、硫酸镁、氯化钡等物质的溶液的导电性。
\end{shiyan}

实验表明,碳酸钠、硫酸镁、氯化钡等物质的溶液都能够导电,它们跟氯化钠一样,都是电解质。它们的电离方程式如下:
\begin{fangchengshi}
    \ce{ \underset{\text{碳酸钠}}{\ce{Na2CO3}} = 2Na+ + CO3^{2-} } \\
    \ce{ \underset{\text{硫酸镁}}{\ce{MgSO4}} = Mg^{2+} + SO4^{2-} } \\
    \ce{ \underset{\text{氯化钡}}{\ce{BaCl2}} = Ba^{2+} + 2Cl- } \\
    \ce{ \underset{\text{氯化钠}}{\ce{NaCl}} = Na+ + Cl- }
\end{fangchengshi}

由上列各式可以看出,碳酸钠、硫酸镁、氯化钡、氯化钠等物质在水溶液里都能电离出金属离子和酸根离子。
象这种\zhongdian{由金属离子和酸根离子组成的化合物叫做盐。}
其中金属离子所带正电荷的总数等于酸根离子所带负电荷的总数。


\begin{xiti}

\xiaoti{写出下列酸的电离方程式:\\
    \ce{HBr}, \quad  \ce{HClO3}, \quad \ce{HI}。
}


\xiaoti{写出下列碱的电离方程式:\\
    \ce{LiOH}, \quad \ce{Ca(OH)2}, \quad \ce{Ba(OH)2}。
}

\xiaoti{写出下列盐的电离方程式:\\
    \ce{FeCl3}, \quad \ce{CuSO4}, \quad  \ce{Ca(NO3)2}, \quad \ce{Al2(SO4)3}。
}

\xiaoti{硫酸氢钠(\ce{NaHSO4}) 溶于水时有 \ce{H+} 生成,它是不是一种酸?为什么?}

\end{xiti}

