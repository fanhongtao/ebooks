\xiaojie

一、金刚石和石墨

一种元素形成几种单质的现象叫做同素异形现象。
由一种元素形成的多种单质,叫做这种元素的同素异形体。
金刚石和看墨都是碳的同素异形体。

纯净的金刚石是一种无色透明的、正八面体形状的固体。
在天然物质里,金刚石的硬度最大。

石墨是一种深灰色的、有金属光泽而不透明的细鳞片状固体。


二、无定形碳 主要有炭黑、木炭、活性炭、焦炭等,都是由石墨的微小晶体和少量杂质构成的。

让木材、煤等含碳的物质隔绝空气加强热的过程,叫做干馏。


三、碳的化学性质

碳能跟氧气、非金属和某些氧化物等起反应。

在跟金属氧化物的反应里,碳可以把金属还原出来。
例如: \ce{ 2CuO + C $\xlongequal{\text{高温}}$ 2Cu + CO2 ^ }

化学上把放出热的反应叫做放热反应,吸收热的反应叫做吸热反应。


四、二氧化碳是一种比空气重的无色气体,在空气中不能燃烧,也不支持燃烧,
溶于水生成碳酸,碳酸容易分解。二氧化碳在加压、降温条件下可液化,甚至凝固成“干冰”。
鉴定二氧化碳的一种最简便的方法是使澄清石灰水变浑浊,生成白色的碳酸钙。
二氧化碳的实验室制法是使碳酸钙跟稀盐酸起反应,工业制法是在石灰窑中高温煅烧石灰石。
二氧化碳用来灭火、制“干冰”以及用作工业原料等。


五、一氧化碳是一种无色有毒的气体,在空气中燃烧生成二氧化碳并放出大量热。
因此是许多气体燃料的主要成分。一氧化碳是一种还原剂,在冶金工业上有重要的用途。

凡有元素化合价升降的化学反应就是氧化-还原反应。


六、碳酸钙在工业生产和建筑材料上有广泛的应用。


七、甲烷是含碳和氢的最简单的有机化合物。甲烷燃烧后生成二氧化碳和水,并放出大量的热,
因此是一种重要的气体燃料。甲烷又名沼气,沼气对于解决农村的燃料问题,改善农村环境卫生,
提高肥料质量等方面具有重要的实际意义。

有机化合物指的是含碳元素的化合物,简称有机物。

