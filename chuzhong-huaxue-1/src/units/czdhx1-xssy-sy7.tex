\section{酸的性质}\label{sec:xssy-sy7}

\begin{shiyanmudi}
    巩固和加深对酸的性质的认识。
\end{shiyanmudi}


\begin{shiyanyongpin}
    试管、试管夹、药匙、酒精灯、玻璃棒、pH 试纸。

    稀盐酸($1:4$)、稀硫酸($1:4$)、稀硝酸($1:4$)、铁片、锌粒、铜片、氧化铜、带锈铁钉、
    氯化钡溶液、硝酸银溶液、碳酸钠、氢氧化钙、石蕊试液、酚酞试液。
\end{shiyanyongpin}


\begin{shiyanbuzhou}
    1. 酸对指示剂的作用

    (1) 在三个试管里分别倒入稀盐酸、稀硫酸和稀硝酸各 2 毫升。观察颜色、状态,并闻气味。
    在每个试管里放入一根玻璃棒。分别用玻璃棒沾一滴酸液到 pH 试纸上。观察试纸颜色的变化
    (显色以半分钟内的变化为准),跟比色卡对比,测出这三种酸的 pH 值。

    (2) 把玻璃棒从上面三个试管里取出,分别滴入 1—2 滴石蕊试液,振荡。观察发生的现象。

    (3) 在三个试管里分别倒入稀盐酸、稀硫酸和稀硝酸各 2 毫升, 分别滴入 1—2 滴酚酞试液, 振荡。观察发生的现象。

    2. 酸跟金属的反应

    在三个试管里分别加入稀盐酸各 2 毫升,依次把铁片、锌粒和铜片放在试管里,观察发生的现象。
    把燃着的火柴移近有现象发生的试管口,又发生什么现象?写出反应的化学方程式,并解释为什么有的试管里没有什么现象发生。

    3. 酸跟碱性氧化物的反应

    (1) 用药匙向一干燥的试管里加入少量氧化铜。然后再向试管里倒入 2 毫升稀硫酸,小心地加热试管
    (注意不要使稀硫酸沸腾),并轻轻振荡。观察发生的现象。写出反应的化学方程式。

    (2) 取一根带锈的铁钉,轻轻地放入试管(应该怎么放)。倒入 2 毫升稀盐酸,加热,直到铁钉上的锈
    (主要成分是 \ce{Fe2O3.H2O}) 去掉为止。写出反应的化学方程式。

    4. 酸跟盐的反应

    (1) 在试管里加入 2 毫升稀硫酸,再滴入几滴氯化钡溶液。观察发生的现象。写出反应的化学方程式。

    (2) 在试管里加入 2 毫升稀盐酸,再滴入几滴硝酸银溶液。观察发生的现象。写出反应的化学方程式。

    (3) 在试管里加入少量碳酸钠粉末,再滴入几滴稀盐酸。观察发生的现象。写出反应的化学方程式。

    5. 酸跟碱的反应

    在试管里加入少量氢氧化钙粉末,再加入少量稀盐酸。观察发生的现象。写出反应的化学方程式。
\end{shiyanbuzhou}


\begin{wentihetaolun}
    1. 根据实验说明检验一种溶液是不是酸溶液,可以用什么方法?

    2. 根据实验说明有些不溶于水的碱性氧化物或碱能溶于酸吗?

    3. 用焊锡进行焊接时,为什么在焊接处先要滴几滴盐酸?
\end{wentihetaolun}

