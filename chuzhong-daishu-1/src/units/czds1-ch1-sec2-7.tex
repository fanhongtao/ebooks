\subsection{加法的运算律}\label{subsec:1-7}

计算: $(+30) + (-20)$, $(-20) + (+30)$。

两次所得的和相同吗?

换两个数再试一试。

关于有理数的加法,有下面的交换律:

\zhongdian{两个数相加,交换加数的位置,和不变。}
\begin{center}
    \framebox{\zhongdian{加法交换律: $\bm{a + b = b + a}$ 。}}
\end{center}
这里 $a$、$b$ 表示任意两个有理数。

计算:$[(+8) + (-5)] + (-4)$,\quad $(+8) + [(-5) + (-4)]$。

两次所得的和相同吗?

换三个数再试一试。

关于有理数的加法,还有下面的结合律:

\zhongdian{三个数相加,先把前两个数相加,或者先把后两个数相加,和不变。}
\begin{center}
    \framebox{\zhongdian{加法结合律: $\bm{(a + b) + c = a + (b + c)}$ 。}}
\end{center}

这里  $a$、$b$、$c$ 表示任意三个有理数。

根据加法交换律和结合律可以推出:三个以上有理数相加,可以任意交换加数的位置,也可先把其中的几个数相加。

\liti 计算 $(+16) + (-25) + (+24) + (-32)$。

\jie $\begin{aligned}[t]
        &(+16) + (-25) + (+24) + (-32) \\
    ={} & [(+16) + (+24)] + [(-25) + (-32)] \\
    ={} & (+40) + (-57) = -17 \juhao
\end{aligned}$

\zhuyi 在上例中,我们把正数和负数分别结合在一起再相加,计算就比较简便。


\liti 10 袋小麦,以每袋 90 千克为准,超过的千克数记作正数,不足的千克数记作负数,称后的记录如图 \ref{fig:1-16}。

\begin{figure}[htbp]
    \centering
    \begin{tikzpicture}
    \tikzset{
        bag/.pic={
            \filldraw [pattern=grid, pattern color=black!80] (0, 0) -- (0.04, 0.1)
                .. controls (-0.04, 0.4) and (-0.04, 0.8) .. (0.04, 0.9)
                -- (0, 1.0)
                -- (0.1, 0.96)
                .. controls (0.2, 1.01) and (0.4, 1.01) .. (0.5, 0.96)
                -- (0.6, 1.0)
                -- (0.56, 0.9)
                .. controls (0.64, 0.8) and (0.64, 0.4) .. (0.56, 0.1)
                -- (0.6, 0)
                -- (0.5, 0.04)
                .. controls (0.4, -0.01) and (0.2, -0.01) .. (0.1, 0.04)
                -- cycle;
        }
    }

    \foreach \pos/\text in {1/+7, 2/+5, 3/-4/, 4/+6, 5/+4} {
        \draw (\pos, 1.3) pic {bag} +(0.3, 1.3) node {$\text$};
    }

    \foreach \pos/\text in {1/+3, 2/-3, 3/-2/, 4/+8, 5/+1} {
        \draw (-0.5+\pos, 0) pic {bag} +(0.3, -0.3) node {$\text$};
    }

    \fill [xshift=6cm, pattern=crosshatch dots] (0, 0)
        [rounded corners=6pt] .. controls (0.8, 1) and (1.8, 1.5) .. (3, 2)
        [sharp corners] .. controls (4, 1.8) and (5, 1.2) .. (6, 0)
        -- (0, 0);
\end{tikzpicture}

    \caption{}\label{fig:1-16}
\end{figure}

总计是超过多少千克或不足多少千克? 10 袋小麦的总重量是多少?

\jie $\begin{aligned}[t]
        &(+7) + (+5) + (-4) + (+6) + (+4) + (+3) + (-3) + (-2) + (+8) + (+1) \\
    ={} & [(-4) + (+4)] + [(+5) + (-3) + (-2)] + [(+7) + (+6) + (+3) + (+8) + (+1)] \\
    ={} & 0 + 0 + (+25) = +25 \juhao
\end{aligned}$

\hspace*{3em} $90 \times 10 + 25 = 925$。

答:总计是超过 25 千克,总重量是 925 千克。

\zhuyi 在上例中,我们把相加得零的数结合起来相加,计算就比较简便。






