\xiaojie

一、本章主要内容是代数式, 整式, 单项式, 多项式的有关概念及整式的加减。

二、代数式是在用字母表示数的基础上建立起来的。
由具体的数到用字母表示数, 就可以简明地表达一些一般的数量和数量关系,
给研究问題和计算带来方便,这是数学上的一个重大发展。

从具体的数的计算到用代数式表示出事物间数量关系, 这是一个由特殊到一般的过程;
用具体的数代替代数式里的字母进行计算, 求出代数式的值,从而解决具体问题, 则是一个由一般到特珠的过程。

三、整式, 单项式和多项式是代数式的基础内容。其中有关项、次数、系数等概念要区别清楚。
判别同类项时要注意: 一是字母相同; 二是同字母的指数分别相同, 两者缺一不可。
合并同类项的要点是: 字母因数不变, 把各个同类项的系数的和作为系数。

四、去括号和添括号在代数式的运算中经常遇到, 去括号和添括号一定要保证原式的值不变。
不论去括号或添括号,都要特别注意, 括号前面放上或去掉 “$-$” 号, 括号里的各项都要变号。

五、整式的加减,实际上就是合并同类项,要注意,在运算时,遇到括号,一般要去掉括号。

整式进行加减的结果还是整式。

