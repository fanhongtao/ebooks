%\subsection{有理数大小的比较}\label{subsec:1-5}
\xiti
\begin{enhancedline}
\begin{xiaotis}

\xiaoti{水库水位上升 $0.07$ 米记作 $+0.07$ 米,下降 $0.04$ 米记作什么?}

\xiaoti{如果 $-50$ 元表示支出 $50$ 元,那么 $+200$ 元表示什么?}

\xiaoti{如果向北为正,那么走 $-70$ 米是什么意思?如果向南为正,那么走 $-70$ 米是什么意思?}

\xiaoti{用正数或负数表示下列具有相反意义的量:}
\begin{xiaoxiaotis}

    \xiaoxiaoti{珠穆朗玛峰高出海平而 $8848.13$ 米(中国登山队在 1975 年测得);}

    \xiaoxiaoti{太平洋最深处低干海平面 $11022$ 米。}

\end{xiaoxiaotis}


\xiaoti{山区气象站测得某一天四个时刻的气温分别为:\\
    \hspace*{2em}零下 2.2 ℃ , 零上 5.7 ℃ , 零下 0.4 ℃, 零下 4。9 ℃。\\
    用正数或负数表示这些温度。
}

\xiaoti{粮库进出粮食的记录如下(运进为正):\\
    \makebox[33em][r]{9月份} \jiange \\
    \begin{tblr}{
        colspec={Q[c, 6em]*{7}{r}},
        column{2-8}={colsep+=0.5em},
        hlines, vlines,
    }
        日期 & 14 & 15 & 16 & 17 & 18 & 19 & 20 \\
        进出(吨) & $+82$ & $-17$ & $-30$ & $+68$ & $-25$ & $+40$ & $-56$
    \end{tblr} \jiange \\
    说明各天的记录的意义。
}


\xiaoti{(1) 任意写出三个正数; (2) 任意写出三个负数。}

\xiaoti{把下列各数中的正数填在左圈正数集合里,负数填在右圈负数集合里:\\
    $-16$,\; $0.004$,\; $+\dfrac{7}{8}$,\; $-\dfrac{1}{2}$,\; $9651$,\; $25.8$,\; $-3.6$,\; $-4$,\; $\dfrac{3}{5}$。
}

\begin{figure}[htbp]
    \centering
    \input{../pic/czds1-ch1-fuxi-8}
    \caption*{(第 8 题)}
\end{figure}

\xiaoti{把下列各数填在相应的大括号里:\\
    $1$,\; $-\dfrac{4}{5}$,\; $8.9$,\; $-7$,\; $\dfrac{5}{6}$,\; $-3.2$,\; $+1008$,\;
    $-0.05$,\; $28$,\; $-9$。\\
    \begin{tblr}{@{}Q[l, 12em]l}
        正整数集合: $\{1, …… \}$  & 负整数集合:$\{\quad ……\}$ \\
        正分数集合: $\{\quad …… \}$  & 负分数集合:\{\quad ……\}
    \end{tblr}
}

\xiaoti{有理数中有没有这样的数,它既不是正数,也不是负数?
    如果有的话,有几个?是什么数?
}


\xiaoti{下面数轴上,$A$、$B$、$C$、$D$、$E$ 各点表示什么数?}

\begin{figure}[htbp]
    \centering
    \begin{tikzpicture}[>=Stealth]
    \draw [->] (-6.5,0) -- (6.5,0);
    \foreach \x in {-6,...,6} {
        \draw (\x,0.3) -- (\x,0);
    }

    \foreach \x in {-5.5,...,5.5} {
        \draw (\x,0.2) -- (\x,0);
    }

    \foreach \x in {-5,...,4} {
        \node at (\x, -0.3) {$\x$};
    }

    \foreach \pos/\text in {-4.5/B, -1.5/C, 0/D, 0.5/E, 3/A} {
        \filldraw [fill=black] (\pos, 0) circle (0.05) +(0, 0.3) node [above] {$\text$};
    }
\end{tikzpicture}

    \caption*{(第 11 题)}
\end{figure}

\xiaoti{在数轴上记出下列各数: $+5.5$,$-6$,$4$,$-3.5$,$0$,$1.5$。}

\xiaoti{$-5$ 的相反数是什么? $+1$ 的相反数是什么? $-3$ 的相反数是什么? $0$ 的相反数是什么?}

\xiaoti{$-1.6$ 是什么数的相反数? 什么数的相反数是 $-0.2$?
    $\dfrac{1}{4}$ 和什么数互为相反敉? $\dfrac{1}{2}$ 和 $-0.5$ 是不是互为相反数?
}

\xiaoti{在数轴上记出 $2$、$-4.5$、$0$ 各数和它们的相反数。}

\xiaoti{简化下列各数的符号:}
\begin{xiaoxiaotis}

    \begin{tblr}{colspec={@{}Q[l, 10em]Q[l, 10em]l}, stretch=1.2}
        \xiaoxiaoti{$-(-16)$;} & \xiaoxiaoti{$-(+20)$;} & \xiaoxiaoti{$+(+50)$;} \\
        \xiaoxiaoti{$-\left( -3\dfrac{1}{2} \right)$;} & \xiaoxiaoti{$+(-8.07)$;} & \xiaoxiaoti{$-\left( +\dfrac{4}{5} \right)$。}
    \end{tblr}
\end{xiaoxiaotis}

\xiaoti{$|+1| = ? \quad |-9|=? \quad \left|-\dfrac{1}{2}\right|=? \quad |10.5|=?$}

\xiaoti{$+3$ 的绝对值是多少? $-3$ 的绝对值是多少?绝对值是 3 的数有几个?
    绝对值是 4 的数有哪几个?绝对值是 0 的数呢?
}

\xiaoti{$|-5| = -5$ 对不对? $|-0.5| = \dfrac{1}{2}$ 对不对?}

\xiaoti{计算:}
\begin{xiaoxiaotis}

    \begin{tblr}{colspec={@{}Q[l, 12em]l}, stretch=1.2}
        \xiaoxiaoti{$|-28| + |-17|$;} & \xiaoxiaoti{$\left|-5\dfrac{3}{8}\right| - \left|-3\dfrac{5}{6}\right|$;} \\
        \xiaoxiaoti{$|-16| \times |-5|$;} & \xiaoxiaoti{$|-0.15| \div |-6|$。}
    \end{tblr}
\end{xiaoxiaotis}

\xiaoti{$-5$ 大于 $-4$,对不对?$-\dfrac{1}{5}$ 大于 $-\dfrac{1}{4}$,对不对?}

\xiaoti{比较下列每对数的大小:}
\begin{xiaoxiaotis}

    \begin{tblr}{colspec={@{}Q[l, 12em]l}, stretch=1.2}
        \xiaoxiaoti{$-9$ 和 $-7$;} & \xiaoxiaoti{$-100$ 和 $+0.01$;} \\
        \xiaoxiaoti{$-\dfrac{5}{8}$ 和 $-\dfrac{3}{8}$;} & \xiaoxiaoti{$\dfrac{4}{5}$ 和 $\dfrac{3}{4}$;} \\
        \xiaoxiaoti{$-1$ 和 $0$;} & \xiaoxiaoti{$-\dfrac{4}{5}$ 和 $\dfrac{3}{4}$;} \\
        \xiaoxiaoti{$-1.9$ 和 $-2.1$;} & \xiaoxiaoti{$-0.75$ 和 $-0.748$;} \\
        \xiaoxiaoti{$0.85$ 和 $-\dfrac{7}{8}$;} & \xiaoxiaoti{$-\dfrac{3}{11}$ 和 $-0.273$;} \\
    \end{tblr}
\end{xiaoxiaotis}

\xiaoti{把三个数从小到大排列,再用“$<$”连接:}
\begin{xiaoxiaotis}

    \twoInLineXxt[12em]{$3$,$-5$,$-4$;}{$-9$,$16$,$-11$。}
\end{xiaoxiaotis}

\xiaoti{比较下列每对数的大小:}
\begin{xiaoxiaotis}

    \begin{tblr}{colspec={@{}Q[l, 16em]l}, stretch=1.5}
        \xiaoxiaoti{$+(-4.8)$ 和 $-\left(+4\dfrac{3}{4}\right)$;} & \xiaoxiaoti{$-\left(-\dfrac{3}{4}\right)$ 和 $-\left(-\dfrac{3}{5}\right)$;} \\
        \xiaoxiaoti{$|-4|$ 和 $-4$;} & \xiaoxiaoti{$-|-2|$ 和 $-(-2)$;} \\
        \xiaoxiaoti{$-\left(-1\dfrac{1}{3}\right)$ 和 $\left|+1\dfrac{2}{3}\right|$;} & \xiaoxiaoti{$-(+3.25)$ 和 $-|-3.245|$。}
    \end{tblr}
\end{xiaoxiaotis}


\xiaoti{下表是我国几个城市某年 1 月份的平均气温,把它们按从高到低的顺序排列。\jiange\\
    \begin{tblr}{
        colspec={*{5}{Q[c, 6em]}},
        hlines, vlines,
    }
        北京 & 武汉 & 广州 & 哈尔滨 & 南京 \\
        $-4.6$℃ & $3.8$℃ & $13.1$℃ & $-19.4$℃ & $2.4$℃
    \end{tblr} \jiange
}


% TODO: wrapfigure 在这里无法正常使用
\begin{minipage}{0.5\textwidth}

\xiaoti{煤矿井下 $A$、$B$、$C$、$D$ 四处的标高分别是: \\
    $A$($-97.4$ 米), \\
    $B$($-159.8$ 米),\\
    $C$($-136.5$ 米),\\
    $D$($-71.3$ 米)。\\
    哪一处最高?哪一处最低?
}

\xiaoti{1 型,2 型,3 型,4 型四个小麦试验品种与 $A$ 型品种的产量比较如下(比 $A$ 型增产为正):\\
    \begin{tblr}{@{}m{10em}l}
        1 型: $+12.4\%$; & 2 型: $-9.8\%$;\\
        3 型: $-6.4\%$;  & 4 型: $+8.6\%$。
    \end{tblr}\\
    四个试验品种哪一个产量最高?哪一个产量最低?
}
\end{minipage}
\quad
\begin{minipage}{0.4\textwidth}
    \centering
    \input{../pic/czds1-ch1-fuxi-26}
    (第 26 题)
\end{minipage}

\end{xiaotis}
\end{enhancedline}
