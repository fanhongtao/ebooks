\subsection{同解方程}\label{subsec:3-2}

看下面的两个方程:
\begin{align}
    & x + 1 = 4,        \label{eq:tjfc-1} \\
    & x + 2 = 5 \juhao  \label{eq:tjfc-2}
\end{align}

可以知道,方程 \eqref{eq:tjfc-1} 的解是 $x = 3$,方程 \eqref{eq:tjfc-2} 的解也是 $x = 3$。
就是说,方程 \eqref{eq:tjfc-1} 和 \eqref{eq:tjfc-2} 的解相同。

如果两个方程的解相同,那么这两个方程叫做\zhongdian{同解方程}。
从上面知道,方程 \eqref{eq:tjfc-1} 和 \eqref{eq:tjfc-2} 是同解方程。

方程 \eqref{eq:tjfc-2} 可以写成
$$ (x + 1) + 1 = 4 + 1 \douhao $$
就是说,方程 \eqref{eq:tjfc-1} 的两边都加上1,就得到与它同解的方程 \eqref{eq:tjfc-2}。

一般地,我们有

\mylabel{dl:tongjie-1}[方程同解原理 1]
\zhongdian{方程同解原理 1 \hspace*{1em}
    方程的两边都加上(或都减去)同一个数或同一个整式,所得方程与原方程是同解方程。
}

再看下面的两个方程:
\begin{align}
    & x + 1 = 3,        \label{eq:tjfc-3} \\
    & 2x + 2 = 6 \juhao \label{eq:tjfc-4}
\end{align}

可以知道,方程 \eqref{eq:tjfc-3} 的解是 $x = 2$,方程 \eqref{eq:tjfc-4} 的解也是 $x = 2$。
就是说,方程 \eqref{eq:tjfc-3} 和 \eqref{eq:tjfc-4} 的解相同,因此它们是同解方程。

方程 \eqref{eq:tjfc-4} 的两边是由方程 \eqref{eq:tjfc-3} 的两边都乘以 2 而得到的。

一般地,我们还有

\mylabel{dl:tongjie-2}[方程同解原理 2]
\zhongdian{方程同解原理 2 \hspace*{1em}
    方程的两边都乘以(或都除以)不等于零的同一个数,所得方程与原方程是同解方程。
}

\lianxi

(口答)根据\nameref{dl:tongjie-1} 和 \hyperref[dl:tongjie-2]{2},说明下面各题里的两个方程是同解方程。
\begin{enhancedline}
\begin{xiaotis}

\xiaoti{$6x - 1= 1$\nsep $6x = 2$。}

\xiaoti{$3x + 2 = -4$\nsep $3x = -6$。}

\xiaoti{$15x = 25$\nsep $3x = 5$。}

\xiaoti{$\dfrac{2}{3}x = 7$\nsep $2x = 21$。}

\end{xiaotis}
\end{enhancedline}
