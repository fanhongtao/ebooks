\subsection{同解不等式}\label{subsec:4-3}

我们已经知道,不等式 $2x < 6$ 的解集是 $x < 3$。
另外,从上一节练习第 2 题的结果可知,不等式 $4x < 12$ 的解集也是 $x < 3$。
所以不等式 $2x < 6$ 与不等式 $4x < 12$ 的解集相同。

一般地说,如果两个不等式的解集相同,那么这两个不等式叫做\zhongdian{同解不等式}。

因此,不等式 $2x < 6$ 与不等式 $4x < 12$ 是同解不等式。
从上一节练习第 2 题的结果还可知道,不等式 $2x < 6$ 与不等式 $2x + 1 < 7$ (即 $2x + 1< 6 + 1$) 也是同解不等式。

关于两个不等式的同解,一般有下面三条原理。

\mylabel{dl:bds-tongjie-1}[不等式同解原理 1]
\zhongdian{不等式同解原理 1 \hspace*{1em}
    不等式的两边都加上(或都减去)同一个数或同一个整式,所得的不等式与原不等式是同解不等式;
}

\mylabel{dl:bds-tongjie-2}[不等式同解原理 2]
\zhongdian{不等式同解原理 2 \hspace*{1em}
    不等式的两边都乘以(或都除以)同一个正数,所得的不等式与原不等式是同解不等式;
}

\mylabel{dl:bds-tongjie-3}[不等式同解原理 3]
\zhongdian{不等式同解原理 3 \hspace*{1em}
    不等式的两边都乘以(或都除以)同一个负数,并且把不等号改变方向后,所得的不等式与原不等式是同解不等式。
}

\liti[0] 为什么下列各题中的两个不等式是同解不等式?

(1) $21x < 14x + 8$ 与 $7x < 8$;

(2) $-5 + x \leqslant -4$ 与 $x \leqslant 1$;

(3) $-16x \geqslant -144$ 与 $x \leqslant 9$。

\jie (1) 因为在不等式 $21x < 14x + 8$ 的两边都减去 $14x$, 就可以得到  $7x < 8$,
所以由\nameref{dl:bds-tongjie-1},可知这两个不等式是同解不等式;

(2) 因为在不等式 $-5 + x \leqslant -4$ 的两边都加上 $5$, 就可以得到 $x \leqslant 1$,
所以由\nameref{dl:bds-tongjie-1},可知这两个不等式是同解不等式;

(3) 因为在不等式 $-16x \geqslant -144$ 的两边都除以 $-16$, 并且把不等号改变方向,就可以得到 $x \leqslant 9$,
所以由\nameref{dl:bds-tongjie-3},可知这两个不等式是同解不等式。

由这个例题中的第 (1),(2) 两个小题可以看出,
\zhongdian{把不等式中的任何一项的符号改变后,从不等号的一边移到另一边,所得的不等式与原不等式是同解不等式。}
就是说,解方程的移项法则对于解不等式同样适用。

\zhuyi 在运用\nameref{dl:bds-tongjie-3} 时,一定不要忘记改变不等号的方向。


\lianxi

为什么下列各题中的两个不等式是同解不等式?

(1) $3x \leqslant 9$ 与 $x \leqslant 3$;

(2) $2x - 7 < 6x$ 与 $-7 < 4x$;

(3) $8 + 3.5x \leqslant 4.5$ 与 $3.5x \leqslant -3.5$;

(4) $-4x < 448$ 与 $x > -112$。


