\xiti

\begin{enhancedline}
\begin{xiaotis}

\xiaoti{用等号 “$=$” 或不等号 “$\neq$” 填空:}
\begin{xiaoxiaotis}

    \begin{tblr}{columns={18em, colsep=0pt}}
        \xxt{$5 + 3 \xhx 12 - 5$;}          & \xxt{$8 + (-4) \xhx 8 - (+4)$;} \\
        \xxt{$1 + 5 \times (-2) \xhx -12$;} & \xxt{$2 \times (3 + 4) \xhx 2 \times 3 + 4$。}
    \end{tblr}

\end{xiaoxiaotis}

\xiaoti{根据下列条件列出方程:}
\begin{xiaoxiaotis}

    \xxt{某数减去 1 再乘以 2, 得 4;}

    \xxt{某数乘以 3 再减去 4, 得 6;}

    \xxt{某数与 6 的和的 3 倍等于 21;}

    \xxt{某数的 $\dfrac{1}{2}$ 与这个数的 $\dfrac{1}{3}$ 的和等于 5。}

\end{xiaoxiaotis}

\xiaoti{检验下列各题括号里的数是不是它们前面方程的解:}
\begin{xiaoxiaotis}

    \begin{tblr}{columns={18em, colsep=0pt}}
        \xxt{$3x = x + 3$,}         & $\left(x = 2\nsep x = \dfrac{3}{2}\right)$; \\
        \xxt{$y = 10 - 4y$,}        & $(y = 1\nsep y = 2\nsep y = 3)$; \\
        \xxt{$(x - 2)(x - 3) = 0$,} & $(x = 0\nsep x = 2 \nsep x = 3)$; \\
        \xxt{$x(x + 1) = 12$,}      & $(x = 3\nsep x = 4\nsep x = -4)$。
    \end{tblr}

\end{xiaoxiaotis}


\xiaoti{根据\nameref{dl:tongjie-1},说明下面各题里的两个方程是同解方程:}
\begin{xiaoxiaotis}

    \begin{tblr}{columns={18em, colsep=0pt}}
        \xxt{$2x - 1 = 3$,}    & $2x = 4$; \\
        \xxt{$4x = 1 + x$,}    & $3x = 1$。
    \end{tblr}

\end{xiaoxiaotis}


\xiaoti{根据\nameref{dl:tongjie-2},说明下面各题里的两个方程是同解方程:}
\begin{xiaoxiaotis}

    \begin{tblr}{columns={18em, colsep=0pt}, stretch=1.5}
        \xxt{$\dfrac{x + 1}{3} = 4$,}      & $x + 1 = 12$; \\
        \xxt{$\dfrac{3}{4}(x - 4) = 3x$,}  & $x - 4 = 4x$。
    \end{tblr}

\end{xiaoxiaotis}

\xiaoti{根据\nameref{dl:tongjie-1} 解下列方程,并写出检验:}
\begin{xiaoxiaotis}

    \begin{tblr}{columns={18em, colsep=0pt}}
        \xxt{$x + 15 = 24$,}       & \xxt{$3x = 4 + 2x$;} \\
        \xxt{$2x - 7 = x$,}        & \xxt{$5y + 8 = 4y$;} \\
        \xxt{$1.8x = 0.8x - 1.2$,} & \xxt{$\dfrac{7}{4}z - \dfrac{1}{2} = \dfrac{3}{4}z$。}
    \end{tblr}

\end{xiaoxiaotis}

\xiaoti{用移项解下列方程,并写出检验:}
\begin{xiaoxiaotis}

    \begin{tblr}{columns={18em, colsep=0pt}}
        \xxt{$2x + 3 = x - 1$;}        & \xxt{$8x - 2 = 7x - 2$;} \\
        \xxt{$3x - 4 + 2x = 4x - 3$;}  & \xxt{$10y + 7 = 12y - 5 - 3y$;} \\
        \xxt{$2.4x - 9.8 = 1.4x - 9$;} & \xxt{$\dfrac{11}{9}z + \dfrac{2}{7} = \dfrac{2}{9}z - \dfrac{5}{7}$。}
    \end{tblr}

\end{xiaoxiaotis}

\xiaoti{根据\nameref{dl:tongjie-2} 解下列方程,并写出检验:}
\begin{xiaoxiaotis}

    \begin{tblr}{columns={18em, colsep=0pt}}
        \xxt{$3x = 12$;}           & \xxt{$-6y = 6$;} \\
        \xxt{$-x = 0$;}            & \xxt{$\dfrac{x}{2} = 8$;} \\[0.5em]
        \xxt{$\dfrac{3}{4}x = 5$;} & \xxt{$-\dfrac{7}{12}x = -1$。}
    \end{tblr}

\end{xiaoxiaotis}

\xiaoti{解下列方程,并用口算检验:}
\begin{xiaoxiaotis}

    \begin{tblr}{columns={18em, colsep=0pt}}
        \xxt{$9x = 6x - 6$;}  & \xxt{$8z = 4z + 1$;} \\
        \xxt{$7x - 6 = -5x$;} & \xxt{$\dfrac{3x}{100} = \dfrac{45}{100}$。}
    \end{tblr}

\end{xiaoxiaotis}

\xiaoti{解下列方程:}
\begin{xiaoxiaotis}

    \begin{tblr}{columns={18em, colsep=0pt}}
        \xxt{$\dfrac{x}{2} + 1 = x$;}  & \xxt{$\dfrac{y}{3} = y - 4$;} \\[0.5em]
        \xxt{$1 = \dfrac{x}{2} - 5$;}  & \xxt{$0.48x - 6 = -0.02x$;} \\
        \xxt{$2x:3 = 6:5$;}            & \xxt{$8:3 = 4x:7$。}
    \end{tblr}

\end{xiaoxiaotis}

\xiaoti{用方程表示下列数量关系,并求出未知数 $x$ 的值:}
\begin{xiaoxiaotis}

    \xxt{$x$ 与 42 的和等于 18;}

    \xxt{$x$ 的 $\dfrac{1}{9}$ 等于 $\dfrac{2}{3}$;}

    \xxt{$x$ 的 4 倍减去 10 等于 30;}

    \xxt{$x$ 的 5 倍等于 $x$ 的 2 倍与 24 的和。}

\end{xiaoxiaotis}

\xiaoti{下列方程的解法对不对?如果不对,错在哪里?应当怎样改正?}
\begin{xiaoxiaotis}

    \xxt{解方程 $\dfrac{x}{2} = 5$。\\
        \jie $\dfrac{x}{2} = 5 = x = 10$。
    }

    \xxt{解方程 $2x - 1 = -x + 5$。\\
        \jie $2x - x = 5 - 1$,\\
        $\therefore \quad x = 4$。
    }

    \xxt{解方程 $\dfrac{6y}{5} = y + 1$。\\
        \jie $6y = 5y + 1$, \\
        \hspace*{1.5em} $6y - 5y = 1$, \\
        $\therefore \quad y = 1$。
    }

\end{xiaoxiaotis}

解方程(第 13 ~ 18 题 ):

\xiaoti{%
    \begin{xiaoxiaotis}%
        \begin{tblr}[t]{columns={18em, colsep=0pt}, column{1}={leftsep=-1.8em}}
            \xxt{$2x + 3 = 11 - 6x$;} & \xxt{$\dfrac{x - 5}{3} = 4$;} \\[0.5em]
            \xxt{$2x - 1 = 5x - 7$;}  & \xxt{$\dfrac{1 - 3x}{2} = 8$。}
        \end{tblr}
    \end{xiaoxiaotis}
}

\xiaoti{%
    \begin{xiaoxiaotis}%
        \begin{tblr}[t]{columns={18em, colsep=0pt}, column{1}={leftsep=-1.8em}}
            \xxt{$3(y + 4) = 12$;}             & \xxt{$\dfrac{3}{4}x - 1 = 7$;} \\
            \xxt{$1 = \dfrac{1 - z}{2} - 1$;}  & \xxt{$-5(x + 1) = \dfrac{1}{2}$。}
        \end{tblr}
    \end{xiaoxiaotis}
}

\xiaoti{}%
\begin{xiaoxiaotis}%
    \xxt[\xxtsep]{$5(x + 8) - 5 = 6(2x - 7)$;}

    \xxt{$2(3y - 4) + 7(4 - y) = 4y$;}

    \xxt{$4x - 3(20 - x) = 6x - 7(9 - x)$;}

    \xxt{$4(2y + 3) = 8(1 - y) - 5(y - 2)$。}

\end{xiaoxiaotis}

\xiaoti{}%
\begin{xiaoxiaotis}%
    \xxt[\xxtsep]{$3x - 4(2x + 5) =7(x - 5) + 4(2x + 1)$;}

    \xxt{$17(2 - 3y) - 5(12 - y) = 8(1 - 7y)$;}

    \xxt{$7(2x - 1) - 3(4x - 1) - 5(3x + 2) + 1 = 0$;}

    \xxt{$5(z - 4) - 7(7 - z) - 9 = 12 - 3(9 - z)$。}

\end{xiaoxiaotis}

\xiaoti{%
    \begin{xiaoxiaotis}%
        \begin{tblr}[t]{columns={18em, colsep=0pt}, column{1}={leftsep=-1.8em}}
            \xxt{$\dfrac{5 - 3x}{2} = \dfrac{3 - 5x}{3}$;}     & \xxt{$y - \dfrac{y - 1}{2} = 2 - \dfrac{y + 2}{5}$;} \\[0.5em]
            \xxt{$\dfrac{x + 2}{4} - \dfrac{2x - 3}{6} = 1$;}  & \xxt{$\dfrac{z - 2}{5} - \dfrac{z + 3}{10} - \dfrac{2z - 5}{3} + 3 = 0$。}
        \end{tblr}
    \end{xiaoxiaotis}
}

\xiaoti{%
    \begin{xiaoxiaotis}%
        \begin{tblr}[t]{columns={18em, colsep=0pt}, column{1}={leftsep=-1.8em}}
            \xxt{$2\dfrac{1}{2} = \dfrac{x + 3}{4} - \dfrac{2 - 3x}{8}$;}      & \xxt{$\dfrac{5y + 1}{6} = \dfrac{9y + 1}{8} - \dfrac{1 - y}{3}$;} \\[0.5em]
            \xxt{$\dfrac{2(x + 3)}{5} = \dfrac{3}{2}x - \dfrac{2(x - 7)}{3}$;} & \xxt{$\dfrac{x - 9}{11} - \dfrac{x + 2}{3} = (x - 1) - \dfrac{x - 2}{2}$。}
        \end{tblr}
    \end{xiaoxiaotis}
}


\xiaoti{下列方程的解法对不对?如果不对,错在哪里?应当怎样改正?}
\begin{xiaoxiaotis}

    \xxt{解方程 $\dfrac{2x - 1}{3} = \dfrac{x + 2}{3} - 1$。\\
        \jie $2x - 1 = x + 2 - 1$,\\
        $\therefore \quad x = 2$。
    }\jiange

    \xxt{解方程 $\dfrac{x - 1}{3} - \dfrac{x + 2}{6} = \dfrac{4 - x}{2}$。\\
        \jie $2x - 2 - x + 2 = 12 - 3x$ \\
        \hspace*{1.5em} $4x = 12$, \\
        $\therefore \quad x = 3$。
    }

\end{xiaoxiaotis}

\xiaoti{根据下列条件列出方程,然后求出某数:}
\begin{xiaoxiaotis}

    \xxt{某数的 5 倍加上 3,等于这个数的 7 倍减去 5;}

    \xxt{某数的 3 倍减去 9,等于这个数的 $\dfrac{1}{3}$ 加上 6。}

\end{xiaoxiaotis}


\xiaoti{根据下列条件列出方程,然后求出某数:}
\begin{xiaoxiaotis}

    \xxt{某数的 8 倍比这个数的 5 倍大 12;}

    \xxt{某数的 $\dfrac{1}{2}$ 加上 4,比这个数的 3 倍少 21。}

\end{xiaoxiaotis}

\xiaoti{}%
\begin{xiaoxiaotis}%
    \xxt[\xxtsep]{$k$ 等于什么数时,代数式 $\dfrac{3k + 5}{7}$ 的值是 2 ?}

    \xxt{$x$ 等于什么数时,代数式 $\dfrac{x - 8}{3}$ 与 $\dfrac{1}{4}x + 5$ 的值相等?}

\end{xiaoxiaotis}


解下列方程(第 23 ~ 25 题):

\xiaoti{}%
\begin{xiaoxiaotis}%
    \xxt[\xxtsep]{$\dfrac{17}{100}x = \dfrac{21}{100}(x - 16)$;}

    \xxt{$\dfrac{65}{100}(y - 1) = \dfrac{37}{100}(y + 1) + 0.1$;}

    \xxt{$3(x + 1) - \dfrac{1}{3}(x - 1) = 2(x - 1) - \dfrac{1}{2}(x + 1)$;}

    \xxt{$\dfrac{3}{4}(z - 1) - \dfrac{2}{5}(3z + 2) = \dfrac{1}{10} - \dfrac{3}{2}(z - 1)$;}

\end{xiaoxiaotis}


\xiaoti{}%
\begin{xiaoxiaotis}%
    \xxt[\xxtsep]{$\dfrac{x - 3}{2} + \dfrac{6 - x}{3} = \dfrac{2}{3}\left(1 + \dfrac{1 + 2x}{4}\right)$;}

    \xxt{$\dfrac{1}{4}\left(1 - \dfrac{3x}{2}\right) - \dfrac{1}{3}\left(2 - \dfrac{x}{4}\right) = 2$;}

    \xxt{$\dfrac{1}{2}\left[x - \dfrac{1}{2}(x - 1)\right] = \dfrac{2}{3}(x - 1)$;}

    \xxt{$\dfrac{3}{2}\left[\dfrac{2}{3}\left(\dfrac{x}{4} - 1\right) - 2\right] - x = 2$。}

\end{xiaoxiaotis}


\xiaoti{}%
\begin{xiaoxiaotis}%
    \xxt[\xxtsep]{$\dfrac{x + 4}{0.2} - \dfrac{x - 3}{0.5} = - 1.6$;}

    \xxt{$\dfrac{4 - 6x}{0.01} - 6.4 = \dfrac{0.02 - 2x}{0.02} - 7.5$。}

\end{xiaoxiaotis}


\xiaoti{}%
\begin{xiaoxiaotis}%
    \xxt[\xxtsep]{在公式 $S = 2\pi r(r + h)$ 中,已知 $S = 942$,$\pi = 3.14$,$r = 10$,求 $h$;}

    \xxt{在公式 $l = l_0 (1 + at)$ 中,已知 $l = 80.096$,$l_0 = 80$,$a = 0.000012$,求 $t$。}

\end{xiaoxiaotis}


\xiaoti{}%
\begin{xiaoxiaotis}%
    \xxt[\xxtsep]{在公式 $v = \dfrac{\pi nD}{1000}$ 中,已知 $v = 120$,$D = 100$,$\pi = 3.142$,求 $n$(得数保留整数);}

    \xxt{在公式 $d = 2a + 1.57(b + c)$ 中,已知 $d = 5200$,$a = 628$,$b = 500$,求 $c$(得数保留整数)。}

\end{xiaoxiaotis}

\xiaoti{如图,求左圈和右圈里的 “$?$”:}

\begin{figure}[htbp]
    \centering
    \begin{tikzpicture}[>=Stealth]
    \tikzset{
        pics/huxian/.style n args={2}{
          code = {
            \draw [->]  (0, 0) node [left] {$#1$} arc [start angle=105, end angle=75, radius=5] node [right] {$#2$};
        }}
    }
    \draw (0, 1.5) ellipse  [x radius=0.8,y radius=1.2];
    \draw (3, 1.5) ellipse  [x radius=0.8,y radius=1.2];
    \draw (0.2, 2.2) pic {huxian={4}{11}};
    \draw (0.2, 1.6) pic {huxian={9}{?}};
    \draw (0.2, 1.0) pic {huxian={?}{17}};
    \node at (1.5, 2.7) {$\times 2 + 3$};
\end{tikzpicture}

    \caption*{(第 28 题)}
\end{figure}


\xiaoti{根据公式 $v = v_0 + at$,填写下表中的空白处:\\[0.5em]
    \begin{tblr}{
        hlines, vlines,
        columns={4em, c, $$},
    }
        v  & v_0 & a & t  \\
           & 0   & 2 & 8  \\
        48 &     & 3 & 14 \\
        15 & 5   &   & 4  \\
        76 & 13  & 7 &    \\
    \end{tblr}\jiange
}


\end{xiaotis}
\end{enhancedline}

