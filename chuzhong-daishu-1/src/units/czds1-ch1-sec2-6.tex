\subsection{有理数加法法则}\label{subsec:1-6}

从一点出发,经过两次运动(向东为正),结果怎样?看下面的例子:

(1) 向东 5 米,再向东 3 米。结果是向东 8 米。

这就是求两次向东运动的和。和小学一样,可以用加法来解答:
$$ (+5) + (+3) = +8 \juhao $$

(2) 向西 5 米,再向西 3 米.结果是向西 8 米。
$$ (-5) + (-3) = -8 \juhao $$

看一看 (1) 和 (2) 中的两个式子,加数的符号有什么特点?
和的符号与加数的符号有什么关系?和的绝对值与加数的绝对值什么关系?

(3) 向东 5 米,再向西 3 米(图 \ref{fig:1-14})。

\begin{figure}[htbp]
    \centering
    \begin{tikzpicture}[>=Stealth]
    \draw [->] (-4,0) -- (6.5,0);
    \foreach \x in {0,...,5} {
        \draw (\x,0.2) -- (\x,0);
    }

    \draw [dashed] (0, 2) -- (0, -1.5)
          (2, 1) -- (2, -1.5);

    \foreach \x in {0} {
        \node [fill=white, inner sep=1pt] at (\x, -0.3) {$\x$};
    }

    \foreach \x in {1,...,5} {
        \node [fill=white, inner sep=1pt] at (\x, -0.3) {$+\x$};
    }

    \draw [->] (0, 1.5) -- (5, 1.5) node [pos=0.5, above] {$+5$};
    \draw (5, 1.5) -- (5, 0.6) node [pos=0.5, right] {$+$};
    \draw [->] (5, 0.6) -- (2, 0.6) node [pos=0.5, above] {$+3$};
    \draw [->] (0, -1.0) -- (2, -1.0) node [pos=0.5, below] {$+2$};
\end{tikzpicture}

    \caption{}\label{fig:1-14}
\end{figure}

结果是向东 2 米。

因为向西 3 米可以看成向东 $-3$ 米, 所以在学了有理数以后,这个问题仍可以用加法来解答:
$$ (+5) + (-3) = +2 \juhao $$

(4) 向东 3 米, 再向西 5 米(图 \ref{fig:1-15})。

\begin{figure}[htbp]
    \centering
    \begin{tikzpicture}[>=Stealth]
    \draw [->] (-4,0) -- (6.5,0);
    \foreach \x in {-2,...,3} {
        \draw (\x,0.2) -- (\x,0);
    }

    \draw [dashed] (-2, 1) -- (-2, -1.5)
          (0, 2) -- (0, -1.5);

    \foreach \x in {-2,...,0} {
        \node [fill=white, inner sep=1pt] at (\x, -0.3) {$\x$};
    }

    \foreach \x in {1,...,3} {
        \node [fill=white, inner sep=1pt] at (\x, -0.3) {$+\x$};
    }

    \draw [->] (0, 1.5) -- (3, 1.5) node [pos=0.5, above] {$+3$};
    \draw (3, 1.5) -- (3, 0.6) node [pos=0.5, right] {$+$};
    \draw [->] (3, 0.6) -- (-2, 0.6) node [pos=0.5, above] {$-5$};
    \draw [->] (0, -1.0) -- (-2, -1.0) node [pos=0.5, below] {$-2$};
\end{tikzpicture}

    \caption{}\label{fig:1-15}
\end{figure}

结果是向西 2 米。
$$ (+3) + (-5) = -2 \juhao $$

看一看 (3) 和 (4) 中的两个式子,加数的符号有什么特点?
和的符号与加数的符号又有什么关系?和的绝对值与加数的绝对值有什么关系?

(5) 向东 5 米,再向西 5 米。结果是 0 米。
$$ (+5) + (-5) = 0 \juhao $$

看上式,两个加数有什么关系?

(6) 向西 5 米,再向东 0 米。结果是向西 5 米。
$$ (-5) + 0 = -5 \juhao $$

综合以上各种情况,得到有理数加法的法则:\jiange

\framebox{\begin{minipage}{0.93\textwidth}
    \zhongdian{1. 同号两数相加,取原来的符号,并把绝对值相加。}

    \zhongdian{2. 异号两数相加,取绝对值较大的加数的符号,并用较大的绝对值减去较小的绝对值。
        互为相反数的两个数相加得零。}

    \zhongdian{3. 一个数同零相加,仍得这个数。}
\end{minipage}}

\lianxi

\begin{xiaotis}

\xiaoti{(口答)上升 8 厘米,再上升 6 厘米, 结果怎样?}
$$ (+8) + (+6) = ? $$

\xiaoti{(口答)下降 8 厘米,再下降 6 厘米, 结果怎样?}
$$ (-8) + (-6) = ? $$

\xiaoti{(口答)上升 8 厘米,再下降 6 厘米, 结果怎样?}
$$ (+8) + (-6) = ? $$

\xiaoti{(口答)上升 6 厘米,再下降 8 厘米, 结果怎样?}
$$ (+6) + (-8) = ? $$

\end{xiaotis}

\lianxijiange

\begin{enhancedline}
\liti[0] 计算:

(1) $(-3) + (-9)$; \quad (2) $\left(-\dfrac{1}{2}\right) + \left(+\dfrac{1}{3}\right)$。

\jie (1) $(-3) + (-9) = -12$;

(2) $\left(-\dfrac{1}{2}\right) + \left(+\dfrac{1}{3}\right) = -\dfrac{1}{6}$。


\lianxi
\begin{xiaotis}
\setcounter{cntxiaoti}{0}

\xiaoti{(口答)\quad $(+4) + (+7)$,\quad $(-4) + (-7)$,\quad $(+4) + (-7)$,\quad
    $(+7) + (-4)$,\quad $(+4) + (-4)$,\quad $(+9) + (-2)$,\quad
    $(-9) + (+2)$,\quad $(-9) + 0$,\quad $0 + (+2)$,\quad $0 + 0$。
}


\xiaoti{计算:\\
    \begin{tblr}{colspec={*{4}{@{}Q[l, 10em]}}}
        $(+12) + (-8)$,  & $(-42) + (+8)$, & $(+84) + (+36)$, & $(-35) + (-25)$, \\
        $(-0.9) + (+1.5)$, & $(+2.7) + (-3)$, & $(-1.1) + (-2.9)$, & $(+2.8) + (+3.7)$, \\
        $\left(+\dfrac{1}{2}\right) + \left(+\dfrac{1}{4}\right)$,
            & $\left(-\dfrac{1}{3}\right) + \left(+\dfrac{1}{2}\right)$,
            & $\left(+\dfrac{1}{2}\right) + \left(-\dfrac{2}{3}\right)$,
            & $\left(-\dfrac{1}{4}\right) + \left(-\dfrac{1}{3}\right)$。
    \end{tblr}
}

\end{xiaotis}

\end{enhancedline}

