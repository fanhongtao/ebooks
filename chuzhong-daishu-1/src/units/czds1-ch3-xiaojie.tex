\xiaojie

一、本章主要内容是方程及其同解原理, 一元一次方程的解法及其应用。

二、含有未知数的等式叫做方程。 如果两个方程的解相同,这两个方程叫做同解方程。 本章介绍的方程的同解原理是:

1. 方程的两边都加上(或都减去)同一个数或同一个整式,所得方程与原方程是同解方程;

2. 方程的两边都乘以(或都除以)不等于零的同一个数,所得方程与原方程是同解方程。

这两条方程同解的原理,是解方程的根据。

\begin{enhancedline}
三、含有一个未知数, 并且未知数的次数是一次的方程叫做一元一次方程。
解一元一次方程, 就是根据方程同解的两条原理, 通过去分母、去括号、移项、合并同类项等步骤,
把原方程化成最简方程 $ax = b \; (a \neq 0)$ 的形式,再在方程的两边都除以未知数的系数 $a$,
从而得出方程的解 $x = \dfrac{b}{a}$。

四、列一元方程解应用题,首先要弄清题意,用字母(例如 $x$)表示问题里的一个未知数;
列出所需要的代数式;然后根据反映这一应用题的包括已知数和未知数的相等关系,列出方程;
通过解方程,求出未知数的值; 并且根据这一应用题的实际意义, 检查求得的值是不是合理;
最后写出答案。
\end{enhancedline}

