\subsection{一元一次方程的应用}\label{subsec:3-4}

\begin{enhancedline}
在小学里, 我们用算术方法和列方程的方法解过一些应用题。 下面进一步介绍怎样列一元一次方程解应用题。

\liti 一种小麦磨成面粉后,重量要减少 $15\%$。为了得到 4250 千克面粉,需要多少千克小麦?

分析:小麦的千克数,减去磨成面粉后减少的千克数,就等于要得到的面粉的千克数。
如果设所需要的小麦为 $x$ 千克,那么根据重量减少 $15\%$,
就可以表示出磨成面粉后减少的千克数,从而可以按照上述相等关系列出方程。

\jie 设需要 $x$ 千克小麦,那么磨成面粉后重量要减少 $\dfrac{15}{100}x$ 千克,根据题意,得
$$ x - \dfrac{15}{100}x = 4250 \juhao $$

解这个方程:
\begin{gather*}
    \dfrac{85}{100}x = 4250 \douhao \\
    x = 5000 \juhao
\end{gather*}

答:需要 5000 千克小麦。

上面的例子说明了列出一元一次方程解应用题的方法:
先用字母(如 $x$)表示题中的未知数;
再列出需要的代数式 \Big( 如例 1 中的 $\dfrac{15}{100}x$ \Big);
然后根据题中包括已知数与未知数的相等关系列出方程;
解这个方程,求出未的值,写出答案(包括单位名称)。

\liti 如图 \ref{fig:3-3},要锻造直径为 60 毫米、高为 20 毫米的圆柱形零件毛坯,
需要截取直径为 40 毫米的圆钢多长(图中 $\phi\;40$ 表示直径的长为 40 毫米)?

\begin{figure}[htbp]
    \centering
    {
\tdplotsetmaincoords{50}{0}
\begin{tikzpicture}[tdplot_main_coords,line cap=butt,line join=round,>=Stealth]
    \tikzset{
        pics/yuanzhuti/.style n args={4}{
        code = {
            \pgfmathsetmacro{\r}{#1}
            \pgfmathsetmacro{\h}{#3}
            \draw[
                left color=gray!50,
                right color=gray!30,
                middle color=white,
                fill opacity=0.6,
                pattern={mylines[angle=90, distance={10pt}]},
            ]
                plot[smooth,variable=\t,domain=\tdplotmainphi:\tdplotmainphi-180]
                    ({\r*cos(\t)},{\r*sin(\t)},\h)
                -- plot[smooth,variable=\t,domain=\tdplotmainphi-180:\tdplotmainphi]
                    ({\r*cos(\t)},{\r*sin(\t)},0) -- cycle;
            \draw plot[smooth,variable=\t,domain=0:360] ({\r*cos(\t)},{\r*sin(\t)},\h);

            \coordinate (O) at (0, 0, \h);
            \coordinate (A) at ({\r*cos(0}, {\r*sin(0)}, \h);
            \coordinate (B) at ({\r*cos(180}, {\r*sin(180)}, \h);

            \coordinate (O') at (0, 0, 0);
            \coordinate (A') at ({\r*cos(0}, {\r*sin(0)}, 0);

            % \draw (A) -- ($(O)!1.5!(A)$);
            % \draw (A') -- ($(O')!1.5!(A')$);
            % \draw [<->] ($(O)!1.4!(A)$) -- ($(O')!1.4!(A')$)
            \draw (A) -- ($(A) + (0.6, 0)$);
            \draw (A') -- ($(A') + (0.6, 0)$);
            \draw [<->] ($(A) + (0.4, 0)$) -- ($(A') + (0.4, 0)$)
                node [pos=0.5, fill=white, inner sep=1pt] {$#4$};

            \draw [<->] (A) -- (B)
                node [pos=0.4, above, fill=white, inner sep=1pt] {$\phi\;#2$};
        }}
    }

    \draw (0, 0) pic {yuanzhuti={1.5}{60}{1.2}{20}};
    \node at (0, -2.3) {零件毛坯};
    \draw (5, 0) pic {yuanzhuti={1}{40}{2.7}{x}};
    \node at (5, -2.3) {圆钢};
\end{tikzpicture}
}

    \caption{}\label{fig:3-3}
\end{figure}

分析:把圆钢锻造成零件毛坯,虽然长度和底面直径发生了变化,但锻造前后的体积是相等的。也就是有等式
$$ \text{圆钢体积} = \text{零件毛坯体积} \juhao $$

如果设应截取的圆钢长为 $x$ 毫米,那么就可以表示出圆钢的体积和零件毛坯的体积,从而可以按照上述相等关系列出方程。

\jie 设应截取的圆钢长为 $x$ 毫米,那么
圆钢的体积是 $\pi \cdot \left(\dfrac{40}{2}\right)^2 \cdot x \; \lfhm$,
零件毛坯的体积是 $\pi \cdot \left(\dfrac{60}{2}\right)^2 \cdot 20 \; \lfhm$,
根据题意,得
$$ \pi \cdot \left(\dfrac{40}{2}\right)^2 \cdot x = \pi \cdot \left(\dfrac{60}{2}\right)^2 \cdot 20 \juhao $$

解这个方程:
\begin{gather*}
    400x = 18000 \douhao \\
    x = 45 \juhao
\end{gather*}

答:应截取的圆钢长为 45 毫米。

实际截料时,要适当留出加工余量,也就是应当比 45 毫米稍长一些。


\lianxi
\begin{xiaotis}

列出一元一次方程解下列应用题:

\xiaoti{买 4 本练习本与 3 攴铅笔一共用了 0.68 元。已知铅笔每的价格是 0.08 元,练习本每本的价格是多少?}

\xiaoti{某地 1982 年粮食平均每公顷产量达到 8160 千克,比 1952 年平均每公顷产量的 4 倍还多 480 千克。
    求该地 1952 年食平均每公顷产量。
}

\xiaoti{把黄豆发成豆芽后,重量可以增加 7.5 倍,要得到 3400 千克这样的豆芽,需要多少千克黄豆?}

\xiaoti{要锻造一个直径为 10 厘米,高为 8 厘米的圆柱形毛坯,应截取直径为 8 厘米的圆钢多长?}

\xiaoti{用直径为 200 毫米的圆钢锻造成长、宽、高分别为 300 毫米、300 毫米、80 毫米的长方体底板,
    应截取圆钢多长(精确到 1 毫米。 对于本节练习和习题,计算时可取 $\pi$ 的近似值为 3.14)?
}

\end{xiaotis}
\lianxijiange


\liti 甲、乙两站相距 360 公里。一列慢车从甲站开出,每小时走 48 公里;
一列快车从乙站开出,每小时走 72 公里。

(1) 两列火车同时开出,相向而行,经过多少小时相遇?

(2)快车先开 25 分,两车相向而行,快车开了几小时与慢车相遇?

分析: (1) 由于两车同时从甲、乙两站开出,相向而行,当它们相遇时,它们所走的路程的和等于两站之间的路程。
如果设两车开了 $x$ 小时相遇,就可以分别表示出两车所走的路程,从而可以按照上述相等关系列出方程。

(2)因为当快车从乙站开了 25 分到达丙地(图 \ref{fig:3-4})时慢车从甲站相向开出,
所以当它们相遇时,快车先走的路程,加上快车从丙地算起所走的路程,再加上慢车走的路程,等于甲、乙两站之间的路程。
如果设慢车开了 $x$ 小时两车相遇,那么上面三段路程都可以表示出来,从而可以按照上述相等关系列出方程。

\begin{figure}[htbp]
    \centering
    \begin{tikzpicture}
    \pgfmathsetmacro{\a}{0}
    \pgfmathsetmacro{\b}{3.4}
    \pgfmathsetmacro{\c}{8.9}
    \pgfmathsetmacro{\d}{10}

    \draw [ultra thick] (\a, 0) node [left] {甲} -- (\d, 0) node[right] {乙};
    \node at (\c, -0.4) {丙};
    \foreach \x in {\a, \b, \c, \d} {
        \draw (\x, 0.2) -- (\x, -0.2);
    }
    \draw[decorate, decoration={brace, amplitude=0.2cm}] (\a, 0.3) -- (\b, 0.3)
        node [pos=0.5, above=0.3em, align=center] {$48x$};
    \draw[decorate, decoration={brace, amplitude=0.2cm}] (\b, 0.3) -- (\c, 0.3)
        node [pos=0.5, above=0.3em, align=center] {$72x$};
    \draw[decorate, decoration={brace, amplitude=0.2cm}] (\c, 0.3) -- (\d, 0.3)
        node [pos=0.5, above=0.3em, align=center] {$72 \times \dfrac{5}{12}$};
    \draw[decorate, decoration={brace,mirror, amplitude=0.8cm}] (\a, -0.3) -- (\d, -0.3)
        node [pos=0.5, below=1cm, align=center] {$360$};
\end{tikzpicture}

    \caption{}\label{fig:3-4}
\end{figure}


\jie (1) 设两车开了 $x$ 小时相遇, 那么慢车走了 $48x$ 公里, 快车走了 $72x$ 公里,根据题意,得
$$ 72x + 48x = 360 \juhao $$

解这个方程:
\begin{gather*}
    120x = 360 \douhao \\
    x = 3 \juhao
\end{gather*}

答: 两车开了 3 小时相遇。

(2) 设慢车开了 $x$ 小时两车相遇,那么快车先走的路程是 $72 \times \dfrac{25}{60}$ 公里,
快车从丙地算起所走的路程是 $72x$ 公里,慢车走的路程是 $48$ 公里,根据题意,得
$$ 72 \times \dfrac{25}{60} + 72x + 48x = 360 \juhao $$

解这个方程:
\begin{align*}
    30 + & 120x = 360 \douhao \\
         & 120x = 330 \douhao \\
         & x = 2\dfrac{3}{4} \juhao \\
    \dfrac{25}{60} + & 2\dfrac{3}{4} = 3\dfrac{1}{6} \juhao
\end{align*}

答: 快车开了 3 小时 10 分与慢车相遇。


\liti 在甲处劳动的有 27 人,在乙处劳动的有 19 人。
现调 20 人去支援,使在甲处劳动的人数为乙处劳动人数的 2 倍,应调往甲、乙两处各多少人?

分析:这个问题里的相等关系可以表示为
$$ \text{调人后甲处人数} = \text{调人后乙处人数的 2 倍} \juhao $$
因为共调了 20 人支援,所以如果设应调往甲处 $x$ 人,就可以表示出应调往乙处的人数,
从而可以按照上面的相等关系列出方程。

\jie 设应调往甲处 $x$ 人,那么应调往乙处 $(20 - x)$ 人。
于是在调人支援后,甲处人数是 $27 + x$,乙处人数是 $19 + (20 - x)$。 根据题意,得
$$ 27 + x = 2 [19 + (20 - x)] \juhao $$

解这个方程:
\begin{align*}
    27 + x &= 2 (39 - x) , \\
    27 + x &= 78 - 2x , \\
        3x &= 51 , \\
         x &= 17 \juhao
\end{align*}

并且,
$$ 20 - x = 20 - 17 = 3 \juhao $$

答:应调往甲处17 人, 调往乙处3 人。


\lianxi
\begin{xiaotis}

列出一元一次方程解下列应用题:

\xiaoti{挖一条长 1210 米的水渠,由甲、乙两个施工队从两头同时施工。
    甲队每天挖 130 米, 乙队每天挖 90 米。 挖好水渠需要几天?
}

\xiaoti{甲、乙两站间的路程为 284 公里,一列慢车从甲站开往乙站,每小时走48 公里;
    慢车走了 1 小时后,另有一列快车从乙站开往甲站,每小时走 70 公里。快车开了几小时与慢车相遇?
}

\xiaoti{有两个运输队,第一队有 32 人, 第二队有 28 人, 现因任务需要,
    要求第一队人数是第二队人数的 2 倍。 需从第二队抽调多少人支援第一队?
}

\xiaoti{两个水池共贮水 40 吨, 甲池注进水 4 吨, 乙池放出水 8 吨,
    甲池水的吨数就与乙池水的吨数相等。两个水池原各有水多少吨?
}

\xiaoti{把面积是 16 公顷的一片地分成两部分, 使它们的面积的比等于 $3:5$,
    每一部分的面积是多少(提示:面积的比为 $3:5$, 表示分别占总面积的 3 份与 5 份,
    于是可设其中的 1 份为 $x$ 公顷) ?
}

\end{xiaotis}
\end{enhancedline}
