\subsection{绝对值}\label{subsec:1-4}

为了区分具有相反意义的量,我们用了正数和负数。
例如,两辆汽车,第一辆向东行驶了 5 千米,第二辆向西行驶了 4 千米。
如果要表示它们行驶的力向(向东为正)和路程,就分别记作 $+5$ 千米和 $-4$ 千米。

但是,有的时候我们只需要研究行驶的路程,不需要考虑方向,就可以分别记作 5 千米和 4 千米。
这里的 5 叫做 $+5$ 的绝对值,4 叫做 $-4$ 的绝对值。

\begin{enhancedline}
我们说,\zhongdian{一个正数的绝对值是它本身;一个负数的绝对值是它的相反数;零的绝对值是零。}

例如,$+5$ 的绝对值就是它本身 $5$,
$-4$ 的绝对值就是它的相反数 $-(-4)$ 即 $4$。
同样,$\dfrac{1}{3}$ 和 $-\dfrac{1}{3}$ 的绝对值都是 $\dfrac{1}{3}$。

从数轴上看,一个数的绝对值就是表示这个数的点离开原点的距离。

\begin{figure}[htbp]
    \centering
    \begin{tikzpicture}[>=Stealth]
    \draw [->] (-5.3,0) -- (6.5,0);
    \foreach \x in {-5,...,6} {
        \draw (\x,0.3) -- (\x,0);
    }

    \foreach \pos/\text in {-4/-4, 0/0, 3/+3} {
        \node at (\pos, 0) [below] {$\text$};
    }

    \draw (-4, 0.4) -- (-4, 1)
          (0, 0.4) -- (0, 1)
          (3, 0.4) -- (3, 0.8);
    \draw [->] (0, 0.8) -- (-4, 0.8);
    \draw [->] (0, 0.6) -- (3, 0.6);
\end{tikzpicture}

    \caption{}\label{fig:1-8}
\end{figure}

例如,$+3$ 的绝对是 $3$ , 表示 $+3$ 的点离开原点的距离是 $3$ 个位长度;
$-4$ 的绝对值是 $4$, 表示 $-4$ 的点离开原点的距离是 $4$ 个单位长度(图 \ref{fig:1-8})。

要表示一个数的绝对值,我们在这个数的两旁各画一条竖线。

例如,$+4$ 的绝对值记作 $|+4|$, $-6$ 的绝对值记作 $|-6|$;
$\left| +\dfrac{2}{3} \right|$表示 $+\dfrac{2}{3}$ 的绝对值,
$|-4.5|$ 表示 $-4.5$ 的绝对值。


\liti[0] $|+8| = ? \quad
        |-8| = ? \quad
        \left| +\dfrac{1}{4} \right| = ? \quad
        \left| -\dfrac{1}{4} \right| = ?$

\jie $\begin{aligned}[t]
    &|+8| = 8, \quad |-8| = 8, \\
    & \left| +\dfrac{1}{4} \right| = \dfrac{1}{4}, \quad \left| -\dfrac{1}{4} \right| = \dfrac{1}{4} \juhao
\end{aligned}$


\lianxi
\begin{xiaotis}

\xiaoti{(口答)说出下列各数的绝对值是多少?\\
    $+7$, $-2$, $\dfrac{3}{4}$, $-9.6$。
}

\xiaoti{$|-3| = ?$ \quad
    $\left| +1\dfrac{1}{2} \right| = ?$ \quad
    $|-1| = ?$ \quad
    $|9| = ?$ \quad
    $|0| = ?$ \quad
    $|-0.4| = ?$
}
\end{xiaotis}
\end{enhancedline}

