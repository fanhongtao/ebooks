\subsection{一元一次不等式和它的解法}\label{subsec:4-4}

\begin{enhancedline}
我们来看下面的不等式:
$$ 2x < 6\nsep 4x - 7 > 3\nsep \dfrac{2y - 1}{3} - y < 0 \juhao $$
这些不等式,只含有一个未知数,并且未知数的次数是一次,这样的不等式,叫做\zhongdian{一元一次不等式}。

解一元一次不等式,就是求这个不等式的解集的过程。它的一般步骤与解一元一次方程类似,
但一定要注意当两边都乘以(或都除以)同一个负数时,不等号的方向必须改变。

\liti 解不等式 $3(1 - x) < 2(x + 9)$, 并把它的解集在数轴上表示出来。

\jie 去括号,得
$$ 3 - 3x < 2x + 18 \juhao $$

移项,得
$$ -3x - 2x < 18 - 3 \juhao $$

合并同类项,得
$$ -5x < 15 \juhao $$

两边都除以 $-5$,得
$$ x > -3 \juhao $$

这个不等式的解集在数轴上表示如下(图 \ref{fig:4-3}):

\begin{figure}[htbp]
    \centering
    \begin{tikzpicture}[>=Stealth,scale=0.8]
    \draw [->] (-7,0) -- (7.5,0);
    \foreach \x in {-6,...,6} {
        \draw (\x,0.2) -- (\x,0) node[anchor=north] {$\x$};
    }

    \pic [transform shape] {infinity interval={start=-3, stop=7, height=0.7}};
    \draw [fill=white] (-3, 0) circle(0.1);
\end{tikzpicture}

    \caption{}\label{fig:4-3}
\end{figure}

\liti 解不等式 $\dfrac{2 + x}{2} \geqslant \dfrac{2x - 1}{3}$,并把它的解集在数轴上表示出来。

\jie 去分母,得
$$ 3(2 + x) \geqslant 2(2x - 1) \juhao $$

去括号,得
$$ 6 + 3x \geqslant 4x - 2 \juhao $$

移项,得
$$ 3x - 4x \geqslant -2 - 6 \juhao $$

合并同类项,得
$$ -x \geqslant -8 \juhao $$

两边同除以 $-1$,得
$$ x \leqslant 8 \juhao $$

这个不等式的解集在数轴上表示如下(图 \ref{fig:4-4}):

\begin{figure}[htbp]
    \centering
    \begin{tikzpicture}[>=Stealth,scale=0.8]
    \draw [->] (-2,0) -- (11,0);
    \foreach \x in {-1,...,10} {
        \draw (\x,0.2) -- (\x,0) node[anchor=north] {$\x$};
    }

    \pic [transform shape] {infinity interval={start=8, stop=-2, height=0.7}};
    \draw [fill=black] (8, 0) circle(0.1);
\end{tikzpicture}

    \caption{}\label{fig:4-4}
\end{figure}

在解例 2 的过程中,在去括号得出 $6 + 3x \geqslant 4x -2$ 后,如果把含 $x$ 的项移到不等号的右边,那么得
$$ 6 + 2 \geqslant 4x - 3x \juhao $$

合并同类项,得
$$ 8 \geqslant x \douhao $$
即
$$ x \leqslant 8 \juhao $$

这两种解法都是正确的,后一种解法比较简便。

\liti $x$ 取什么值时,代数式 $2x - 5$ 的值:

\twoInLine{(1) 大于 0 ?}{(2)不大于 0 ?}

分析:问 “$x$ 取什么值时,代数式 $2x - 5$ 的值大于 0”,就是问 “$x$ 取什么值时,不等式
$$ 2x - 5 > 0 $$
成立”。为此就要求这个不等式的解集。同样,问 “$x$ 取什么值时,代数 $2x - 5$ 的值不大于 0”,
就是求不等式
$$ 2x - 5 \leqslant 0$$
的解集。

\jie (1)根据题意,要求不等式
$$ 2x - 5 > 0$$
的解集。解这个不等式,得
\begin{align*}
    2x &> 5 \douhao \\
     x &> \dfrac{5}{2} \juhao
\end{align*}
所以当 $x$ 取大于 $\dfrac{5}{2}$ 的值时,$2x - 5$ 的值大于 0 。

(2)根据题意,要求不等式
$$ 2x - 5 \leqslant 0$$
的解集。解这个不等式,得
\begin{align*}
    2x &\leqslant 5 \douhao \\
     x &\leqslant \dfrac{5}{2} \juhao
\end{align*}
所以当 $x$ 取不大于 $\dfrac{5}{2}$ 的值时,$2x - 5$ 的值不大于 0 。


\liti 求不等式 $3x - 10 \leqslant 0$ 的正整数解。

\jie 解不等式 $3x - 10 \leqslant 0$,得
$$ x \leqslant 3\dfrac{1}{3} \juhao $$

因为不大于 $3\dfrac{1}{3}$ 的正整数有 $1$,$2$,$3$ 三个,
所以不等式 $3x - 10 \leqslant 0$ 的正整数解是 $1$,$2$,$3$ 。

\lianxi
\begin{xiaotis}

\xiaoti{解下列不等式,并把它们的解集在数轴上表示出来:}
\begin{xiaoxiaotis}

    \begin{tblr}{columns={18em, colsep=0pt}}
        \xxt{$x + 3 > 2$;} & \xxt{$-2x < 10$;} \\
        \xxt{$3x + 1 < 2x - 5$;} & \xxt{$2 - 5x \geqslant 8 - 2x$;} \\
        \xxt{$\dfrac{1}{2}(3 - x) \geqslant 3$;} & \xxt{$1 + \dfrac{x}{3} \geqslant 5 - \dfrac{x - 2}{2}$。}
    \end{tblr}

\end{xiaoxiaotis}

\xiaoti{$x$ 取什么值时,代数式 $3x + 7$ 的值:}
\begin{xiaoxiaotis}

    \twoInLineXxt[18em]{不小于 1 ?}{不大于 1 ?}

\end{xiaoxiaotis}

\xiaoti{求不等式 $10(x + 4) + x \leqslant 84$ 的非负整数解。}

\end{xiaotis}
\end{enhancedline}

