\subsection{加减法统一成加法}\label{subsec:1-9}

在式子 $(-20) - (+5) + (+3) - (-7)$ 里,有加法,也有减法。
根据有理数减法的法则,可以把它改写成:
$$ (-20) + (-5) + (+3) + (+7) \juhao \footnote{象这样把加减法统一写成加法的式子,有时也叫做\textbf{代数和}。}$$

这样一来,式子里的减法就都转化成为加法。

因此,一切加法和减法的运算,都可以统一成加法运算。

在一个和里,通常把各个加号省略不写。例如
$$ (-20) + (-5) + (+3) + (+7) $$
可以写成省略加号的和的形式:
$$ -20 -5 + 3 + 7 \juhao $$
读作 “负 20、负 5、正 3、正 7 的和”。
事实上,上式又可看作是 $(-20) - (+5) + (+3) + (+7)$,所以也可读作
“负 20 减 5 加 3 加 7”。


\lianxi

把 $(-8) - (+4) + (-6) - (-1)$ 中的减法改成加法,再写成省略加号的和。

\lianxijiange


\begin{enhancedline}
\liti[0] 计算:

(1) \quad $12 + 7 - 5 - 30 + 2$;

(2) \quad $\left(+\dfrac{1}{3}\right) - \left(+\dfrac{1}{2}\right) + \left(-\dfrac{3}{4}\right) - \left(-\dfrac{2}{3}\right)$。

\jie (1) \quad $12 + 7 - 5 - 30 + 2 = 12 + 7 + 2 - 5 - 30 = 21 - 35 = -14$;

(2) \quad $\begin{aligned}[t]
        & \left(+\dfrac{1}{3}\right) - \left(+\dfrac{1}{2}\right) + \left(-\dfrac{3}{4}\right) - \left(-\dfrac{2}{3}\right) \\
    ={} & \dfrac{1}{3} - \dfrac{1}{2} - \dfrac{3}{4} + \dfrac{2}{3} = \dfrac{1}{3} + \dfrac{2}{3} - \dfrac{1}{2} - \dfrac{3}{4} \\
    ={} & 1 - 1\dfrac{1}{4} = -\dfrac{1}{4} \juhao
\end{aligned}$

\zhuyi 在这里,我们运用了加法运算律,这是因为在代数中,加减法都可以统一成加法。
\end{enhancedline}

\lianxi

计算: (1) \, $5 - 8$; \quad (2) \, $-4 + 7 - 6$; \quad (3) \, $6 + 9 - 15 + 3$。

