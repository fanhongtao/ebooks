\subsection{乘法的运算律}\label{subsec:1-11}

小学学过的乘法的运算律有哪些?

小学学过的加法的运算律对有理数仍适用, 乘法的运算律适用不适用呢?

我们看下面的例子:

\hspace*{2em} $(+5) \times (-6) = -30$, $(-6) \times (+5) = -30$,

就是 \quad $(+5) \times (-6) = (-6) \times (+5)$。

\hspace*{2em} $[(+3) \times (-4)] \times (-5) = (-12) \times (-5) = 60$,

\hspace*{2em} $(+3) \times [(-4) \times (-5)] = (+3) \times (+20) = 60$,

就是 \quad $\begin{aligned}[t]
        & [(+3) \times (-4)] \times (-5) \\
    ={} & (+3) \times [(-4) \times (-5)] \juhao
\end{aligned}$

换一些数再试一试。 一般地, 我们有:

\zhongdian{两个数相乘,交换因数的位置,积不变。}
\begin{center}
    \framebox{\zhongdian{乘法交换律: $\bm{ab = ba}$ 。}}
\end{center}

\zhongdian{三个数相乘,先把前两个数乘,或者先把后两个数相乘,积不变。}
\begin{center}
    \framebox{\zhongdian{乘法结合律: $\bm{(ab)c = a(bc)}$ 。}}
\end{center}

在上面, 我们把 “$a \times b$” 写成 $ab$。 在不引起误会的时候,乘号可以用 “$\cdot$” ,或者省略不写。

再看下面的例子:

\hspace*{2em} $5 \cdot [(+3) + (-7)] = 5 \cdot (-4) = -20$,

\hspace*{2em} $5 \dot (+3) + 5 \cdot (-7) = 15 + (-35) = -20$,

就是 \quad $5 \cdot [(+3) + (-7)] = 5 \dot (+3) + 5 \cdot (-7)$。

换一些数再试一试。 一般地, 我们有:

\zhongdian{一个数同两个数的和相乘,等于把这个数分别同这两个数相乘,再把积相加。}
\begin{center}
    \framebox{\zhongdian{分配律: $\bm{a(b + c) = ab + ac$}。}}
\end{center}


\lianxi

下列式子各说明哪一条运算律? 怎样用字母表示这条运算律?

\begin{xiaotis}

\xiaoti{(口答)\; $(-4) \cdot 8 = 8 \cdot (-4)$。}

\xiaoti{(口答)\; $(3 + 9) + (-5) = 3 + [9 + (-5)]$。}

\xiaoti{(口答)\; $(-6) \cdot (7 + 2) = (-6) \cdot 7 + (-6) \cdot 2$。}

\xiaoti{(口答)\; $(5 \times 4) \times 6 = 5 \times (4 \times 6)$。}

\xiaoti{(口答)\; $(-8) + (-9) = (-9) + (-8)$。}

\end{xiaotis}

\lianxijiange

\begin{enhancedline}

\liti 计算 $\left(\dfrac{1}{4} + \dfrac{1}{6} - \dfrac{1}{2}\right) \times 12$。

\jie $\begin{aligned}[t]
        & \left(\dfrac{1}{4} + \dfrac{1}{6} - \dfrac{1}{2}\right) \times 12 \\
    ={} & \dfrac{1}{4} \times 12 + \dfrac{1}{6} \times 12 - \dfrac{1}{2} \times 12 \\
    ={} & 3 + 2 - 6 = -1 \juhao
\end{aligned}$

\liti 计算 $9\dfrac{18}{19} \times 15$。

\jie $\begin{aligned}[t]
        & 9\dfrac{18}{19} \times 15 \\
    ={} & (10 - \dfrac{1}{19}) \times 15 \\
    ={} & 150 - \dfrac{15}{19} = 149\dfrac{4}{19} \juhao
\end{aligned}$

\zhuyi 应用运算律,有时可使运算简便。


\lianxi

计算:

\begin{xiaotis}
\setcounter{cntxiaoti}{0}

\xiaoti{$(-85)(-25)(-4)$}。

\xiaoti{$\left(-\dfrac{7}{8}\right) \times 15 \times \left(-1\dfrac{1}{7}\right)$}。

\xiaoti{$\left(\dfrac{9}{10} - \dfrac{1}{15}\right) \times 30$}。

\xiaoti{$\dfrac{24}{25} \times 7$}。

\end{xiaotis}

\end{enhancedline}

