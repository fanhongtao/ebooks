\subsection{不等式的解集}\label{subsec:4-2}

看不等式 $2x < 6$。 这是一个含有未知数的不等式。
用 2 代替 $x$,不等式能够成立; 用 3 代替 $x$,不等式不能成立。
同方程类似,我们可以说,2 是不等式 $2x < 6$ 的解,3 不是不等式 $2x < 6$ 的解。

可以发现: $1$,$0$,$-2.5$,$-4$ 等数也都是不等式 $2x < 6$ 的解;$3.5$,$4$,$4.5$ 等数都不是不等式 $2x < 6$ 的解。
实际上,用小于 3 的任何一个数代替 $x$, 不等式都能成立;而用等于或大于 3 的任何一个数代替 $x$,不等式都不能成立。
因此,小于 3 的每一个数都是不等式 $2x < 6$ 的解; 而大于或等于 3 的任何一个数,都不是不等式 $2x < 6$ 的解。
可以看出,不等式 $2x < 6$ 有无限多个解。

我们说,不等式 $2x < 6$ 的所有的解,组成不等式 $2x < 6$ 的解的集合,简称不等式 $2x < 6$ 的解集。
一般地说,一个含有未知数的不等式的所有的解,组成这个不等式的\zhongdian{解的集合}, 简称这个不等式的\zhongdian{解集}。

不等式 $2x < 6$ 的解集,可以记作 $x < 3$。

求不等式的解集的过程,叫做\zhongdian{解不等式}。

不等式的解集可以在数轴上直观地表示出来。例如:

如果不等式的解集是 $x < 3$, 就可以用数轴上表示 3 的点的左边部分来表示(图 \ref{fig:4-1}), 这里的圆圈表示不包括 3 这一点。

\begin{figure}[htbp]
    \centering
    \begin{tikzpicture}[>=Stealth,scale=0.8]
    \draw [->] (-5,0) -- (10,0);
    \foreach \x in {-4,...,9} {
        \draw (\x,0.2) -- (\x,0) node[anchor=north] {$\x$};
    }

    \pic [transform shape] {infinity interval={start=3, stop=-5, height=0.7}};
    \draw [fill=white] (3, 0) circle(0.1);
\end{tikzpicture}

    \caption{}\label{fig:4-1}
\end{figure}

如果不等式的解集是 $x \geqslant -2$(记号 “$\geqslant$” 读作 “大于或等于”, 意思也可说是 “不小于”;
类似地,记号 “$\leqslant$” 读作 “小于或等于”, 意思也可说是 “不大于”),
就可以用数轴上表示 $-2$ 的点和它的右边部分来表示(图 \ref{fig:4-2}),这里的黑点表示包括 $-2$ 这一点。

\begin{figure}[htbp]
    \centering
    \begin{tikzpicture}[>=Stealth,scale=0.8]
    \draw [->] (-6,0) -- (9,0);
    \foreach \x in {-5,...,8} {
        \draw (\x,0.2) -- (\x,0) node[anchor=north] {$\x$};
    }

    \pic [transform shape] {infinity interval={start=-2, stop=8, height=0.7}};
    \draw [fill=black] (-2, 0) circle(0.1);
\end{tikzpicture}

    \caption{}\label{fig:4-2}
\end{figure}

\lianxi
\begin{xiaotis}

\xiaoti{根据下列数量关系,列出不等式:}
\begin{xiaoxiaotis}

    \begin{tblr}{columns={18em, colsep=0pt}}
        \xxt{$x$ 的 3 倍大于 1;}   & \xxt{$x$ 与 5 的和是负数;} \\
        \xxt{$y$ 与 1 的差是正数;} & \xxt{$x$ 的一半不大于 10。}
    \end{tblr}

\end{xiaoxiaotis}

\xiaoti{通过试验求出下列不等式的解集,并与不等式 $2x < 6$ 的解集进行比较:}
\begin{xiaoxiaotis}

    \twoInLineXxt[18em]{$2x + 1 < 7$;}{$4x < 12$。}

\end{xiaoxiaotis}

\xiaoti{在数轴上表示下列不等式的解集:}
\begin{xiaoxiaotis}

    \begin{tblr}{columns={18em, colsep=0pt}}
        \xxt{$x > 5$;}         & \xxt{$x \geqslant 0$;} \\
        \xxt{$x \leqslant 3$;} & \xxt{$x < -2\dfrac{1}{2}$。}
    \end{tblr}

\end{xiaoxiaotis}

\end{xiaotis}

