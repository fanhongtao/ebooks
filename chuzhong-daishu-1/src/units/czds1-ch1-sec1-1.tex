\subsection{正数和负数}\label{subsec:1-1}

我们在生产劳动和日常生活中需要计算物体的个数,就使用了自然数 1、2、3、……;
为了用数表示没有物体,就使用了数 0 ;
在测量物体的长度、重量等的时候,往往不能正好得到整数的结果,就使用了分数和小数。
这些数我们在小学已经学习过了。

只有这些数能不能满足实际需要呢?我们看下面的例子:
有一天最高温度是零上 5 ℃(℃ 读作摄氏度),最低温度是零下 5 ℃(图 \ref{fig:1-1})。
要表示出这两个温度,如果只用小学学过的数,把它们都记作 5 ℃,就不能把它们区别清楚。

\begin{figure}[htbp]
    \centering
    \begin{minipage}{7cm}
    \centering
    \begin{tikzpicture}
    \draw (0,0) pic {thermometer={5}};
    \draw (3,0) pic {thermometer={-5}};
\end{tikzpicture}

    \caption{}\label{fig:1-1}
    \end{minipage}
    \qquad
    \begin{minipage}{7cm}
    \centering
    \begin{tikzpicture}[>=Stealth, scale=1.5]
    \draw [rounded corners=6pt, ground]
        (0, 0) .. controls (1,0) and (1,0) .. (1.5,3)
            -- (3, 3) .. controls (3,2) and (3,1) .. (4, 0.2);
    \draw [<->] (0.5, 0.05) -- (0.5, 1.2)
        node[pos=.5, rotate=90, fill=white, inner sep=0] {\small 3.6}
        node[pos=-0.05, fill=white, inner sep=1pt, below] {\small 乙};
    \draw [dashed] (0.2, 1.2) -- (4, 1.2);
    \draw [<->] (3.8, 1.2) -- (3.8, 3)
        node[pos=.5, rotate=90, fill=white, inner sep=0] {\small 5.2};
    \node at (3, 3) [fill=white, inner sep=1pt, anchor=south east] {\small 甲};
    \draw [dashed] (3, 3) -- (4.5, 3);
    \draw (4.1, 1.4) -- (4.3, 1.4) -- (4.2, 1.2) -- cycle;
    \path [path picture={\draw (path picture bounding box.north west) pic {waterwave};}] (3.2, 1.2) -- (4.5, 1.2) -- (3.9, 0.6) -- cycle;
\end{tikzpicture}

    \caption{}\label{fig:1-2}
    \end{minipage}
\end{figure}

零上 5 ℃ 和零下 5 ℃ 虽然都是 5 ℃,但是它们的意义是相反的,
一个在 0 ℃ 的上面, 一个在 0 ℃ 的下面。
为了区别这种具有相反意义的量,我们把零上 5 ℃ 记作 $+5$ ℃ (读作正 5 摄氏度)或 5 ℃;
把零下 5 ℃ 记作 $-5$ ℃(读作负 5 摄氏度)。
也就是说,我们把一种意义的量——零上温度规定为正的,把另一种与它相反意义的量——零下温度规定为负的。
正的量用小学学过的数的前面放上 “$+$”(读作正)号来表示,也可以把 “$+$” 号省略不写,仍旧用以前学过的数表示;
负的量就用小学学过的数的前而放上“$-$” (读作负)号来表示。

\begin{enhancedline}
具有相反意义的量的例子很多,例如,甲地高出海平面 5.2 米,乙地低于海平面 3.6 米(图 \ref{fig:1-2});
昨天运进货物 $8\dfrac{1}{2}$ 吨, 今天运出货物 $4\dfrac{1}{2}$ 吨; 等等。
我们可以把高出海平面 5.2 米记作 $+5.2$ 米或 5.2 米,低于海平面 3.6 米记作 $-3.6$ 米;
运进货物 $8\dfrac{1}{2}$ 吨记作 $+8\dfrac{1}{2}$ 吨 或 $8\dfrac{1}{2}$,
运出货物 $4\dfrac{1}{2}$ 吨记作  $-4\dfrac{1}{2}$ 吨。
\end{enhancedline}


\lianxi
1.(口答) 举出一些具有相反意义的量的实例。

2.(口答) 如果向东走 5 千米记作 $+5$ 千米,那么向西走 6 千米记作什么?

3.(口答) 如果下降 400 米记作 $-400$ 米,那么上升 800 米记作什么?

4.(口答) 如果节余 10.32 元记作 $+10.32$ 元,那么亏损 4.15 元记作什么?

\jiange

\begin{enhancedline}
象 $+5$、$+8\dfrac{1}{2}$、$+5.2$ 等带有正号的数叫做\zhongdian{正数}(正号也可省略不写)。
象 $-5$、$-4\dfrac{1}{2}$、$-3.6$ 等带有负号的数叫做\zhongdian{负数}。
零既不是正数, 也不是负数。
\end{enhancedline}

\lianxi
(口答)读出下列各数,它们各是正还是负数?

\begin{enhancedline}
$+6$,$-8.75$,$-0.4$,$0$,$\dfrac{3}{7}$,$9.15$,$-\dfrac{2}{3}$,$+1\dfrac{4}{5}$。

\jiange

\liti[0] 所有的正数组成正数集合,所有的负数组成负数集合。
把下列各数中的正数和负数分别填在表示正数集合和负数集合的圈里:

$-11$,$4.8$,$+73$,$-2.7$,$\dfrac{1}{6}$,$+\dfrac{7}{12}$,$-8.12$,$-\dfrac{3}{4}$。
\end{enhancedline}

\begin{figure}[htbp]
    \centering
    \begin{tikzpicture}
    \draw (2, 0) circle [x radius=2, y radius=1];
    \node at (2, -1.5) {正数集合};
    \draw (8,0) circle [x radius=2, y radius=1];
    \node at (8, -1.5) {负数集合};
\end{tikzpicture}

    \caption{}\label{fig:1-3}
\end{figure}

\jie
\begin{figure}[htbp]
    \centering
    \begin{tikzpicture}
    \draw (2, 0) circle [x radius=2, y radius=1];
    \node at (2, -1.5) {正数集合};
    \draw (8,0) circle [x radius=2, y radius=1];
    \node at (8, -1.5) {负数集合};

    \node at (0.8, 0.5) {$4.8$};
    \node at (1.8, 0.5) {$+73$};
    \node at (2.8, 0.5) {\Large $\frac{1}{6}$};
    \node at (1.8, -0.5) {\Large $+\frac{7}{12}$};
    \node at (2.8, -0.5) {……};

    \node at (7.0, 0.5) {$-11$};
    \node at (8.0, 0.5) {$-2.7$};
    \node at (6.9, -0.5) {$-8.12$};
    \node at (7.9, -0.5) {\Large $-\frac{3}{4}$};
    \node at (8.9, -0.5) {……};
\end{tikzpicture}

    \caption{}\label{fig:1-4}
\end{figure}

到现在为止,我们学过的数有:

正整数(也叫自然数),如 $+1$ 、$+2$ 、$+3$、……;

零, $0$;

负整数,如 $-1$ 、$-2$ 、$-3$、……;

\begin{enhancedline}
正分数,如 $+8\dfrac{1}{2}$、 $+5.2 \; \left( \text{即} +5\dfrac{1}{5} \right)$ 、$\dfrac{2}{3}$、……;

负分数,如 $-4\dfrac{1}{2}$、 $-3.6 \; \left( \text{即} -3\dfrac{3}{5} \right)$ 、$-\dfrac{6}{7}$、……;

正整数、零、负整数统称\zhongdian{整数},正分数、负分数统称\zhongdian{分数}。
\end{enhancedline}

整数和分数统称\zhongdian{有理数}。

\zhuyi 整数也可看作是分母为 1 的分数,因此分数包括整数。
有时为了研究需要,也把整数和分数分开,这里的分数是指不包括整数的分数。


\lianxi
\begin{xiaotis}
\begin{enhancedline}
\xiaoti{(口答)下列各数,是整数还是分数,是正数还是负数?\\
    $-7$, $10.1$, $-\dfrac{1}{6}$, $89$, $0$, $-0.67$, $1\dfrac{3}{5}$。
}

\xiaoti{(口答)说出几个正整数、负整数、正分数、负分数。}
\end{enhancedline}
\end{xiaotis}

