% 原书的目录结构就是如此(缺少 section)
% 忽略这里的报错: Difference (2) between bookmark levels is greater (hyperref)	than one, level fixed.
\subsection{不等式}\label{subsec:4-1}

看下面的式子:

\hspace*{2em}\begin{tblr}{columns={8em}}
    $-7 < -5$, & $3 + 4 > 1 + 4$, \\
    $2x < 6$, & $a + 2 > a + 1$, \\
    $5 + 3 \neq 12 - 5$, & $a \neq 0$。
\end{tblr} \\
这些式子含有我们已经学过的符号 “$<$”, “$>$” 或 “$\neq$”,这些符号都叫做\zhongdian{不等号}。
在这些式子中, 有的只表明不等号的左、右两边不相等;
有的不仅表明不等号的左、右两边不相等, 而且表明哪边大哪边小。
象这种表示不相等关系的式子,叫做\zhongdian{不等式}。

小于号 “$<$” 和大于号 “$>$” , 都是表示大小关系的不等号。
在中学数学中研究的不等式, 如果不特别说明, 都是指表示大小关系的不等式。

\lianxi
\begin{xiaotis}

\xiaoti{(口答) 用小于号 “$<$” 或大于号 “$>$” 填空:}
\begin{xiaoxiaotis}

    \begin{tblr}{columns={18em, colsep=0pt}}
        \xxt{$-5 \xhx -3$;} & \xxt{$-2 \xhx 0$;} \\
        \xxt{$9 \xhx -14$;} & \xxt{$3.254 \xhx 3\dfrac{1}{4}$;} \\
        \xxt{$\dfrac{2}{7} \xhx \dfrac{1}{3}$;} & \xxt{$-\dfrac{1}{2} \xhx -\dfrac{1}{3}$。}
    \end{tblr}

\end{xiaoxiaotis}

\xiaoti{(口答) 用小于号 “$<$” 或大于号 “$>$” 填空:}
\begin{xiaoxiaotis}

    \begin{tblr}{columns={18em, colsep=0pt}}
        \xxt{$7 + 3 \xhx 4 + 3$;} & \xxt{$7 - 3 \xhx 4 - 3$;} \\
        \xxt{$7 \times 3 \xhx 4 \times 3$;} & \xxt{$7 \times (-3) \xhx 4 \times (-3)$。}
    \end{tblr}

\end{xiaoxiaotis}

\xiaoti{用不等式表示:}
\begin{xiaoxiaotis}

    \begin{tblr}{columns={18em, colsep=0pt}}
        \xxt{$a$ 是正数;}        & \xxt{$a$ 是负数;} \\
        \xxt{$x$ 不等于 1;}      & \xxt{$m + n$ 是正数;} \\
        \xxt{$x$ 的 4 倍大于 7;} & \xxt{$b$ 与 6 的和小于 5。}
    \end{tblr}

\end{xiaoxiaotis}

\end{xiaotis}
\lianxijiange

下面我们来研究不等式的基本性质。

看不等式
$$ 7 > 4 \juhao $$

我们先看看上面做过的练习第 2 题的答案, 然后把 3 换成 5, 做同样的试验:

1. 两边都加上(或都减去)5, 结果怎样? 不等号的方向变了吗?
$$ 7 + 5 \xhx 4 + 5\nsep 7 - 5 \xhx 4 - 5 \juhao $$

2. 两边都乘以 5 , 结果怎样? 不等号的方向变了吗?
$$ 7 \times 5 \xhx 4 \times 5 \juhao $$

3. 两边都乘以 $-5$, 结果怎样? 不等号的方向变了吗?
$$ 7 \times (-5) \xhx 4 \times (-5) \juhao $$

我们发现:在第 1 种情况和第 2 种情况下,不等号的方向不变;
在第 3 种情况下, 在不等式的两边都乘以同一个负数后,不等号的方向改变了。

换一个不等式 $-2 < 6$ 再试一试。

一般地说,不等式有下面三条基本性质:

\mylabel{dl:budengshi-1}[不等式基本性质 1]
\zhongdian{1. 不等式的两边都加上(或都减去)同一个数,不等号的方向不变。}

这就是说:
如果 $a < b$, 那么 $a + c < b + c$ (或 $a - c < b - c$);
如果 $a > b$, 那么 $a + c > b + c$ (或 $a - c > b - c$)。

\begin{enhancedline}
\mylabel{dl:budengshi-2}[不等式基本性质 2]
\zhongdian{2. 不等式的两边都乘以(或都除以)同一个正数,不等号的方向不变。}

这就是说:
如果 $a < b$, 并且 c > 0, 那么 $ac < bc$  \Big( 或 $\dfrac{a}{c} < \dfrac{b}{c}$ \Big);
如果 $a > b$, 并且 c > 0, 那么 $ac > bc$  \Big( 或 $\dfrac{a}{c} > \dfrac{b}{c}$ \Big)。

\mylabel{dl:budengshi-3}[不等式基本性质 3]
\zhongdian{3. 不等式的两边都乘以(或都除以)同一个负数,不等号的方向改变。}

这就是说:
如果 $a < b$, 并且 c < 0, 那么 $ac > bc$ \Big( 或 $\dfrac{a}{c} > \dfrac{b}{c}$ \Big);
如果 $a > b$, 并且 c < 0, 那么 $ac < bc$ \Big( 或 $\dfrac{a}{c} < \dfrac{b}{c}$ \Big)。

想一想:如果不等式的两边都乘以零,会出现什么结果呢?
\end{enhancedline}

\liti 按照下列条件, 写出仍能成立的不等式:

(1) $5 < 9$, 两边都加上 $-2$ ;

(2) $9 > 5$, 两边都减去 10 ;

(3) $-5 < 3$, 两边都乘以 4 ;

(4) $14 > -8$, 两边都除以 $-2$。

解: (1) 根据\nameref{dl:budengshi-1} , 在不等式 $5 < 9$ 的两边都加上 $-2$, 不等号的方向不变, 所以
$$ 5 + (-2) < 9 + (-2) \douhao $$
即
$$ 3 < 7 \fenhao $$

(2) 根据\nameref{dl:budengshi-1} , 得
$$ 9 - 10 > 5 - 10 \douhao $$
即
$$ -1 > -5 \fenhao $$

(3) 根据\nameref{dl:budengshi-2} , 得
$$ -5 \times 4 < 3 \times 4 \douhao $$
即
$$ -20 < 12 \fenhao $$

(4) 根据\nameref{dl:budengshi-3} , 得
$$ 14 \div (-2) < -8 \div (-2) \douhao $$
即
$$ -7 < 4 \juhao $$


\liti 设 $a > b$,用不等号连结下列各题中的两式:

\begin{tblr}{columns={18em, colsep=0pt}}
    (1)$a - 3$ 与 $b - 3$; & (2)$2a$ 与 $2b$; \\
    (3)$-a$ 与 $-b$。
\end{tblr}

\jie (1)因为 $a > b$,两边都减去 3,由\nameref{dl:budengshi-1},得
$$ a - 3 > b - 3 \fenhao $$

(2)因为 $a > b$,而 $2 > 0$,由\nameref{dl:budengshi-2},得
$$ 2a > 2b \fenhao $$

(3)因为 $a > b$,而 $-1 < 0$,由\nameref{dl:budengshi-3},得
$$ -a < -b \juhao $$

\lianxi
\begin{xiaotis}

\xiaoti{按照下列条件,写出仍能成立的不等式:}
\begin{xiaoxiaotis}

    \xxt{$-7 < 8$,两边都加上 $9$;\\
         $-7 < 8$,两边都加上 $-9$。
    }

    \xxt{$5 > -2$,两边都减去 $6$;\\
        $5 > -2$,两边都减去 $-6$。
    }

    \xxt{$-3 > -4$,两边都乘以 $7$;\\
         $-3 > -4$,两边都乘以 $-7$。
    }

    \xxt{$-8 < 0$,两边都除以 $8$;\\
         $-8 < 0$,两边都除以 $-8$。
    }

\end{xiaoxiaotis}

\xiaoti{设 $a < b$,用不等号连结下列各题中的两式:}
\begin{xiaoxiaotis}

    \begin{tblr}{columns={18em, colsep=0pt}}
        \xxt{$a + 5$ 与 $b + 5$;} & \xxt{$3a$ 与 $3b$;} \\
        \xxt{$-5a$ 与 $-5b$;}     & \xxt{$\dfrac{a}{3}$ 与 $\dfrac{b}{3}$。}
    \end{tblr}

\end{xiaoxiaotis}

\end{xiaotis}

