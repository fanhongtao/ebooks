\subsection{相反数}\label{subsec:1-3}

我们看 $+6$ 和 $-6$ 这两个数,只有符号不同,一正一负。
在数轴上表示这两个数的点,分别在原点的两旁,离开原点的距离相等。

$2\dfrac{1}{2}$ 和 $-2\dfrac{1}{2}$ 也是这样。

象这样只有符号不同的两个数,我们说其中一个是另一个的\zhongdian{相反数}。
$+6$ 是 $-6$ 的相反数, $-6$ 是 $+6$ 的相反数,$+6$ 和 $-6$ 互为相反数。
同样, $2\dfrac{1}{2}$ 和 $-2\dfrac{1}{2}$ 互为相反数。

\zhongdian{零的相反数是零}。


\lianxi

1. (口答)$+9$ 的相反数是什么? $-7$ 的相反数是什么?

2. (口答) $-2.4$ 是什么数的相反数?$\dfrac{3}{5}$ 是什么数的相反数?

\vspace{2em}

我们知道, $+2$ 和 $2$ 是一样的,就是说 $+2 = 2$,同样 $+(+3) = +3$,$+(-4) = -4$。

$-2$ 是 $2$ 的相反数。 同样,
$-(+3)$ 是 $+3$ 的相反数,就是 $-(+3) = -3$;
$-(-4)$ 是 $-4$ 的相反数,就是 $-(-4) = 4$。

\zhongdian{在一个数前面添上一个 “$+$” 号,仍与原数相同;
在一个数前面添上一个 “$-$” 号,就成为原数的相反数。}

$\bm{+0 = 0}$,$\bm{-0 = 0}$。


\lianxi
\begin{xiaotis}

\xiaoti{简化下列各数的符号:\\
    $-(+8)$, $+(-9)$, $-(-6)$, $-(+7)$, $+(+\dfrac{2}{3})$。
}

\xiaoti{下列各对数中,哪些是相等的数?哪些互为相反数?\\
    \begin{tblr}{Q[l, 12em]l}
        $+(-8)$ 和 $-8$,       & $-(-8)$ 和 $-8$, \\
        $-(-8)$ 和 $+(-8)$,    & $-(+8)$ 和 $+(-8)$, \\
        $-(-8)$ 和 $+(+8)$,    & $+8$    和 $+(-8)$。 \\
    \end{tblr}
}

\end{xiaotis}

