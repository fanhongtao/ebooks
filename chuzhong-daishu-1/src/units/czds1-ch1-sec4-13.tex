\subsection{有理数的乘方}\label{subsec:1-13}

我们来计算:
1. 边长是 7 厘米的正方形的面积;
2. 棱长是 5 厘米的正方体的体积(图 \ref{fig:1-20})。
\begin{align*}
    & 7 \times 7 = 49 \; (\text{平方厘米}); \\
    & 5 \times 5 \times 5 = 125 \; (\text{立方厘米}) \juhao
\end{align*}

$7 \times 7$,$5 \times 5 \times 5$ 都是相同因数的乘法。

\begin{figure}[htbp]
    \centering
    \begin{tikzpicture}
        \draw [step=0.5] (0, 0) grid (3.5, 3.5);
        \draw [step=0.5] (6, 1) pic {cubes={0}{2.5}{0}{2.5}{0}{2.5}};
    \end{tikzpicture}
    \caption{}\label{fig:1-20}
\end{figure}

为了简便起见,相同的因数相乘,可以只写一个因数,而在它的右上角写上相同的因数的个数。
例如,$7 \times 7$ 记作 $7^2$,$5 \times 5 \times 5$ 记作 $5^3$。

同样,$(-2)(-2)(-2)(-2)$ 记作 $(-2)^4$,
$\dfrac{3}{4} \times \dfrac{3}{4} \times \dfrac{3}{4} \times \dfrac{3}{4} \times \dfrac{3}{4}$
记作 $\left(\dfrac{3}{4}\right)^5$。


\lianxi
\begin{xiaotis}

\xiaoti{(口答)$8 \times 8 \times 8$ 怎样记?}

\xiaoti{(口答)$(-6)(-6)(-6)(-6)$ 怎样记?}

\xiaoti{(口答)$0.1^2$ 表示什么意思?}

\xiaoti{(口答)$\left(-\dfrac{2}{5}\right)^3$ 表示什么意思?}

\end{xiaotis}
\lianxijiange

$n$ 个相同的因数相乘,即
\begin{minipage}[t]{6em}
    $a \cdot a \cdot a \cdot \cdots \cdot a$ \\[0.3em]
    \hspace*{0.3em}\begin{tikzpicture}[>=Stealth]
        \draw[decorate,decoration={brace,mirror,amplitude=0.2cm}] (-1.0, -0.2) -- (1.0, -0.2)
            node [pos=0.5, below=0.3em, align=center] {$n$ 个};
    \end{tikzpicture}
\end{minipage}
记作 $a^n$。

这种求 $n$ 个相同因数的积的运算,叫做\zhongdian{乘方},乘方的结果叫做\zhongdian{幂}。
在 $a^n$ 中, $a$ 叫做\zhongdian{底数},$n$ 叫做\zhongdian{指数},$a^n$ 读作 $a$ 的 $n$ 次方。
$a^n$ 看作是 $a$ 的 $n$ 次方的结果时,也可读作 $a$ 的 $n$ 次幂。

\begin{figure}[htbp]
    \centering
    \begin{tikzpicture}[>=Stealth]
        %\draw (0, 0) rectangle (-1, 1) node [pos=0.5] {$a^n$};
        \node [rectangle, minimum size=1cm, draw=black] {$a^n$};
        \draw [->] (-1, 0) -- (-0.5, 0) node [pos=0, left] {幂};
        \draw [->] (1, 0.1) -- (0.2, 0.1) node [pos=0, right] {指数};
        \draw [->] (-0.1, -1) -- (-0.1, -0.2) node [pos=0, below] {底数};
    \end{tikzpicture}
\end{figure}

例如,在 $9^4$ 中,底数是 $9$,指数是 $4$, $9^4$ 读作 $9$ 的 $4$ 次方,或 $9$ 的 $4$ 次幂。


\lianxi
\begin{xiaotis}
\setcounter{cntxiaoti}{0}

\xiaoti{(口答) $10^2$ 读作什么? 其中底数是什么, 指数是什么?}

\xiaoti{(口答) $7^3$ 读作什么? 其中 $3$ 叫什么数, $7$ 叫什么数?}

\end{xiaotis}

\lianxijiange

二次方也叫\zhongdian{平方}, 三次方也叫\zhongdian{立方}。
象上面的 $10^2$ 可以读作 “10 的平方”, $7^3$ 可以读作 “7 的立方”。

一个数可以看作这个数本身的一次方。例如, 5 就是 $5^1$。 指数 1 通常省略不写。

\lianxi
\begin{xiaotis}
\setcounter{cntxiaoti}{0}

\xiaoti{计算:$2^3$,\; $3^2$,\; $0.1^3$,\; $5^4$,\; $\left(\dfrac{2}{3}\right)^2$,\; $1.2^3$,\; $\left(1\dfrac{1}{2}\right)^3$,\; $9^1$ 。}

\xiaoti{计算:\begin{tblr}[t]{columns={4em, l, $$}}
    (+2)^1,  &  (+2)^2, &  (+2)^3, &  (+2)^4, &  (+2)^5; \\
    (-2)^1,  &  (-2)^2, &  (-2)^3, &  (-2)^4, &  (-2)^5 \juhao
\end{tblr}}

\end{xiaotis}

\lianxijiange


想一想,正数的 2 次幂、3 次幂、…… 是正数还是负数?
负数的 2 次幂、3 次幂、…… 是正数还是负数?有些什么规律?

\zhongdian{正数的任何次幂都是正数;负数的奇次幂是负数,负数的偶次幂是正数。}


\liti[0] 计算:
\begin{xiaoxiaotis}

    \begin{tblr}{columns={14em, l, colsep=0pt}}
        \xxt{$(-3)^4$;} & \xxt{$-3^4$;} \\
        \xxt{$3 \times 2^3$;} & \xxt{$(3 \times 2)^3$;} \\
        \xxt{$-2 \times 3^4$;} & \xxt{$(-2 \times 3)^4$;} \\
        \xxt{$8 \div 2^2$;} & \xxt{$(8 \div 2)^2$。}
    \end{tblr}

\setcounter{cntxiaoxiaoti}{0}
\jie \begin{tblr}[t]{columns={l, colsep=0pt}}
    \xxt{$(-3)^4 = 81$;} & \xxt{$-3^4 = -81$;} \\
    \xxt{$3 \times 2^3 = 3 \times 8 = 24$;} & \xxt{$(3 \times 2)^3 = 6^3 = 216$;} \\
    \xxt{$-2 \times 3^4 = -2 \times 81 = -162$;} & \xxt{$(-2 \times 3)^4 = (-6)^4 = 1296$;} \\
    \xxt{$8 \div 2^2 = 8 \div 4 = 2$;} & \xxt{$(8 \div 2)^2 = 4^2 = 16$。}
\end{tblr}

\end{xiaoxiaotis}


\zhuyi 乘方与乘除在一起的时候,要先算乘方,再算乘除。如果有括号,就先算括号里面的。

\lianxi

计算:

\begin{xiaoxiaotis}
\setcounter{cntxiaoxiaoti}{0}

    \begin{tblr}{columns={10em, l, colsep=0pt}}
        \xxt{$-8^2$;} & \xxt{$(-8)^2$;} & \xxt{$4 \times 2^2$;}\\
        \xxt{$(4 \times 2)^2$;} & \xxt{$-3 \times 2^3$;} & \xxt{$(-3 \times 2)^3$;}\\
        \xxt{$(6 \div 3)^2$;} & \xxt{$6 \div 3^2$。}
    \end{tblr}
\end{xiaoxiaotis}

