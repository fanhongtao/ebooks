\subsection{实系数方程虚根成对定理}\label{subsec:1-8}

我们知道,如果 $\Delta = b^2 - 4ac < 0$,那么实系数一元二次方程
$ ax^2 + bx + c = 0 $ 有一对虚数根,它们互为共轭虚数,即
$$ x = \dfrac{-b \pm \sqrt{4ac - b^2}\,i}{2a} \text{。} $$

一般地说,关于实系数一元 $n$ 次方程的虚数根,有下面的性质:

\begin{theorem} \label{theorem:xgcd}
    如果虚数 $a + b\,i$ 是实系数一元 $n$ 次方程 $f(x) = 0$ 的根,那么 $a - b\,i$ 也这个方程的根,并且它们的重数相等。
\end{theorem}

\zhengming 由 $a + b\,i$ 是实系数一元 $n$ 次方程 $f(x) = 0$ 的根,可知 $f(a + b\,i) = 0$。
我们先来证明 $a - b\,i$ 也是方程 $f(x) = 0$ 的根,为此只需证明 $f(a - b\,i) = 0$。

考虑多项式
\begin{align}
    g(x) &= [x - (a + b\,i)] [x - (a - b\,i)] \notag \\
         &= (x - a)^2 - (b\,i)^2 \notag \\
         &= x^2 - 2ax + (a^2 + b^2) \text{,} \label{eq:sxsfc-1}
\end{align}
这是一个实系数二次三项式。用 $g(x)$ 除 $f(x)$,设商式为 $q(x)$,那么余式
的次数不大于 $1$,可以表示为 $mx + n$ (其中 $m,\, n$ 为实数)。于是
\begin{equation}
    f(x) = g(x) \cdot q(x) + mx + n \text{。}
\end{equation}

把 $x = a + b\,i$ 代入上式,左边 $f(a + b\,i) = 0$,又由 (1) 式知
右边的 $g(a + b\,i) = 0$,从而右边的 $m(a + b\,i) + n = 0$,即
$$ am + n + bm\,i = 0 \text{。} $$

根据复数等于零的条件,得
$$ am + n = 0, \quad bm = 0 \text{。} $$
由于 $b \neq 0$ (否则 $a + b\,i$ 不是虚数),所以 $m = 0$,由此
$n = 0$。又由 (1) 式知 $g(a - b\,i) = 0$,所以由 (2) 式,
\begin{align*}
    f(a - b\,i) &= g(a - b\,i) \cdot q(a - b\,i) + m(a - b\,i) + n \\
        &= 0 \cdot q(a - b\,i) + 0 \cdot (a - b\,i) + 0 = 0,
\end{align*}
即 $a - b\,i$ 是方程 $f(x) = 0$ 的根。

现在再证明 $a + b\,i$ 与 $a - b\,i$ 的重数相等,由上面的证明可知
$m = 0$,$n = 0$,代入 (2) 式,得
$$ f(x) = g(x) \cdot q(x) \text{。} $$
这说明 $g(x)$ 整除 $f(x)$。因为 $f(x)$,$g(x)$ 的系数都是实数,非零
实系数多项式除以实系数多项式,商式仍然是实系数多项式,所以 $q(x)$ 的
系数也都是实数。如果 $a + b\,i$ 是方程 $f(x) = 0$ 的重根,那么它必然
也是方程 $q(x) = 0$ 的根,根据上面的证明,$a - b\,i$ 也必然是方程
$q(x) = 0$ 的根。这要 $a - b\,i$ 也是方程 $f(x) = 0$ 的重根。
设 $a + b\,i$ 与 $a - b\,i$ 分别是方程 $f(x) = 0$  的 $s$ 重根与
$t$ 重根,重复运用这个推理方法,可知 $s \leqslant t$;同理可证
$t \leqslant s$。所以 $s = t$。

由上面的定理可知,在实系数一元 $n$ 次方程中,虚数根总是成对出现的。


\liti 求方程 $2x^4 - 6x^3 + 21x^2 + 14x + 39 = 0$ 在复数集 $C$
中的解集,已知它的根中有一个是 $2 - 3\,i$。

\textbf{解法一:} 这是一个一元四次方程,在复数集 $C$ 中有且仅有
四个根。因为它的系数都是实数,且 $2 - 3\,i$ 是它的根,可知
$2 + 3\,i$ 也是它的根。

把 $2x^4 - 6x^3 + 21x^2 + 14x + 39$ 除以
$$ [x - (2 - 3\,i)] [x - (2 + 3\,i)] \text{,} $$
也就是除以 $x^2 - 4x + 13$,得商式 $2x^2 + 2x + 3$。
因此原方程可以化为
$$ [x - (2 - 3\,i)] [x - (2 + 3\,i)] (2x^2 + 2x + 3) = 0 \text{。} $$

解方程 $2x^2 + 2x + 3 = 0$,得 $x = \dfrac{-1 \pm \sqrt{5}\,i}{2}$,
所以原方程的解集是
$$ \left\{ 2 - 3\,i,\, 2 + 3\,i,\, \dfrac{-1 + \sqrt{5}\,i}{2},\, \dfrac{-1 - \sqrt{5}\,i}{2}\right\} \text{。} $$

\textbf{解法二:} 原方程有两个根 $2 - 3\,i,\, 2 + 3\,i$,设另外两个根为
$\alpha$,$\beta$,由根与系数的关系,有
$$\begin{cases}
    \alpha + \beta + (2 - 3\,i) + (2 + 3\,i) = 3, \\
    \alpha \cdot \beta \cdot (2 - 3\,i) \cdot (2 + 3\,i) = \dfrac{29}{2},
\end{cases}$$
即
$$\begin{cases}
    \alpha + \beta = -1, \\
    \alpha\beta = \dfrac{3}{2} \text{。}
\end{cases}$$
所以 $\alpha$,$\beta$ 是一元二次方程 $2x^2 + 2x + 3 = 0$ 的根。
解这个一元二次方程,得两个根 $x = \dfrac{-1 \pm \sqrt{5}\,i}{2}$。
从而原方程的解集是
$$ \left\{ 2 - 3\,i,\, 2 + 3\,i,\, \dfrac{-1 + \sqrt{5}\,i}{2},\, \dfrac{-1 - \sqrt{5}\,i}{2}\right\} \text{。} $$



\liti 求次数最低的实系数方程 $f(x) = 0$,已知它在复数集 $C$
中的解集含有 $i$,$-1 + i$,$0$ 这三个数。

\jie 根据\hyperref[theorem:xgcd]{实系数方程虚根成对定理},如果 $i$,$-1 + i$
是所求实系数方程 $f(x) = 0$ 的根,那么它们的共轭虚数 $-i$,$-1 - i$
也是这个方程的根,所以所求的实系数方程至少有五个根 $\pm i$,$-1 \pm i$,$0$,
也就是说,$f(x)$ 至少有五个一次因式 $x \mp i$,$x + 1 \mp i$,$x$。
把 $f(x)$ 写成这五个一次因式与一个常数 $a \; (a \in C \text{,且} a \neq 0)$ 的积
$$ f(x) = a (x - i) (x + i) (x + 1 - i) (x + 1 + i) x \text{。} $$

取 $a = 1$,那么,实系数一元五次方程
$$ (x - i) (x + i) (x + 1 - i) (x + 1 + i) x = 0 \text{,} $$
即
$$ x^5 + 2x^4 + 3x^3 + 2x^2 + 2x = 0 \text{,} $$
就是所求的方程。

\lianxi
\begin{xiaotis}

\xiaoti{已知方程 $3x^4 - 2x^3 + 10x^2 -2x + 7 = 0$ 的根中有一个是 $i$,
    求它在复数集 $C$ 中的解集。
}

\xiaoti{求次数最低的实系数方程 $f(x) = 0$,已知它在复数集 $C$ 中的解集含有下列数:}
\begin{xiaoxiaotis}

    \twoInLineXxt[16em]{$3 + 2\,i$;}{$-2,\; 1 - \,i$。}

\end{xiaoxiaotis}

\xiaoti{已知虚数 $-1 + \sqrt{2}\,i$ 是实系数方程 $x^3 + 3x^2 + ax + b = 0$ 的根,
    求 $a$,$b$ 的值以及这个方程在复数集 $C$ 中的解集。
}

\end{xiaotis}

