\subsection{组合}\label{subsec:2-4}

我们看下面的问题:

在北京、上海、广州三个民航站之间的直达航线,有多少种不同的飞机票价?

这个问题与第 \ref{subsec:2-2} 节中计算飞机票种数的问题不同。飞机票的种数与起点站、终点站有关,
从北京到上海和从上海到北京,飞机票是不同的,也就是与顺序有关;但飞机票价只与起点站和终点站之间的距离有关,
从北京到上海和从上海到北京,飞机票价是相同的,也就是与顺序无关。
因此,第 \ref{subsec:2-2} 节中计算飞机票种数的问题,是从三个不同的元素中任取两个,然后按照一定的顺序排列,
求一共有多少种不同的排列方法,这是排列问题;而本节这个问题,是从三个不同的元素中任取两个,不管怎样的顺序
并成一组,求一共有多少个不同的组,这就是要研究的组合问题。

一般地说,从 $n$ 个不同元素中,任取 $m \; (m \leqslant n)$ 个元素并成一组,
叫做从 $n$ 个不同元素中取出 $m$ 个元素的一个\textbf{组合}。

上面问题中要确定有几种不同的飞机票价,就是要求从 $3$ 个不同的元素中取出 $2$ 个元素的所有组合的个数。
因为上海到广州和广州到上海的飞机票价是相同的,所以这两站间的飞机票价就是从北京、上海、广州这三个
不同元素中取出上海、广州这两个元素的一个组合。

如果两个组合中的元素完全相同,不管元素的顺序如何,都是相同的组合;
只有当两个组合中的元素不完全相同时,才是不同的组合。例如,从 $a,\, b,\, c$
三个不同的元素中取出两个元素的所有组合有 $3$ 个,它们分别是:
$$ ab,\quad ac,\quad bc \text{。} $$
组合 $ab$ 与组合 $ba$ 是相同的组合,而组合 $ab$ 与组合 $ac$ 是不同的组合。

从排列和组合的定义可以知道,排列与元素的顺序有关,组合与顺序无关。
例如 $ab$ 与 $ba$ 是两个不同的排列,但它们却是同一个组合。

在实际问题中,有时需要写出某个组合问题的所有组合。例如,已知 $a,\, b,\, c,\, d$ 这 $4$ 个元素,
写出每次取出 $2$ 个元素的所有组合,可以先列出下图(图\ref{fig:2-7}):

\begin{figure}[htbp]
    \centering
    \begin{tikzpicture}[>=Stealth]
    \node at (0, 0) {$a$};
    \node at (1, 0) {$b$};
    \node at (2, 0) {$c$};
    \node at (3, 0) {$d$};
    \draw [->] (0, -0.2) arc(240:300:1);
    \draw [->] (0, -0.2) arc(240:300:2);
    \draw [->] (0, -0.2) arc(240:300:3);


    \node at (5, 0) {$b$};
    \node at (6, 0) {$c$};
    \node at (7, 0) {$d$};
    \draw [->] (5, -0.2) arc(240:300:1);
    \draw [->] (5, -0.2) arc(240:300:2);

    \node at (9, 0) {$c$};
    \node at (10, 0) {$d$};
    \draw [->] (9, -0.2) arc(240:300:1);
\end{tikzpicture}

    \caption{}\label{fig:2-7}
\end{figure}

如图\ref{fig:2-7} 所表示的,先把 $a$ 从左到右依次与 $b,\, c,\, d$ 组合,
再把 $b$ 依次与 $c,\, d$ 组合, 再把 $c$ 与 $d$ 组合,由此可以写出所有的组合:
$$ ab,\quad ac,\quad ad,\quad bc,\quad bd,\quad cd \text{。} $$

