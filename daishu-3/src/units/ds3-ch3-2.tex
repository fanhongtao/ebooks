\subsection{等可能性事件的概率}\label{subsec:3-2}

随机事件的概率,一般可以通过大量重复试验求得其近似值。但对于某些随机事件,
也可以不通过重复试验,而只通过对一次试验中可能出现的结果的分析来计算其概率。


例如,掷一枚均匀的硬币,它要么出现正面,要么出现反面,出现这两种结果的可能性是相等的。
因此,可以认为出现正面的概率是 $\dfrac{1}{2}$,出现反面的概率也是$\dfrac{1}{2}$。
这和大量重复试验的结果是一致的。有人做过掷一枚均匀硬币的大量重复试验,结果硬币出现正面
的频率总是接近于 $\dfrac{1}{2}$, 在它附近摆动。其中当掷币 $24000$ 次时,
硬币出现正面 $12012$ 次,其频率为 $0.5005$。


又如,有 $10$ 个型号相同的杯子,其中一等品 $6$ 个,二等品 $3$ 个,三等品 $1$ 个。
从中任取 $1$ 个,取到各个杯子的可能性是相等的。由于是从 $10$ 个杯子中任取 $1$ 个,
共有 $10$ 种等可能的结果。又由于其中有 $6$ 个一等品,从这 $10$ 个杯子中取到一等品
的结果有 $6$ 种。因此,可以认为取到一等品的概率是 $\dfrac{6}{10}$。
同理,可以认为取到二等品的概率是 $\dfrac{3}{10}$,取到三等品的概率是 $\dfrac{1}{10}$。
这和大量重复试验的结果也是一致的。

一般地,\textbf{如果一次试验中共有 $n$ 种等可能出现的结果,其中事件 $A$ 包含的结果有 $m$ 种,
那么事件 $A$ 的概率 $P(A)$ 是 $\dfrac{m}{n}$。}



\liti 先后抛掷两枚均匀的硬币,计算:\mylabel{li-ti-1}

(1) 两枚都出现正面的概率;

(2) 一枚出现正面、一枚出现反面的概率。

分析:抛掷一枚硬币,可能出现正面或反面这两种结果。因而先后抛掷两枚硬币可能出现的结果数,
可根据乘法原理得出。由于硬币是均匀的,所有结果出现的可能性都相等。
又在所有等可能的结果中,两枚都出现正面这一事件包含的结果数是可以知道的,从而可以求出这个事件的概率。
同样,一枚出现正面、一枚出现反面这一事件包含的结果数是可以知道的,从而也可求出这个事件的概率。

\jie 由乘法原理,先后抛掷两枚硬币可能出现的结果共有$2 \times 2 = 4$ 种(图 \ref{fig:3-1}),
且这 $4$ 种结果出现的可能性都相等。

\begin{figure}[htbp]
    \centering
    \begin{tikzpicture}[>=Stealth, scale=0.8, transform shape]
    \tikzset{
        pics/coin/.style args={#1/#2}{
        code = {
            \draw (0, 0) node {\Large #1} circle (0.4);
            \draw (0, -0.7) node {#2};
        }}}

    \draw (0, 0) pic {coin={正/1}};
    \draw (1, 0) pic {coin={正/2}};

    \draw (3, 0) pic {coin={正/1}};
    \draw (4, 0) pic {coin={反/2}};

    \draw (0, -2) pic {coin={反/1}};
    \draw (1, -2) pic {coin={正/2}};

    \draw (3, -2) pic {coin={反/1}};
    \draw (4, -2) pic {coin={反/2}};
\end{tikzpicture}

    \caption{}\label{fig:3-1}
\end{figure}


(1) 记“抛掷两枚硬币,都出现正面”为事件 $A$,那么在上面 $4$ 种结果中,事件 $A$ 包含的结果有 $1$ 种,
因此事件 $A$ 的概率
$$ P(A) = \dfrac{1}{4} \text{。} $$

答:两枚都出现正面的概率是 $\dfrac{1}{4}$ 。

(2) 记“抛掷两枚硬币,一枚出现正面、一枚出现反面” 为事件 $B$ 。
那么事件 $B$ 包含的结果有 $2$ 种, 因此事件 $B$ 的概率
$$ P(B) = \dfrac{2}{4} = \dfrac{1}{2} \text{。} $$

答:一枚出现正面、一枚出现反面的概率是 $\dfrac{1}{2}$ 。


\liti 在 $100$ 件产品中,有 $95$ 件合格品,$5$ 件次品。从中任取 $2$ 件,计算:

(1) $2$ 件都是合格品的概率;

(2) $2$ 件都是次品的概率;

(3) $1$ 件是合格品、$1$ 件是次品的概率。

分析:从 $100$ 件产品中任取 $2$ 件可能出现的结果数,就是从 $100$ 个元素中任取 $2$ 个的组合数。
由于是任意抽取,这些结果出现的可能性都相等。又由于在所有产品中有 $95$ 件合格品、$5$ 件次品,
取到 $2$ 件合格品的结果数,就是从 $95$ 个元素中任取 $2$ 个的组合数;
取到 $2$ 件次品的结果数,就是从 $5$ 个元素中任取 $2$ 个的组合数;
取到 $1$ 件合格品、$1$ 件次品的结果数,就是从 $95$ 个元素中任取 $1$ 个元素的组合数与从 $5$ 个元素中任取 $1$ 个元素的组合数的积,
从而可以分别得到所求各个事件的概率。

\jie (1)  从 $100$ 件产品中任取 $2$ 件,可能出现的结果共有 $C_{100}^2$种,
且这些结果出现的可能性都相等。又在 $C_{100}^2$ 种结果中,取到 $2$ 件合格品的结果有 $C_{95}^2 $种。
记“任取 $2$ 件,都是合格品”为事件 $A$,那么事件 $A$ 的概率
$$ P(A) = \dfrac{C_{95}^2}{C_{100}^2} = \dfrac{893}{990} \text{。} $$

答:$2$ 件都是合格品的概率为 $\dfrac{893}{990}$。


(2) 记 “任取 $2$ 件,都是次品” 为事件 $B$。由于在 $C_{100}^2$ 种结果中,
取到 $2$ 件次品的结果有 $C_5^2$ 种,事件 $B$ 的概率
$$ P(B) = \dfrac{C_5^2}{C_{100}^2} = \dfrac{1}{495} \text{。} $$

答:$2$ 件都是次品的概率为 $\dfrac{1}{495}$。


(3) 记 “任取 $2$ 件,$1$ 件是合格品、$1$ 件是次品” 为事件 $C$。
由于在 $C_{100}^2$ 种结果中,取到 $1$ 件合格品、$1$ 件次品的结果有
$C_{95}^1 \cdot C_5^1$ 种,事件 $C$ 的概率
$$ P(C) = \dfrac{C_{95}^1 \cdot C_5^1}{C_{100}^2} = \dfrac{19}{198} \text{。} $$

答:$1$ 件是合格品、$1$ 件是次品概率为 $\dfrac{19}{198}$。



\liti 某号码锁有 $6$ 个拨盘,每个拨盘上有从 $0$ 到 $9$ 共十个数字。
当 $6$ 个拨盘上的数字组成某一个六位数字号码(开锁号码)时,锁才能打开。
如果不知道开锁号码,试开一次就把锁打开的概率是多少?

分析:号码锁每个拨盘上的数字,从 $0$ 到 $9$ 共有十个。
$6$ 个拨盘上的各一个数字排在一起,就是一个六位数字号码。
根据乘法原理,这种号码共有 $10^6$ 个。
由于不知道开锁号码,试开时采用每一个号码的可能性都相等。
又开锁号码只有一个,从而可以求出试开一次就把锁打开的概率。

\jie 号码锁每个拨盘上的数字有 $10$ 种可能的取法。
根据乘法原理, $6$ 个拨盘上的数字组成的六位数字号码共有 $10^6$个。
又试开时采用每一个号码的可能性都相等,且开锁号码只有一个,
所以试开一次就把锁开的概率
$$ P = \dfrac{1}{10^6} \text{。} $$

答:试开一次就把锁打开的概率是 $\dfrac{1}{10^6}$。



\lianxi
\begin{xiaotis}

\xiaoti{( 口答)在 $40$ 根纤维中,有 $12$ 根的长超过 $30$ 毫米。
    从中任取 $1$ 根,取到长度超过 $30$ 毫米的纤维的概率是多少?}

\xiaoti{在 $10$ 支铅笔中,有 $8$ 支正品和 $2$ 支副品。从中任取 $2$ 支,
    恰好都取到正品的概率是多少?}

\xiaoti{对于第 \pageref{li-ti-1} 页 例 1 ,有人说,先后抛掷枚硬币,共出现
    “两枚都是正面”,“两枚都是反面”,“一枚正面、一枚反面” 等 $3$ 种结果,
    因此,“两枚都出现正面” 这一事件的概率是 $\dfrac{1}{3}$。这种说法错在哪里?
}

\end{xiaotis}

