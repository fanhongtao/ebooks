\subsection{二项式系数的性质}\label{subsec:2-8}

我们已经知道,$(a + b)^n$ 的展开式的二项式系数是
$$ C_n^0,\; C_n^1,\; C_n^2,\; \cdots,\; C_n^{n-1},\; C_n^n \text{。} $$

二项式系数有下列性质:

\textbf{1. 在二项展开式中,与首末两端“等距离”的两项的二项式系数相等。}

由已知公式
$$ C_n^m = C_n^{n-m} \text{,} $$
分别取 $m = 0,\, 1,\, \cdots,\, k,\, \cdots$,从而得

$C_n^0 = C_n^n,\; C_n^1 = C_n^{n-1},\; C_n^2 = C_n^{n-2},\; \cdots,\; C_n^k = C_n^{n-k},\; \cdots $。

\textbf{2. 如果二项式的幂指数是偶数,中间一项的二项式系数最大;
如果二项式的幂指数是奇数,中间两项的二项式系数相等并且最大。}

由于展开式各项的二项式系数顺次是
\begin{align*}
    & C_n^0 = 1,\; C_n^1 = n,\; C_n^2 = \dfrac{n(n-1)}{1 \cdot 2},\; \cdots, \\
    C_n^k =& \dfrac{n (n-1) (n-2) \cdots (n-k+1)}{1 \cdot 2 \cdot \cdots \cdot k},\; \cdots,\; C_n^n = 1 \text{。}
\end{align*}
其中,后一个二项式系数的分子是前一个二项式系数的分子乘以逐次减小 $1$ 的数(如,$n,\, n-1,\, n-2,\, \cdots$),
分母是乘以逐次增大 $1$ 的数(如 $1,\, 2,\, 3,\, \cdots$),因而,各项的二项式系数从开始起是逐渐增大,
又因为与首末两端“等距离”的两项的二项式系数相等,所以二项式系数增大到某一项时就逐渐减小,
且二项式系数最大的项必在中间。

当 $n$ 是偶数时,$n+1$ 是奇数,展开式共有 $n+1$ 项,所以展开式有中间一项,并且这一项的二项式系数最大。

当 $n$ 是奇数时,$n+1$ 是偶数,展开式共有 $n+1$ 项,所以有中间两项,这两项的二项式系数相等并且最大。


\liti 证明
$$ C_n^0 + C_n^1 + C_n^2 + \cdots  C_n^k + \cdots + C_n^n = 2^n \text{。} $$

\zhengming 运用 $(1 + x)^n$ 的展开式
$$ (1 + x)^n = C_n^0 + C_n^1 x + C_n^2 x^2 + \cdots + C_n^r x^r + \cdots + C_n^n x^n \text{,} $$
设 $x = 1$,则
$$ 2^n = C_n^0 + C_n^1 + C_n^2 + \cdots  C_n^r + \cdots + C_n^n \text{。} $$

例 1 说明,$(a + b)^n$ 的展开式的所有二项式系数的和等于 $2^n$。



\liti 证明在 $(a + b)^n$ 的展开式中,奇数项的二项式系数的和等于偶数项的二项式系数的和。

\zhengming 在展开式
$$ (a + b)^n = C_n^0 a^n + C_n^1 a^{n-1}b + C_n^2 a^{n-2}b^2 + \cdots + C_n^n b^n $$
中,令 $a = 1$,$b = -1$,则得
$$ (1 - 1)^n = C_n^0 - C_n^1 + C_n^2 - C_n^3 + \cdots + (-1)^n C_n^n \text{,} $$
就是
$$ 0 = (C_n^0 + C_n^2 + \cdots) - (C_n^1 + C_n^3 + \cdots) \text{,} $$

$\therefore \quad C_n^0 + C_n^2 + \cdots = C_n^1 + C_n^3 + \cdots$。

即 $(a + b)^n$ 的展开式中,奇数项的二项式系数的和等于偶数项的二项式系数的和。



\lianxi
\begin{xiaotis}

\xiaoti{求 $(1 - x)^{13}$ 的展开式中的含 $x$ 的奇次项系数的和。}

\xiaoti{证明 $C_n^0 + C_n^2 + C_n^4 + \cdots + C_n^n = 2^{n-1}$,($n$ 是偶数)。}

\xiaoti{求 $C_{11}^1 + C_{11}^3 + \cdots + C_{11}^{11}$。}

\end{xiaotis}

