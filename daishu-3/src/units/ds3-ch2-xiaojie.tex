\xiaojie

一、本章主要内容是排列、组合、二项式定理。


二、加法原理与乘法原理是两个基本原理,它们不仅是推导排列数公式、组合数公式的基础,
而且还常常需要直接运用它们去解某些问题。

这两个原理的区别在于一个与分类洧关,一个与分步有关。
如完成一件事有 $n$ 类办法,这 $n$ 类办法彼此之间是相互独立的,
不论哪一类办法中的哪一种方法都能单独完成这件事,求完成这件事的方法种数,就用加法原理;
如果完成一件事需分成 $n$ 个步骤,各步骤都不可缺少,需要依次完成所有步骤,才能完成这件事,
而完成每一个步骤各有若干方法,求完成这件事的方法种数就用乘法原理。


三、排列与组合是研究从一些不同的元素中,任取几个元素进行排列或并组有多少种方法的问题。
本章所研究的主要是不同元素不允许重复的排列或组合。排列与组合的区别要看问题是否与顺序有关。
与顺序有关就属于排列,与顺序无关就属于组合。

在求应用题中的排列数或组合数时,注意防止重复与遗漏。


四、排列与组合的主要公式有:

1. 排列数公式
\begin{align*}
    & P_n^m = n (n-1) (n-2) \cdots (n-m+1), \quad (m \leqslant n); \\
    & P_n^m = \dfrac{n!}{(n-m)!}, \quad (m \leqslant n); \\
    & P_n^n = n! = n (n-1) (n-2) \cdots 2 \cdot 1;
\end{align*}


2. 组合数公式 \quad $C_n^m = \dfrac{P_n^m}{P_m^m},\quad (m \leqslant n)$;


3. 组合数性质 \quad $\begin{aligned}[t]
    & C_n^m = C_n^{n-m}, \quad (m \leqslant n); \\
    & C_{n+1}^m = C_n^m + C_n^{m-1}, \quad (m \leqslant n) \text{。}
\end{aligned}$



五、二项式定理通过公式的形式,表示出二项式的幂展开在项数、系数、各项中的指数等方面的联系。


二项式定理为
$$ (a + b)^n = C_n^0 a^n + C_n^1 a^{n-1}b^1 + \cdots + C_n^r a^{n-r}b^r + \cdots + C_n^n b^n \text{,} $$
其中 $C_n^r$ 叫做第 $r+1$ 项的二项式系数,展开式的第 $r+1$ 项为
$$ T_{r+1} = C_n^r a^{n-r} b^r \text{。} $$

二项式系数的主要性质有:

l. 在二项展开式中,与首末两端“等距离”的两项的二项式系数相等。

2. 如果二项式的幂指数是偶数,中间一项的二项式系数最大;
如果二项式的幂指数是奇数,中间两项的二项式系数相等并且最大。

