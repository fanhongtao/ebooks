\subsection{排列}\label{subsec:2-2}

我们看下面的问题:

1. 北京、上海、广州三个民航站之间的直达航线,需要准备多少种不同的飞机票?

这个问题就是从北京、上海、广州三个民航站中, 每次取出两个站,按照起点站在前、终点站在后的顺序排列,
求一共有多少种不同的排法。

首先确定起点站,在三个站中,任选一个站为起点站,有 $3$ 种方法;
其次确定终点站,当选定起点站以后,终点站就只能在其余的两个站中去选,因此,有 $2$ 种方法。
那么根据乘法原理,在三个民航站中,每次取两个,按起点站在前、终点站在后的顺序排列的不同方法共有
$$ 3 \times 2 = 6 $$
种。也就是说,需要备如下 $6$ 种不同的飞机票:

\begin{figure}[htbp]
    \centering
    \begin{tikzpicture}[>=Stealth]
    \tikzset{
        pics/tickets/.style args={#1/#2/#3}{
        code = {
            \node at (0, 0.35) {#1};
            \node at (2, 0.7) {#2};
            \node at (2, 0) {#3};
            \draw (1.5, 0.7) -- (0.5, 0.35) -- (1.5, 0);

            \node at (5, 0) {#1};
            \node at (7, 0) {#3};
            \draw (5.5, 0) -- (6.5, 0);

            \node at (5, 0.7) {#1};
            \node at (7, 0.7) {#2};
            \draw (5.5, 0.7) -- (6.5, 0.7);
        }}}

    \draw (0, 0) pic[transform shape] {tickets=广州/北京/上海};
    \draw (0, 1.5) pic[transform shape] {tickets=上海/北京/广州};
    \draw (0, 3) pic[transform shape] {tickets=北京/上海/广州};
    \node at (0, 4.3) {起点站};
    \node at (2, 4.3) {终点站};
    \node at (6, 4.3) {飞机票};
\end{tikzpicture}

\end{figure}


2. 由数字 $1,\, 2,\, 3,\, 4$ 可以组成多少个没有重复数字的三位数?

这个问题就是从 $1,\, 2,\, 3,\, 4$ 这四个数字中,每次取出三个,按照百位、十位、个位的顺序排列起来,求一共有多少种不同的排法。

第一步,先确定百位上的数字, 在 $1,\, 2,\, 3,\, 4$ 这四个数字中任取一个,有 $4$ 种方法;

第二步,确定十位上的数字,当百位上的数字确定以后,十位上的数字只能从余下的三个数字中去取,有 $3$ 种方法;

第三步,确定个位上的数字,当百位、十位上的数字都确定以后,个位上的数字只能从余下的两个数字中去取,有 $2$ 种方法。

根据乘法原理,从四个不同的数字中,每次取出三个排成一个三位数的方法共有
$$ 4 \times 3 \times 2 = 24 $$
种。也就是说,可以排成 $24$ 个不同的三位数。具体排法如下:

\begin{figure}[htbp]
    \centering
    \begin{tikzpicture}[>=Stealth,
    every text node part/.style={font=\tiny},
    transform shape,
    scale=1.6]
    \tikzset{
        pics/boxes/.style n args={2}{
            code = {
                \draw (0, 0) rectangle (0.9, 0.3);
                \draw (0.3, 0) -- (0.3, 0.3);
                \draw (0.6, 0) -- (0.6, 0.3);
                \def\first{\the\numexpr(#1)/100\relax};
                \draw (0.15, 0.15) node{\first};
                \ifnum #2>1
                    \draw (0.45, 0.15) node{\the\numexpr(#1)/10 - \first*10\relax};
                    \ifnum #2>2
                        \draw (0.75, 0.15) node{\the\numexpr(#1) - \numexpr(#1)/10*10\relax};
                    \fi
                \fi
            }
        }
    }
    \tikzset{
        pics/numbers/.style n args={6}{
            code = {
                \draw (0, 0) pic {boxes={#1}{1}};

                \draw[decorate,decoration={brace,mirror,amplitude=0.2cm}] (1.2, 1.2) -- (1.2, -0.9);
                \draw (1.3,  1) pic {boxes={#1}{2}};
                \draw (1.3,  0) pic {boxes={#3}{2}};
                \draw (1.3, -1) pic {boxes={#5}{2}};

                \draw[decorate,decoration={brace,mirror,amplitude=0.1cm}] (2.4, 1.5) -- (2.4, 0.8);
                \draw (2.5, 1.25) pic {boxes={#1}{3}};
                \draw (2.5, 0.75) pic {boxes={#2}{3}};

                \draw[decorate,decoration={brace,mirror,amplitude=0.1cm}] (2.4, 0.5) -- (2.4, -0.2);
                \draw (2.5, 0.25) pic {boxes={#3}{3}};
                \draw (2.5, -0.25) pic {boxes={#4}{3}};

                \draw[decorate,decoration={brace,mirror,amplitude=0.1cm}] (2.4, -0.5) -- (2.4, -1.2);
                \draw (2.5, -0.75) pic {boxes={#5}{3}};
                \draw (2.5, -1.25) pic {boxes={#6}{3}};
            }
        }
    }

    \draw (0, 0) pic {numbers={123}{124}{132}{134}{142}{143}};
    \draw (4, 0) pic {numbers={213}{214}{231}{234}{241}{243}};

    \draw (0, -3) pic {numbers={312}{314}{321}{324}{341}{342}};
    \draw (4, -3) pic {numbers={412}{413}{421}{423}{431}{432}};
\end{tikzpicture}

\end{figure}

我们把被取的对象(如上面问题中的民航站、数字)叫做\textbf{元素}。
上面第一个问题,就是从 $3$ 个不同的元素中,任取 $2$ 个,然后按一定的顺序排成一列,求一共有多少种不同的排法;
    第二个问题,就是从 $4$ 个不同的元素中,任取 $3$ 个,然后按一定的顺序排成一列,求一共有多少种不同的排法。

一般地说, 从 $n$ 个不同元素中,任取 $m \; (m \leqslant n)$ 个元素(本章只研究被取出的元素各不相同的情况),
按照一定的顺序排成一列,叫做从 $n$ 个不同元素中取出 $m$ 个元素的一个\textbf{排列}。

从排列的定义知道,如果两个排列相同,不仅这两个排列的元素必须完全相同,而且排列的顺序也必须完全相同。
如果所取的元素不完全相同,例如问题 1 中的飞机票“上海——北京”和“上海——广州”,它们就是两个不同的排列。
即使所取的元素完全相同,但排列顺序不同,也不是相同的排列。如问題 2 中的三位数“213”和“231”,
虽然它们的元素相同,但排列顺序不同,也是两个不同的排列。

在实际问题中,有时需要写出某个排列问题的所有排列。例如,已知 $a,\, b,\, c,\, d$ 这 $4$ 个元素,
写出每次取 $3$ 个元素的所有排列,可以先列出下图(见图 \ref{fig:2-2}):

\begin{figure}[htbp]
    \centering
    \begin{tikzpicture}[>=Stealth]
    \ExplSyntaxOn
    \tikzset{
       pics/letters/.style~n~args={3}{
            code = {
                \node at (0, 0) {$(#1)$};

                \draw (0, 0.3) -- (-0.8, 1.2) node[above] {$\clist_item:nn{#2}{1}$};
                \draw (0, 0.3) -- (0, 1.2) node[above] {$\clist_item:nn{#2}{2}$};
                \draw (0, 0.3) -- (0.8, 1.2) node[above] {$\clist_item:nn{#2}{3}$};

                \draw (-0.8, 1.8) -- (-1.0, 2.7) node[above] {$\clist_item:nn{#3}{1}$};
                \draw (-0.8, 1.8) -- (-0.6, 2.7) node[above] {$\clist_item:nn{#3}{2}$};

                \draw (0, 1.8) -- (-0.2, 2.7) node[above] {$\clist_item:nn{#3}{3}$};
                \draw (0, 1.8) -- (0.2, 2.7) node[above] {$\clist_item:nn{#3}{4}$};

                \draw (0.8, 1.8) -- (0.6, 2.7) node[above] {$\clist_item:nn{#3}{5}$};
                \draw (0.8, 1.8) -- (1.0, 2.7) node[above] {$\clist_item:nn{#3}{6}$};
            }}}
    \ExplSyntaxOff

    \draw (0,0) pic {letters={a}{b,c,d}{c,d,b,d,b,c}};
    \draw (3,0) pic {letters={b}{a,c,d}{c,d,a,d,a,c}};
    \draw (6,0) pic {letters={c}{a,b,d}{b,d,a,d,a,b}};
    \draw (9,0) pic {letters={d}{a,b,c}{b,c,a,c,a,b}};
\end{tikzpicture}

    \caption{}\label{fig:2-2}
\end{figure}

由此可以写出所有的排列:
\begin{table}[H]
    \centering
    \begin{tabular}{*{4}{w{c}{3em}}}
        $abc$ & $bac$ & $cab$ & $dab$ \\
        $abd$ & $bad$ & $cad$ & $dac$ \\
        $acb$ & $bca$ & $cba$ & $dba$ \\
        $acd$ & $bcd$ & $cbd$ & $dbc$ \\
        $adb$ & $bda$ & $cda$ & $dca$ \\
        $adc$ & $bdc$ & $cdb$ & $dcb$
    \end{tabular}
\end{table}




