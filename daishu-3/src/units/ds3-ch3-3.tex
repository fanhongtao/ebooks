\subsection{互斥事件\footnotemark 有一个发生的概率}\label{subsec:3-3}
\footnotetext{有的书上也称为\textbf{互不相容事件}。}


在 $10$ 个乒乓球中,有 $7$ 个一等品,$2$ 个二等品,$1$ 个三等品。
我们把从中任取一个,取出一等品叫做事件 $A$, 取出二等品叫做事件 $B$,取出三等品叫做事件 $C$。
我们看到,如果取出的乒乓球是一等品,即事件 $A$发生,那么事件 $B$ 就不发生;
如果取出的是二等品, 即事件 $B$ 发生,那么事件 $A$ 就不发生。
也就是说,事件 $A,\, B$ 不可能同时发生。
这种不可能同时发生的两个事件叫做\textbf{互斥事件}。
同理,事件 $B,\, C$ 是互斥事件,事件 $A,\, C$ 是互斥事件。
换句话说, 事件 $A,\, B,\, C$ 中,任何两个都是互斥事件。
这时我们说事件 $A,\, B,\, C$ 彼此互斥。一般地,
如果事件 $A_1,\, A_2,\, \cdots,\, A_n$ 中任何两个都是互斥事件,
那么就说事件 $A_1,\, A_2,\, \cdots,\, A_n$ \textbf{彼此互斥。}


在上面的问题里,因为是任取一个,共有 $10$ 种等可能的取法,
其中得到一等品,二等品,三等品的取法分别有 $7$ 种,$2$ 种,$1$ 种,
因此,$P(A) = \dfrac{7}{10}$,$P(B) = \dfrac{2}{10}$,$P(C) = \dfrac{1}{10}$。


现在问:“任取一个乒乓球,取出一等品或二等品” 这一事件的概率是多少?
这一事件,我们记作“$A + B$”。因为不论取出一等品还是二等品,都表示这个事件发生,
而得到一等品或二等品的取法共有 $7 + 2$ 种,所以取出一等品或二等品的概率
$P(A+B) = \dfrac{7+2}{10}$。由 $\dfrac{7+2}{10} = \dfrac{7}{10} + \dfrac{2}{10}$,
我们看到
\begin{align}
    \boxed{P(A + B) = P(A) + P(B) \text{。}}  \tag{1} \label{eq:hu-chi}
\end{align}
它告诉我们:\textbf{如果事件 $A,\, B$ 互斥,那么事件“$A + B$” 发生
(即 $A,\, B$ 中有一个发生)的概率,等于事件 $A,\, B$ 分别发生的概率的和。}


一般地,\textbf{如果事件 $A_1,\, A_2,\, \cdots,\, A_n$ 彼此互斥,那么事件 “$A_1 + A_2 + \cdots + A_n$”发生
〈即 $A_1,\, A_2,\, \cdots,\, A_n$ 中有一个发生)的概率,等于这 $n$ 个事件分别发生的概率的和,} 即
\begin{align}
    P(A_1 + A_2 + \cdots + A_n) = P(A_1) + P(A_2) + \cdots + P(A_n) \text{。}  \tag{$1'$} \label{eq:hu-chi-ex}
\end{align}


\liti 某地区的年降水量, 在 $100 \sim 150$\footnotemark 毫米范围内的概率是 $0.12$,
在 $150 \sim 200$ 毫米范围内的概率是 $0.25$,
在 $200 \sim 250$ 毫米范围内的概率是 $0.16$,
在 $250 \sim 300$ 毫米范围内的概率是 $0.14$。
计算年降水量在 $100 \sim 200$ 毫米范围内的概率与在 $150 \sim 300$ 毫米范围内的概率。
\footnotetext{在本章内, $a \sim b$ 表示大于或等于 $a$ 而小干 $b$ 的一个实数范围。}


\jie 记这个地区的年降水量在 $100 \sim 150$ 毫米,$150 \sim 200$ 毫米,$200 \sim 250$ 毫米,
$250 \sim 300$ 毫米范围内分别为事件 $A,\, B,\, C,\, D$。这四个事件是彼此互斥的。
根据公式 \eqref{eq:hu-chi-ex} 年降水量在 $100 \sim 200$ 毫米范围内的概率是
$$ P(A + B) = P(A) + P(B) = 0.12 + 0.25 = 0.37 \text{;} $$
年降水量在 $150 \sim 300$ 毫米范围内的概率是
$$ P(B + C + D) = P(B) + P(C) + P(D) = 0.25 + 0.16 + 0.14 = 0.55 \text{。} $$

答:年降水量在 $100 \sim 200$ 毫米范围内的概率为 $0.37$,
在 $150 \sim 300$ 毫米范围内的概率为 $0.55$。


\liti 在 $20$ 件产品中,有 $15$ 件一级品,$5$ 件二级品。从中任取 $3$ 件,
其中至少有 $1$ 件为二级品的概率是多少?

\jie 记从 $20$ 件产品中任取 $3$ 件,其中
恰有 $1$ 件二级品为事件 $A_1$,
恰有 $2$ 件二级品为事件 $A_2$,
$3$ 件全是二级品为事件 $A_3$。
这样,事件 $A_1,\, A_2,\, A_3$ 的概率分别是
\begin{align*}
    P(A_1) &= \dfrac{C_5^1 \cdot C_{15}^2}{C_{20}^3} = \dfrac{105}{228} ; \\
    P(A_2) &= \dfrac{C_5^2 \cdot C_{15}^1}{C_{20}^3} - \dfrac{30}{228} ; \\
    P(A_3) &= \dfrac{C_5^3}{C_{20}^3} = \dfrac{2}{228} \text{。}
\end{align*}

根据题意,事件 $A_1,\, A_2,\, A_3$ 彼此互斥。由公式 \eqref{eq:hu-chi-ex},$3$ 件产品中至少有 $1$ 件为二级品的概率是
$$ P(A_1 + A_2 + A_3) = P(A_1) + P(A_2) + P(A_3) = \dfrac{105}{228} + \dfrac{30}{228} + \dfrac{2}{228} = \dfrac{137}{228} \text{。} $$

答:其中至少有一件为二级品的概率是 $\dfrac{137}{228}$。


在 例 2 中,从 $20$ 件产品中任取 $3$ 件,或者都是一级品,或者不都是一级品(即其中至少有一件是二级品),
这两个互斥事件必有一个发生。 这种其中必有一个发生的两个互斥事件叫做\textbf{对立事件}。

一个事件 $A$ 的对立事件通常记作 $\buji{A}$。互根据对立事件的意义,$A + \buji{A}$ 是一个必然事件,
它的概率等于 $1$。又由于 $A$ 与 $\buji{A}$ 互斥,我们得到 \jiange
\begin{align}
    \boxed{P(A) + P(\buji{A}) = P(A + \buji{A}) = 1 \text{。}} \tag{2} \label{eq:dui-li}
\end{align}
这就是说,\textbf{两个对立事件的概率的和等于 $1$。}

从公式 \eqref{eq:dui-li} 还可得到
\begin{align}
    P(\buji{A}) = 1 - P(A) \text{。} \tag{$2'$} \label{eq:dui-li-ex}
\end{align}

运用公式 \eqref{eq:dui-li-ex} 计算事件的概率,有时比较简便。如 例 2 还可以这样来解:

从 $20$ 件产品中任取 $3$ 件,$3$ 件全是一级品(记作事件 $A$)的概率:
$$ P(A) = \dfrac{C_{15}^3}{C_{20}^3} = \dfrac{91}{228} \text{,} $$
由于“任取 $3$ 件,至少有 $1$ 件为二级品” 是事件 $A$ 的对立事件 $\buji{A}$,
根据公式 \eqref{eq:dui-li-ex},
$$ P(\buji{A}) = 1 - P(A) = 1 - \dfrac{91}{228} = \dfrac{137}{228} \text{。} $$



\lianxi
\begin{xiaotis}

\xiaoti{判别下列每对事件是不是互斥事件,如果是,再判别它们是不是对立事件。\\
    从一堆产品(其中正品与次品都多于 $2$ 个)中任取 $2$ 件,其中:}
\begin{xiaoxiaotis}

    \xiaoxiaoti{恰有 $1$ 件次品和恰有 $2$ 件次品;}

    \xiaoxiaoti{至少有 $1$ 件次品和全是次品;}

    \xiaoxiaoti{至少有 $1$ 件正品和至少有 $1$ 件次品;}

    \xiaoxiaoti{至少有 $1$ 件次品和全是正品。}

\end{xiaoxiaotis}


\xiaoti{在某一时期内,一条河流某处的年最高水位在各个范围内的概率如下:\\
    \begin{tabular}{|w{c}{7em}|*{6}{w{c}{5em}|}}
        \hline
        年最高水位 & 低于 $10$ 米 & $10 \sim 12$ 米 & $12 \sim 14$ 米 & $14 \sim 16$ 米 & 不低于 $16$ 米 \\ \hline
        概率       & $0.1$       & $0.28$          & $0.38$          & $0.16$          & $0.08$        \\ \hline
    \end{tabular} \\
    计算在同一时期内,河流这一处的年最高水位在下列范围内的概率:
    (1) $10 \sim 16$ 米;(2) 低于 $12$ 米;(3) 不低于 $14$ 米。
}

\end{xiaotis}

