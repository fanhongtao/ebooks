\subsection{综合除法}\label{subsec:1-2}

我们在初中已经学过实系数多项式的加、减、乘、除等运算。复系数多项式同样有这些运算。
一元多项式相加(包括相减)、相乘的结果仍是一元多项式,并且加乘运算满足交换律、结合律
以及乘法对加法的分配律。

一个一元多项式除以另一个一元多项式, 并不是总能整除,当被除式 $f(x)$ 除以
除式 $g(x)$(不是零多项式),得商式 $q(x)$ 及余式 $r(x)$ 时,就有下列等式:
$$ f(x) = g(x) q(x) + r(x) \text{,} $$
\textbf{其中 $r(x)$ 的次数小于 $g(x)$ 的次数,或者 $r(x)$ 是零多项式。}
当 $r(x)$ 是零多项式时,就是 $f(x)$ 能被 $g(x)$ 整除。

一个一元多项式除以一个一元多项式,有一种简便的计算方法 —— 综合除法。

先用一般的除法来计算 $a_3x^3 + a_2x^2 + a_1x + a_0$ 除以 $x - b$:
$$
\begin{array}{r@{\;}r@{}rrrr}
    & & \multicolumn{3}{l}{a_3x^2 + (a_2 + a_3b)x + [a_1 + (a_2 + a_3b)b]} \\
    \cline{3-5}
    x-b & \vstretch{1.3}{\big)}\mkern-7.3mu & \; a_3x^3 + a_2x^2 & + a_1x & + a_0 \\
    & & a_3x^3 - a_3bx^2 & \\
    \cline{3-5}
    & & (a_2 + a_3b)x^2 & + a_1x & \\
    & & (a_2 + a_3b)x^2 & - (a_2 + a_3b)bx & \\
    \cline{3-5}
    & & & [a_1 + (a_2 + a_3b)b]x & + a_0  \\
    & & & [a_1 + (a_2 + a_3b)b]x & - [a_1 + (a_2 + a_3b)b]b \\
    \cline{4-5}
    & & & & a_0 + [a_1 + (a_2 + a_3b)b]b
\end{array}
$$

这里所得的商式是 $a_3x^2 + (a_2 + a_3b)x + [a_1 + (a_2 + a_3b)b]$;余式是 $a_0 + [a_1 + (a_2 + a_3b)b]b$,
它不含 $x$,所以它是一个常数,下面把它叫做余数。

商式中各项的系数及余数分别是
\begin{align*}
    & a_3, \qquad a_2 + a_3b, \qquad a_1 + (a_2 + a_3b)b ; \\
    & a_0 + [a_1 + (a_2 + a_3b)b]b \text{。}
\end{align*}
其中第一个数就是被除式中第一项的系数,把这个数乘以 $b$ 再加上被除式中下一项的系数就得第二个数,
依此类推,最后得到余数。

因此,上面的除法可以用下面的简便算式来进行:
$$
\renewcommand\arraystretch{1.2}
\begin{array}{*{3}{c@{\hspace{0.8cm}}}c|l}
    a_3 & a_2 & a_1 & a_0 & b\\
        & a_3b & (a_2 + a_3b)b & [a_1 + (a_2 + a_3b)b]b & \\
    \cline{1-4}
    a_3 & a_2 + a_3b & a_1 + (a_2 + a_3b)b & \multicolumn{1}{|r}{ a_0 + [a_1 + (a_2 + a_3b)b]b } & \\
    \cline{4-4}
\end{array}
$$
这里,第一行是被除式按降幂排列时各项的系数,如果有缺项,必须用零补足。
移下第一个系数,乘以 $b$,加上第二个系数,依次进行,算得的第三行就是商式
各项的系数及余数。用这种算式进行的除法叫做\textbf{综合除法}。

被除式的次数不是三次时,综合除法同样适用。


\liti 用综合除法计算:
\begin{xiaoxiaotis}

    \xiaoxiaoti{$(x^3 + 8x^2 - 2x - 14) \div (x + 1)$;}

    \xiaoxiaoti{$(2x^4 + 5x^3 - 24x^2 + 15) \div (x - 2)$。}

\end{xiaoxiaotis}

\jie (1) $x+1$ 就是 $x - (-1)$。
$$
\renewcommand\arraystretch{1.2}
\begin{array}{*{3}{c@{\hspace{0.8cm}}}c|l}
    1 & +8 & -2 & -14 & -1\\
      & -1 & -7 &  +9 & \\
    \cline{1-4}
    1 & +7 & -9 & \multicolumn{1}{|r}{ -5 } & \\
    \cline{4-4}
\end{array}
$$

$\therefore$\quad 商式是 $x^2 + 7x - 9$,余式是 $-5$。


(2) 被除式缺一次项,用 $0$ 补足,得
$$
\renewcommand\arraystretch{1.2}
\begin{array}{*{4}{c@{\hspace{0.8cm}}}c|l}
    2 & +5 & -24 &  +0 & +15 & 2 \\
      & +4 & +18 & -12 & -24 &   \\
    \cline{1-5}
    2 & +9 &  -6 & -12 & \multicolumn{1}{|r}{ -9 } & \\
    \cline{5-5}
\end{array}
$$

$\therefore$\quad 商式是 $2x^3 + 9x^2 - 6x - 12$,余式是 $-9$。


\liti 用综合除法计算下列各式,并把结果写成 “$f(x) = g(x)q(x) + r(x)$” 的形式:
\begin{xiaoxiaotis}

    \xiaoxiaoti{$(4x^4 - 7x^2 - 7x - 5) \div \left( x - \dfrac{3}{2} \right)$;}

    \xiaoxiaoti{$(6x^4 - 5x^3 - 3x^2 -x + 4) \div (2x + 1)$。}

\end{xiaoxiaotis}

\jie (1) \quad $
\renewcommand\arraystretch{1.2}
\begin{array}[t]{*{4}{c@{\hspace{0.8cm}}}c|l}
    4 & +0 & -7 & -7 & -5 & \dfrac{3}{2} \\
      & +6 & +9 & +3 & -6 &   \\
    \cline{1-5}
    4 & +6 & +2 & -4 & \multicolumn{1}{|r}{ -11 } & \\
    \cline{5-5}
\end{array}
$

$\therefore$\quad $4x^4 - 7x^2 - 7x - 5 = \left( x - \dfrac{3}{2} \right) (4x^3 + 6x^2 + 2x - 4) - 11$ 。


(2)  $2x+1$ 就是 $2 \left( x + \dfrac{1}{2} \right)$,先将 $6x^4 - 5x^3 - 3x^2 - x + 4$ 除以 $x + \dfrac{1}{2}$。
$$
%\renewcommand\arraystretch{1.2}
\begin{array}[t]{*{4}{c@{\hspace{0.8cm}}}c|l}
    6 & -5 & -3 &            -1 &            +4 & -\dfrac{1}{2} \\[0.6em]
      & -3 & +4 & -\dfrac{1}{2} & +\dfrac{3}{4} &   \\[0.8em]
    \cline{1-5}
    6 & -8 & +1 & -\dfrac{3}{2} & \multicolumn{1}{|r}{ \rule[-1em]{0pt}{3em} +\dfrac{19}{4} } & \\
    \cline{5-5}
\end{array}
$$

$\therefore \quad \begin{aligned}[t]
        & 6x^4 - 5x^3 - 3x^2 -x + 4 \\
    ={} & \left( x + \dfrac{1}{2} \right) \left( 6x^3 - 8x^2 + x - \dfrac{3}{2} \right) + \dfrac{19}{4} \\
    ={} & 2 \left( x + \dfrac{1}{2} \right) \cdot \dfrac{1}{2} \left( 6x^3 - 8x^2 + x - \dfrac{3}{2} \right) + \dfrac{19}{4} \\
    ={} & (2x + 1) \left( 3x^3 - 4x^2 + \dfrac{x}{2} - \dfrac{3}{4} \right) + \dfrac{19}{4} \text{。}
\end{aligned}$

由例 2 的第 (2) 小题可知,$f(x)$ 除以一般的一元一次式 $px \pm q$,也可利用综合除法:
先将 $f(x)$ 除以 $x \pm \dfrac{q}{p}$,所得的商除以 $p$ 就是所求的商式,所得的余数就是所求的余数。



\lianxi
\begin{xiaotis}

用综合除法计算(第 $1 \sim 3$ 题):

\xiaoti{$(x^3 + 6x^2 - 11x - 14) \div (x - 3)$。}

\xiaoti{$(x^5 - 4x^3 - 8) \div (x - 2)$。}

\xiaoti{$(3x^4 + 7x^3 - 15x - 20) \div (x + 2)$。}

用综合除法计算下列各式,并把所得的结果写成 “$f(x) = g(x)q(x) + r(x)$” 的形式(第 $4 \sim 6$ 题):

\xiaoti{$(x^6 + 1) \div (x + 1)$。}

\xiaoti{$(27x^3 - 10) \div (3x - 2)$。}

\xiaoti{$(20x^5 + 9x^4 - 8x^3 + 12x^2 -35x - 12) \div (5x + 6)$。}

\end{xiaotis}


