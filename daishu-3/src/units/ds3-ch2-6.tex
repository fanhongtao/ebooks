\subsection{组合数的两个性质}\label{subsec:2-6}

\begin{theorem}\label{theorem:zhs-1}
\end{theorem}
\begin{center}
    \framebox{\begin{minipage}{12em}
        \begin{gather*}
            C_n^m = C_n^{n-m} \text{。}
        \end{gather*}
    \end{minipage}}
\end{center}

\zhengming $\begin{aligned}[t]
    &\because \quad C_n^m = \dfrac{n!}{m! (n-m)!} \text{,} \\
    &C_n^{n-m} = \dfrac{n!}{(n-m)! [n - (n-m)]!} = \dfrac{n!}{m! (n-m)!} \text{,}
\end{aligned}$

$\therefore \quad C_n^m = C_n^{n-m}$。

这个性质也可以根据组合的定义得出。从 $n$ 个不同元素中取出 $m$ 个元素后,剩下 $n-m$ 个元素,
也就是说,从 $n$ 个不同元素中取出 $m$ 个元素的每一个组合,都对应着从 $n$ 个不同元素中取出
$n-m$ 个元素的唯一的一个组合;反过来也是一样。因此,从 $n$ 个不同元素中取出 $m$ 个元素的
组合数 $C_n^m$,等于从 $n$ 个不同元素中取出 $n-m$ 个元素的组合数 $C_n^{n-m}$,即
$$ C_n^m = C_n^{n-m} \text{。} $$

当 $m > \dfrac{n}{2}$ 时,通常不直接计算 $C_n^m$,而是改为计算 $C_n^{n-m}$,这样比较简便。
例如,$C_9^7$ 可以这样计算:
$$ C_9^7 = C_9^{9-7} = C_9^2 = \dfrac{9 \times 8}{2!} = 36 \text{。} $$

\textbf{注意} \quad 为了使这个公式在 $n=m$ 时也成立,我们规定
$$ C_n^0 = 1 \text{。} $$


\begin{theorem}\label{theorem:zhs-2}
\end{theorem}
\begin{center}
    \framebox{\begin{minipage}{12em}
        \begin{gather*}
            C_{n+1}^m = C_n^m + C_n^{m-1} \text{。}
        \end{gather*}
    \end{minipage}}
\end{center}

\zhengming \quad $\begin{aligned}[t]
    C_n^m + C_n^{m-1} &= \dfrac{n!}{m! (n-m)!} + \dfrac{n!}{(m-1)! [n - (m-1)]!} \\
        &= \dfrac{n! (n-m+1) + n! m}{m! (n-m+1)!} \\
        &= \dfrac{(n-m+1+m) n!}{m! (n+1-m)!} \\
        &= \dfrac{(n+1)!}{m! [(n+1) - m]!} \\
        &= C_{n+1}^m \text{,}
\end{aligned}$

$\therefore \quad C_{n+1}^m = C_n^m + C_n^{m-1}$。


这个性质也可以根据组合的定义和加法原理得出。
从 $a_1,\, a_2,\, \cdots,\, a_{n+1}$ 这 $n+1$ 个不同的元素中取出 $m$ 个的组合数是 $C_{n+1}^m$,
这些组合可以分成两类,一类含有 $a_1$,一类不含 $a_1$。
含有 $a_1$ 的组合是从 $a_2,\, a_3,\, \cdots,\, a_{n+1}$ 这 $n$ 个元素中取出 $m-1$ 个元素与 $a_1$ 组成的,共有 $C_n^{m-1}$ 个;
不含 $a_1$ 的组合是从 $a_2,\, a_3,\, \cdots,\, a_{n+1}$ 这 $n$ 个元素中取出 $m$ 个元素组成的,共有 $C_n^m$ 个。
根据加法原理,得
$$ C_{n+1}^m = C_n^{m-1} + C_n^m \text{。} $$


\liti 计算 $C_{200}^{198}$ 及 $C_{99}^3 + C_{99}^2$。

解:由 \nameref{theorem:zhs-1},得
$$ C_{200}^{198} = C_{200}^2 = \dfrac{200 \times 199}{2 \times 1} = 19900 ; $$
由 \nameref{theorem:zhs-2},得
$$ C_{99}^3 + C_{99}^2 = C_{100}^3 = \dfrac{100 \times 99 \times 98}{3 \times 2 \times 1} = 161700 \text{。} $$



\liti 平面内有 $12$ 个点,任何 $3$ 点不在同一直线上,以每 $3$ 点为顶点画一个三角形,一共可画多少个三角形?

\jie 以平面内 $12$ 个点中的每 $3$ 个点为顶点画三角形,可画的三角形的个数,
就是从 $12$ 个不同的元素中取出 $3$ 个元素的组合数,即
$$ C_{12}^3 = \dfrac{12 \times 11 \times 10}{3 \times 2 \times 1} = 220 \text{。} $$

答:一共可画 $220$ 个三角形。


\liti 有 $13$ 个队参加篮球赛,比赛时先分成两组,第一组 $7$ 个队,第二组 $6$ 个队。
各组都进行单循环赛(即每队都要与本组其他各队比赛一场),然后由各组的前两名共 $4$ 个队
进行单循环赛决定冠军、亚军。共需要比赛多少场?

\jie 根据题意,
第一组是 $7$ 个队,单循环赛的比赛场数是 $C_7^2$,
第二组   $6$ 个队,单循环赛的比赛场数是 $C_6^2$;
各组的前两名共 $4$ 个队再进行单循环赛时,还要比赛 $C_4^2$ 场。所以共需要比赛的场数是
$$ C_7^2 + C_6^2 + C_4^2 = 21 + 15 + 6 = 42 \text{。} $$

答:这次篮球赛共需要比赛 $42$ 场。


\liti 在产品检验时,常从产品中抽出一部分进行检查。现在从 $100$ 件产品中任意抽出 $3$ 件:

(1) 一共有多少种不同的抽法?

(2) 如果 $100$ 件产品中有 $2$ 件次品,抽出的 $3$ 件中恰好有 $1$ 件是次品的抽法有多少种?

(3) 如果 $100$ 件产品中有 $2$ 件次品,抽出的 $3$ 件中至少有 $1$ 件是次品的抽法有多少种?


\jie  (1) 所求的不同抽法的种数, 就是从 $100$ 件产品中取出 $3$ 件的组合数:
$$ C_{100}^3 = \dfrac{100 \times 99 \times 98}{3 \times 2 \times 1} = 161700 \text{。} $$

答:共有 $161700$ 种抽法。

(2) 从 $2$ 件次品中抽出 $1$ 件次品的抽法有 $C_2^1$ 种,
从 $98$ 件合格品中抽出 $2$ 件合格品的抽法有 $C_{98}^2$ 种,
因此抽出的 $3$ 件中恰好有 $1$ 件是次品的抽法的种数是
$$ C_2^1 \cdot C_{98}^2 = 2 \times 4753 = 9506 \text{。} $$

答:$3$ 件中恰好有 $1$ 件是次品的抽法有 $9506$ 种。

(3) 从 $100$ 件产品中抽出 $3$ 件,一共有 $C_{100}^3$ 种抽法,在这些抽法里,
除掉抽出的 $3$ 件都是合格品的抽法 $C_{98}^3$ 种,便得抽出的 $3$ 件中至少有 $1$ 件是次品的抽法的种数,即
$$ C_{100}^3 - C_{98}^3 = 161700 - 152096 = 9604 \text{。} $$

本小题也可以这样来解:

从 $100$ 件产品抽出的 $3$ 件中至少有 $1$ 件是次品的抽法,包括 $1$ 件是次品的和 $2$ 件是次品的,
其中 1 件是次品的抽法有 $C_{98}^2 \cdot C_2^1$ 种,
$2$ 件是次品的抽法有 $C_{98}^1 \cdot C_2^2$ 种。
因此,至少有 $1$ 件是次品的抽法的种数为
$$ C_{98}^2 \cdot C_2^1 + C_{98}^1 \cdot C_2^2 = 9506 + 98 = 9604 \text{。} $$

答:$3$ 件中至少有 $1$ 件是次品的抽法有 $9604$ 种。



\lianxi
\begin{xiaotis}

\xiaoti{北京、上海、天津、广东四个足球队举行单循环赛:}
\begin{xiaoxiaotis}

    \xiaoxiaoti{列出所有各场比赛的双方;}

    \xiaoxiaoti{列出所有冠亚军的可能情况。}

\end{xiaoxiaotis}


\xiaoti{已知平面内不在同一直上的三点 $A,\, B,\, C$:}
\begin{xiaoxiaotis}

    \xiaoxiaoti{写出连结任意两点的所有线段;}

    \xiaoxiaoti{写出连结任意两点的所有的有向线段。}

\end{xiaoxiaotis}


\xiaoti{写出:}
\begin{xiaoxiaotis}

    \xiaoxiaoti{从五个元素 $a,\, b,\, c,\, d,\, e$ 中任取两个元素的所有组合;}

    \xiaoxiaoti{从五个元素 $a,\, b,\, c,\, d,\, e$ 中任取三个元素的所有组合。}

\end{xiaoxiaotis}


\xiaoti{利用第 3 题第 (1) 小题的结果写出从五个元素 $a,\, b,\, c,\, d,\, e$ 中任取两个元素的所有排列。}


\xiaoti{计算:}
\begin{xiaoxiaotis}

    \renewcommand\arraystretch{1.2}
    \begin{tabular}[t]{*{2}{@{}p{16em}}}
        \xiaoxiaoti{$C_6^2$;} & \xiaoxiaoti{$C_8^3$;} \\
        \xiaoxiaoti{$C_{100}^{96}$;} & \xiaoxiaoti{$C_7^3 - C_6^2$;} \\
        \xiaoxiaoti{$C_5^1 + C_5^2 + C_5^3 + C_5^4 + C_5^5$;} & \xiaoxiaoti{$3 C_8^3 - 2 C_5^2$。}
    \end{tabular}

\end{xiaoxiaotis}


\xiaoti{从 $3,\, 5,\, 7,\, 11$ 这四个质数中任取两个相乘,可以得到多少个不相等的积?}


\xiaoti{某校举行排球单循环赛,有 $8$ 个队参加,共需要举行多少场比赛?}

\end{xiaotis}

