\subsection{余数定理}\label{subsec:1-3}

设有多项式 $f(x) = x^3 - 7x^2 + 12x + 27$,那么 $f(5) = 5^3 - 7 \times 5^2 + 12 \times 5 + 27 = 125 - 175 + 60 + 27 = 37$;
另一方面,如果把这个多项式除以 $x - 5$,求余数,那么用综合除法可得:
$$
\begin{array}{*{3}{c@{\hspace{0.8cm}}}c|l}
    1 & -7 & +12 & +27 & 5\\
      & +5 & -10 & +10 & \\
    \cline{1-4}
    1 & -2 & +2 & \multicolumn{1}{|r}{ +37 } & \\
    \cline{4-4}
\end{array}
$$
我们发现,所得的余数正好也是 $37$。这就是说,多项式 $f(x) = x^3 - 7x^2 + 12x + 27$ 除以
$x - 5$ 所得的余数正好等于 $f(5)$。

对一般的多项式,有下面的重要定理:

\textbf{余数定理\mylabel{theorem:ysdl}[余数定理]\footnotemark\quad 多项式 $f(x)$ 除以 $x - b$ 所得的余数等于 $f(b)$。}
\footnotetext{此定理又叫\textbf{余式定理},\textbf{剩余定理}或\textbf{裴蜀定理}。裴蜀(Etienne Bezout, 1730 - 1783 年),法国数学家。}

\zhengming 设多项式 $f(x)$ 除以 $x - b$ 所得的商式为 $q(x)$,余数为 $r$,则有
$$ f(x) = (x - b) \cdot q(x) + r \text{。} $$

用 $x = b$ 代入等式的两边,得
$$ f(b) = (b - b) \cdot q(b) + r \text{。} $$

由此即得余数 $r = f(b)$。

根据余数定理,既然多项式 $f(x)$ 除以 $x - b$ 所得的余数 $r$ 等于 $f(x)$ 在 $x = b$ 时
的值 $f(b)$,那么 $r$ 就可以由 $f(b)$ 来求得,反过来,$f(b)$ 也可以由 $r$ 来求得。

\liti 设 $f(x) = x^8 + 3$,求 $f(x)$ 除以 $x + 1$ 所得的余数。

\jie 根据余数定理,所求的余数等于 $f(-1) = (-1)^8 + 3 = 4$。


\liti 设 $f(x) = x^5 - 12x^3 + 15x - 8$,求 $f(6)$。

\jie 用综合除法求 $f(x)$ 除以 $x - 6$ 所得的余数:
$$
\begin{array}{*{5}{c@{\hspace{0.8cm}}}c|l}
    1 & +0 & -12 &   +0 &  +15 &    -8 & 6 \\
      & +6 & +36 & +144 & +864 & +5274 &   \\
    \cline{1-6}
    1 & +6 & +24 & +144 & +879 & \multicolumn{1}{|r}{ +5266 } & \\
    \cline{6-6}
\end{array}
$$

根据余数定理,余数 $5266$ 等于 $f(6)$,所以
$$ f(6) = 5266 \text{。} $$


\lianxi
\begin{xiaotis}

\xiaoti{设 $f(x) = 5x^4 - x^2 + 6$,求 $f(x)$ 除以 $x - 1$ 所得的余数。}

\xiaoti{设 $f(x) = x^4 - 3x^3 + 6x^2 - 10x + 9$,求 $f(4)$。}

\xiaoti{已知 $f(x) = 16x^4 - 14x^3 - 15x^2 - 24x + 38$,求 $f\left( \dfrac{3}{2} \right)$。}

\xiaoti{设 $f(x) = x^6 + a^6$,求 $f(x)$ 除以 $x - a\,i$ 所得的余数。}

\end{xiaotis}

